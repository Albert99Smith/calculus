\documentclass{ximera}

%\usepackage{todonotes}
%\usepackage{mathtools} %% Required for wide table Curl and Greens
%\usepackage{cuted} %% Required for wide table Curl and Greens
\newcommand{\todo}{}

\usepackage{esint} % for \oiint
\ifxake%%https://math.meta.stackexchange.com/questions/9973/how-do-you-render-a-closed-surface-double-integral
\renewcommand{\oiint}{{\large\bigcirc}\kern-1.56em\iint}
\fi


\graphicspath{
  {./}
  {ximeraTutorial/}
  {basicPhilosophy/}
  {functionsOfSeveralVariables/}
  {normalVectors/}
  {lagrangeMultipliers/}
  {vectorFields/}
  {greensTheorem/}
  {shapeOfThingsToCome/}
  {dotProducts/}
  {partialDerivativesAndTheGradientVector/}
  {../productAndQuotientRules/exercises/}
  {../normalVectors/exercisesParametricPlots/}
  {../continuityOfFunctionsOfSeveralVariables/exercises/}
  {../partialDerivativesAndTheGradientVector/exercises/}
  {../directionalDerivativeAndChainRule/exercises/}
  {../commonCoordinates/exercisesCylindricalCoordinates/}
  {../commonCoordinates/exercisesSphericalCoordinates/}
  {../greensTheorem/exercisesCurlAndLineIntegrals/}
  {../greensTheorem/exercisesDivergenceAndLineIntegrals/}
  {../shapeOfThingsToCome/exercisesDivergenceTheorem/}
  {../greensTheorem/}
  {../shapeOfThingsToCome/}
  {../separableDifferentialEquations/exercises/}
  {vectorFields/}
}

\newcommand{\mooculus}{\textsf{\textbf{MOOC}\textnormal{\textsf{ULUS}}}}

\usepackage{tkz-euclide}\usepackage{tikz}
\usepackage{tikz-cd}
\usetikzlibrary{arrows}
\tikzset{>=stealth,commutative diagrams/.cd,
  arrow style=tikz,diagrams={>=stealth}} %% cool arrow head
\tikzset{shorten <>/.style={ shorten >=#1, shorten <=#1 } } %% allows shorter vectors

\usetikzlibrary{backgrounds} %% for boxes around graphs
\usetikzlibrary{shapes,positioning}  %% Clouds and stars
\usetikzlibrary{matrix} %% for matrix
\usepgfplotslibrary{polar} %% for polar plots
\usepgfplotslibrary{fillbetween} %% to shade area between curves in TikZ
\usetkzobj{all}
\usepackage[makeroom]{cancel} %% for strike outs
%\usepackage{mathtools} %% for pretty underbrace % Breaks Ximera
%\usepackage{multicol}
\usepackage{pgffor} %% required for integral for loops



%% http://tex.stackexchange.com/questions/66490/drawing-a-tikz-arc-specifying-the-center
%% Draws beach ball
\tikzset{pics/carc/.style args={#1:#2:#3}{code={\draw[pic actions] (#1:#3) arc(#1:#2:#3);}}}



\usepackage{array}
\setlength{\extrarowheight}{+.1cm}
\newdimen\digitwidth
\settowidth\digitwidth{9}
\def\divrule#1#2{
\noalign{\moveright#1\digitwidth
\vbox{\hrule width#2\digitwidth}}}





\newcommand{\RR}{\mathbb R}
\newcommand{\R}{\mathbb R}
\newcommand{\N}{\mathbb N}
\newcommand{\Z}{\mathbb Z}

\newcommand{\sagemath}{\textsf{SageMath}}


%\renewcommand{\d}{\,d\!}
\renewcommand{\d}{\mathop{}\!d}
\newcommand{\dd}[2][]{\frac{\d #1}{\d #2}}
\newcommand{\pp}[2][]{\frac{\partial #1}{\partial #2}}
\renewcommand{\l}{\ell}
\newcommand{\ddx}{\frac{d}{\d x}}

\newcommand{\zeroOverZero}{\ensuremath{\boldsymbol{\tfrac{0}{0}}}}
\newcommand{\inftyOverInfty}{\ensuremath{\boldsymbol{\tfrac{\infty}{\infty}}}}
\newcommand{\zeroOverInfty}{\ensuremath{\boldsymbol{\tfrac{0}{\infty}}}}
\newcommand{\zeroTimesInfty}{\ensuremath{\small\boldsymbol{0\cdot \infty}}}
\newcommand{\inftyMinusInfty}{\ensuremath{\small\boldsymbol{\infty - \infty}}}
\newcommand{\oneToInfty}{\ensuremath{\boldsymbol{1^\infty}}}
\newcommand{\zeroToZero}{\ensuremath{\boldsymbol{0^0}}}
\newcommand{\inftyToZero}{\ensuremath{\boldsymbol{\infty^0}}}



\newcommand{\numOverZero}{\ensuremath{\boldsymbol{\tfrac{\#}{0}}}}
\newcommand{\dfn}{\textbf}
%\newcommand{\unit}{\,\mathrm}
\newcommand{\unit}{\mathop{}\!\mathrm}
\newcommand{\eval}[1]{\bigg[ #1 \bigg]}
\newcommand{\seq}[1]{\left( #1 \right)}
\renewcommand{\epsilon}{\varepsilon}
\renewcommand{\phi}{\varphi}


\renewcommand{\iff}{\Leftrightarrow}

\DeclareMathOperator{\arccot}{arccot}
\DeclareMathOperator{\arcsec}{arcsec}
\DeclareMathOperator{\arccsc}{arccsc}
\DeclareMathOperator{\si}{Si}
\DeclareMathOperator{\scal}{scal}
\DeclareMathOperator{\sign}{sign}


%% \newcommand{\tightoverset}[2]{% for arrow vec
%%   \mathop{#2}\limits^{\vbox to -.5ex{\kern-0.75ex\hbox{$#1$}\vss}}}
\newcommand{\arrowvec}[1]{{\overset{\rightharpoonup}{#1}}}
%\renewcommand{\vec}[1]{\arrowvec{\mathbf{#1}}}
\renewcommand{\vec}[1]{{\overset{\boldsymbol{\rightharpoonup}}{\mathbf{#1}}}\hspace{0in}}

\newcommand{\point}[1]{\left(#1\right)} %this allows \vector{ to be changed to \vector{ with a quick find and replace
\newcommand{\pt}[1]{\mathbf{#1}} %this allows \vec{ to be changed to \vec{ with a quick find and replace
\newcommand{\Lim}[2]{\lim_{\point{#1} \to \point{#2}}} %Bart, I changed this to point since I want to use it.  It runs through both of the exercise and exerciseE files in limits section, which is why it was in each document to start with.

\DeclareMathOperator{\proj}{\mathbf{proj}}
\newcommand{\veci}{{\boldsymbol{\hat{\imath}}}}
\newcommand{\vecj}{{\boldsymbol{\hat{\jmath}}}}
\newcommand{\veck}{{\boldsymbol{\hat{k}}}}
\newcommand{\vecl}{\vec{\boldsymbol{\l}}}
\newcommand{\uvec}[1]{\mathbf{\hat{#1}}}
\newcommand{\utan}{\mathbf{\hat{t}}}
\newcommand{\unormal}{\mathbf{\hat{n}}}
\newcommand{\ubinormal}{\mathbf{\hat{b}}}

\newcommand{\dotp}{\bullet}
\newcommand{\cross}{\boldsymbol\times}
\newcommand{\grad}{\boldsymbol\nabla}
\newcommand{\divergence}{\grad\dotp}
\newcommand{\curl}{\grad\cross}
%\DeclareMathOperator{\divergence}{divergence}
%\DeclareMathOperator{\curl}[1]{\grad\cross #1}
\newcommand{\lto}{\mathop{\longrightarrow\,}\limits}

\renewcommand{\bar}{\overline}

\colorlet{textColor}{black}
\colorlet{background}{white}
\colorlet{penColor}{blue!50!black} % Color of a curve in a plot
\colorlet{penColor2}{red!50!black}% Color of a curve in a plot
\colorlet{penColor3}{red!50!blue} % Color of a curve in a plot
\colorlet{penColor4}{green!50!black} % Color of a curve in a plot
\colorlet{penColor5}{orange!80!black} % Color of a curve in a plot
\colorlet{penColor6}{yellow!70!black} % Color of a curve in a plot
\colorlet{fill1}{penColor!20} % Color of fill in a plot
\colorlet{fill2}{penColor2!20} % Color of fill in a plot
\colorlet{fillp}{fill1} % Color of positive area
\colorlet{filln}{penColor2!20} % Color of negative area
\colorlet{fill3}{penColor3!20} % Fill
\colorlet{fill4}{penColor4!20} % Fill
\colorlet{fill5}{penColor5!20} % Fill
\colorlet{gridColor}{gray!50} % Color of grid in a plot

\newcommand{\surfaceColor}{violet}
\newcommand{\surfaceColorTwo}{redyellow}
\newcommand{\sliceColor}{greenyellow}




\pgfmathdeclarefunction{gauss}{2}{% gives gaussian
  \pgfmathparse{1/(#2*sqrt(2*pi))*exp(-((x-#1)^2)/(2*#2^2))}%
}


%%%%%%%%%%%%%
%% Vectors
%%%%%%%%%%%%%

%% Simple horiz vectors
\renewcommand{\vector}[1]{\left\langle #1\right\rangle}


%% %% Complex Horiz Vectors with angle brackets
%% \makeatletter
%% \renewcommand{\vector}[2][ , ]{\left\langle%
%%   \def\nextitem{\def\nextitem{#1}}%
%%   \@for \el:=#2\do{\nextitem\el}\right\rangle%
%% }
%% \makeatother

%% %% Vertical Vectors
%% \def\vector#1{\begin{bmatrix}\vecListA#1,,\end{bmatrix}}
%% \def\vecListA#1,{\if,#1,\else #1\cr \expandafter \vecListA \fi}

%%%%%%%%%%%%%
%% End of vectors
%%%%%%%%%%%%%

%\newcommand{\fullwidth}{}
%\newcommand{\normalwidth}{}



%% makes a snazzy t-chart for evaluating functions
%\newenvironment{tchart}{\rowcolors{2}{}{background!90!textColor}\array}{\endarray}

%%This is to help with formatting on future title pages.
\newenvironment{sectionOutcomes}{}{}



%% Flowchart stuff
%\tikzstyle{startstop} = [rectangle, rounded corners, minimum width=3cm, minimum height=1cm,text centered, draw=black]
%\tikzstyle{question} = [rectangle, minimum width=3cm, minimum height=1cm, text centered, draw=black]
%\tikzstyle{decision} = [trapezium, trapezium left angle=70, trapezium right angle=110, minimum width=3cm, minimum height=1cm, text centered, draw=black]
%\tikzstyle{question} = [rectangle, rounded corners, minimum width=3cm, minimum height=1cm,text centered, draw=black]
%\tikzstyle{process} = [rectangle, minimum width=3cm, minimum height=1cm, text centered, draw=black]
%\tikzstyle{decision} = [trapezium, trapezium left angle=70, trapezium right angle=110, minimum width=3cm, minimum height=1cm, text centered, draw=black]


\outcome{Compute limits of vector-valued functions.}
\outcome{Determine continuity of vector-valued functions.}
\outcome{Compute the derivative of a vector-valued function.}
\outcome{Compute the tangent vector of a vector-valued function.}
\outcome{Find the unit tangent vector of a vector-valued function.}
\outcome{Compute integrals of vector-valued functions.}


\title[Dig-In:]{Calculus and vector-valued functions}

\begin{document}
\begin{abstract}
  With one input, and vector outputs, we work component-wise.
\end{abstract}
\maketitle


Since we are currently thinking about vector-valued functions that
only have a single input, we can work \dfn{component-wise}. This means
that to build a theory of calculus for vector-valued functions, we
simply treat each component of a vector-valued function as a regular,
single-variable function. Let's see this in action.

\section{Limits of vector-valued functions}

\begin{definition}
  Let $\vec{f}:\R \to \R^3$ be a vector-valued function
  \[
  \vec{f}(t) = \vector{x(t),y(t),z(t)}
  \]
  then the vector-valued \dfn{limit} as $t$ goes to $a$ is given by 
  \[
  \lim_{t \to a} \vec{f}(t) = \vector{\lim_{t \to a}x(t),\lim_{t \to a}y(t),\lim_{t \to a}z(t)}.
  \]
  This limit exists if and only if each of the limits of the
  components exist.
\end{definition}

We evaluate limits by just taking the limit of each component
separately.

\begin{question}
  Let $\vec{f}(t) = \vector{\sin(t),\cos(t),\frac{\sin(t)}{t}}$.
  Compute:
  \[
  \lim_{t \to 0} \vec{f}(t)
  \begin{prompt}
    =\vector{\answer{0},\answer{1},\answer{1}}
  \end{prompt}
  \]
  \begin{hint}
    Take the limit of each component separately.
  \end{hint}
\end{question}


Now that we have the notion of limits, we may also define the concept
of continuity of vector-valued functions:

\begin{definition}
  A vector-valued function $\vec{f}$ is \dfn{continuous} at $t= a$ if
  and only if
  \[
  \lim_{t \to a} \vec{f}(t)  = \vec{f}(a)
  \]
\end{definition}
Because of the component-wise nature of limits, we can see that a
function $\vec{f}$ is continuous if and only if each component
function $x(t)$, $y(t)$, $z(t)$ is also continuous at $t=a$.

\begin{question}
Which of the following vector-valued functions are continuous for all
real values of $t$?
\begin{selectAll}
  \choice{$\vector{t,t^2,\tan(t)}$}
  \choice{$\vector{\frac{2}{t-5},\sin(t),0}$}
  \choice[correct]{$\vector{\cos(5 t), t^2-3t+1, e^t}$}
  \choice{$\vector{e^{\sqrt{t}}, 4\sin(t), t^7}$}
  \choice{$\vector{\ln(t^2), \ln(t), t^{3/2}}$}
  \choice[correct]{$\vector{0, 0, 0}$}
\end{selectAll}
\end{question}


\section{Derivatives}

We also compute derivatives component-wise:

\begin{definition}
  Let $\vec{f}:\R \to \R^3$ be a vector-valued function
  \[
  \vec{f}(t) = \vector{x(t),y(t),z(t)},
  \]
  then the vector-valued \dfn{derivative} is given by
  \[
  \vec{f}'(t) = \vector{x'(t),y'(t),z'(t)}.
  \]
\end{definition}

\begin{question}
  Let $\vec{f}(t) = \vector{\arctan(t),7^t, \sec(t^2)}$.
  Compute:
  \[
  \dd{t} \vec{f}(t)
  \begin{prompt}
    =\vector{\answer{\frac{1}{1+t^2}},\answer{7^t \ln(7)},\answer{ \sec(t^2)\tan(t^2)2t}}
  \end{prompt}
  \]
\end{question}

We also have some (additional) derivative rules:
\begin{theorem}
  Let $\vec{f}$ and $\vec{g}$ be differentiable vector-valued
  functions, $s(t)$ be a differentiable scalar function, and $c$ be a
  constant. Then:
  \begin{itemize}
  \item $\dd{t} \left(\vec{f}(t) + \vec{g}(t) \right) = \vec{f}'(t) + \vec{g}'(t)$
  \item $\dd{t} c\vec{f}(t) = c\vec{f}'(t)$
  \item $\dd{t} s(t)\vec{f}(t) = s'(t)\vec{f}(t) + s(t)\vec{f}'(t)$
  \item $\dd{t} \left(\vec{f}(t)\dotp \vec{g}(t)\right) = \vec{f}'(t)\dotp \vec{g}(t) + \vec{f}(t)\dotp\vec{g}'(t)$
  \item $\dd{t} \left(\vec{f}(t)\cross \vec{g}(t)\right) = \vec{f}'(t)\cross \vec{g}(t) + \vec{f}(t)\cross\vec{g}'(t)$
  \item $\dd{t} \vec{f}(s(t)) = \vec{f}'(s(t))\cdot s'(t)$
  \end{itemize}
\end{theorem}

The derivative of a vector-valued function gives a \textit{tangent
  vector} that points in the direction that the curve is drawn.

\begin{definition}
Let $\vec{f}(t)$ be a differentiable vector-valued function on an open
interval $I$ containing $c$, where $\vec{f}'(c)\neq \vec{0}$.
\begin{itemize}
\item A vector $\vec v$ is a \dfn{tangent vector} to the graph of
  $\vec{f}(t)$ at $t=c$, if $\vec v$ is parallel to $\vec{f}'(c)$.
\item The \dfn{tangent line} to the graph of $\vec f(t)$ at $t=c$
  is the line through $\vec f(c)$ in a direction parallel to
  $\vec{f}'(c)$. An equation of the tangent line is
  \[
  \boldsymbol{\l}(t) = \vec f(c) + t\vec{f}'(c).
  \]
\item The \dfn{unit tangent vector} of $\vec{f}$ is given by
  \[
  \utan(t)=\frac{\vec{f}'(t)}{|\vec{f}'(t)|} = \frac{\vec{f}'(t)}{\sqrt{\vec{f}'(t)\dotp \vec{f}'(t)}}.
  \]
\end{itemize}
\end{definition}

\begin{question}
  Let $\vec{f}(t) = \vector{t,t^2,t^3}$ on $[-1.5,1.5]$.  Find the
  vector equation of the line tangent to the graph of $\vec f$ at
  $t=-1$.
  \begin{prompt}
    \[
    \boldsymbol{\l}(t) = \vector{\answer{-1},\answer{1},-1} + t\vector{\answer{1},-2,\answer{3}}
    \]
  \end{prompt}
\end{question}

If the tangent vector is ever the zero vector, then there is a
problem, since there is no ``tangent line'' at that point. Because of
this, we have a special name for functions where the tangent vector is never zero:

\begin{definition}
Let $\vec{f}(t)$ be a differentiable vector-valued function on an open
interval $I$. $\vec f(t)$ is \dfn{smooth} on $I$ if
\[
\vec{f}'(t)\neq\vec{0}
\]
for $t$ in $I$.
\end{definition}

\begin{question}
  Is the vector-valued function $\vec{f}(t) = \vector{t^2,t^2}$ smooth on
  $\R$?
  \begin{prompt}
    \begin{multipleChoice}
      \choice{yes}
      \choice[correct]{no}
    \end{multipleChoice}
    \begin{feedback}
      This function looks like a ``ray.'' Conceptually it isn't smooth
      because it goes to the origin and turns around ``abruptly.'' Check it out:
      \[
      \graph{(t^2,t^2)}
      \]
    \end{feedback}
  \end{prompt}
\end{question}


Finally, we should point out that if a vector-valued function has
constant length, then it traces out part of a circle.
\begin{theorem}
  Let $\vec f(t)$ be a differentiable vector-valued function on an
  open interval $I$. $\vec{f}(t)$ is of constant length if and only if
  the function and its tangent vector are orthogonal.
  \begin{explanation}
    Suppose that $|\vec f(t)| = c$ for all $t$ in $I$. In particular,
    we see
    \[
    \vec{f}(t) \dotp \vec{f}(t) = c^2.
    \]
    Now take the derivative of both sides. Write with me
    \begin{align*}
      \vec{f}'(t) \dotp \vec{f}(t) + \vec{f}(t)\dotp \vec{f}'(t) &= 0\\
      2\vec{f}'(t) \dotp \vec{f}(t) &= 0\\
      \vec{f}'(t) \dotp \vec{f}(t) &= 0
    \end{align*}
    and this means that the vectors $\vec{f}'(t)$ and $\vec{f}(t)$ are
    \wordChoice{\choice{parallel}\choice[correct]{perpendicular}}.

    Now suppose that for all $t$ in $I$, $\vec{f}'(t)$ and
    $\vec{f}(t)$ are orthogonal. This means
    \begin{align*}
      \vec{f}'(t) \dotp \vec{f}(t) &= 0\\
      2\vec{f}'(t) \dotp \vec{f}(t) &= 0\\
      \vec{f}'(t) \dotp \vec{f}(t) + \vec{f}(t)\dotp \vec{f}'(t) &= 0,
    \end{align*}
    but we know where the derivative on the left came from! So we may
    write
    \[
    \vec{f}(t) \dotp \vec{f}(t) = c^2
    \]
    for some constant $c$. Now we see that $\vec{f}$ is a
    vector-valued function of constant length on an open interval
    $I$.
  \end{explanation}
\end{theorem}
The theorem above says a lot in a very little. To start, it says that
if the magnitude of a vector-valued function is constant, then the
vector-value of the function and its tangent vector are
orthogonal. Moreover, this means that the curve looks locally like a
circle. However, it does not mean the curve looks like a circle globally.
\begin{onlineOnly}
  For example, consider this curve:

  \begin{center}
    \geogebra{sAtZbVuM}{800}{600}
  \end{center}

  By moving the slider around you can see the vector of constant
  length that draws the curve.
\end{onlineOnly}






\section{Integrals}

Indefinite and definite integrals of vector-valued functions are also evaluated component-wise.

\begin{theorem}
  Let $\vec{f}:\R \to \R^3$ be a continuous vector-valued function
  \[
  \vec{f}(t) = \vector{x(t),y(t),z(t)},
  \]
  then the vector-valued \dfn{integral} is given by
  \[
  \int \vec{f}(t) \d t= \vector{\int x(t) \d t,\int y(t)\d t,\int z(t)\d t}.
  \]
\end{theorem}

\begin{question}
  Let $\vec f(t) = \vector{e^{2t},\sin t}$. Compute:
  \[
  \int_0^1 \vec f(t) \d t
  \begin{prompt}
    = \vector{\answer{\frac{1}{2}(e^2-1)},\answer{-\cos(1)+1}}
  \end{prompt}
  \]
\end{question}

We can also solve initial value problems, check out our next example:

\begin{example}
  Let $\vec{f}''(t) = \vector{2, \cos(t), 12t}$. Find $\vec f(t)$ where:
  \begin{itemize}
  \item $\vec f(0) = \vector{-7,-1,2}$ and
  \item $\vec{f}'(0) = \vector{5,3,0}.$
  \end{itemize}
  \begin{explanation}
    Knowing $\vec{f}''(t) = \vector{2,\cos(t), 12t}$, we find
    $\vec{f}'(t)$ by evaluating the indefinite integral.
    \begin{align*}
      \int \vec{f}''(t)\d t &= \vector{\int \answer[given]{2}\d t, \int \answer[given]{\cos(t)}\d t , \int \answer[given]{12t}\d t} \\
      &= \vector{\answer[given]{2t}+C_1, \answer[given]{\sin(t)}+ C_2, \answer[given]{6t^2} + C_3} \\
      &= \vector{2t,\sin(t),6t^2} + \vector{C_1,C_2,C_3} \\
      &= \vector{2t,\sin(t),6t^2} + \vec{C}.
    \end{align*}
    Note how each indefinite integral creates its own constant which
    we collect as one constant vector $\vec C$. Knowing
    \[
    \vec{f}'(0) = \vector{5,3,0}
    \]
    allows us to solve for $\vec C$:
    \begin{align*}
      \vec{f}'(t) & = \vector{2t,\sin(t),6t^2} + \vec C\\
      \vec{f}'(0) &= \vector{\answer[given]{0},\answer[given]{0},\answer[given]{0}} + \vec C\\
      \vector{5,3,0} &= \vec C.
    \end{align*}
    So
    \begin{align*}
      \vec{f}'(t) &= \vector{2t,\sin(t),6t^2} + \vector{\answer[given]{5},\answer[given]{3},\answer[given]{0}}\\
      &=\vector{2t+5, \sin(t) + 3, 6t^2}.
    \end{align*}
    To find $\vec f(t)$, we integrate once more.
    \begin{align*}
      \int \vec{f}'(t)\d t &= \vector{\int\answer[given]{2t+5}\d t, \int \answer[given]{\sin(t) + 3}\d t, \int \answer[given]{6t^2}\d t }\\
      &= \vector{\answer[given]{t^2+5t}, \answer[given]{-\cos(t) + 3t}, \answer[given]{2t^3}} + \vec C
    \end{align*}
    With $\vec f(0) = \vector{-7,-1,2}$, we solve for $\vec C$:
    \begin{align*}
      \vec f(t) &= \vector{\answer[given]{t^2+5t}, \answer[given]{-\cos(t) + 3t}, \answer[given]{2t^3}} + \vec C\\
      \vec f(0) &= \vector{\answer[given]{0},\answer[given]{-1},\answer[given]{0}} + \vec C\\
      \vector{-7,-1,2} &= \vector{0,-1,0} + \vec C.
    \end{align*}
    Hence:
    \[
    \vec{C}=\vector{\answer[given]{-7},\answer[given]{0},\answer[given]{2}}
    \]
    So
    \begin{align*}
      \vec f(t) &= \vector{t^2+5t, -\cos(t) + 3t, 2t^3} + \vector{-7,0,2}\\
      &= \vector{\answer[given]{t^2+5t-7},\answer[given]{-\cos(t)+3t},\answer[given]{2t^3+2}}.
    \end{align*}
  \end{explanation}
\end{example}

We now leave you with a question: What does the integration of a
vector-valued function \textit{mean}?  There are many applications,
but none as direct as ``the area under the curve'' that we used in
understanding the integral of a real-valued function. We will explore
this later in our study of calculus.

\end{document}
