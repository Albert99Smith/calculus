\documentclass{ximera}

%\usepackage{todonotes}
%\usepackage{mathtools} %% Required for wide table Curl and Greens
%\usepackage{cuted} %% Required for wide table Curl and Greens
\newcommand{\todo}{}

\usepackage{esint} % for \oiint
\ifxake%%https://math.meta.stackexchange.com/questions/9973/how-do-you-render-a-closed-surface-double-integral
\renewcommand{\oiint}{{\large\bigcirc}\kern-1.56em\iint}
\fi


\graphicspath{
  {./}
  {ximeraTutorial/}
  {basicPhilosophy/}
  {functionsOfSeveralVariables/}
  {normalVectors/}
  {lagrangeMultipliers/}
  {vectorFields/}
  {greensTheorem/}
  {shapeOfThingsToCome/}
  {dotProducts/}
  {partialDerivativesAndTheGradientVector/}
  {../productAndQuotientRules/exercises/}
  {../normalVectors/exercisesParametricPlots/}
  {../continuityOfFunctionsOfSeveralVariables/exercises/}
  {../partialDerivativesAndTheGradientVector/exercises/}
  {../directionalDerivativeAndChainRule/exercises/}
  {../commonCoordinates/exercisesCylindricalCoordinates/}
  {../commonCoordinates/exercisesSphericalCoordinates/}
  {../greensTheorem/exercisesCurlAndLineIntegrals/}
  {../greensTheorem/exercisesDivergenceAndLineIntegrals/}
  {../shapeOfThingsToCome/exercisesDivergenceTheorem/}
  {../greensTheorem/}
  {../shapeOfThingsToCome/}
  {../separableDifferentialEquations/exercises/}
  {vectorFields/}
}

\newcommand{\mooculus}{\textsf{\textbf{MOOC}\textnormal{\textsf{ULUS}}}}

\usepackage{tkz-euclide}\usepackage{tikz}
\usepackage{tikz-cd}
\usetikzlibrary{arrows}
\tikzset{>=stealth,commutative diagrams/.cd,
  arrow style=tikz,diagrams={>=stealth}} %% cool arrow head
\tikzset{shorten <>/.style={ shorten >=#1, shorten <=#1 } } %% allows shorter vectors

\usetikzlibrary{backgrounds} %% for boxes around graphs
\usetikzlibrary{shapes,positioning}  %% Clouds and stars
\usetikzlibrary{matrix} %% for matrix
\usepgfplotslibrary{polar} %% for polar plots
\usepgfplotslibrary{fillbetween} %% to shade area between curves in TikZ
\usetkzobj{all}
\usepackage[makeroom]{cancel} %% for strike outs
%\usepackage{mathtools} %% for pretty underbrace % Breaks Ximera
%\usepackage{multicol}
\usepackage{pgffor} %% required for integral for loops



%% http://tex.stackexchange.com/questions/66490/drawing-a-tikz-arc-specifying-the-center
%% Draws beach ball
\tikzset{pics/carc/.style args={#1:#2:#3}{code={\draw[pic actions] (#1:#3) arc(#1:#2:#3);}}}



\usepackage{array}
\setlength{\extrarowheight}{+.1cm}
\newdimen\digitwidth
\settowidth\digitwidth{9}
\def\divrule#1#2{
\noalign{\moveright#1\digitwidth
\vbox{\hrule width#2\digitwidth}}}





\newcommand{\RR}{\mathbb R}
\newcommand{\R}{\mathbb R}
\newcommand{\N}{\mathbb N}
\newcommand{\Z}{\mathbb Z}

\newcommand{\sagemath}{\textsf{SageMath}}


%\renewcommand{\d}{\,d\!}
\renewcommand{\d}{\mathop{}\!d}
\newcommand{\dd}[2][]{\frac{\d #1}{\d #2}}
\newcommand{\pp}[2][]{\frac{\partial #1}{\partial #2}}
\renewcommand{\l}{\ell}
\newcommand{\ddx}{\frac{d}{\d x}}

\newcommand{\zeroOverZero}{\ensuremath{\boldsymbol{\tfrac{0}{0}}}}
\newcommand{\inftyOverInfty}{\ensuremath{\boldsymbol{\tfrac{\infty}{\infty}}}}
\newcommand{\zeroOverInfty}{\ensuremath{\boldsymbol{\tfrac{0}{\infty}}}}
\newcommand{\zeroTimesInfty}{\ensuremath{\small\boldsymbol{0\cdot \infty}}}
\newcommand{\inftyMinusInfty}{\ensuremath{\small\boldsymbol{\infty - \infty}}}
\newcommand{\oneToInfty}{\ensuremath{\boldsymbol{1^\infty}}}
\newcommand{\zeroToZero}{\ensuremath{\boldsymbol{0^0}}}
\newcommand{\inftyToZero}{\ensuremath{\boldsymbol{\infty^0}}}



\newcommand{\numOverZero}{\ensuremath{\boldsymbol{\tfrac{\#}{0}}}}
\newcommand{\dfn}{\textbf}
%\newcommand{\unit}{\,\mathrm}
\newcommand{\unit}{\mathop{}\!\mathrm}
\newcommand{\eval}[1]{\bigg[ #1 \bigg]}
\newcommand{\seq}[1]{\left( #1 \right)}
\renewcommand{\epsilon}{\varepsilon}
\renewcommand{\phi}{\varphi}


\renewcommand{\iff}{\Leftrightarrow}

\DeclareMathOperator{\arccot}{arccot}
\DeclareMathOperator{\arcsec}{arcsec}
\DeclareMathOperator{\arccsc}{arccsc}
\DeclareMathOperator{\si}{Si}
\DeclareMathOperator{\scal}{scal}
\DeclareMathOperator{\sign}{sign}


%% \newcommand{\tightoverset}[2]{% for arrow vec
%%   \mathop{#2}\limits^{\vbox to -.5ex{\kern-0.75ex\hbox{$#1$}\vss}}}
\newcommand{\arrowvec}[1]{{\overset{\rightharpoonup}{#1}}}
%\renewcommand{\vec}[1]{\arrowvec{\mathbf{#1}}}
\renewcommand{\vec}[1]{{\overset{\boldsymbol{\rightharpoonup}}{\mathbf{#1}}}\hspace{0in}}

\newcommand{\point}[1]{\left(#1\right)} %this allows \vector{ to be changed to \vector{ with a quick find and replace
\newcommand{\pt}[1]{\mathbf{#1}} %this allows \vec{ to be changed to \vec{ with a quick find and replace
\newcommand{\Lim}[2]{\lim_{\point{#1} \to \point{#2}}} %Bart, I changed this to point since I want to use it.  It runs through both of the exercise and exerciseE files in limits section, which is why it was in each document to start with.

\DeclareMathOperator{\proj}{\mathbf{proj}}
\newcommand{\veci}{{\boldsymbol{\hat{\imath}}}}
\newcommand{\vecj}{{\boldsymbol{\hat{\jmath}}}}
\newcommand{\veck}{{\boldsymbol{\hat{k}}}}
\newcommand{\vecl}{\vec{\boldsymbol{\l}}}
\newcommand{\uvec}[1]{\mathbf{\hat{#1}}}
\newcommand{\utan}{\mathbf{\hat{t}}}
\newcommand{\unormal}{\mathbf{\hat{n}}}
\newcommand{\ubinormal}{\mathbf{\hat{b}}}

\newcommand{\dotp}{\bullet}
\newcommand{\cross}{\boldsymbol\times}
\newcommand{\grad}{\boldsymbol\nabla}
\newcommand{\divergence}{\grad\dotp}
\newcommand{\curl}{\grad\cross}
%\DeclareMathOperator{\divergence}{divergence}
%\DeclareMathOperator{\curl}[1]{\grad\cross #1}
\newcommand{\lto}{\mathop{\longrightarrow\,}\limits}

\renewcommand{\bar}{\overline}

\colorlet{textColor}{black}
\colorlet{background}{white}
\colorlet{penColor}{blue!50!black} % Color of a curve in a plot
\colorlet{penColor2}{red!50!black}% Color of a curve in a plot
\colorlet{penColor3}{red!50!blue} % Color of a curve in a plot
\colorlet{penColor4}{green!50!black} % Color of a curve in a plot
\colorlet{penColor5}{orange!80!black} % Color of a curve in a plot
\colorlet{penColor6}{yellow!70!black} % Color of a curve in a plot
\colorlet{fill1}{penColor!20} % Color of fill in a plot
\colorlet{fill2}{penColor2!20} % Color of fill in a plot
\colorlet{fillp}{fill1} % Color of positive area
\colorlet{filln}{penColor2!20} % Color of negative area
\colorlet{fill3}{penColor3!20} % Fill
\colorlet{fill4}{penColor4!20} % Fill
\colorlet{fill5}{penColor5!20} % Fill
\colorlet{gridColor}{gray!50} % Color of grid in a plot

\newcommand{\surfaceColor}{violet}
\newcommand{\surfaceColorTwo}{redyellow}
\newcommand{\sliceColor}{greenyellow}




\pgfmathdeclarefunction{gauss}{2}{% gives gaussian
  \pgfmathparse{1/(#2*sqrt(2*pi))*exp(-((x-#1)^2)/(2*#2^2))}%
}


%%%%%%%%%%%%%
%% Vectors
%%%%%%%%%%%%%

%% Simple horiz vectors
\renewcommand{\vector}[1]{\left\langle #1\right\rangle}


%% %% Complex Horiz Vectors with angle brackets
%% \makeatletter
%% \renewcommand{\vector}[2][ , ]{\left\langle%
%%   \def\nextitem{\def\nextitem{#1}}%
%%   \@for \el:=#2\do{\nextitem\el}\right\rangle%
%% }
%% \makeatother

%% %% Vertical Vectors
%% \def\vector#1{\begin{bmatrix}\vecListA#1,,\end{bmatrix}}
%% \def\vecListA#1,{\if,#1,\else #1\cr \expandafter \vecListA \fi}

%%%%%%%%%%%%%
%% End of vectors
%%%%%%%%%%%%%

%\newcommand{\fullwidth}{}
%\newcommand{\normalwidth}{}



%% makes a snazzy t-chart for evaluating functions
%\newenvironment{tchart}{\rowcolors{2}{}{background!90!textColor}\array}{\endarray}

%%This is to help with formatting on future title pages.
\newenvironment{sectionOutcomes}{}{}



%% Flowchart stuff
%\tikzstyle{startstop} = [rectangle, rounded corners, minimum width=3cm, minimum height=1cm,text centered, draw=black]
%\tikzstyle{question} = [rectangle, minimum width=3cm, minimum height=1cm, text centered, draw=black]
%\tikzstyle{decision} = [trapezium, trapezium left angle=70, trapezium right angle=110, minimum width=3cm, minimum height=1cm, text centered, draw=black]
%\tikzstyle{question} = [rectangle, rounded corners, minimum width=3cm, minimum height=1cm,text centered, draw=black]
%\tikzstyle{process} = [rectangle, minimum width=3cm, minimum height=1cm, text centered, draw=black]
%\tikzstyle{decision} = [trapezium, trapezium left angle=70, trapezium right angle=110, minimum width=3cm, minimum height=1cm, text centered, draw=black]


\outcome{Understand the connection between curves and parameterizations.}

\author{Jim Talamo}

\begin{document}
\begin{exercise}

In the following statements, properties of a differentiable vector-valued function $\vec{r}(t)$ are given.  Select all of the following statements that \emph{must} be true.

\begin{selectAll}
\choice{$\frac{\d}{\d t}  |\vec{r}(t)| = |\vec{r}'(t)|.$}
\choice{If it is known that $| \vec{r}(t) | = 2t^2+4$.  Then,  $| \vec{r}  ' (t) | = 4t$.}
\choice{If $\vec{r}(t)$ is a unit vector for all $t$.  Then $\vec{r}  ' (t)$  is a unit vector for all $t$.}
\choice[correct]{If $|\vec{r}(t)|=2$ for all $t$, then $\dfrac{d}{dt} |\vec{r}(t) | =0$.}
\end{selectAll}

(make sure to use the hint if you would like an explanation of each)

\begin{hint}

By introducing a parameter to describe a curve, we introduce a notion of how to trace out the curve.  We thus have reason to think of the quantities of speed and velocity.

\begin{itemize}
\item $\vec{r}(t)$ gives the position $(x,y,z)$ at a particular time $t$.
\item $|\vec{r}(t)|$ gives the distance the corresponding point $(x,y,z)$ is from the origin.
\item $\vec{r}'(t) = \frac{\d}{\d t} \vec{r}(t)$ is the velocity.
\item  $\left|\vec{r}'(t)\right| = \left|  \frac{\d}{\d t}\vec{r}(t)\right|$ is the speed.
\end{itemize}
Pay close attention to the difference between the last two expressions.  Before tackling the true and false questions below, recall that the same curve can be traced out in many ways.  

\begin{question}
Are you ready to proceed?

\begin{multipleChoice}
\choice[correct]{Yes}
\choice{No}
\end{multipleChoice}

\begin{question}
\begin{itemize}
%%%%%%%%%%%%%%%%%%%%%%%%%%%%%%%%%%%%%%%%%%%%%%%
\item Consider the statement

 \begin{quote}
$\frac{\d}{\d t}  |\vec{r}(t)|  = |\vec{r}'(t)|.$  
\end{quote}

Note that the expression $\frac{\d}{\d t} |\vec{r}(t)|$ is to be
understood as $\frac{\d}{\d t} |\vec{r}(t)|$, which measures how the
distance the particle is from the origin changes in time.  The
righthand side is the speed at which the particle is moving; this
generally does not tell us how the particle is moving away from the
origin.  For instance, if a particle moves in a circle of radius $1$,
the distance from the origin is always $1$, but the particle could be
moving around the circle in many different ways.

Algebraically, note what must be computed on each side. 

\begin{itemize}
\item $\frac{\d}{\d t}  |\vec{r}(t)| = \frac{\d}{\d t} \sqrt{x(t)^2+y(t)^2+z(t)^2}$
\item $|\vec{r}'(t)| =\sqrt{\left(\frac{\d x}{\d t}\right)^2+\left(\frac{\d y}{\d t}\right)^2+\left(\frac{\d z}{\d t}\right)^2}$
\end{itemize}

The chain rule is needed to compute the derivative of the first expression, and there is no reason to believe that the result will be the expression on the righthand side.  To see that this is the case, consider $\vec{r}(t) = \vector{\cos(\omega t),\sin(\omega t)}$, which traces out a circle of radius $1$ in a counterclockwise fashion, and compute both expressions.
%%%%%%%%%%%%%%%%%%%%%%%%%%%%%%%%%%%%%%%%%%%%%%%
\item Consider the statement

\begin{quote}
If it is known that $| \vec{r}(t) | = 2t^2+4$.  Then,  $| \vec{r}  ' (t) | = 4t$.
\end{quote}


Intuitively, $| \vec{r}(t) | =  2t^2+4$ tells us that the distance from the origin a point $(x(t),y(t),z(t))$ is at time $t$.  Knowing this distance should have no relationship with exactly how fast the point on the curve is moving.
 
For a specific counterexample, take $\vec{r}(t)=\vector{(2t^2+4)\cos(t),(2t^2+4)\sin(t)}$.  Then, $|\vec{r}(t)| = 2t^2+4$, but $\vec{r}'(t) = \vector{4t\cos(t)-(2t^2+4)\sin(t), 4t\sin(t)+(2t^2+4)\cos(t)}$.  Now, if $|\vec{r}'(t)|=4t$ for all $t$, it certainly should be true that $|\vec{r}'(\pi)|=4\pi$ (where the choice $t=\pi$ was made since we can check whether $|\vec{r}'(t)| = 4t$ easily there).  However, $\vec{r}(\pi) = \vector{-4\pi,-2\pi^2+4}$, so  $|\vec{r}'(\pi)| = \sqrt{16\pi^2+(2\pi^2+4)^2} > 4 \pi$.

%%%%%%%%%%%%%%%%%%%%%%%%%%%%%%%%%%%%%%%%%%%%%%%
\item Consider the statement

\begin{quote}
If $\vec{r}(t)$ is a unit vector for all $t$.  Then $\vec{r}'(t)$  is a unit vector for all $t$.
\end{quote}

Intuitively, if $\vec{r}(t)$ is a unit vector for all time $t$, this means that the point on the curve at any time $t$ is $1$ unit away from the origin.  In three dimensions, this means that the point could be moving along a sphere of radius $1$.  No matter how quickly it is moving, as long as it stays on the sphere, it will be $1$ unit away from the origin.  Thus, knowing the magnitude of $\vec{r}(t)$ should not give an information about the magnitude of $\vec{r}'(t)$ (i.e. the speed).
 
 For a specific counterexample, take $\vec{r}(t) = \vector{\cos(\omega t),\sin(\omega t)}$ (this describes a particle moving counterclockwise in a circle with angular speed $\omega$).  You can check the following.
 
 \begin{itemize}
 \item $|\vec{r}(t)| = 1$.
 \item $\vec{r}'(t) = \vector{-\omega \sin(\omega t), \omega \cos(\omega t)}$, so $|\vec{r}'(t)| = \omega$.
 \end{itemize}
%%%%%%%%%%%%%%%%%%%%%%%%%%%%%%%%%%%%%%%%%%%%%%%
\item Consider the statement

 \begin{quote}
If $|\vec{r}(t)|=2$ for all $t$, then $\dfrac{d}{dt} |\vec{r}(t) | =0$.
\end{quote}

Make sure to pay attention to what we are given and being asked to find.  We are given that the distance the point on the curve is from the origin is always $2$.  The derivative $\dfrac{\d}{\d t} |\vec{r}(t) |$ will give us how this magnitude changes in time, so it must be $0$.  Note in particular that $\dfrac{\d}{\d t} |\vec{r}(t) |$ is \emph{not} the speed.  Said more explicitly,

\[
\frac{\d}{\d t} |\vec{r}(t)| \neq |\vec{r}'(t)|.
\]
\end{itemize}

\end{question}
\end{question}
 \end{hint}
 \end{exercise}
\end{document}
