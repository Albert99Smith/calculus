\documentclass{ximera}

%\usepackage{todonotes}
%\usepackage{mathtools} %% Required for wide table Curl and Greens
%\usepackage{cuted} %% Required for wide table Curl and Greens
\newcommand{\todo}{}

\usepackage{esint} % for \oiint
\ifxake%%https://math.meta.stackexchange.com/questions/9973/how-do-you-render-a-closed-surface-double-integral
\renewcommand{\oiint}{{\large\bigcirc}\kern-1.56em\iint}
\fi


\graphicspath{
  {./}
  {ximeraTutorial/}
  {basicPhilosophy/}
  {functionsOfSeveralVariables/}
  {normalVectors/}
  {lagrangeMultipliers/}
  {vectorFields/}
  {greensTheorem/}
  {shapeOfThingsToCome/}
  {dotProducts/}
  {partialDerivativesAndTheGradientVector/}
  {../productAndQuotientRules/exercises/}
  {../normalVectors/exercisesParametricPlots/}
  {../continuityOfFunctionsOfSeveralVariables/exercises/}
  {../partialDerivativesAndTheGradientVector/exercises/}
  {../directionalDerivativeAndChainRule/exercises/}
  {../commonCoordinates/exercisesCylindricalCoordinates/}
  {../commonCoordinates/exercisesSphericalCoordinates/}
  {../greensTheorem/exercisesCurlAndLineIntegrals/}
  {../greensTheorem/exercisesDivergenceAndLineIntegrals/}
  {../shapeOfThingsToCome/exercisesDivergenceTheorem/}
  {../greensTheorem/}
  {../shapeOfThingsToCome/}
  {../separableDifferentialEquations/exercises/}
  {vectorFields/}
}

\newcommand{\mooculus}{\textsf{\textbf{MOOC}\textnormal{\textsf{ULUS}}}}

\usepackage{tkz-euclide}\usepackage{tikz}
\usepackage{tikz-cd}
\usetikzlibrary{arrows}
\tikzset{>=stealth,commutative diagrams/.cd,
  arrow style=tikz,diagrams={>=stealth}} %% cool arrow head
\tikzset{shorten <>/.style={ shorten >=#1, shorten <=#1 } } %% allows shorter vectors

\usetikzlibrary{backgrounds} %% for boxes around graphs
\usetikzlibrary{shapes,positioning}  %% Clouds and stars
\usetikzlibrary{matrix} %% for matrix
\usepgfplotslibrary{polar} %% for polar plots
\usepgfplotslibrary{fillbetween} %% to shade area between curves in TikZ
\usetkzobj{all}
\usepackage[makeroom]{cancel} %% for strike outs
%\usepackage{mathtools} %% for pretty underbrace % Breaks Ximera
%\usepackage{multicol}
\usepackage{pgffor} %% required for integral for loops



%% http://tex.stackexchange.com/questions/66490/drawing-a-tikz-arc-specifying-the-center
%% Draws beach ball
\tikzset{pics/carc/.style args={#1:#2:#3}{code={\draw[pic actions] (#1:#3) arc(#1:#2:#3);}}}



\usepackage{array}
\setlength{\extrarowheight}{+.1cm}
\newdimen\digitwidth
\settowidth\digitwidth{9}
\def\divrule#1#2{
\noalign{\moveright#1\digitwidth
\vbox{\hrule width#2\digitwidth}}}





\newcommand{\RR}{\mathbb R}
\newcommand{\R}{\mathbb R}
\newcommand{\N}{\mathbb N}
\newcommand{\Z}{\mathbb Z}

\newcommand{\sagemath}{\textsf{SageMath}}


%\renewcommand{\d}{\,d\!}
\renewcommand{\d}{\mathop{}\!d}
\newcommand{\dd}[2][]{\frac{\d #1}{\d #2}}
\newcommand{\pp}[2][]{\frac{\partial #1}{\partial #2}}
\renewcommand{\l}{\ell}
\newcommand{\ddx}{\frac{d}{\d x}}

\newcommand{\zeroOverZero}{\ensuremath{\boldsymbol{\tfrac{0}{0}}}}
\newcommand{\inftyOverInfty}{\ensuremath{\boldsymbol{\tfrac{\infty}{\infty}}}}
\newcommand{\zeroOverInfty}{\ensuremath{\boldsymbol{\tfrac{0}{\infty}}}}
\newcommand{\zeroTimesInfty}{\ensuremath{\small\boldsymbol{0\cdot \infty}}}
\newcommand{\inftyMinusInfty}{\ensuremath{\small\boldsymbol{\infty - \infty}}}
\newcommand{\oneToInfty}{\ensuremath{\boldsymbol{1^\infty}}}
\newcommand{\zeroToZero}{\ensuremath{\boldsymbol{0^0}}}
\newcommand{\inftyToZero}{\ensuremath{\boldsymbol{\infty^0}}}



\newcommand{\numOverZero}{\ensuremath{\boldsymbol{\tfrac{\#}{0}}}}
\newcommand{\dfn}{\textbf}
%\newcommand{\unit}{\,\mathrm}
\newcommand{\unit}{\mathop{}\!\mathrm}
\newcommand{\eval}[1]{\bigg[ #1 \bigg]}
\newcommand{\seq}[1]{\left( #1 \right)}
\renewcommand{\epsilon}{\varepsilon}
\renewcommand{\phi}{\varphi}


\renewcommand{\iff}{\Leftrightarrow}

\DeclareMathOperator{\arccot}{arccot}
\DeclareMathOperator{\arcsec}{arcsec}
\DeclareMathOperator{\arccsc}{arccsc}
\DeclareMathOperator{\si}{Si}
\DeclareMathOperator{\scal}{scal}
\DeclareMathOperator{\sign}{sign}


%% \newcommand{\tightoverset}[2]{% for arrow vec
%%   \mathop{#2}\limits^{\vbox to -.5ex{\kern-0.75ex\hbox{$#1$}\vss}}}
\newcommand{\arrowvec}[1]{{\overset{\rightharpoonup}{#1}}}
%\renewcommand{\vec}[1]{\arrowvec{\mathbf{#1}}}
\renewcommand{\vec}[1]{{\overset{\boldsymbol{\rightharpoonup}}{\mathbf{#1}}}\hspace{0in}}

\newcommand{\point}[1]{\left(#1\right)} %this allows \vector{ to be changed to \vector{ with a quick find and replace
\newcommand{\pt}[1]{\mathbf{#1}} %this allows \vec{ to be changed to \vec{ with a quick find and replace
\newcommand{\Lim}[2]{\lim_{\point{#1} \to \point{#2}}} %Bart, I changed this to point since I want to use it.  It runs through both of the exercise and exerciseE files in limits section, which is why it was in each document to start with.

\DeclareMathOperator{\proj}{\mathbf{proj}}
\newcommand{\veci}{{\boldsymbol{\hat{\imath}}}}
\newcommand{\vecj}{{\boldsymbol{\hat{\jmath}}}}
\newcommand{\veck}{{\boldsymbol{\hat{k}}}}
\newcommand{\vecl}{\vec{\boldsymbol{\l}}}
\newcommand{\uvec}[1]{\mathbf{\hat{#1}}}
\newcommand{\utan}{\mathbf{\hat{t}}}
\newcommand{\unormal}{\mathbf{\hat{n}}}
\newcommand{\ubinormal}{\mathbf{\hat{b}}}

\newcommand{\dotp}{\bullet}
\newcommand{\cross}{\boldsymbol\times}
\newcommand{\grad}{\boldsymbol\nabla}
\newcommand{\divergence}{\grad\dotp}
\newcommand{\curl}{\grad\cross}
%\DeclareMathOperator{\divergence}{divergence}
%\DeclareMathOperator{\curl}[1]{\grad\cross #1}
\newcommand{\lto}{\mathop{\longrightarrow\,}\limits}

\renewcommand{\bar}{\overline}

\colorlet{textColor}{black}
\colorlet{background}{white}
\colorlet{penColor}{blue!50!black} % Color of a curve in a plot
\colorlet{penColor2}{red!50!black}% Color of a curve in a plot
\colorlet{penColor3}{red!50!blue} % Color of a curve in a plot
\colorlet{penColor4}{green!50!black} % Color of a curve in a plot
\colorlet{penColor5}{orange!80!black} % Color of a curve in a plot
\colorlet{penColor6}{yellow!70!black} % Color of a curve in a plot
\colorlet{fill1}{penColor!20} % Color of fill in a plot
\colorlet{fill2}{penColor2!20} % Color of fill in a plot
\colorlet{fillp}{fill1} % Color of positive area
\colorlet{filln}{penColor2!20} % Color of negative area
\colorlet{fill3}{penColor3!20} % Fill
\colorlet{fill4}{penColor4!20} % Fill
\colorlet{fill5}{penColor5!20} % Fill
\colorlet{gridColor}{gray!50} % Color of grid in a plot

\newcommand{\surfaceColor}{violet}
\newcommand{\surfaceColorTwo}{redyellow}
\newcommand{\sliceColor}{greenyellow}




\pgfmathdeclarefunction{gauss}{2}{% gives gaussian
  \pgfmathparse{1/(#2*sqrt(2*pi))*exp(-((x-#1)^2)/(2*#2^2))}%
}


%%%%%%%%%%%%%
%% Vectors
%%%%%%%%%%%%%

%% Simple horiz vectors
\renewcommand{\vector}[1]{\left\langle #1\right\rangle}


%% %% Complex Horiz Vectors with angle brackets
%% \makeatletter
%% \renewcommand{\vector}[2][ , ]{\left\langle%
%%   \def\nextitem{\def\nextitem{#1}}%
%%   \@for \el:=#2\do{\nextitem\el}\right\rangle%
%% }
%% \makeatother

%% %% Vertical Vectors
%% \def\vector#1{\begin{bmatrix}\vecListA#1,,\end{bmatrix}}
%% \def\vecListA#1,{\if,#1,\else #1\cr \expandafter \vecListA \fi}

%%%%%%%%%%%%%
%% End of vectors
%%%%%%%%%%%%%

%\newcommand{\fullwidth}{}
%\newcommand{\normalwidth}{}



%% makes a snazzy t-chart for evaluating functions
%\newenvironment{tchart}{\rowcolors{2}{}{background!90!textColor}\array}{\endarray}

%%This is to help with formatting on future title pages.
\newenvironment{sectionOutcomes}{}{}



%% Flowchart stuff
%\tikzstyle{startstop} = [rectangle, rounded corners, minimum width=3cm, minimum height=1cm,text centered, draw=black]
%\tikzstyle{question} = [rectangle, minimum width=3cm, minimum height=1cm, text centered, draw=black]
%\tikzstyle{decision} = [trapezium, trapezium left angle=70, trapezium right angle=110, minimum width=3cm, minimum height=1cm, text centered, draw=black]
%\tikzstyle{question} = [rectangle, rounded corners, minimum width=3cm, minimum height=1cm,text centered, draw=black]
%\tikzstyle{process} = [rectangle, minimum width=3cm, minimum height=1cm, text centered, draw=black]
%\tikzstyle{decision} = [trapezium, trapezium left angle=70, trapezium right angle=110, minimum width=3cm, minimum height=1cm, text centered, draw=black]


\author{Jim Talamo}
\license{Creative Commons 3.0 By-NC}


\outcome{Understand the integral test remainder estimates.}

\begin{document}

\begin{exercise}

Consider the series $\sum_{k=1}^{\infty} \frac{2k}{(k^2+1)^2}$.  

Does the integral test apply?

\begin{multipleChoice}
\choice[correct]{Yes.}
\choice{No.}
\end{multipleChoice}

What does the integral test tell us?
\begin{multipleChoice}
\choice{The series converges to $\frac{1}{2}$.}
\choice[correct]{The series converges, but we do not know its value yet.}
\choice{The series diverges.}
\end{multipleChoice}

\begin{feedback}
Note that $\frac{2x}{(x^2+1)^2}$ is positive, decreasing(check for yourself!), and continuous for all $x \geq 1$, and

\[
\int_1^\infty \frac{2x}{(x^2+1)^2} \d x = \lim_{b \to \infty} \eval{-\frac{1}{x^2+1}}_1^b = \lim_{b \to \infty} \eval{-\frac{1}{b^2+1}+\frac{1}{2}}.
\]
Hence, the improper integral converges, so $\sum_{k=1}^{\infty}  \frac{2k}{(k^2+1)^2}$ converges by the integral test.  

Suppose that we want to estimate the value of the series.  Since we do not have a way to find an explicit formula for $s_n=\sum_{k=1}^n \frac{2k}{(k^2+1)^2}$, we need to turn to the remainder estimate that comes with the integral test.

\begin{theorem}[Integral Remainder Estimates]
If $f(x)$ is a function that is positive, increasing, and continuous for $x \geq n_0$,  and $f(n) = a_n$ for every $n \geq n_0$, then 

\[
\int_{n+1}^{\infty} f(x) \d x \leq  r_n \leq \int_{n}^{\infty} f(x) \d x \qquad \textrm{ for all } n \geq n_0,
\]
where $r_n = \sum_{k=n+1}^{\infty} a_k$.
\end{theorem}

\end{feedback}


\begin{exercise}
Suppose we want to approximate $\sum_{k=1}^{\infty} \frac{2k}{(k^2+1)^2}$ as best as we can by using $\sum_{k=1}^{15} \frac{2k}{(k^2+1)^2}$.  

How will we find our approximation?

\begin{multipleChoice}
\choice{Our approximation is $\sum_{k=1}^{15} \frac{2k}{(k^2+1)^2}$.}
\choice[correct]{Our approximation will be made from $\sum_{k=1}^{N} \frac{2k}{(k^2+1)^2}$ and the upper and lower bounds for the error.}
\end{multipleChoice}

\textbf{First, we can determine an upper bound for the error from the inequality $ r_n \leq \int_{n}^{\infty} f(x) \d x$. } 

We can mimic the calculation done earlier

\[
\int_n^\infty \frac{2x}{(x^2+1)^2} \d x = \lim_{b \to \infty} \eval{\answer{-\frac{1}{x^2+1}}}_n^b = \lim_{b \to \infty} \eval{-\frac{1}{b^2+1}+\frac{1}{n^2+1}} =\answer{\frac{1}{n^2+1}}. 
\]

Since we want to approximate $\sum_{k=1}^{\infty} \frac{2k}{(k^2+1)^2}$ by using $\sum_{k=1}^{15} \frac{2k}{(k^2+1)^2}$, we should use $n=\answer{15}$, and doing so gives us that

\[
error \leq \answer{\frac{1}{15^2+1}} \approx .0044
\]

This tells us that $\sum_{k=1}^{15} \frac{2k}{(k^2+1)^2}$ will be no more than $.0044$ off from $\sum_{k=1}^{\infty} \frac{2k}{(k^2+1)^2}$.

\textbf{Secondly, we can determine a lower bound for the error from the inequality $ \int_{n+1}^{\infty} f(x) \d x  \leq r_n $. } 

We can mimic the calculation done earlier

\[
\int_{n+1}^\infty \frac{2x}{(x^2+1)^2} \d x = \lim_{b \to \infty} \eval{\answer{-\frac{1}{x^2+1}}}_{n+1}^b = \lim_{b \to \infty} \eval{-\frac{1}{b^2+1}+\frac{1}{(n+1)^2+1}} =\answer{\frac{1}{(n+1)^2+1}}. 
\]

Since we want to approximate $\sum_{k=1}^{\infty} \frac{2k}{(k^2+1)^2}$ by using $\sum_{k=1}^{15} \frac{2k}{(k^2+1)^2}$, we should use $n=\answer{15}$, and doing so gives us that

\[
error \geq \answer{\frac{1}{16^2+1}} \approx .0039
\]

This tells us that $\sum_{k=1}^{15} \frac{2k}{(k^2+1)^2}$ will be at least $.0039$ off from the exact value of  $\sum_{k=1}^{\infty} \frac{2k}{(k^2+1)^2}$.

\begin{exercise}
Now, we can use technology to compute that to four decimal places, $\sum_{k=1}^{15} \frac{2k}{(k^2+1)^2} = \answer[tolerance=.0002]{.7901}$.  This allows us to determine bounds for the value of $\sum_{k=1}^{\infty} \frac{2k}{(k^2+1)^2}$.

\[
\textrm{appx} + \textrm{lower error bound}  \leq \sum_{k=1}^{\infty} \frac{2k}{(k^2+1)^2}  \leq \textrm{appx} + \textrm{upper error bound} 
\]
\begin{align*}
.7901 +.0039 &\leq \sum_{k=1}^{\infty} \frac{2k}{(k^2+1)^2} \leq .7901 +.0044 \\
.7939 &\leq  \sum_{k=1}^{\infty} \frac{2k}{(k^2+1)^2} \leq .7945 
\end{align*}

\begin{exercise}
Now suppose that we want to ensure that we can approximate $\sum_{k=1}^{\infty} \frac{2k}{(k^2+1)^2}$ to within $.001$ of its exact value.  

To explore how the error  bounds sharpen our estimate, suppose that we find $N$ so $\sum_{k=1}^{N} \frac{2k}{(k^2+1)^2}$ is within $.001$ of $\sum_{k=1}^{\infty} \frac{2k}{(k^2+1)^2}$.  To do so, we will use the upper bound for error; that is, we will set $r_N \leq \int_N^\infty \frac{2x}{(x^2+1)^2} \d x \leq .001$.

From our previous work, we have found that

\[
 \int_N^\infty \frac{2x}{(x^2+1)^2} \d x= \answer{\frac{1}{N^2+1}},
\]

so we just have to solve the inequality below.

\begin{align*}
 \int_N^\infty \frac{2x}{(x^2+1)^2} \d x =  \answer{\frac{1}{N^2+1}} &\leq .001 \\
N^2+1 &\geq 1000 \\
N \geq \sqrt{\answer{999}}
\end{align*}

Using technology, we find $N \geq 31.6$, so we use $N=32$.

Our approximation is thus $\sum_{k=1}^{32} \frac{2k}{(k^2+1)^2} = \answer[tolerance=.002]{.7933}$

\begin{exercise}
Of course, there's a much more efficient way to estimate by using both error bounds.  Note that since the terms in the series are all positive, this estimate is must be an \wordChoice{\choice{overestimate}\choice[correct]{underestimate}}.  Had we used $N=32$ and repeat the construction in the previous part to find new error bounds for the actual value of the series, which gives us a much better idea of the value of the series!

The lower bound for the error
\[
\textrm{appx} + \textrm{lower error bound}  \leq \sum_{k=1}^{\infty} \frac{2k}{(k^2+1)^2}  \leq \textrm{appx} + \textrm{upper error bound} 
\]
\begin{align*}
.7933 +.0009 &\leq \sum_{k=1}^{\infty} \frac{2k}{(k^2+1)^2} \leq .7933 +.001 \\
.7942 &\leq  \sum_{k=1}^{\infty} \frac{2k}{(k^2+1)^2} \leq .7943 
\end{align*}

A much more computationally efficient is to require that the difference between the upper estimate for the series and the lower estimate be no more than $.001$; that is note that for every $n$,

\[
\int_{n+1}^\infty \frac{2x}{(x^2+1)^2} \d x  +s_n \leq \sum_{k=1}^{\infty} a_k \leq \int_{n}^\infty \frac{2x}{(x^2+1)^2} \d x  +s_n.
\]
 
If the difference between the left and right sides of the inequality is less than $.001$, then the infinite series must be within $.001$ of its minimum possible value (the lefthand side) and its maximum possible value (the righthand side).
 
 Now, note that

\begin{align*}
RHS-LHS &\leq .001 \\
\left( \int_{n}^\infty \frac{2x}{(x^2+1)^2} \d x  +s_n\right) - \left( \int_{n+1}^\infty \frac{2x}{(x^2+1)^2} \d x  +s_n \right) &\leq .001 \\
 \int_{n}^\infty \frac{2x}{(x^2+1)^2} \d x  - \int_{n+1}^\infty \frac{2x}{(x^2+1)^2} \d x   &\leq .001 \\
  \int_{n}^\infty \frac{2x}{(x^2+1)^2} \d x  +  \int_{\infty}^{n+1} \frac{2x}{(x^2+1)^2} \d x   &\leq .001 \\
    \int_{n}^{n+1} \frac{2x}{(x^2+1)^2}  &\leq .001 \\
\end{align*} 

Noting that $\int_N^{N+1}  \frac{2x}{(x^2+1)^2} \d x = \answer{\frac{1}{N^2+1}}  - \frac{1}{(N+1)^2+1}$, we can find that the smallest integer value for $N$ for which $\int_N^{N+1}  \frac{2x}{(x^2+1)^2} \d x \leq .001$.  We simply must solve

\[
 \frac{1}{N^2+1}-\frac{1}{(N+1)^2+1}  \leq .001,
\]

and technology shows us that $N=13.06$, so we take is $N= \answer{14}$.  This is much better than the previous value found of $N=32$!
 
 We now know that the actual value for the series must be between $s_{14} + \int_{15}^\infty \frac{2x}{(x^2+1)^2} \d x = \answer[tolerance=.002]{.7934}$ and $s_{14} + \int_{14}^\infty \frac{2x}{(x^2+1)^2} \d x = \answer[tolerance=.002]{.7939}$

\end{exercise}
\end{exercise}
\end{exercise}








\end{exercise}
\end{exercise}
\end{document}