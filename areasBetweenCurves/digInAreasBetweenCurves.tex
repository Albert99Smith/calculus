\documentclass{ximera}
%\usepackage{todonotes}
%\usepackage{mathtools} %% Required for wide table Curl and Greens
%\usepackage{cuted} %% Required for wide table Curl and Greens
\newcommand{\todo}{}

\usepackage{esint} % for \oiint
\ifxake%%https://math.meta.stackexchange.com/questions/9973/how-do-you-render-a-closed-surface-double-integral
\renewcommand{\oiint}{{\large\bigcirc}\kern-1.56em\iint}
\fi


\graphicspath{
  {./}
  {ximeraTutorial/}
  {basicPhilosophy/}
  {functionsOfSeveralVariables/}
  {normalVectors/}
  {lagrangeMultipliers/}
  {vectorFields/}
  {greensTheorem/}
  {shapeOfThingsToCome/}
  {dotProducts/}
  {partialDerivativesAndTheGradientVector/}
  {../productAndQuotientRules/exercises/}
  {../normalVectors/exercisesParametricPlots/}
  {../continuityOfFunctionsOfSeveralVariables/exercises/}
  {../partialDerivativesAndTheGradientVector/exercises/}
  {../directionalDerivativeAndChainRule/exercises/}
  {../commonCoordinates/exercisesCylindricalCoordinates/}
  {../commonCoordinates/exercisesSphericalCoordinates/}
  {../greensTheorem/exercisesCurlAndLineIntegrals/}
  {../greensTheorem/exercisesDivergenceAndLineIntegrals/}
  {../shapeOfThingsToCome/exercisesDivergenceTheorem/}
  {../greensTheorem/}
  {../shapeOfThingsToCome/}
  {../separableDifferentialEquations/exercises/}
  {vectorFields/}
}

\newcommand{\mooculus}{\textsf{\textbf{MOOC}\textnormal{\textsf{ULUS}}}}

\usepackage{tkz-euclide}\usepackage{tikz}
\usepackage{tikz-cd}
\usetikzlibrary{arrows}
\tikzset{>=stealth,commutative diagrams/.cd,
  arrow style=tikz,diagrams={>=stealth}} %% cool arrow head
\tikzset{shorten <>/.style={ shorten >=#1, shorten <=#1 } } %% allows shorter vectors

\usetikzlibrary{backgrounds} %% for boxes around graphs
\usetikzlibrary{shapes,positioning}  %% Clouds and stars
\usetikzlibrary{matrix} %% for matrix
\usepgfplotslibrary{polar} %% for polar plots
\usepgfplotslibrary{fillbetween} %% to shade area between curves in TikZ
\usetkzobj{all}
\usepackage[makeroom]{cancel} %% for strike outs
%\usepackage{mathtools} %% for pretty underbrace % Breaks Ximera
%\usepackage{multicol}
\usepackage{pgffor} %% required for integral for loops



%% http://tex.stackexchange.com/questions/66490/drawing-a-tikz-arc-specifying-the-center
%% Draws beach ball
\tikzset{pics/carc/.style args={#1:#2:#3}{code={\draw[pic actions] (#1:#3) arc(#1:#2:#3);}}}



\usepackage{array}
\setlength{\extrarowheight}{+.1cm}
\newdimen\digitwidth
\settowidth\digitwidth{9}
\def\divrule#1#2{
\noalign{\moveright#1\digitwidth
\vbox{\hrule width#2\digitwidth}}}





\newcommand{\RR}{\mathbb R}
\newcommand{\R}{\mathbb R}
\newcommand{\N}{\mathbb N}
\newcommand{\Z}{\mathbb Z}

\newcommand{\sagemath}{\textsf{SageMath}}


%\renewcommand{\d}{\,d\!}
\renewcommand{\d}{\mathop{}\!d}
\newcommand{\dd}[2][]{\frac{\d #1}{\d #2}}
\newcommand{\pp}[2][]{\frac{\partial #1}{\partial #2}}
\renewcommand{\l}{\ell}
\newcommand{\ddx}{\frac{d}{\d x}}

\newcommand{\zeroOverZero}{\ensuremath{\boldsymbol{\tfrac{0}{0}}}}
\newcommand{\inftyOverInfty}{\ensuremath{\boldsymbol{\tfrac{\infty}{\infty}}}}
\newcommand{\zeroOverInfty}{\ensuremath{\boldsymbol{\tfrac{0}{\infty}}}}
\newcommand{\zeroTimesInfty}{\ensuremath{\small\boldsymbol{0\cdot \infty}}}
\newcommand{\inftyMinusInfty}{\ensuremath{\small\boldsymbol{\infty - \infty}}}
\newcommand{\oneToInfty}{\ensuremath{\boldsymbol{1^\infty}}}
\newcommand{\zeroToZero}{\ensuremath{\boldsymbol{0^0}}}
\newcommand{\inftyToZero}{\ensuremath{\boldsymbol{\infty^0}}}



\newcommand{\numOverZero}{\ensuremath{\boldsymbol{\tfrac{\#}{0}}}}
\newcommand{\dfn}{\textbf}
%\newcommand{\unit}{\,\mathrm}
\newcommand{\unit}{\mathop{}\!\mathrm}
\newcommand{\eval}[1]{\bigg[ #1 \bigg]}
\newcommand{\seq}[1]{\left( #1 \right)}
\renewcommand{\epsilon}{\varepsilon}
\renewcommand{\phi}{\varphi}


\renewcommand{\iff}{\Leftrightarrow}

\DeclareMathOperator{\arccot}{arccot}
\DeclareMathOperator{\arcsec}{arcsec}
\DeclareMathOperator{\arccsc}{arccsc}
\DeclareMathOperator{\si}{Si}
\DeclareMathOperator{\scal}{scal}
\DeclareMathOperator{\sign}{sign}


%% \newcommand{\tightoverset}[2]{% for arrow vec
%%   \mathop{#2}\limits^{\vbox to -.5ex{\kern-0.75ex\hbox{$#1$}\vss}}}
\newcommand{\arrowvec}[1]{{\overset{\rightharpoonup}{#1}}}
%\renewcommand{\vec}[1]{\arrowvec{\mathbf{#1}}}
\renewcommand{\vec}[1]{{\overset{\boldsymbol{\rightharpoonup}}{\mathbf{#1}}}\hspace{0in}}

\newcommand{\point}[1]{\left(#1\right)} %this allows \vector{ to be changed to \vector{ with a quick find and replace
\newcommand{\pt}[1]{\mathbf{#1}} %this allows \vec{ to be changed to \vec{ with a quick find and replace
\newcommand{\Lim}[2]{\lim_{\point{#1} \to \point{#2}}} %Bart, I changed this to point since I want to use it.  It runs through both of the exercise and exerciseE files in limits section, which is why it was in each document to start with.

\DeclareMathOperator{\proj}{\mathbf{proj}}
\newcommand{\veci}{{\boldsymbol{\hat{\imath}}}}
\newcommand{\vecj}{{\boldsymbol{\hat{\jmath}}}}
\newcommand{\veck}{{\boldsymbol{\hat{k}}}}
\newcommand{\vecl}{\vec{\boldsymbol{\l}}}
\newcommand{\uvec}[1]{\mathbf{\hat{#1}}}
\newcommand{\utan}{\mathbf{\hat{t}}}
\newcommand{\unormal}{\mathbf{\hat{n}}}
\newcommand{\ubinormal}{\mathbf{\hat{b}}}

\newcommand{\dotp}{\bullet}
\newcommand{\cross}{\boldsymbol\times}
\newcommand{\grad}{\boldsymbol\nabla}
\newcommand{\divergence}{\grad\dotp}
\newcommand{\curl}{\grad\cross}
%\DeclareMathOperator{\divergence}{divergence}
%\DeclareMathOperator{\curl}[1]{\grad\cross #1}
\newcommand{\lto}{\mathop{\longrightarrow\,}\limits}

\renewcommand{\bar}{\overline}

\colorlet{textColor}{black}
\colorlet{background}{white}
\colorlet{penColor}{blue!50!black} % Color of a curve in a plot
\colorlet{penColor2}{red!50!black}% Color of a curve in a plot
\colorlet{penColor3}{red!50!blue} % Color of a curve in a plot
\colorlet{penColor4}{green!50!black} % Color of a curve in a plot
\colorlet{penColor5}{orange!80!black} % Color of a curve in a plot
\colorlet{penColor6}{yellow!70!black} % Color of a curve in a plot
\colorlet{fill1}{penColor!20} % Color of fill in a plot
\colorlet{fill2}{penColor2!20} % Color of fill in a plot
\colorlet{fillp}{fill1} % Color of positive area
\colorlet{filln}{penColor2!20} % Color of negative area
\colorlet{fill3}{penColor3!20} % Fill
\colorlet{fill4}{penColor4!20} % Fill
\colorlet{fill5}{penColor5!20} % Fill
\colorlet{gridColor}{gray!50} % Color of grid in a plot

\newcommand{\surfaceColor}{violet}
\newcommand{\surfaceColorTwo}{redyellow}
\newcommand{\sliceColor}{greenyellow}




\pgfmathdeclarefunction{gauss}{2}{% gives gaussian
  \pgfmathparse{1/(#2*sqrt(2*pi))*exp(-((x-#1)^2)/(2*#2^2))}%
}


%%%%%%%%%%%%%
%% Vectors
%%%%%%%%%%%%%

%% Simple horiz vectors
\renewcommand{\vector}[1]{\left\langle #1\right\rangle}


%% %% Complex Horiz Vectors with angle brackets
%% \makeatletter
%% \renewcommand{\vector}[2][ , ]{\left\langle%
%%   \def\nextitem{\def\nextitem{#1}}%
%%   \@for \el:=#2\do{\nextitem\el}\right\rangle%
%% }
%% \makeatother

%% %% Vertical Vectors
%% \def\vector#1{\begin{bmatrix}\vecListA#1,,\end{bmatrix}}
%% \def\vecListA#1,{\if,#1,\else #1\cr \expandafter \vecListA \fi}

%%%%%%%%%%%%%
%% End of vectors
%%%%%%%%%%%%%

%\newcommand{\fullwidth}{}
%\newcommand{\normalwidth}{}



%% makes a snazzy t-chart for evaluating functions
%\newenvironment{tchart}{\rowcolors{2}{}{background!90!textColor}\array}{\endarray}

%%This is to help with formatting on future title pages.
\newenvironment{sectionOutcomes}{}{}



%% Flowchart stuff
%\tikzstyle{startstop} = [rectangle, rounded corners, minimum width=3cm, minimum height=1cm,text centered, draw=black]
%\tikzstyle{question} = [rectangle, minimum width=3cm, minimum height=1cm, text centered, draw=black]
%\tikzstyle{decision} = [trapezium, trapezium left angle=70, trapezium right angle=110, minimum width=3cm, minimum height=1cm, text centered, draw=black]
%\tikzstyle{question} = [rectangle, rounded corners, minimum width=3cm, minimum height=1cm,text centered, draw=black]
%\tikzstyle{process} = [rectangle, minimum width=3cm, minimum height=1cm, text centered, draw=black]
%\tikzstyle{decision} = [trapezium, trapezium left angle=70, trapezium right angle=110, minimum width=3cm, minimum height=1cm, text centered, draw=black]

\pgfplotsset{compat=1.13}
\author{Jim Talamo}

\outcome{Introduce the method of ``Slice, Approximate, Integrate" to set up Riemann integrals}
\outcome{Find the bounded area between two curves.}
\outcome{Express the area between curves as an integral or sum of integrals with respect to $x$ or $y$.}
\outcome{Decide whether to integrate with respect to $x$ or $y$.}


\title[Dig-In:]{Area between curves}
 
\begin{document}
\begin{abstract}
  We introduce the procedure of ``Slice, Approximate, Integrate" and use it study the area of a region between two curves using the definite integral.
\end{abstract}
\maketitle

We have seen previously that for continuous functions defined on closed intervals, the Fundamental Theorem of Calculus relates the process of finding antiderivatives to calculating certain areas.  As it turns out the process used to transcribe Riemann sums that \emph{approximate} such areas to definite integrals that give an \emph{exact} answer is a fundamental procedure that applies to numerous examples in both mathematics and other STEM fields.  In the upcoming sections, we will apply this process to many different cases of interest.  As an opening remark, we deal only with piecewise continuously differentiable functions unless otherwise noted.  
%%%%FINISH THIS

\section{The Fundamental Theorem of Calculus and Areas}

We begin the section with an example that illustrates the concepts that are fundamental in setting up definite integrals.  Consider a continuous function $f(x)$ that is positive on a closed interval $[a,b]$ in its domain.   Suppose that we are interested in finding the area bounded by the graph of $y=f(x)$ and the $x$-axis between $x=a$ and $x=b$.  Below is an example of such an example.


\begin{image}
\begin{tikzpicture}

\begin{axis}
	[
	domain=0:6, ymax=1.75,xmax=6, ymin=0, xmin=0,
	axis lines=center, xlabel=$x$, ylabel=$y$,
	xtick={1.5,4.5},
	xticklabels={$a$,$b$},
	ymajorticks=false,
	every axis y label/.style={at=(current axis.above origin),anchor=south},
	every axis x label/.style={at=(current axis.right of origin),anchor=west},
	axis on top,
	typeset ticklabels with strut,
	]

	\addplot [draw=penColor,very thick, smooth] {.4*sin(deg(x)) + 1};
	
	\addplot [name path=A,domain=1.5:4.5,draw=none] {.4*sin(deg(x)) + 1};   
	\addplot [name path=B,domain=1.5:4.5,draw=none] {0};
	\addplot [fillp] fill between[of=A and B];
	
	\draw[penColor,thick] (1.5,0) -- (1.5,{.4*sin(deg(1.5)) + 1});
	\draw[penColor,thick] (4.5,0) -- (4.5,{.4*sin(deg(4.5)) + 1});
	
	\node at (axis cs:3,1.5) [penColor] {$y=f(x)$};
\end{axis}

\end{tikzpicture}
\end{image}

We can write this area as the definite integral 

\[A = \int_{x=a}^{x=b} f(x) \d x , \]

and we may use the Fundamental Theorem of Calculus to evaluate it.  However, recalling how this result was obtained in the first place is instructive.  Understanding the logic behind it is essential in order to apply a similar method to set up integrals to model other types of situations.  We thus give a detailed conceptual outline of the argument here.

\paragraph{Step 1: Slice}

Since we have expressed $y$ as an function of $x$, we divide the interval $[a,b]$ into $n$ pieces of uniform width $\Delta x$. 

\begin{image}
\begin{tikzpicture}

\begin{axis}
	[
	domain=0:6, ymax=1.75,xmax=6, ymin=-.25, xmin=0,
	axis lines=center, xlabel=$x$, ylabel=$y$,
	xtick={1.5,4.5},
	xticklabels={$a$,$b$},
	ymajorticks=false,
	every axis y label/.style={at=(current axis.above origin),anchor=south},
	every axis x label/.style={at=(current axis.right of origin),anchor=west},
	axis on top,
	typeset ticklabels with strut,
	]

	\addplot [draw=penColor,very thick, smooth] {.4*sin(deg(x)) + 1};
	
	\node at (axis cs:3,1.5) [penColor] {$y=f(x)$};
	
	\addplot [name path=A,domain=1.5:4.5,draw=none] {.4*sin(deg(x)) + 1};   
	\addplot [name path=B,domain=1.5:4.5,draw=penColor,thick] {0};
	\addplot [fillp] fill between[of=A and B];

	\pgfplotsinvokeforeach{1.5,2.25,3,3.75,4.5}	
	{
		\draw[black,thick] 
		(axis cs:#1,0) -- (axis cs:#1,{.4*sin(deg(#1)) + 1});
	}
	
	\node at (axis cs:2.625,-.15) [penColor] {$\Delta x$};
	
	\draw[penColor, |-|] (axis cs:2.25,-.07) -- (axis cs:3,-.07);
\end{axis}

\end{tikzpicture}
\end{image}


As a note, using rectangles of equal width is not is not a requirement.  In a more theoretical treatment, this can be treated, but it suffices for us to use uniform widths in order to present more conceptually tractable examples.  As such, we adopt this convention here as well as in the coming sections. 


\paragraph{Step 2: Approximate}

We cannot determine the exact area of the slice, but we can approximate each slice by a rectangle whose heights are determined by the value of the function $y = f(x)$ at some $x$-value on the base of the rectangle.  Here, we use the righthand endpoint to determine the height of each rectangle.

\begin{image}
\begin{tikzpicture}

\begin{axis}
	[
	domain=0:6, ymax=1.75,xmax=6, ymin=-.25, xmin=0,
	axis lines=center, xlabel=$x$, ylabel=$y$,
	xtick={1.5,4.5},
	xticklabels={$a$,$b$},
	ymajorticks=false,
	every axis y label/.style={at=(current axis.above origin),anchor=south},
	every axis x label/.style={at=(current axis.right of origin),anchor=west},
	axis on top,
	typeset ticklabels with strut,
	]

	\addplot [draw=penColor,very thick, smooth] {.4*sin(deg(x)) + 1};
	
	\node at (axis cs:3,1.5) [penColor] {$y=f(x)$};

	\pgfplotsinvokeforeach{2.25,3,3.75,4.5}	
	{
		\draw[black,thick,fill=fillp] 
		(axis cs:#1-.75,0)
		-- (axis cs:#1-.75,{.4*sin(deg(#1)) + 1}) 
		-- (axis cs:#1,{.4*sin(deg(#1)) + 1}) 
		-- (axis cs:#1,0) 
		-- (axis cs:#1-.75,0);
	}
	
	\node at (axis cs:2.625,-.15) [penColor] {$\Delta x$};
	
	\draw[penColor, |-|] (axis cs:2.25,-.07) -- (axis cs:3,-.07);
\end{axis}

\end{tikzpicture}
\end{image}

We can now find the area $\Delta A_k$ of the $k$-th rectangle. 
\begin{align}
\Delta A_k & = (height) \times (width) \nonumber \\
\Delta A_k &= f\left(x_k^*\right)\Delta x 
\end{align}
where $x_k^*$ here is the $x$-value of the righthand endpoint of each rectangle.  The height of the rectangle is thus $f(x_k^*)$. 

Let $S_n$ denote the total area obtained by adding the areas of the $n$ rectangles together.   Then, we can compute $S_n$ easily by adding up the areas of all of the rectangles: $$S_n = \Delta A_1 + \Delta A_2 + \ldots \Delta A_n$$
or if you prefer using sigma notation: 
\begin{equation}
S_n =\sum_{k=1}^{n} \Delta A_k =  \sum_{k=1}^n f(x_k^*) \Delta x 
\end{equation}
Note that as we use more rectangles, the following occur \emph{simultaneously}:

\begin{itemize}
\item[1.] The width $\Delta x$ of each rectangle decreases.
\item[2.] The total number of rectangles increases.
\item[3.] The sum of the areas of the rectangles becomes closer to the actual area.
\end{itemize}

The actual area is then $A = \lim_{n \rightarrow \infty} \left[ \sum_{k=1}^n f(x_k^*) \Delta x \right]$.

%%%INTERACTIVE FIGURE%%%%%%%

\paragraph{Step 3: Integrate}

While this infinite limit can be quite cumbersome to work out in even the simplest cases, the Fundamental Theorem of Calculus comes to the rescue.  It guarantees that since $y=f(x)$ is continuous on $[a,b]$, we may find the area by finding antiderivatives and evaluating the difference of an antiderivative of $f(x)$ at the endpoints.

\[A = \int_{x=a}^{x=b} f(x) \d x = F(b)-F(a) ,\]

where $F(x)$ is an antiderivative of $f(x)$.

This can now be interpreted conceptually as follows:
\begin{itemize}
\item[1.] The notation ``$\Delta x$" represents the \emph{finite} but small width of a rectangle.  The notation ``$\d x$" represents the \emph{infinitesimal} width of a rectangle.
\item[2.] The procedure of definite integration can be thought of conceptually as follows.  We \emph{simultaneously} shrink the widths of the rectangles while adding all of the areas together.  
\item[3.] The integrand $f(x) \d x$ can be thought of as the area of an \emph{infinitesimal} rectangle of height $f(x)$ and thickness $\d x$.  As such, we cannot think of a rectangle of width $\d x$ as having width zero since we must add infinitely many such rectangles together. 
\end{itemize}

This same procedure can be used to model many other situations, which will be the subject of the coming sections.  It is therefore important to understand the logic behind it.


%%%%%%%%%%%%%%%%%%%%%%%%%%%%%%


\section{The Area Between Two Curves}

The above procedure also can be used to find areas between two curves as well.  Henceforth, by ``area'', we will mean ``total area''; the area bounded by the curves should be taken to be positive.  For example, the area bounded by $y=-x$ and $y=0$ from $x=0$ and $x=-1$ is shown below.  

\begin{image}
\begin{tikzpicture}

\begin{axis}
	[
	domain=-.5:1.3, ymax=.5,xmax=1.3, ymin=-1.3, xmin=-.5,
	axis lines=center, xlabel=$x$, ylabel=$y$,
	xtick={-1,1},
	ymajorticks=false,
	every axis y label/.style={at=(current axis.above origin),anchor=south},
	every axis x label/.style={at=(current axis.right of origin),anchor=west},
	axis on top,
	]

	\addplot [draw=penColor,very thick, smooth] {-x};	
	\addplot [draw=penColor2,very thick, smooth] {0};
	
	\draw [penColor,thick] (1,-1) -- (1,0);
	
	\addplot [name path=A,domain=0:1,draw=none] {-x};	
	\addplot [name path=B,domain=0:1,draw=none] {0};
	\addplot [fillp] fill between[of=A and B];
	
	\node at (axis cs:.6,24) [penColor] {$y=-x$};
	\node at (axis cs:1.4,2.5) [penColor2] {$y = 0$};
\end{axis}

\end{tikzpicture}
\end{image}

By noticing that this is the area of a triangle whose base and height have length 1, the total area should be $+\frac{1}{2}$.

As it turns out, the previous procedure can be used to find the total area between two curves.  We explore this in the context of another motivating example.

%\paragraph{Motivating Example:}
\begin{model}
Suppose now that we have two functions, $y=4x^3+15$ and $y=3x^2-2$ and suppose we want to find the area between the two curves over the interval $-1 \leq x \leq 1$.  The area is shown below.

%%%%%%%%PICTURE%%%%%%%%%%%

\begin{image}
\begin{tikzpicture}

\begin{axis}
	[
	domain=-1.2:1.2, ymax=25,xmax=2.2, ymin=-4, xmin=-1.2,
	axis lines=center, xlabel=$x$, ylabel=$y$,
	xtick={-1,1},
	ymajorticks=false,
	every axis y label/.style={at=(current axis.above origin),anchor=south},
	every axis x label/.style={at=(current axis.right of origin),anchor=west},
	axis on top,
	]

	\addplot [draw=penColor,very thick, smooth] {4*x^3 + 15};	
	\addplot [draw=penColor2,very thick, smooth] {3*x^2-2};
	
	\draw [penColor,thick] (-1,1) -- (-1,11);
	\draw [penColor,thick] (1,1) -- (1,19);
	
	\addplot [name path=A,domain=-1:1,draw=none] {4*x^3 + 15};	
	\addplot [name path=B,domain=-1:1,draw=none] {3*x^2-2};
	\addplot [fillp] fill between[of=A and B];
	
	\node at (axis cs:.6,22) [penColor] {$y=4x^3 + 15$};
	\node at (axis cs:1.7,4) [penColor2] {$y = 3x^2-2$};
\end{axis}

\end{tikzpicture}
\end{image}

%%%%%%%%PICTURE%%%%%%%%%%%



So how should we do this? Let's apply the procedure of ``Slice, Approximate, Integrate".

\paragraph{Step 1: Slice}

We divide $[-1,1]$ up into $n$ pieces of equal width $\Delta x$.

%%%%%%%%PICTURE%%%%%%%%%%%
\begin{image}
\begin{tikzpicture}

\begin{axis}
	[
	domain=-1.2:1.2, ymax=25,xmax=2.2, ymin=-6, xmin=-1.2,
	axis lines=center, xlabel=$x$, ylabel=$y$,
	xtick={-1,1},
	ymajorticks=false,
	every axis y label/.style={at=(current axis.above origin),anchor=south},
	every axis x label/.style={at=(current axis.right of origin),anchor=west},
	axis on top,
	]

	\addplot [draw=penColor,very thick, smooth] {4*x^3 + 15};	
	\addplot [draw=penColor2,very thick, smooth] {3*x^2-2};
	
	\pgfplotsinvokeforeach{-1,-.75,-.5,-.25,0,.25,.5,.75,1}
	{
		\draw[penColor,thick] (#1, {3*(#1)^2-2}) -- (#1, {4*(#1)^3 + 15});
	}
	
	\addplot [name path=A,domain=-1:1,draw=none] {4*x^3 + 15};	
	\addplot [name path=B,domain=-1:1,draw=none] {3*x^2-2};
	\addplot [fillp] fill between[of=A and B];
	
	\addplot [name path=darkerA,domain=.5:.75,draw=none] {4*x^3 + 15};
	\addplot [name path=darkerB,domain=.5:.75,draw=none] {3*x^2-2};
	\addplot [blue!50!black!50] fill between[of=darkerA and darkerB];
	
	\node at (axis cs:.6,22) [penColor] {$y=4x^3 + 15$};
	\node at (axis cs:1.7,4) [penColor2] {$y = 3x^2-2$};
	
	\node at (axis cs:.625,-4) [penColor] {$\Delta x$};
	
	\draw[penColor, |-|] (axis cs:.5,-2.75) -- (axis cs:.75,-2.75);
\end{axis}

\end{tikzpicture}
\end{image}

%%%%%%%%PICTURE%%%%%%%%%%%

\paragraph{Step 2: Approximate}

We cannot determine the exact area of the slice, but just as before, we can approximate that each slice by a rectangle whose heights are determined by the value of the function $y = f(x)$ at some $x$-value on its base.  Here, we choose to use lefthand endpoints.

%%%%%%%%PICTURE%%%%%%%%%%%

\begin{image}
\begin{tikzpicture}

\begin{axis}
	[
	domain=-1.2:1.2, ymax=25,xmax=2.2, ymin=-6, xmin=-1.2,
	axis lines=center, xlabel=$x$, ylabel=$y$,
	xtick={-1,1},
	ymajorticks=false,
	every axis y label/.style={at=(current axis.above origin),anchor=south},
	every axis x label/.style={at=(current axis.right of origin),anchor=west},
	axis on top,
	]

	\pgfplotsinvokeforeach{-1,-.75,-.5,-.25,0,.25,.75}
	{
		\draw[penColor,thick,fill=fillp] (#1, {3*(#1)^2-2}) -- (#1, {4*(#1)^3 + 15}) -- ({#1 + .25}, {4*(#1)^3 + 15}) -- ({#1 + .25}, {3*(#1)^2-2}) -- (#1, {3*(#1)^2-2});
	}
	\pgfplotsinvokeforeach{.5}
	{
		\draw[penColor,thick,fill=blue!50!black!50] (#1, {3*(#1)^2-2}) -- (#1, {4*(#1)^3 + 15}) -- ({#1 + .25}, {4*(#1)^3 + 15}) -- ({#1 + .25}, {3*(#1)^2-2}) -- (#1, {3*(#1)^2-2});
	}
	
	\draw[black, thick, |-|] (axis cs:.82,{3*(.5)^2-2}) -- (axis cs:.82,{4*(.5)^3 + 15});
	\node at (axis cs:.91,9) [black] {$h$};

	\addplot [draw=penColor,very thick, smooth] {4*x^3 + 15};	
	\addplot [draw=penColor2,very thick, smooth] {3*x^2-2};
	
	%\addplot [name path=A,domain=-1:1,draw=none] {4*x^3 + 15};	
	%\addplot [name path=B,domain=-1:1,draw=none] {3*x^2-2};
	%\addplot [fillp] fill between[of=A and B];
	
	\node at (axis cs:.6,22) [penColor] {$y=4x^3 + 15$};
	\node at (axis cs:1.7,4) [penColor2] {$y = 3x^2-2$};
	
	\node at (axis cs:.625,-4) [penColor] {$\Delta x$};
	
	\draw[penColor, |-|] (axis cs:.5,-2.75) -- (axis cs:.75,-2.75);
\end{axis}

\end{tikzpicture}
\end{image}

%%%%%%%%PICTURE%%%%%%%%%%%

The area $\Delta A_k$ of one the $k$th rectangles is given by 
\begin{align*}
\Delta A_k & = (height) \times (width), \nonumber \\
\end{align*}
where $x_k^*$ is the $x$-value of the lefthand endpoint of the $k$-th rectangle. 

To find the height of the darkly shaded rectangle, notice that this height is just the vertical distance between the curves.  Vertical distances can always be found by taking the top $y$-value minus the bottom $y$-value.  Since these $y$-values lie on the graphs of the given functions, at a given $x$-value, we have \wordChoice{\choice[correct]{$y_{top}=4x^3+15$}\choice{$y_{bot}=3x^2-2$}} and \wordChoice{
\choice{$y_{top}=4x^3+15$}\choice[correct]{$y_{bot}=3x^2-2$}}.

Thus, the height $h$ of the rectangle is $h=y_{top}-y_{bot} = \answer[given]{(4x^3+15)-(3x^2-2)}$.

The approximate total area obtained by adding the areas of the $n$ rectangles between $x=-1$ and $x=1$ together.  Note that as we use more rectangles, the following occur \emph{simultaneously}.

\begin{itemize}
\item[1.] The width $\Delta x$ of each rectangle decreases.
\item[2.] The total number of rectangles increases.
\item[3.] The sum of the areas of the rectangles becomes closer to the actual area.
\end{itemize}

The approximate area $A$ is given by the finite sum $A = \sum_{k=1}^n f(x_k^*) \Delta x $. 


\paragraph{Step 3: Integrate}

The actual area $A$ is found by taking the limit of the above sum, that is

\[
A = \lim_{n \rightarrow \infty} \left[ \sum_{k=1}^n f(x_k^*) \Delta x \right],
\]

and we can write this as the definite integral

\[
A = \int_{x=-1}^{x=1} (4x^3+15)-(3x^2-2) \d x
\]

Once again the Fundamental Theorem of Calculus allows us to find the area using antiderivatives.

\begin{align*}
A &= \int_{x=-1}^{x=1} 4x^3-3x^2+17 \d x \\
&= \eval{ \answer[given]{x^4-x^3+17x}}_{x= -1}^{x=1} \\
&= \left[(1)^4-(1)^3+17(1)\right] -  \left[(-1)^4-(-1)^3+17(-1)\right] \\
&=\answer[given]{32}
\end{align*}


Note that $A = \int_{x=-1}^{x=1} (4x^3+15)-(3x^2-2) \d x$ can be interpreted as follows:
\begin{itemize}
\item[1.] The integrand is the area of a rectangle, whose height is determined as the difference between the top and bottom $y$-values of the bounding curves, and whose thickness $\d x$ is \emph{infinitesimal}. 
\item[2.] Since we integrate with respect to $x$, the limits of integration tell us the range of the $x$-values of the rectangles to be added.
\begin{itemize}
\item The lower limit gives the $x$-position of the first rectangle.
\item The upper limit gives the $x$-position of the last rectangle.
\end{itemize}
\end{itemize}


A similar procedure can be used in many other examples for the rest of the chapter.  The major point here is that once we find the approximate area for a \emph{single} rectangle, we can immediately write down the integral that gives the \emph{exact} area of the region.
\end{model}

\begin{remark} Since the thickness is $\d x$, we must express the curves as functions of $x$; that is, we must write $y_{top}$ and $y_{bot}$ in terms of $x$.  This is a common theme that runs throughout this chapter.  Once we choose a variable of integration, \emph{every} quantity (limits of integration, functions in the integrand) \emph{must} be written in terms of that variable!
\end{remark}


\section{Integrating with respect to \textit{x}}

We can summarize the above procedure with a formula that respects the geometrical reasoning described previously.

\begin{formula}
The area of a region bounded by continuous functions on $a \le x \le b$ is given by: 

\[A=\int_{x=a}^{x=b} h(x) \d x \]
where $h(x)$ is the height of each rectangle as a function of $x$, $x=a$ gives the $x$-value of the first slice, and $x=b$ gives the $x$-value of the last slice.

By using vertical rectangles, we have 
\[
h(x) = y_{top}-y_{bot}
\] 
where $y_{top}$ is found from the top curve, expressed as a function of $x$, and $y_{bot}$ is found form the bottom curve.
\end{formula}

Let's look at a few more examples that demonstrate how to apply the formula.


\begin{example}
Find the area bounded by the curves $y=\frac{9}{x}$, $y=2x-3$, and $x=1$.

\begin{explanation}
The best way to start is to draw a picture of the region, and draw a representative rectangle that will be used to build the area of the region.  

%%%%%%%%PICTURE%%%%%%%%%%%
\begin{image}
\begin{tikzpicture}
\begin{axis}[
            domain=0:4, ymax=12,xmax=4.5, ymin=-4, xmin=0,
            axis lines =center, xlabel=$x$, ylabel=$y$,
            every axis y label/.style={at=(current axis.above origin),anchor=south},
            every axis x label/.style={at=(current axis.right of origin),anchor=west},
            axis on top,
          ]
          
\addplot [draw=penColor,very thick,smooth] {2*x-3};
\addplot [draw=penColor2,very thick,smooth] {9/x};
\addplot [draw=black!60!green,very thick,smooth] coordinates {(1,-4)(1,12)};
           
\addplot [name path=A,domain=1:3,draw=none] {9/x};   
\addplot [name path=B,domain=1:3,draw=none] {2*x-3};
\addplot [fillp] fill between[of=A and B];

  \addplot [draw=penColor, fill = gray!50] plot coordinates {(1.5,.3) (1.6,.3) (1.6, 5.5) (1.5,5.5) (1.5, .3)};
          
        

          \draw[decoration={brace,raise=.2cm},decorate,thin] (axis cs:1.51,.3)--(axis cs:1.51,5.5);
          \node[anchor=east] at (axis cs:1.35,3) {$h$};
                    
          
          \node at (axis cs:2,-2) [penColor] {$y=2x-3$};
          \node at (axis cs:2,7) [penColor2] {$y=\frac{9}{x}$};
          \node at (axis cs:0.5,4) [black!60!green] {$x=1$};
        \end{axis}
\end{tikzpicture}

\end{image}

%%%%%%%%PICTURE%%%%%%%%%%%

Note that there is a ``natural" righthand boundary here given by the $x$-value where the curves intersect.

   \begin{align*}
    \frac{9}{x} &= 2x-3\\
  9 &= 2x^2-3x\\
   2x^2-3x-9 &= 0\\
   (2x+\answer[given]{3})(x+\answer[given]{-3})  &= 0 \\
   x   &= -\frac{3}{2} \text{ or }\answer[given]{3}.
  \end{align*}
  From the picture, note that the other intersection point $x=-\frac{3}{2}$ is not relevant.

We must now express $h$ in terms of the variable of integration.  Since $h$ is a vertical distance, $h(x)=y_{top}-y_{bot}$.  The function used to determine the upper $y$-value is \wordChoice{\choice{$y_{top}=2x-3$}\choice[correct]{$y_{top}=\frac{9}{x}$}\choice{$y_{top}=1$}}.  The function used to determine the lower $y$-value is \wordChoice{\choice[correct]{$y_{bot}=2x-3$}\choice{$y_{bot}=\frac{9}{x}$}\choice{$y_{bot}=1$}}

The height $h$ of the rectangle is thus $h=y_{top}-y_{bot} = \answer[given]{\frac{9}{x}-(2x-3)}$, and the area is given by:
  \[
 \int_{x=a}^{x=b} h \d x =  \int_{x=\answer[given]{1}}^{x=\answer[given]{3}} \answer[given]{\frac{9}{x}-(2x-3)} \d x .  \]
  
 Evaluating the integral, we find that the area is $A= \answer[given]{-2+9 \ln(3)}$.
  \begin{hint}
    \begin{align*}
      \int_1^3 \frac{9}{x}-(2x-3) \d x &= \int_1^3 \frac{9}{x}-2x+3 \d x\\
      &=\eval{\answer[given]{9\ln(x)-x^2+3x}}_1^3\\
    \end{align*}
  \end{hint}
\end{explanation}
\end{example}

\begin{example} Set up, but do not evaluate, an integral or sum of integrals that expresses the area bounded by $y=\sqrt{2x}$, $x+2y=6$ and $y=0$.

As usual, we begin by drawing a picture and indicating the type of rectangle that will be used to build the area:

%%%%PICTURE%%%%%%%%%%


   \begin{image}
            \begin{tikzpicture}
            	\begin{axis}[
            domain=-.9:6.5, ymax=2.4,xmax=6.9, ymin=-.2, xmin=-.2,
            axis lines =center, xlabel=$x$, ylabel=$y$,
            every axis y label/.style={at=(current axis.above origin),anchor=south},
            every axis x label/.style={at=(current axis.right of origin),anchor=west},
            axis on top,
          ]                      
            	\addplot [draw=penColor,very thick,smooth,samples=100] {sqrt(2*x)};
	        \addplot [draw=penColor,very thick,smooth,samples=100,domain=0:.5] {sqrt(2*x)};
            	\addplot [draw=penColor2,very thick,smooth] {3-1/2*x};
	                            
            	\addplot [name path=A,domain=0:2,draw=none,samples=200] {sqrt(2*x)};   
            	\addplot [name path=B,domain=2:6,draw=none] {3-1/2*x};
	        \addplot [name path=C,domain=0:6,draw=none] {0};
            	\addplot [fillp] fill between[of=A and C];
	        \addplot [fillp] fill between[of=B and C];
                      
                      
            	\node at (axis cs:3.5,2.2) [penColor] {$y=\sqrt{2x}$};
            	\node at (axis cs:5.5,.82) [penColor2] {$x+2y=6$};
	
	%%%Draws the rectangles
	  \addplot [draw=penColor, fill = gray!50] plot coordinates {(.4,0) (.6,0) (.6, .86) (.4,.86) (.4, 0)};     
          \addplot [draw=penColor, fill = gray!50] plot coordinates {(4.6,0) (4.8,0) (4.8, .58) (4.6,.58) (4.6, 0)};    
	    
	      \end{axis}
            \end{tikzpicture}
            \end{image}
            
            
As you can see, something interesting happens here; the curve used to determine the height of the top rectangle changes.  In order to express this area by integrating with respect to $x$, we have to split it into two pieces.

%%%%PICTURE%%%%%%%%%%

The top curve will change at the $x$-value where the two curves intersect.  To find this $x$-value, we first must express each curve as a function of $x$.  The function $y=\sqrt{2x}$ is already a function of $x$.  For the line $x+2y=6$, we can solve $y=\answer[given]{3- \frac{1}{2}x}$.

We can now find the $x$-value of the intersection point. Write with me:

\begin{align*}
\sqrt{2x} &= 3 -\frac{1}{2}x \\
2x &= \left(3 -\frac{1}{2}x\right)^2 \\
2x&= 9-3x+\frac{1}{4} x^2 \\
0 &= \frac{1}{4} x^2 -5x+9  \\
0 &= x^2 -20x+36  \\
0 &= (x-2)(x-18) 
\end{align*}
Hence, $x=2$ or $x=18$.  

Note that by squaring both sides to eliminate the square root, we may have introduced an extraneous root.  We can check this easily enough:

By substituting $x=2$ into the equation $\sqrt{2x} = 3-\frac{1}{2}x$, we obtain $2=2$, which is a true statement.  However, doing the same for $x=18$ gives $6 = -6$, which is not true (though it should be clear why $x=18$ is a solution to the equation that results from squaring both sides!) 

Thus, we use $x=\answer[given]{2}$.  

For the second region, we find the rightmost $x$-value is $x=6$ (set $3-\frac{1}{2}x=0$).

We can now think of the original region in two separate parts and write down an integral that gives the area of each.

   \begin{image}
            \begin{tikzpicture}
            	\begin{axis}[
            domain=-.9:6.5, ymax=2.4,xmax=6.9, ymin=-.2, xmin=-.2,
            axis lines =center, xlabel=$x$, ylabel=$y$,
            every axis y label/.style={at=(current axis.above origin),anchor=south},
            every axis x label/.style={at=(current axis.right of origin),anchor=west},
            axis on top,
          ]                      
            	\addplot [draw=penColor,very thick,smooth,samples=100] {sqrt(2*x)};
	        \addplot [draw=penColor,very thick,smooth,samples=100,domain=0:.5] {sqrt(2*x)};
            	\addplot [draw=penColor2,very thick,smooth] {3-1/2*x};
	                            
            	\addplot [name path=A,domain=0:2,draw=none,samples=200] {sqrt(2*x)};   
            	\addplot [name path=B,domain=2:6,draw=none] {3-1/2*x};
	        \addplot [name path=C,domain=2:6,draw=none] {0};
            	\addplot [penColor!30] fill between[of=A and C];
	        \addplot [penColor2!30] fill between[of=B and C];
                      
                      
            	\node at (axis cs:3.5,2.2) [penColor] {$y=\sqrt{2x}$};
            	\node at (axis cs:5.5,.82) [penColor2] {$x+2y=6$};
	\node at (axis cs:1.2,.7) [black!75] {I};
	\node at (axis cs:3,.7) [black!75] {II};
	
	%%%Draws the rectangles
	\addplot [draw=penColor, fill = gray!50] plot coordinates {(.4,0) (.6,0) (.6, .86) (.4,.86) (.4, 0)};        
          \addplot [draw=penColor, fill = gray!50] plot coordinates {(4.6,0) (4.8,0) (4.8, .58) (4.6,.58) (4.6, 0)};    
	
	%%%Draws dashed line
	\addplot [draw=black!75,thick,dashed] coordinates {(2,0)(2,7)};
	    
	      \end{axis}
            \end{tikzpicture}
            \end{image}

Using the picture above: 
\[
\begin{array}{ll}
\text{In Region I:}  & \text{In Region II:}  \\
0 \le x \le 2  & 2 \le x \le  6\\
y_{top} = \sqrt{2x}    &  y_{top} = 3-\frac{1}{2}x \\
y_{bot} = 0  & y_{bot} = 0 \\
A_I = \int_{x= \answer[given]{0}}^{x=\answer[given]{2}} \answer[given]{\sqrt{2x}} \d x   & A_{II} = \int_{x= \answer[given]{2}}^{x=\answer[given]{6}}  \answer[given]{3-\frac{1}{2}x} \d x
\end{array}
\]

Putting this together, we can write down a sum of integrals that gives the area.

\[A= \int_{x= \answer[given]{0}}^{x=\answer[given]{2}} \answer[given]{\sqrt{2x}} \d x + \int_{x= \answer[given]{2}}^{x=\answer[given]{6}}\answer[given]{3-\frac{1}{2}x} \d x\]

\end{example}


\section{Integrating with respect to \textit{y}}

We needed to use two integrals to find the area because we used vertical slices to build up the area of the region and the top curve changed.  Mercifully, there is an easier way to find this area. Instead of using vertical slices to build the area, we could instead use horizontal ones.

  \begin{image}
            \begin{tikzpicture}
            	\begin{axis}[
            domain=-.9:6.5, ymax=2.4,xmax=6.9, ymin=-.2, xmin=-.2,
            axis lines =center, xlabel=$x$, ylabel=$y$,
            every axis y label/.style={at=(current axis.above origin),anchor=south},
            every axis x label/.style={at=(current axis.right of origin),anchor=west},
            axis on top,
          ]                      
            	\addplot [draw=penColor,very thick,smooth,samples=100] {sqrt(2*x)};
	        \addplot [draw=penColor,very thick,smooth,samples=100,domain=0:.5] {sqrt(2*x)};
            	\addplot [draw=penColor2,very thick,smooth] {3-1/2*x};
	                            
            	\addplot [name path=A,domain=0:2,draw=none,samples=200] {sqrt(2*x)};   
            	\addplot [name path=B,domain=2:6,draw=none] {3-1/2*x};
	        \addplot [name path=C,domain=0:6,draw=none] {0};
            	\addplot [fillp] fill between[of=A and C];
	        \addplot [fillp] fill between[of=B and C];
                      
                      
            	\node at (axis cs:3.5,2.2) [penColor] {$y=\sqrt{2x}$};
            	\node at (axis cs:4.4,1.4) [penColor2] {$x+2y=6$};
	
	%%%Draws the rectangles
	  \addplot [draw=penColor, fill = gray!50] plot coordinates {(.58,1.05) (4.15,1.05) (4.15, .85) (.58,.85) (.58, .85)};        
	    \addplot [draw=penColor,thick] coordinates {(.58,.85)(.58,1.05)};
	    
	    %%%Draws Label
	          \draw[decoration={brace,mirror, raise=.2cm},decorate,thin] (axis cs:.58,.85)--(axis cs:4.15,.85);
          \node[anchor=east] at (axis cs:2.62,.6) {$h$};
             \draw[decoration={brace, mirror, raise=.2cm},decorate,thin] (axis cs:4.15,.85)--(axis cs:4.15,1.05);
          \node[anchor=east] at (axis cs:5.15,.92) {$\Delta y$};
          
	      \end{axis}
            \end{tikzpicture}
            \end{image}

So what are we doing?  Instead of making slices with respect to $x$ as we did before, we are slicing with respect to $y$.  The area of one of these rectangles is $\Delta A = h \Delta y$.  To find the exact area, we simultaneous need to shrink the widths of the rectangles and add all of them together.  The same procedure as before produces a convenient result.

\begin{formula}
The area of a region bounded by continuous functions on $c \le y \le d$ is given by

\[A=\int_{y=c}^{y=d} h(y) \d y \]
where $h(y)$ is the height of each rectangle as a function of $y$ , $y=c$ gives the $y$-position of the lowest rectangle, and $y=d$ gives the $y$-position of the highest rectangle.

By using horizontal rectangles, we have 
\[
h(y) = x_{right}-x_{left}
\] 
where $x_{right}$ is found from the right curve, described as a function of $y$,  and $x_{left}$ is found form the left curve.




\end{formula}

Now, let's revisit the last example.

\begin{example} Set up, but do not evaluate, an integral or sum of integrals with respect to $y$ that expresses the area bounded by $y=\sqrt{2x}$, $x+2y=6$ and $y=0$.

\begin{explanation}

Since we are going to integrate with respect to $y$, we must describe the curves as functions of $y$:

For $y=\sqrt{2x}$, we can solve for $x$ to obtain $x= \answer[given]{\frac{1}{2}y^2}$.

For $x+2y=6$, we can solve for $x$ to obtain $x= \answer[given]{6-2y}$.

We can note that the lowest slice occurs at $y=0$ and the upper slice occurs at the $y$-value where the curves intersect.  Setting these new expressions equal to each other gives $y= \answer[given]{2}$.  

Now that we have our limits of integration, we must express $h$ in terms of the variable of integration!  Since $h$ is a horizontal distance, $h=x_{right}-x_{left}$

  \begin{image}
            \begin{tikzpicture}
            	\begin{axis}[
            domain=-.9:6.5, ymax=2.4,xmax=6.9, ymin=-.2, xmin=-.2,
            axis lines =center, xlabel=$x$, ylabel=$y$,
            every axis y label/.style={at=(current axis.above origin),anchor=south},
            every axis x label/.style={at=(current axis.right of origin),anchor=west},
            axis on top,
          ]                      
            	\addplot [draw=penColor,very thick,smooth,samples=100] {sqrt(2*x)};
	        \addplot [draw=penColor,very thick,smooth,samples=100,domain=0:.5] {sqrt(2*x)};
            	\addplot [draw=penColor2,very thick,smooth] {3-1/2*x};
	                            
            	\addplot [name path=A,domain=0:2,draw=none,samples=200] {sqrt(2*x)};   
            	\addplot [name path=B,domain=2:6,draw=none] {3-1/2*x};
	        \addplot [name path=C,domain=0:6,draw=none] {0};
            	\addplot [fillp] fill between[of=A and C];
	        \addplot [fillp] fill between[of=B and C];
                      
                      
            	\node at (axis cs:3.5,2.2) [penColor] {$x=\frac{1}{2}y^2$};
            	\node at (axis cs:4.4,1.4) [penColor2] {$x=6-2y$};
	
	%%%Draws the rectangles
	  \addplot [draw=penColor, fill = gray!50] plot coordinates {(.58,1.05) (4.15,1.05) (4.15, .85) (.58,.85) (.58, .85)};        
	    \addplot [draw=penColor,thick] coordinates {(.58,.85)(.58,1.05)};
	    
	    %%%Draws Label
	          \draw[decoration={brace,mirror, raise=.2cm},decorate,thin] (axis cs:.58,.85)--(axis cs:4.15,.85);
          \node[anchor=east] at (axis cs:2.62,.6) {$h$};
             \draw[decoration={brace, mirror, raise=.2cm},decorate,thin] (axis cs:4.15,.85)--(axis cs:4.15,1.05);
          \node[anchor=east] at (axis cs:5.15,.92) {$\Delta y$};
          
	      \end{axis}
            \end{tikzpicture}
            \end{image}
            
            
The function used to determine the rightmost $x$-value is \wordChoice{\choice{$x_{right}=\frac{1}{2}y^2$}\choice[correct]{$x_{right}=6-2y$}} and the function used to determine the leftmost $x$-value is \wordChoice{\choice[correct]{$x_{left}=\frac{1}{2}y^2$}\choice{$x_{left}=6-2y$}}.

The length $h$ of the rectangle is thus $h=x_{right}-x_{left} = \answer[given]{(6-2y)-\left(\frac{1}{2}y^2\right)}$ and the area is given by the integral
  \[
 \int_{y=c}^{y=d} h \d y =  \int_{y=0}^{y=2} 6-2y-\frac{1}{2}y^2 \d y .
  \]
  Evaluating this (which you should work out yourself and confirm) gives $A= \frac{20}{3}.$
 \end{explanation}
 
\end{example}

\section{Choosing a variable of integration}
As we have seen, choosing a particular type of slice may be more advantageous than another.  To make this more explicit, draw a vertical rectangle or horizontal rectangle in the region. 

\begin{itemize}
\item If the top or bottom curve of the region depends on where you draw the slice, you'll need more than one integral with respect to $x$ to find the area.
\item If the left or right curve of the region depends on where you draw the slice, you'll need more than one integral with respect to $y$ to find the area.
\end{itemize}

\begin{example} Consider the region bounded by $y=\sin(x)$, $y=x$, and the $y$-axis.


 \begin{image}
            \begin{tikzpicture}
            	\begin{axis}[
            domain=-.9:1.5, ymax=1.4,xmax=.45, ymin=-.2, xmin=-.08,
            axis lines =center, xlabel=$x$, ylabel=$y$,
            every axis y label/.style={at=(current axis.above origin),anchor=south},
            every axis x label/.style={at=(current axis.right of origin),anchor=west},
            axis on top,
          ]                      
            	\addplot [draw=penColor,very thick,smooth,samples=100] {sin(deg(3*x))};
	        \addplot [draw=penColor2,very thick,smooth,samples=100,domain=-1:2] {1-x};
            	
	                            
            	\addplot [name path=A,domain=0:.28,draw=none,samples=200] {sin(deg(3*x))};   
            	\addplot [name path=B,domain=0:.28,draw=none] {1-x};

            	\addplot [fillp] fill between[of=A and B];

                      
                      
            	\node at (axis cs:.22,.3) [penColor] {$y=\sin(3x)$};
            	\node at (axis cs:.1,1.05) [penColor2] {$y=1-x$};
          
	      \end{axis}
            \end{tikzpicture}
            \end{image}

How many integrals with respect to $x$ are needed to compute this area? $\answer[given]{1}$
	
How many integrals with respect to $y$ are needed to compute this area? $\answer[given]{2}$


\end{example}


Sometimes, you will need multiple integrals to find the area of a region no matter which type of slice you use.

\begin{example} Consider the region bounded by $3x-y=0$, $x-y=3$, $x+y=4$, and $x+y=0$.


 \begin{image}
            \begin{tikzpicture}
            	\begin{axis}[
            domain=-.9:4, ymax=3.8,xmax=4.4, ymin=-2.4, xmin=-.95,
            axis lines =center, xlabel=$x$, ylabel=$y$,
            every axis y label/.style={at=(current axis.above origin),anchor=south},
            every axis x label/.style={at=(current axis.right of origin),anchor=west},
            axis on top,
          ]                      
            	\addplot [draw=penColor,very thick,smooth,samples=100] {-x};
	        \addplot [draw=penColor2,very thick,smooth,samples=100] {4-x};
            	\addplot [draw=penColor3,very thick,smooth,samples=100] {x-3};
	        \addplot [draw=penColor4,very thick,smooth,samples=100] {3*x};
	                            
            	\addplot [name path=A,domain=0:1.5,draw=none] {-x};   
            	\addplot [name path=B,domain=0:1,draw=none] {3*x};
	        \addplot [name path=C,domain=1:1.5,draw=none] {4-x};
	      	\addplot [name path=D,domain=1:1.5,draw=none] {-x};
        		\addplot [name path=E,domain=1.5:3.5,draw=none] {4-x};
	        \addplot [name path=F,domain=1.5:3.5,draw=none] {x-3};
	        
            	\addplot [fillp] fill between[of=A and B];
	 	\addplot [fillp] fill between[of=C and D];
		\addplot [fillp] fill between[of=E and F];
                      
                 
	      \end{axis}
            \end{tikzpicture}
            \end{image}

How many integrals with respect to $x$ are needed to compute this area? $\answer[given]{3}$
	
How many integrals with respect to $y$ are needed to compute this area? $\answer[given]{3}$

\end{example}

We conclude the section by summarizing a few important facts.

\begin{itemize}
\item To find vertical distances, we always take $y_{top} - y_{bot}$.  To find horizontal distances, we always take $x_{right}-x_{left}$.

\item When we integrate with respect to $x$, we use vertical slices and when we use vertical slices, we integrate with respect to $x$.  When we integrate with respect to $y$, we use horizontal slices and when we use horizontal slices, we integrate with respect to $y$.

\item Once we choose a variable of integration, everything in the integrand must be expressed in terms of that variable!  This includes both the limits of integration and any functions that arise in the integrand.
\end{itemize}


\begin{quote}
``Mathematics is not about numbers, equations, computations, or algorithms; it is about understanding." - William Paul Thurston

\end{quote}

\end{document}
