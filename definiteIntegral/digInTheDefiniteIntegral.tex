\documentclass{ximera}

%\usepackage{todonotes}
%\usepackage{mathtools} %% Required for wide table Curl and Greens
%\usepackage{cuted} %% Required for wide table Curl and Greens
\newcommand{\todo}{}

\usepackage{esint} % for \oiint
\ifxake%%https://math.meta.stackexchange.com/questions/9973/how-do-you-render-a-closed-surface-double-integral
\renewcommand{\oiint}{{\large\bigcirc}\kern-1.56em\iint}
\fi


\graphicspath{
  {./}
  {ximeraTutorial/}
  {basicPhilosophy/}
  {functionsOfSeveralVariables/}
  {normalVectors/}
  {lagrangeMultipliers/}
  {vectorFields/}
  {greensTheorem/}
  {shapeOfThingsToCome/}
  {dotProducts/}
  {partialDerivativesAndTheGradientVector/}
  {../productAndQuotientRules/exercises/}
  {../normalVectors/exercisesParametricPlots/}
  {../continuityOfFunctionsOfSeveralVariables/exercises/}
  {../partialDerivativesAndTheGradientVector/exercises/}
  {../directionalDerivativeAndChainRule/exercises/}
  {../commonCoordinates/exercisesCylindricalCoordinates/}
  {../commonCoordinates/exercisesSphericalCoordinates/}
  {../greensTheorem/exercisesCurlAndLineIntegrals/}
  {../greensTheorem/exercisesDivergenceAndLineIntegrals/}
  {../shapeOfThingsToCome/exercisesDivergenceTheorem/}
  {../greensTheorem/}
  {../shapeOfThingsToCome/}
  {../separableDifferentialEquations/exercises/}
  {vectorFields/}
}

\newcommand{\mooculus}{\textsf{\textbf{MOOC}\textnormal{\textsf{ULUS}}}}

\usepackage{tkz-euclide}\usepackage{tikz}
\usepackage{tikz-cd}
\usetikzlibrary{arrows}
\tikzset{>=stealth,commutative diagrams/.cd,
  arrow style=tikz,diagrams={>=stealth}} %% cool arrow head
\tikzset{shorten <>/.style={ shorten >=#1, shorten <=#1 } } %% allows shorter vectors

\usetikzlibrary{backgrounds} %% for boxes around graphs
\usetikzlibrary{shapes,positioning}  %% Clouds and stars
\usetikzlibrary{matrix} %% for matrix
\usepgfplotslibrary{polar} %% for polar plots
\usepgfplotslibrary{fillbetween} %% to shade area between curves in TikZ
\usetkzobj{all}
\usepackage[makeroom]{cancel} %% for strike outs
%\usepackage{mathtools} %% for pretty underbrace % Breaks Ximera
%\usepackage{multicol}
\usepackage{pgffor} %% required for integral for loops



%% http://tex.stackexchange.com/questions/66490/drawing-a-tikz-arc-specifying-the-center
%% Draws beach ball
\tikzset{pics/carc/.style args={#1:#2:#3}{code={\draw[pic actions] (#1:#3) arc(#1:#2:#3);}}}



\usepackage{array}
\setlength{\extrarowheight}{+.1cm}
\newdimen\digitwidth
\settowidth\digitwidth{9}
\def\divrule#1#2{
\noalign{\moveright#1\digitwidth
\vbox{\hrule width#2\digitwidth}}}





\newcommand{\RR}{\mathbb R}
\newcommand{\R}{\mathbb R}
\newcommand{\N}{\mathbb N}
\newcommand{\Z}{\mathbb Z}

\newcommand{\sagemath}{\textsf{SageMath}}


%\renewcommand{\d}{\,d\!}
\renewcommand{\d}{\mathop{}\!d}
\newcommand{\dd}[2][]{\frac{\d #1}{\d #2}}
\newcommand{\pp}[2][]{\frac{\partial #1}{\partial #2}}
\renewcommand{\l}{\ell}
\newcommand{\ddx}{\frac{d}{\d x}}

\newcommand{\zeroOverZero}{\ensuremath{\boldsymbol{\tfrac{0}{0}}}}
\newcommand{\inftyOverInfty}{\ensuremath{\boldsymbol{\tfrac{\infty}{\infty}}}}
\newcommand{\zeroOverInfty}{\ensuremath{\boldsymbol{\tfrac{0}{\infty}}}}
\newcommand{\zeroTimesInfty}{\ensuremath{\small\boldsymbol{0\cdot \infty}}}
\newcommand{\inftyMinusInfty}{\ensuremath{\small\boldsymbol{\infty - \infty}}}
\newcommand{\oneToInfty}{\ensuremath{\boldsymbol{1^\infty}}}
\newcommand{\zeroToZero}{\ensuremath{\boldsymbol{0^0}}}
\newcommand{\inftyToZero}{\ensuremath{\boldsymbol{\infty^0}}}



\newcommand{\numOverZero}{\ensuremath{\boldsymbol{\tfrac{\#}{0}}}}
\newcommand{\dfn}{\textbf}
%\newcommand{\unit}{\,\mathrm}
\newcommand{\unit}{\mathop{}\!\mathrm}
\newcommand{\eval}[1]{\bigg[ #1 \bigg]}
\newcommand{\seq}[1]{\left( #1 \right)}
\renewcommand{\epsilon}{\varepsilon}
\renewcommand{\phi}{\varphi}


\renewcommand{\iff}{\Leftrightarrow}

\DeclareMathOperator{\arccot}{arccot}
\DeclareMathOperator{\arcsec}{arcsec}
\DeclareMathOperator{\arccsc}{arccsc}
\DeclareMathOperator{\si}{Si}
\DeclareMathOperator{\scal}{scal}
\DeclareMathOperator{\sign}{sign}


%% \newcommand{\tightoverset}[2]{% for arrow vec
%%   \mathop{#2}\limits^{\vbox to -.5ex{\kern-0.75ex\hbox{$#1$}\vss}}}
\newcommand{\arrowvec}[1]{{\overset{\rightharpoonup}{#1}}}
%\renewcommand{\vec}[1]{\arrowvec{\mathbf{#1}}}
\renewcommand{\vec}[1]{{\overset{\boldsymbol{\rightharpoonup}}{\mathbf{#1}}}\hspace{0in}}

\newcommand{\point}[1]{\left(#1\right)} %this allows \vector{ to be changed to \vector{ with a quick find and replace
\newcommand{\pt}[1]{\mathbf{#1}} %this allows \vec{ to be changed to \vec{ with a quick find and replace
\newcommand{\Lim}[2]{\lim_{\point{#1} \to \point{#2}}} %Bart, I changed this to point since I want to use it.  It runs through both of the exercise and exerciseE files in limits section, which is why it was in each document to start with.

\DeclareMathOperator{\proj}{\mathbf{proj}}
\newcommand{\veci}{{\boldsymbol{\hat{\imath}}}}
\newcommand{\vecj}{{\boldsymbol{\hat{\jmath}}}}
\newcommand{\veck}{{\boldsymbol{\hat{k}}}}
\newcommand{\vecl}{\vec{\boldsymbol{\l}}}
\newcommand{\uvec}[1]{\mathbf{\hat{#1}}}
\newcommand{\utan}{\mathbf{\hat{t}}}
\newcommand{\unormal}{\mathbf{\hat{n}}}
\newcommand{\ubinormal}{\mathbf{\hat{b}}}

\newcommand{\dotp}{\bullet}
\newcommand{\cross}{\boldsymbol\times}
\newcommand{\grad}{\boldsymbol\nabla}
\newcommand{\divergence}{\grad\dotp}
\newcommand{\curl}{\grad\cross}
%\DeclareMathOperator{\divergence}{divergence}
%\DeclareMathOperator{\curl}[1]{\grad\cross #1}
\newcommand{\lto}{\mathop{\longrightarrow\,}\limits}

\renewcommand{\bar}{\overline}

\colorlet{textColor}{black}
\colorlet{background}{white}
\colorlet{penColor}{blue!50!black} % Color of a curve in a plot
\colorlet{penColor2}{red!50!black}% Color of a curve in a plot
\colorlet{penColor3}{red!50!blue} % Color of a curve in a plot
\colorlet{penColor4}{green!50!black} % Color of a curve in a plot
\colorlet{penColor5}{orange!80!black} % Color of a curve in a plot
\colorlet{penColor6}{yellow!70!black} % Color of a curve in a plot
\colorlet{fill1}{penColor!20} % Color of fill in a plot
\colorlet{fill2}{penColor2!20} % Color of fill in a plot
\colorlet{fillp}{fill1} % Color of positive area
\colorlet{filln}{penColor2!20} % Color of negative area
\colorlet{fill3}{penColor3!20} % Fill
\colorlet{fill4}{penColor4!20} % Fill
\colorlet{fill5}{penColor5!20} % Fill
\colorlet{gridColor}{gray!50} % Color of grid in a plot

\newcommand{\surfaceColor}{violet}
\newcommand{\surfaceColorTwo}{redyellow}
\newcommand{\sliceColor}{greenyellow}




\pgfmathdeclarefunction{gauss}{2}{% gives gaussian
  \pgfmathparse{1/(#2*sqrt(2*pi))*exp(-((x-#1)^2)/(2*#2^2))}%
}


%%%%%%%%%%%%%
%% Vectors
%%%%%%%%%%%%%

%% Simple horiz vectors
\renewcommand{\vector}[1]{\left\langle #1\right\rangle}


%% %% Complex Horiz Vectors with angle brackets
%% \makeatletter
%% \renewcommand{\vector}[2][ , ]{\left\langle%
%%   \def\nextitem{\def\nextitem{#1}}%
%%   \@for \el:=#2\do{\nextitem\el}\right\rangle%
%% }
%% \makeatother

%% %% Vertical Vectors
%% \def\vector#1{\begin{bmatrix}\vecListA#1,,\end{bmatrix}}
%% \def\vecListA#1,{\if,#1,\else #1\cr \expandafter \vecListA \fi}

%%%%%%%%%%%%%
%% End of vectors
%%%%%%%%%%%%%

%\newcommand{\fullwidth}{}
%\newcommand{\normalwidth}{}



%% makes a snazzy t-chart for evaluating functions
%\newenvironment{tchart}{\rowcolors{2}{}{background!90!textColor}\array}{\endarray}

%%This is to help with formatting on future title pages.
\newenvironment{sectionOutcomes}{}{}



%% Flowchart stuff
%\tikzstyle{startstop} = [rectangle, rounded corners, minimum width=3cm, minimum height=1cm,text centered, draw=black]
%\tikzstyle{question} = [rectangle, minimum width=3cm, minimum height=1cm, text centered, draw=black]
%\tikzstyle{decision} = [trapezium, trapezium left angle=70, trapezium right angle=110, minimum width=3cm, minimum height=1cm, text centered, draw=black]
%\tikzstyle{question} = [rectangle, rounded corners, minimum width=3cm, minimum height=1cm,text centered, draw=black]
%\tikzstyle{process} = [rectangle, minimum width=3cm, minimum height=1cm, text centered, draw=black]
%\tikzstyle{decision} = [trapezium, trapezium left angle=70, trapezium right angle=110, minimum width=3cm, minimum height=1cm, text centered, draw=black]


\outcome{Use integral notation for both antiderivatives and definite integrals.}
\outcome{Compute definite integrals using geometry.}
\outcome{Compute definite integrals using the properties of integrals.}
\outcome{Justify the properties of definite integrals using algebra or geometry.}
\outcome{Understand how Riemann sums are used to find exact area.}
\outcome{Define net area.}
\outcome{Split the area under a curve into several pieces to aid with calculations.}
\outcome{Use symmetry to calculate definite integrals.}
\outcome{Explain geometrically why symmetry of a function simplifies calculation of some definite integrals.}


\title[Dig-In:]{The definite integral}

\begin{document}
\begin{abstract}
  Definite integrals compute net area.
\end{abstract}
\maketitle




The process of approximating areas under curves led to the notion of a Riemann sum

\[
A\approx\sum_{k=1}^n f(x_k^*)\Delta x,
\]
where $f$ is a nonnegative, continuous function on the interval $[a,b]$, and

 $x_k^*$ is a sample point for the $k^{th}$ rectangle, $k=1,2,..., n$.
 
 The limit of Riemann sums, as $n\to\infty$, gives the exact area between the curve $y=f(x)$ and the interval on the $x-$axis:
 
\[
A=\lim_{n\to\infty}\sum_{k=1}^n f(x_k^*)\Delta x.
\]
It does not matter whether we consider only right Riemann sums, or left Riemann sums, or midpoint Riemann sums, or others: the limit of any kind of Riemann sum as $n\to\infty$  is equal to the area, as long as $f$ is nonnegative and continuous on $[a,b]$. 


 
What happens when a continuous function $f$ assumes negative values  on the interval $[a,b]$?

We can still form Riemann sums and take the limit. 

The question is: What is the meaning of a Riemann sum in that case?

 \begin{example}
The graph of the function $f$ on the interval $[0,10]$ is given in the figure below.

 \begin{image}
  \begin{tikzpicture}[
      declare function = {f(\x) = 6+\x/2 - pow(\x,2)/4;}]
    \begin{axis}[  
        domain=0:10, xmin =-1,xmax=10.5,ymax=10,ymin=-15,
        width=6in,
        height=3in,xtick={0,2,4,...,10},
        xticklabels={$x_1^*=0$,$x_2^*=2$,$x_3^*=4$,$x_4^*=6$,$x_5^*=8$},
        %% ytick style={draw=none},
        %% yticklabels={},
        axis lines=center, xlabel=$x$, ylabel=$y$,
        every axis y label/.style={at=(current axis.above origin),anchor=south},
        every axis x label/.style={at=(current axis.right of origin),anchor=west},
        axis on top,
      ]
      \addplot [draw=penColor,fill=fill1] plot coordinates
               {({(0) * 2},{f(2)})
                 ({(1) * 2},{f(2) })} \closedcycle;

               \addplot [draw=penColor,fill=fill1] plot coordinates
               {({2-1) * 2},{f(4)})
                 ({(2) * 2},{f(4) })} \closedcycle;

               \addplot [draw=penColor,fill=fill1] plot coordinates
               {({3-1) * 2},{f(6)})
                 ({(3) * 2},{f(6) })} \closedcycle;

               \addplot [draw=penColor,fill=fill2] plot coordinates
               {({4-1) * 2},{f(8)})
                 ({(4) * 2},{f(8) })} \closedcycle;

               \addplot [draw=penColor,fill=fill2] plot coordinates
               {({5-1) * 2},{f(10)})
                 ({(5) * 2},{f(10) })} \closedcycle;
               
               \addplot [ultra thick,penColor, smooth] {f(x)};
              \node at (axis cs:1,7) {\large$f(2)\cdot 2$};
              \node at (axis cs:3,7) {\large$f(4)\cdot 2$};
              \node at (axis cs:5,5) {\large$f(6)\cdot 2$};
              \node at (axis cs:7,1) {\large$f(8)\cdot 2$};
              \node at (axis cs:9,1) {\large$f(10)\cdot 2$}; 
                \node at (axis cs:6.6,6.2) {\large$y=f(x)$};
    \end{axis}
  \end{tikzpicture}
\end{image}

 The figure illustrates a right Riemann sum with $n=5$ rectangles for $f$ on $[0,10]$.
\[
\sum_{k=1}^5 f(x_k^*)\Delta x= f(2)\Delta x+ f(4)\Delta x+ f(6)\Delta x+f(8)\Delta x+ f(10)\Delta x
\]

So, the Riemann sum is the sum of \textbf{signed areas} of rectangles:
rectangles that lie above the $x$-axis contribute positive values, and
rectangles that lie below the $x$-axis contribute negative values to
the Riemann sum.
\begin{align*}
  \sum_{k=1}^5 f(x_k^*)\Delta x= &\left(\underbrace{f(2)\Delta x}_{\text{nonnegative}}+  \underbrace{f(4)\Delta x}_{\text{nonnegative}}+  \underbrace{f(6)\Delta x}_{\text{nonnegative}}\right)\\
  &+ \left(\underbrace{f(8)\Delta x}_{\text{negative}}+  \underbrace{f(10)\Delta x}_{\text{negative}}\right)
\end{align*}
 \end{example}

 When we take the limit of Riemann sums, it seems that we should get that
 
 \[
\lim_{n\to\infty}\sum_{k=1}^n f(x_k^*)\Delta x= A_1+(-A_2),
\]
where the areas $A_1$ and $A_2$ are depicted in the figure below. 

In other words, the limit of Riemann sums seems to be equal to the sum of signed areas of the regions that lie entirely above  or below  the $x$-axis.
The signed area of  regions that lie above the $x$-axis is positive, and the signed area of regions that lie below the $x$-axis is negative.

This sum of signed areas is called the net area of the region  between the graph of $f$ and the interval on the $x$- axis .
 \begin{image}
  \begin{tikzpicture}[
      declare function = {f(\x) = 6+\x/2 - pow(\x,2)/4;}]
    \begin{axis}[  
        domain=0:10, xmin =-1,xmax=10.5,ymax=10,ymin=-15,
        width=6in,
        height=3in,xtick={0,2,4,...,10},
        xticklabels={0,2,4,...,10},
        %% ytick style={draw=none},
        %% yticklabels={},
        axis lines=center, xlabel=$x$, ylabel=$y$,
        every axis y label/.style={at=(current axis.above origin),anchor=south},
        every axis x label/.style={at=(current axis.right of origin),anchor=west},
        axis on top,
      ]
       \addplot [draw=none,fill=fillp,domain=0:6, smooth] {f(x)} \closedcycle;
            \addplot [draw=none,fill=fill2,domain=6:10, smooth] {f(x)} \closedcycle;
          \addplot [ultra thick,penColor, smooth] {f(x)};
            \node at (axis cs:2,2) {\huge$A_1$};
              \node at (axis cs:9,-5) {\huge$A_2$};
                \node at (axis cs:5,6.2) {\large$y=f(x)$};
        \end{axis}  
  \end{tikzpicture}
\end{image}


The limit of Riemann sums will exist for any continuous functions on the interval $[a,b]$, even if $f$ assumes negative values on $[a,b]$.
The limit of Riemann sums  gives the  \textbf{net area} of the region between the graph of $f$ and an interval on the $x$- axis.

This leads to the following definition.
\begin{definition}
\index{integral}\index{definite integral}
Let $f$ be a function which is continuous on the interval $[a,b]$. We define the \dfn{definite integral} of $f$ on $[a,b]$ by
\[
\int_a^b f(x) \d x=\lim_{n\to\infty}\sum_{k=1}^n f(x_k^*)\Delta x.
\]
\end{definition}
The definite integral is a number that gives the net area of the region between the curve $y=f(x)$ and the $x$-axis on the interval $[a,b]$.  
 
\begin{example} The graph a function $f$ on the interval $[0,9]$ is given in the figure. The areas of  four regions that lie either above or below the $x$-axis are labeled in the figure.

 \begin{image}
  \begin{tikzpicture}[
      declare function = {f(\x) = (1/4)*(x-1)*(x-5)* (x-8);}]
    \begin{axis}[  
        domain=0:9, xmin =-1,xmax=9.1,ymax=10,ymin=-10,
        width=6in,
        height=3in,xtick={0,3,6,9},
        xticklabels={0,3,6,9},
        %% ytick style={draw=none},
    %% yticklabels={},
        axis lines=center, xlabel=$x$, ylabel=$y$,
        every axis y label/.style={at=(current axis.above origin),anchor=south},
        every axis x label/.style={at=(current axis.right of origin),anchor=west},
        axis on top,
      ]
        \addplot [draw=none,fill=fill2,domain=0:1, smooth] {f(x)} \closedcycle;
       \addplot [draw=none,fill=fillp,domain=1:5, smooth] {f(x)} \closedcycle;
            \addplot [draw=none,fill=fill2,domain=5:8, smooth] {f(x)} \closedcycle;
              \addplot [draw=none,fill=fillp,domain=8:9, smooth] {f(x)} \closedcycle;
          \addplot [ultra thick,penColor, smooth] {f(x)};
            \node at (axis cs:0.5,-1.2) {\large$A_1$};
              \node at (axis cs:3,1.2) {\large$A_2$};
              \node at (axis cs:6.5,-1.2) {\large$A_3$};
              \node at (axis cs:8.5,1.2) {\large$A_4$};
                \node at (axis cs:6,6.2) {\large$y=f(x)$};
        \end{axis}  
  \end{tikzpicture}
\end{image}
Consider the integral
\[
\int_0^9 f(x) \d x
\]
Express the integral in terms of areas $A_1$, $A_2$, $A_3$ and $A_4$.
\begin{explanation}
\begin{align*}
    \int_0^9 f(x) \d x&= -A_1+A_2-A_3+A_4\\
\end{align*}

\end{explanation}
\end{example}

\begin{example}
Consider the integral
\[
\int_0^{10}-x \d x.
\]
Compute the integral in two ways:
\begin{enumerate}
\item by interpreting the integral as the net area of the region between the curve $y=-x$ and the interval $[0,10]$ on  the $x$-axis;
\item using the definition of the definite integral, i.e. by computing the limit of Riemann sums.
\end{enumerate}
\begin{explanation}
\begin{enumerate}
\item The  area between the $x$-axis and the curve can be easily computed, since it is the area of a triangle.
\begin{image}
  \begin{tikzpicture}[
      declare function = {f(\x) = -x;}]
    \begin{axis}[  
        domain=0:10, xmin =-1,xmax=10.5,ymax=10,ymin=-15,
        width=6in,
        height=3in,xtick={0,2,4,...,10},
        xticklabels={0,2,4,...,10},
        %% ytick style={draw=none},
        %% yticklabels={},
        axis lines=center, xlabel=$x$, ylabel=$y$,
        every axis y label/.style={at=(current axis.above origin),anchor=south},
        every axis x label/.style={at=(current axis.right of origin),anchor=west},
        axis on top,
      ]
       \addplot [draw=none,fill=fill2,domain=0:10, smooth] {f(x)} \closedcycle;
         
          \addplot [ultra thick,penColor, smooth] {f(x)};
       \node at (axis cs:8,-5) {\huge$A$};
            
        \end{axis}  
  \end{tikzpicture}
\end{image}

\[
A=\frac{1}{2}\cdot 10\cdot10=\answer[given]{50}.
\]

Then, it follows that

\[
\int_0^{10}-x \d x=\answer[given]{-}A
\]
\item We use the \textbf{definition of the definite integral} and write
\[
\int_0^{10}-x \d x=\lim_{n\to\infty}\sum_{k=1}^n f(x_k^*)\Delta x=\lim_{n\to\infty}\sum_{k=1}^n -x_k^*\Delta x
\]
It does not matter what  type of a Riemann sum we use, so we choose a right Riemann sum.

\[
\int_0^{10}-x \d x=\lim_{n\to\infty}\sum_{k=1}^n -x_k\Delta x
\]
A right Riemann sum with $n=5$ is illustrated in the figure below.
\begin{image}
  \begin{tikzpicture}[
      declare function = {f(\x) = -x;}]
    \begin{axis}[  
        domain=0:10, xmin =-1,xmax=10.2,ymax=10,ymin=-10,
        width=6in,
        height=3in,xtick={2,4,...,10},
        xticklabels={$x_1^*=2$,$x_2^*=4$,$x_3^*=6$,$x_4^*=8$,$x_5^*=10$},
        %% ytick style={draw=none},
        %% yticklabels={},
        axis lines=center, xlabel=$x$, ylabel=$y$,
        every axis y label/.style={at=(current axis.above origin),anchor=south},
        every axis x label/.style={at=(current axis.right of origin),anchor=west},
        axis on top,
      ]
      \addplot [draw=penColor,fill=fill2] plot coordinates
               {({(1-1) * 2},{-2)})
                 ({(1) * 2},{-2})} \closedcycle;

               \addplot [draw=penColor,fill=fill2] plot coordinates
               {({(2-1) * 2},{-4})
                 ({(2) * 2},{-4 })} \closedcycle;

               \addplot [draw=penColor,fill=fill2] plot coordinates
               {({(3-1) * 2},{-6)})
                 ({(3) * 2},{-6 })} \closedcycle;

               \addplot [draw=penColor,fill=fill2] plot coordinates
               {({(4-1) * 2},{-8)})
                 ({(4) * 2},{-8) })} \closedcycle;

               \addplot [draw=penColor,fill=fill2] plot coordinates
               {({(5-1) * 2},{-10})
                 ({(5) * 2},{-10 })} \closedcycle;
               
               \addplot [ultra thick,penColor, smooth] {f(x)};
             
    \end{axis}
  \end{tikzpicture}
\end{image}

We can apply the constant multiple rule for sums and limits:
\begin{align*}
  \int_0^{10}-x \d x &=\lim_{n\to\infty}\sum_{k=1}^n -x_k\Delta x\\
  &=\lim_{n\to\infty}\left(-\sum_{k=1}^n x_k\Delta x\right)\\
  &=-\lim_{n\to\infty}\sum_{k=1}^n x_k\Delta x
\end{align*}

We can now finish our computation of the limit of right Riemann sums.
\begin{align*}
\int_0^{10}-x \d x&=-\lim_{n\to\infty}\sum_{k=1}^n x_k\Delta x\\
&=-\lim_{n\to\infty}\sum_{k=1}^n k\cdot\frac{10}{n}\cdot\frac{10}{n}\\
&=-\lim_{n\to\infty}\left(\frac{10}{n}\right)^2\sum_{k=1}^n k\\
&=-\lim_{n\to\infty}\left(\frac{10}{n}\right)^2\frac{n(n+1)}{2}\\
&=-\lim_{n\to\infty}50\frac{n+1}{n}\\
&=\answer[given]{-50}\\
\end{align*}
\end{enumerate}
\end{explanation}
\end{example}
\begin{example}
\begin{enumerate}


\item Express the limit as a definite integral.
 \[
  \lim_{n\to \infty} \sum_{k=1}^n \left(\sqrt{1-\left(-1+\frac{2k}{n}\right)^2}\right)
  \left(\frac{2}{n}\right)
  \]
  \item Compute this limit:
  \[
  \lim_{n\to \infty} \sum_{k=1}^n \left(\sqrt{1-\left(-1+\frac{2k}{n}\right)^2}\right)
  \left(\frac{2}{n}\right)
  \]

  \begin{explanation}
  \begin{enumerate}
  \item  This is a limit of Riemann sums!  Specifically, it is a limit of
    Riemann sums of $n$ rectangles, where
    \[
    \Delta x = \answer[given]{\frac{2}{n}},
    \]
 
    \[
    x_k^* = -1+\frac{2k}{n},
    \]
    \[
   a= x_0^* = \answer[given]{-1},
    \]
    and
     \[
    b=x_n^* = \answer[given]{1}.
    \]
    Hence, we may rewrite this as
    \[
    \lim_{n\to \infty} \sum_{k=1}^n \left(\sqrt{1-(x_k^*)^2}\right)=\int_{\answer[given]{-1}}^{\answer[given]{1}}\sqrt{1-x^2}
    \d x.
    \]
  \item The limit  computes the area between the $x$-axis and
    the curve
    
     $y = \sqrt{1-x^2}$. Let's see it:
    \begin{image}
  \begin{tikzpicture}
    \begin{axis}[
        width=6in,
        %height=3in,
        unit vector ratio*=1 1 1,            
        xmin=-1.1, xmax=1.1,ymin=-.1,ymax=1.1,
        axis lines =center, xlabel=$x$, ylabel=$y$,
        every axis y label/.style={at=(current axis.above origin),anchor=south},
        every axis x label/.style={at=(current axis.right of origin),anchor=west},
        axis on top,
    ] 
      \addplot [draw=none, fill=fillp,samples=200,domain=-1:1] {sqrt(1-x^2)} \closedcycle;
      
      \addplot [penColor,ultra thick,samples=200,domain=-1:1] {sqrt(1-x^2)};
    \end{axis}
  \end{tikzpicture}
    \end{image}
    By geometry, we know that this semicircle has area $\answer[given]{\pi/2}$. Hence
    \[
    \lim_{n\to \infty} \sum_{k=1}^n \left(\sqrt{1-(x_k^*)^2}\right)\Delta x =\answer[given]{\pi/2}.
    \]
     \end{enumerate}
  \end{explanation}
  \end{enumerate}
\end{example}




The definite integral computes the net area (sum of signed areas) between $y=f(x)$ and the $x$-axis on the
interval $[a,b]$.
%\begin{itemize}
 % \item If the region is above the $x$-axis, then the area has
   % positive sign.
  %\item If the region is below the $x$-axis, then the area has
  %  negative sign.
%\end{itemize}
\begin{question}
Consider the graph of  a function $f$ on the interval $[0,5]$.
\begin{image}
  \begin{tikzpicture}
    \begin{axis}[
        width=6in,
        height=3in,
        xmin=-.5, xmax=5.5,ymin=-1.2,ymax=2.2,domain=0:5,
        axis lines =center, xlabel=$x$, ylabel=$y$,
        every axis y label/.style={at=(current axis.above origin),anchor=south},
        every axis x label/.style={at=(current axis.right of origin),anchor=west},
        axis on top,
    ] 
      %\addplot [draw=none, %pattern=north west lines, pattern color=blue,
        fill=fillp,
       %domain=0:1] {x} \closedcycle;
      \addplot [draw=none, %pattern=north west lines, pattern color=blue,
        fill=fillp,
        domain=0:5] {1.5-x/2} \closedcycle;
      
    \addplot [draw=none,fill=fillp,domain=0:3, smooth] {1.5-x/2} \closedcycle;
            \addplot [draw=none,fill=fill2,domain=3:5, smooth] {1.5-x/2} \closedcycle;
          \addplot [ultra thick,penColor, smooth] {1.5-x/2};
            \node at (axis cs:1,0.5) {\huge$A_1$};
              \node at (axis cs:4.75,-0.45) {\huge$A_2$};
      \addplot [penColor,ultra thick,domain=0:5] {1.5-x/2};
  \end{axis}
  \end{tikzpicture}
\end{image}
Compute the definite integral
\[
\int_0^5 f(x) \d x.
\]
\begin{explanation}
The net area in the figure above is given by
\[
\int_0^5 f(x) \d x=A_1-A_2
\]
 
The areas $A_1$ and $A_2$ are easily computed, since both are the  areas of triangles. Therefore,
\[
\int_0^5 f(x) \d x=\answer[given]{1.25}
\]
\end{explanation}
\end{question}
There is more to this example, than just a value of the integral.


Notice that 
\[
 \int_0^3 f(x) \d x= A_1,
 \]
 \[
 \int_3^5 f(x) \d x=-A_2,
 \]
and
\[
\int_0^5 f(x) \d x=A_1-A_2.
\]
It follows that
 \[
\int_0^5 f(x) \d x= \int_0^3 f(x) \d x+ \int_3^5 f(x) \d x.
\]

This looks like a property of the definite integral. Are there other properties?

\begin{theorem}[Properties of the definite integral]
Let $f$ and $g$ be defined on a closed interval $[a,b]$ that contains the
value $c$, and let $k$ be a constant. The following
hold:
\begin{enumerate}
\item $\int_a^a f(x)\d x = 0$
\item $\int_a^c f(x)\d x + \int_c^b f(x)\d x = \int_a^b f(x)\d x$
\item $\int_a^bf(x)\d x = -\int_b^a f(x)\d x$
\item $\int_a^bk\cdot f(x)\d x = k\cdot\int_a^bf(x)\d x$
\item $\int_a^b f(x)\pm g(x)\d x = \int_a^bf(x)\d x \pm \int_a^bg(x)\d x$
\end{enumerate}
\begin{explanation}
  We will address each property in turn:
\begin{enumerate}
\item Here, there is no ``area under the curve'' when the region has
  no width; hence this definite integral is $0$.
\item This states that total area is the sum of the areas of
  subregions. Here a picture is worth a thousand words:
  \begin{image}
    \begin{tikzpicture}[
        declare function = {f(\x) = -sin(deg(\x)) + 3;} ]
      \begin{axis}[
          domain=-.2:7, xmin =-.2,xmax=7,ymax=5,ymin=-.2,
          width=6in,
          height=3in,
          xtick={1,3.5,6}, 
          xticklabels={$a$,$c$,$b$},
          ytick style={draw=none},
          yticklabels={},
          axis lines=center, xlabel=$x$, ylabel=$y$,
          every axis y label/.style={at=(current axis.above origin),anchor=south},
          every axis x label/.style={at=(current axis.right of origin),anchor=west},
          axis on top,
          ]
        \addplot [draw=none,fill=fill4,domain=1:3.5, smooth] {f(x)} \closedcycle;
        
        \addplot [draw=none,fill=fill5,domain=3.5:6, smooth] {f(x)} \closedcycle;
        
        \addplot [ultra thick,penColor, smooth] {f(x)};
        \addplot [dashed] plot coordinates {(3.5,0) (3.5,{f(3.5)})};
        
        \node at (axis cs:2.25,1) {\large$\int_a^c f(x)\d x$};
        \node at (axis cs:4.75,1) {\large$\int_c^b f(x)\d x$};
      \end{axis}
    \end{tikzpicture}
  \end{image}		
  It is important to note that this still holds true even if
  $a<b<c$. We discuss this in the next point.
  
\item For now, this property can be viewed a merely a convention to
  make other properties work well. However, later we will see how this
  property has a justification all its own.

\item This states that when one scales a function by, for instance, $7$,
  the area of the enclosed region also is scaled by a factor of
  $7$.
\item This states that the integral of the sum is the sum of the
  integrals.
\end{enumerate}
\end{explanation}
\end{theorem}

Due to the geometric nature of integration, geometric properties of
functions can help us compute integrals.

\begin{example}
  Compute:
  \[
  \int_0^6 |x-3| \d x
  \]
  \begin{explanation}
    This may seem difficult at first. Perhaps the first thing to do is
    look at a graph of $y=x-3$:
  \begin{image}
  \begin{tikzpicture}
    \begin{axis}[
        width=6in,
        height=3in,
        xmin=-.5, xmax=6.5,ymin=-4,ymax=4,domain=0:6,
        axis lines =center, xlabel=$x$, ylabel=$y$,
        every axis y label/.style={at=(current axis.above origin),anchor=south},
        every axis x label/.style={at=(current axis.right of origin),anchor=west},
        axis on top,
    ] 
      \addplot [penColor,ultra thick,domain=0:6] {x-3};
    \end{axis}
  \end{tikzpicture}
    \end{image}
    Now we can graph $y=|x-3|$:
    \begin{image}
  \begin{tikzpicture}
    \begin{axis}[
        width=6in,
        height=3in,
        xmin=-.5, xmax=6.5,ymin=-4,ymax=4,domain=0:6,
        axis lines =center, xlabel=$x$, ylabel=$y$,
        every axis y label/.style={at=(current axis.above origin),anchor=south},
        every axis x label/.style={at=(current axis.right of origin),anchor=west},
        axis on top,
    ] 
      \addplot [draw=none, fill=fillp,domain=0:3] {3-x} \closedcycle;
      \addplot [draw=none, fill=fillp,domain=3:6] {x-3} \closedcycle;
      \addplot [penColor,ultra thick,domain=3:6] {x-3};
      \addplot [penColor,ultra thick,domain=0:3] {3-x};
    \end{axis}
  \end{tikzpicture}
    \end{image}
    Now we see that we really have two triangles, each with base $3$
    and height $3$.  Hence
    \begin{align*}
    \int_0^6 |x-3| \d x &= \int_0^3 \answer[given]{3-x} \d x + \int_3^6 \answer[given]{x-3} \d x\\
    &= \frac{3\cdot 3}{2} + \frac{3\cdot 3}{2}\\
    &=\answer[given]{9}.
    \end{align*}
  \end{explanation}
\end{example}

\begin{definition}
  A function $f$ is an \dfn{odd} function if
  \[
  f(-x) = -f(x),
  \]
  and a function $g$ is an \dfn{even} function if
  \[
  g(-x) = g(x).
  \]
\end{definition}

The names \textit{odd} and \textit{even} come from the fact that these
properties are shared by functions of the form $x^n$ where $n$ is
either odd or even. For example, if $f(x) = x^3$, then
\[
f(-7) = -f(7),
\]
and if $g(x) = x^4$, then
\[
g(-7) = g(7).
\]
Geometrically, even functions have 
  symmetry with respect to the $y-$axis
  . Cosine is an even function:
\begin{image}
 \begin{tikzpicture}
	\begin{axis}[
            xmin=-6.75,xmax=6.75,ymin=-1.5,ymax=1.5,
            axis lines=center,
            xtick={-6.28, -4.71, -3.14, -1.57, 0, 1.57, 3.142, 4.71, 6.28},
            xticklabels={$-2\pi$,$-3\pi/2$,$-\pi$, $-\pi/2$, $0$, $\pi/2$, $\pi$, $3\pi/2$, $2\pi$},
            ytick={-1,1},
            %ticks=none,
            width=6in,
            height=3in,
            unit vector ratio*=1 1 1,
            xlabel=$\theta$, ylabel=$x$,
            every axis y label/.style={at=(current axis.above origin),anchor=south},
            every axis x label/.style={at=(current axis.right of origin),anchor=west},
          ]        
          \addplot [ultra thick, penColor2, samples=100,smooth, domain=(-6.75:6.75)] {cos(deg(x))};
          \node at (axis cs:-1.57,.75) [penColor2] {$\cos(\theta)$};
        \end{axis}
\end{tikzpicture}
\end{image}
On the other hand, odd functions have $180^\circ$ \textit{rotational symmetry}
around the origin. Sine is an odd function:
\begin{image}
\begin{tikzpicture}
	\begin{axis}[
            xmin=-6.75,xmax=6.75,ymin=-1.5,ymax=1.5,
            axis lines=center,
            xtick={-6.28, -4.71, -3.14, -1.57, 0, 1.57, 3.142, 4.71, 6.28},
            xticklabels={$-2\pi$,$-3\pi/2$,$-\pi$, $-\pi/2$, $0$, $\pi/2$, $\pi$, $3\pi/2$, $2\pi$},
            ytick={-1,1},
            %ticks=none,
            width=6in,
            height=3in,
            unit vector ratio*=1 1 1,
            xlabel=$\theta$, ylabel=$x$,
            every axis y label/.style={at=(current axis.above origin),anchor=south},
            every axis x label/.style={at=(current axis.right of origin),anchor=west},
          ]        
          \addplot [ultra thick, penColor, samples=100,smooth, domain=(-6.75:6.75)] {sin(deg(x))};
          
          \node at (axis cs:3.14,.75) [penColor] {$\sin(\theta)$};
        \end{axis}
\end{tikzpicture}
\end{image}
\begin{question}
  Let $f$ be an odd function defined for all real numbers. Compute:
  \[
  \int_{-2}^2 f(x) \d x \begin{prompt}=\answer{0}\end{prompt}
  \]
  \begin{hint}
    Since our function is odd, it must look something like:
    \begin{image}
      \begin{tikzpicture}
        \begin{axis}[
            xmin=-2.5, xmax=2.5,ymin=-1,ymax=1,domain=-2.2:2.2,
            axis lines =center, xlabel=$x$, ylabel=$y$,
            every axis y label/.style={at=(current axis.above origin),anchor=south},
            every axis x label/.style={at=(current axis.right of origin),anchor=west},
            axis on top,
          ] 
          \addplot [penColor,ultra thick,smooth] {sin(deg(x))*sin(deg(x^2/1.3))};
        \end{axis}
      \end{tikzpicture}
    \end{image}
  \end{hint}
  \begin{hint}
    The integral above computes the following net area:
    \begin{image}
      \begin{tikzpicture}
        \begin{axis}[
            xmin=-2.5, xmax=2.5,ymin=-1,ymax=1,domain=-2.2:2.2,
            axis lines =center, xlabel=$x$, ylabel=$y$,
            every axis y label/.style={at=(current axis.above origin),anchor=south},
            every axis x label/.style={at=(current axis.right of origin),anchor=west},
            axis on top,
          ]
          \addplot [draw=none,fill=fillp,domain=0:2, smooth] {sin(deg(x))*sin(deg(x^2/1.3))} \closedcycle;
          \addplot [draw=none,fill=filln,domain=-2:0, smooth] {sin(deg(x))*sin(deg(x^2/1.3))} \closedcycle;
          \addplot [penColor,ultra thick,smooth] {sin(deg(x))*sin(deg(x^2/1.3))};
        \end{axis}
      \end{tikzpicture}
    \end{image}
  \end{hint}
\end{question}


\begin{question}
  Let $f$ be an odd function defined for all real numbers. Which of
  the following are equal to
  \[
  \int_2^4 f(x) \d x ?
  \]
  \begin{selectAll}
    \choice{$\int_{4}^{2} f(x) \d x$}
    \choice{$\int_{-4}^{-2} f(x) \d x$}
    \choice[correct]{$\int_{-2}^{-4} f(x) \d x$}
    \choice[correct]{$\int_{-2}^{4} f(x) \d x$}
    \choice{$\int_{4}^{-2} f(x) \d x$}
    \choice[correct]{$\int_{2}^{-4} f(x) \d x$}
    \choice{$\int_{-4}^{2} f(x) \d x$}
    \choice[correct]{$-\int_{-4}^{2} f(x) \d x$}
    \choice[correct]{$-\int_{-4}^{-2} f(x) \d x$}
  \end{selectAll}
\end{question}




\section{Net area versus geometric area}


We know that the net area of the region between a curve $y=f(x)$ and the $x$-axis
on $[a,b]$ is given by
\[
\int_a^b f(x) \d x.
\]
On the other hand, if we want to know the \textit{geometric area},
meaning the ``actual'' area, we compute
\[
\int_a^b |f(x)| \d x.
\]
\begin{example}
The graph of a function $f$ is given in the figure below.
 \begin{image}
  \begin{tikzpicture}[
      declare function = {f(\x) = (1/4)*(x-1)*(x-5)* (x-8);}]
    \begin{axis}[  
        domain=0:9, xmin =-1,xmax=9.1,ymax=10,ymin=-10,
        width=6in,
        height=3in,xtick={0,1,...,9},
        xticklabels={0,1,...,9},
        %% ytick style={draw=none},
    %% yticklabels={},
        axis lines=center, xlabel=$x$, ylabel=$y$,
        every axis y label/.style={at=(current axis.above origin),anchor=south},
        every axis x label/.style={at=(current axis.right of origin),anchor=west},
        axis on top,
      ]
        \addplot [draw=none,fill=fillp,domain=0:1, smooth] {f(x)} \closedcycle;
       \addplot [draw=none,fill=fillp,domain=1:5, smooth] {f(x)} \closedcycle;
            \addplot [draw=none,fill=fillp,domain=5:8, smooth] {f(x)} \closedcycle;
              \addplot [draw=none,fill=fillp,domain=8:9, smooth] {f(x)} \closedcycle;
          \addplot [ultra thick,penColor, smooth] {f(x)};
            \node at (axis cs:0.5,-1.2) {\large$A_1$};
              \node at (axis cs:3,1.2) {\large$A_2$};
              \node at (axis cs:6.5,-1.2) {\large$A_3$};
              \node at (axis cs:8.5,1.2) {\large$A_4$};
                \node at (axis cs:6,6.2) {\large$y=f(x)$};
        \end{axis}  
  \end{tikzpicture}
\end{image}
\begin{enumerate}
\item Express the  geometric  area of the region between the curve $y=f(x)$ and the $x$-axis on the interval $[0,9]$ as a definite integral.
\item Express the  geometric  area of the region between the curve $y=f(x)$ and the $x$-axis on the interval $[0,9]$ in terms of  definite integrals of $f$.
\item Express the  geometric  area of the region between the curve $y=f(x)$ and the $x$-axis on the interval $[0,9]$ in terms of areas $A_1$, $A_2$, $A_3$ and $A_4$.

\begin{explanation}
\begin{enumerate}
\item The geometric area is given by the  integral
 \[
  \int_0^9 |f(x)| \d x 
  \]
The figure below depicts the graph of $|f|$.
 \begin{image}
  \begin{tikzpicture}[
      declare function = {f(\x) = (1/4)*(x-1)*(x-5)* (x-8);}]
    \begin{axis}[  
        domain=0:9, xmin =-1,xmax=9.1,ymax=10,ymin=-10,
        width=6in,
        height=3in,xtick={0,1,...,9},
        xticklabels={0,1,...,9},
        %% ytick style={draw=none},
    %% yticklabels={},
        axis lines=center, xlabel=$x$, ylabel=$y$,
        every axis y label/.style={at=(current axis.above origin),anchor=south},
        every axis x label/.style={at=(current axis.right of origin),anchor=west},
        axis on top,
      ]
        \addplot [draw=none,fill=fillp,domain=0:1, smooth] {-f(x)} \closedcycle;
       \addplot [draw=none,fill=fillp,domain=1:5, smooth] {f(x)} \closedcycle;
            \addplot [draw=none,fill=fillp,domain=5:8, smooth] {-f(x)} \closedcycle;
              \addplot [draw=none,fill=fillp,domain=8:9, smooth] {f(x)} \closedcycle;
          \addplot [ultra thick,penColor,domain=0:1, smooth] {-f(x)};
            \addplot [ultra thick,penColor,domain=1:5, smooth] {f(x)};
              \addplot [ultra thick,penColor,domain=5:8, smooth] {-f(x)};
                \addplot [ultra thick,penColor,domain=8:9, smooth] {f(x)};
            \node at (axis cs:0.5,1.2) {\large$A_1$};
              \node at (axis cs:3,1.2) {\large$A_2$};
              \node at (axis cs:6.5,1.2) {\large$A_3$};
              \node at (axis cs:8.5,1.2) {\large$A_4$};
                \node at (axis cs:6,6.2) {\large$y=|f(x)|$};
        \end{axis}  
  \end{tikzpicture}
\end{image}
\item Using the properties of definite integrals, we can write
\begin{align*}
  \int_0^9 |f(x)| \d x = \int_0^1 |f(x) |\d x &+ \int_1^5 |f(x)| \d x\\
  &+ \int_5^8 |f(x)| \d x\\
  &+ \int_8^9 |f(x)| \d x 
\end{align*}
Rewriting the absolute value signs:
\begin{align*}
  &= \int_0^1 (-f(x)) \d x+ \int_1^5 f(x) \d x+ \int_5^8 (-f(x)) \d x+ \int_8^9 f(x) \d x\\
  &= -\int_0^1 f(x) \d x+ \int_1^5 f(x) \d x- \int_5^8 f(x) \d x+ \int_8^9 f(x) \d x
 \end{align*}
  \item
   \[
  \int_0^9 |f(x)| \d x =A_1 +A_2+A_3+A_4
  \]

 \end{enumerate}
\end{explanation}
 \end{enumerate}
\end{example}
\begin{question}
  True or false:
  \[
  \int_a^b |f(x)| \d x = \left|\int_a^b f(x) \d x\right|
  \]
  \begin{prompt}
  \begin{multipleChoice}
    \choice{true}
    \choice[correct]{false}
  \end{multipleChoice}
  \begin{feedback}
    Consider $f(x) = x$ on the interval $[-1,1]$. Here
    \[
    \int_a^b |f(x)| \d x = 1 \qquad\text{but}\qquad \left|\int_a^b
    f(x) \d x\right| = 0.
    \]
  \end{feedback}
  \end{prompt}
\end{question}


\end{document}
