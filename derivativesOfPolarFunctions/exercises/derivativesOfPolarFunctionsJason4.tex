\documentclass{ximera}

%\usepackage{todonotes}
%\usepackage{mathtools} %% Required for wide table Curl and Greens
%\usepackage{cuted} %% Required for wide table Curl and Greens
\newcommand{\todo}{}

\usepackage{esint} % for \oiint
\ifxake%%https://math.meta.stackexchange.com/questions/9973/how-do-you-render-a-closed-surface-double-integral
\renewcommand{\oiint}{{\large\bigcirc}\kern-1.56em\iint}
\fi


\graphicspath{
  {./}
  {ximeraTutorial/}
  {basicPhilosophy/}
  {functionsOfSeveralVariables/}
  {normalVectors/}
  {lagrangeMultipliers/}
  {vectorFields/}
  {greensTheorem/}
  {shapeOfThingsToCome/}
  {dotProducts/}
  {partialDerivativesAndTheGradientVector/}
  {../productAndQuotientRules/exercises/}
  {../normalVectors/exercisesParametricPlots/}
  {../continuityOfFunctionsOfSeveralVariables/exercises/}
  {../partialDerivativesAndTheGradientVector/exercises/}
  {../directionalDerivativeAndChainRule/exercises/}
  {../commonCoordinates/exercisesCylindricalCoordinates/}
  {../commonCoordinates/exercisesSphericalCoordinates/}
  {../greensTheorem/exercisesCurlAndLineIntegrals/}
  {../greensTheorem/exercisesDivergenceAndLineIntegrals/}
  {../shapeOfThingsToCome/exercisesDivergenceTheorem/}
  {../greensTheorem/}
  {../shapeOfThingsToCome/}
  {../separableDifferentialEquations/exercises/}
  {vectorFields/}
}

\newcommand{\mooculus}{\textsf{\textbf{MOOC}\textnormal{\textsf{ULUS}}}}

\usepackage{tkz-euclide}\usepackage{tikz}
\usepackage{tikz-cd}
\usetikzlibrary{arrows}
\tikzset{>=stealth,commutative diagrams/.cd,
  arrow style=tikz,diagrams={>=stealth}} %% cool arrow head
\tikzset{shorten <>/.style={ shorten >=#1, shorten <=#1 } } %% allows shorter vectors

\usetikzlibrary{backgrounds} %% for boxes around graphs
\usetikzlibrary{shapes,positioning}  %% Clouds and stars
\usetikzlibrary{matrix} %% for matrix
\usepgfplotslibrary{polar} %% for polar plots
\usepgfplotslibrary{fillbetween} %% to shade area between curves in TikZ
\usetkzobj{all}
\usepackage[makeroom]{cancel} %% for strike outs
%\usepackage{mathtools} %% for pretty underbrace % Breaks Ximera
%\usepackage{multicol}
\usepackage{pgffor} %% required for integral for loops



%% http://tex.stackexchange.com/questions/66490/drawing-a-tikz-arc-specifying-the-center
%% Draws beach ball
\tikzset{pics/carc/.style args={#1:#2:#3}{code={\draw[pic actions] (#1:#3) arc(#1:#2:#3);}}}



\usepackage{array}
\setlength{\extrarowheight}{+.1cm}
\newdimen\digitwidth
\settowidth\digitwidth{9}
\def\divrule#1#2{
\noalign{\moveright#1\digitwidth
\vbox{\hrule width#2\digitwidth}}}





\newcommand{\RR}{\mathbb R}
\newcommand{\R}{\mathbb R}
\newcommand{\N}{\mathbb N}
\newcommand{\Z}{\mathbb Z}

\newcommand{\sagemath}{\textsf{SageMath}}


%\renewcommand{\d}{\,d\!}
\renewcommand{\d}{\mathop{}\!d}
\newcommand{\dd}[2][]{\frac{\d #1}{\d #2}}
\newcommand{\pp}[2][]{\frac{\partial #1}{\partial #2}}
\renewcommand{\l}{\ell}
\newcommand{\ddx}{\frac{d}{\d x}}

\newcommand{\zeroOverZero}{\ensuremath{\boldsymbol{\tfrac{0}{0}}}}
\newcommand{\inftyOverInfty}{\ensuremath{\boldsymbol{\tfrac{\infty}{\infty}}}}
\newcommand{\zeroOverInfty}{\ensuremath{\boldsymbol{\tfrac{0}{\infty}}}}
\newcommand{\zeroTimesInfty}{\ensuremath{\small\boldsymbol{0\cdot \infty}}}
\newcommand{\inftyMinusInfty}{\ensuremath{\small\boldsymbol{\infty - \infty}}}
\newcommand{\oneToInfty}{\ensuremath{\boldsymbol{1^\infty}}}
\newcommand{\zeroToZero}{\ensuremath{\boldsymbol{0^0}}}
\newcommand{\inftyToZero}{\ensuremath{\boldsymbol{\infty^0}}}



\newcommand{\numOverZero}{\ensuremath{\boldsymbol{\tfrac{\#}{0}}}}
\newcommand{\dfn}{\textbf}
%\newcommand{\unit}{\,\mathrm}
\newcommand{\unit}{\mathop{}\!\mathrm}
\newcommand{\eval}[1]{\bigg[ #1 \bigg]}
\newcommand{\seq}[1]{\left( #1 \right)}
\renewcommand{\epsilon}{\varepsilon}
\renewcommand{\phi}{\varphi}


\renewcommand{\iff}{\Leftrightarrow}

\DeclareMathOperator{\arccot}{arccot}
\DeclareMathOperator{\arcsec}{arcsec}
\DeclareMathOperator{\arccsc}{arccsc}
\DeclareMathOperator{\si}{Si}
\DeclareMathOperator{\scal}{scal}
\DeclareMathOperator{\sign}{sign}


%% \newcommand{\tightoverset}[2]{% for arrow vec
%%   \mathop{#2}\limits^{\vbox to -.5ex{\kern-0.75ex\hbox{$#1$}\vss}}}
\newcommand{\arrowvec}[1]{{\overset{\rightharpoonup}{#1}}}
%\renewcommand{\vec}[1]{\arrowvec{\mathbf{#1}}}
\renewcommand{\vec}[1]{{\overset{\boldsymbol{\rightharpoonup}}{\mathbf{#1}}}\hspace{0in}}

\newcommand{\point}[1]{\left(#1\right)} %this allows \vector{ to be changed to \vector{ with a quick find and replace
\newcommand{\pt}[1]{\mathbf{#1}} %this allows \vec{ to be changed to \vec{ with a quick find and replace
\newcommand{\Lim}[2]{\lim_{\point{#1} \to \point{#2}}} %Bart, I changed this to point since I want to use it.  It runs through both of the exercise and exerciseE files in limits section, which is why it was in each document to start with.

\DeclareMathOperator{\proj}{\mathbf{proj}}
\newcommand{\veci}{{\boldsymbol{\hat{\imath}}}}
\newcommand{\vecj}{{\boldsymbol{\hat{\jmath}}}}
\newcommand{\veck}{{\boldsymbol{\hat{k}}}}
\newcommand{\vecl}{\vec{\boldsymbol{\l}}}
\newcommand{\uvec}[1]{\mathbf{\hat{#1}}}
\newcommand{\utan}{\mathbf{\hat{t}}}
\newcommand{\unormal}{\mathbf{\hat{n}}}
\newcommand{\ubinormal}{\mathbf{\hat{b}}}

\newcommand{\dotp}{\bullet}
\newcommand{\cross}{\boldsymbol\times}
\newcommand{\grad}{\boldsymbol\nabla}
\newcommand{\divergence}{\grad\dotp}
\newcommand{\curl}{\grad\cross}
%\DeclareMathOperator{\divergence}{divergence}
%\DeclareMathOperator{\curl}[1]{\grad\cross #1}
\newcommand{\lto}{\mathop{\longrightarrow\,}\limits}

\renewcommand{\bar}{\overline}

\colorlet{textColor}{black}
\colorlet{background}{white}
\colorlet{penColor}{blue!50!black} % Color of a curve in a plot
\colorlet{penColor2}{red!50!black}% Color of a curve in a plot
\colorlet{penColor3}{red!50!blue} % Color of a curve in a plot
\colorlet{penColor4}{green!50!black} % Color of a curve in a plot
\colorlet{penColor5}{orange!80!black} % Color of a curve in a plot
\colorlet{penColor6}{yellow!70!black} % Color of a curve in a plot
\colorlet{fill1}{penColor!20} % Color of fill in a plot
\colorlet{fill2}{penColor2!20} % Color of fill in a plot
\colorlet{fillp}{fill1} % Color of positive area
\colorlet{filln}{penColor2!20} % Color of negative area
\colorlet{fill3}{penColor3!20} % Fill
\colorlet{fill4}{penColor4!20} % Fill
\colorlet{fill5}{penColor5!20} % Fill
\colorlet{gridColor}{gray!50} % Color of grid in a plot

\newcommand{\surfaceColor}{violet}
\newcommand{\surfaceColorTwo}{redyellow}
\newcommand{\sliceColor}{greenyellow}




\pgfmathdeclarefunction{gauss}{2}{% gives gaussian
  \pgfmathparse{1/(#2*sqrt(2*pi))*exp(-((x-#1)^2)/(2*#2^2))}%
}


%%%%%%%%%%%%%
%% Vectors
%%%%%%%%%%%%%

%% Simple horiz vectors
\renewcommand{\vector}[1]{\left\langle #1\right\rangle}


%% %% Complex Horiz Vectors with angle brackets
%% \makeatletter
%% \renewcommand{\vector}[2][ , ]{\left\langle%
%%   \def\nextitem{\def\nextitem{#1}}%
%%   \@for \el:=#2\do{\nextitem\el}\right\rangle%
%% }
%% \makeatother

%% %% Vertical Vectors
%% \def\vector#1{\begin{bmatrix}\vecListA#1,,\end{bmatrix}}
%% \def\vecListA#1,{\if,#1,\else #1\cr \expandafter \vecListA \fi}

%%%%%%%%%%%%%
%% End of vectors
%%%%%%%%%%%%%

%\newcommand{\fullwidth}{}
%\newcommand{\normalwidth}{}



%% makes a snazzy t-chart for evaluating functions
%\newenvironment{tchart}{\rowcolors{2}{}{background!90!textColor}\array}{\endarray}

%%This is to help with formatting on future title pages.
\newenvironment{sectionOutcomes}{}{}



%% Flowchart stuff
%\tikzstyle{startstop} = [rectangle, rounded corners, minimum width=3cm, minimum height=1cm,text centered, draw=black]
%\tikzstyle{question} = [rectangle, minimum width=3cm, minimum height=1cm, text centered, draw=black]
%\tikzstyle{decision} = [trapezium, trapezium left angle=70, trapezium right angle=110, minimum width=3cm, minimum height=1cm, text centered, draw=black]
%\tikzstyle{question} = [rectangle, rounded corners, minimum width=3cm, minimum height=1cm,text centered, draw=black]
%\tikzstyle{process} = [rectangle, minimum width=3cm, minimum height=1cm, text centered, draw=black]
%\tikzstyle{decision} = [trapezium, trapezium left angle=70, trapezium right angle=110, minimum width=3cm, minimum height=1cm, text centered, draw=black]


\author{Jason Miller}
\license{Creative Commons 3.0 By-bC}


\outcome{}

\begin{document}
\begin{exercise}


Consider the polar curve $r=1+\sin(\theta)$. The graph is show below. 




\begin{image}  
  \begin{tikzpicture}  
    \begin{axis}[  
        xmin=-2.5,  
        xmax=2.5,  
        ymin=-1.5,  
        ymax=2.5,  
        axis lines=center,  
        xlabel=$x$,  
        ylabel=$y$,  
        every axis y label/.style={at=(current axis.above origin),anchor=south},  
        every axis x label/.style={at=(current axis.right of origin),anchor=west},  
      ]  
      \addplot[data cs=polar,blue,domain=0:360,samples=360,smooth, thick] (x,{1+sin(x)});
            \end{axis}  
  \end{tikzpicture}  
\end{image} 

We want to find all the horizontal and vertical tangent lines to this curve.

First we find the $\theta$ values where $r=1+\sin(\theta)$ has a horizontal tangent line and give the equation of the line. 

Since our curve can be generated by letting $\theta$ vary from over the interval $[0, 2\pi)$, we can assume the $\theta$ values lie in $[0, 2\pi)$. List the $\theta$ values in order from smallest to largest below: 


When $\theta=\answer{\frac{\pi}{2}}$ the equation of the tangent line is $\answer{y=2  }$.

When $\theta=\answer{\frac{7\pi}{6}}$ the equation of the tangent line is $\answer{y=-\frac{1}{4}}$ and when $\theta=\answer{ \frac{11\pi}{6}}$ the equation of 
the tangent line is $\answer{ y=-\frac{1}{4}}$. 




\begin{hint}

Consider the formulas for changing from polar coordinates to Cartesian coordinates:

\begin{align*}
x&=r\cos(\theta) \\
y&=r\sin(\theta)
\end{align*}

Using the equation of our curve $r=\cos(\theta)$, we can substitute for $r$ to obtain:

\begin{align*}
x&=\answer{(1+\sin(\theta))\cos(\theta)} \\
y&=\answer{(1+\sin(\theta))\sin(\theta) }
\end{align*}

Note that we have expressed both the $x$ and $y$ coordinates of the points of the curve in terms of functions of a single parameter $\theta$.

Since we have a parametric description of our curve in terms of $\theta$, 
we can use the chain rule to express $\dd[y]{x}$ in terms of $\theta$ to get $\dd[y]{x}=\frac{ dy/d\theta}{dx/d\theta}$.

Calculating we get

\begin{align*}
\dd[x]{\theta}&=\answer{ -\sin(\theta)+\cos^2(\theta)-\sin^2(\theta)  } \\
\dd[y]{\theta}&=\answer{ \cos(\theta) + 2\sin(\theta)\cos(\theta)   }
\end{align*} 

Thus we find $\dd[y]{x}=\answer{\frac{ \cos(\theta) + 2\sin(\theta)\cos(\theta) }{-\sin(\theta)+\cos^2(\theta)-\sin^2(\theta) }}$. 

The tangent line is horizontal when the slope is $0$. Therefore we need to find where $\dd[y]{x}$ is equal to $0$.

Setting the numerator of $\dd[y]{x}$ equal to $0$ gives us $\cos(\theta) + 2\sin(\theta)\cos(\theta)=0$. 

We can pull a $\cos(\theta)$ out of each term to get $\answer{\cos(\theta)(1+2\sin(\theta))=0 }$. 

Thus we have $\cos(\theta)=0$ or $1+2\sin(\theta)=0$. 


(Notice that our curve $r=\cos(\theta)$ can be generated by letting $\theta$ vary over the interval $[0, 2\pi)$, we solve these equations for $\theta$ values in  $[0, 2\pi)$).

Below list the $\theta$ values in order from smallest to largest.  

When $\cos(\theta)=0$, we get $\theta=\answer{\frac{\pi}{2}}$ and $\theta=\answer{\frac{3\pi}{2}}$.  

When $1+2\sin(\theta)=0$, we get $\theta=\answer{ \frac{7\pi}{6}}$ and $\theta=\answer{\frac{11\pi}{6}}$.

Notice that when $\theta=\frac{3\pi}{2}$ then the denominator of $\dd[y]{x}$ is $0$ as well. This corresponds to the cusp (sharp corner) on the graph at the origin. Recall that the tangent line at a given point is supposed to be the best linear approximation of the graph at the point. Our curve at $(0,0)$ does not look more and more like a line as we zoom in on the origin. 

\end{hint}

\begin{exercise}



Now we want to find the $\theta$ values where $r=\cos(\theta)$ has a vertical tangent line and give the equation of the line. 

Again assume the $\theta$ values lie in $[0, 2\pi)$. List the $\theta$ values in order below: 

When $\theta=\answer{\frac{\pi}{6}}$ the equation of the tangent line is $\answer{x=\frac{3\sqrt{3}}{4} }$ 

and when $\theta=\answer{ \frac{5\pi}{6}}$ the equation of the tangent line is $\answer{x=\frac{-3\sqrt{3}}{4}}$. 





\begin{hint}


In the hint for exercise 1 we derived a parametric description of our curve in terms of $\theta$: 

\begin{align*}
\dd[x]{\theta}&=\answer{  -\sin(\theta)+\cos^2(\theta)-\sin^2(\theta) } \\
\dd[y]{\theta}&=\answer{ \cos(\theta) + 2\sin(\theta)\cos(\theta)    }
\end{align*} 

Where we calculated $\dd[y]{x}=\answer{ \frac{ \cos(\theta) + 2\sin(\theta)\cos(\theta) }{-\sin(\theta)+\cos^2(\theta)-\sin^2(\theta) }}$. 

The tangent line is vertical when the slope becomes infinite. We can check for this by looking for where the denominator (but not the numerator) of $\dd[y]{x}$ is equal to $0$.

Setting the denominator equal to $0$ gives us $ -\sin(\theta)+\cos^2(\theta)-\sin^2(\theta)=0$.  

Since two of the terms involve $\sin(\theta)$ we can convert the cosine term by writing $\cos^2(\theta)=1-\sin^2(\theta)$. 

We then get $-2\sin^2(\theta)-\sin(\theta)+1=0$. 

This is a quadratic polynomial in $\sin(\theta)$ so we can try to factor it. We obtain $(2\sin(\theta)-1)(\sin(\theta)+1)=0$. 


Setting $2\sin(\theta)-1=0$ gives us $\theta=\answer{ \frac{\pi}{6}}$ and $\theta=\answer{\frac{5\pi}{6}}$. 


and setting $\sin(\theta)+1=0$ gives us $\theta=\answer{ \frac{3\pi}{2}}$. 

(Above list the $\theta$ values in order from smallest to largest and note we take $\theta$ to lie in the interval $[0, 2\pi)$).


\end{hint}

\end{exercise}
\end{exercise}
\end{document}
