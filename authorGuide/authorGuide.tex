\documentclass{article}


\usepackage{amsmath,amssymb}

\newcommand{\RR}{\mathbb R}
\renewcommand{\d}{\,d}
\newcommand{\dd}[2][]{\frac{d #1}{d #2}}
\renewcommand{\l}{\ell}
\newcommand{\ddx}{\frac{d}{dx}}
\everymath{\displaystyle}
\newcommand{\dfn}{\textbf}
\newcommand{\eval}[1]{\bigg[ #1 \bigg]}


\usepackage{hyperref}


\newcommand{\mooculus}{\textsf{\textbf{MOOC}\textnormal{\textsf{ULUS}}}}

\begin{document}

\section*{Author guide for \mooculus}

\subsection*{Arbitration policy}

There is a tremendous danger of, for instance, otherwise productive
weekly Code Reviews being overtaken by
\href{http://en.wikipedia.org/wiki/Parkinson's_law_of_triviality}{bikeshedding}---or more substantive disagreements.

If, in the process of producing the MOOCulus textbook, it happens that
there is a disagreement which cannot be abstracted away by
incorporating appropriate \LaTeX\ macros, then it is time for one of
those present to propose a rule addressing the disagreement.  Those
involved in the disagreement at the Code Review then vote: if 85\% of
those present agree, the proposed rule is incorporated into this
document.  If near consensus is not achieved, then three textbooks
(e.g., Spivak, Stewart, Briggs/Cochran) are consulted to formulate the
rule.

\subsection*{Basic guidelines}

The first letter of the first word of a title is capitalized, then all
other words are lowercase, except for proper nouns. For example:
\begin{quote}
Introduction to Newton's method
\end{quote}
No punctuation is used at the end, except for perhaps a question mark.


The abstract is a one-sentence description of the activity. It is
intended to give the instructor an idea of what the activity is about.


Do not hack the \LaTeX\ document to make it appear in a customized
way. Do not add vertical space, boxes around formulas, etc. This
should be done by the Ximera conversion. Hence, if some special
formatting is needed, it should be done at the level of the
ximera.cls.


\subsection*{Structure guidelines}

Each lecture corresponds in \mooculus to three distinct sections:
\begin{description}
\item[Break Ground] An introduction to the topic.
\item[Dig In] A detailed discussion of the topic.
\item[Reinforce] Practice problems.
\end{description}

\paragraph{Launch}

The Break-Ground should somehow present a ``mystery'' for the students
to solve.  This gives an ``intellectual need'' for the material being
covered. Moreover, it should lead them to the right path. Ideally if
we had a bright student working on the Break-Ground, they might even develop
the techniques from the lesson to solve the problem.  By the end of
the dig-in phase, the mystery is solved!

Typically the Break-Ground will be between $5$ and $15$ \textbf{Problems}.

Each Break-Ground needs to end with a \textbf{Xarma Boost} asking the
students to generate their own questions.


\paragraph{Dig-In}


The dig-in section is most like a traditional textbook. However, some key differences:


Every definition and every theorem should have a \textbf{Question}
following it to check for student comprehension.


The ``mystery'' presented in the Break-Ground must be solved here. 



\paragraph{Reinforce}

This section is a list of \textbf{Problems}. Though maybe we want to
ask \textbf{exercises} and \textbf{explorations} here too.



\subsection*{Problem-types}

Ximera supports a variety of problem-types. It is least confusing for
students if the author sticks to $1--2$ problem-types per activity.



For the Break-Ground, ``Problem'' should be used along with a final ``Xarma
Boost.''



For the Dig-In, ``Question'' and ``Examples'' should be used, though
the author must ensure that the ``questions'' they write are actually
questions. Moreover in ``Examples'' one actually writes statements,
where the students fill-in the blanks. In the handout version of the
textbook, these ``Examples'' become complete prose. 



For the Reinforce, ``Exercise'' should be used. Though ``Exploration''
might be a good option for difficult or open-ended problems.



In all cases, ``Hint'' can also be used as an additional problem-type.


\subsection*{Nested and delayed questions}


If a problem has more than one answer-type in it, all answers will
need to be....

\begin{verbatim}
\begin{problem}
$2+2 =$\answer{4}. The previous answer plus $1$ is \answer{5}.  
\end{problem}
\end{verbatim}

I THINK IT IS AN OPTIONAL ARG FOR ANSWER!!!


To write follow up questions, nest problem enviroments.

\begin{verbatim}
\begin{problem}
$2+2 =$\answer{4}. 
\begin{problem}
The previous answer plus $1$ is \answer{5}.  
\end{problem}
\end{problem}
\end{verbatim}
Note, \verb|hint| will not block future questions.



\subsection*{Images}

All images should be set within \verb|image| tags.
\begin{verbatim}
\begin{image}
\includegraphics{myFunkyImage.pdf}
\end{image}
\end{verbatim}
This will allow Ximera to ``know'' that an image is inserted, and
handle the image appropriately. All supporting files should be within
the same directory as the \LaTeX\ file that generates the Ximera
activity. This will allow for maximum usability of the Ximera
documents.

\subsection*{\LaTeX\ Macros}

\renewcommand{\arraystretch}{2}
\begin{tabular*}{1.0\textwidth}{lll}
\hline
Command & Example & Typeset \\
\hline
%\verb|\fxn| & \verb|\fxn{f}(x)| & $\fxn{f}(x)$\\
\verb|\RR| & \verb|f:\RR\to\RR| & $f:\RR\to\RR$\\ 
\verb|\ddx| & \verb|\ddx f(x)| & $\ddx f(x)$\\
\verb|\dd[_]{_}| & \verb|\dd[y]{x}| & $\dd[y]{x}$ \\
\verb|\d | & \verb|\int f(x) \d x| & $\int f(x) \d x$\\
\verb|\l| & \verb|\l(x) = mx+b| & $\l(x) = mx +b$\\
\verb|\dfn| & \verb|we define \dfn{this}| & we define \dfn{this}\\
\verb|\eval| & \verb|\eval{f(x)}_a^b| & $\eval{f(x)}_a^b$
\end{tabular*}



\end{document}
