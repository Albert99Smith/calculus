\documentclass{ximera}

\author{Bart Snapp}

%\usepackage{todonotes}
%\usepackage{mathtools} %% Required for wide table Curl and Greens
%\usepackage{cuted} %% Required for wide table Curl and Greens
\newcommand{\todo}{}

\usepackage{esint} % for \oiint
\ifxake%%https://math.meta.stackexchange.com/questions/9973/how-do-you-render-a-closed-surface-double-integral
\renewcommand{\oiint}{{\large\bigcirc}\kern-1.56em\iint}
\fi


\graphicspath{
  {./}
  {ximeraTutorial/}
  {basicPhilosophy/}
  {functionsOfSeveralVariables/}
  {normalVectors/}
  {lagrangeMultipliers/}
  {vectorFields/}
  {greensTheorem/}
  {shapeOfThingsToCome/}
  {dotProducts/}
  {partialDerivativesAndTheGradientVector/}
  {../productAndQuotientRules/exercises/}
  {../normalVectors/exercisesParametricPlots/}
  {../continuityOfFunctionsOfSeveralVariables/exercises/}
  {../partialDerivativesAndTheGradientVector/exercises/}
  {../directionalDerivativeAndChainRule/exercises/}
  {../commonCoordinates/exercisesCylindricalCoordinates/}
  {../commonCoordinates/exercisesSphericalCoordinates/}
  {../greensTheorem/exercisesCurlAndLineIntegrals/}
  {../greensTheorem/exercisesDivergenceAndLineIntegrals/}
  {../shapeOfThingsToCome/exercisesDivergenceTheorem/}
  {../greensTheorem/}
  {../shapeOfThingsToCome/}
  {../separableDifferentialEquations/exercises/}
  {vectorFields/}
}

\newcommand{\mooculus}{\textsf{\textbf{MOOC}\textnormal{\textsf{ULUS}}}}

\usepackage{tkz-euclide}\usepackage{tikz}
\usepackage{tikz-cd}
\usetikzlibrary{arrows}
\tikzset{>=stealth,commutative diagrams/.cd,
  arrow style=tikz,diagrams={>=stealth}} %% cool arrow head
\tikzset{shorten <>/.style={ shorten >=#1, shorten <=#1 } } %% allows shorter vectors

\usetikzlibrary{backgrounds} %% for boxes around graphs
\usetikzlibrary{shapes,positioning}  %% Clouds and stars
\usetikzlibrary{matrix} %% for matrix
\usepgfplotslibrary{polar} %% for polar plots
\usepgfplotslibrary{fillbetween} %% to shade area between curves in TikZ
\usetkzobj{all}
\usepackage[makeroom]{cancel} %% for strike outs
%\usepackage{mathtools} %% for pretty underbrace % Breaks Ximera
%\usepackage{multicol}
\usepackage{pgffor} %% required for integral for loops



%% http://tex.stackexchange.com/questions/66490/drawing-a-tikz-arc-specifying-the-center
%% Draws beach ball
\tikzset{pics/carc/.style args={#1:#2:#3}{code={\draw[pic actions] (#1:#3) arc(#1:#2:#3);}}}



\usepackage{array}
\setlength{\extrarowheight}{+.1cm}
\newdimen\digitwidth
\settowidth\digitwidth{9}
\def\divrule#1#2{
\noalign{\moveright#1\digitwidth
\vbox{\hrule width#2\digitwidth}}}





\newcommand{\RR}{\mathbb R}
\newcommand{\R}{\mathbb R}
\newcommand{\N}{\mathbb N}
\newcommand{\Z}{\mathbb Z}

\newcommand{\sagemath}{\textsf{SageMath}}


%\renewcommand{\d}{\,d\!}
\renewcommand{\d}{\mathop{}\!d}
\newcommand{\dd}[2][]{\frac{\d #1}{\d #2}}
\newcommand{\pp}[2][]{\frac{\partial #1}{\partial #2}}
\renewcommand{\l}{\ell}
\newcommand{\ddx}{\frac{d}{\d x}}

\newcommand{\zeroOverZero}{\ensuremath{\boldsymbol{\tfrac{0}{0}}}}
\newcommand{\inftyOverInfty}{\ensuremath{\boldsymbol{\tfrac{\infty}{\infty}}}}
\newcommand{\zeroOverInfty}{\ensuremath{\boldsymbol{\tfrac{0}{\infty}}}}
\newcommand{\zeroTimesInfty}{\ensuremath{\small\boldsymbol{0\cdot \infty}}}
\newcommand{\inftyMinusInfty}{\ensuremath{\small\boldsymbol{\infty - \infty}}}
\newcommand{\oneToInfty}{\ensuremath{\boldsymbol{1^\infty}}}
\newcommand{\zeroToZero}{\ensuremath{\boldsymbol{0^0}}}
\newcommand{\inftyToZero}{\ensuremath{\boldsymbol{\infty^0}}}



\newcommand{\numOverZero}{\ensuremath{\boldsymbol{\tfrac{\#}{0}}}}
\newcommand{\dfn}{\textbf}
%\newcommand{\unit}{\,\mathrm}
\newcommand{\unit}{\mathop{}\!\mathrm}
\newcommand{\eval}[1]{\bigg[ #1 \bigg]}
\newcommand{\seq}[1]{\left( #1 \right)}
\renewcommand{\epsilon}{\varepsilon}
\renewcommand{\phi}{\varphi}


\renewcommand{\iff}{\Leftrightarrow}

\DeclareMathOperator{\arccot}{arccot}
\DeclareMathOperator{\arcsec}{arcsec}
\DeclareMathOperator{\arccsc}{arccsc}
\DeclareMathOperator{\si}{Si}
\DeclareMathOperator{\scal}{scal}
\DeclareMathOperator{\sign}{sign}


%% \newcommand{\tightoverset}[2]{% for arrow vec
%%   \mathop{#2}\limits^{\vbox to -.5ex{\kern-0.75ex\hbox{$#1$}\vss}}}
\newcommand{\arrowvec}[1]{{\overset{\rightharpoonup}{#1}}}
%\renewcommand{\vec}[1]{\arrowvec{\mathbf{#1}}}
\renewcommand{\vec}[1]{{\overset{\boldsymbol{\rightharpoonup}}{\mathbf{#1}}}\hspace{0in}}

\newcommand{\point}[1]{\left(#1\right)} %this allows \vector{ to be changed to \vector{ with a quick find and replace
\newcommand{\pt}[1]{\mathbf{#1}} %this allows \vec{ to be changed to \vec{ with a quick find and replace
\newcommand{\Lim}[2]{\lim_{\point{#1} \to \point{#2}}} %Bart, I changed this to point since I want to use it.  It runs through both of the exercise and exerciseE files in limits section, which is why it was in each document to start with.

\DeclareMathOperator{\proj}{\mathbf{proj}}
\newcommand{\veci}{{\boldsymbol{\hat{\imath}}}}
\newcommand{\vecj}{{\boldsymbol{\hat{\jmath}}}}
\newcommand{\veck}{{\boldsymbol{\hat{k}}}}
\newcommand{\vecl}{\vec{\boldsymbol{\l}}}
\newcommand{\uvec}[1]{\mathbf{\hat{#1}}}
\newcommand{\utan}{\mathbf{\hat{t}}}
\newcommand{\unormal}{\mathbf{\hat{n}}}
\newcommand{\ubinormal}{\mathbf{\hat{b}}}

\newcommand{\dotp}{\bullet}
\newcommand{\cross}{\boldsymbol\times}
\newcommand{\grad}{\boldsymbol\nabla}
\newcommand{\divergence}{\grad\dotp}
\newcommand{\curl}{\grad\cross}
%\DeclareMathOperator{\divergence}{divergence}
%\DeclareMathOperator{\curl}[1]{\grad\cross #1}
\newcommand{\lto}{\mathop{\longrightarrow\,}\limits}

\renewcommand{\bar}{\overline}

\colorlet{textColor}{black}
\colorlet{background}{white}
\colorlet{penColor}{blue!50!black} % Color of a curve in a plot
\colorlet{penColor2}{red!50!black}% Color of a curve in a plot
\colorlet{penColor3}{red!50!blue} % Color of a curve in a plot
\colorlet{penColor4}{green!50!black} % Color of a curve in a plot
\colorlet{penColor5}{orange!80!black} % Color of a curve in a plot
\colorlet{penColor6}{yellow!70!black} % Color of a curve in a plot
\colorlet{fill1}{penColor!20} % Color of fill in a plot
\colorlet{fill2}{penColor2!20} % Color of fill in a plot
\colorlet{fillp}{fill1} % Color of positive area
\colorlet{filln}{penColor2!20} % Color of negative area
\colorlet{fill3}{penColor3!20} % Fill
\colorlet{fill4}{penColor4!20} % Fill
\colorlet{fill5}{penColor5!20} % Fill
\colorlet{gridColor}{gray!50} % Color of grid in a plot

\newcommand{\surfaceColor}{violet}
\newcommand{\surfaceColorTwo}{redyellow}
\newcommand{\sliceColor}{greenyellow}




\pgfmathdeclarefunction{gauss}{2}{% gives gaussian
  \pgfmathparse{1/(#2*sqrt(2*pi))*exp(-((x-#1)^2)/(2*#2^2))}%
}


%%%%%%%%%%%%%
%% Vectors
%%%%%%%%%%%%%

%% Simple horiz vectors
\renewcommand{\vector}[1]{\left\langle #1\right\rangle}


%% %% Complex Horiz Vectors with angle brackets
%% \makeatletter
%% \renewcommand{\vector}[2][ , ]{\left\langle%
%%   \def\nextitem{\def\nextitem{#1}}%
%%   \@for \el:=#2\do{\nextitem\el}\right\rangle%
%% }
%% \makeatother

%% %% Vertical Vectors
%% \def\vector#1{\begin{bmatrix}\vecListA#1,,\end{bmatrix}}
%% \def\vecListA#1,{\if,#1,\else #1\cr \expandafter \vecListA \fi}

%%%%%%%%%%%%%
%% End of vectors
%%%%%%%%%%%%%

%\newcommand{\fullwidth}{}
%\newcommand{\normalwidth}{}



%% makes a snazzy t-chart for evaluating functions
%\newenvironment{tchart}{\rowcolors{2}{}{background!90!textColor}\array}{\endarray}

%%This is to help with formatting on future title pages.
\newenvironment{sectionOutcomes}{}{}



%% Flowchart stuff
%\tikzstyle{startstop} = [rectangle, rounded corners, minimum width=3cm, minimum height=1cm,text centered, draw=black]
%\tikzstyle{question} = [rectangle, minimum width=3cm, minimum height=1cm, text centered, draw=black]
%\tikzstyle{decision} = [trapezium, trapezium left angle=70, trapezium right angle=110, minimum width=3cm, minimum height=1cm, text centered, draw=black]
%\tikzstyle{question} = [rectangle, rounded corners, minimum width=3cm, minimum height=1cm,text centered, draw=black]
%\tikzstyle{process} = [rectangle, minimum width=3cm, minimum height=1cm, text centered, draw=black]
%\tikzstyle{decision} = [trapezium, trapezium left angle=70, trapezium right angle=110, minimum width=3cm, minimum height=1cm, text centered, draw=black]


\title[Dig-In:]{Interpreting the gradient vector}
\begin{document}
\begin{abstract}
  The gradient is the fundamental notion of a derivative for a
  function of several variables.
\end{abstract}
\maketitle

\section{Three things about the gradient vector}

We have now learned much about the gradient vector. However,
\textbf{there are three things you must know about the gradient
  vector:}

\paragraph{First: You must know how to compute the gradient vector.}
Remember given a function $F:\R^n\to\R$:
\[
\grad F  = \vector{\pp[F]{x_1},\pp[F]{x_2},\dots,\pp[F]{x_n}}
\]
This is a vector-valued function of $n$ variables. This means when you
compute the gradient, you should express it as a vector!

\paragraph{Second: The gradient vector points in the initial direction of greatest increase for a function.}
Remember, the gradient vector of a function of $n$ variables is a
vector that lives in $\R^n$. The gradient vector tells you how to
immediately change the values of the inputs of a function to find the
initial greatest increase in the output of the function.
\begin{onlineOnly}
  We can see this in the interactive below. 
  \begin{center}
    \geogebra{wd5mrudh}{800}{600} %https://ggbm.at/wd5mrudh
  \end{center}
  The gradient at each point shows you which direction to change the
  $(x,y)$-values to get the greatest initial change in the $z$-value.
\end{onlineOnly}


\paragraph{Third: The gradient vector is orthogonal to level sets.}

In particular, given $F:\R^2\to\R$, the gradient vector $\grad
F\in\R^2$ is always orthogonal to the level curves $c =
F(x,y)$. Moreover, given $F:\R^3\to\R$, $\grad F \in \R^3$ is always
orthogonal to level surfaces.



\section{Computing the gradient vector}

Given a function of several variables, say $F:\R^2\to\R$, the
gradient, when evaluated at a point in the domain of $F$, is a vector
in $\R^2$.
\begin{onlineOnly}
  We can see this in the interactive below. 
  \begin{center}
    \geogebra{wd5mrudh}{800}{600} %https://ggbm.at/wd5mrudh
  \end{center}
  The gradient at each point is a vector pointing in the $(x,y)$-plane.
\end{onlineOnly}
You compute the gradient vector, by writing the vector:
\[
\grad F  = \vector{\pp[F]{x_1},\pp[F]{x_2},\dots,\pp[F]{x_n}}
\]
You've done this sort of direct computation many times before. So now,
try your hand at these puzzlers:

\begin{question}
  Consider a differentiable function $F:\R^2\to\R$ whose tangent plane
  at $(x,y) = (2,-1)$ is given by:
  \[
  z = 3x - 2y -1
  \]
  In this case what is $F(2,-1)$?
  \begin{prompt}
  \[
  F(2,-1) = \answer{7}
  \]
  \end{prompt}
  \begin{question}
    Suppose you know that $F^{(1,0)}(2,-1)>0$. What is $\grad
    F(2,-1)$?
    \begin{prompt}
      \[
      \grad F (2,-1) = \vector{\answer{3},\answer{-2}}
      \]
    \end{prompt}
  \end{question}
\end{question}


\begin{question}
  Consider a differentiable function $G:\R^2\to\R$ and the unit vector
  $\uvec{u} = \vector{1/\sqrt{2},1/\sqrt{2}}$. Suppose that
  $D_{\uvec{u}} (G(1,-3)) = 0$ and that $G^{(0,1)}(1,-3)=2$. Compute:
  \[
  \grad G(1,-3) \begin{prompt}
    = \vector{\answer{-2},\answer{2}}
  \end{prompt}
  \]
\end{question}
  
\begin{question}
  Consider a differentiable function $H:\R^2\to\R$ where
  $H^{(0,1)}(-5,6) = 3$ and the line $\vecl(t) =
  \vector{1-2t,3+t}$. Suppose that
  \[
  \eval{\dd{t} H(\vecl(t)) }_{t=3} = 5
  \]
  Compute:
  \[
  \grad H (-5,6) \begin{prompt}
    =\vector{\answer{-1},\answer{3}}
  \end{prompt}
  \]
  \begin{hint}
    Use the chain rule.
  \end{hint}
\end{question}



\section{The initial greatest increase}

Given a function $F:\R^n\to\R$ and point in $\R^n$, the gradient
vector tells you which initial direction to leave the point in order
to get the greatest increase in $F$. Why is this so? Well, to compute
the change in the output of a function when changing the inputs in a
specific direction, we should use the directional derivative. Recall:
\[
D_\uvec{u}(F) = \grad{F} \dotp \uvec{u}
\]
To make this change as large as possible, $\uvec{u}$ must be the
same direction as $\grad F$. Hence, it is the gradient vector that
points in the initial direction of greatest increase for the function.

We can directly witness that the gradient vector points in the initial
direction of greatest increase by looking at a differentiable function
$F:\R^2\to\R$ that is described by a table of values.

\begin{example}
  Let $F:\R^2\to\R$ be a differentiable function described by the
    following table of values:
    \begin{image}
      \begin{tikzpicture}[x=1cm,y=.75cm]
        \draw (0,0) grid [step=1] (6,5);
        
        \draw[ultra thick] (0,1)--(6,1);
        \draw[ultra thick] (1,0)--(1,5);
        
        \draw (0,0) -- (1,1);
        \node at (.4,.9) [below left,inner sep=1pt] {\small$y$};
        \node at (0.6,.1) [above right,inner sep=1pt] {\small$x$};
    
        %% y-values
        \node at (0.5,4.5) {$7$};
        \node at (0.5,3.5) {$6$};
        \node at (0.5,2.5) {$5$};
        \node at (0.5,1.5) {$4$};
        
        
        %% z-values
        %% top
        \node at (1.5,4.5) {$18$};
        \node at (2.5,4.5) {$10$};
        \node at (3.5,4.5) {$-1$};
        \node at (4.5,4.5) {$-3$};
        \node at (5.5,4.5) {$-7$};
        
        %% 
        \node at (1.5,3.5) {$24$};
        \node at (2.5,3.5) {$16$};
        \node at (3.5,3.5) {$5$};
        \node at (4.5,3.5) {$1$};
        \node at (5.5,3.5) {$-2$};
        
        %% 
        \node at (1.5,2.5) {$18$};
        \node at (2.5,2.5) {$5$};
        \node at (3.5,2.5) {$0$};
        \node at (4.5,2.5) {$-1$};
        \node at (5.5,2.5) {$-4$};
        
        %% 
        \node at (1.5,1.5) {$12$};
        \node at (2.5,1.5) {$2$};
        \node at (3.5,1.5) {$-3$};
        \node at (4.5,1.5) {$-4$};
        \node at (5.5,1.5) {$-6$};
        
        %% bottom row
        \node at (1.5,.5) {$1$};
        \node at (2.5,.5) {$2$};
        \node at (3.5,.5) {$3$};
        \node at (4.5,.5) {$4$};
        \node at (5.5,.5) {$5$};
      \end{tikzpicture}
    \end{image}
    Estimate $\grad F(3,5)$.
    \begin{explanation}
      We estimate $\grad F(3,5)$ by estimating the partial derivatives. 
      To estimate $F^{(1,0)}(3,5)$, we examine the change in $F(x,5)$
    between $x=4$ and $x=3$:
    \[
    \frac{F(4,5)-F\left(\answer[given]{3},5\right)}{\answer[given]{4}-3}= \answer[given]{-1}
    \]
    We should also examine the change in $F(x,5)$ between $x=3$ and
    $x=2$:
    \[
      \frac{F(3,5)-F\left(\answer[given]{2},5\right)}{\answer[given]{3}-2} =\answer[given]{-5}  
    \]
    Now if we average these values together, we see:
    \[
    \eval{\pp{x} F(x,y)}_{(x,y)=(3,5)} \approx \answer[given]{-3}
    \]
    On the other hand, using a similar procedure, we find that:
    \[
    \eval{\pp{y} F(x,y)}_{(x,y)=(3,5)} \approx \answer[given]{4}
    \]
    Thus the gradient is
    \[
    \grad F(3,5) = \vector{\answer[given]{-3},\answer[given]{4}}
    \]
    Note if you leave the point $(3,5)$ in the direction of $\grad
    F(3,5) = \vector{\answer[given]{-3},\answer[given]{4}}$, you head
    toward $F(2,6)= \answer[given]{16}$, the greatest initial increase
    from $(3,5)$.
    \end{explanation}
\end{example}

\begin{question}
  Here is a plot of an elliptic paraboloid $G(x,y) = x^2 + y^2$ along
  with a vector attached to a point on the surface:
  \begin{image}
    \begin{tikzpicture}
      \begin{axis}%
        [tick label style={font=\scriptsize},axis on top,
	  axis lines=center,
	  view={155}{25},
	  name=myplot,
	  %xtick=\empty,
	  %ytick={5},
	  %ztick={.7,-.7},
	  %minor xtick=1,
	  %minor ytick=1,
	  ymin=-4.4,ymax=4.5,
	  xmin=-4.5,xmax=4.5,
	  zmin=-1.1, zmax=17,
	  every axis x label/.style={at={(axis cs:\pgfkeysvalueof{/pgfplots/xmax},0,0)},xshift=-5pt,yshift=-1pt},
	  xlabel={\scriptsize $x$},
	  every axis y label/.style={at={(axis cs:0,\pgfkeysvalueof{/pgfplots/ymax},0)},xshift=4pt,yshift=-4pt},
	  ylabel={\scriptsize $y$},
	  every axis z label/.style={at={(axis cs:0,0,\pgfkeysvalueof{/pgfplots/zmax})},xshift=0pt,yshift=4pt},
	  zlabel={\scriptsize $z$},colormap/cool
        ]
        \addplot3[domain=-3:3,y domain=-3:3,mesh,samples y=15,very thin,z buffer=sort,%opacity=.6,
          samples=15,] (x,y,{x^2+y^2});
        \addplot3[ultra thick, penColor, ->] coordinates {(-2,2,8) (-3,3,16)};
        \filldraw [black] (axis cs:-2,2,8) circle (2.5pt);        
      \end{axis}
    \end{tikzpicture}
  \end{image}
  True or false: The vector above could be the gradient vector for $G$
  at the given point.
  \begin{prompt}
  \begin{multipleChoice}
    \choice{True}
    \choice[correct]{False}
  \end{multipleChoice}
  \begin{feedback}
    The answer is ``False.'' Here the graph of the function is three
    dimensional. The gradient vector is in one less dimension than the
    function's graph. Hence the gradient of $G$ is in fact always a
    two dimensional vector.
  \end{feedback}
  \end{prompt}
\end{question}

So far we have mostly talked about the direction of the gradient
vector. Now let's talk about the \textit{magnitude} of the gradient
vector. The magnitude of the gradient vector tells you ``how fast''
the function is increasing.


\begin{question}
  Suppose you have a differentiable function $F:\R^2\to\R$ with the
  following set of level curves.  You should interpolate reasonable
  values of the function $F$ between the level curves which are shown:
  \begin{image}
    \begin{tikzpicture}	
      \draw[ultra thick, penColor] (0,0) ellipse (1.1cm and .8cm);
      \draw[ultra thick, penColor] (.3,0) ellipse (1.5cm and 1cm);
      \draw[ultra thick, penColor] (.7,0) ellipse (2cm and 1.2cm);
      \draw[ultra thick, penColor] (1,0) ellipse (2.4cm and 1.5cm);
      \draw[ultra thick, penColor] (1.3,0) ellipse (2.8cm and 1.8cm);
      \draw[ultra thick, penColor] (1.6,0) ellipse (3.2cm and 2.1cm);

      \node[penColor,fill=white] at (.6,-.6) {\small$7$};
      \node[penColor,fill=white] at (1.2,-.8) {\small$6$};
      \node[penColor,fill=white] at (1.8,-1) {\small$5$};
      \node[penColor,fill=white] at (2.4,-1.2) {\small$4$};
      \node[penColor,fill=white] at (3,-1.4) {\small$3$};
      \node[penColor,fill=white] at (3.6,-1.6) {\small$2$};

      \draw[fill=black,black] (3.7,0) circle (.1cm);
      \draw[fill=black,black] (1.3,1.65) circle (.1cm);
      \draw[fill=black,black] (-1.3,0) circle (.1cm);

      \node[black] at (-2,0) {$B$};
      \node[black] at (.8,2.5) {$A$};
      \node[black,above] at (3.7,0) {$C$};

      \draw[thick,->] (-1.8,0) -- (-1.5,0);
      \draw[thick,->] (.9,2.3) -- (1.2,1.7);
    \end{tikzpicture}
  \end{image}
  Consider the points $A$, $B$, and $C$ on the surface $z=F(x,y)$.
  Where $|\grad F|$ largest?
  \begin{prompt}
    The magnitude of the gradient vector of $F$ is largest at point
    $\answer[format=string]{B}$.
  \end{prompt}
  Where is $|\grad F|$ smallest?
  \begin{prompt}
    The magnitude of the gradient vector of $F$ is smallest at point
    $\answer[format=string]{C}$.
  \end{prompt}
\end{question}




Now, stand back. We're going to do some serious calculus. Just
read, relax and enjoy.

\begin{example}
  Consider the surface given by $F(x,y)= 20-x^2-2y^2$:
  \begin{image}
    \begin{tikzpicture}
      \begin{axis}%
        [tick label style={font=\scriptsize},axis on top,
	  axis lines=center,
	  view={155}{25},
	  name=myplot,
	  %xtick=\empty,
	  %ytick={5},
	  %ztick={.7,-.7},
	  %minor xtick=1,
	  %minor ytick=1,
	  ymin=-1,ymax=5.5,
	  xmin=-1,xmax=5.5,
	  zmin=-1.1, zmax=21,
	  every axis x label/.style={at={(axis cs:\pgfkeysvalueof{/pgfplots/xmax},0,0)},xshift=-5pt,yshift=-1pt},
	  xlabel={\scriptsize $x$},
	  every axis y label/.style={at={(axis cs:0,\pgfkeysvalueof{/pgfplots/ymax},0)},xshift=4pt,yshift=-4pt},
	  ylabel={\scriptsize $y$},
	  every axis z label/.style={at={(axis cs:0,0,\pgfkeysvalueof{/pgfplots/zmax})},xshift=0pt,yshift=4pt},
	  zlabel={\scriptsize $z$},colormap/cool
        ]
        
        %\addplot3[domain=0:180,smooth,y domain=0:360,surf,%fill=white,
        %colormap={mp2}{\colormapplaneone},faceted color=black!40,samples=30,samples y=25,very thin,z buffer=sort] ({cos(x)*1.5*cos(y)},{sin(x)*cos(y)},{sin(y)});
        
        \addplot3[domain=-1:4,y domain=-1:3,mesh,samples y=15,very thin,z buffer=sort,%opacity=.6,
          samples=15,] (x,y,{20-x^2-2*y^2});
        
        \addplot3 [thick, penColor, smooth,domain=1:4,samples=20,samples y=0] ({x},{x^2/4},{20-x^2-0.125*x^4});
        %%        
        \filldraw [black] (axis cs:1,.25,18.875) circle (1pt);
        %\filldraw [black] (axis cs:.5,.25,19.625) circle (1pt);
        %
        %\filldraw [black] (axis cs:.5,.5,19.25) circle (1pt);
        
        
        %\addplot3 [thick,{\colorone}, smooth,domain=-3:3,samples=20,samples y=0] ({x},{2},{x^2+8});
        %
        %\addplot3 [thick,{\colorone}, smooth,domain=-30:170,samples=60,samples y=0] ({2.93*(cos(x))},{1.96*(sin(x))},.2);
        %
        %\addplot3 [thick,{\colorone}, smooth,domain=-30:170,samples=60,samples y=0] ({2.75*(cos(x))},{1.83*(sin(x))},.4);
        %
        %\addplot3 [thick,{\colorone}, smooth,domain=-35:170,samples=60,samples y=0] ({2.4*(cos(x))},{1.6*(sin(x))},.6);
        %
        %\addplot3 [thick,{\colorone}, smooth,domain=-40:170,samples=60,samples y=0] ({1.8*(cos(x))},{1.2*(sin(x))},.8);
        %
        %\filldraw [{\colorone}] (axis cs: 0,0,1) circle (1pt);
      \end{axis}
    \end{tikzpicture}
  \end{image}
  
  Water is poured on the surface at $(1,1/4)$. What path does it take
  as it flows downhill?
  \begin{explanation}
    Let $\vec{w}(t) = \vector{x(t), y(t)}$ be the vector-valued
    function describing the path of the water in the $(x,y)$-plane. We
    seek $x(t)$ and $y(t)$. We know that water will always flow
    downhill in the initial steepest direction. Therefore, at any
    point on its path, it will be moving in the direction of
    \[
    -\grad F(x,y)
    \]
    We'll ignore the physical effects of momentum on the water.  Thus
    $\vec{w}(t)$ will be parallel to $\grad F$. Ah! This means there
    is some constant $c$ such that
    \[
    c\grad F(x(t),y(t)) = \vec{w}'(t) = \vector{x'(t), y'(t)}.
    \]
    Computing the gradient,
    \[
    \grad F(x(t),y(t)) = \vector{-2x(t), -4y(t)}
    \]
    Then
    \begin{align*}
      c\cdot \grad F(x(t),y(t)) &= \vector{ x'(t), y'(t)}\\
      c\cdot \vector{-2x(t),-4y(t)} &= \vector{ x'(t), y'(t)}\\
      \vector{-2cx(t),-4cy(t)} &= \vector{ x'(t), y'(t)}\\
          \end{align*}
    This implies
    \[
    -2cx(t) = x'(t) \quad \text{and} \quad  -4cy(t) =y'(t)
    \]
    so
    \[
    c = -\frac{x'(t)}{2x(t)} \quad \text{and} \quad  c =-\frac{y'(t)}{4y(t)}.
    \]
    Now recall that the differentials $\d x = x'(t) \d t$, and $\d
    y=y'(t)\d t$, so we may write
    \begin{align*}
      \int \frac{1}{2x}x'(t)\d t &=\int \frac{1}{4y} y'(t)\d t \\
      \int \frac{1}{2x}\d x &=\int\frac{1}{4y}\d y \\
      \frac{1}{2}\ln|x| +C &= \frac{1}{4}\ln|y|\\
      2\ln|x| + C &= \ln|y|\\
      \ln|x^2| + C &= \ln|y|
    \end{align*}
    Raising $e$ to the left-hand and right hand sides, we see
    \begin{align*}
    e^{\ln|x^2| + C} &= e^{\ln|y|}\\
    x^2\cdot e^C &= \ln|y|,
    \end{align*}
    setting $K = e^C$, we write
    \[
    K\cdot x^2 = y.
    \]
  We are so close to being done, $y=K\cdot x^2$, this is the path
  described in the $(x,y)$-plane. Since the water started at the point
  $(1,1/4)$, we can solve for $K$:
\[
K\cdot 1^2 = \frac14 \quad \Rightarrow \quad K = \frac14.
\]
Thus the water follows the curve
\[
y=x^2/4
\]
in the $(x,y)$-plane.
  \end{explanation}
\end{example}

\begin{question}
  What were you supposed to learn from that last example?
  \begin{prompt}
  \begin{multipleChoice}
    \choice[correct]{I've thought about this.}
    \choice{I've not thought about this.}
  \end{multipleChoice}
  \begin{feedback}[correct]
  There are two key take-aways from the example above:
  \begin{itemize}
    \item First, that the negative of the gradient points in the
      initial direction of greatest decrease.
    \item Second, is just to observe how the problem combines
      many aspects of calculus.
  \end{itemize}
  \end{feedback}
  \end{prompt}
\end{question}







\section{Orthogonality and the gradient}


Now that we know gradient vectors point in the initial direction of
the greatest increase of the function, let's think about the geometry
of the gradient vector. Previously we used the chain rule to show that
the gradient vector is always orthogonal to level sets. The argument
went like this: Suppose that a vector-valued function
$\vec{c}(t)=\vector{x(t),y(t)}$ runs along a level surface for the
surface $F(x,y)$.  If we ask ourselves: ``What is the change in $F$ as
$t$ varies?'' We must conclude that 
\[
\dd{t} F(\vec{c}(t)) = 0 
\]
since the value of $F$ doesn't change on the curve drawn by $\vec{c}$ (remember, $\vec{c}$ draws a level curve).
On the other hand, by the chain rule:
\[
\dd{t} F(\vec{c}(t)) = \grad F(\vec{c}(t)) \dotp \vec{c}'(t)
\]
The vector $\vec{c}'$ is tangent to the curve drawn by $\vec{c}$, and
putting the two equations above together we see
\[
0 = \grad F(\vec{c}(t)) \dotp \vec{c}'(t)
\]
so $\grad F(\vec{c}(t))$ must be orthogonal to $\vec{c}'$, and hence
orthogonal to the curve drawn by $\vec{c}$.


The explanation we just gave is a good one, but let's give one more.
In this book, we are always thinking about differentiable
functions. Remember, a function $F:\R^2\to\R$ is differentiable if one
can ``zoom-in'' and eventually the function will look like a plane. So
let's imagine that we've ``zoomed-in'' on a differentiable function
and it looks like a plane.  The contour plot of a plane looks like a
bunch of parallel lines:
\begin{image}
  \begin{tikzpicture}
    \begin{axis}%
      [tick label style={font=\scriptsize},axis on top,
	axis lines=center,
        width=3in,
        height=3in,
	xtick={-3,-2,...,3},
        ytick={-3,-2,...,3},
	%ytick={5},
	%ztick={.7,-.7},
	ymin=-1.2,ymax=1.2,
	xmin=-1.2,xmax=1.2,
        grid=major,%width=3in,height=3in,
        grid style={dashed, gridColor},
	every axis x label/.style={at={(axis cs:\pgfkeysvalueof{/pgfplots/xmax},0,0)},xshift=5pt,yshift=0pt},
	xlabel={\scriptsize $x$},
	every axis y label/.style={at={(axis cs:0,\pgfkeysvalueof{/pgfplots/ymax},0)},xshift=4pt,yshift=2pt},
	ylabel={\scriptsize $y$},
      ]
      \addplot[very thick, penColor,smooth] {-x};
      \addplot[very thick, penColor,smooth] {-x+1};
      \addplot[very thick, penColor,smooth] {-x+2};
      \addplot[very thick, penColor,smooth] {-x+3};
      \addplot[very thick, penColor,smooth] {-x-1};
      \addplot[very thick, penColor,smooth] {-x-2};
      \foreach \i in {-2.5,-2.4,...,2.5}
               {
                 \addplot[penColor!50!white,thin,smooth] {-x+\i};
               }
               \node[penColor,fill=white] at (axis cs:.5,-.5) {$0$};
               \node[penColor,fill=white] at (axis cs:0.8,.2) {$1$};
               \node[penColor,fill=white] at (axis cs:1,1) {$2$};
               \node[penColor,fill=white] at (axis cs:-.2,-.8) {$-1$};
               \node[penColor,fill=white] at (axis cs:-1,-1) {$-2$};

               \addplot[color=penColor2,fill=penColor2,only marks,mark=*] coordinates{(-.5,.5)};  %% closed hole          
    \end{axis}
  \end{tikzpicture}
    \end{image}
If we wish to leave the point above in the direction of the initial
greatest increase, then we should move in a direction perpendicular to
the level curves:
\begin{image}
  \begin{tikzpicture}
    \begin{axis}%
      [tick label style={font=\scriptsize},axis on top,
	axis lines=center,
	%view={30}{30},
	%name=myplot,
        width=3in,
        height=3in,
	xtick={-3,-2,...,3},
        ytick={-3,-2,...,3},
	    %ytick={5},
	%ztick={.7,-.7},
	ymin=-1.2,ymax=1.2,
	xmin=-1.2,xmax=1.2,
        grid=major,%width=3in,height=3in,
        grid style={dashed, gridColor},
	every axis x label/.style={at={(axis cs:\pgfkeysvalueof{/pgfplots/xmax},0,0)},xshift=5pt,yshift=0pt},
	xlabel={\scriptsize $x$},
	every axis y label/.style={at={(axis cs:0,\pgfkeysvalueof{/pgfplots/ymax},0)},xshift=4pt,yshift=2pt},
	ylabel={\scriptsize $y$},
      ]
      \addplot[very thick, penColor,smooth] {-x};
      \addplot[very thick, penColor,smooth] {-x+1};
      \addplot[very thick, penColor,smooth] {-x+2};
      \addplot[very thick, penColor,smooth] {-x+3};
      \addplot[very thick, penColor,smooth] {-x-1};
      \addplot[very thick, penColor,smooth] {-x-2};
      \foreach \i in {-2.5,-2.4,...,2.5}
               {
                 \addplot[penColor!50!white,thin,smooth] {-x+\i};
               }
               
               \node[penColor,fill=white] at (axis cs:.5,-.5) {$0$};
               \node[penColor,fill=white] at (axis cs:0.8,.2) {$1$};
               \node[penColor,fill=white] at (axis cs:1,1) {$2$};
               \node[penColor,fill=white] at (axis cs:-.2,-.8) {$-1$};
               \node[penColor,fill=white] at (axis cs:-1,-1) {$-2$};

               \addplot[penColor2,ultra thick, ->] coordinates{(-.5,.5) (0,1)};
               
          \addplot[color=penColor2,fill=penColor2,only marks,mark=*] coordinates{(-.5,.5)};  %% closed hole          
    \end{axis}
  \end{tikzpicture}
\end{image}
Gradient vectors point in the initial direction of greatest
increase and the fastest way to leave a line is perpendicular to that line.

The fact that the gradient is always orthogonal to level surfaces is
very powerful.  In fact it gives new (easier!) solutions to old
problems. Let's use this fact to find a plane tangent to a surface.

\begin{example}
  Find an implicit equation for the tangent plane to the elliptic
  paraboloid $z = x^2 + y^2$
  \begin{image}
    \begin{tikzpicture}
      \begin{axis}%
        [width=175pt,tick label style={font=\scriptsize},axis on top,
	  axis lines=center,
	  view={145}{20},
	  name=myplot,
	  xtick=\empty,
	  ytick=\empty,
	  ztick=\empty,
	  ymin=-2.5,ymax=2.5,
	  xmin=-3.5,xmax=3.5,
	  zmin=-.1, zmax=5.5,
	  every axis x label/.style={at={(axis cs:\pgfkeysvalueof{/pgfplots/xmax},0,0)},xshift=-3pt,yshift=-3pt},
	  xlabel={\scriptsize $x$},
	  every axis y label/.style={at={(axis cs:0,\pgfkeysvalueof{/pgfplots/ymax},0)},xshift=5pt,yshift=-2pt},
	  ylabel={\scriptsize $y$},
	  every axis z label/.style={at={(axis cs:0,0,\pgfkeysvalueof{/pgfplots/zmax})},xshift=0pt,yshift=4pt},
	  zlabel={\scriptsize $z$},
          colormap/cool
        ]
        
        \addplot3[domain=0:360,y domain=0:2,color=black!40,mesh,samples=40,samples y=10,very thin,z buffer=sort] ({2*cos(x)*y},{sin(x)*y},{y^2});
      \end{axis}
    \end{tikzpicture}
  \end{image}
  at $\vec{p} = \vector{2,3,13}$.
  \begin{explanation}
    Consider
        \[
        F(x,y,z) =x^2 +y^2 -z
        \]
        and imagine the elliptic paraboloid as the level surface
        \[
        F(x,y,z) = \answer[given]{0}
        \]
        Remember, \textbf{the gradient is perpendicular to level
          surfaces}.  We'll use this fact to find a normal vector to
        the surface, and with this vector we'll find the tangent
        plane.  The gradient is:
        \begin{align*}
          \grad F(x,y,z) &= \vector{\pp[F]{x}, \pp[F]{y},\pp[F]{z}}\\
          &= \vector{\answer[given]{2x},\answer[given]{2y}, \answer[given]{-1}}.
        \end{align*}
        Since this vector is normal to the surface, we can use it to
        find an implicit formula for the tangent plane to the surface
        by computing
        \[
        \vec{n}\dotp(\vec{x}-\vec{p}) = 0
        \]
        where $\vec{p} = \vector{2,3,13}$ and
        \begin{align*}
          \vec{n} &= \grad F(\vec{p})\\
          &=\vector{\answer[given]{4}, \answer[given]{6},\answer[given]{-1}}
        \end{align*}
        Thus the equation of the plane tangent to the ellipsoid at
        $\vec{p}$ is:
        \[
        \answer[given]{4}(x-2) + 6(y-3) - \left(z-\answer[given]{13}\right) = \answer[given]{0}
        \]
  \end{explanation}
\end{example}

Now let's see a more in-depth problem.

\begin{example}
  A plane perpendicular to the $(x,y)$-plane contains the point
  $(-8,3,-8)$ on the hyperbolic paraboloid $z = x^2-8y^2$. The line
  tangent to the intersection of the paraboloid and the plane is
  parallel to the $(x,y)$-plane at this point. Find an equation of the
  plane.
  \begin{explanation}
    Planes are determined by a point and their normal vector
    $\vec{n}$.  Since this plane is perpendicular to the
    $(x,y)$-plane, we know that:
    \[
    \vec{n} = \vector{a,b,\answer[given]{0}}
    \]
    Moreover the hyperbolic paraboloid
    \[
    z = x^2-8y^2
    \]
    can be thought of as a level surface of
    \[
    G(x,y,z) = x^2-8y^2 -z,
    \]
    in particular, $G(x,y,z) = 0$. Since gradient vectors are normal
    to level surfaces, we compute $\grad G$ to find:
    \[
    \grad G (x,y,z) = \vector{2x, -16y,-1}
    \]
    Now if we compute:
    \begin{align*}
      \vec{n} \cross \grad G(-8,3,-8) &= \vector{a,b,0}\cross\vector{-16,-48,-1}\\
      &=\vector{\answer[given]{-b},\answer[given]{a},\answer[given]{16b-48a}}
    \end{align*}
    But we know that the $z$-component of the vector above must be
    $\answer[given]{0}$. So, write with me:
    \begin{align*}
      16b-48a &= 0\\
      b &= \answer[given]{3a}
    \end{align*}
    So $\vec{n}$ is parallel to $\vector{a,3a,0}$. Hence one formula for the plane is
    \[
    (x+8)+3(y-3) = 0
    \]
  \end{explanation}
\end{example}

\section{Summary}

To conclude, we will repeat ourselves: \textbf{There are three things
  you must know about the gradient vector:}

\paragraph{First: You must know how to compute the gradient vector.}



\paragraph{Second: The gradient vector points in the initial direction of greatest increase for a function.}


\paragraph{Third: The gradient vector is orthogonal to level sets.}






\end{document}
