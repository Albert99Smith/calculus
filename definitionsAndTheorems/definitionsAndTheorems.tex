\documentclass{ximera}

\usepackage{multicol}

\newcommand{\RR}{\mathbb R}
\newcommand{\R}{\mathbb R}
\renewcommand{\d}{\,d}
\newcommand{\dd}[2][]{\frac{d #1}{d #2}}
\renewcommand{\l}{\ell}
\newcommand{\ddx}{\frac{d}{dx}}
\everymath{\displaystyle}
\newcommand{\dfn}{\textbf}
\newcommand{\eval}[1]{\bigg[ #1 \bigg]}

\DeclareMathOperator{\arccot}{arccot}
\DeclareMathOperator{\arcsec}{arcsec}
\DeclareMathOperator{\arccsc}{arccsc}
\DeclareMathOperator{\sech}{sech}
\DeclareMathOperator{\csch}{csch}
\DeclareMathOperator{\arcsinh}{arcsinh}
\DeclareMathOperator{\arcsech}{arcsech}
\DeclareMathOperator{\arccosh}{arccosh}


\title{Draft of Definitions and Theorems from MOOCulus}
\begin{document}
\begin{abstract}

This document contains a draft of every definition and theorem from
MOOCulus.

\end{abstract}

\maketitle






\section{Functions}




\begin{definition}
A \textbf{function} is a relation between sets, where for each input,
there is exactly one output.
\end{definition}



\begin{definition}
A function is \textbf{one-to-one} if for every value in the range,
there is exactly one value in the domain.
\end{definition}






\section{Limits}





\begin{definition}
The \textbf{limit} of $f(x)$ as $x$ goes to $a$ is $L$,
\[
\lim_{x\to a}f(x)=L,
\]
if for every $\epsilon>0$ there is a $\delta > 0$ so that whenever
\[
0 < |x-a| < \delta, \qquad\text{we have} \qquad |f(x)-L|<\epsilon.
\]
If no such value of $L$ can be
found, then we say that $\lim_{x\to a}f(x)$ \textbf{does not exist}.
\end{definition}



\begin{definition}
We say that the \textbf{limit} of $f(x)$ as $x$ goes to $a$ from the
\textbf{left} is $L$,
\[
\lim_{x\to a-}f(x)=L
\]
if for every $\epsilon>0$ there is a $\delta > 0$ so that whenever $x< a$ and
\[
a-\delta < x \qquad\text{we have}\qquad |f(x)-L|<\epsilon.
\]
We say that the \textbf{limit} of $f(x)$ as $x$ goes to $a$ from the \textbf{right} is $L$,
\[
\lim_{x\to a+}f(x)=L
\]
if for every $\epsilon>0$ there is a $\delta > 0$ so that whenever $x > a$ and
\[
x<a+\delta \qquad\text{we have}\qquad |f(x)-L|<\epsilon.
\]
\end{definition}




\begin{theorem}[Limit Product Law]
Suppose $\lim_{x\to a} f(x)=L$ and $\lim_{x\to a}g(x)=M$. Then
\[
\lim_{x\to a} f(x)g(x) = LM.
\]
\end{theorem}




\begin{theorem}[Limit Composition Law]
Suppose that $\lim_{x\to a}g(x)=M$ and $\lim_{x\to M}f(x) =
f(M)$. Then
\[
\lim_{x\to a} f(g(x)) = f(M).
\]
\end{theorem}




\begin{theorem}[Limit Root Law]
Suppose that $n$ is a positive integer. Then
$$\lim_{x\to a}\root n\of{x} = \root n\of{a},$$ provided that $a$ is
positive if $n$ is even.
\end{theorem}




\begin{theorem}[Limit Laws]
Suppose that $\lim_{x\to a}f(x)=L$, $\lim_{x\to a}g(x)=M$, $k$
is some constant, and $n$ is a positive integer.
\begin{itemize}
\item[\textbf{Constant Law}] $\lim_{x\to a} kf(x) = k\lim_{x\to a}f(x)=kL$.
\item[\textbf{Sum Law}] $\lim_{x\to a} (f(x)+g(x)) = \lim_{x\to a}f(x)+\lim_{x\to a}g(x)=L+M$.
\item[\textbf{Product Law}] $\lim_{x\to a} (f(x)g(x)) = \lim_{x\to a}f(x)\cdot\lim_{x\to a}g(x)=LM$.
\item[\textbf{Quotient Law}] $\lim_{x\to a} \frac{f(x)}{g(x)} =
\frac{\lim_{x\to a}f(x)}{\lim_{x\to a}g(x)}=\frac{L}{M}$, if $M\ne0$.
\item[\textbf{Power Law}] $\lim_{x\to a} f(x)^n = \left(\lim_{x\to a}f(x)\right)^n=L^n$.
\item[\textbf{Root Law}] $\lim_{x\to a} \sqrt[n]{f(x)} = \sqrt[n]{\lim_{x\to
a}f(x)}=\sqrt[n]{L}$ provided if $n$ is even, then $f(x)\ge 0$
near $a$.
\item[\textbf{Composition Law}] If $\lim_{x\to a}g(x)=M$ and
$\lim_{x\to M}f(x) = f(M)$, then $\lim_{x\to a} f(g(x)) = f(M)$.
\end{itemize}
\end{theorem}




\begin{theorem}[Squeeze Theorem]
Suppose that $g(x) \le f(x) \le h(x)$ for all $x$
close to $a$ but not necessarily equal to $a$. If
\[
\lim_{x\to a} g(x) = L = \lim_{x\to a} h(x),
\]
then $\lim_{x\to a} f(x) = L$.
\end{theorem}




\begin{theorem}[Squeeze Theorem]
Suppose that $g(x) \le f(x) \le h(x)$ for all $x$
close to $a$ but not necessarily equal to $a$. If
\[
\lim_{x\to a} g(x) = L = \lim_{x\to a} h(x),
\]
then $\lim_{x\to a} f(x) = L$.
\end{theorem}






\section{Infinite limits}






\begin{definition}
If $f(x)$ grows arbitrarily large as $x$ approaches $a$, we write
\[
\lim_{x\to a} f(x) = \infty
\]
and say that the limit of $f(x)$ \textbf{approaches infinity} as $x$
goes to $a$.
If $|f(x)|$ grows arbitrarily large as $x$ approaches $a$ and $f(x)$ is
negative, we write
\[
\lim_{x\to a} f(x) = -\infty
\]
and say that the limit of $f(x)$ \textbf{approaches negative infinity}
as $x$ goes to $a$.
\end{definition}




\begin{definition}
If
\[
\lim_{x\to a} f(x) = \pm\infty, \qquad \lim_{x\to a+} f(x) = \pm\infty, \qquad\text{or}\qquad \lim_{x\to a-} f(x) = \pm\infty,
\]
then the line $x=a$ is a \textbf{vertical asymptote} of $f(x)$.
\end{definition}






\section{Limits at infinity}







\begin{definition}
If $f(x)$ becomes arbitrarily close to a specific value $L$ by making
$x$ sufficiently large, we write
\[
\lim_{x\to \infty} f(x) = L
\]
and we say, the \textbf{limit at infinity} of $f(x)$ is $L$.
If $f(x)$ becomes
arbitrarily close to a specific value $L$ by making $x$ sufficiently
large and negative, we write
\[
\lim_{x\to -\infty} f(x) = L
\]
and we say, the \textbf{limit at negative infinity} of $f(x)$ is $L$.
\end{definition}




\begin{definition}
If
\[
\lim_{x\to \infty} f(x) = L \qquad\text{or}\qquad \lim_{x\to -\infty} f(x) = L,
\]
then the line $y=L$ is a \textbf{horizontal asymptote} of $f(x)$.
\end{definition}







\section{Continuity}






\begin{definition}
A function $f$ is \textbf{continuous at a point} $a$ if $\lim_{x\to a}
f(x) = f(a)$.
\end{definition}




\begin{definition}
A function $f$ is \textbf{continuous on an interval} if it is
continuous at every point in the interval.
\end{definition}




\begin{theorem}[Intermediate Value Theorem]
If $f(x)$ is a continuous function for all $x$ in the closed interval
$[a,b]$ and $d$ is between $f(a)$ and $f(b)$, then there is a number
$c$ in $[a, b]$ such that $f(c) = d$.
\end{theorem}







\section{Slopes of tangent lines via limits}







\begin{definition}
The \textbf{derivative} of $f(x)$ is the function
\[
\ddx f(x) = \lim_{h\to 0} \frac{f(x+h) - f(x)}{h}.
\]
If this limit does not exist for a given value of $x$, then $f(x)$ is
not \textbf{differentiable} at $x$.
\end{definition}




\begin{definition}
There are several different notations for the derivative, we'll mainly
use
\[
\ddx f(x) = f'(x).
\]
If one is working with a function of a variable other than $x$, say $t$ we write
\[
\dd{t} f(t) = f'(t).
\]
However, if $y = f(x)$, $\dd[y]{x}$, $\dot{y}$, and $D_x f(x)$ are
also used.
\end{definition}




\begin{theorem}[Differentiability Implies Continuity]
If $f(x)$ is a differentiable function at $x = a$, then $f(x)$ is
continuous at $x=a$.
\end{theorem}







\section{Basic derivative rules}







\begin{theorem}[The Constant Rule]
Given a constant $c$,
\[
\ddx c = 0.
\]
\end{theorem}




\begin{theorem}[The Power Rule]
For any real number $n$,
\[
\ddx x^n = n x^{n-1}.
\]
\end{theorem}




\begin{theorem}[The Sum Rule]
If $f(x)$ and $g(x)$ are differentiable and $c$ is a constant, then
\begin{enumerate}
\item\label{SR:1} $\ddx \big( f(x) + g(x)\big) = f'(x) + g'(x)$,
\item $\ddx \big( f(x) - g(x)\big) = f'(x) - g'(x)$,
\item $\ddx \big(c\cdot f(x)\big) = c\cdot f'(x)$.
\end{enumerate}
\end{theorem}




\begin{definition}
Euler's number is defined to be the number $e$ such that
\[
\lim_{h\to 0} \frac{e^h-1}{h} = 1.
\]
\end{definition}




\begin{theorem}[The Derivative of $\textit{e}^\textit{x}$]
\[
\ddx e^x = e^x.
\]
\end{theorem}







\section{Curve Sketching}





\begin{definition}\hfil
\begin{enumerate}
\item A point $(x,f(x))$ is a \textbf{local maximum} if there is an interval $a<x<b$ with $f(x)\ge f(z)$ for
every $z$ in $(a,b)$.
\item A point $(x,f(x))$ is a \textbf{local minimum} if
there is an interval $a<x<b$ with $f(x)\le f(z)$ for every $z$ in
$(a,b)$.
\end{enumerate}
A \textbf{local extremum}\index{extremum!local} is either a local
maximum or a local minimum.
\end{definition}



\begin{theorem}[Fermat's Theorem]
If $f(x)$ has a local extremum at $x=a$ and $f(x)$ is differentiable
at $a$, then $f'(a)=0$.
\end{theorem}



\begin{definition}
Any value of $x$ for which $f'(x)$ is zero or undefined is called a
\textbf{critical point} for $f(x)$.
% critical 'number' and also most books insist that the crit num be in the domain of f.  Is that necessary? This def is nicer
\end{definition}



\begin{theorem}[First Derivative Test]\index{first derivative test}\label{T:fdt}\hfil
Suppose that $f(x)$ is continuous on an interval, and that $f'(a)=0$
for some value of $a$ in that interval.
\begin{itemize}
\item If $f'(x)>0$ to the left of $a$ and $f'(x)<0$ to the right of
$a$, then $f(a)$ is a local maximum.
\item If $f'(x)<0$ to the left of $a$ and $f'(x)>0$ to the right of
$a$, then $f(a)$ is a local minimum.
\item If $f'(x)$ has the same sign to the left and right of $a$,
then $f(a)$ is not a local extremum.
\end{itemize}
\end{theorem}



\begin{theorem}[Test for Concavity]\index{concavity test}
Suppose that $f''(x)$ exists on an interval.
\begin{enumerate}
\item If $f''(x)>0$ on an interval, then $f(x)$ is concave up on that interval.
\item If $f''(x)<0$ on an interval, then $f(x)$ is concave down on that interval.
\end{enumerate}
\end{theorem}



\begin{theorem}[Second Derivative Test]\index{second derivative test}\label{T:sdt}
Suppose that $f''(x)$ is continuous on an open interval and that
$f'(a)=0$ for some value of $a$ in that interval.
\begin{itemize}
\item If $f''(a) <0$, then $f(x)$ has a local maximum at $a$.
\item If $f''(a) >0$, then $f(x)$ has a local minimum at $a$.
\item If $f''(a) =0$, then the test is inconclusive. In this case,
$f(x)$ may or may not have a local extremum at $x=a$.
\end{itemize}
\end{theorem}






\section{Product Rule}





\begin{theorem}[The Product Rule]\index{derivative rules!product}\index{product rule}\label{theorem:product-rule}
If $f(x)$ and $g(x)$ are differentiable, then
\[
\ddx f(x)g(x) = f(x)g'(x)+f'(x)g(x).
\]
\end{theorem}







\section{Quotient Rule}





\begin{theorem}[The Quotient Rule]\index{derivative rules!quotient}\index{quotient rule}\label{theorem:quotient-rule}
If $f(x)$ and $g(x)$ are differentiable, then
\[
\ddx \frac{f(x)}{g(x)} = \frac{f'(x)g(x)-f(x)g'(x)}{g(x)^2}.
\]
\end{theorem}








\section{The Chain Rule}






\begin{theorem}[Chain Rule]\index{chain rule}\index{derivative rules!chain}
If $f(x)$ and $g(x)$ are differentiable, then
\[
\ddx f(g(x)) = f'(g(x))g'(x).
\]
\end{theorem}







\section{Implicit Differentiation}






\begin{theorem}[The Derivative of the Natural Logrithm]\index{derivative!of the natural logarithm}
\[
\ddx \ln(x) = \frac{1}{x}.
\]
\end{theorem}



\begin{theorem}[Inverse Function Theorem]\index{Inverse Function Theorem}\label{theorem:IFT}
If $f(x)$ is a differentiable function, and $f'(x)$ is continuous, and
$f'(a) \neq 0$, then
\begin{enumerate}
\item $f^{-1}(y)$ is defined for $y$ near $f(a)$,
\item $f^{-1}(y)$ is differentiable near $f(a)$,
\item $\dd{y} f^{-1}(y)$ is continuous near $f(a)$, and
\item $\dd{y} f^{-1}(y) = \displaystyle\frac{1}{f'(f^{-1}(y))}$.
\end{enumerate}
\end{theorem}






\section{Logarithmic Differentiation}





\begin{theorem}[Power Rule]
For any real number $n$,
\[
\ddx x^n = n x^{n-1}.
\]
\end{theorem}








\section{The Derivatives of Trigonometric Functions}





\begin{theorem}[The Derivative of sin(\textit{x})]\index{derivative!of sine}\label{theorem:deriv sin}
\[
\ddx \sin(x) = \cos(x).
\]
\end{theorem}



\begin{theorem}[The Derivative of cos(\textit{x})]\index{derivative!of cosine}
\[
\ddx \cos(x) = -\sin(x).
\]
\end{theorem}



\begin{theorem}[The Derivative of tan(\textit{x})]\index{derivative!of tangent}
\[
\ddx \tan(x) = \sec^2(x).
\]
\end{theorem}



\begin{theorem}[The Derivative of sec(\textit{x})]\index{derivative!of secant}
\[
\ddx \sec(x) = \sec(x)\tan(x).
\]
\end{theorem}






\section{Inverse Trigonometric Functions}







\section{The Derivatives of Inverse Trigonometric Functions}






\begin{theorem}[The Derivative of arcsin(\textit{y})]\index{derivative!of arcsine}
\[
\dd{y} \arcsin(y) = \frac{1}{\sqrt{1-y^2}}.
\]
\end{theorem}



\begin{theorem}[The Derivative of arccos(\textit{y})]\index{derivative!of arccosine}
\[
\dd{y} \arccos(y) = \frac{-1}{\sqrt{1-y^2}}.
\]
\end{theorem}



\begin{theorem}[The Derivative of arctan(\textit{y})]\index{derivative!of arctangent}
\[
\dd{y} \arctan(y) = \frac{1}{1+y^2}.
\]
\end{theorem}



\begin{theorem}[The Derivatives of Inverse Trigonometric Functions] \hfil
\begin{itemize}
\item $\dd{y} \arcsin(y) = \frac{1}{\sqrt{1-y^2}}$.
\item $\dd{y} \arccos(y) = \frac{-1}{\sqrt{1-y^2}}$.
\item $\dd{y} \arctan(y) = \frac{1}{1+y^2}$.
\item $\dd{y} \arcsec(y) = \frac{1}{|y|\sqrt{y^2-1}}$ for $|y|>1$.
\item $\dd{y} \arccsc(y) = \frac{-1}{|y|\sqrt{y^2-1}}$ for $|y|>1$.
\item $\dd{y} \arccot(y) = \frac{-1}{1+y^2}$.
\end{itemize}
\end{theorem}







\section{L'Hopital's Rule}





\begin{theorem}[L'H\^opital's Rule]\index{l'H\^opital's Rule}
Let $f(x)$ and $g(x)$ be functions that are differentiable near $a$. If
\[
\lim_{x \to a} f(x) = \lim_{x \to a}g(x) = 0 \qquad \text{or} \pm \infty,
\]
and $\lim_{x \to a} \frac{f'(x)}{g'(x)}$ exists, and $g'(x) \neq 0$
for all $x$ near $a$, then
\[
\lim_{x \to a} \frac{f(x)}{g(x)} = \lim_{x \to a} \frac{f'(x)}{g'(x)}.
\]
\end{theorem}




\begin{definition}[List of Indeterminate Forms]\index{indeterminate form}\hfil
\begin{itemize}
\item[\textbf{0/0}] This refers to a limit of the form $\lim_{x\to a}
\frac{f(x)}{g(x)}$ where $f(x)\to 0$ and $g(x)\to 0$ as $x\to a$.
\item[\textbf{$\pmb\infty$/$\pmb\infty$}] This refers to a limit of the form $\lim_{x\to a}
\frac{f(x)}{g(x)}$ where $f(x)\to \infty$ and $g(x)\to \infty$ as $x\to a$.
\item[\textbf{0\,$\pmb{\cdot\infty}$}] This refers to a limit of the form $\lim_{x\to a}
\left(f(x)\cdot g(x)\right)$ where $f(x)\to 0$ and $g(x)\to \infty$ as $x\to a$.
\item[\textbf{$\pmb\infty$--$\pmb\infty$}] This refers to a limit of the form $\lim_{x\to a}\left(
f(x)-g(x)\right)$ where $f(x)\to \infty$ and $g(x)\to \infty$ as $x\to a$.
\item[\textbf{1$^{\pmb\infty}$}] This refers to a limit of the form $\lim_{x\to a}
f(x)^{g(x)}$ where $f(x)\to 1$ and $g(x)\to \infty$ as $x\to a$.
\item[\textbf{0$^\text{0}$}] This refers to a limit of the form $\lim_{x\to a}
f(x)^{g(x)}$ where $f(x)\to 0$ and $g(x)\to 0$ as $x\to a$.
\item[\textbf{$\pmb\infty^\text{0}$}] This refers to a limit of the form $\lim_{x\to a}
f(x)^{g(x)}$ where $f(x)\to \infty$ and $g(x)\to 0$ as $x\to a$.
\end{itemize}
In each of these cases, the value of the limit is \textbf{not} immediately
obvious. Hence, a careful analysis is required!
\end{definition}







\section{The Derivative as a Rate}







\begin{definition}\index{average rate of change}
Given a function $f(x)$, the \textbf{average rate of change} over the
interval $[a, a+\Delta x]$ is given by
\[
\frac{f(a+\Delta x) - f(a)}{\Delta x}.
\]
\end{definition}



\begin{definition}\index{instantaneous rate of change}
Given a function, the \textbf{instantaneous rate of change} at $x=a$ is given by
\[
\left.\ddx f(x) \right|_{x=a}.
\]
\end{definition}





\section{Related Rates Problems}








\section{Maximum and Minimum Values of Curves}







\begin{definition}\hfil\index{maximum/minimum!absolute}
\begin{enumerate}
\item A point $(x,f(x))$ is an \textbf{absolute maximum} on an interval
if $f(x)\ge f(z)$ for every $z$ in that interval.
\item A point $(x,f(x))$ is an \textbf{absolute minimum} on an interval if
$f(x)\le f(z)$ for every $z$ in that interval.
\end{enumerate}
An \textbf{absolute extremum}\index{extremum!absolute} is either an
absolute maximum or an absolute minimum.
\end{definition}



\begin{theorem}[Extreme Value Theorem]\label{theorem:evt}\index{Extreme Value Theorem}
If $f(x)$ is a continuous function for all $x$ in the closed interval
$[a,b]$, then there are points $c$ and $d$ in $[a,b]$, such that
$(c,f(c))$ is an absolute maximum and $(d,f(d))$ is an absolute
minimum on $[a, b]$.
\end{theorem}






\section{Basic Optimization Problems}







\section{Linear Approximation and Differentials}






\begin{definition}\index{linear approximation}
If $f(x)$ is a differentiable function at $x=a$, then a \textbf{linear
approximation} for $f(x)$ at $x=a$ is given by
\[
\l(x) = f'(a)(x-a) +f(a).
\]
\end{definition}



\begin{definition}\index{differential}  %in subsection: Differentials
Let $f(x)$ be a differentiable function. We define a new
independent variable $dx$, and a new dependent variable
\[
dy=f'(x)\cdot dx.
\]
The variables $dx$ and $dy$ are called \textbf{differentials}, see
Figure~\ref{figure:differentials}.
\end{definition}







\section{Iterative Methods}








\section{The Mean Value Theorem}







\begin{theorem}[Rolle's Theorem]\index{Rolle's Theorem}
Suppose that $f(x)$ is differentiable on the interval $(a,b)$, is
continuous on the interval $[a,b]$, and $f(a)=f(b)$. Then
\[
f'(c)=0
\]
for some $a<c<b$.
\label{thm:rolle}
\end{theorem}




\begin{theorem}[Mean Value Theorem]\label{thm:mvt}\index{Mean Value Theorem}
Suppose that $f(x)$ has a derivative on the interval $(a,b)$ and is
continuous on the interval $[a,b]$. Then
\[
f'(c)=\frac{f(b)-f(a)}{b-a}
\]
for some $a<c<b$.
\end{theorem}



\begin{theorem}
If $f'(x)=0$ for all $x$ in an interval $I$, then $f(x)$ is constant
on $I$.
\end{theorem}






\section{Basic Antiderivatives}







\begin{definition}\index{antiderivative}
A function $F(x)$ is called an \textbf{antiderivative} of $f(x)$ on an
interval if
\[
F'(x) = f(x)
\]
for all $x$ in the interval.
\end{definition}




\begin{definition}\index{antiderivative!notation}\index{indefinite integral}
The antiderivative is denoted by
\[
\int f(x) \d x = F(x)+C,
\]
where $dx$ identifies $x$ as the variable and $C$ is a constant
indicating that there are many possible antiderivatives, each varying by
the addition of a constant. This is often called the
\textbf{indefinite integral}.
\end{definition}



\begin{theorem}[Basic Antiderivatives]\label{theorem:basicAnti} \hfil
\begin{multicols}{3}
\begin{itemize}
\item $\int k \d x= kx+C$.
\item $\int x^n \d x= \frac{x^{n+1}}{n+1}+C\qquad(n\ne-1)$.
\item $\int e^x \d x= e^x + C$.
\item $\int a^x \d x= \frac{a^x}{\ln(a)}+C$.
\item $\int \frac{1}{x} \d x= \ln|x|+C$.
\item $\int \cos(x) \d x = \sin(x) + C$.
\item $\int \sin(x) \d x = -\cos(x) + C$.
\item $\int \tan(x) \d x = -\ln|\cos(x)| + C$.
\item $\int \sec^2(x) \d x = \tan(x) + C$.
\item $\int \csc^2(x) \d x = -\cot(x) + C$.
\item $\int \sec(x)\tan(x) \d x = \sec(x) + C$.
\item $\int \csc(x)\cot(x) \d x = -\csc(x) + C$.
\item $\int \frac{1}{x^2+1}\d x = \arctan(x) + C$.
\item $\int \frac{1}{\sqrt{1-x^2}}\d x= \arcsin(x)+C$.
\end{itemize}
\end{multicols}
\end{theorem}




\begin{theorem}[The Sum Rule for Antiderivatives]\label{theorem:SRA}
Given two functions $f(x)$ and $g(x)$ where $k$ is a constant:
\begin{itemize}
\item $\int k f(x) \d x= kF(x) + C$.
\item $\int \left(f(x) + g(x)\right) \d x = F(x) + G(x) + C$.
\end{itemize}
\end{theorem}






\section{Differential Equations}







\section{Definite Integrals Compute Signed Area}





\begin{definition}\index{integral}\index{definite integral}
The \textbf{definite integral}
\[
\int_a^b f(x) \d x
\]
computes the signed area in the region $[a,b]$ between $f(x)$ and the
$x$-axis. If the region is above the $x$-axis, then the area has
positive sign. If the region is below the $x$-axis, then the area has
negative sign.
\end{definition}


\begin{theorem}[Properties of Definite Integrals]
\begin{enumerate}
\item $\int_a^b k \d x= kb-ka$, where $k$ is a constant.
\item $\int_a^b \left( f(x) + g(x) \right) \d x = \int_a^b f(x) \d x + \int_a^b
g(x) \d x$.
\item $\int_a^b k \cdot f(x) \d x = k \int_a^b f(x) \d x$.
\end{enumerate}
\end{theorem}



\begin{definition} %in the subsection: Accumulation Functions
Given a function $f(x)$, an \textbf{accumulation function} for
$f(x)$ is given by
\[
F(x) = \int_a^x f(t) \d t.
\]
\end{definition}






\section{Riemann Sums}





\begin{definition}\index{Riemann sum}
Given an interval $[a,b]$ and a partition defined by
\[
a = x_0 < x_1 <x_2 < \cdots x_{n-1}< x_n = b,
\]
a \textbf{Riemann sum} for $f(x)$ is a sum of the form
\[
\sum_{i=0}^{n-1} f(x_i^*) \cdot (x_{i+1}-x_i)
\]
where $x_i^*\in [x_i,x_{i+1}]$.
\end{definition}




\begin{definition}
Consider the following Riemann sum:
\[
\sum_{i=0}^{n-1} f(x_i^*) \cdot (x_{i+1}-x_i)
\]
\begin{itemize}
\item This is called a \textbf{left} Riemann sum if each $x_i^* =
x_i$.
\item This is called a \textbf{right} Riemann sum if each $x_i^* =
x_{i+1}$.
\item This is called a \textbf{midpoint} Riemann sum if each $x_i^*
= \frac{x_i+x_{i+1}}{2}$.
\item This is called a \textbf{upper} Riemann sum if each $x_i^*$ is
a point that gives a maximum value $f(x)$ on the interval
$[x_i,x_{i+1}]$.
\item This is called a \textbf{lower} Riemann sum if each $x_i^*$ is a
point that gives a minimum value $f(x)$ on the interval
$[x_i,x_{i+1}]$.
\end{itemize}
\end{definition}







\section{The Fundamental Theorem}







\begin{theorem}[Fundamental Theorem of Calculus---Version I]
\index{fundamental theorem of calculus---version 1}
\label{thm:fundamental_theorem_I}\hfil
\noindent Suppose that $f(x)$ is continuous on the real numbers and let
\[
F(x)=\int_a^x f(t)\d t.
\]
Then $F'(x)=f(x)$.
\end{theorem}



\begin{theorem}[Fundamental Theorem of Calculus---Version II]\index{fundamental theorem of calculus---version 2}
\label{thm:fundamental_theorem_II}\hfil
\noindent Suppose that $f(x)$ is continuous on the interval $[a,b]$. If $F(x)$
is any antiderivative of $f(x)$, then
\[
\left.\int_a^b f(x)\d x = F(x) \right|_a^b = F(b)-F(a).
\]
\end{theorem}






\section{Integration by Substitution}







\begin{theorem}[Integral Substitution Formula]
If $u(x)$ is differentiable on the interval $[a,b]$ and $f(x)$ is
differentiable on the interval $[u(a),u(b)]$, then
\[
\int_a^b f'(u(x)) u'(x) \d x =\int_{u(a)}^{u(b)} f'(u) \d u.
\]
\end{theorem}







\section{Powers of Sine and Cosine}








\section{Integration by Parts}







\begin{theorem}[Integration by Parts Formula]
If $f(x)g(x)$ is differentiable on the interval $[a,b]$, then
\[
\int_a^b f(x) g'(x) \d x =f(x)g(x) \bigg|_a^b - \int_a^b f'(x) g(x) \d x.
\]
\end{theorem}





\section{Volume}







\section{Arc Length}















\end{document}

   

