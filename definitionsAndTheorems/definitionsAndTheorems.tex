\documentclass{ximera}

\newcommand{\RR}{\mathbb R}
\newcommand{\R}{\mathbb R}
\renewcommand{\d}{\,d}
\newcommand{\dd}[2][]{\frac{d #1}{d #2}}
\renewcommand{\l}{\ell}
\newcommand{\ddx}{\frac{d}{dx}}
\everymath{\displaystyle}
\newcommand{\dfn}{\textbf}
\newcommand{\eval}[1]{\bigg[ #1 \bigg]}
\newcommand{\fxn}{}


\title{Draft of Definitions and Theorems from MOOCulus}

\begin{document}
\begin{abstract}
This document contains a draft of every definition and theorem from
MOOCulus.
\end{abstract}
\maketitle


\section*{Functions}

\begin{definition}
A \textbf{function} is a relation between sets, where for each input,
there is exactly one output.
\end{definition}

\begin{definition}
A function is \textbf{one-to-one} if for every value in the range,
there is exactly one value in the domain.
\end{definition}




\section*{Limits}

\begin{definition}
The \textbf{limit} of $f(x)$ as $x$ goes to $a$ is $L$,
\[
\lim_{x\to a}f(x)=L,
\] 
if for every $\epsilon>0$ there is a $\delta > 0$ so that whenever
\[
0 < |x-a| < \delta, \qquad\text{we have} \qquad |f(x)-L|<\epsilon.
\] 
If no such value of $L$ can be
found, then we say that $\lim_{x\to a}f(x)$ \textbf{does not exist}.
\end{definition}



\begin{definition} 
We say that the \textbf{limit} of $f(x)$ as $x$ goes to $a$ from the
\textbf{left} is $L$,
\[
\lim_{x\to a-}f(x)=L
\]
if for every $\epsilon>0$ there is a $\delta > 0$ so that whenever $x< a$ and 
\[
a-\delta < x \qquad\text{we have}\qquad |f(x)-L|<\epsilon.
\]

We say that the \textbf{limit} of $f(x)$ as $x$ goes to $a$ from the \textbf{right} is $L$,
\[
\lim_{x\to a+}f(x)=L
\] 
if for every $\epsilon>0$ there is a $\delta > 0$ so that whenever $x > a$ and 
\[
x<a+\delta \qquad\text{we have}\qquad |f(x)-L|<\epsilon.
\]
\end{definition}





\begin{theorem}[Limit Product Law]
Suppose $\lim_{x\to a} f(x)=L$ and $\lim_{x\to a}g(x)=M$. Then
\[
\lim_{x\to a} f(x)g(x) = LM.
\]
\end{theorem}

\begin{theorem}[Limit Composition Law]
Suppose that $\lim_{x\to a}g(x)=M$ and $\lim_{x\to M}f(x) =
f(M)$. Then
\[
\lim_{x\to a} f(g(x)) = f(M).
\]
\end{theorem}


\begin{theorem}[Limit Root Law]
Suppose that $n$ is a positive integer. Then
$$\lim_{x\to a}\root n\of{x} = \root n\of{a},$$ provided that $a$ is
positive if $n$ is even.
\end{theorem}


\begin{theorem}[Limit Laws]
Suppose that $\lim_{x\to a}f(x)=L$, $\lim_{x\to a}g(x)=M$, $k$
is some constant, and $n$ is a positive integer.
\begin{itemize}
\item[\textbf{Constant Law}] $\lim_{x\to a} kf(x) = k\lim_{x\to a}f(x)=kL$.
\item[\textbf{Sum Law}] $\lim_{x\to a} (f(x)+g(x)) = \lim_{x\to a}f(x)+\lim_{x\to a}g(x)=L+M$.  
\item[\textbf{Product Law}] $\lim_{x\to a} (f(x)g(x)) = \lim_{x\to a}f(x)\cdot\lim_{x\to a}g(x)=LM$. 
\item[\textbf{Quotient Law}] $\lim_{x\to a} \frac{f(x)}{g(x)} =
  \frac{\lim_{x\to a}f(x)}{\lim_{x\to a}g(x)}=\frac{L}{M}$, if $M\ne0$.
\item[\textbf{Power Law}] $\lim_{x\to a} f(x)^n = \left(\lim_{x\to a}f(x)\right)^n=L^n$.
\item[\textbf{Root Law}] $\lim_{x\to a} \sqrt[n]{f(x)} = \sqrt[n]{\lim_{x\to
    a}f(x)}=\sqrt[n]{L}$ provided if $n$ is even, then $f(x)\ge 0$
  near $a$.
\item[\textbf{Composition Law}] If $\lim_{x\to a}g(x)=M$ and
  $\lim_{x\to M}f(x) = f(M)$, then $\lim_{x\to a} f(g(x)) = f(M)$.
\end{itemize}
\end{theorem}

\begin{theorem}[Squeeze Theorem]
Suppose that $g(x) \le f(x) \le h(x)$ for all $x$
close to $a$ but not necessarily equal to $a$. If 
\[
\lim_{x\to a} g(x) = L = \lim_{x\to a} h(x),
\] 
then $\lim_{x\to a} f(x) = L$.
\end{theorem}



\begin{theorem}[Squeeze Theorem]
Suppose that $g(x) \le f(x) \le h(x)$ for all $x$
close to $a$ but not necessarily equal to $a$. If 
\[
\lim_{x\to a} g(x) = L = \lim_{x\to a} h(x),
\] 
then $\lim_{x\to a} f(x) = L$.
\end{theorem}


\section*{Infinite limits}

\begin{definition}
If $f(x)$ grows arbitrarily large as $x$ approaches $a$, we write
\[
\lim_{x\to a} f(x) = \infty
\]
and say that the limit of $f(x)$ \textbf{approaches infinity} as $x$
goes to $a$.

If $|f(x)|$ grows arbitrarily large as $x$ approaches $a$ and $f(x)$ is
negative, we write
\[
\lim_{x\to a} f(x) = -\infty
\]
and say that the limit of $f(x)$ \textbf{approaches negative infinity}
as $x$ goes to $a$.
\end{definition}



\begin{definition}
If 
\[
\lim_{x\to a} f(x) = \pm\infty, \qquad \lim_{x\to a+} f(x) = \pm\infty, \qquad\text{or}\qquad \lim_{x\to a-} f(x) = \pm\infty,
\]
then the line $x=a$ is a \textbf{vertical asymptote} of $f(x)$.
\end{definition}


\section*{Limits at infinity}



\begin{definition}
If $f(x)$ becomes arbitrarily close to a specific value $L$ by making
$x$ sufficiently large, we write
\[
\lim_{x\to \infty} f(x) = L
\]
and we say, the \textbf{limit at infinity} of $f(x)$ is $L$.  

If $f(x)$ becomes
arbitrarily close to a specific value $L$ by making $x$ sufficiently
large and negative, we write
\[
\lim_{x\to -\infty} f(x) = L
\]
and we say, the \textbf{limit at negative infinity} of $f(x)$ is $L$.  
\end{definition}



\begin{definition}
If  
\[
\lim_{x\to \infty} f(x) = L \qquad\text{or}\qquad \lim_{x\to -\infty} f(x) = L,
\]
then the line $y=L$ is a \textbf{horizontal asymptote} of $f(x)$.
\end{definition}





\section*{Continuity}


\begin{definition} 
A function $f$ is \textbf{continuous at a point} $a$ if $\lim_{x\to a}
f(x) = f(a)$.
\end{definition}



\begin{definition} 
A function $f$ is \textbf{continuous on an interval} if it is
continuous at every point in the interval.
\end{definition}

\begin{theorem}[Intermediate Value Theorem]
If $f(x)$ is a continuous function for all $x$ in the closed interval
$[a,b]$ and $d$ is between $f(a)$ and $f(b)$, then there is a number
$c$ in $[a, b]$ such that $f(c) = d$.
\end{theorem}



\section*{Slopes of tangent lines via limits}

\begin{definition}
The \textbf{derivative} of $f(x)$ is the function
\[
\ddx f(x) = \lim_{h\to 0} \frac{f(x+h) - f(x)}{h}.
\]
If this limit does not exist for a given value of $x$, then $f(x)$ is
not \textbf{differentiable} at $x$.
\end{definition}


\begin{definition}
There are several different notations for the derivative, we'll mainly
use
\[
\ddx f(x) = f'(x).
\]
If one is working with a function of a variable other than $x$, say $t$ we write
\[
\dd{t} f(t) = f'(t).
\]
However, if $y = f(x)$, $\dd[y]{x}$, $\dot{y}$, and $D_x f(x)$ are
also used.
\end{definition}


\begin{theorem}[Differentiability Implies Continuity]
If $f(x)$ is a differentiable function at $x = a$, then $f(x)$ is
continuous at $x=a$.
\end{theorem}



\section*{Basic derivative rules}


\begin{theorem}[The Constant Rule]
Given a constant $c$,
\[
\ddx c = 0.
\]
\end{theorem}


\begin{theorem}[The Power Rule]
For any real number $n$,
\[
\ddx x^n = n x^{n-1}.
\]
\end{theorem}



\begin{theorem}[The Sum Rule]
If $f(x)$ and $g(x)$ are differentiable and $c$ is a constant, then 
\begin{enumerate}
\item\label{SR:1} $\ddx \big( f(x) + g(x)\big) = f'(x) + g'(x)$,
\item $\ddx \big( f(x) - g(x)\big) = f'(x) - g'(x)$,
\item $\ddx \big(c\cdot f(x)\big) = c\cdot f'(x)$.
\end{enumerate}
\end{theorem}


\begin{definition}
Euler's number is defined to be the number $e$ such that
\[
\lim_{h\to 0} \frac{e^h-1}{h} = 1.
\]
\end{definition}

\begin{theorem}[The Derivative of $\textit{e}^\textit{x}$]
\[
\ddx e^x = e^x.
\]
\end{theorem}

\section*{Curve Sketching}



\begin{definition}\hfil
\begin{enumerate}
\item A point $(x,f(x))$ is a \textbf{local maximum} if there is an interval $a<x<b$ with $f(x)\ge f(z)$ for
  every $z$ in $(a,b)$.
\item A point $(x,f(x))$ is a \textbf{local minimum} if
  there is an interval $a<x<b$ with $f(x)\le f(z)$ for every $z$ in
  $(a,b)$.
\end{enumerate}
A \textbf{local extremum}\index{extremum!local} is either a local
maximum or a local minimum.
\end{definition}


\begin{theorem}[Fermat's Theorem]
If $f(x)$ has a local extremum at $x=a$ and $f(x)$ is differentiable
at $a$, then $f'(a)=0$.
\end{theorem}


\begin{definition}
Any value of $x$ for which $f'(x)$ is zero or undefined is called a
\textbf{critical point} for $f(x)$.
\end{definition}



\begin{theorem}[First Derivative Test]\index{first derivative test}\label{T:fdt}\hfil
Suppose that $f(x)$ is continuous on an interval, and that $f'(a)=0$
for some value of $a$ in that interval.
\begin{itemize}
\item If $f'(x)>0$ to the left of $a$ and $f'(x)<0$ to the right of
  $a$, then $f(a)$ is a local maximum.
\item If $f'(x)<0$ to the left of $a$ and $f'(x)>0$ to the right of
  $a$, then $f(a)$ is a local minimum.
\item If $f'(x)$ has the same sign to the left and right of $a$,
  then $f(a)$ is not a local extremum.
\end{itemize}
\end{theorem}


\begin{theorem}[Test for Concavity]\index{concavity test}
Suppose that $f''(x)$ exists on an interval.
\begin{enumerate}
\item If $f''(x)>0$ on an interval, then $f(x)$ is concave up on that interval.
\item If $f''(x)<0$ on an interval, then $f(x)$ is concave down on that interval.
\end{enumerate}
\end{theorem}


\begin{theorem}[Second Derivative Test]\index{second derivative test}\label{T:sdt}
Suppose that $f''(x)$ is continuous on an open interval and that
$f'(a)=0$ for some value of $a$ in that interval.
\begin{itemize}
\item If $f''(a) <0$, then $f(x)$ has a local maximum at $a$.
\item If $f''(a) >0$, then $f(x)$ has a local minimum at $a$.
\item If $f''(a) =0$, then the test is inconclusive. In this case,
  $f(x)$ may or may not have a local extremum at $x=a$.
\end{itemize}
\end{theorem}



\section*{Product Rule}











\end{document}

