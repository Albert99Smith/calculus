\chapter{Integration by Substitution}

Computing antiderivatives is not as easy as computing derivatives. One
issue is that the chain rule can be difficult to ``undo.'' Sometimes 
it is helpful to transform the integral in question via substitution. 

\marginnote[.5in]{Here as is customary in calculus courses, we are abusing
  notation slightly, allowing $u$ to both be a name of a function
  $u(x)$, and a variable in the second integral.}
\begin{mainTheorem}[Integral Substitution Formula] 
If $u(x)$ is differentiable on the interval $[a,b]$ and $f(x)$ is
differentiable on the interval $[u(a),u(b)]$, then
\[
\int_a^b f'(u(x)) u'(x) \d x =\int_{u(a)}^{u(b)} f'(u) \d u.
\]
\end{mainTheorem}
\begin{proof} First we recognize the chain rule
\[
\int_a^b f'(u(x)) u'(x) \d x = \int_a^b (f\circ u)'(x) \d x.
\]
Next we apply the Fundamental Theorem of Calculus. 
\begin{align*} 
\int_a^b (f\circ u)'(x) \d x &= f(u(x)) \bigg|_a^b \\
&= f(x) \bigg|_{u(a)}^{u(b)}\\ 
&= \int_{g(a)}^{g(b)} f'(u) \d u.
\end{align*}
\end{proof}


There are several different ways to think about substitution. The
first is using the formula given above. Let's see an example. 
\begin{example}
Compute
\[
\int_1^3 x\cos(x^2)\d x.
\]
\end{example}

\marginnote[1.25in]{Here we are directly using the equation
\[
\int_a^b f'(u(x)) u'(x) \d x = \int_{u(a)}^{u(b)} f'(u) \d u.
\]}

\begin{solution}
A little thought reveals that if $x\cos(x^2)$ is the derivative of
some function, then it must have come from an application of the chain
rule. Here we have $x$ on the ``outside,'' which is the derivative of
$x^2$ on the ``inside,'' 
\[
\int \underbrace{x}_{\text{outside}}\cos(\underbrace{x^2}_{\text{inside}})\d x.
\]
Set $u(x) = x^2$ so $u'(x) = 2x$ and now it must be that $f(u) =
\frac{\cos(u)}{2}$. Now we see
\begin{align*}
\int_1^3 x\cos(x^2)\d x &= \int_1^9 \frac{\cos(u)}{2}\d u\\
&= \frac{\sin(u)}{2} \bigg|_1^9 \\
&= \frac{\sin(9) -\sin(1)}{2}.
\end{align*}
\end{solution}

Sometimes we frame the solution in a different way. Let's do the same
example again, this time we'll think in terms of differentials.

\begin{example}
Compute
\[
\int_1^3 x\cos(x^2)\d x.
\]
\end{example}
\begin{solution}
Here we will set $u=x^2$. Now $du = 2x \d x$, we are thinking in terms
of differentials. Now we see
\[
\int_{u(1)}^{u(3)} \frac{\cos(u)}{2}\d u = \int_1^3\frac{\cos(x^2)}{2}2x\d x.
\]
At this point, we can continue as we did before and write
\[
\int_1^3 x\cos(x^2)\d x= \frac{\sin(9) -\sin(1)}{2}.
\]
\end{solution}

Finally, sometimes we simply want to deal with the antiderivative on
its own, we'll repeat the example one more time demonstrating this.

\begin{example}
Compute
\[
\int_1^3 x\cos(x^2)\d x.
\]
\end{example}
\begin{solution}
Here we start as we did before, setting $u=x^2$. Now $du = 2x \d x$,
again thinking in terms of differentials. Now we see
\[
\int  \frac{\cos(u)}{2}\d u = \int \frac{\cos(x^2)}{2}2x\d x.
\]
Hence 
\[
\int x\cos(x^2)\d x = \frac{\sin(u)}{2} = \frac{\sin(x^2)}{2}.
\]
Now we see
\begin{align*}
\int_1^3 x\cos(x^2)\d x &=\frac{\sin(x^2)}{2}\bigg|_1^3\\
&= \frac{\sin(9) -\sin(1)}{2}.
\end{align*}
\end{solution}

With some experience, it is not hard to see which function is $f(x)$
and which is $u(x)$, let's see another example.
\begin{example}
Compute
\[
\int x^4(x^5+1)^{99} \d x.
\]
\end{example}

\begin{solution}
Here we set $u = x^5+1$ so $du = 5x^4 \d x$, and $f(u) = \frac{u^{99}}{5}$. Now
\begin{align*}
\int x^4(x^5+1)^{99} \d x &= \int \frac{u^{99}}{5} \d u\\
&= \frac{u^{100}}{500}.
\end{align*}
Recalling that $u = x^5+1$, we have our final answer
\[
\int x^4(x^5+1)^{99} \d x= \frac{(x^5+1)^{100}}{500}+C.
\]
\end{solution}


Our next example is a bit different.

\begin{example}
Compute
\[
\int_{2}^{3} \frac{1}{x\ln(x)} \d x.
\]
\end{example}

\begin{solution}
Let $u=\ln(x)$ so $du=\frac{1}{x}\d x$. Write
\begin{align*}
\int_{2}^{3} \frac{1}{x\ln(x)} \d x = \int_{\ln(2)}^{\ln(3)} \frac{1}{u} \d u\\
&= \ln(u) \bigg|_{\ln(2)}^{\ln(3)}\\
& = \ln(\ln(3)) - \ln(\ln(2)).
\end{align*}
\end{solution}


On the other hand our next example is much harder.

\begin{example} Compute
\[
\int x^3\sqrt{1-x^2}\d x.
\]
\end{example}

\begin{solution} 
Here it is not apparent that the chain rule is involved. However, if
it was involved, perhaps a good guess for $u$ would be
\[
u = 1-x^2
\]
in this case
\[
du = -2x \d x.
\]
Now consider our indefinite integral
\[
\int x^3\sqrt{1-x^2}\d x,
\]
immediately we can substitute. Write
\[
\int x^3\sqrt{1-x^2}\d x = \int -\frac{x^2\sqrt{u}}{2}\d u.
\]
However, we cannot continue until each $x$ is replaced. We know however that 
\begin{align*}
u &= 1-x^2 \\
u -1 &= -x^2\\
1- u &= x^2
\end{align*}
so now we may write
\[
\int x^3\sqrt{1-x^2}\d x = \int -\frac{(1-u)\sqrt{u}}{2}\d u.
\]
At this point, we are close to being done. Write
\begin{align*}
\int -\frac{(1-u)\sqrt{u}}{2}\d u &= \int \left(\frac{u\sqrt{u}}{2} - \frac{\sqrt{u}}{2}\right) \d u \\
&= \int \frac{u^{3/2}}{2} \d u - \int \frac{\sqrt{u}}{2} \d u \\
&= \frac{u^{5/2}}{5} - \frac{u^{3/2}}{3}.
\end{align*}
Now recall that $u = 1-x^2$. Hence our final answer is
\[
\int x^3\sqrt{1-x^2}\d x = \frac{(1-x^2)^{5/2}}{5} - \frac{(1-x^2)^{3/2}}{3}+C.
\]
\end{solution}

To summarize, if we suspect that a given function is the derivative of
another via the chain rule, we let $u$ denote a likely candidate for
the inner function, then translate the given function so that it is
written entirely in terms of $u$, with no $x$ remaining in the
expression. If we can integrate this new function of $u$, then the
antiderivative of the original function is obtained by replacing $u$
by the equivalent expression in $x$.



\begin{exercises}

\twocol

\begin{exercise} $\int (1-t)^9\d t$
\begin{answer} $-(1-t)^{10}/10+C$
\end{answer}\end{exercise}

\begin{exercise} $\int (x^2+1)^2\d x$
\begin{answer} $x^5/5+2x^3/3+x+C$
\end{answer}\end{exercise}

\begin{exercise} $\int x(x^2+1)^{100}\d x$
\begin{answer} $(x^2+1)^{101}/202+C$
\end{answer}\end{exercise}

\begin{exercise} $\int {1\over\root 3 \of {1-5t}}\d t$ 
\begin{answer} $-3(1-5t)^{2/3}/10+C$
\end{answer}\end{exercise}

\begin{exercise} $\int \sin^3x\cos x\d x$
\begin{answer} $(\sin^4x)/4+C$
\end{answer}\end{exercise}

\begin{exercise} $\int x\sqrt{100-x^2}\d x$
\begin{answer} $-(100-x^2)^{3/2}/3+C$
\end{answer}\end{exercise}

\begin{exercise} $\int {x^2\over\sqrt{1-x^3}}\d x$
\begin{answer} $-2\sqrt{1-x^3}/3+C$
\end{answer}\end{exercise}

\begin{exercise} $\int \cos(\pi t)\cos\bigl(\sin(\pi t)\bigr)\d t$
\begin{answer} $\sin(\sin\pi t)/\pi+C$
\end{answer}\end{exercise}

\begin{exercise} $\int {\sin x\over\cos^3 x}\d x$
\begin{answer} $1/(2\cos^2 x)=(1/2)\sec^2x+C$
\end{answer}\end{exercise}

\begin{exercise} $\int\tan x\d x$
\begin{answer} $-\ln|\cos x|+C$
\end{answer}\end{exercise}

\begin{exercise}  $\int_0^\pi\sin^5(3x)\cos(3x)\d x$
\begin{answer} $0$
\end{answer}\end{exercise}

\begin{exercise} $\int\sec^2x\tan x\d x$
\begin{answer} $\tan^2(x)/2+C$
\end{answer}\end{exercise}

\begin{exercise} $\int_0^{\sqrt{\pi}/2} x\sec^2(x^2)\tan(x^2)\d x$
\begin{answer} $1/4$
\end{answer}\end{exercise}

\begin{exercise} $\int {\sin(\tan x)\over\cos^2x}\d x$
\begin{answer} $-\cos(\tan x)+C$
\end{answer}\end{exercise}

\begin{exercise} $\int_3^4 {1\over(3x-7)^2}\d x$
\begin{answer} $1/10$
\end{answer}\end{exercise}

\begin{exercise} $\int_0^{\pi/6}(\cos^2x - \sin^2x)\d x$
\begin{answer} $\sqrt3/4$
\end{answer}\end{exercise}

\begin{exercise} $\int {6x\over(x^2 - 7)^{1/9}}\d x$
\begin{answer} $(27/8)(x^2-7)^{8/9}$
\end{answer}\end{exercise}

\begin{exercise} $\int_{-1}^1 (2x^3-1)(x^4-2x)^6\d x$
\begin{answer} $-(3^7+1)/14$
\end{answer}\end{exercise}

\begin{exercise} $\int_{-1}^1 \sin^7 x\d x$
\begin{answer} $0$
\end{answer}\end{exercise}

\begin{exercise} $\int f(x) f'(x)\d x$ 
\begin{answer} $f(x)^2/2$
\end{answer}\end{exercise}

\endtwocol

\end{exercises}
