\documentclass{ximera}

%\usepackage{todonotes}
%\usepackage{mathtools} %% Required for wide table Curl and Greens
%\usepackage{cuted} %% Required for wide table Curl and Greens
\newcommand{\todo}{}

\usepackage{esint} % for \oiint
\ifxake%%https://math.meta.stackexchange.com/questions/9973/how-do-you-render-a-closed-surface-double-integral
\renewcommand{\oiint}{{\large\bigcirc}\kern-1.56em\iint}
\fi


\graphicspath{
  {./}
  {ximeraTutorial/}
  {basicPhilosophy/}
  {functionsOfSeveralVariables/}
  {normalVectors/}
  {lagrangeMultipliers/}
  {vectorFields/}
  {greensTheorem/}
  {shapeOfThingsToCome/}
  {dotProducts/}
  {partialDerivativesAndTheGradientVector/}
  {../productAndQuotientRules/exercises/}
  {../normalVectors/exercisesParametricPlots/}
  {../continuityOfFunctionsOfSeveralVariables/exercises/}
  {../partialDerivativesAndTheGradientVector/exercises/}
  {../directionalDerivativeAndChainRule/exercises/}
  {../commonCoordinates/exercisesCylindricalCoordinates/}
  {../commonCoordinates/exercisesSphericalCoordinates/}
  {../greensTheorem/exercisesCurlAndLineIntegrals/}
  {../greensTheorem/exercisesDivergenceAndLineIntegrals/}
  {../shapeOfThingsToCome/exercisesDivergenceTheorem/}
  {../greensTheorem/}
  {../shapeOfThingsToCome/}
  {../separableDifferentialEquations/exercises/}
  {vectorFields/}
}

\newcommand{\mooculus}{\textsf{\textbf{MOOC}\textnormal{\textsf{ULUS}}}}

\usepackage{tkz-euclide}\usepackage{tikz}
\usepackage{tikz-cd}
\usetikzlibrary{arrows}
\tikzset{>=stealth,commutative diagrams/.cd,
  arrow style=tikz,diagrams={>=stealth}} %% cool arrow head
\tikzset{shorten <>/.style={ shorten >=#1, shorten <=#1 } } %% allows shorter vectors

\usetikzlibrary{backgrounds} %% for boxes around graphs
\usetikzlibrary{shapes,positioning}  %% Clouds and stars
\usetikzlibrary{matrix} %% for matrix
\usepgfplotslibrary{polar} %% for polar plots
\usepgfplotslibrary{fillbetween} %% to shade area between curves in TikZ
\usetkzobj{all}
\usepackage[makeroom]{cancel} %% for strike outs
%\usepackage{mathtools} %% for pretty underbrace % Breaks Ximera
%\usepackage{multicol}
\usepackage{pgffor} %% required for integral for loops



%% http://tex.stackexchange.com/questions/66490/drawing-a-tikz-arc-specifying-the-center
%% Draws beach ball
\tikzset{pics/carc/.style args={#1:#2:#3}{code={\draw[pic actions] (#1:#3) arc(#1:#2:#3);}}}



\usepackage{array}
\setlength{\extrarowheight}{+.1cm}
\newdimen\digitwidth
\settowidth\digitwidth{9}
\def\divrule#1#2{
\noalign{\moveright#1\digitwidth
\vbox{\hrule width#2\digitwidth}}}





\newcommand{\RR}{\mathbb R}
\newcommand{\R}{\mathbb R}
\newcommand{\N}{\mathbb N}
\newcommand{\Z}{\mathbb Z}

\newcommand{\sagemath}{\textsf{SageMath}}


%\renewcommand{\d}{\,d\!}
\renewcommand{\d}{\mathop{}\!d}
\newcommand{\dd}[2][]{\frac{\d #1}{\d #2}}
\newcommand{\pp}[2][]{\frac{\partial #1}{\partial #2}}
\renewcommand{\l}{\ell}
\newcommand{\ddx}{\frac{d}{\d x}}

\newcommand{\zeroOverZero}{\ensuremath{\boldsymbol{\tfrac{0}{0}}}}
\newcommand{\inftyOverInfty}{\ensuremath{\boldsymbol{\tfrac{\infty}{\infty}}}}
\newcommand{\zeroOverInfty}{\ensuremath{\boldsymbol{\tfrac{0}{\infty}}}}
\newcommand{\zeroTimesInfty}{\ensuremath{\small\boldsymbol{0\cdot \infty}}}
\newcommand{\inftyMinusInfty}{\ensuremath{\small\boldsymbol{\infty - \infty}}}
\newcommand{\oneToInfty}{\ensuremath{\boldsymbol{1^\infty}}}
\newcommand{\zeroToZero}{\ensuremath{\boldsymbol{0^0}}}
\newcommand{\inftyToZero}{\ensuremath{\boldsymbol{\infty^0}}}



\newcommand{\numOverZero}{\ensuremath{\boldsymbol{\tfrac{\#}{0}}}}
\newcommand{\dfn}{\textbf}
%\newcommand{\unit}{\,\mathrm}
\newcommand{\unit}{\mathop{}\!\mathrm}
\newcommand{\eval}[1]{\bigg[ #1 \bigg]}
\newcommand{\seq}[1]{\left( #1 \right)}
\renewcommand{\epsilon}{\varepsilon}
\renewcommand{\phi}{\varphi}


\renewcommand{\iff}{\Leftrightarrow}

\DeclareMathOperator{\arccot}{arccot}
\DeclareMathOperator{\arcsec}{arcsec}
\DeclareMathOperator{\arccsc}{arccsc}
\DeclareMathOperator{\si}{Si}
\DeclareMathOperator{\scal}{scal}
\DeclareMathOperator{\sign}{sign}


%% \newcommand{\tightoverset}[2]{% for arrow vec
%%   \mathop{#2}\limits^{\vbox to -.5ex{\kern-0.75ex\hbox{$#1$}\vss}}}
\newcommand{\arrowvec}[1]{{\overset{\rightharpoonup}{#1}}}
%\renewcommand{\vec}[1]{\arrowvec{\mathbf{#1}}}
\renewcommand{\vec}[1]{{\overset{\boldsymbol{\rightharpoonup}}{\mathbf{#1}}}\hspace{0in}}

\newcommand{\point}[1]{\left(#1\right)} %this allows \vector{ to be changed to \vector{ with a quick find and replace
\newcommand{\pt}[1]{\mathbf{#1}} %this allows \vec{ to be changed to \vec{ with a quick find and replace
\newcommand{\Lim}[2]{\lim_{\point{#1} \to \point{#2}}} %Bart, I changed this to point since I want to use it.  It runs through both of the exercise and exerciseE files in limits section, which is why it was in each document to start with.

\DeclareMathOperator{\proj}{\mathbf{proj}}
\newcommand{\veci}{{\boldsymbol{\hat{\imath}}}}
\newcommand{\vecj}{{\boldsymbol{\hat{\jmath}}}}
\newcommand{\veck}{{\boldsymbol{\hat{k}}}}
\newcommand{\vecl}{\vec{\boldsymbol{\l}}}
\newcommand{\uvec}[1]{\mathbf{\hat{#1}}}
\newcommand{\utan}{\mathbf{\hat{t}}}
\newcommand{\unormal}{\mathbf{\hat{n}}}
\newcommand{\ubinormal}{\mathbf{\hat{b}}}

\newcommand{\dotp}{\bullet}
\newcommand{\cross}{\boldsymbol\times}
\newcommand{\grad}{\boldsymbol\nabla}
\newcommand{\divergence}{\grad\dotp}
\newcommand{\curl}{\grad\cross}
%\DeclareMathOperator{\divergence}{divergence}
%\DeclareMathOperator{\curl}[1]{\grad\cross #1}
\newcommand{\lto}{\mathop{\longrightarrow\,}\limits}

\renewcommand{\bar}{\overline}

\colorlet{textColor}{black}
\colorlet{background}{white}
\colorlet{penColor}{blue!50!black} % Color of a curve in a plot
\colorlet{penColor2}{red!50!black}% Color of a curve in a plot
\colorlet{penColor3}{red!50!blue} % Color of a curve in a plot
\colorlet{penColor4}{green!50!black} % Color of a curve in a plot
\colorlet{penColor5}{orange!80!black} % Color of a curve in a plot
\colorlet{penColor6}{yellow!70!black} % Color of a curve in a plot
\colorlet{fill1}{penColor!20} % Color of fill in a plot
\colorlet{fill2}{penColor2!20} % Color of fill in a plot
\colorlet{fillp}{fill1} % Color of positive area
\colorlet{filln}{penColor2!20} % Color of negative area
\colorlet{fill3}{penColor3!20} % Fill
\colorlet{fill4}{penColor4!20} % Fill
\colorlet{fill5}{penColor5!20} % Fill
\colorlet{gridColor}{gray!50} % Color of grid in a plot

\newcommand{\surfaceColor}{violet}
\newcommand{\surfaceColorTwo}{redyellow}
\newcommand{\sliceColor}{greenyellow}




\pgfmathdeclarefunction{gauss}{2}{% gives gaussian
  \pgfmathparse{1/(#2*sqrt(2*pi))*exp(-((x-#1)^2)/(2*#2^2))}%
}


%%%%%%%%%%%%%
%% Vectors
%%%%%%%%%%%%%

%% Simple horiz vectors
\renewcommand{\vector}[1]{\left\langle #1\right\rangle}


%% %% Complex Horiz Vectors with angle brackets
%% \makeatletter
%% \renewcommand{\vector}[2][ , ]{\left\langle%
%%   \def\nextitem{\def\nextitem{#1}}%
%%   \@for \el:=#2\do{\nextitem\el}\right\rangle%
%% }
%% \makeatother

%% %% Vertical Vectors
%% \def\vector#1{\begin{bmatrix}\vecListA#1,,\end{bmatrix}}
%% \def\vecListA#1,{\if,#1,\else #1\cr \expandafter \vecListA \fi}

%%%%%%%%%%%%%
%% End of vectors
%%%%%%%%%%%%%

%\newcommand{\fullwidth}{}
%\newcommand{\normalwidth}{}



%% makes a snazzy t-chart for evaluating functions
%\newenvironment{tchart}{\rowcolors{2}{}{background!90!textColor}\array}{\endarray}

%%This is to help with formatting on future title pages.
\newenvironment{sectionOutcomes}{}{}



%% Flowchart stuff
%\tikzstyle{startstop} = [rectangle, rounded corners, minimum width=3cm, minimum height=1cm,text centered, draw=black]
%\tikzstyle{question} = [rectangle, minimum width=3cm, minimum height=1cm, text centered, draw=black]
%\tikzstyle{decision} = [trapezium, trapezium left angle=70, trapezium right angle=110, minimum width=3cm, minimum height=1cm, text centered, draw=black]
%\tikzstyle{question} = [rectangle, rounded corners, minimum width=3cm, minimum height=1cm,text centered, draw=black]
%\tikzstyle{process} = [rectangle, minimum width=3cm, minimum height=1cm, text centered, draw=black]
%\tikzstyle{decision} = [trapezium, trapezium left angle=70, trapezium right angle=110, minimum width=3cm, minimum height=1cm, text centered, draw=black]



\outcome{Compute dot products.}
\outcome{Use dot products to compute the angle between vectors.}
\outcome{Find orthogonal projections.}
\outcome{Use the dot product in applied settings.}

\title[Dig-In:]{The dot product}

\begin{document}
\begin{abstract}
  The dot product is one way to multiply two vectors.
\end{abstract}
\maketitle


\section{The definition of the dot product}

We have already seen how to add vectors and how to multiply vectors by
scalars.

\begin{warning}
We have not yet defined how to multiply a vector by a vector.  You
might think it is reasonable to define
\[
\begin{bmatrix}
  a_1\\
  a_2\\
  \vdots\\
  a_n
\end{bmatrix}
\cdot
\begin{bmatrix}
  b_1\\
  b_2\\
  \vdots\\
  b_n
\end{bmatrix}
=
\begin{bmatrix}
  a_1b_1\\
  a_2b_2\\
  \vdots\\
  a_nb_n
\end{bmatrix}
\] 
but this operation is not especially useful, and will \textbf{never be
  utilized in this course}.
\end{warning}

In this section we will define a way to ``multiply'' two vectors
called the \textit{dot product}.

\begin{definition}
  The \dfn{dot product} of two vectors is given by:
  \begin{align*}
  \begin{bmatrix}
    a_1\\
    a_2\\
    \vdots\\
    a_n
  \end{bmatrix}
  \dotp
  \begin{bmatrix}
    b_1\\
    b_2\\
    \vdots\\
    b_n
  \end{bmatrix}
  &= \sum_{i=1}^n a_ib_i\\
  &= a_1b_1 + a_2b_2 +\dots+a_nb_n
  \end{align*}
\end{definition}

The first thing you should notice about the the dot product is that
\[
\mathbf{vector}\dotp \mathbf{vector} = \mathbf{number}.
\]
\begin{question}
  Compute:
  \[
  \vector{1,2,3}\dotp\vector{4,5,6}
  \begin{prompt}
    = \answer{32}
  \end{prompt}
  \]
  \begin{question}
  Compute:
  \[
  \vector{1,1,-1}\dotp\vector{1,1,2}
  \begin{prompt}
    = \answer{0}
  \end{prompt}
  \]
\end{question}
\end{question}

\begin{question}
  Let $\vec{u},\vec{v},\vec{w}$ be nonzero vectors in $\R^3$. Which of
  the following expressions make sense?
  \begin{selectAll}
    \choice[correct]{$(\vec{w} \dotp \vec{u} ) \vec{u}$}
    \choice[correct]{$5(\vec{u} +\vec{w}) \dotp {\vec{u}}$}
    \choice{$\vec{w} / \vec{u}$}
    \choice[correct]{$\vector{2,3} \dotp \vector{4,2} + 7$}
    \choice[correct]{$\vec{w} / ( \vec{u} \dotp \vec{u})$}
    \choice{$\vector{1,3} \dotp \vector{-1,2,5}$}
    \choice{$\vec{u}\dotp \vec{v}+\vec{w}$}
  \end{selectAll}
\end{question}

The dot product allows us to write some complex formulas more simply.

\begin{theorem}
  The magnitude of vector $\vec{v}$ is given by
  \[
  |\vec{v}|=\sqrt{\vec{v}\dotp\vec{v}}
  \]
  \begin{explanation}
    We already know that if $\vec{v} = \vector{v_1,v_2,\dots,v_n}$,
    then
    \[
    |\vec{v}| = \sqrt{v_1^2+v_2^2+v_3^2+\dots+v_n^2}
    \]
    but
    \[
    \vec{v} \dotp \vec{v} = v_1^2+v_2^2+v_3^2+\dots+v_n^2,
    \]
    so
    \[
    |\vec{v}|=\sqrt{\vec{v}\dotp\vec{v}}.
    \]
  \end{explanation}
\end{theorem}

\begin{question}
  Compute the magnitude of the vector $\vec{v} = \vector{1,2,3,4}$.
  \begin{prompt}
    \[
    |\vec{v}| = \answer{\sqrt{30}}
    \]
  \end{prompt}
\end{question}



\section{The geometry of the dot product}

Now let's see if we can figure out what the dot product tells us. As
an appetizer, we give the next theorem, recall the law of cosines:

\begin{theorem}[Law of Cosines]
  Given a triangle with sides of length $a$, $b$, and $c$, and with
  $0\le\theta\le\pi$ being the measure of the angle between the sides
  of length $a$ and $b$,
  \begin{image}
    \begin{tikzpicture}
    \draw (.5,.1) arc[radius=.5cm,start angle=11.3,end angle=56.3];
    \draw[ultra thick,penColor] (0,0) -- (5,1) -- (2,3)--(0,0)--cycle;
    \node[below,penColor] at (2.5,.5) {$a$}; %% <a,b>
    \node[above left,penColor] at (1,1.5) {$b$}; %% <c,d>
    \node[above right,penColor] at (3.5,2) {$c$}; %% <c,d>
    \node[above right] at (.4,.2) {$\theta$}; %% <c,d>
\end{tikzpicture}
  \end{image}
  we have
  \[
  c^2 = a^2+b^2-2ab\cos(\theta)
  \]
\end{theorem}
\begin{question}
  When $\theta = \pi/2$ what does the law of cosines say?
  \begin{prompt}
    \begin{multipleChoice}
      \choice[correct]{It is the Pythagorean theorem.}
      \choice{It is the law of sines.}
      \choice{It is undefined.}
    \end{multipleChoice}
  \end{prompt}
\end{question}

We can rephrase the law of cosines in the language of vectors.  The
vectors $\vec{v}$, $\vec{w}$, and $\vec{v} - \vec{w}$ form a triangle:
\begin{image}
  \begin{tikzpicture}
    \draw (.5,.1) arc[radius=.5cm,start angle=11.3,end angle=56.3];
    \draw[->,ultra thick,penColor] (0,0) -- (5,1);
    \draw[->,ultra thick,penColor2] (0,0) -- (2,3);
    \draw[->,ultra thick,penColor3] (2,3) -- (5,1);
    \node[below,penColor] at (2.5,.5) {$\vec{v}$}; %% <a,b>
    \node[above left,penColor2] at (1,1.5) {$\vec{w}$}; %% <c,d>
    \node[above right,penColor3] at (3.5,2) {$\vec{v}-\vec{w}$}; %% <c,d>
    \node[above right] at (.4,.2) {$\theta$}; %% <c,d>
\end{tikzpicture}
\end{image}
so if $\theta$ is the angle between $\vec{v}$ and $\vec{w}$ we must
have
\[
|\vec{v} - \vec{w}|^2=|\vec{v}|^2+|\vec{w}|^2-2|\vec{w}||\vec{v}|\cos(\theta)
\]



\begin{theorem}[Geometric Interpretation of the Dot Product]\index{dot product}
  For any two vectors $\vec{v}$ and $\vec{w}$,
  \[
  \vec{v} \dotp \vec{w} = |\vec{v}||\vec{w}|\cos(\theta)
  \]
  where $0\le \theta\le\pi$ is the angle between $\vec{v}$ and
  $\vec{w}$.
  \begin{explanation}
    First note that
    \[
    |\vec{v} - \vec{w}|^2 =  (\vec{v} - \vec{w})\dotp(\vec{v} - \vec{w})
    \]
    Now use the law of cosines to write
    \begin{align*}
      |\vec{v} - \vec{w}|^2&=|\vec{v}|^2+|\vec{w}|^2-2|\vec{v}||\vec{w}|\cos(\theta)\\
      (\vec{v} - \vec{w})\dotp(\vec{v} - \vec{w}) &=|\vec{v}|^2+|\vec{w}|^2-2|\vec{v}||\vec{w}|\cos(\theta)\\
      \vec{v}\dotp\vec{v} -2\vec{v}\dotp\vec{w}+\vec{w}\dotp\vec{w}&=|\vec{v}|^2+|\vec{w}|^2-2|\vec{v}||\vec{w}|\cos(\theta)\\
      |\vec{v}|^2+|\vec{w}|^2 -2\vec{v}\dotp\vec{w} &=|\vec{v}|^2+|\vec{w}|^2-2|\vec{v}||\vec{w}|\cos(\theta)\\
      \vec{v} \dotp \vec{w} &= |\vec{v}||\vec{w}|\cos(\theta).
    \end{align*}
  \end{explanation}
\end{theorem}
	
This tells us something interesting about the dot product:

\begin{theorem}
  If $\vec{v}$ and $\vec{w}$ are two nonzero vectors, and $\theta$ is
  the angle between them,
  \[
  \vec{v}\dotp \vec{w} = 0 \text{ if and only if } \theta=
  \frac{\pi}{2}.
  \]
\end{theorem}

We have a special buzz-word for when the angle between two vectors is $\pi/2$:

\begin{definition}
  Two vectors are called \dfn{orthogonal} if the angle between them is
  $\pi/2$.
\end{definition}

From this we see that the dot product of two vectors is zero if those
vectors are orthogonal.  Moreover, if the dot product is not zero,
using the formula
\[
\vec{v} \dotp \vec{w} = |\vec{v}||\vec{w}|\cos(\theta)
\]
allows us to compute the angle between these vectors via
\[
\theta = \arccos\left(\frac{\vec{v} \dotp \vec{w} }{|\vec{v}|\cdot|\vec{w}|}\right).
\]

\begin{question}
  Find the angle between the vectors
  \begin{align*}
  \vec{v} &= 2\veci+3\vecj+6\veck\\
  \vec{w} &= 1\veci+2\vecj+2\veck
  \end{align*}
  \begin{prompt}
  \[
  \theta = \answer{ \arccos\left(\frac{20}{21}\right)}
  \]
  \end{prompt}
  \begin{feedback}
    Think about how hard this question would have been before you read this section!
  \end{feedback}
\end{question}



\section{Projections and components}

\subsection{Projections}
One of the major uses of the dot product is to let us \textit{project}
one vector in the direction of another.

\begin{definition}
  The \dfn{orthogonal projection}\index{projection} in the direction
  of vector $\vec{w}$ of vector $\vec{v}$ is a new vector denoted
  $\proj_\vec{w}(\vec{v})$,
  \begin{image}
    \begin{tikzpicture}
      \draw[dashed] (-.5,-.5) -- (3.5,3.5);
      \draw[ultra thick,penColor4,->] (0,0) -- (3,1);
      \draw[ultra thick,penColor2,->] (-0,0) -- (0.707,0.707);
      %\draw[textColor, dashed] (3,1) -- (2,2);
      \node[below] at (1.5, 0.5) [penColor4] {$\vec{v}$};
      \node at (0.2, .5) [penColor2] {$\vec{w}$};
    \end{tikzpicture}
    \qquad
    \begin{tikzpicture}
      \draw[textColor, dashed] (3,1) -- (2,2);
      \draw[textColor, thin] (2.2,1.8) -- (2,1.6)--(1.8,1.8);
      \draw[draw=none] (-.5,-.5) -- (3.5,3.5);
      \draw[very thick, penColor,->] (0,0) -- (2,2);
      \draw[ultra thick,penColor4,->] (0,0) -- (3,1);
      \draw[ultra thick,penColor2,->] (-0,0) -- (0.707,0.707);
      %\node[above] at (1.5, 0.5) [penColor] {$\vec{v}$};
      %\node at (0.2, .5) [penColor2] {$\vec{w}$};
      \node[above left] at (1, 1) [penColor] {$\proj_\vec{w}(\vec{v})$};
    \end{tikzpicture}
  \end{image}
  that lies on the line containing $\vec{w}$, with the vector
  $\proj_\vec{w}(\vec{v}) - \vec{v}$ perpendicular to $\vec{w}$.
\end{definition}

\begin{question}
  Consider the vector $\vec{v}=\vector{3,2,1}$ and the vector $\veci =
  \vector{1,0,0}$.  Compute $\proj_\veci(\vec{v})$.
  \begin{hint}
    Draw a picture.
  \end{hint}
  \begin{prompt}
    \[
    \proj_\veci(\vec{v}) = \vector{\answer{3},\answer{0},\answer{0}}
    \]
  \end{prompt}
  \begin{question}
    Let $\vec{v} = \vector{1,1}$ and $\vec{w}=\vector{-1,1}$. Compute
    $\proj_\vec{w}(\vec{v})$.
    \begin{hint}
      Draw a picture.
    \end{hint}
      \begin{prompt}
        \[
        \proj_\vec{w}(\vec{v}) = \vector{\answer{0},\answer{0}}
        \]
      \end{prompt}
  \end{question}
\end{question}

To compute the projection of a vector along another, we use the dot
product:

\begin{theorem}
  Given two vectors, $\vec{v}$ and $\vec{w}$,
  \[
  \proj_\vec{w}(\vec{v}) =\left(\frac{\vec{v} \dotp \vec{w}}{\vec{w}\dotp \vec{w}}\right) \vec{w}=\left(\frac{\vec{v} \dotp \vec{w}}{|\vec{w}|^2}\right) \vec{w}.
  \]
  \begin{explanation}
    First note that the direction of $\proj_\vec{w}(\vec{v})$ is given by:
    \[
    \frac{\vec{w}}{|\vec{w}|}
    \]
    and the magnitude of $\proj_\vec{w}(\vec{v})$
    \begin{image}
      \begin{tikzpicture}
        \draw (.5,.1) arc[radius=.5cm,start angle=11.3,end angle=56.3];
        \draw[->,ultra thick,penColor] (0,0) -- (5,1);
        \draw[->,ultra thick,penColor2] (0,0) -- (2,3);
        \node[below,penColor] at (2.5,.5) {$\vec{w}$}; %% <a,b>
        \node[above left,penColor2] at (1,1.5) {$\vec{v}$}; %% <c,d>
        \node[above right] at (.4,.2) {$\theta$}; %% <c,d>
      \end{tikzpicture}
    \end{image}
    is given by:
    \[
    |\proj_{\vec{w}}(\vec{v})| = |\vec{v}|\cdot \answer[given]{\cos(\theta)}
    \]
    Thus
    \begin{align*}
      \proj_\vec{w}(\vec{v}) &= \mathrm{direction}\cdot\mathrm{magnitude}\\
      &= \frac{\vec{w}}{|\vec{w}|}\cdot|\vec{v}|\cdot\answer[given]{\cos(\theta)}\\
      &= \frac{\vec{w}}{|\vec{w}|^2}\cdot |\vec{w}|\cdot |\vec{v}|\cdot\answer[given]{\cos(\theta)}\\
      &= \left(\frac{\vec{v} \dotp \vec{w}}{|\vec{w}|^2}\right) \vec{w}.
    \end{align*}
  \end{explanation}
\end{theorem}

\begin{question}
  Consider the formula:
  \[
  \proj_\vec{w}(\vec{v}) =\left(\frac{\vec{v} \dotp \vec{w}}{|\vec{w}|^2}\right) \vec{w}
  \]
  Is $\vec{v}\dotp \vec{w}$ a number or a vector?
  \begin{prompt}
  \begin{multipleChoice}
    \choice[correct]{number}
    \choice{vector}
  \end{multipleChoice}
  \end{prompt}
  \begin{question}
    Is $|\vec{w}|^2$ a number or a vector?
     \begin{prompt}
    \begin{multipleChoice}  
      \choice[correct]{number}
      \choice{vector}
      \end{multipleChoice}
     \end{prompt}
    \begin{question}
      Is $\vec{w}$ a number or a vector?
      \begin{prompt}
        \begin{multipleChoice}
          \choice{number}
          \choice[correct]{vector}
        \end{multipleChoice}
      \end{prompt}
    \end{question}
  \end{question}
\end{question}

\begin{question}
  Find the projection of the vector $\vec{v} = \vector{2,3,1}$ in the
  direction of the vector $\vec{w} = \vector{3,-1,1}$.
  \begin{prompt}
  \[
  \proj_\vec{w}(\vec{v}) = \vector{\answer{\frac{12}{11}},\answer{\frac{-4}{11}},\answer{\frac{4}{11}}}
  \]
  \end{prompt}
\end{question}

\subsection{Components}

\begin{definition}
  Let $\vec{v}$ and $\vec{w}$ be vectors and let $0\le\theta\le\pi$ be
  the angle between them.  The \dfn{scalar component}\index{component}
  in the direction of $\vec{w}$ of vector $\vec{v}$ is denoted
  \[
  \scal_\vec{w}(\vec{v})=
  \begin{cases}
    |\proj_\vec{w}(\vec{v})| &\text{when $0\le\theta\le \pi/2$}\\
    -|\proj_\vec{w}(\vec{v})| &\text{when $\pi/2<\theta\le \pi$.}
  \end{cases}
  \]
\end{definition}

\begin{question}
  Let $\vec{v} = \vector{3,-2,1}$. Compute $\scal_\veci(\vec{v})$.
  \begin{prompt}
    \[
    \scal_\veci(\vec{v}) = \answer{3}
    \]
  \end{prompt}
  \begin{question}
    Compute $\scal_\vecj(\vec{v})$.
    \begin{prompt}
      \[
      \scal_\vecj(\vec{v}) = \answer{-2}
      \]
    \end{prompt}
    \begin{question}
      Compute $\scal_\veck(\vec{v})$.
      \begin{prompt}
        \[
        \scal_\veck(\vec{v}) = \answer{1}
        \]
      \end{prompt}
    \end{question}
  \end{question}
\end{question}
To compute the scalar component of a vector in the direction of
another, you use the dot product:

\begin{theorem}
  Given two vectors, $\vec{v}$ and $\vec{w}$,
  \[
  \scal_\vec{w}(\vec{v}) =\frac{\vec{v} \dotp \vec{w}}{|\vec{w}|}.
  \]
\end{theorem}

\begin{question}
  Let $\vec{v}$ and $\vec{w}$ be nonzero vectors and let $\theta$ be
  the angle between them. Which of the following are true?
  \begin{selectAll}
    \choice{$|\proj_\vec{w}(\vec{v})| = \scal_{\vec{w}}(\vec{v})$}
    \choice[correct]{$|\proj_\vec{w}(\vec{v})| = |\scal_{\vec{w}}(\vec{v})|$}
    \choice[correct]{$\proj_\vec{w}(\vec{v}) =|\vec{v}|\cos(\theta)\left(\frac{\vec{w}}{|\vec{w}|}\right)$}
    \choice[correct]{$\scal_\vec{w}(\vec{v}) = |\vec{v}|\cos(\theta)$}
  \end{selectAll}
\end{question}


\subsection{Orthogonal decomposition}

\begin{definition}
Let $\vec v$ and $\vec w$ be vectors. The \dfn{orthogonal
  decomposition} of $\vec v$ in terms of $\vec{w}$ is the sum
\[
\vec v = \underbrace{\proj_{\vec{w}}(\vec{v})}_{\parallel \vec w} +  (\underbrace{\vec v-\proj_{\vec{w}}(\vec{v})}_{\perp \vec w}).
\]
where $\vec{x} \parallel \vec{y}$ means that ``$\vec{x}$ is parallel
to $\vec{y}$'' and $\vec{x} \perp\vec{y}$ means that ``$\vec{x}$ is
perpendicular to $\vec{y}$.''
\end{definition}

\begin{question}
Let $\vec u = \vector{-2,1}$ and $\vec v = \vector{3,1}$.  What is the
orthogonal decomposition of $\vec{u}$ in terms of $\vec{v}$?
\begin{prompt}
\[
\vec u= \underbrace{\vector{\answer{-1.5},\answer{-0.5}}}_{\parallel \vec v} + \underbrace{\vector{\answer{-0.5},\answer{1.5}}}_{\perp \vec v}.
\]
\end{prompt}
\begin{question}
  Let $\vec w =\vector{2,1,3}$ and $\vec x  =\vector{ 1,1,1}$. What is the
  orthogonal decomposition of $\vec{w}$ in terms of $\vec{x}$?
  \begin{prompt}
  \[
  \vec w  = \underbrace{\vector{\answer{2},\answer{2},\answer{2}}}_{\parallel \vec x} + \underbrace{\vector{ \answer{0},\answer{-1},\answer{1}}}_{\perp\vec x}
  \]
  \end{prompt}
\end{question}
\end{question}


Now we give an example of where this decomposition is useful.

\begin{example}
  Consider a box weighing $50\unit{lb}$ on a ramp that rises
  $5\unit{ft}$ over a span of $20\unit{ft}$.
  \begin{image}
    \begin{tikzpicture}
      \begin{scope}[scale=.25]
	\draw [thin] (20,5) -- node [right,pos=.5] {\scriptsize $5$} (20,0);
        \draw [very thick,->,penColor] (0,0) --  (20,5) node [above] {\scriptsize $\vec r$};
        \draw [thin] (0,0) --  node [below,pos=.6] {\scriptsize $20$}(20,0);
	\draw [thin] (10,2.5) -- (9.25,5.5) -- (12.25,6.25) -- (13,3.25); %box
	\draw [very thick,->,penColor2] (11.125,4.375) -- (11.125,-3.625) node [below] {\scriptsize $\vec g$}; %g
      \end{scope}
    \end{tikzpicture}
  \end{image}
  Find the orthogonal decomposition of $\vec{g}$ in terms of $\vec{r}$.
  \begin{explanation}
    To find the force of gravity in the direction of the ramp, we
    compute $\proj_\vec{r}(\vec g)$:
    \begin{align*}
      \proj_\vec{r}(\vec{g}) &= \left(\frac{\vec{g}\dotp\vec{r}}{\vec{r}\dotp\vec{r}}\right)\vec r\\
      &=  \answer[given]{\frac{-250}{425}}\vector{20,5}\\
      &= \vector{\answer[given]{\frac{-200}{17}},\answer[given]{\frac{-50}{17}}}.
    \end{align*}
    To find the component $\vec z$ of gravity orthogonal to the ramp,
    write with me
    \begin{align*}
      \vec z &= \vec g - \proj_\vec{r}(\vec{g}) \\
      &= \vector{\answer[given]{\frac{200}{17}},\answer[given]{\frac{-800}{17}}}.
    \end{align*}
    \begin{image}
    \begin{tikzpicture}
      \begin{scope}[scale=.25]
        \draw [thin] (20,5) -- node [right,pos=.5] {\scriptsize $5$} (20,0);
        \draw [very thick,->,penColor] (0,0) --  (20,5) node [above] {\scriptsize $\vec r$};
        \draw [thin] (0,0) --  node [below,pos=.6] {\scriptsize $20$}(20,0);
	\draw [thin] (10,2.5) -- (9.25,5.5) -- (12.25,6.25) -- (13,3.25); %box
	\draw [very thick,->,penColor2] (11.125,4.375) -- (11.125,-3.625) node [below] {\scriptsize $\vec g$}; %g
	\draw [penColor3,very thick,->] (11.125,4.375) -- (13,-3.15) node [right,pos=.4,penColor3] {\scriptsize $\vec z$};
	\draw [penColor4,very thick,->] (13,-3.15) -- (11.125,-3.625)node [shift={(10pt,-5pt)} ,penColor4,pos=0] {\scriptsize $\proj_{\vec r}(\vec g)$};
      \end{scope}
    \end{tikzpicture}
    \end{image}
\end{explanation}
\end{example}






\section{The algebra of the dot product}

We summarize the arithmetic and algebraic properties of the dot
product below:
\begin{theorem}
  The following are true for all scalars $s$ and vectors
  $\vec{u}$, $\vec{v}$, and $\vec{w}$ in $\R^n$:
  \begin{description}
  \item[Commutativity:] $\vec{v} \dotp \vec{w} = \vec{w} \dotp
    \vec{v}$.
  \item[Linear in first argument:] $(\vec{u}+\vec{v})\dotp \vec{w} = \vec{u}\dotp \vec{w} +
    \vec{v}\dotp \vec{w}$ and $(s\vec{v})\dotp \vec{w} = s(\vec{v}
    \dotp \vec{w})$.
  \item[Linear in second argument:] $\vec{u} \dotp (\vec{v}+\vec{w}) = \vec{u}\dotp \vec{v}+
    \vec{u}\dotp \vec{w}$ and $\vec{v} \dotp (s\vec{w}) = s(\vec{v}
    \dotp \vec{w})$.
  \item[Relation to magnitude:] $\vec{v} \dotp \vec{v} = |\vec{v}|^2$.
  \item[Relation to orthogonality:] If $\vec{v}$ is orthogonal to
    $\vec{w}$ then $\vec{v} \dotp \vec{w} = 0$.
  \end{description}
\end{theorem}

Instead of defining the dot product by a formula, we could have
defined it by the properties above!  While this is common practice in
mathematics, it is a bit abstract, and is perhaps beyond the scope of
this course. Nevertheless, we know that you are an intrepid young
mathematician, and we will not hold back.  We will now show that there
is only one formula which gives us all of these properties, and it
will be our formula for the dot product.

\begin{theorem}
  The dot product is given by the following formula:
    \begin{align*}
  \begin{bmatrix}
    a_1\\
    a_2\\
    \vdots\\
    a_n
  \end{bmatrix}
  \dotp
  \begin{bmatrix}
    b_1\\
    b_2\\
    \vdots\\
    b_n
  \end{bmatrix}
  &= \sum_{i=1}^n a_ib_i\\
  &= a_1b_1 + a_2b_2 +\dots+a_nb_n
    \end{align*}
\begin{explanation}
  Let 
  \[
  \vec{\hat e}_j = \vector{0,0,0,\dots,1,\dots,0}
  \]	
  be the vector whose $j$th coordinate is $1$, with all other
  coordinates being $0$. Then
  \[ 
  \vector{a_1,a_2, \dots,a_n} = \sum_{i=1}^n a_i \vec{\hat e}_i
  \]
  and
  \[ 
  \vector{b_1,b_2, \dots,b_n} = \sum_{i=1}^n b_j \vec{\hat e}_j
  \]	 
    Then
    \[
    \begin{bmatrix}
      a_1\\
      a_2\\
      \vdots\\
      a_n
    \end{bmatrix}
    \dotp
    \begin{bmatrix}
      b_1\\
      b_2\\
      \vdots\\
      b_n
    \end{bmatrix} = \left(\sum_{i=1}^n a_i \vec{\hat e}_i\right) \dotp \left(\sum_{j=1}^n b_j \vec{\hat e}_j\right).
    \]
    Now by the linearity properties of the dot product,
    \[
    =\sum_{i,j =1}^n a_ib_j(\vec{\hat e}_i \dotp \vec{\hat e}_j)
    \]
    since $\vec{\hat e}_i \dotp \vec{\hat e}_j = 1$ if $i=j$ (because they are
    parallel in this case) and $0$ otherwise (because then they are
    orthogonal),
    \begin{align*}
    &=\sum_{i=1}^n a_ib_i \\
    &=a_1b_1 + a_2b_2 +\dots+a_nb_n.
    \end{align*}
\end{explanation}
\end{theorem}



\end{document}
