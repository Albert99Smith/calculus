\documentclass{ximera}
%\usepackage{todonotes}
%\usepackage{mathtools} %% Required for wide table Curl and Greens
%\usepackage{cuted} %% Required for wide table Curl and Greens
\newcommand{\todo}{}

\usepackage{esint} % for \oiint
\ifxake%%https://math.meta.stackexchange.com/questions/9973/how-do-you-render-a-closed-surface-double-integral
\renewcommand{\oiint}{{\large\bigcirc}\kern-1.56em\iint}
\fi


\graphicspath{
  {./}
  {ximeraTutorial/}
  {basicPhilosophy/}
  {functionsOfSeveralVariables/}
  {normalVectors/}
  {lagrangeMultipliers/}
  {vectorFields/}
  {greensTheorem/}
  {shapeOfThingsToCome/}
  {dotProducts/}
  {partialDerivativesAndTheGradientVector/}
  {../productAndQuotientRules/exercises/}
  {../normalVectors/exercisesParametricPlots/}
  {../continuityOfFunctionsOfSeveralVariables/exercises/}
  {../partialDerivativesAndTheGradientVector/exercises/}
  {../directionalDerivativeAndChainRule/exercises/}
  {../commonCoordinates/exercisesCylindricalCoordinates/}
  {../commonCoordinates/exercisesSphericalCoordinates/}
  {../greensTheorem/exercisesCurlAndLineIntegrals/}
  {../greensTheorem/exercisesDivergenceAndLineIntegrals/}
  {../shapeOfThingsToCome/exercisesDivergenceTheorem/}
  {../greensTheorem/}
  {../shapeOfThingsToCome/}
  {../separableDifferentialEquations/exercises/}
  {vectorFields/}
}

\newcommand{\mooculus}{\textsf{\textbf{MOOC}\textnormal{\textsf{ULUS}}}}

\usepackage{tkz-euclide}\usepackage{tikz}
\usepackage{tikz-cd}
\usetikzlibrary{arrows}
\tikzset{>=stealth,commutative diagrams/.cd,
  arrow style=tikz,diagrams={>=stealth}} %% cool arrow head
\tikzset{shorten <>/.style={ shorten >=#1, shorten <=#1 } } %% allows shorter vectors

\usetikzlibrary{backgrounds} %% for boxes around graphs
\usetikzlibrary{shapes,positioning}  %% Clouds and stars
\usetikzlibrary{matrix} %% for matrix
\usepgfplotslibrary{polar} %% for polar plots
\usepgfplotslibrary{fillbetween} %% to shade area between curves in TikZ
\usetkzobj{all}
\usepackage[makeroom]{cancel} %% for strike outs
%\usepackage{mathtools} %% for pretty underbrace % Breaks Ximera
%\usepackage{multicol}
\usepackage{pgffor} %% required for integral for loops



%% http://tex.stackexchange.com/questions/66490/drawing-a-tikz-arc-specifying-the-center
%% Draws beach ball
\tikzset{pics/carc/.style args={#1:#2:#3}{code={\draw[pic actions] (#1:#3) arc(#1:#2:#3);}}}



\usepackage{array}
\setlength{\extrarowheight}{+.1cm}
\newdimen\digitwidth
\settowidth\digitwidth{9}
\def\divrule#1#2{
\noalign{\moveright#1\digitwidth
\vbox{\hrule width#2\digitwidth}}}





\newcommand{\RR}{\mathbb R}
\newcommand{\R}{\mathbb R}
\newcommand{\N}{\mathbb N}
\newcommand{\Z}{\mathbb Z}

\newcommand{\sagemath}{\textsf{SageMath}}


%\renewcommand{\d}{\,d\!}
\renewcommand{\d}{\mathop{}\!d}
\newcommand{\dd}[2][]{\frac{\d #1}{\d #2}}
\newcommand{\pp}[2][]{\frac{\partial #1}{\partial #2}}
\renewcommand{\l}{\ell}
\newcommand{\ddx}{\frac{d}{\d x}}

\newcommand{\zeroOverZero}{\ensuremath{\boldsymbol{\tfrac{0}{0}}}}
\newcommand{\inftyOverInfty}{\ensuremath{\boldsymbol{\tfrac{\infty}{\infty}}}}
\newcommand{\zeroOverInfty}{\ensuremath{\boldsymbol{\tfrac{0}{\infty}}}}
\newcommand{\zeroTimesInfty}{\ensuremath{\small\boldsymbol{0\cdot \infty}}}
\newcommand{\inftyMinusInfty}{\ensuremath{\small\boldsymbol{\infty - \infty}}}
\newcommand{\oneToInfty}{\ensuremath{\boldsymbol{1^\infty}}}
\newcommand{\zeroToZero}{\ensuremath{\boldsymbol{0^0}}}
\newcommand{\inftyToZero}{\ensuremath{\boldsymbol{\infty^0}}}



\newcommand{\numOverZero}{\ensuremath{\boldsymbol{\tfrac{\#}{0}}}}
\newcommand{\dfn}{\textbf}
%\newcommand{\unit}{\,\mathrm}
\newcommand{\unit}{\mathop{}\!\mathrm}
\newcommand{\eval}[1]{\bigg[ #1 \bigg]}
\newcommand{\seq}[1]{\left( #1 \right)}
\renewcommand{\epsilon}{\varepsilon}
\renewcommand{\phi}{\varphi}


\renewcommand{\iff}{\Leftrightarrow}

\DeclareMathOperator{\arccot}{arccot}
\DeclareMathOperator{\arcsec}{arcsec}
\DeclareMathOperator{\arccsc}{arccsc}
\DeclareMathOperator{\si}{Si}
\DeclareMathOperator{\scal}{scal}
\DeclareMathOperator{\sign}{sign}


%% \newcommand{\tightoverset}[2]{% for arrow vec
%%   \mathop{#2}\limits^{\vbox to -.5ex{\kern-0.75ex\hbox{$#1$}\vss}}}
\newcommand{\arrowvec}[1]{{\overset{\rightharpoonup}{#1}}}
%\renewcommand{\vec}[1]{\arrowvec{\mathbf{#1}}}
\renewcommand{\vec}[1]{{\overset{\boldsymbol{\rightharpoonup}}{\mathbf{#1}}}\hspace{0in}}

\newcommand{\point}[1]{\left(#1\right)} %this allows \vector{ to be changed to \vector{ with a quick find and replace
\newcommand{\pt}[1]{\mathbf{#1}} %this allows \vec{ to be changed to \vec{ with a quick find and replace
\newcommand{\Lim}[2]{\lim_{\point{#1} \to \point{#2}}} %Bart, I changed this to point since I want to use it.  It runs through both of the exercise and exerciseE files in limits section, which is why it was in each document to start with.

\DeclareMathOperator{\proj}{\mathbf{proj}}
\newcommand{\veci}{{\boldsymbol{\hat{\imath}}}}
\newcommand{\vecj}{{\boldsymbol{\hat{\jmath}}}}
\newcommand{\veck}{{\boldsymbol{\hat{k}}}}
\newcommand{\vecl}{\vec{\boldsymbol{\l}}}
\newcommand{\uvec}[1]{\mathbf{\hat{#1}}}
\newcommand{\utan}{\mathbf{\hat{t}}}
\newcommand{\unormal}{\mathbf{\hat{n}}}
\newcommand{\ubinormal}{\mathbf{\hat{b}}}

\newcommand{\dotp}{\bullet}
\newcommand{\cross}{\boldsymbol\times}
\newcommand{\grad}{\boldsymbol\nabla}
\newcommand{\divergence}{\grad\dotp}
\newcommand{\curl}{\grad\cross}
%\DeclareMathOperator{\divergence}{divergence}
%\DeclareMathOperator{\curl}[1]{\grad\cross #1}
\newcommand{\lto}{\mathop{\longrightarrow\,}\limits}

\renewcommand{\bar}{\overline}

\colorlet{textColor}{black}
\colorlet{background}{white}
\colorlet{penColor}{blue!50!black} % Color of a curve in a plot
\colorlet{penColor2}{red!50!black}% Color of a curve in a plot
\colorlet{penColor3}{red!50!blue} % Color of a curve in a plot
\colorlet{penColor4}{green!50!black} % Color of a curve in a plot
\colorlet{penColor5}{orange!80!black} % Color of a curve in a plot
\colorlet{penColor6}{yellow!70!black} % Color of a curve in a plot
\colorlet{fill1}{penColor!20} % Color of fill in a plot
\colorlet{fill2}{penColor2!20} % Color of fill in a plot
\colorlet{fillp}{fill1} % Color of positive area
\colorlet{filln}{penColor2!20} % Color of negative area
\colorlet{fill3}{penColor3!20} % Fill
\colorlet{fill4}{penColor4!20} % Fill
\colorlet{fill5}{penColor5!20} % Fill
\colorlet{gridColor}{gray!50} % Color of grid in a plot

\newcommand{\surfaceColor}{violet}
\newcommand{\surfaceColorTwo}{redyellow}
\newcommand{\sliceColor}{greenyellow}




\pgfmathdeclarefunction{gauss}{2}{% gives gaussian
  \pgfmathparse{1/(#2*sqrt(2*pi))*exp(-((x-#1)^2)/(2*#2^2))}%
}


%%%%%%%%%%%%%
%% Vectors
%%%%%%%%%%%%%

%% Simple horiz vectors
\renewcommand{\vector}[1]{\left\langle #1\right\rangle}


%% %% Complex Horiz Vectors with angle brackets
%% \makeatletter
%% \renewcommand{\vector}[2][ , ]{\left\langle%
%%   \def\nextitem{\def\nextitem{#1}}%
%%   \@for \el:=#2\do{\nextitem\el}\right\rangle%
%% }
%% \makeatother

%% %% Vertical Vectors
%% \def\vector#1{\begin{bmatrix}\vecListA#1,,\end{bmatrix}}
%% \def\vecListA#1,{\if,#1,\else #1\cr \expandafter \vecListA \fi}

%%%%%%%%%%%%%
%% End of vectors
%%%%%%%%%%%%%

%\newcommand{\fullwidth}{}
%\newcommand{\normalwidth}{}



%% makes a snazzy t-chart for evaluating functions
%\newenvironment{tchart}{\rowcolors{2}{}{background!90!textColor}\array}{\endarray}

%%This is to help with formatting on future title pages.
\newenvironment{sectionOutcomes}{}{}



%% Flowchart stuff
%\tikzstyle{startstop} = [rectangle, rounded corners, minimum width=3cm, minimum height=1cm,text centered, draw=black]
%\tikzstyle{question} = [rectangle, minimum width=3cm, minimum height=1cm, text centered, draw=black]
%\tikzstyle{decision} = [trapezium, trapezium left angle=70, trapezium right angle=110, minimum width=3cm, minimum height=1cm, text centered, draw=black]
%\tikzstyle{question} = [rectangle, rounded corners, minimum width=3cm, minimum height=1cm,text centered, draw=black]
%\tikzstyle{process} = [rectangle, minimum width=3cm, minimum height=1cm, text centered, draw=black]
%\tikzstyle{decision} = [trapezium, trapezium left angle=70, trapezium right angle=110, minimum width=3cm, minimum height=1cm, text centered, draw=black]

\author{Jim Talamo and Alex Beckwith}
\license{Creative Commons 3.0 By-NC}
\outcome{Set up a volume integral using the Washer Method}
\begin{document}
\begin{exercise}

	Let $R$ be the region in the $xy$-plane bounded by $y=0$, $y=\ln x$, $y=2$, and $x=1$. This exercise will walk you through setting up an integral using the Shell Method that will give the volume of the solid generated when $R$ is revolved about the line $x=-1$.
            \begin{image}
            \begin{tikzpicture}
            	\begin{axis}[
            		domain=-2.5:8.5, ymax=2.5,xmax=8.4, ymin=-0.5, xmin=-2.4,
            		axis lines =center, xlabel=$x$, ylabel=$y$,
            		every axis y label/.style={at=(current axis.above origin),anchor=south},
            		every axis x label/.style={at=(current axis.right of origin),anchor=west},
            		axis on top,
            		]
                      
            	\addplot [draw=penColor,very thick,smooth] {2};
            	\addplot [draw=penColor2,very thick,smooth] {ln(x)};
		\addplot [draw=penColor3,very thick,smooth] {0};
		\addplot [draw=penColor4,very thick,smooth] coordinates {(1,-4)(1,12)};
		\addplot [draw=penColor5,very thick,dotted] coordinates {(-1,-0.5)(-1,2.5)};
                       
            	\addplot [name path=A,domain=1:7.4,draw=none] {2};   
            	\addplot [name path=B,domain=1:7.4,draw=none] {ln(x)};
            	\addplot [fillp] fill between[of=A and B];
	                
            	\node at (axis cs:5,1.2) [penColor2] {$y=\ln(x)$};
		\node at (axis cs:4.5,2.15) [penColor] {$y=2$};
		\node at (axis cs:1.8,2.3) [penColor4] {$x=1$};
            	\end{axis}
            \end{tikzpicture}
            \end{image}

Since we are using the Shell Method, the slices must be:
\begin{multipleChoice}
\choice[correct]{parallel}
\choice{perpendicular}
\end{multipleChoice}
to the axis of rotation.  

Slices that are parallel to the axis of rotation $x=-1$ are:
\begin{multipleChoice}
\choice[correct]{vertical}
\choice{horizontal}
\end{multipleChoice}

Since the slices are vertical, we must: 
\begin{multipleChoice}
\choice[correct]{integrate with respect to $x$.}
\choice{integrate with respect to $y$.}
\end{multipleChoice}

Since we must integrate with respect to $x$, we will use the result:

\[V = \int_{x=a}^{x=b}2\pi \rho h \d x \]

to set up the volume.  We must now find the limits of integration as express the radius $\rho$ and the height $h$ in terms of the variable of integration $x$. 

\begin{exercise}
The limits of integration are: $a= \answer{1}$ and $b = \answer{e^2}$. 
\end{exercise}

\begin{exercise}

We thus have a helpful version of the picture of the region $R$ below:

  \begin{image}
            \begin{tikzpicture}
            	\begin{axis}[
            		domain=-2.5:8.5, ymax=2.5,xmax=8.4, ymin=-0.5, xmin=-2.4,
            		axis lines =center, xlabel=$x$, ylabel=$y$,
            		every axis y label/.style={at=(current axis.above origin),anchor=south},
            		every axis x label/.style={at=(current axis.right of origin),anchor=west},
            		axis on top,
            		]
                      
            	\addplot [draw=penColor,very thick,smooth] {2};
            	\addplot [draw=penColor2,very thick,smooth] {ln(x)};
		\addplot [draw=penColor3,very thick,smooth] {0};
		\addplot [draw=penColor4,very thick,smooth] coordinates {(1,-4)(1,12)};
		\addplot [draw=penColor5,very thick,dotted] coordinates {(-1,-0.5)(-1,2.5)};
                       
            	\addplot [name path=A,domain=1:7.4,draw=none] {2};   
            	\addplot [name path=B,domain=1:7.4,draw=none] {ln(x)};
            	\addplot [fillp] fill between[of=A and B];
	                
            	\node at (axis cs:3,.5) [penColor2] {$y=\ln(x)$};
		\node at (axis cs:4.5,2.15) [penColor] {$y=2$};
		\node at (axis cs:1.8,2.3) [penColor4] {$x=1$};
	
		\addplot [draw=penColor, fill = gray!50] plot coordinates {(2.7, 1.1) (3,1.1) (3,2) (2.7,2) (2.7, 1.1)};
          
          %Draw R and r
          \addplot [draw=black!30!red,very thick] coordinates {(-1,1.5)(2.7,1.5)};
          \node at (axis cs:1.7,1.35) [black!30!red] {$\rho$};
          

	 \draw[decoration={brace,raise=.1cm,mirror},decorate,thin] (axis cs:3.05,1.1)--(axis cs:3.05,2);

	 \node at (axis cs:3.6,1.55)  [black!30!blue]  {$h$};
                      
                  	\end{axis}
            \end{tikzpicture}
  \end{image}
            
 We see from the picture that $\rho$ is a:
 \begin{multipleChoice}
 \choice{vertical distance}
 \choice[correct]{horizontal distance}
 \end{multipleChoice}           
            
\begin{exercise}
Since $\rho$ is the distance from the axis of rotation to the slice, and this is a horizontal distance, we find $\rho = x_{right}-x_{left}$.
\begin{multipleChoice}
 \choice[correct]{$x_{right} = x$}
 \choice{$x_{right} = \ln(x)$}
  \choice{$x_{right} = -1$}
\end{multipleChoice}       

\begin{multipleChoice}
 \choice{$x_{left} = \ln(x)$}
 \choice{$x_{left} = 1$}
  \choice[correct]{$x_{left} = -1$}
\end{multipleChoice}   

\begin{hint}
Remember that the process of find the volume requires us to express the volume of each shell in terms of the $x$-value where the shell is located.  An arbitrary shell is located at $x$!
\end{hint}

So, $\rho= \answer{x-(-1)}$.
 \end{exercise}
 
  We see from the picture that $h$ is a:
 \begin{multipleChoice}
 \choice[correct]{vertical distance}
 \choice{horizontal distance}
 \end{multipleChoice}           
 
 \begin{exercise}
Since $h$ is the height of the slice, and this is a vertical distance, we find $h = y_{top}-y_{bot}$.
\begin{multipleChoice}
 \choice[correct]{$y_{top} = 2$}
 \choice{$y_{top} = \ln(x)$}
\end{multipleChoice}       

\begin{multipleChoice}
 \choice{$y_{bot} =2$}
 \choice[correct]{$y_{bot} =  \ln(x)$}
\end{multipleChoice}   

So, $h= \answer{2-\ln(x)}$.
 \end{exercise}
           
\begin{exercise}

Using \[V = \int_{x=a}^{x=b} 2\pi \rho h \d x, \] we find that an integral that gives the volume of the solid of revolution is:            
	\[
	V= \int_{x=\answer{1}}^{x=\answer{e^2}}
	\answer{2\pi(x+1)(2-\ln(x))}\d x
	\]
\end{exercise}
\end{exercise}
\end{exercise}

\end{document}