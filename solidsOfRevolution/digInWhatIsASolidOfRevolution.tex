\documentclass{ximera}

%\usepackage{todonotes}
%\usepackage{mathtools} %% Required for wide table Curl and Greens
%\usepackage{cuted} %% Required for wide table Curl and Greens
\newcommand{\todo}{}

\usepackage{esint} % for \oiint
\ifxake%%https://math.meta.stackexchange.com/questions/9973/how-do-you-render-a-closed-surface-double-integral
\renewcommand{\oiint}{{\large\bigcirc}\kern-1.56em\iint}
\fi


\graphicspath{
  {./}
  {ximeraTutorial/}
  {basicPhilosophy/}
  {functionsOfSeveralVariables/}
  {normalVectors/}
  {lagrangeMultipliers/}
  {vectorFields/}
  {greensTheorem/}
  {shapeOfThingsToCome/}
  {dotProducts/}
  {partialDerivativesAndTheGradientVector/}
  {../productAndQuotientRules/exercises/}
  {../normalVectors/exercisesParametricPlots/}
  {../continuityOfFunctionsOfSeveralVariables/exercises/}
  {../partialDerivativesAndTheGradientVector/exercises/}
  {../directionalDerivativeAndChainRule/exercises/}
  {../commonCoordinates/exercisesCylindricalCoordinates/}
  {../commonCoordinates/exercisesSphericalCoordinates/}
  {../greensTheorem/exercisesCurlAndLineIntegrals/}
  {../greensTheorem/exercisesDivergenceAndLineIntegrals/}
  {../shapeOfThingsToCome/exercisesDivergenceTheorem/}
  {../greensTheorem/}
  {../shapeOfThingsToCome/}
  {../separableDifferentialEquations/exercises/}
  {vectorFields/}
}

\newcommand{\mooculus}{\textsf{\textbf{MOOC}\textnormal{\textsf{ULUS}}}}

\usepackage{tkz-euclide}\usepackage{tikz}
\usepackage{tikz-cd}
\usetikzlibrary{arrows}
\tikzset{>=stealth,commutative diagrams/.cd,
  arrow style=tikz,diagrams={>=stealth}} %% cool arrow head
\tikzset{shorten <>/.style={ shorten >=#1, shorten <=#1 } } %% allows shorter vectors

\usetikzlibrary{backgrounds} %% for boxes around graphs
\usetikzlibrary{shapes,positioning}  %% Clouds and stars
\usetikzlibrary{matrix} %% for matrix
\usepgfplotslibrary{polar} %% for polar plots
\usepgfplotslibrary{fillbetween} %% to shade area between curves in TikZ
\usetkzobj{all}
\usepackage[makeroom]{cancel} %% for strike outs
%\usepackage{mathtools} %% for pretty underbrace % Breaks Ximera
%\usepackage{multicol}
\usepackage{pgffor} %% required for integral for loops



%% http://tex.stackexchange.com/questions/66490/drawing-a-tikz-arc-specifying-the-center
%% Draws beach ball
\tikzset{pics/carc/.style args={#1:#2:#3}{code={\draw[pic actions] (#1:#3) arc(#1:#2:#3);}}}



\usepackage{array}
\setlength{\extrarowheight}{+.1cm}
\newdimen\digitwidth
\settowidth\digitwidth{9}
\def\divrule#1#2{
\noalign{\moveright#1\digitwidth
\vbox{\hrule width#2\digitwidth}}}





\newcommand{\RR}{\mathbb R}
\newcommand{\R}{\mathbb R}
\newcommand{\N}{\mathbb N}
\newcommand{\Z}{\mathbb Z}

\newcommand{\sagemath}{\textsf{SageMath}}


%\renewcommand{\d}{\,d\!}
\renewcommand{\d}{\mathop{}\!d}
\newcommand{\dd}[2][]{\frac{\d #1}{\d #2}}
\newcommand{\pp}[2][]{\frac{\partial #1}{\partial #2}}
\renewcommand{\l}{\ell}
\newcommand{\ddx}{\frac{d}{\d x}}

\newcommand{\zeroOverZero}{\ensuremath{\boldsymbol{\tfrac{0}{0}}}}
\newcommand{\inftyOverInfty}{\ensuremath{\boldsymbol{\tfrac{\infty}{\infty}}}}
\newcommand{\zeroOverInfty}{\ensuremath{\boldsymbol{\tfrac{0}{\infty}}}}
\newcommand{\zeroTimesInfty}{\ensuremath{\small\boldsymbol{0\cdot \infty}}}
\newcommand{\inftyMinusInfty}{\ensuremath{\small\boldsymbol{\infty - \infty}}}
\newcommand{\oneToInfty}{\ensuremath{\boldsymbol{1^\infty}}}
\newcommand{\zeroToZero}{\ensuremath{\boldsymbol{0^0}}}
\newcommand{\inftyToZero}{\ensuremath{\boldsymbol{\infty^0}}}



\newcommand{\numOverZero}{\ensuremath{\boldsymbol{\tfrac{\#}{0}}}}
\newcommand{\dfn}{\textbf}
%\newcommand{\unit}{\,\mathrm}
\newcommand{\unit}{\mathop{}\!\mathrm}
\newcommand{\eval}[1]{\bigg[ #1 \bigg]}
\newcommand{\seq}[1]{\left( #1 \right)}
\renewcommand{\epsilon}{\varepsilon}
\renewcommand{\phi}{\varphi}


\renewcommand{\iff}{\Leftrightarrow}

\DeclareMathOperator{\arccot}{arccot}
\DeclareMathOperator{\arcsec}{arcsec}
\DeclareMathOperator{\arccsc}{arccsc}
\DeclareMathOperator{\si}{Si}
\DeclareMathOperator{\scal}{scal}
\DeclareMathOperator{\sign}{sign}


%% \newcommand{\tightoverset}[2]{% for arrow vec
%%   \mathop{#2}\limits^{\vbox to -.5ex{\kern-0.75ex\hbox{$#1$}\vss}}}
\newcommand{\arrowvec}[1]{{\overset{\rightharpoonup}{#1}}}
%\renewcommand{\vec}[1]{\arrowvec{\mathbf{#1}}}
\renewcommand{\vec}[1]{{\overset{\boldsymbol{\rightharpoonup}}{\mathbf{#1}}}\hspace{0in}}

\newcommand{\point}[1]{\left(#1\right)} %this allows \vector{ to be changed to \vector{ with a quick find and replace
\newcommand{\pt}[1]{\mathbf{#1}} %this allows \vec{ to be changed to \vec{ with a quick find and replace
\newcommand{\Lim}[2]{\lim_{\point{#1} \to \point{#2}}} %Bart, I changed this to point since I want to use it.  It runs through both of the exercise and exerciseE files in limits section, which is why it was in each document to start with.

\DeclareMathOperator{\proj}{\mathbf{proj}}
\newcommand{\veci}{{\boldsymbol{\hat{\imath}}}}
\newcommand{\vecj}{{\boldsymbol{\hat{\jmath}}}}
\newcommand{\veck}{{\boldsymbol{\hat{k}}}}
\newcommand{\vecl}{\vec{\boldsymbol{\l}}}
\newcommand{\uvec}[1]{\mathbf{\hat{#1}}}
\newcommand{\utan}{\mathbf{\hat{t}}}
\newcommand{\unormal}{\mathbf{\hat{n}}}
\newcommand{\ubinormal}{\mathbf{\hat{b}}}

\newcommand{\dotp}{\bullet}
\newcommand{\cross}{\boldsymbol\times}
\newcommand{\grad}{\boldsymbol\nabla}
\newcommand{\divergence}{\grad\dotp}
\newcommand{\curl}{\grad\cross}
%\DeclareMathOperator{\divergence}{divergence}
%\DeclareMathOperator{\curl}[1]{\grad\cross #1}
\newcommand{\lto}{\mathop{\longrightarrow\,}\limits}

\renewcommand{\bar}{\overline}

\colorlet{textColor}{black}
\colorlet{background}{white}
\colorlet{penColor}{blue!50!black} % Color of a curve in a plot
\colorlet{penColor2}{red!50!black}% Color of a curve in a plot
\colorlet{penColor3}{red!50!blue} % Color of a curve in a plot
\colorlet{penColor4}{green!50!black} % Color of a curve in a plot
\colorlet{penColor5}{orange!80!black} % Color of a curve in a plot
\colorlet{penColor6}{yellow!70!black} % Color of a curve in a plot
\colorlet{fill1}{penColor!20} % Color of fill in a plot
\colorlet{fill2}{penColor2!20} % Color of fill in a plot
\colorlet{fillp}{fill1} % Color of positive area
\colorlet{filln}{penColor2!20} % Color of negative area
\colorlet{fill3}{penColor3!20} % Fill
\colorlet{fill4}{penColor4!20} % Fill
\colorlet{fill5}{penColor5!20} % Fill
\colorlet{gridColor}{gray!50} % Color of grid in a plot

\newcommand{\surfaceColor}{violet}
\newcommand{\surfaceColorTwo}{redyellow}
\newcommand{\sliceColor}{greenyellow}




\pgfmathdeclarefunction{gauss}{2}{% gives gaussian
  \pgfmathparse{1/(#2*sqrt(2*pi))*exp(-((x-#1)^2)/(2*#2^2))}%
}


%%%%%%%%%%%%%
%% Vectors
%%%%%%%%%%%%%

%% Simple horiz vectors
\renewcommand{\vector}[1]{\left\langle #1\right\rangle}


%% %% Complex Horiz Vectors with angle brackets
%% \makeatletter
%% \renewcommand{\vector}[2][ , ]{\left\langle%
%%   \def\nextitem{\def\nextitem{#1}}%
%%   \@for \el:=#2\do{\nextitem\el}\right\rangle%
%% }
%% \makeatother

%% %% Vertical Vectors
%% \def\vector#1{\begin{bmatrix}\vecListA#1,,\end{bmatrix}}
%% \def\vecListA#1,{\if,#1,\else #1\cr \expandafter \vecListA \fi}

%%%%%%%%%%%%%
%% End of vectors
%%%%%%%%%%%%%

%\newcommand{\fullwidth}{}
%\newcommand{\normalwidth}{}



%% makes a snazzy t-chart for evaluating functions
%\newenvironment{tchart}{\rowcolors{2}{}{background!90!textColor}\array}{\endarray}

%%This is to help with formatting on future title pages.
\newenvironment{sectionOutcomes}{}{}



%% Flowchart stuff
%\tikzstyle{startstop} = [rectangle, rounded corners, minimum width=3cm, minimum height=1cm,text centered, draw=black]
%\tikzstyle{question} = [rectangle, minimum width=3cm, minimum height=1cm, text centered, draw=black]
%\tikzstyle{decision} = [trapezium, trapezium left angle=70, trapezium right angle=110, minimum width=3cm, minimum height=1cm, text centered, draw=black]
%\tikzstyle{question} = [rectangle, rounded corners, minimum width=3cm, minimum height=1cm,text centered, draw=black]
%\tikzstyle{process} = [rectangle, minimum width=3cm, minimum height=1cm, text centered, draw=black]
%\tikzstyle{decision} = [trapezium, trapezium left angle=70, trapezium right angle=110, minimum width=3cm, minimum height=1cm, text centered, draw=black]

\author{Jim Talamo}

%%I like to capitalize Washer Method and Shell Method.  As a result, reference to these methods appears with that convention in those section titles
\outcome{Understand what a solid of revolution is and the two ways to generate one.}

\title[Dig-In:]{What is a solid of revolution?}

\begin{document}
\begin{abstract}
  We define a solid of revolution and discuss how to find the volume of one in two different ways.
\end{abstract}
\maketitle


\section{Solids of revolution}\index{solid of revolution}
\index{volume of a solid of revolution}

Given a region $R$ in the $xy$-plane, we built solids by stacking ``slabs" with given cross sections on top of $R$.  Another way to generate a solid from the region $R$ is to revolve it about a vertical or horizontal axis of revolution.  A solid generated this way is often called a \emph{solid of revolution}.  We will be interested in computing the volume of such solids.  Before we dive into the details, we first study two ways we can generate such a solid. 



\begin{model}
Consider a cone with a height of $4$ units and a base radius of $2$ units.  We can superimpose the cone on a coordinate system, as shown below. 

 \begin{image}
            \begin{tikzpicture}
            	\begin{axis}[
            		domain=-10:10, ymax=4.8,xmax=2.8, ymin=-.8, xmin=-2.8,
            		axis lines =center, xlabel=$x$, ylabel=$y$,
            		every axis y label/.style={at=(current axis.above origin),anchor=south},
            		every axis x label/.style={at=(current axis.right of origin),anchor=west},
            		axis on top,
            		]
                      
            	\addplot [draw=penColor,domain=0:2,very thick,smooth] {4-2*x};
		\addplot [draw=penColor,domain=-2:0,very thick,smooth] {4+2*x};
		
		%shades outer part	
		\addplot [name path=A,domain=0:2,draw=none] {4-2*x};
		\addplot [name path=B,domain=-2:-0,draw=none] {4+2*x};
            	\addplot [name path=C,domain=-1.99:1.99,draw=none,samples=75, smooth] {-sqrt(.25- .25/4*x^2)};
	       	\addplot [fillp] fill between[of=A and C];
		\addplot [fillp] fill between[of=B and C];
		
                 %outer ellipses
                   \addplot [draw=penColor,domain=-1.99:1.99,very thick,smooth,samples=100,dashed] {sqrt(.25- .25/4*x^2)};
                  \addplot [draw=penColor,domain=-1.99:1.99,very thick,smooth,samples=100] {-sqrt(.25- .25/4*x^2)};
                  
                  %%%%%%%%%%%%%%%%%%%%                 
%            	\node at (axis cs:11,1.55) [penColor] {$x=4y^2$};
%            	\node at (axis cs:15,-.9) [penColor2] {$x+4y=8$};
	    
	      \end{axis}
            \end{tikzpicture}
            \end{image}

We can imagine this as a solid of revolution as follows.  First, note that the line segment in Quadrant I of the $xy$-plane that joins the points $(0,4)$ and $(2,0)$ has slope $\answer[given]{-2}$.  Using the point-slope form for a line and the point $(2,0)$ gives the equation of this line segment, which holds for $0 \leq x \leq 2$.

\begin{align*}
y-y_0 &= m(x-x_0) \\
y - 0 & = -2(x-2) \\
2x+y &= 4 
\end{align*}

Now, consider the region $R$ in the $xy$-plane bounded by $2x+y=4$, $y=0$, and $x=0$.  

 \begin{image}
            \begin{tikzpicture}
            	\begin{axis}[
            		domain=-10:10, ymax=4.8,xmax=2.8, ymin=-.8, xmin=-2.8,
            		axis lines =center, xlabel=$x$, ylabel=$y$,
            		every axis y label/.style={at=(current axis.above origin),anchor=south},
            		every axis x label/.style={at=(current axis.right of origin),anchor=west},
            		axis on top,
            		]
                      
            	\addplot [draw=penColor,domain=0:2,very thick,smooth] {4-2*x};
		\addplot [draw=penColor,very thick,smooth] coordinates {(0,0)(0,4)};
		\addplot [draw=penColor,very thick,smooth] coordinates {(0,0)(2,0)};
		
		%shades outer part	
		\addplot [name path=A,domain=0:2,draw=none] {4-2*x};
            	\addplot [name path=C,domain=0:2,draw=none] {0};
	       	\addplot [fillp] fill between[of=A and C];
		
                 
                  %%%%%%%%%%%%%%%%%%%%                 
          	\node at (axis cs:2,2) [penColor] {$2x+y=4$};
%            	\node at (axis cs:15,-.9) [penColor2] {$x+4y=8$};
	    
	      \end{axis}
            \end{tikzpicture}
            \end{image}


Note that when we revolve $R$ about the \wordChoice{\choice{$x$-axis}\choice[correct]{$y$-axis}}, we obtain the cone in question.  Since revolving this region about the axis of revolution produced the solid of revolution, we call $R$ in the $xy$-plane the \emph{region of revolution}. 

There are two ways that we can imagine the solid.

\end{model}

\section{Slicing Perpendicular to the axis of revolution}

We can slice the solid perpendicular to the axis of revolution.  The axis of revolution is the $y$-axis, so perpendicular slices are horizontal.  A representative slice is shown on the solid below.

 \begin{image}
            \begin{tikzpicture}
            	\begin{axis}[
            		domain=-10:10, ymax=4.8,xmax=2.8, ymin=-.8, xmin=-2.8,
            		axis lines =center, xlabel=$x$, ylabel=$y$,
            		every axis y label/.style={at=(current axis.above origin),anchor=south},
            		every axis x label/.style={at=(current axis.right of origin),anchor=west},
            		axis on top,
            		]
                      
            	\addplot [draw=penColor,domain=0:2,very thick,smooth] {4-2*x};
		\addplot [draw=penColor,domain=-2:0,very thick,smooth] {4+2*x};
		
		%shades outer part	
		\addplot [name path=A,domain=0:1.99,draw=none] {4-2*x};
		\addplot [name path=B,domain=-1.99:-0,draw=none] {4+2*x};
            	\addplot [name path=C,domain=-1.99:1.99,draw=none,samples=100] {-sqrt(.25- .25/4*x^2)};
	       	\addplot [fillp] fill between[of=A and C];
		\addplot [fillp] fill between[of=B and C];
		
                 %outer ellipses
                   \addplot [draw=penColor,domain=-1.99:1.99,very thick,smooth,samples=100,dashed] {sqrt(.25- .25/4*x^2)};
                  \addplot [draw=penColor,domain=-1.99:1.99,very thick,smooth,samples=100] {-sqrt(.25- .25/4*x^2)};
                  
                 %Slice
                  \addplot [draw=penColor,domain=-1.224:1.224,very thick,smooth,samples=100] {1.5+ sqrt(.1- .1/1.5*x^2)};
                  \addplot [draw=penColor,domain=-1.2649:1.2649,very thick,smooth,samples=100] {1.5- sqrt(.1- .1/1.6*x^2)};
                  \addplot [draw=penColor,domain=-1.341:1.341,very thick,smooth,samples=100] {1.3- sqrt(.1- .1/1.8*x^2)};
                  
                  %Shading
                	\addplot [name path=D,domain=-1.264:1.264,draw=none] {1.5- sqrt(.1- .1/1.6*x^2)};
            	\addplot [name path=E,domain=-1.34:1.34,draw=none,samples=100] {1.3- sqrt(.1- .1/1.8*x^2)};
                	\addplot [name path=F,domain=-1.4:-1.2,draw=none] {4+2*x};	
                	\addplot [name path=G,domain=1.2:1.4,draw=none] {4-2*x};
		
	       	\addplot [gray] fill between[of=D and E];
	       	\addplot [gray] fill between[of=D and F];		
	       	\addplot [gray] fill between[of=D and G];
		                  
                	\addplot [name path=F,domain=-1.22:1.22,draw=none] {1.5+ sqrt(.1- .1/1.5*x^2)};
            	\addplot [name path=G,domain=-1.22:1.22,draw=none,samples=100] {1.5- sqrt(.1- .1/1.5*x^2)};
	       	\addplot [gray!50] fill between[of=F and G];
                  %%%%%%%%%%%%%%%%%%%%                 
%            	\node at (axis cs:11,1.55) [penColor] {$x=4y^2$};
%            	\node at (axis cs:15,-.9) [penColor2] {$x+4y=8$};
	    
	      \end{axis}
            \end{tikzpicture}
            \end{image}
                       
The region in the $xy$-plane that generates the slice once rotated about the $y$-axis is shown as well.            
            
 \begin{image}
            \begin{tikzpicture}
            	\begin{axis}[
            		domain=-10:10, ymax=4.8,xmax=2.8, ymin=-.8, xmin=-2.8,
            		axis lines =center, xlabel=$x$, ylabel=$y$,
            		every axis y label/.style={at=(current axis.above origin),anchor=south},
            		every axis x label/.style={at=(current axis.right of origin),anchor=west},
            		axis on top,
            		]
                      
            	\addplot [draw=penColor,domain=0:2,very thick,smooth] {4-2*x};
		\addplot [draw=penColor,very thick,smooth] coordinates {(0,0)(0,4)};
		\addplot [draw=penColor,very thick,smooth] coordinates {(0,0)(2,0)};
		
		%shades outer part	
		\addplot [name path=A,domain=0:2,draw=none] {4-2*x};
            	\addplot [name path=C,domain=0:2,draw=none] {0};
	       	\addplot [fillp] fill between[of=A and C];
		
		\addplot [draw=penColor,very thick] coordinates {(0,1)(1.5,1)(1.4,1.2)(0,1.2)(0,1)};
                 \addplot [name path=D,domain=0:1.4,draw=none] {1.2};
            	\addplot [name path=E,domain=0:1.5,draw=none] {1};
	       	\addplot [gray] fill between[of=D and E];
		
                  %%%%%%%%%%%%%%%%%%%%                 
          	\node at (axis cs:2,2) [penColor] {$2x+y=4$};
%            	\node at (axis cs:15,-.9) [penColor2] {$x+4y=8$};
	    
	      \end{axis}
            \end{tikzpicture}
            \end{image}
            
Ultimately, we need to approximate the slices by a shape whose volume we can compute.  Note that the slice on the solid here can be approximated by a \emph{disk}.
            
            \begin{image}
            \begin{tikzpicture}
            	\begin{axis}[
            		domain=-10:10, ymax=4.8,xmax=2.8, ymin=-.8, xmin=-2.8,
            		axis lines =center, xlabel=$x$, ylabel=$y$,
            		every axis y label/.style={at=(current axis.above origin),anchor=south},
            		every axis x label/.style={at=(current axis.right of origin),anchor=west},
            		axis on top,
            		]
                      
            	\addplot [draw=penColor,domain=0:2,very thick,smooth] {4-2*x};
		\addplot [draw=penColor,domain=-2:0,very thick,smooth] {4+2*x};
		
		%shades outer part	
		\addplot [name path=A,domain=0:2,draw=none] {4-2*x};
		\addplot [name path=B,domain=-2:-0,draw=none] {4+2*x};
            	\addplot [name path=C,domain=-2:2,draw=none,samples=100] {-sqrt(.25- .25/4*x^2)};
	       	\addplot [fillp] fill between[of=A and C];
		\addplot [fillp] fill between[of=B and C];
		
                 %outer ellipses
                   \addplot [draw=penColor,domain=-2:2,very thick,smooth,samples=100,dashed] {sqrt(.25- .25/4*x^2)};
                  \addplot [draw=penColor,domain=-2:2,very thick,smooth,samples=100] {-sqrt(.25- .25/4*x^2)};
                  
                 %Slice
                  \addplot [draw=penColor,domain=-1.224:1.224,very thick,smooth,samples=100] {1.5+ sqrt(.1- .1/1.5*x^2)};
                  \addplot [draw=penColor,domain=-1.224:1.224,very thick,smooth,samples=100] {1.5- sqrt(.1- .1/1.5*x^2)};
                  \addplot [draw=penColor,domain=-1.224:1.224,very thick,smooth,samples=100] {1.3- sqrt(.1- .1/1.5*x^2)};
                  
       		\addplot [draw=penColor,very thick,smooth] coordinates {(-1.22,1.25)(-1.22,1.55)};
		\addplot [draw=penColor,very thick,smooth] coordinates {(1.22,1.25)(1.22,1.55)};
                  
                  %Shading
                	\addplot [name path=D,domain=-1.224:1.224,draw=none] {1.5- sqrt(.1- .1/1.5*x^2)};
            	\addplot [name path=E,domain=-1.224:1.224,draw=none,samples=100] {1.3- sqrt(.1- .1/1.5*x^2)};
	       	\addplot [gray] fill between[of=D and E];
                  
                	\addplot [name path=F,domain=-1.224:1.224,draw=none] {1.5+ sqrt(.1- .1/1.5*x^2)};
            	\addplot [name path=G,domain=-1.224:1.224,draw=none,samples=100] {1.5- sqrt(.1- .1/1.5*x^2)};
	       	\addplot [gray!50] fill between[of=F and G];
                  %%%%%%%%%%%%%%%%%%%%                 
%            	\node at (axis cs:11,1.55) [penColor] {$x=4y^2$};
%            	\node at (axis cs:15,-.9) [penColor2] {$x+4y=8$};
	    
	      \end{axis}
            \end{tikzpicture}
            \end{image}
            
Note that if we approximate the slice in the region of revolution by a \emph{rectangle}, the disk on the solid is generated by revolving this rectangle about the axis of revolution.            
            
 \begin{image}
            \begin{tikzpicture}
            	\begin{axis}[
            		domain=-10:10, ymax=4.8,xmax=2.8, ymin=-.8, xmin=-2.8,
            		axis lines =center, xlabel=$x$, ylabel=$y$,
            		every axis y label/.style={at=(current axis.above origin),anchor=south},
            		every axis x label/.style={at=(current axis.right of origin),anchor=west},
            		axis on top,
            		]
                      
            	\addplot [draw=penColor,domain=0:2,very thick,smooth] {4-2*x};
		\addplot [draw=penColor,very thick,smooth] coordinates {(0,0)(0,4)};
		\addplot [draw=penColor,very thick,smooth] coordinates {(0,0)(2,0)};
		
		%shades outer part	
		\addplot [name path=A,domain=0:2,draw=none] {4-2*x};
            	\addplot [name path=C,domain=0:2,draw=none] {0};
	       	\addplot [fillp] fill between[of=A and C];
		
		\addplot [draw=penColor,very thick] coordinates {(0,1)(1.4,1)(1.4,1.2)(0,1.2)(0,1)};
                 \addplot [name path=D,domain=0:1.4,draw=none] {1.2};
            	\addplot [name path=E,domain=0:1.4,draw=none] {1};
	       	\addplot [gray] fill between[of=D and E];
		
                  %%%%%%%%%%%%%%%%%%%%                 
          	\node at (axis cs:2,2) [penColor] {$2x+y=4$};
%            	\node at (axis cs:15,-.9) [penColor2] {$x+4y=8$};
	    
	      \end{axis}
            \end{tikzpicture}
            \end{image}
            
Since we can find the volume of a disk, this seems like a fruitful approach.  Before continuing, there are a few important points to summarize.

\begin{itemize}
\item We slice the solid into many pieces, and we can view each slice as being generated by revolving a certain region in the $xy$-plane about the axis of revolution.
\item We can approximate each slice on the solid by a disk, and we can generate this disk by approximating the corresponding slice in the $xy$-plane by a rectangle.  
\item Thus, we can work with the solid of revolution by analyzing the slices in the $xy$-plane.  
\end{itemize}

\begin{remark}
Note here that since one of the sides of the solid lies on the axis of revolution, we obtain a solid disk, but if the axis of revolution does not coincide with the boundary of the region, we instead obtain a disk with a hole in it.

 %ADD PICTURE% 

\end{remark}
 
 We'll explore the details in a bit, but let's first explore another way to generate the solid.


\section{Slicing Parallel to the axis of revolution}            
            
We instead can choose to slice the region of revolution in the $xy$-plane by using slices that are \emph{parallel} to the axis of revolution.  we can approximate each of those slices by a rectangle.        
            
 \begin{image}
            \begin{tikzpicture}
            	\begin{axis}[
            		domain=-10:10, ymax=4.8,xmax=2.8, ymin=-.8, xmin=-2.8,
            		axis lines =center, xlabel=$x$, ylabel=$y$,
            		every axis y label/.style={at=(current axis.above origin),anchor=south},
            		every axis x label/.style={at=(current axis.right of origin),anchor=west},
            		axis on top,
            		]
                      
            	\addplot [draw=penColor,domain=0:2,very thick,smooth] {4-2*x};
		\addplot [draw=penColor,very thick,smooth] coordinates {(0,0)(0,4)};
		\addplot [draw=penColor,very thick,smooth] coordinates {(0,0)(2,0)};
		
		%shades outer part	
		\addplot [name path=A,domain=0:2,draw=none] {4-2*x};
            	\addplot [name path=C,domain=0:2,draw=none] {0};
	       	\addplot [fillp] fill between[of=A and C];
		
		\addplot [draw=penColor,very thick] coordinates {(.6,0)(.8,0)(.8,2.4)(.6,2.4)(.6,0)};
                 \addplot [name path=D,domain=.6:.8,draw=none] {2.4};
            	\addplot [name path=E,domain=.6:.8,draw=none] {0};
	       	\addplot [gray] fill between[of=D and E];
		
                  %%%%%%%%%%%%%%%%%%%%                 
          	\node at (axis cs:2,2) [penColor] {$2x+y=4$};
%            	\node at (axis cs:15,-.9) [penColor2] {$x+4y=8$};
	    
	      \end{axis}
            \end{tikzpicture}
            \end{image}

Revolving these about the axis of revolution gives very thin, hollow cylinders.

            \begin{image}
            \begin{tikzpicture}
            	\begin{axis}[
            		domain=-10:10, ymax=4.8,xmax=2.8, ymin=-.8, xmin=-2.8,
            		axis lines =center, xlabel=$x$, ylabel=$y$,
            		every axis y label/.style={at=(current axis.above origin),anchor=south},
            		every axis x label/.style={at=(current axis.right of origin),anchor=west},
            		axis on top,
            		]
                      
            	\addplot [draw=penColor,domain=0:2,very thick,smooth] {4-2*x};
		\addplot [draw=penColor,domain=-2:0,very thick,smooth] {4+2*x};
		
		%shades outer part	
		\addplot [name path=A,domain=0:1.99,draw=none] {4-2*x};
		\addplot [name path=B,domain=-1.99:-0,draw=none] {4+2*x};
            	\addplot [name path=C,domain=-1.99:1.99,draw=none,samples=100] {-sqrt(.25- .25/4*x^2)};
	       	\addplot [fillp] fill between[of=A and C];
		\addplot [fillp] fill between[of=B and C];
		
                 %outer ellipses
                   \addplot [draw=penColor,domain=-1.99:1.99,very thick,smooth,samples=100,dashed] {sqrt(.25- .25/4*x^2)};
                  \addplot [draw=penColor,domain=-1.99:1.99,very thick,smooth,samples=100] {-sqrt(.25- .25/4*x^2)};
                              
                 %Slice
                  \addplot [draw=penColor,domain=-1.224:1.224,very thick,smooth,samples=100] {1.5+ sqrt(.1- .1/1.5*x^2)};
                  \addplot [draw=penColor,domain=-1.224:1.224,very thick,smooth,samples=100] {1.5- sqrt(.1- .1/1.5*x^2)};
                  
                  \addplot [draw=penColor,domain=-1.05:1.05,very thick,smooth,samples=100] {1.5+ sqrt(.05- .05/1.2*x^2)};
                  \addplot [draw=penColor,domain=-1.05:1.05,very thick,smooth,samples=100] {1.5- sqrt(.05- .05/1.2*x^2)};
                  
                  \addplot [draw=penColor,domain=-1.0954:-1.05,very thick,smooth,samples=100] {1.5+ sqrt(.05- .05/1.2*x^2)};
                  \addplot [draw=penColor,domain=-1.0954:-1.05,very thick,smooth,samples=100] {1.5- sqrt(.05- .05/1.2*x^2)};                  
                                   
                  \addplot [draw=penColor,domain=1.05:1.0954,very thick,smooth,samples=100] {1.5+ sqrt(.05- .05/1.2*x^2)};
                  \addplot [draw=penColor,domain=1.05:1.0954,very thick,smooth,samples=100] {1.5 - sqrt(.05- .05/1.2*x^2)};   
                  
                  \addplot [draw=penColor,domain=-1.224:1.224,very thick,smooth,samples=100] {- sqrt(.1- .1/1.5*x^2)}; 
                                    
                  \addplot [draw=penColor,very thick,smooth] coordinates {(-1.22,-.05)(-1.22,1.55)};
		 \addplot [draw=penColor,very thick,smooth] coordinates {(1.22,-.05)(1.22,1.55)};
                  
                  %Shading
                	\addplot [name path=D,domain=-1.224:1.224,draw=none] {1.5- sqrt(.1- .1/1.5*x^2)};
            	\addplot [name path=E,domain=-1.224:1.224,draw=none,samples=100] {- sqrt(.1- .1/1.5*x^2)};
	       	\addplot [gray] fill between[of=D and E];
                  
                	\addplot [name path=F,domain=-1.224:1.224,draw=none] {1.5+ sqrt(.1- .1/1.5*x^2)};
            	\addplot [name path=G,domain=-1.224:1.224,draw=none,samples=100] {1.5- sqrt(.1- .1/1.5*x^2)};
	       	\addplot [gray!75] fill between[of=F and G];
		
                	\addplot [name path=H,domain=-1.095:1.095,draw=none] {1.5+ sqrt(.05- .05/1.2*x^2)};
            	\addplot [name path=I,domain=-1.095:1.095,draw=none,samples=100] {1.5- sqrt(.05- .05/1.2*x^2)};
	       	\addplot [fillp] fill between[of=H and I];
		
                  %%%%%%%%%%%%%%%%%%%%                 
%            	\node at (axis cs:11,1.55) [penColor] {$x=4y^2$};
%            	\node at (axis cs:15,-.9) [penColor2] {$x+4y=8$};
	    
	      \end{axis}
            \end{tikzpicture}
            \end{image}
            
Now, the solid is built by nesting these hollow cylinders inside of each other, whereas before, when the slices were taken perpendicular to the axis of revolution, we built the solid by stacking the disks on top of each other.   
   
\section{Volumes of Hollow Cylinders}   

We have seen that we can slice the region of revolution perpendicular or parallel to the axis of revolution.  We can then approximate each slice by a rectangle and revolve them about the axis of revolution.  In each case, we obtain a (potentially) hollow cylinder. Let's turn our attention to studying the volume of a hollow cylinder with outer radius $R$, inner radius $r$, and height $h$.  A picture is shown below.

\begin{image} %%%Thanks to Hans Parshall
\begin{tikzpicture}
\node[name=cyl,cylinder,draw,rotate=90,minimum height=6cm,minimum width=6cm,aspect=7] {};
\draw [dashed] (-1.7,-1.7) arc (180:360:1.7 and 0.2);
\draw [dashed] (-1.7,-1.7) arc (180:0:1.7 and 0.2);
\draw [dashed] (-1.7,-1.7) -- (-1.7,2.7);
\draw [dashed] (1.7,-1.7) -- (1.7,2.7);
\draw (-1.7,2.7) arc (180:360:1.7 and 0.2);
\draw (-1.7,2.7) arc (180:0:1.7 and 0.2);

\draw[|-|] (3.5,-1.7) -- (3.5,2.7);
\node at (3.7,0.7) {$h$};

\draw[thick,penColor] (0,2.7) -- (-1.7,2.7);
\node[penColor] at (-.85,3) {$r$};

\draw[thick,penColor2] (0,2.7) -- (1.5,2);
\node[penColor2] at (.5,2.2) {$R$};

\draw[|-|] (3.5,-1.7) -- (3.5,2.7);
\node at (3.7,0.7) {$h$};
\end{tikzpicture}\end{image}

To find the volume of the hollow cylinder, note that:
\begin{align*}
\left< \textrm{Volume of the hollow cylinder} \right>  &=   \\
 \left< \textrm{ Volume of the outer} \right. & \left. \textrm{cylinder } \right> - \left< \textrm{ Volume of the inner cylinder} \right>.
\end{align*}
 
 The volume of the outer cylinder, which has radius $R$ is $V_{outer} = \pi R^2 h$, while the volume of the inner cylinder, which has radius $r$ is $V_{inner} = \pi r^2 h$.  
 
\[
\left< \textrm{Volume of the hollow cylinder} \right> = \pi R^2h-\pi r^2h=\pi(R^2-r^2)h 
\]

Note that this requires only three geometric quantities of interest - an outer radius, and inner radius, and a thickness.  It does not require that we work with a coordinate system.  To use calculus, however, we must work with functions described using a coordinate system.  We thus will continue to imagine solids of revolution by revolving regions in the $xy$-plane about an axis of revolution.  One of the essential skills to find the resulting volumes will be a familiar one; we must express $R$, $r$, and $h$ in terms of our variable of integration.  We explore this in detail in the coming sections.

\end{document}

