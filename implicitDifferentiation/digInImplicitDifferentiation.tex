\documentclass{ximera}

%\usepackage{todonotes}
%\usepackage{mathtools} %% Required for wide table Curl and Greens
%\usepackage{cuted} %% Required for wide table Curl and Greens
\newcommand{\todo}{}

\usepackage{esint} % for \oiint
\ifxake%%https://math.meta.stackexchange.com/questions/9973/how-do-you-render-a-closed-surface-double-integral
\renewcommand{\oiint}{{\large\bigcirc}\kern-1.56em\iint}
\fi


\graphicspath{
  {./}
  {ximeraTutorial/}
  {basicPhilosophy/}
  {functionsOfSeveralVariables/}
  {normalVectors/}
  {lagrangeMultipliers/}
  {vectorFields/}
  {greensTheorem/}
  {shapeOfThingsToCome/}
  {dotProducts/}
  {partialDerivativesAndTheGradientVector/}
  {../productAndQuotientRules/exercises/}
  {../normalVectors/exercisesParametricPlots/}
  {../continuityOfFunctionsOfSeveralVariables/exercises/}
  {../partialDerivativesAndTheGradientVector/exercises/}
  {../directionalDerivativeAndChainRule/exercises/}
  {../commonCoordinates/exercisesCylindricalCoordinates/}
  {../commonCoordinates/exercisesSphericalCoordinates/}
  {../greensTheorem/exercisesCurlAndLineIntegrals/}
  {../greensTheorem/exercisesDivergenceAndLineIntegrals/}
  {../shapeOfThingsToCome/exercisesDivergenceTheorem/}
  {../greensTheorem/}
  {../shapeOfThingsToCome/}
  {../separableDifferentialEquations/exercises/}
  {vectorFields/}
}

\newcommand{\mooculus}{\textsf{\textbf{MOOC}\textnormal{\textsf{ULUS}}}}

\usepackage{tkz-euclide}\usepackage{tikz}
\usepackage{tikz-cd}
\usetikzlibrary{arrows}
\tikzset{>=stealth,commutative diagrams/.cd,
  arrow style=tikz,diagrams={>=stealth}} %% cool arrow head
\tikzset{shorten <>/.style={ shorten >=#1, shorten <=#1 } } %% allows shorter vectors

\usetikzlibrary{backgrounds} %% for boxes around graphs
\usetikzlibrary{shapes,positioning}  %% Clouds and stars
\usetikzlibrary{matrix} %% for matrix
\usepgfplotslibrary{polar} %% for polar plots
\usepgfplotslibrary{fillbetween} %% to shade area between curves in TikZ
\usetkzobj{all}
\usepackage[makeroom]{cancel} %% for strike outs
%\usepackage{mathtools} %% for pretty underbrace % Breaks Ximera
%\usepackage{multicol}
\usepackage{pgffor} %% required for integral for loops



%% http://tex.stackexchange.com/questions/66490/drawing-a-tikz-arc-specifying-the-center
%% Draws beach ball
\tikzset{pics/carc/.style args={#1:#2:#3}{code={\draw[pic actions] (#1:#3) arc(#1:#2:#3);}}}



\usepackage{array}
\setlength{\extrarowheight}{+.1cm}
\newdimen\digitwidth
\settowidth\digitwidth{9}
\def\divrule#1#2{
\noalign{\moveright#1\digitwidth
\vbox{\hrule width#2\digitwidth}}}





\newcommand{\RR}{\mathbb R}
\newcommand{\R}{\mathbb R}
\newcommand{\N}{\mathbb N}
\newcommand{\Z}{\mathbb Z}

\newcommand{\sagemath}{\textsf{SageMath}}


%\renewcommand{\d}{\,d\!}
\renewcommand{\d}{\mathop{}\!d}
\newcommand{\dd}[2][]{\frac{\d #1}{\d #2}}
\newcommand{\pp}[2][]{\frac{\partial #1}{\partial #2}}
\renewcommand{\l}{\ell}
\newcommand{\ddx}{\frac{d}{\d x}}

\newcommand{\zeroOverZero}{\ensuremath{\boldsymbol{\tfrac{0}{0}}}}
\newcommand{\inftyOverInfty}{\ensuremath{\boldsymbol{\tfrac{\infty}{\infty}}}}
\newcommand{\zeroOverInfty}{\ensuremath{\boldsymbol{\tfrac{0}{\infty}}}}
\newcommand{\zeroTimesInfty}{\ensuremath{\small\boldsymbol{0\cdot \infty}}}
\newcommand{\inftyMinusInfty}{\ensuremath{\small\boldsymbol{\infty - \infty}}}
\newcommand{\oneToInfty}{\ensuremath{\boldsymbol{1^\infty}}}
\newcommand{\zeroToZero}{\ensuremath{\boldsymbol{0^0}}}
\newcommand{\inftyToZero}{\ensuremath{\boldsymbol{\infty^0}}}



\newcommand{\numOverZero}{\ensuremath{\boldsymbol{\tfrac{\#}{0}}}}
\newcommand{\dfn}{\textbf}
%\newcommand{\unit}{\,\mathrm}
\newcommand{\unit}{\mathop{}\!\mathrm}
\newcommand{\eval}[1]{\bigg[ #1 \bigg]}
\newcommand{\seq}[1]{\left( #1 \right)}
\renewcommand{\epsilon}{\varepsilon}
\renewcommand{\phi}{\varphi}


\renewcommand{\iff}{\Leftrightarrow}

\DeclareMathOperator{\arccot}{arccot}
\DeclareMathOperator{\arcsec}{arcsec}
\DeclareMathOperator{\arccsc}{arccsc}
\DeclareMathOperator{\si}{Si}
\DeclareMathOperator{\scal}{scal}
\DeclareMathOperator{\sign}{sign}


%% \newcommand{\tightoverset}[2]{% for arrow vec
%%   \mathop{#2}\limits^{\vbox to -.5ex{\kern-0.75ex\hbox{$#1$}\vss}}}
\newcommand{\arrowvec}[1]{{\overset{\rightharpoonup}{#1}}}
%\renewcommand{\vec}[1]{\arrowvec{\mathbf{#1}}}
\renewcommand{\vec}[1]{{\overset{\boldsymbol{\rightharpoonup}}{\mathbf{#1}}}\hspace{0in}}

\newcommand{\point}[1]{\left(#1\right)} %this allows \vector{ to be changed to \vector{ with a quick find and replace
\newcommand{\pt}[1]{\mathbf{#1}} %this allows \vec{ to be changed to \vec{ with a quick find and replace
\newcommand{\Lim}[2]{\lim_{\point{#1} \to \point{#2}}} %Bart, I changed this to point since I want to use it.  It runs through both of the exercise and exerciseE files in limits section, which is why it was in each document to start with.

\DeclareMathOperator{\proj}{\mathbf{proj}}
\newcommand{\veci}{{\boldsymbol{\hat{\imath}}}}
\newcommand{\vecj}{{\boldsymbol{\hat{\jmath}}}}
\newcommand{\veck}{{\boldsymbol{\hat{k}}}}
\newcommand{\vecl}{\vec{\boldsymbol{\l}}}
\newcommand{\uvec}[1]{\mathbf{\hat{#1}}}
\newcommand{\utan}{\mathbf{\hat{t}}}
\newcommand{\unormal}{\mathbf{\hat{n}}}
\newcommand{\ubinormal}{\mathbf{\hat{b}}}

\newcommand{\dotp}{\bullet}
\newcommand{\cross}{\boldsymbol\times}
\newcommand{\grad}{\boldsymbol\nabla}
\newcommand{\divergence}{\grad\dotp}
\newcommand{\curl}{\grad\cross}
%\DeclareMathOperator{\divergence}{divergence}
%\DeclareMathOperator{\curl}[1]{\grad\cross #1}
\newcommand{\lto}{\mathop{\longrightarrow\,}\limits}

\renewcommand{\bar}{\overline}

\colorlet{textColor}{black}
\colorlet{background}{white}
\colorlet{penColor}{blue!50!black} % Color of a curve in a plot
\colorlet{penColor2}{red!50!black}% Color of a curve in a plot
\colorlet{penColor3}{red!50!blue} % Color of a curve in a plot
\colorlet{penColor4}{green!50!black} % Color of a curve in a plot
\colorlet{penColor5}{orange!80!black} % Color of a curve in a plot
\colorlet{penColor6}{yellow!70!black} % Color of a curve in a plot
\colorlet{fill1}{penColor!20} % Color of fill in a plot
\colorlet{fill2}{penColor2!20} % Color of fill in a plot
\colorlet{fillp}{fill1} % Color of positive area
\colorlet{filln}{penColor2!20} % Color of negative area
\colorlet{fill3}{penColor3!20} % Fill
\colorlet{fill4}{penColor4!20} % Fill
\colorlet{fill5}{penColor5!20} % Fill
\colorlet{gridColor}{gray!50} % Color of grid in a plot

\newcommand{\surfaceColor}{violet}
\newcommand{\surfaceColorTwo}{redyellow}
\newcommand{\sliceColor}{greenyellow}




\pgfmathdeclarefunction{gauss}{2}{% gives gaussian
  \pgfmathparse{1/(#2*sqrt(2*pi))*exp(-((x-#1)^2)/(2*#2^2))}%
}


%%%%%%%%%%%%%
%% Vectors
%%%%%%%%%%%%%

%% Simple horiz vectors
\renewcommand{\vector}[1]{\left\langle #1\right\rangle}


%% %% Complex Horiz Vectors with angle brackets
%% \makeatletter
%% \renewcommand{\vector}[2][ , ]{\left\langle%
%%   \def\nextitem{\def\nextitem{#1}}%
%%   \@for \el:=#2\do{\nextitem\el}\right\rangle%
%% }
%% \makeatother

%% %% Vertical Vectors
%% \def\vector#1{\begin{bmatrix}\vecListA#1,,\end{bmatrix}}
%% \def\vecListA#1,{\if,#1,\else #1\cr \expandafter \vecListA \fi}

%%%%%%%%%%%%%
%% End of vectors
%%%%%%%%%%%%%

%\newcommand{\fullwidth}{}
%\newcommand{\normalwidth}{}



%% makes a snazzy t-chart for evaluating functions
%\newenvironment{tchart}{\rowcolors{2}{}{background!90!textColor}\array}{\endarray}

%%This is to help with formatting on future title pages.
\newenvironment{sectionOutcomes}{}{}



%% Flowchart stuff
%\tikzstyle{startstop} = [rectangle, rounded corners, minimum width=3cm, minimum height=1cm,text centered, draw=black]
%\tikzstyle{question} = [rectangle, minimum width=3cm, minimum height=1cm, text centered, draw=black]
%\tikzstyle{decision} = [trapezium, trapezium left angle=70, trapezium right angle=110, minimum width=3cm, minimum height=1cm, text centered, draw=black]
%\tikzstyle{question} = [rectangle, rounded corners, minimum width=3cm, minimum height=1cm,text centered, draw=black]
%\tikzstyle{process} = [rectangle, minimum width=3cm, minimum height=1cm, text centered, draw=black]
%\tikzstyle{decision} = [trapezium, trapezium left angle=70, trapezium right angle=110, minimum width=3cm, minimum height=1cm, text centered, draw=black]


\outcome{Implicitly differentiate expressions.}
\outcome{Find the equation of the tangent line for curves that
  are not plots of functions.}
\outcome{Understand how changing the variable changes how we take
  the derivative.}
\outcome{Understand the derivatives of expressions that are not
  functions or not ``solved for $y$''.}

\title[Dig-In:]{Implicit differentiation}

\begin{document}
\begin{abstract}
In this section we differentiate equations that contain more than one variable on one side.
\end{abstract}
\maketitle
\section{Review of the chain rule}

Implicit differentiation is really just an application of the chain rule.
So recall:

\begin{theorem}[Chain Rule]\index{chain rule}\index{derivative rules!chain}
If $f(x)$ and $g(x)$ are differentiable, then
\[
\ddx f(g(x)) = f'(g(x))g'(x).
\]
\end{theorem}

Of particular use in this section is the following.  
If $y$ is a differentiable function of $x$ and if $f$ is a differentiable function, then
\[
\ddx (f(y)) = f'(y) \cdot \ddx (y) = f'(y) \dd[y]{x}.
\]


\section{Implicit differentiation}

The functions we've been dealing with so far have been
\textit{explicit functions}\index{explicit function}, meaning that the
dependent variable is written in terms of the independent
variable. For example:
\[
y=3x^2-2x+1,\qquad y=e^{3x}, \qquad y = \frac{x-2}{x^2-3x+2}.
\]
However, there is another type of function, called an \textit{implicit
  function}. In this case, the dependent variable is not stated
explicitly in terms of the independent variable. Some examples are:
\[
x^2+y^2 = 4,\qquad x^3+y^3 = 9xy, \qquad x^4+3x^2 = x^{2/3}+y^{2/3} + 1.
\]
Your inclination might be simply to solve each of these for $y$ and go
merrily on your way. However this can be difficult and it may require
two \textit{branches}, for example to explicitly plot $x^2+y^2 = 4$,
one needs both $y= \sqrt{4-x^2}$ and $y=-\sqrt{4-x^2}$. Moreover, it
may not even be possible to solve for $y$. To deal with such
situations, we use \index{implicit differentiation}\textit{implicit
  differentiation}. We'll start with a basic example.

\begin{example}
Consider the curve defined by:
\[
x^2 + y^2 = 1
\]
\begin{enumerate}
\item Compute $\dd[y]{x}$.
\item Find the slope of the tangent line at $\left(\frac{\sqrt{2}}{2},\frac{\sqrt{2}}{2}\right)$.
\end{enumerate}
\begin{explanation}
  Starting with 
\[
x^2 + y^2 = 1
\]
we differentiate both sides of the
equation with respect to $x$ to obtain
\[
\ddx \left(x^2+y^2\right) = \ddx 1.
\]
Applying the sum rule we see
\[
\ddx x^2+\ddx y^2 = 0.
\]
Let's examine each of these terms in turn. To start
\[
\ddx x^2 = \answer[given]{2x}.
\]
On the other hand, $\ddx y^2$ is somewhat different. Here you imagine that $y = f(x)$, and hence by the chain rule
\begin{align*}
\ddx y^2 &= \ddx (f(x))^2 \\ 
&= 2\cdot f(x) \cdot f'(x) \\
&= 2y\dd[y]{x}.
\end{align*}
Putting this together we are left with the equation
\[
2x + 2y\dd[y]{x} =0
\]
At this point, we solve for $\dd[y]{x}$. Write
\begin{align*}
2x + 2y\dd[y]{x} &= 0\\
2y \dd[y]{x} &= -2x\\
\dd[y]{x} &= \answer[given]{\frac{-x}{y}}.
\end{align*}

For the second part of the problem, we simply plug $x=\frac{\sqrt{2}}{2}$ and $y=\frac{\sqrt{2}}{2}$
into the formula above, hence the slope of the tangent line at this point 
is $\answer[given]{-1}$.
\begin{onlineOnly}
  We can confirm our results by looking at the graph of the curve and
  our tangent line:
  \[
  \graph{x^2+y^2=1,-x+\sqrt{2}}
  \]
\end{onlineOnly}
\end{explanation}
\end{example}

Let's see another illustrative example:


\begin{example}
Consider the curve defined by:
\[
x^3+y^3 = 9xy
\]
\begin{image}
\begin{tikzpicture}
	\begin{axis}[
            xmin=-6,xmax=6,ymin=-6,ymax=6,
            axis lines=center,
            xlabel=$x$, ylabel=$y$,
            every axis y label/.style={at=(current axis.above origin),anchor=south},
            every axis x label/.style={at=(current axis.right of origin),anchor=west},
          ]        
          \addplot [very thick, penColor, smooth, samples=100, domain=(-.99:0)] ({9*x/(1+x^3)},{9*x^2/(1+x^3)});
          \addplot [very thick, penColor, smooth, samples=100, domain=(-.99:0)] ({9*x^2/(1+x^3)},{9*x/(1+x^3)});
          \addplot [very thick, penColor, smooth, samples=100, domain=(0:1)] ({9*x/(1+x^3)},{9*x^2/(1+x^3)});
          \addplot [very thick, penColor, smooth, samples=100, domain=(0:1)] ({9*x^2/(1+x^3)},{9*x/(1+x^3)});
        \end{axis}
\end{tikzpicture}
%% \caption{A plot of $x^3+y^3 = 9xy$. While this is not a function of
%%   $y$ in terms of $x$, the equation still defines a relation between
%%   $x$ and $y$.}
\end{image}
\begin{enumerate}
\item Compute $\dd[y]{x}$.
\item Find the slope of the tangent line at $(4,2)$.
\end{enumerate}



\begin{explanation}
Starting with 
\[
x^3+y^3 = 9xy,
\]
we differentiate both sides of the
equation with respect to $x$ to obtain
\[
\ddx \left(x^3+y^3\right) = \ddx 9xy.
\]
Applying the sum rule we see
\[
\ddx x^3+\ddx y^3 = \ddx 9xy.
\]
Let's examine each of these terms in turn. To start
\[
\ddx x^3 = \answer[given]{3x^2}.
\]
On the other hand $\ddx y^3$ is somewhat different. Here you imagine that $y = f(x)$, and hence by the chain rule
\begin{align*}
\ddx y^3 &= \ddx (f(x))^3 \\ 
&= 3(f(x))^2 \cdot f'(x) \\
&= 3y^2\dd[y]{x}.
\end{align*}
Considering the final term $\ddx 9xy$, we again imagine that $y=f(x)$. Hence 
\begin{align*}
\ddx 9xy &= 9\ddx \Bigl(x\cdot f(x)\Bigr) \\
&= 9 \left(x\cdot f'(x) + f(x)\right)\\
&= 9x \dd[y]{x} + 9y.
\end{align*}
Putting this all together we are left with the equation
\[
3x^2 + 3y^2\dd[y]{x} =9x \dd[y]{x} + 9y.
\]
At this point, we solve for $\dd[y]{x}$. Write
\begin{align*}
3x^2 + 3y^2\dd[y]{x} &= 9x \dd[y]{x} + 9y\\
3y^2\dd[y]{x} -  9x \dd[y]{x} &= 9y - 3x^2\\
\dd[y]{x}\left(3y^2-9x\right)&= 9y - 3x^2\\
\dd[y]{x} &=\frac{9y - 3x^2}{3y^2-9x} = \frac{3y - x^2}{y^2-3x}.
\end{align*}

For the second part of the problem, we simply plug $x=4$ and $y=2$
into the formula above, hence the slope of the tangent line at $(4,2)$
is $\frac{5}{4}$. We've included a plot for your viewing pleasure:
\begin{image}
\begin{tikzpicture}
	\begin{axis}[
            xmin=-6,xmax=6,ymin=-6,ymax=6,
            axis lines=center,
            xlabel=$x$, ylabel=$y$,
            every axis y label/.style={at=(current axis.above origin),anchor=south},
            every axis x label/.style={at=(current axis.right of origin),anchor=west},
          ]        
          \addplot [very thick, penColor, smooth, samples=100, domain=(-.99:0)] ({9*x/(1+x^3)},{9*x^2/(1+x^3)});
          \addplot [very thick, penColor, smooth, samples=100, domain=(-.99:0)] ({9*x^2/(1+x^3)},{9*x/(1+x^3)});
          \addplot [very thick, penColor, smooth, samples=100, domain=(0:1)] ({9*x/(1+x^3)},{9*x^2/(1+x^3)});
          \addplot [very thick, penColor, smooth, samples=100, domain=(0:1)] ({9*x^2/(1+x^3)},{9*x/(1+x^3)});

          \addplot [very thick, penColor2, domain=(-6:6)] {(5/4)*(x-4)+2};

          \addplot[color=penColor,fill=penColor,only marks,mark=*] coordinates{(4,2)};  %% closed hole          
        \end{axis}
\end{tikzpicture}
%% \caption{A plot of $x^3+y^3 = 9xy$ along with the tangent line at
%%   $(4,2)$.}
%% \label{plot:x^3+y^3=9xy}
\end{image}
\end{explanation}

\end{example}


You might think that the step in which we solve for $\dd[y]{x}$ could
sometimes be difficult. In fact, \textit{this never happens}. All
occurrences $\dd[y]{x}$ arise from applying the chain rule, and
whenever the chain rule is used it deposits a single $\dd[y]{x}$
multiplied by some other expression. Hence our expression is linear in
$\dd[y]{x}$, it will always be possible to group the terms containing
$\dd[y]{x}$ together and factor out the $\dd[y]{x}$, just as in the
previous examples.

One more last example:

\begin{example}
Consider the curve defined by
\[
\cos(xy) - \frac{y}{x} = 4x^2 y^3.
\]
Compute $\dd[x]{y}$.
\begin{explanation}
First, notice that the problem asks for $\dd[x]{y}$, \textbf{not}
$\dd[y]{x}$.  So we are considering $x$ as a function of $y$. This
means the variables have changed places!  Not to worry, everything is
exactly the same.  We apply $\dd{y}$ to both sides of the equation to
get
\[
\dd{y} \left( \cos(xy) - \frac{y}{x} \right) = \dd{y} (4x^2 y^3)
\]
which gives us
\[
-\answer[given]{\sin(xy)} \left(y \dd[x]{y} + x \right) - \frac{x - y \dd[x]{y}}{x^2}
= 8xy^3 \dd[x]{y} + 12 x^2 y^2.
\]
Distributing and multiplying by $x^2$ yields
\begin{align*}
  -x^2 y \sin(xy) \dd[x]{y} &- x^3 \sin(xy) - x + y \dd[x]{y}\\
  &= 8x^3y^3 \dd[x]{y} + 12x^4y^2.
\end{align*}
Grouping terms, factoring, and dividing finally gives us
\begin{align*}
  -x^2 y \sin(xy) \dd[x]{y} &+ y \dd[x]{y} - 8x^3y^3 \dd[x]{y} \\
  &= x^3 \sin(xy) + x + 12x^4 y^2
\end{align*}
so,
\[
\left( y - x^2y\sin(xy) - 8x^3 y^3 \right) \dd[x]{y} = x^3 \sin(xy) + x + 12x^4 y^2 
\]
and now we see
\[
\dd[x]{y} = \frac{x^3 \sin(xy) + x + 12x^4 y^2}{y - x^2y\sin(xy) - 8x^3 y^3}.
\]
\end{explanation}
\end{example}

\end{document}
