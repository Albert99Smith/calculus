\documentclass{ximera}

%\usepackage{todonotes}
%\usepackage{mathtools} %% Required for wide table Curl and Greens
%\usepackage{cuted} %% Required for wide table Curl and Greens
\newcommand{\todo}{}

\usepackage{esint} % for \oiint
\ifxake%%https://math.meta.stackexchange.com/questions/9973/how-do-you-render-a-closed-surface-double-integral
\renewcommand{\oiint}{{\large\bigcirc}\kern-1.56em\iint}
\fi


\graphicspath{
  {./}
  {ximeraTutorial/}
  {basicPhilosophy/}
  {functionsOfSeveralVariables/}
  {normalVectors/}
  {lagrangeMultipliers/}
  {vectorFields/}
  {greensTheorem/}
  {shapeOfThingsToCome/}
  {dotProducts/}
  {partialDerivativesAndTheGradientVector/}
  {../productAndQuotientRules/exercises/}
  {../normalVectors/exercisesParametricPlots/}
  {../continuityOfFunctionsOfSeveralVariables/exercises/}
  {../partialDerivativesAndTheGradientVector/exercises/}
  {../directionalDerivativeAndChainRule/exercises/}
  {../commonCoordinates/exercisesCylindricalCoordinates/}
  {../commonCoordinates/exercisesSphericalCoordinates/}
  {../greensTheorem/exercisesCurlAndLineIntegrals/}
  {../greensTheorem/exercisesDivergenceAndLineIntegrals/}
  {../shapeOfThingsToCome/exercisesDivergenceTheorem/}
  {../greensTheorem/}
  {../shapeOfThingsToCome/}
  {../separableDifferentialEquations/exercises/}
  {vectorFields/}
}

\newcommand{\mooculus}{\textsf{\textbf{MOOC}\textnormal{\textsf{ULUS}}}}

\usepackage{tkz-euclide}\usepackage{tikz}
\usepackage{tikz-cd}
\usetikzlibrary{arrows}
\tikzset{>=stealth,commutative diagrams/.cd,
  arrow style=tikz,diagrams={>=stealth}} %% cool arrow head
\tikzset{shorten <>/.style={ shorten >=#1, shorten <=#1 } } %% allows shorter vectors

\usetikzlibrary{backgrounds} %% for boxes around graphs
\usetikzlibrary{shapes,positioning}  %% Clouds and stars
\usetikzlibrary{matrix} %% for matrix
\usepgfplotslibrary{polar} %% for polar plots
\usepgfplotslibrary{fillbetween} %% to shade area between curves in TikZ
\usetkzobj{all}
\usepackage[makeroom]{cancel} %% for strike outs
%\usepackage{mathtools} %% for pretty underbrace % Breaks Ximera
%\usepackage{multicol}
\usepackage{pgffor} %% required for integral for loops



%% http://tex.stackexchange.com/questions/66490/drawing-a-tikz-arc-specifying-the-center
%% Draws beach ball
\tikzset{pics/carc/.style args={#1:#2:#3}{code={\draw[pic actions] (#1:#3) arc(#1:#2:#3);}}}



\usepackage{array}
\setlength{\extrarowheight}{+.1cm}
\newdimen\digitwidth
\settowidth\digitwidth{9}
\def\divrule#1#2{
\noalign{\moveright#1\digitwidth
\vbox{\hrule width#2\digitwidth}}}





\newcommand{\RR}{\mathbb R}
\newcommand{\R}{\mathbb R}
\newcommand{\N}{\mathbb N}
\newcommand{\Z}{\mathbb Z}

\newcommand{\sagemath}{\textsf{SageMath}}


%\renewcommand{\d}{\,d\!}
\renewcommand{\d}{\mathop{}\!d}
\newcommand{\dd}[2][]{\frac{\d #1}{\d #2}}
\newcommand{\pp}[2][]{\frac{\partial #1}{\partial #2}}
\renewcommand{\l}{\ell}
\newcommand{\ddx}{\frac{d}{\d x}}

\newcommand{\zeroOverZero}{\ensuremath{\boldsymbol{\tfrac{0}{0}}}}
\newcommand{\inftyOverInfty}{\ensuremath{\boldsymbol{\tfrac{\infty}{\infty}}}}
\newcommand{\zeroOverInfty}{\ensuremath{\boldsymbol{\tfrac{0}{\infty}}}}
\newcommand{\zeroTimesInfty}{\ensuremath{\small\boldsymbol{0\cdot \infty}}}
\newcommand{\inftyMinusInfty}{\ensuremath{\small\boldsymbol{\infty - \infty}}}
\newcommand{\oneToInfty}{\ensuremath{\boldsymbol{1^\infty}}}
\newcommand{\zeroToZero}{\ensuremath{\boldsymbol{0^0}}}
\newcommand{\inftyToZero}{\ensuremath{\boldsymbol{\infty^0}}}



\newcommand{\numOverZero}{\ensuremath{\boldsymbol{\tfrac{\#}{0}}}}
\newcommand{\dfn}{\textbf}
%\newcommand{\unit}{\,\mathrm}
\newcommand{\unit}{\mathop{}\!\mathrm}
\newcommand{\eval}[1]{\bigg[ #1 \bigg]}
\newcommand{\seq}[1]{\left( #1 \right)}
\renewcommand{\epsilon}{\varepsilon}
\renewcommand{\phi}{\varphi}


\renewcommand{\iff}{\Leftrightarrow}

\DeclareMathOperator{\arccot}{arccot}
\DeclareMathOperator{\arcsec}{arcsec}
\DeclareMathOperator{\arccsc}{arccsc}
\DeclareMathOperator{\si}{Si}
\DeclareMathOperator{\scal}{scal}
\DeclareMathOperator{\sign}{sign}


%% \newcommand{\tightoverset}[2]{% for arrow vec
%%   \mathop{#2}\limits^{\vbox to -.5ex{\kern-0.75ex\hbox{$#1$}\vss}}}
\newcommand{\arrowvec}[1]{{\overset{\rightharpoonup}{#1}}}
%\renewcommand{\vec}[1]{\arrowvec{\mathbf{#1}}}
\renewcommand{\vec}[1]{{\overset{\boldsymbol{\rightharpoonup}}{\mathbf{#1}}}\hspace{0in}}

\newcommand{\point}[1]{\left(#1\right)} %this allows \vector{ to be changed to \vector{ with a quick find and replace
\newcommand{\pt}[1]{\mathbf{#1}} %this allows \vec{ to be changed to \vec{ with a quick find and replace
\newcommand{\Lim}[2]{\lim_{\point{#1} \to \point{#2}}} %Bart, I changed this to point since I want to use it.  It runs through both of the exercise and exerciseE files in limits section, which is why it was in each document to start with.

\DeclareMathOperator{\proj}{\mathbf{proj}}
\newcommand{\veci}{{\boldsymbol{\hat{\imath}}}}
\newcommand{\vecj}{{\boldsymbol{\hat{\jmath}}}}
\newcommand{\veck}{{\boldsymbol{\hat{k}}}}
\newcommand{\vecl}{\vec{\boldsymbol{\l}}}
\newcommand{\uvec}[1]{\mathbf{\hat{#1}}}
\newcommand{\utan}{\mathbf{\hat{t}}}
\newcommand{\unormal}{\mathbf{\hat{n}}}
\newcommand{\ubinormal}{\mathbf{\hat{b}}}

\newcommand{\dotp}{\bullet}
\newcommand{\cross}{\boldsymbol\times}
\newcommand{\grad}{\boldsymbol\nabla}
\newcommand{\divergence}{\grad\dotp}
\newcommand{\curl}{\grad\cross}
%\DeclareMathOperator{\divergence}{divergence}
%\DeclareMathOperator{\curl}[1]{\grad\cross #1}
\newcommand{\lto}{\mathop{\longrightarrow\,}\limits}

\renewcommand{\bar}{\overline}

\colorlet{textColor}{black}
\colorlet{background}{white}
\colorlet{penColor}{blue!50!black} % Color of a curve in a plot
\colorlet{penColor2}{red!50!black}% Color of a curve in a plot
\colorlet{penColor3}{red!50!blue} % Color of a curve in a plot
\colorlet{penColor4}{green!50!black} % Color of a curve in a plot
\colorlet{penColor5}{orange!80!black} % Color of a curve in a plot
\colorlet{penColor6}{yellow!70!black} % Color of a curve in a plot
\colorlet{fill1}{penColor!20} % Color of fill in a plot
\colorlet{fill2}{penColor2!20} % Color of fill in a plot
\colorlet{fillp}{fill1} % Color of positive area
\colorlet{filln}{penColor2!20} % Color of negative area
\colorlet{fill3}{penColor3!20} % Fill
\colorlet{fill4}{penColor4!20} % Fill
\colorlet{fill5}{penColor5!20} % Fill
\colorlet{gridColor}{gray!50} % Color of grid in a plot

\newcommand{\surfaceColor}{violet}
\newcommand{\surfaceColorTwo}{redyellow}
\newcommand{\sliceColor}{greenyellow}




\pgfmathdeclarefunction{gauss}{2}{% gives gaussian
  \pgfmathparse{1/(#2*sqrt(2*pi))*exp(-((x-#1)^2)/(2*#2^2))}%
}


%%%%%%%%%%%%%
%% Vectors
%%%%%%%%%%%%%

%% Simple horiz vectors
\renewcommand{\vector}[1]{\left\langle #1\right\rangle}


%% %% Complex Horiz Vectors with angle brackets
%% \makeatletter
%% \renewcommand{\vector}[2][ , ]{\left\langle%
%%   \def\nextitem{\def\nextitem{#1}}%
%%   \@for \el:=#2\do{\nextitem\el}\right\rangle%
%% }
%% \makeatother

%% %% Vertical Vectors
%% \def\vector#1{\begin{bmatrix}\vecListA#1,,\end{bmatrix}}
%% \def\vecListA#1,{\if,#1,\else #1\cr \expandafter \vecListA \fi}

%%%%%%%%%%%%%
%% End of vectors
%%%%%%%%%%%%%

%\newcommand{\fullwidth}{}
%\newcommand{\normalwidth}{}



%% makes a snazzy t-chart for evaluating functions
%\newenvironment{tchart}{\rowcolors{2}{}{background!90!textColor}\array}{\endarray}

%%This is to help with formatting on future title pages.
\newenvironment{sectionOutcomes}{}{}



%% Flowchart stuff
%\tikzstyle{startstop} = [rectangle, rounded corners, minimum width=3cm, minimum height=1cm,text centered, draw=black]
%\tikzstyle{question} = [rectangle, minimum width=3cm, minimum height=1cm, text centered, draw=black]
%\tikzstyle{decision} = [trapezium, trapezium left angle=70, trapezium right angle=110, minimum width=3cm, minimum height=1cm, text centered, draw=black]
%\tikzstyle{question} = [rectangle, rounded corners, minimum width=3cm, minimum height=1cm,text centered, draw=black]
%\tikzstyle{process} = [rectangle, minimum width=3cm, minimum height=1cm, text centered, draw=black]
%\tikzstyle{decision} = [trapezium, trapezium left angle=70, trapezium right angle=110, minimum width=3cm, minimum height=1cm, text centered, draw=black]


\author{Jim Talamo}
\license{CC-By-SA-NC}

\outcome{Use Lagrange multipliers to solve constrained optimization
  problems.}

\begin{document}
\begin{exercise}
Consider the function $F(x,y) = x^2+y^2-2x+4y$ defined on $R = \left\{ (x,y) \in \R^2 ~ \bigg| ~ x^2+y^2 \leq 9 \right\}$.  Then, $F(x,y)$ \wordChoice{\choice[correct]{attains}\choice{does not attain}} an absolute minimum and an absolute maximum on $R$ since 

\begin{multipleChoice}
\choice{$F(x,y)$ is continuous on $R$.}
\choice{$R$ is closed and bounded.}
\choice[correct]{$F(x,y)$ is continuous on $R$ and $R$ is closed and bounded.}
\end{multipleChoice}

Where could the absolute extrema occur?

\begin{selectAll}
\choice{at all of the critical points of $F(x,y)$}
\choice[correct]{at the critical points of $F(x,y)$ that occur in the region $R$.}
\choice[correct]{on the boundary of $R$.}
\end{selectAll}

\begin{exercise}
To find the critical points on $F(x,y)$ that lie within $R$, we compute the gradient.

\[
\grad{F}(x,y) = \vector{\answer{2x-2},\answer{2y+4}}
\]

Hence, the critical point of $F(x,y)$ is $\left(\answer{1},\answer{-2}\right)$.

\begin{exercise}
To determine if this critical point lies in $R$, we must check if $x^2+y^2 \leq 9$.  When $x=1$ and $y=-2$, we have that $x^2+y^2 = \answer{5}$, so the critical point $(1,-2)$ \wordChoice{\choice[correct]{lies}\choice{does not lie}} in $R$ and hence it \wordChoice{\choice[correct]{is}\choice{is not}} a candidate for the location of an absolute extrema.

Note that we do not have to use the Second Derivative Test to classify the extrema; we only need to determine the value $F(x,y)$ takes at this critical point and compare it to the values that $F(x,y)$ takes at the other candidates for absolute extrema (which occur on the boundary).  To this end, we note that

\[
F(1,-2) = \answer{-5}.
\]

\end{exercise}
\end{exercise}

\begin{exercise}
We now must determine the extreme values $F(x,y)$ attains on the boundary.  We can do this using two methods.

\begin{exercise}
\textbf{Method 1: Parameterize the boundary}

The boundary is the circle $x^2+y^2 =9$, so a good choice of parameterization is $\vec{r}(t) = \vector{3 \cos(t), 3 \sin(t)}$ for $0 \leq t \leq 2\pi$.  Along the boundary, we thus have $F(x,y) = F(3 \cos(t),3 \sin(t)) = \answer{9-6 \cos(t) + 12 \sin(t)}$. 

\begin{exercise}
We thus must find the absolute extrema for $f(t) = 9-6 \cos(t) + 12 \sin(t)$ for $0 \leq t \leq 2 \pi$.  Note that $f'(t) = \answer{6 \sin(t) +12 \cos(t)}$, so $f'(t) = 0$ when $\tan(t) = \answer{-2}$.

\begin{itemize}
\item The maximum value $F(x,y)$ takes on the boundary is $ \answer{9+\frac{30}{\sqrt{5}}}$.
\item The maximum value $F(x,y)$ takes on the boundary is $ \answer{9-\frac{30}{\sqrt{5}}}$.
\end{itemize}

\begin{hint}
This is hard to work with in general, but we can draw triangles in Quadrants II and IV.

The triangle in Quadrant I is shown below.
\begin{image}
\resizebox {6cm} {!} { 
\begin{tikzpicture}
    \begin{axis}[
        xmin=-2.5,xmax=.5,ymin=-.5,ymax=2.5,
            axis lines=center,
            ticks=none,
            unit vector ratio*=1 1 1,
            xlabel=$x$, ylabel=$y$,
            ytick={-2,-1,...,7},
	    %yticklabels={$0.5$,$1$,$1.5$,$2$},
	    xtick={-2,-1,...,10},
	    %xticklabels={$0.5$,$1$,$1.5$,$2$},
	    grid = major,
            every axis y label/.style={at=(current axis.above origin),anchor=south},
            every axis x label/.style={at=(current axis.right of origin),anchor=west},
          ]

          \addplot[very thick,penColor] plot coordinates {(0,0) (-1,0)(-1,2)(0,0)};

            \node[below] at (axis cs:-.6, -.1) [penColor] {\Large $-1$};
           \node[left] at (axis cs: -1.2, 1 ) [penColor] {\Large $2$};
           \node[above right] at (axis cs:-.5, 1 ) [penColor] {\Large $\sqrt{5}$};
                      \node[left] at (axis cs:-.2, .2 ) [penColor] {\Large $t$};
    \end{axis}
\end{tikzpicture}}
\end{image}

Here, we find that $\cos(t) = \answer{\frac{-1}{\sqrt{5}}}$ and $\sin(t) = \answer{\frac{2}{\sqrt{5}}}$.  

Since $F(x,y) = F(3 \cos(t),3 \sin(t)) = 9-6 \cos(t) + 12 \sin(t)$ along the boundary, we use these values for $\cos(t)$ and $\sin(t)$ to find that $F(x,y) = \answer{9+\frac{6}{\sqrt{5}}+\frac{24}{\sqrt{5}}}$ at the critical point in Quadrant II. 

A similar analysis for the triangle in Quadrant IV shows that $F(x,y) = \answer{9-\frac{6}{\sqrt{5}}-\frac{24}{\sqrt{5}}}$ at the critical point in Quadrant IV.

Finally, we note that the endpoints $t=0$ and $t=2\pi$ could yield absolute extrema, so we check $F(x,y)$ there as well and find that when $t=0$, $F(x(t),y(t)) = \answer{3}$ and when $t=2 \pi$, $F(x(t),y(t)) = \answer{3}$.

\end{hint}

Thus, by considering the absolute extrema on the boundary as well as the extrema that occurs within the interior of $R$, we find the following.

\begin{itemize}
\item The absolute maximum value of $F(x,y)$ over $R$ is $\answer{ 9 +6\sqrt{5} }$ and occurs when $(x,y) = \left( \answer{\frac{-3}{\sqrt{5}}},  \answer{\frac{6}{\sqrt{5}}} \right) $.
\item The absolute minimum value of $F(x,y)$ over $R$ is $\answer{ -5 }$ and occurs when $(x,y) = \left(\answer{1},\answer{-2}\right)$.
\end{itemize}

\end{exercise}

\end{exercise}

\begin{exercise}
\textbf{Method 2: Use Lagrange multipliers by writing the boundary as a constraint.}

By setting $G(x,y) = x^2+y^2$, we recognize the boundary as the level curve defined by $G(x,y) = 9$.  We thus compute

\begin{itemize}
\item $\grad{F}(x,y) = \vector{\answer{2x-2},\answer{2y+4}}$.
\item $\grad{G}(x,y) =\vector{\answer{2x},\answer{2y}}$. 
\end{itemize}

and must solve the system of equations that arises from the condition $\grad{F}(x,y) = \lambda \grad{G}(x,y)$ and the original constraint.

\begin{align}
2x-2 & = 2\lambda x \\
2y+4 & = 2\lambda y \\
x^2+y^2 &= 9
\end{align}

Note that it looks like a variable could ``cancel'' in the top two equations. If we do this, we might lose potential solutions if the quantity we would like to divide by is $0$, so we will instead cross-multiply the top equations to obtain

\begin{align*}
\lambda xy-\lambda y &= \lambda xy+2\lambda x \\
-\lambda y &=+2\lambda x \\
\lambda \left( \answer{2x+y} \right) &= 0 
\end{align*}


Note that if $\lambda = 0$, Eqn (1) tells us that $x=\answer{1}$ and Eqn (2) tells us that $y=-2$, which corresponds to our earlier critical point.  Since this point lies in the interior of the region and not on the boundary, we may disregard it.

The interesting case occurs when $2x+y= 0$.  We may thus take $y = \answer{-2x}$ and substitute this into the constraint equation $x^2+y^2 = 9$ to find that we have two critical points,  $y= -\answer{\frac{6}{\sqrt{5}}}$ and $y= \answer{\frac{6}{\sqrt{5}}}$.  The candidates for critical points are thus $(x,y) = \left(  \answer{\frac{-3}{\sqrt{5}}}, \frac{6}{\sqrt{5}}\right)$ and $(x,y) = \left(  \answer{\frac{3}{\sqrt{5}}}, -\frac{6}{\sqrt{5}}\right)$.

\begin{exercise}
We now check $F(x,y)$ at each of these critical points and find

\begin{itemize}
\item $F\left(\frac{3}{\sqrt{5}}, - \frac{6}{\sqrt{5}} \right) = \answer{9-6 \sqrt{5}}$.
\item $F\left(-\frac{3}{\sqrt{5}},  \frac{6}{\sqrt{5}} \right) = \answer{9+6 \sqrt{5}}$.
\end{itemize}

\end{exercise}

Thus, by considering the absolute extrema on the boundary as well as the extrema that occurs within the interior of $R$, we find the following.

\begin{itemize}
\item The absolute maximum value of $F(x,y)$ over $R$ is $\answer{ 9 +6\sqrt{5} }$ and occurs when $(x,y) = \left( \answer{\frac{-3}{\sqrt{5}}},  \answer{\frac{6}{\sqrt{5}}} \right) $.
\item The absolute minimum value of $F(x,y)$ over $R$ is $\answer{ -5 }$ and occurs when $(x,y) = \left(\answer{1},\answer{-2}\right)$.
\end{itemize}
\end{exercise}


\end{exercise}


\end{exercise}
\end{document}
