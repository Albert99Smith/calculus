\documentclass{ximera}

%\usepackage{todonotes}
%\usepackage{mathtools} %% Required for wide table Curl and Greens
%\usepackage{cuted} %% Required for wide table Curl and Greens
\newcommand{\todo}{}

\usepackage{esint} % for \oiint
\ifxake%%https://math.meta.stackexchange.com/questions/9973/how-do-you-render-a-closed-surface-double-integral
\renewcommand{\oiint}{{\large\bigcirc}\kern-1.56em\iint}
\fi


\graphicspath{
  {./}
  {ximeraTutorial/}
  {basicPhilosophy/}
  {functionsOfSeveralVariables/}
  {normalVectors/}
  {lagrangeMultipliers/}
  {vectorFields/}
  {greensTheorem/}
  {shapeOfThingsToCome/}
  {dotProducts/}
  {partialDerivativesAndTheGradientVector/}
  {../productAndQuotientRules/exercises/}
  {../normalVectors/exercisesParametricPlots/}
  {../continuityOfFunctionsOfSeveralVariables/exercises/}
  {../partialDerivativesAndTheGradientVector/exercises/}
  {../directionalDerivativeAndChainRule/exercises/}
  {../commonCoordinates/exercisesCylindricalCoordinates/}
  {../commonCoordinates/exercisesSphericalCoordinates/}
  {../greensTheorem/exercisesCurlAndLineIntegrals/}
  {../greensTheorem/exercisesDivergenceAndLineIntegrals/}
  {../shapeOfThingsToCome/exercisesDivergenceTheorem/}
  {../greensTheorem/}
  {../shapeOfThingsToCome/}
  {../separableDifferentialEquations/exercises/}
  {vectorFields/}
}

\newcommand{\mooculus}{\textsf{\textbf{MOOC}\textnormal{\textsf{ULUS}}}}

\usepackage{tkz-euclide}\usepackage{tikz}
\usepackage{tikz-cd}
\usetikzlibrary{arrows}
\tikzset{>=stealth,commutative diagrams/.cd,
  arrow style=tikz,diagrams={>=stealth}} %% cool arrow head
\tikzset{shorten <>/.style={ shorten >=#1, shorten <=#1 } } %% allows shorter vectors

\usetikzlibrary{backgrounds} %% for boxes around graphs
\usetikzlibrary{shapes,positioning}  %% Clouds and stars
\usetikzlibrary{matrix} %% for matrix
\usepgfplotslibrary{polar} %% for polar plots
\usepgfplotslibrary{fillbetween} %% to shade area between curves in TikZ
\usetkzobj{all}
\usepackage[makeroom]{cancel} %% for strike outs
%\usepackage{mathtools} %% for pretty underbrace % Breaks Ximera
%\usepackage{multicol}
\usepackage{pgffor} %% required for integral for loops



%% http://tex.stackexchange.com/questions/66490/drawing-a-tikz-arc-specifying-the-center
%% Draws beach ball
\tikzset{pics/carc/.style args={#1:#2:#3}{code={\draw[pic actions] (#1:#3) arc(#1:#2:#3);}}}



\usepackage{array}
\setlength{\extrarowheight}{+.1cm}
\newdimen\digitwidth
\settowidth\digitwidth{9}
\def\divrule#1#2{
\noalign{\moveright#1\digitwidth
\vbox{\hrule width#2\digitwidth}}}





\newcommand{\RR}{\mathbb R}
\newcommand{\R}{\mathbb R}
\newcommand{\N}{\mathbb N}
\newcommand{\Z}{\mathbb Z}

\newcommand{\sagemath}{\textsf{SageMath}}


%\renewcommand{\d}{\,d\!}
\renewcommand{\d}{\mathop{}\!d}
\newcommand{\dd}[2][]{\frac{\d #1}{\d #2}}
\newcommand{\pp}[2][]{\frac{\partial #1}{\partial #2}}
\renewcommand{\l}{\ell}
\newcommand{\ddx}{\frac{d}{\d x}}

\newcommand{\zeroOverZero}{\ensuremath{\boldsymbol{\tfrac{0}{0}}}}
\newcommand{\inftyOverInfty}{\ensuremath{\boldsymbol{\tfrac{\infty}{\infty}}}}
\newcommand{\zeroOverInfty}{\ensuremath{\boldsymbol{\tfrac{0}{\infty}}}}
\newcommand{\zeroTimesInfty}{\ensuremath{\small\boldsymbol{0\cdot \infty}}}
\newcommand{\inftyMinusInfty}{\ensuremath{\small\boldsymbol{\infty - \infty}}}
\newcommand{\oneToInfty}{\ensuremath{\boldsymbol{1^\infty}}}
\newcommand{\zeroToZero}{\ensuremath{\boldsymbol{0^0}}}
\newcommand{\inftyToZero}{\ensuremath{\boldsymbol{\infty^0}}}



\newcommand{\numOverZero}{\ensuremath{\boldsymbol{\tfrac{\#}{0}}}}
\newcommand{\dfn}{\textbf}
%\newcommand{\unit}{\,\mathrm}
\newcommand{\unit}{\mathop{}\!\mathrm}
\newcommand{\eval}[1]{\bigg[ #1 \bigg]}
\newcommand{\seq}[1]{\left( #1 \right)}
\renewcommand{\epsilon}{\varepsilon}
\renewcommand{\phi}{\varphi}


\renewcommand{\iff}{\Leftrightarrow}

\DeclareMathOperator{\arccot}{arccot}
\DeclareMathOperator{\arcsec}{arcsec}
\DeclareMathOperator{\arccsc}{arccsc}
\DeclareMathOperator{\si}{Si}
\DeclareMathOperator{\scal}{scal}
\DeclareMathOperator{\sign}{sign}


%% \newcommand{\tightoverset}[2]{% for arrow vec
%%   \mathop{#2}\limits^{\vbox to -.5ex{\kern-0.75ex\hbox{$#1$}\vss}}}
\newcommand{\arrowvec}[1]{{\overset{\rightharpoonup}{#1}}}
%\renewcommand{\vec}[1]{\arrowvec{\mathbf{#1}}}
\renewcommand{\vec}[1]{{\overset{\boldsymbol{\rightharpoonup}}{\mathbf{#1}}}\hspace{0in}}

\newcommand{\point}[1]{\left(#1\right)} %this allows \vector{ to be changed to \vector{ with a quick find and replace
\newcommand{\pt}[1]{\mathbf{#1}} %this allows \vec{ to be changed to \vec{ with a quick find and replace
\newcommand{\Lim}[2]{\lim_{\point{#1} \to \point{#2}}} %Bart, I changed this to point since I want to use it.  It runs through both of the exercise and exerciseE files in limits section, which is why it was in each document to start with.

\DeclareMathOperator{\proj}{\mathbf{proj}}
\newcommand{\veci}{{\boldsymbol{\hat{\imath}}}}
\newcommand{\vecj}{{\boldsymbol{\hat{\jmath}}}}
\newcommand{\veck}{{\boldsymbol{\hat{k}}}}
\newcommand{\vecl}{\vec{\boldsymbol{\l}}}
\newcommand{\uvec}[1]{\mathbf{\hat{#1}}}
\newcommand{\utan}{\mathbf{\hat{t}}}
\newcommand{\unormal}{\mathbf{\hat{n}}}
\newcommand{\ubinormal}{\mathbf{\hat{b}}}

\newcommand{\dotp}{\bullet}
\newcommand{\cross}{\boldsymbol\times}
\newcommand{\grad}{\boldsymbol\nabla}
\newcommand{\divergence}{\grad\dotp}
\newcommand{\curl}{\grad\cross}
%\DeclareMathOperator{\divergence}{divergence}
%\DeclareMathOperator{\curl}[1]{\grad\cross #1}
\newcommand{\lto}{\mathop{\longrightarrow\,}\limits}

\renewcommand{\bar}{\overline}

\colorlet{textColor}{black}
\colorlet{background}{white}
\colorlet{penColor}{blue!50!black} % Color of a curve in a plot
\colorlet{penColor2}{red!50!black}% Color of a curve in a plot
\colorlet{penColor3}{red!50!blue} % Color of a curve in a plot
\colorlet{penColor4}{green!50!black} % Color of a curve in a plot
\colorlet{penColor5}{orange!80!black} % Color of a curve in a plot
\colorlet{penColor6}{yellow!70!black} % Color of a curve in a plot
\colorlet{fill1}{penColor!20} % Color of fill in a plot
\colorlet{fill2}{penColor2!20} % Color of fill in a plot
\colorlet{fillp}{fill1} % Color of positive area
\colorlet{filln}{penColor2!20} % Color of negative area
\colorlet{fill3}{penColor3!20} % Fill
\colorlet{fill4}{penColor4!20} % Fill
\colorlet{fill5}{penColor5!20} % Fill
\colorlet{gridColor}{gray!50} % Color of grid in a plot

\newcommand{\surfaceColor}{violet}
\newcommand{\surfaceColorTwo}{redyellow}
\newcommand{\sliceColor}{greenyellow}




\pgfmathdeclarefunction{gauss}{2}{% gives gaussian
  \pgfmathparse{1/(#2*sqrt(2*pi))*exp(-((x-#1)^2)/(2*#2^2))}%
}


%%%%%%%%%%%%%
%% Vectors
%%%%%%%%%%%%%

%% Simple horiz vectors
\renewcommand{\vector}[1]{\left\langle #1\right\rangle}


%% %% Complex Horiz Vectors with angle brackets
%% \makeatletter
%% \renewcommand{\vector}[2][ , ]{\left\langle%
%%   \def\nextitem{\def\nextitem{#1}}%
%%   \@for \el:=#2\do{\nextitem\el}\right\rangle%
%% }
%% \makeatother

%% %% Vertical Vectors
%% \def\vector#1{\begin{bmatrix}\vecListA#1,,\end{bmatrix}}
%% \def\vecListA#1,{\if,#1,\else #1\cr \expandafter \vecListA \fi}

%%%%%%%%%%%%%
%% End of vectors
%%%%%%%%%%%%%

%\newcommand{\fullwidth}{}
%\newcommand{\normalwidth}{}



%% makes a snazzy t-chart for evaluating functions
%\newenvironment{tchart}{\rowcolors{2}{}{background!90!textColor}\array}{\endarray}

%%This is to help with formatting on future title pages.
\newenvironment{sectionOutcomes}{}{}



%% Flowchart stuff
%\tikzstyle{startstop} = [rectangle, rounded corners, minimum width=3cm, minimum height=1cm,text centered, draw=black]
%\tikzstyle{question} = [rectangle, minimum width=3cm, minimum height=1cm, text centered, draw=black]
%\tikzstyle{decision} = [trapezium, trapezium left angle=70, trapezium right angle=110, minimum width=3cm, minimum height=1cm, text centered, draw=black]
%\tikzstyle{question} = [rectangle, rounded corners, minimum width=3cm, minimum height=1cm,text centered, draw=black]
%\tikzstyle{process} = [rectangle, minimum width=3cm, minimum height=1cm, text centered, draw=black]
%\tikzstyle{decision} = [trapezium, trapezium left angle=70, trapezium right angle=110, minimum width=3cm, minimum height=1cm, text centered, draw=black]


\title[Dig-In:]{Spherical coordinates}

\outcome{Work in spherical coordinates.}
\outcome{Compute triple integrals in spherical coordinates.}

\begin{document}
\begin{abstract}
  We integrate over regions in spherical coordinates.
\end{abstract}
\maketitle

Another way to generalize polar coordinates to three dimensions is
with \textit{spherical} coordinates.

\begin{definition}
  An ordered triple consisting of a radius, an angle, and a height
  $(\rho,\varphi,\theta)$ can be graphed as
  \begin{align*}
    x &= \rho\cdot \cos(\theta)\sin(\varphi)\\
    y &= \rho\cdot \sin(\theta)\sin(\varphi)\\
    z &= \rho\cdot \cos(\varphi)
  \end{align*}
  meaning:
  \begin{image}
    \begin{tikzpicture}
      \begin{axis}[tick label style={font=\scriptsize},axis on top,
	axis lines=center,
	view={110}{25},
	name=myplot,
	xtick=\empty,
        ytick=\empty,
        ztick=\empty,
	ymin=-.1,ymax=1.2,
	xmin=-.1,xmax=1.2,
	zmin=-.2, zmax=2.1,
	every axis x label/.style={at={(axis cs:\pgfkeysvalueof{/pgfplots/xmax},0,0)},xshift=-1pt,yshift=-4pt},
	xlabel={\scriptsize $x$},
	every axis y label/.style={at={(axis cs:0,\pgfkeysvalueof{/pgfplots/ymax},0)},xshift=5pt,yshift=-3pt},
	ylabel={\scriptsize $y$},
	every axis z label/.style={at={(axis cs:0,0,\pgfkeysvalueof{/pgfplots/zmax})},xshift=0pt,yshift=4pt},
	zlabel={\scriptsize $z$},
        colormap/cool,
        ]
        \addplot3[gray,->,domain=0:45,samples y=0] ({.3*cos(x)},{.3*sin(x)},0); %% angle for theta
        \addplot3[gray,->,domain=0:27,samples y=0] ({.3*cos(45)*sin(x)},{.3*sin(45)*sin(x)},{.3*cos(x)}); %% angle for phi
        \addplot3[gray,domain=0:2.1,samples y=0,->] ({x*cos(45)*sin(27)},{x*sin(45)*sin(27)},{x*cos(27)}); %% line for rho

        
        \addplot3[gray,domain=0:1,samples y=0,dashed] ({x*cos(45)},{x*sin(45)},0); %% line for theta
        \addplot3[gray,domain=0:2,samples y=0,dashed] ({1*cos(45)},{1*sin(45)},x); %% line for z
        \node at (axis cs:{.5*cos(22.5)},{.5*sin(22.5)},0) {$\theta$};
        \node[above] at (axis cs:{.3*cos(45)*sin(17)},{.3*sin(45)*sin(17)},{.3*cos(17)}) {$\varphi$};

        \node[above] at (axis cs:{1*cos(45)*sin(27)},{1*sin(45)*sin(27)},{1*cos(27)}) {$\rho$};

        \filldraw [black] (axis cs:{1*cos(45)},{1*sin(45)},1.95) circle (2.5pt);        
        \node[right] at (axis cs:{1*cos(45)},{1*sin(45)},2) {$(\rho,\theta,\phi)$};
      \end{axis}
    \end{tikzpicture}
  \end{image}
  Coordinates of this type are called \dfn{spherical coordinates}.
\end{definition}

\begin{question}
  Conside the point $(\rho,\theta,\phi)=(2,-\pi/4,\pi/4)$ in spherical coordinates. What is
  this point when expressed in $(x,y,z)$-coordinates?
  \begin{prompt}
    \[
    (x,y,z) = (\answer{2\cos(-\pi/4)\sin(\pi/4)}, \answer{2 \sin(-\pi/4)\sin(\pi/4)},\answer{2\cos(\pi/4)})
    \]
  \end{prompt}
  %% \begin{question}
  %%   Consider the point $(1, -1,5)$ in $(x,y,z)$-coordinates. What is this
  %%   point when expressed in spherical coordinates?
  %%   \begin{prompt}
  %%     \[
  %%     (\rho,\varphi,\theta) = BADBAD
  %%     \]
  %%   \end{prompt}
  %% \end{question}
\end{question}

\section{Triple integrals in spherical coordinates}

If you want to evaluate this integral
\[
\iiint_R F \d V,
\]
you have to change $R$ to a region defined in $(x,y,z)$-coordinates,
and change $\d V$ to some combination of $\d x\d y\d z$ leaving you
with some iterated integral:
\[
\int_a^b\int_c^d\int_p^q F(x,y,z) \d y \d x\d z
\]
Now consider representing a region $R$ in spherical coordinates and
let's express $\d V$ in terms of $\d \rho$, $\d\theta$, and $\d z$. To
do this, consider the diagram below:
  \begin{image}
    \begin{tikzpicture}
      \begin{axis}[tick label style={font=\scriptsize},axis on top,
	axis lines=center,
	view={120}{25},
	name=myplot,
        clip=false,
	xtick=\empty,
        ytick=\empty,
        ztick=\empty,
	ymin=-.1,ymax=1,
	xmin=-.1,xmax=.7,
	zmin=-.2, zmax=2.1,
	every axis x label/.style={at={(axis cs:\pgfkeysvalueof{/pgfplots/xmax},0,0)},xshift=-1pt,yshift=-4pt},
	xlabel={\scriptsize $x$},
	every axis y label/.style={at={(axis cs:0,\pgfkeysvalueof{/pgfplots/ymax},0)},xshift=5pt,yshift=-3pt},
	ylabel={\scriptsize $y$},
	every axis z label/.style={at={(axis cs:0,0,\pgfkeysvalueof{/pgfplots/zmax})},xshift=0pt,yshift=4pt},
	zlabel={\scriptsize $z$},
        colormap/cool,
        ]
        \filldraw [fill=fill1]
        (axis cs:{1.8*cos(60)*sin(30)},{1.8*sin(60)*sin(30)},{1.8*cos(30)})--
        (axis cs:{1.8*cos(70)*sin(30)},{1.8*sin(70)*sin(30)},{1.8*cos(30)})--
        (axis cs:{2.24*cos(70)*sin(30)},{2.24*sin(70)*sin(30)},{2.24*cos(30)})--
        (axis cs:{2.24*cos(70)*sin(22)},{2.24*sin(70)*sin(22)},{2.24*cos(22)})--
        (axis cs:{2.24*cos(60)*sin(22)},{2.24*sin(60)*sin(22)},{2.24*cos(22)})--
        (axis cs:{1.8*cos(60)*sin(22)},{1.8*sin(60)*sin(22)},{1.8*cos(22)});

        \addplot3[->,domain=60:70,samples y=0] ({.3*cos(x)},{.3*sin(x)},0); %% dtheta
        \addplot3[domain=60:70,samples y=0,dashed] ({.8*cos(x)},{.8*sin(x)},0); %% dtheta
        \addplot3[domain=60:70,samples y=0] ({1*cos(x)},{1*sin(x)},0); %% dtheta
        %% \addplot3[domain=60:70,samples y=0,dashed] ({.8*cos(x)},{.8*sin(x)},1.7); %% dtheta
        %% \addplot3[domain=60:70,samples y=0] ({1*cos(x)},{1*sin(x)},1.7); %% dtheta
        %% \addplot3[domain=60:70,samples y=0] ({.8*cos(x)},{.8*sin(x)},2); %% dtheta
        %% \addplot3[domain=60:70,samples y=0] ({1*cos(x)},{1*sin(x)},2); %% dtheta

        \addplot3[->,domain=22:30,samples y=0] ({.3*cos(60)*sin(x)},{.3*sin(60)*sin(x)},{.3*cos(x)}); %% angle for phi
        \addplot3[domain=0:2.24,samples y=0] ({x*cos(60)*sin(22)},{x*sin(60)*sin(22)},{x*cos(22)}); %% line for rho
        \addplot3[domain=0:2.24,samples y=0,dashed] ({x*cos(70)*sin(22)},{x*sin(70)*sin(22)},{x*cos(22)}); %% line for rho
        \addplot3[domain=0:2.24,samples y=0] ({x*cos(60)*sin(30)},{x*sin(60)*sin(30)},{x*cos(30)}); %% line for rho
        \addplot3[domain=0:2.24,samples y=0] ({x*cos(70)*sin(30)},{x*sin(70)*sin(30)},{x*cos(30)}); %% line for rho

        \addplot3[domain=22:30,samples y=0] ({1.8*cos(60)*sin(x)},{1.8*sin(60)*sin(x)},{1.8*cos(x)}); %% angle for phi
        \addplot3[domain=22:30,samples y=0,dashed] ({1.8*cos(70)*sin(x)},{1.8*sin(70)*sin(x)},{1.8*cos(x)}); %% angle for phi
        \addplot3[domain=22:30,samples y=0] ({2.24*cos(60)*sin(x)},{2.24*sin(60)*sin(x)},{2.24*cos(x)}); %% angle for phi
        \addplot3[domain=22:30,samples y=0] ({2.24*cos(70)*sin(x)},{2.24*sin(70)*sin(x)},{2.24*cos(x)}); %% angle for phi

        \addplot3[domain=60:70,samples y=0] ({2.24*cos(x)*sin(22)},{2.24*sin(x)*sin(22)},{2.24*cos(22)}); %% angle for theta/box/top
        \addplot3[domain=60:70,samples y=0] ({2.24*cos(x)*sin(30)},{2.24*sin(x)*sin(30)},{2.24*cos(30)}); %% angle for theta/box/top
        \addplot3[domain=60:70,samples y=0] ({1.8*cos(x)*sin(30)},{1.8*sin(x)*sin(30)},{1.8*cos(30)}); %% angle for theta/box/top
        \addplot3[domain=60:70,samples y=0,dashed] ({1.8*cos(x)*sin(22)},{1.8*sin(x)*sin(22)},{1.8*cos(22)}); %% angle for theta/box/top

        
        \addplot3[domain=0:1,samples y=0] ({x*cos(60)},{x*sin(60)},0); %% r
        \addplot3[domain=0:1,samples y=0] ({x*cos(70)},{x*sin(70)},0); %% r
        

        \node[left] at (axis cs:{.3*cos(60)},{.3*sin(60)},0) {$\d\theta$};
        \node[below] at (axis cs:{1.8*cos(65)*sin(35)},{1.8*sin(65)*sin(35)},{1.8*cos(35)}) {$\rho\sin(\varphi)\d\theta$};
        \node[below right] at (axis cs:{2*cos(70)*sin(30)},{2*sin(70)*sin(30)},{2*cos(30)}) {$\d \rho$};
        \node[right] at (axis cs:{2.24*cos(75)*sin(24)},{2.24*sin(75)*sin(24)},{2.24*cos(24}) {$\rho \d \varphi$};
        \node[above,penColor] at (axis cs:{2.24*cos(65)*sin(22)},{2.24*sin(65)*sin(22)},{2.24*cos(22)}) {$\d V$};

        \draw [->] (axis cs:{1.8*cos(60)*sin(35)},{1.8*sin(60)*sin(35)},{1.8*cos(35)})--(axis cs:{1.8*cos(65)*sin(30)},{1.8*sin(65)*sin(30)},{1.8*cos(30)});
        \draw [->] (axis cs:{2*cos(70)*sin(32)},{2*sin(70)*sin(32)},{2*cos(32)})--(axis cs:{2*cos(70)*sin(30)},{2*sin(70)*sin(30)},{2*cos(30)});
        \draw [->] (axis cs:{2.24*cos(73)*sin(26)},{2.24*sin(73)*sin(26)},{2.24*cos(26})--(axis cs:{2.24*cos(70)*sin(26)},{2.24*sin(70)*sin(26)},{2.24*cos(26});
        
      \end{axis}
    \end{tikzpicture}
  \end{image}
Here we see 
\begin{align*}
\d V &= (\rho\sin(\varphi)\d\theta)\cdot(\rho \d \varphi)\cdot (\d \rho)\\
&= \rho^2 \sin(\varphi)\d\rho\d\varphi\d\theta.
\end{align*}


\begin{remark}
Recalling that the determiant of a $3\times 3$ matrix gives
the volume of a parallelepiped, we could also deude the correct for
for $\d V$ by setting
\begin{align*}
  x(\rho,\theta,\phi) &= \rho \cos(\theta)\sin(\phi)\\
  y(\rho,\theta,\phi) &= \rho \sin(\theta)\sin(\phi)\\
  z(\rho,\theta,\phi) &= \rho \cos(\phi)
\end{align*}
and computing:
\begin{align*}
  \d V &= \left| \det
  \begin{bmatrix}
    \pp[x]{\rho} \d \rho & \pp[y]{\rho} \d \rho & \pp[z]{\rho} \d \rho\\
    \pp[x]{\theta} \d \theta & \pp[y]{\theta} \d \theta & \pp[z]{\theta} \d \theta\\
    \pp[x]{\phi} \d \phi & \pp[y]{\phi} \d \phi & \pp[z]{\phi} \d \phi
  \end{bmatrix}
  \right|\\
  &= \left| \det
  \begin{bmatrix}
    \answer[given]{\cos(\theta)\sin(\phi)} \d \rho & \answer[given]{\sin(\theta)\sin(\phi)} \d \rho & \answer[given]{\cos(\phi)} \d \rho \\
    \answer[given]{-\rho\sin(\theta)\sin(\phi)} \d \theta & \answer[given]{\rho\cos(\theta)\sin(\phi)} \d \theta & 0\\
    \answer[given]{\rho\cos(\theta)\cos(\phi)} \d \phi & \answer[given]{\rho\sin(\theta)\cos(\phi)} \d \phi & \answer[given]{-\rho\sin(\phi)}\d \phi
  \end{bmatrix}
  \right|\\
  &= \rho^2 \sin(\varphi)\d\rho\d\varphi\d\theta
\end{align*}
\end{remark}



We may now state at theorem:
\begin{theorem}[Fubini]
  Let $F:\R^3\to\R$ be continuous on the region
  \[
  R=\{(\rho,\theta,\phi):\text{$\alpha\leq\theta\leq\beta$, $a\leq \phi\leq b$, $G_1(\theta,\phi)\le \rho\le G_2(\theta,\phi)$}\}
  \]
  Then: 
  \begin{align*}
  \iiint_R &F(\rho,\theta,\phi)\d V\\
  &= \int_\alpha^\beta\int_{a}^{b} \int_{G_1(\theta,\phi)}^{G_2(\theta,\phi)}F(\rho,\theta,\phi) \rho^2 \sin(\phi)\d\rho\d\varphi\d\theta.
  \end{align*}
\end{theorem}

\begin{question}
  Write down a triple integral in spherical coordinates that will
  compute the volume of a sphere of radius $a$.
  \begin{prompt}
  \[
  \iiint_R \d V = \int_{\answer{0}}^{\answer{2\pi}}
  \int_{\answer{0}}^{\answer{\pi}}
  \int_{\answer{0}}^{\answer{a}}
  \rho^2 \sin(\varphi)\d\rho\d\varphi\d\theta
  \]
  \end{prompt}
\end{question}


%% \begin{question}
%%   Consider the following region:
%%   \begin{image}
%%     \begin{tikzpicture}
%%       \begin{axis}%
%%         [
%%           tick label style={font=\scriptsize},%axis on top,
%% 	  axis lines=center,
%% 	  view={135}{25},
%% 	  name=myplot,
%% 	  %% xtick=\empty,
%% 	  %% ytick=\empty,
%% 	  %% ztick=\empty,
%% 	  %% extra x ticks={1},
%% 	  %% extra x tick labels={$a$},
%% 	  %% extra y ticks={1},
%% 	  %% extra y tick labels={$a$},
%% 	  %% extra z ticks={1},
%% 	  %% extra z tick labels={$h$},
%% 	  ymin=-1.3,ymax=1.3,
%% 	  xmin=-1.3,xmax=1.3,
%% 	  zmin=-.1, zmax=1.1,
%% 	  every axis x label/.style={at={(axis cs:\pgfkeysvalueof{/pgfplots/xmax},0,0)},xshift=-1pt,yshift=-4pt},
%% 	  xlabel={\scriptsize $x$},
%% 	  every axis y label/.style={at={(axis cs:0,\pgfkeysvalueof{/pgfplots/ymax},0)},xshift=5pt,yshift=-3pt},
%% 	  ylabel={\scriptsize $y$},
%% 	  every axis z label/.style={at={(axis cs:0,0,\pgfkeysvalueof{/pgfplots/zmax})},xshift=0pt,yshift=4pt},
%% 	  zlabel={\scriptsize $z$},colormap/cool
%% 	]
        
%%         \addplot3[domain=0:1,,y domain=0:360,mesh,samples=10,samples y=36,very thin,z buffer=sort] ({x*cos(y)}, {x*sin(y)},{1-x});
%%         \addplot3[domain=0:1,,y domain=0:360,mesh,samples=10,samples y=36,very thin,z buffer=sort] ({x*cos(y)}, {x*sin(y)},{1-x});
%%       \end{axis}
%%     \end{tikzpicture}
%%   \end{image}
%%   Above we see a cone with its base in the $(x,y)$-plane within the
%%   unit circle. Write three integrals that compute its volume.
%%   \begin{prompt}
%%     \[
%%     \int_{\answer{-1}}^{\answer{1}}
%%     \int_{\answer{-\sqrt{1-x^2}}}^{\answer{\sqrt{1-x^2}}}
%%     \int_{\answer{0}}^{\answer{1}}
%%     \d z \d y \d x
%%     \]
%%     \[
%%     \int_{\answer{0}}^{\answer{2\pi}}
%%     \int_{\answer{0}}^{\answer{1}}
%%     \int_{\answer{0}}^{\answer{1-r}}
%%     r\d z \d r \d \theta
%%     \]
%%     \[
%%     \int_{\answer{0}}^{\answer{2\pi}}
%%     \int_{\answer{0}}^{\answer{1}}
%%     \int_{\answer{0}}^{\answer{1-r}}
%%     \rho^2\sin(\varphi) \d\rho\d\varphi\d\theta
%%     \]
%%   \end{prompt}
%%   \end{prompt}
%% \end{question}



\end{document}
