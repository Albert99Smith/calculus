\documentclass{ximera}

%\usepackage{todonotes}
%\usepackage{mathtools} %% Required for wide table Curl and Greens
%\usepackage{cuted} %% Required for wide table Curl and Greens
\newcommand{\todo}{}

\usepackage{esint} % for \oiint
\ifxake%%https://math.meta.stackexchange.com/questions/9973/how-do-you-render-a-closed-surface-double-integral
\renewcommand{\oiint}{{\large\bigcirc}\kern-1.56em\iint}
\fi


\graphicspath{
  {./}
  {ximeraTutorial/}
  {basicPhilosophy/}
  {functionsOfSeveralVariables/}
  {normalVectors/}
  {lagrangeMultipliers/}
  {vectorFields/}
  {greensTheorem/}
  {shapeOfThingsToCome/}
  {dotProducts/}
  {partialDerivativesAndTheGradientVector/}
  {../productAndQuotientRules/exercises/}
  {../normalVectors/exercisesParametricPlots/}
  {../continuityOfFunctionsOfSeveralVariables/exercises/}
  {../partialDerivativesAndTheGradientVector/exercises/}
  {../directionalDerivativeAndChainRule/exercises/}
  {../commonCoordinates/exercisesCylindricalCoordinates/}
  {../commonCoordinates/exercisesSphericalCoordinates/}
  {../greensTheorem/exercisesCurlAndLineIntegrals/}
  {../greensTheorem/exercisesDivergenceAndLineIntegrals/}
  {../shapeOfThingsToCome/exercisesDivergenceTheorem/}
  {../greensTheorem/}
  {../shapeOfThingsToCome/}
  {../separableDifferentialEquations/exercises/}
  {vectorFields/}
}

\newcommand{\mooculus}{\textsf{\textbf{MOOC}\textnormal{\textsf{ULUS}}}}

\usepackage{tkz-euclide}\usepackage{tikz}
\usepackage{tikz-cd}
\usetikzlibrary{arrows}
\tikzset{>=stealth,commutative diagrams/.cd,
  arrow style=tikz,diagrams={>=stealth}} %% cool arrow head
\tikzset{shorten <>/.style={ shorten >=#1, shorten <=#1 } } %% allows shorter vectors

\usetikzlibrary{backgrounds} %% for boxes around graphs
\usetikzlibrary{shapes,positioning}  %% Clouds and stars
\usetikzlibrary{matrix} %% for matrix
\usepgfplotslibrary{polar} %% for polar plots
\usepgfplotslibrary{fillbetween} %% to shade area between curves in TikZ
\usetkzobj{all}
\usepackage[makeroom]{cancel} %% for strike outs
%\usepackage{mathtools} %% for pretty underbrace % Breaks Ximera
%\usepackage{multicol}
\usepackage{pgffor} %% required for integral for loops



%% http://tex.stackexchange.com/questions/66490/drawing-a-tikz-arc-specifying-the-center
%% Draws beach ball
\tikzset{pics/carc/.style args={#1:#2:#3}{code={\draw[pic actions] (#1:#3) arc(#1:#2:#3);}}}



\usepackage{array}
\setlength{\extrarowheight}{+.1cm}
\newdimen\digitwidth
\settowidth\digitwidth{9}
\def\divrule#1#2{
\noalign{\moveright#1\digitwidth
\vbox{\hrule width#2\digitwidth}}}





\newcommand{\RR}{\mathbb R}
\newcommand{\R}{\mathbb R}
\newcommand{\N}{\mathbb N}
\newcommand{\Z}{\mathbb Z}

\newcommand{\sagemath}{\textsf{SageMath}}


%\renewcommand{\d}{\,d\!}
\renewcommand{\d}{\mathop{}\!d}
\newcommand{\dd}[2][]{\frac{\d #1}{\d #2}}
\newcommand{\pp}[2][]{\frac{\partial #1}{\partial #2}}
\renewcommand{\l}{\ell}
\newcommand{\ddx}{\frac{d}{\d x}}

\newcommand{\zeroOverZero}{\ensuremath{\boldsymbol{\tfrac{0}{0}}}}
\newcommand{\inftyOverInfty}{\ensuremath{\boldsymbol{\tfrac{\infty}{\infty}}}}
\newcommand{\zeroOverInfty}{\ensuremath{\boldsymbol{\tfrac{0}{\infty}}}}
\newcommand{\zeroTimesInfty}{\ensuremath{\small\boldsymbol{0\cdot \infty}}}
\newcommand{\inftyMinusInfty}{\ensuremath{\small\boldsymbol{\infty - \infty}}}
\newcommand{\oneToInfty}{\ensuremath{\boldsymbol{1^\infty}}}
\newcommand{\zeroToZero}{\ensuremath{\boldsymbol{0^0}}}
\newcommand{\inftyToZero}{\ensuremath{\boldsymbol{\infty^0}}}



\newcommand{\numOverZero}{\ensuremath{\boldsymbol{\tfrac{\#}{0}}}}
\newcommand{\dfn}{\textbf}
%\newcommand{\unit}{\,\mathrm}
\newcommand{\unit}{\mathop{}\!\mathrm}
\newcommand{\eval}[1]{\bigg[ #1 \bigg]}
\newcommand{\seq}[1]{\left( #1 \right)}
\renewcommand{\epsilon}{\varepsilon}
\renewcommand{\phi}{\varphi}


\renewcommand{\iff}{\Leftrightarrow}

\DeclareMathOperator{\arccot}{arccot}
\DeclareMathOperator{\arcsec}{arcsec}
\DeclareMathOperator{\arccsc}{arccsc}
\DeclareMathOperator{\si}{Si}
\DeclareMathOperator{\scal}{scal}
\DeclareMathOperator{\sign}{sign}


%% \newcommand{\tightoverset}[2]{% for arrow vec
%%   \mathop{#2}\limits^{\vbox to -.5ex{\kern-0.75ex\hbox{$#1$}\vss}}}
\newcommand{\arrowvec}[1]{{\overset{\rightharpoonup}{#1}}}
%\renewcommand{\vec}[1]{\arrowvec{\mathbf{#1}}}
\renewcommand{\vec}[1]{{\overset{\boldsymbol{\rightharpoonup}}{\mathbf{#1}}}\hspace{0in}}

\newcommand{\point}[1]{\left(#1\right)} %this allows \vector{ to be changed to \vector{ with a quick find and replace
\newcommand{\pt}[1]{\mathbf{#1}} %this allows \vec{ to be changed to \vec{ with a quick find and replace
\newcommand{\Lim}[2]{\lim_{\point{#1} \to \point{#2}}} %Bart, I changed this to point since I want to use it.  It runs through both of the exercise and exerciseE files in limits section, which is why it was in each document to start with.

\DeclareMathOperator{\proj}{\mathbf{proj}}
\newcommand{\veci}{{\boldsymbol{\hat{\imath}}}}
\newcommand{\vecj}{{\boldsymbol{\hat{\jmath}}}}
\newcommand{\veck}{{\boldsymbol{\hat{k}}}}
\newcommand{\vecl}{\vec{\boldsymbol{\l}}}
\newcommand{\uvec}[1]{\mathbf{\hat{#1}}}
\newcommand{\utan}{\mathbf{\hat{t}}}
\newcommand{\unormal}{\mathbf{\hat{n}}}
\newcommand{\ubinormal}{\mathbf{\hat{b}}}

\newcommand{\dotp}{\bullet}
\newcommand{\cross}{\boldsymbol\times}
\newcommand{\grad}{\boldsymbol\nabla}
\newcommand{\divergence}{\grad\dotp}
\newcommand{\curl}{\grad\cross}
%\DeclareMathOperator{\divergence}{divergence}
%\DeclareMathOperator{\curl}[1]{\grad\cross #1}
\newcommand{\lto}{\mathop{\longrightarrow\,}\limits}

\renewcommand{\bar}{\overline}

\colorlet{textColor}{black}
\colorlet{background}{white}
\colorlet{penColor}{blue!50!black} % Color of a curve in a plot
\colorlet{penColor2}{red!50!black}% Color of a curve in a plot
\colorlet{penColor3}{red!50!blue} % Color of a curve in a plot
\colorlet{penColor4}{green!50!black} % Color of a curve in a plot
\colorlet{penColor5}{orange!80!black} % Color of a curve in a plot
\colorlet{penColor6}{yellow!70!black} % Color of a curve in a plot
\colorlet{fill1}{penColor!20} % Color of fill in a plot
\colorlet{fill2}{penColor2!20} % Color of fill in a plot
\colorlet{fillp}{fill1} % Color of positive area
\colorlet{filln}{penColor2!20} % Color of negative area
\colorlet{fill3}{penColor3!20} % Fill
\colorlet{fill4}{penColor4!20} % Fill
\colorlet{fill5}{penColor5!20} % Fill
\colorlet{gridColor}{gray!50} % Color of grid in a plot

\newcommand{\surfaceColor}{violet}
\newcommand{\surfaceColorTwo}{redyellow}
\newcommand{\sliceColor}{greenyellow}




\pgfmathdeclarefunction{gauss}{2}{% gives gaussian
  \pgfmathparse{1/(#2*sqrt(2*pi))*exp(-((x-#1)^2)/(2*#2^2))}%
}


%%%%%%%%%%%%%
%% Vectors
%%%%%%%%%%%%%

%% Simple horiz vectors
\renewcommand{\vector}[1]{\left\langle #1\right\rangle}


%% %% Complex Horiz Vectors with angle brackets
%% \makeatletter
%% \renewcommand{\vector}[2][ , ]{\left\langle%
%%   \def\nextitem{\def\nextitem{#1}}%
%%   \@for \el:=#2\do{\nextitem\el}\right\rangle%
%% }
%% \makeatother

%% %% Vertical Vectors
%% \def\vector#1{\begin{bmatrix}\vecListA#1,,\end{bmatrix}}
%% \def\vecListA#1,{\if,#1,\else #1\cr \expandafter \vecListA \fi}

%%%%%%%%%%%%%
%% End of vectors
%%%%%%%%%%%%%

%\newcommand{\fullwidth}{}
%\newcommand{\normalwidth}{}



%% makes a snazzy t-chart for evaluating functions
%\newenvironment{tchart}{\rowcolors{2}{}{background!90!textColor}\array}{\endarray}

%%This is to help with formatting on future title pages.
\newenvironment{sectionOutcomes}{}{}



%% Flowchart stuff
%\tikzstyle{startstop} = [rectangle, rounded corners, minimum width=3cm, minimum height=1cm,text centered, draw=black]
%\tikzstyle{question} = [rectangle, minimum width=3cm, minimum height=1cm, text centered, draw=black]
%\tikzstyle{decision} = [trapezium, trapezium left angle=70, trapezium right angle=110, minimum width=3cm, minimum height=1cm, text centered, draw=black]
%\tikzstyle{question} = [rectangle, rounded corners, minimum width=3cm, minimum height=1cm,text centered, draw=black]
%\tikzstyle{process} = [rectangle, minimum width=3cm, minimum height=1cm, text centered, draw=black]
%\tikzstyle{decision} = [trapezium, trapezium left angle=70, trapezium right angle=110, minimum width=3cm, minimum height=1cm, text centered, draw=black]


\title[Dig-In:]{Cylindrical coordinates}

\outcome{Work in cylindrical coordinates.}
\outcome{Compute double integrals in cylindrical coordinates.}

\begin{document}
\begin{abstract}
  We integrate over regions in cylindrical coordinates.
\end{abstract}
\maketitle

The first way we will generalize polar coordinates to three dimensions is with \textit{cylindrical} coordinates.

\begin{definition}
  An ordered triple consisting of a radius, an angle, and a height
  $(r,\theta,z)$ can be graphed as
  \begin{align*}
    x &= r\cdot \cos(\theta)\\
    y &= r\cdot \sin(\theta)\\
    z &= z
  \end{align*}
  meaning:
  \begin{image}
    \begin{tikzpicture}
      \begin{axis}[tick label style={font=\scriptsize},axis on top,
	axis lines=center,
	view={110}{25},
	name=myplot,
	xtick=\empty,
        ytick=\empty,
        ztick=\empty,
	ymin=-.1,ymax=1.2,
	xmin=-.1,xmax=1.2,
	zmin=-.2, zmax=2.1,
	every axis x label/.style={at={(axis cs:\pgfkeysvalueof{/pgfplots/xmax},0,0)},xshift=-1pt,yshift=-4pt},
	xlabel={\scriptsize $x$},
	every axis y label/.style={at={(axis cs:0,\pgfkeysvalueof{/pgfplots/ymax},0)},xshift=5pt,yshift=-3pt},
	ylabel={\scriptsize $y$},
	every axis z label/.style={at={(axis cs:0,0,\pgfkeysvalueof{/pgfplots/zmax})},xshift=0pt,yshift=4pt},
	zlabel={\scriptsize $z$},
        colormap/cool,
        ]
        \addplot3[->,domain=0:45,samples y=0,gray] ({.3*cos(x)},{.3*sin(x)},0);
        \addplot3[->,domain=0:1,samples y=0,gray] ({x*cos(45)},{x*sin(45)},0);
        \addplot3[->,domain=0:.95,samples y=0,gray] ({cos(45)},{sin(45)},2*x);
        \node at (axis cs:{.5*cos(22.5)},{.5*sin(22.5)},0) {$\theta$};

        \node[above] at (axis cs:{.7*cos(45)},{.7*sin(45)},0) {$r$};

        \node[right] at (axis cs:{1*cos(45)},{1*sin(45)},1) {$z$};
        \filldraw [black] (axis cs:{1*cos(45)},{1*sin(45)},2) circle (3pt);        
        \node[right] at (axis cs:{1*cos(45)},{1*sin(45)},2) {$(r,\theta,z)$};
      \end{axis}
    \end{tikzpicture}
  \end{image}
  Coordinates of this type are called \dfn{cylindrical coordinates}.
\end{definition}

\begin{question}
  Consider the point $(2, \pi/3,5)$ in cylindrical coordinates. What is
  this point when expressed in $(x,y,z)$-coordinates?
  \begin{prompt}
    \[
    (x,y,z) = (\answer{2\cos(\pi/3)}, \answer{2 \sin(\pi/3)},\answer{5})
    \]
  \end{prompt}
  \begin{question}
    Consider the point $(1, -1,5)$ in $(x,y,z)$-coordinates. What is
    this point when expressed in cylindrical coordinates where
    $0\le\theta<2\pi$?
    \begin{prompt}
      \[
      (r,\theta,z) = (\answer{\sqrt{2}}, \answer{7\pi/4},\answer{5})
      \]
    \end{prompt}
  \end{question}
\end{question}

\section{Triple integrals in cylindrical coordinates}

If you want to evaluate this integral
\[
\iiint_R F \d V,
\]
you have to change $R$ to a region defined in $(x,y,z)$-coordinates,
and change $\d V$ to some combination of $\d x\d y\d z$ leaving you
with some iterated integral:
\[
\int_a^b\int_c^d\int_p^q F(x,y,z) \d y \d x\d z
\]
Now consider representing a region $R$ in cylindrical coordinates
and let's express $\d V$ in terms of $\d r$, $\d\theta$, and $\d z$. To do this, consider the diagram below:
  \begin{image}
    \begin{tikzpicture}
      \begin{axis}[tick label style={font=\scriptsize},axis on top,
	axis lines=center,
	view={120}{25},
	name=myplot,
        clip=false,
	xtick=\empty,
        ytick=\empty,
        ztick=\empty,
	ymin=-.1,ymax=1,
	xmin=-.1,xmax=.7,
	zmin=-.2, zmax=2.1,
	every axis x label/.style={at={(axis cs:\pgfkeysvalueof{/pgfplots/xmax},0,0)},xshift=-1pt,yshift=-4pt},
	xlabel={\scriptsize $x$},
	every axis y label/.style={at={(axis cs:0,\pgfkeysvalueof{/pgfplots/ymax},0)},xshift=5pt,yshift=-3pt},
	ylabel={\scriptsize $y$},
	every axis z label/.style={at={(axis cs:0,0,\pgfkeysvalueof{/pgfplots/zmax})},xshift=0pt,yshift=4pt},
	zlabel={\scriptsize $z$},
        colormap/cool,
        ]
        \filldraw [fill=fill1]
        (axis cs:{1*cos(60)},{1*sin(60)},1.7)--
        (axis cs:{1*cos(70)},{1*sin(70)},1.7)--
        (axis cs:{1*cos(70)},{1*sin(70)},2)--
        (axis cs:{.8*cos(70)},{.8*sin(70)},2)--
        (axis cs:{.8*cos(60)},{.8*sin(60)},2)--
        (axis cs:{.8*cos(60)},{.8*sin(60)},1.7);

        \addplot3[->,domain=60:70,samples y=0] ({.3*cos(x)},{.3*sin(x)},0); %% dtheta
        \addplot3[domain=60:70,samples y=0,dashed] ({.8*cos(x)},{.8*sin(x)},0); %% dtheta
        \addplot3[domain=60:70,samples y=0] ({1*cos(x)},{1*sin(x)},0); %% dtheta
        \addplot3[domain=60:70,samples y=0,dashed] ({.8*cos(x)},{.8*sin(x)},1.7); %% dtheta
        \addplot3[domain=60:70,samples y=0] ({1*cos(x)},{1*sin(x)},1.7); %% dtheta
        \addplot3[domain=60:70,samples y=0] ({.8*cos(x)},{.8*sin(x)},2); %% dtheta
        \addplot3[domain=60:70,samples y=0] ({1*cos(x)},{1*sin(x)},2); %% dtheta
        \addplot3[domain=0:1,samples y=0] ({x*cos(60)},{x*sin(60)},0); %% r
        \addplot3[domain=0:1,samples y=0] ({x*cos(70)},{x*sin(70)},0); %% r
        \addplot3[domain=.8:1,samples y=0,dashed] ({x*cos(70)},{x*sin(70)},1.7); %% dr
        \addplot3[domain=.8:1,samples y=0] ({x*cos(60)},{x*sin(60)},1.7); %% dr
        \addplot3[domain=.8:1,samples y=0] ({x*cos(60)},{x*sin(60)},2); %% dr
        \addplot3[domain=.8:1,samples y=0] ({x*cos(70)},{x*sin(70)},2); %% dr
        \addplot3[domain=0:2,samples y=0] ({.8*cos(60)},{.8*sin(60)},x); %% z and dz
        \addplot3[domain=0:2,samples y=0,dashed] ({.8*cos(70)},{.8*sin(70)},x); %% z and dz
        \addplot3[domain=0:2,samples y=0] ({1*cos(60)},{1*sin(60)},x); %% z and dz
        \addplot3[domain=0:2,samples y=0] ({1*cos(70)},{1*sin(70)},x); %% z and dz
        

        \node[left] at (axis cs:{.3*cos(60)},{.3*sin(60)},0) {$\d\theta$};
        \node[below left] at (axis cs:{.9*cos(62)},{.9*sin(62)},0) {$\d r$};
        \node[below] at (axis cs:{1*cos(65)},{1*sin(65)},0) {$r\d\theta$};
        \node[right] at (axis cs:{1*cos(70)},{1*sin(70)},1.85) {$\d z$};
        \node[above,penColor] at (axis cs:{.9*cos(65)},{.9*sin(65)},2.2) {$\d V$};
      \end{axis}
    \end{tikzpicture}
  \end{image}
Here we see 
\begin{align*}
\d V &= \d r \cdot (r \d \theta)\cdot \d z \\
&= \d z\ r \d r \d \theta.
\end{align*}
\begin{theorem}[Fubini]
  Let $F:\R^3\to\R$ be continuous on the region
  \[
  R=\{(r,\theta,z):\alpha\leq\theta\leq\beta, g_1(\theta)\leq r\leq g_2(\theta), G_1(x,y)\le z\le G_2(x,y)\}
  \]
  Then: 
  \[
  \iiint_R F(r,\theta,z)\d V = \int_\alpha^\beta\int_{g_1(\theta)}^{g_2(\theta)} \int_{G_1(r\cos(\theta),r\sin(\theta))}^{G_2(r\cos(\theta),r\sin(\theta))}F(r,\theta,z) \d z \ r\d r\d \theta.
  \]
\end{theorem}

\begin{question}
  Write down a triple integral in cylindrical coordinates that will
  compute the volume of a cylinder of radius $a$ and height $h$.
  \begin{prompt}
  \[
  \iiint_R \d V = \int_{\answer{0}}^{\answer{2\pi}}
  \int_{\answer{0}}^{\answer{a}}
  \int_{\answer{0}}^{\answer{h}}
  \d z \ r \d r \d \theta 
  \]
  \end{prompt}
\end{question}


\begin{example}
  Find the volume under $z= \sqrt{4-r^2}$ above the quarter circle
  inside $x^2 + y^2 = 4$ in the first quadrant.
  \begin{explanation}
    In this case
    \[
    R=\{(r,\theta,z):\answer[given]{0}\leq\theta\leq\answer[given]{\pi/2}, \answer[given]{0}\leq r\leq \answer[given]{2}, \answer[given]{0}\le z\le \sqrt{4-r^2}\}
    \]
    So our integral is
    \begin{align*}
      \int_{\answer[given]{0}}^{\answer[given]{\pi/2}}
      \int_{\answer[given]{0}}^{\answer[given]{2}}
      \int_{\answer[given]{0}}^{\answer[given]{\sqrt{4-r^2}}} r \d z \d r \d \theta
      &=
      \int_{\answer[given]{0}}^{\answer[given]{\pi/2}}
      \int_{\answer[given]{0}}^{\answer[given]{2}}
      \answer[given]{\sqrt{4-r^2}} r \d r \d \theta\\
      &=\int_{\answer[given]{0}}^{\answer[given]{\pi/2}}
      \eval{\answer[given]{\frac{-(4-r^2)^{3/2}}{3}}}_{\answer[given]{0}}^{\answer[given]{2}}  \d \theta\\
      &=\int_{\answer[given]{0}}^{\answer[given]{\pi/2}}
      \answer[given]{\frac{8}{3}}  \d \theta\\
      &=\answer[given]{4\pi/3}.
    \end{align*}
  \end{explanation}
\end{example}

\begin{example}
  Find the volume of the object defined as the intersection of the
  cylinder $x^2+y^2=1$ and the sphere $x^2+y^2+z^2=4$.
  \begin{explanation}
    First note that we can express our region as
    \begin{align*}
      R=\{(r,\theta,z):&\answer[given]{0}\leq\theta\leq\answer[given]{2\pi}, \\
      &\answer[given]{0}\leq r\leq \answer[given]{1}, \\
      &-\sqrt{4-r^2}\le z\le \answer[given]{\sqrt{4-r^2}}\}
    \end{align*}
    Now write with me,
    \begin{align*}
      \int_{\answer[given]{0}}^{\answer[given]{2\pi}}
      \int_{\answer[given]{0}}^{\answer[given]{1}}
      \int_{\answer[given]{-\sqrt{4-r^2}}}^{\answer[given]{\sqrt{4-r^2}}} r \d z \d r \d \theta
      &=
      \int_{\answer[given]{0}}^{\answer[given]{2\pi}}
      \int_{\answer[given]{0}}^{\answer[given]{1}}
      \answer[given]{2\sqrt{4-r^2}} r \d r \d \theta\\
      &=\int_{\answer[given]{0}}^{\answer[given]{2\pi}}
      \eval{\answer[given]{\frac{-2(4-r^2)^{3/2}}{3}}}_{\answer[given]{0}}^{\answer[given]{1}}  \d \theta\\
      &=\int_{\answer[given]{0}}^{\answer[given]{2\pi}}
      \answer[given]{\frac{16}{3}-2\sqrt{3}}  \d \theta\\
      &=\answer[given]{2\pi\left(\frac{16}{3}-2\sqrt{3}\right)}.
    \end{align*}
  \end{explanation}
\end{example}



\end{document}
