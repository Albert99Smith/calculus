\documentclass{ximera}

%\usepackage{todonotes}
%\usepackage{mathtools} %% Required for wide table Curl and Greens
%\usepackage{cuted} %% Required for wide table Curl and Greens
\newcommand{\todo}{}

\usepackage{esint} % for \oiint
\ifxake%%https://math.meta.stackexchange.com/questions/9973/how-do-you-render-a-closed-surface-double-integral
\renewcommand{\oiint}{{\large\bigcirc}\kern-1.56em\iint}
\fi


\graphicspath{
  {./}
  {ximeraTutorial/}
  {basicPhilosophy/}
  {functionsOfSeveralVariables/}
  {normalVectors/}
  {lagrangeMultipliers/}
  {vectorFields/}
  {greensTheorem/}
  {shapeOfThingsToCome/}
  {dotProducts/}
  {partialDerivativesAndTheGradientVector/}
  {../productAndQuotientRules/exercises/}
  {../normalVectors/exercisesParametricPlots/}
  {../continuityOfFunctionsOfSeveralVariables/exercises/}
  {../partialDerivativesAndTheGradientVector/exercises/}
  {../directionalDerivativeAndChainRule/exercises/}
  {../commonCoordinates/exercisesCylindricalCoordinates/}
  {../commonCoordinates/exercisesSphericalCoordinates/}
  {../greensTheorem/exercisesCurlAndLineIntegrals/}
  {../greensTheorem/exercisesDivergenceAndLineIntegrals/}
  {../shapeOfThingsToCome/exercisesDivergenceTheorem/}
  {../greensTheorem/}
  {../shapeOfThingsToCome/}
  {../separableDifferentialEquations/exercises/}
  {vectorFields/}
}

\newcommand{\mooculus}{\textsf{\textbf{MOOC}\textnormal{\textsf{ULUS}}}}

\usepackage{tkz-euclide}\usepackage{tikz}
\usepackage{tikz-cd}
\usetikzlibrary{arrows}
\tikzset{>=stealth,commutative diagrams/.cd,
  arrow style=tikz,diagrams={>=stealth}} %% cool arrow head
\tikzset{shorten <>/.style={ shorten >=#1, shorten <=#1 } } %% allows shorter vectors

\usetikzlibrary{backgrounds} %% for boxes around graphs
\usetikzlibrary{shapes,positioning}  %% Clouds and stars
\usetikzlibrary{matrix} %% for matrix
\usepgfplotslibrary{polar} %% for polar plots
\usepgfplotslibrary{fillbetween} %% to shade area between curves in TikZ
\usetkzobj{all}
\usepackage[makeroom]{cancel} %% for strike outs
%\usepackage{mathtools} %% for pretty underbrace % Breaks Ximera
%\usepackage{multicol}
\usepackage{pgffor} %% required for integral for loops



%% http://tex.stackexchange.com/questions/66490/drawing-a-tikz-arc-specifying-the-center
%% Draws beach ball
\tikzset{pics/carc/.style args={#1:#2:#3}{code={\draw[pic actions] (#1:#3) arc(#1:#2:#3);}}}



\usepackage{array}
\setlength{\extrarowheight}{+.1cm}
\newdimen\digitwidth
\settowidth\digitwidth{9}
\def\divrule#1#2{
\noalign{\moveright#1\digitwidth
\vbox{\hrule width#2\digitwidth}}}





\newcommand{\RR}{\mathbb R}
\newcommand{\R}{\mathbb R}
\newcommand{\N}{\mathbb N}
\newcommand{\Z}{\mathbb Z}

\newcommand{\sagemath}{\textsf{SageMath}}


%\renewcommand{\d}{\,d\!}
\renewcommand{\d}{\mathop{}\!d}
\newcommand{\dd}[2][]{\frac{\d #1}{\d #2}}
\newcommand{\pp}[2][]{\frac{\partial #1}{\partial #2}}
\renewcommand{\l}{\ell}
\newcommand{\ddx}{\frac{d}{\d x}}

\newcommand{\zeroOverZero}{\ensuremath{\boldsymbol{\tfrac{0}{0}}}}
\newcommand{\inftyOverInfty}{\ensuremath{\boldsymbol{\tfrac{\infty}{\infty}}}}
\newcommand{\zeroOverInfty}{\ensuremath{\boldsymbol{\tfrac{0}{\infty}}}}
\newcommand{\zeroTimesInfty}{\ensuremath{\small\boldsymbol{0\cdot \infty}}}
\newcommand{\inftyMinusInfty}{\ensuremath{\small\boldsymbol{\infty - \infty}}}
\newcommand{\oneToInfty}{\ensuremath{\boldsymbol{1^\infty}}}
\newcommand{\zeroToZero}{\ensuremath{\boldsymbol{0^0}}}
\newcommand{\inftyToZero}{\ensuremath{\boldsymbol{\infty^0}}}



\newcommand{\numOverZero}{\ensuremath{\boldsymbol{\tfrac{\#}{0}}}}
\newcommand{\dfn}{\textbf}
%\newcommand{\unit}{\,\mathrm}
\newcommand{\unit}{\mathop{}\!\mathrm}
\newcommand{\eval}[1]{\bigg[ #1 \bigg]}
\newcommand{\seq}[1]{\left( #1 \right)}
\renewcommand{\epsilon}{\varepsilon}
\renewcommand{\phi}{\varphi}


\renewcommand{\iff}{\Leftrightarrow}

\DeclareMathOperator{\arccot}{arccot}
\DeclareMathOperator{\arcsec}{arcsec}
\DeclareMathOperator{\arccsc}{arccsc}
\DeclareMathOperator{\si}{Si}
\DeclareMathOperator{\scal}{scal}
\DeclareMathOperator{\sign}{sign}


%% \newcommand{\tightoverset}[2]{% for arrow vec
%%   \mathop{#2}\limits^{\vbox to -.5ex{\kern-0.75ex\hbox{$#1$}\vss}}}
\newcommand{\arrowvec}[1]{{\overset{\rightharpoonup}{#1}}}
%\renewcommand{\vec}[1]{\arrowvec{\mathbf{#1}}}
\renewcommand{\vec}[1]{{\overset{\boldsymbol{\rightharpoonup}}{\mathbf{#1}}}\hspace{0in}}

\newcommand{\point}[1]{\left(#1\right)} %this allows \vector{ to be changed to \vector{ with a quick find and replace
\newcommand{\pt}[1]{\mathbf{#1}} %this allows \vec{ to be changed to \vec{ with a quick find and replace
\newcommand{\Lim}[2]{\lim_{\point{#1} \to \point{#2}}} %Bart, I changed this to point since I want to use it.  It runs through both of the exercise and exerciseE files in limits section, which is why it was in each document to start with.

\DeclareMathOperator{\proj}{\mathbf{proj}}
\newcommand{\veci}{{\boldsymbol{\hat{\imath}}}}
\newcommand{\vecj}{{\boldsymbol{\hat{\jmath}}}}
\newcommand{\veck}{{\boldsymbol{\hat{k}}}}
\newcommand{\vecl}{\vec{\boldsymbol{\l}}}
\newcommand{\uvec}[1]{\mathbf{\hat{#1}}}
\newcommand{\utan}{\mathbf{\hat{t}}}
\newcommand{\unormal}{\mathbf{\hat{n}}}
\newcommand{\ubinormal}{\mathbf{\hat{b}}}

\newcommand{\dotp}{\bullet}
\newcommand{\cross}{\boldsymbol\times}
\newcommand{\grad}{\boldsymbol\nabla}
\newcommand{\divergence}{\grad\dotp}
\newcommand{\curl}{\grad\cross}
%\DeclareMathOperator{\divergence}{divergence}
%\DeclareMathOperator{\curl}[1]{\grad\cross #1}
\newcommand{\lto}{\mathop{\longrightarrow\,}\limits}

\renewcommand{\bar}{\overline}

\colorlet{textColor}{black}
\colorlet{background}{white}
\colorlet{penColor}{blue!50!black} % Color of a curve in a plot
\colorlet{penColor2}{red!50!black}% Color of a curve in a plot
\colorlet{penColor3}{red!50!blue} % Color of a curve in a plot
\colorlet{penColor4}{green!50!black} % Color of a curve in a plot
\colorlet{penColor5}{orange!80!black} % Color of a curve in a plot
\colorlet{penColor6}{yellow!70!black} % Color of a curve in a plot
\colorlet{fill1}{penColor!20} % Color of fill in a plot
\colorlet{fill2}{penColor2!20} % Color of fill in a plot
\colorlet{fillp}{fill1} % Color of positive area
\colorlet{filln}{penColor2!20} % Color of negative area
\colorlet{fill3}{penColor3!20} % Fill
\colorlet{fill4}{penColor4!20} % Fill
\colorlet{fill5}{penColor5!20} % Fill
\colorlet{gridColor}{gray!50} % Color of grid in a plot

\newcommand{\surfaceColor}{violet}
\newcommand{\surfaceColorTwo}{redyellow}
\newcommand{\sliceColor}{greenyellow}




\pgfmathdeclarefunction{gauss}{2}{% gives gaussian
  \pgfmathparse{1/(#2*sqrt(2*pi))*exp(-((x-#1)^2)/(2*#2^2))}%
}


%%%%%%%%%%%%%
%% Vectors
%%%%%%%%%%%%%

%% Simple horiz vectors
\renewcommand{\vector}[1]{\left\langle #1\right\rangle}


%% %% Complex Horiz Vectors with angle brackets
%% \makeatletter
%% \renewcommand{\vector}[2][ , ]{\left\langle%
%%   \def\nextitem{\def\nextitem{#1}}%
%%   \@for \el:=#2\do{\nextitem\el}\right\rangle%
%% }
%% \makeatother

%% %% Vertical Vectors
%% \def\vector#1{\begin{bmatrix}\vecListA#1,,\end{bmatrix}}
%% \def\vecListA#1,{\if,#1,\else #1\cr \expandafter \vecListA \fi}

%%%%%%%%%%%%%
%% End of vectors
%%%%%%%%%%%%%

%\newcommand{\fullwidth}{}
%\newcommand{\normalwidth}{}



%% makes a snazzy t-chart for evaluating functions
%\newenvironment{tchart}{\rowcolors{2}{}{background!90!textColor}\array}{\endarray}

%%This is to help with formatting on future title pages.
\newenvironment{sectionOutcomes}{}{}



%% Flowchart stuff
%\tikzstyle{startstop} = [rectangle, rounded corners, minimum width=3cm, minimum height=1cm,text centered, draw=black]
%\tikzstyle{question} = [rectangle, minimum width=3cm, minimum height=1cm, text centered, draw=black]
%\tikzstyle{decision} = [trapezium, trapezium left angle=70, trapezium right angle=110, minimum width=3cm, minimum height=1cm, text centered, draw=black]
%\tikzstyle{question} = [rectangle, rounded corners, minimum width=3cm, minimum height=1cm,text centered, draw=black]
%\tikzstyle{process} = [rectangle, minimum width=3cm, minimum height=1cm, text centered, draw=black]
%\tikzstyle{decision} = [trapezium, trapezium left angle=70, trapezium right angle=110, minimum width=3cm, minimum height=1cm, text centered, draw=black]


\author{Jim Talamo}
\license{Creative Commons 3.0 By-NC}


\outcome{Set up an integral that gives the area of a surface of revolution with respect to both $x$ and $y$.}
\outcome{Find  the area of a surface of revolution}

\begin{document}
\begin{exercise}

The portion of the curve $x=4e^{3y}$ from $y=\ln(2)$ to $y=1$ is revolved around the line $y=2$.

 \begin{image}
      \begin{tikzpicture}
        \begin{axis}[
            xmin=-3, xmax=90,
            domain=-1:1,
            ymin=-.3, ymax=2.4,
            clip=false,
            xtick = {10,20,30,40,50,60,70,80},
            ytick = {-1,1,2,3},
            axis lines =center,
            xlabel=$y$, ylabel=$y$, every axis y label/.style={at=(current axis.above origin),anchor=south},
            every axis x label/.style={at=(current axis.right of origin),anchor=west},
            axis on top,
          ]
                              
         \addplot [penColor,thick,smooth,domain=32:80]{1/3*ln(x/4)};
          
         % ds and points
          	\addplot[color=penColor2,fill=penColor2,only marks,mark=*] coordinates{(40,.767)};
		\addplot[color=penColor2,fill=penColor2,only marks,mark=*] coordinates{(45,.806)};
		\addplot[ultra thick, penColor2] plot coordinates {(40,.767) (45,.806)};
          	\node[anchor=north, penColor2] at (axis cs:45,.75) {$\Delta s$};
	%axis
	\addplot[ultra thick, penColor5, dotted] plot coordinates {(.3,2) (90,2)};
	
          %r and point 
           	\addplot[thick, penColor2] plot coordinates {(42.5,2) (42.5,.79)};
		\node[anchor=north, penColor2] at (axis cs:39,1.4) {$r$};
          
          %endpoints
        \addplot[color=penColor,fill=penColor,only marks,mark=*] coordinates{(32,.69)};
	\addplot[color=penColor,fill=penColor,only marks,mark=*] coordinates{(80,1)};
          
          \node[penColor] at (axis cs:65,1.15) {$x=4e^{3y}$};
        \end{axis}
      \end{tikzpicture}
    \end{image}
 
To set up an integral with respect to $x$ that gives the area of the surface of revolution, do the following:  

Since we have chosen to integrate with respect to $x$, we use the result:

\[ SA = \int_{x=a}^{x=b} 2 \pi r \d s\]

and we must express $r$ in terms of $x$ and $\d s$ in terms of $x$ and $\d x$.  


Let's start by describing the curve as a function of $x$.  Since $x=4e^{3y}$, we find:

\[
y= \answer{\frac{1}{3}\ln\left(\frac{x}{4}\right)}
\]

Calculating $\d s$ gives: 

\[
\d s = \sqrt{1+ \answer{\frac{1}{9x^2}}} \d x
\]


\begin{exercise}
Note that is $r$ is the distance from the axis to the curve. This is a:

\begin{multipleChoice}
\choice[correct]{vertical distance}
\choice{horizontal distance}
\end{multipleChoice} 
Thus $r=y_{top}-y_{bot} = \answer{2-\frac{1}{3}\ln\left(\frac{x}{4}\right)}.$


\begin{hint}
Since we have to express $r$ in terms of $x$, and we note that $y_{top}$ is on the axis of rotation, we must express it in terms of $x$.  Hence, $y_{top} = \answer{2}$.

Similarly, $y_{bot} =\answer{ \frac{1}{3}\ln\left(\frac{x}{4}\right)}$ since $y_{bot}$ is on the curve.
\end{hint}



\end{exercise}

\begin{exercise}
Now we see that an integral that gives the surface area is: 
\[
SA= \int_{x=a}^{x=b} 2 \pi r \d s = \int_{x=\answer{32}}^{x=\answer{4e^3}} \answer{2 \pi \left(2-\frac{1}{3}\ln\left(\frac{x}{4}\right)\right)} \sqrt{\answer{1+\frac{1}{9x^2}}} \d x
\]

\begin{hint}
When $y=\ln(2)$, we have $x= 4e^{3 \ln(2)}$.  There are two ways to simplify this:

Way 1: Use the properties of logarithms to write: $3 \ln(2) = \ln\left(2^3\right)$ so $4e^{3 \ln(2)} = 4e^{\ln(8)} =\answer{32}$.

Way 2: Use the properties of exponents to write $4e^{3 \ln(2)} = 4\left(e^{\ln(2)}\right)^3 = 4(2)^3 = \answer{32}$.

Now, make sure that your limits of integration are in increasing order.
\end{hint}

\begin{exercise}
Using computational software of your choice, the integral to 4 decimal places shows that the surface area is $\answer{343.3015}$ square units.  
\end{exercise}
\end{exercise}


%%%%%%%%%%%%%%%%%

To set up an integral with respect to $y$ that gives the area of the surface of revolution, do the following:  

Since we have chosen to integrate with respect to $y$, we use the result:

\[ SA = \int_{x=a}^{x=b} 2 \pi r \d s\]

and we must express $r$ in terms of $y$ and $\d s$ in terms of $y$ and $\d y$.  


Let's start by finding $\d s$.  Since we integrate with respect to $y$, we use $\d s = \sqrt{1+\left(\frac{\d x}{\d y}\right)^2} \d y$. So: 

\[
\d s = \sqrt{\answer{1+144e^{6y}}} \d y
\]


\begin{exercise}
For the radius $r$, we find: $r=\answer{2-y}$.  


\begin{exercise}
Now we see that an integral that gives the surface area is: 
\[
SA= \int_{y=c}^{y=d} 2 \pi r \d s = \int_{y=\answer{\ln(2)}}^{y=\answer{1}} \answer{ 2\pi (2-y)} \sqrt{\answer{1+144e^{6y}}} \d y
\]

\begin{exercise}
Using computational software of your choice, the integral to 4 decimal places shows that the surface area is $\answer{343.3015}$ square units.  

Does this agree with he previous result?

\begin{multipleChoice}
\choice[correct]{Yes}
\choice{No}
\end{multipleChoice}

\end{exercise}



\end{exercise}
\end{exercise}
\end{exercise}
\end{document}
