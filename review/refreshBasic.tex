\documentclass{ximera}

%\usepackage{todonotes}
%\usepackage{mathtools} %% Required for wide table Curl and Greens
%\usepackage{cuted} %% Required for wide table Curl and Greens
\newcommand{\todo}{}

\usepackage{esint} % for \oiint
\ifxake%%https://math.meta.stackexchange.com/questions/9973/how-do-you-render-a-closed-surface-double-integral
\renewcommand{\oiint}{{\large\bigcirc}\kern-1.56em\iint}
\fi


\graphicspath{
  {./}
  {ximeraTutorial/}
  {basicPhilosophy/}
  {functionsOfSeveralVariables/}
  {normalVectors/}
  {lagrangeMultipliers/}
  {vectorFields/}
  {greensTheorem/}
  {shapeOfThingsToCome/}
  {dotProducts/}
  {partialDerivativesAndTheGradientVector/}
  {../productAndQuotientRules/exercises/}
  {../normalVectors/exercisesParametricPlots/}
  {../continuityOfFunctionsOfSeveralVariables/exercises/}
  {../partialDerivativesAndTheGradientVector/exercises/}
  {../directionalDerivativeAndChainRule/exercises/}
  {../commonCoordinates/exercisesCylindricalCoordinates/}
  {../commonCoordinates/exercisesSphericalCoordinates/}
  {../greensTheorem/exercisesCurlAndLineIntegrals/}
  {../greensTheorem/exercisesDivergenceAndLineIntegrals/}
  {../shapeOfThingsToCome/exercisesDivergenceTheorem/}
  {../greensTheorem/}
  {../shapeOfThingsToCome/}
  {../separableDifferentialEquations/exercises/}
  {vectorFields/}
}

\newcommand{\mooculus}{\textsf{\textbf{MOOC}\textnormal{\textsf{ULUS}}}}

\usepackage{tkz-euclide}\usepackage{tikz}
\usepackage{tikz-cd}
\usetikzlibrary{arrows}
\tikzset{>=stealth,commutative diagrams/.cd,
  arrow style=tikz,diagrams={>=stealth}} %% cool arrow head
\tikzset{shorten <>/.style={ shorten >=#1, shorten <=#1 } } %% allows shorter vectors

\usetikzlibrary{backgrounds} %% for boxes around graphs
\usetikzlibrary{shapes,positioning}  %% Clouds and stars
\usetikzlibrary{matrix} %% for matrix
\usepgfplotslibrary{polar} %% for polar plots
\usepgfplotslibrary{fillbetween} %% to shade area between curves in TikZ
\usetkzobj{all}
\usepackage[makeroom]{cancel} %% for strike outs
%\usepackage{mathtools} %% for pretty underbrace % Breaks Ximera
%\usepackage{multicol}
\usepackage{pgffor} %% required for integral for loops



%% http://tex.stackexchange.com/questions/66490/drawing-a-tikz-arc-specifying-the-center
%% Draws beach ball
\tikzset{pics/carc/.style args={#1:#2:#3}{code={\draw[pic actions] (#1:#3) arc(#1:#2:#3);}}}



\usepackage{array}
\setlength{\extrarowheight}{+.1cm}
\newdimen\digitwidth
\settowidth\digitwidth{9}
\def\divrule#1#2{
\noalign{\moveright#1\digitwidth
\vbox{\hrule width#2\digitwidth}}}





\newcommand{\RR}{\mathbb R}
\newcommand{\R}{\mathbb R}
\newcommand{\N}{\mathbb N}
\newcommand{\Z}{\mathbb Z}

\newcommand{\sagemath}{\textsf{SageMath}}


%\renewcommand{\d}{\,d\!}
\renewcommand{\d}{\mathop{}\!d}
\newcommand{\dd}[2][]{\frac{\d #1}{\d #2}}
\newcommand{\pp}[2][]{\frac{\partial #1}{\partial #2}}
\renewcommand{\l}{\ell}
\newcommand{\ddx}{\frac{d}{\d x}}

\newcommand{\zeroOverZero}{\ensuremath{\boldsymbol{\tfrac{0}{0}}}}
\newcommand{\inftyOverInfty}{\ensuremath{\boldsymbol{\tfrac{\infty}{\infty}}}}
\newcommand{\zeroOverInfty}{\ensuremath{\boldsymbol{\tfrac{0}{\infty}}}}
\newcommand{\zeroTimesInfty}{\ensuremath{\small\boldsymbol{0\cdot \infty}}}
\newcommand{\inftyMinusInfty}{\ensuremath{\small\boldsymbol{\infty - \infty}}}
\newcommand{\oneToInfty}{\ensuremath{\boldsymbol{1^\infty}}}
\newcommand{\zeroToZero}{\ensuremath{\boldsymbol{0^0}}}
\newcommand{\inftyToZero}{\ensuremath{\boldsymbol{\infty^0}}}



\newcommand{\numOverZero}{\ensuremath{\boldsymbol{\tfrac{\#}{0}}}}
\newcommand{\dfn}{\textbf}
%\newcommand{\unit}{\,\mathrm}
\newcommand{\unit}{\mathop{}\!\mathrm}
\newcommand{\eval}[1]{\bigg[ #1 \bigg]}
\newcommand{\seq}[1]{\left( #1 \right)}
\renewcommand{\epsilon}{\varepsilon}
\renewcommand{\phi}{\varphi}


\renewcommand{\iff}{\Leftrightarrow}

\DeclareMathOperator{\arccot}{arccot}
\DeclareMathOperator{\arcsec}{arcsec}
\DeclareMathOperator{\arccsc}{arccsc}
\DeclareMathOperator{\si}{Si}
\DeclareMathOperator{\scal}{scal}
\DeclareMathOperator{\sign}{sign}


%% \newcommand{\tightoverset}[2]{% for arrow vec
%%   \mathop{#2}\limits^{\vbox to -.5ex{\kern-0.75ex\hbox{$#1$}\vss}}}
\newcommand{\arrowvec}[1]{{\overset{\rightharpoonup}{#1}}}
%\renewcommand{\vec}[1]{\arrowvec{\mathbf{#1}}}
\renewcommand{\vec}[1]{{\overset{\boldsymbol{\rightharpoonup}}{\mathbf{#1}}}\hspace{0in}}

\newcommand{\point}[1]{\left(#1\right)} %this allows \vector{ to be changed to \vector{ with a quick find and replace
\newcommand{\pt}[1]{\mathbf{#1}} %this allows \vec{ to be changed to \vec{ with a quick find and replace
\newcommand{\Lim}[2]{\lim_{\point{#1} \to \point{#2}}} %Bart, I changed this to point since I want to use it.  It runs through both of the exercise and exerciseE files in limits section, which is why it was in each document to start with.

\DeclareMathOperator{\proj}{\mathbf{proj}}
\newcommand{\veci}{{\boldsymbol{\hat{\imath}}}}
\newcommand{\vecj}{{\boldsymbol{\hat{\jmath}}}}
\newcommand{\veck}{{\boldsymbol{\hat{k}}}}
\newcommand{\vecl}{\vec{\boldsymbol{\l}}}
\newcommand{\uvec}[1]{\mathbf{\hat{#1}}}
\newcommand{\utan}{\mathbf{\hat{t}}}
\newcommand{\unormal}{\mathbf{\hat{n}}}
\newcommand{\ubinormal}{\mathbf{\hat{b}}}

\newcommand{\dotp}{\bullet}
\newcommand{\cross}{\boldsymbol\times}
\newcommand{\grad}{\boldsymbol\nabla}
\newcommand{\divergence}{\grad\dotp}
\newcommand{\curl}{\grad\cross}
%\DeclareMathOperator{\divergence}{divergence}
%\DeclareMathOperator{\curl}[1]{\grad\cross #1}
\newcommand{\lto}{\mathop{\longrightarrow\,}\limits}

\renewcommand{\bar}{\overline}

\colorlet{textColor}{black}
\colorlet{background}{white}
\colorlet{penColor}{blue!50!black} % Color of a curve in a plot
\colorlet{penColor2}{red!50!black}% Color of a curve in a plot
\colorlet{penColor3}{red!50!blue} % Color of a curve in a plot
\colorlet{penColor4}{green!50!black} % Color of a curve in a plot
\colorlet{penColor5}{orange!80!black} % Color of a curve in a plot
\colorlet{penColor6}{yellow!70!black} % Color of a curve in a plot
\colorlet{fill1}{penColor!20} % Color of fill in a plot
\colorlet{fill2}{penColor2!20} % Color of fill in a plot
\colorlet{fillp}{fill1} % Color of positive area
\colorlet{filln}{penColor2!20} % Color of negative area
\colorlet{fill3}{penColor3!20} % Fill
\colorlet{fill4}{penColor4!20} % Fill
\colorlet{fill5}{penColor5!20} % Fill
\colorlet{gridColor}{gray!50} % Color of grid in a plot

\newcommand{\surfaceColor}{violet}
\newcommand{\surfaceColorTwo}{redyellow}
\newcommand{\sliceColor}{greenyellow}




\pgfmathdeclarefunction{gauss}{2}{% gives gaussian
  \pgfmathparse{1/(#2*sqrt(2*pi))*exp(-((x-#1)^2)/(2*#2^2))}%
}


%%%%%%%%%%%%%
%% Vectors
%%%%%%%%%%%%%

%% Simple horiz vectors
\renewcommand{\vector}[1]{\left\langle #1\right\rangle}


%% %% Complex Horiz Vectors with angle brackets
%% \makeatletter
%% \renewcommand{\vector}[2][ , ]{\left\langle%
%%   \def\nextitem{\def\nextitem{#1}}%
%%   \@for \el:=#2\do{\nextitem\el}\right\rangle%
%% }
%% \makeatother

%% %% Vertical Vectors
%% \def\vector#1{\begin{bmatrix}\vecListA#1,,\end{bmatrix}}
%% \def\vecListA#1,{\if,#1,\else #1\cr \expandafter \vecListA \fi}

%%%%%%%%%%%%%
%% End of vectors
%%%%%%%%%%%%%

%\newcommand{\fullwidth}{}
%\newcommand{\normalwidth}{}



%% makes a snazzy t-chart for evaluating functions
%\newenvironment{tchart}{\rowcolors{2}{}{background!90!textColor}\array}{\endarray}

%%This is to help with formatting on future title pages.
\newenvironment{sectionOutcomes}{}{}



%% Flowchart stuff
%\tikzstyle{startstop} = [rectangle, rounded corners, minimum width=3cm, minimum height=1cm,text centered, draw=black]
%\tikzstyle{question} = [rectangle, minimum width=3cm, minimum height=1cm, text centered, draw=black]
%\tikzstyle{decision} = [trapezium, trapezium left angle=70, trapezium right angle=110, minimum width=3cm, minimum height=1cm, text centered, draw=black]
%\tikzstyle{question} = [rectangle, rounded corners, minimum width=3cm, minimum height=1cm,text centered, draw=black]
%\tikzstyle{process} = [rectangle, minimum width=3cm, minimum height=1cm, text centered, draw=black]
%\tikzstyle{decision} = [trapezium, trapezium left angle=70, trapezium right angle=110, minimum width=3cm, minimum height=1cm, text centered, draw=black]


\title[Refresh:]{Basic material}

\begin{document}
\begin{abstract}
  We review basic material for this course. 
\end{abstract}
\maketitle

\begin{problem}
  Compute
  \[
  9-16\cdot 0 + 8\div 2 = \answer{13}
  \]
\end{problem}


\begin{problem}
  Select all that are correct:
  \begin{selectAll}
    \choice[correct]{$\frac{4x^2-2x}{3x}= \frac{x(4x-2)}{3x}=\frac{4x-2}{3}$}
    \choice{$\frac{2x}{1+x^2}=\frac{2x}{1}+\frac{2x}{x^2}=2x+\frac{2}{x}$}
    \choice{$(x-1)^3 = x^3-1$}
    \choice{$\frac{1}{5x} = 5x^{-1}$}
    \choice[correct]{$(2x)^3=2^3x^3=8x^3$}
    \choice{$2^{3+2k}=2^3+2^{2k}$}
  \end{selectAll}
\end{problem}


\begin{problem}
  Find the missing quantities that complete the square:
  \[
   x^2 + 20x +\answer{100} = (\answer{x+10})^2 
   \]
\end{problem}


\begin{problem}
  Find the missing quantities that complete the square:
  \[
  x^2-12x + \answer{36} = (\answer{x-6})^2
  \]
\end{problem}


\begin{problem}
  Use a change of variables to compute the following indefinite
  integral.
  \[
  \int \frac{4x^2}{\sqrt{5 - 8x^3}} \d x
  \]
  What is the best choice of $u$ for the change of variables?
  \[
  u = \answer{5 - 8x^3}
  \]
  Find $\d u$.
  \[
  \d u = \answer{-24x^2} \d x
  \]
  Rewrite the given integral using this change of variables:
  
  \[
  \int \frac{4x^2}{\sqrt{5 - 8x^3}} \d x = \int \answer{\frac{-1}{6} u^{-1/2}} \d u
  \]
  Find the indefinite integral:
  \[
  \int \frac{4x^2}{\sqrt{5 - 8x^3}} \d x = \answer{\frac{-1}{3}(5-8x^3)^{1/2} + C}
  \]
  (Use $C$ as the arbitrary constant.)
\end{problem}


\begin{problem}
  Find the indefinite integral.
  \[
    \int \frac{x}{\sqrt{x - 1}} \d x = \answer{\frac{2}{3}(x-1)^{3/2} + 2(x - 1)^{1/2} + C}
  \]  
  (Use $C$ as the arbitrary constant.
\end{problem}


\begin{problem}
  Use a change of variable to evaluate the following definite integral.
  \[
    \int_{-\pi/2}^{-\pi/4} -3 \cot(x) \csc^2(x) \d x = \answer{3/2} \text{(Type an exact answer.)}
  \]
\end{problem}

\begin{problem}
  Find a function $f$ such that $\int f(x) \d x = \cos(x^2) + C$.
  \[
  f(x) = \answer{-2 x \sin(x^2)}
  \]
\end{problem}

\begin{problem}
  A student claims that $\int \frac{1}{1-x^2} \d x = \ln(1 - x^2) + C$.
  Determine if the student is correct.
  If the student is incorrect, choose the best explanation that follows.
  \begin{multipleChoice}
    \choice{The student is correct.}
    \choice[correct]{The student is incorrect.}
  \end{multipleChoice}
  \begin{problem}
    If the student's claim had been correct, then
    \[
    \dd{x} \ln(1 - x^2) = \answer{\frac{1}{1-x^2}}.
    \]
    However,
    \[
    \dd{x} \ln(1 - x^2) = \answer{\frac{-2x}{1 - x^2}}
    \]
  \end{problem}
\end{problem}


\begin{problem}
  A student is asked to compute $\int x \sqrt{1 - x^2} \d x$ on a midterm and provides the following solution:
  \[
    \int x \sqrt{1 - x^2} \d x= \int x(1 -x) \d x = \int (x - x^2 )\d x = \frac{1}{2}x^2 - \frac{1}{3}x^3 + C.
  \]
  Determine if the student is correct.
  If the student is not correct, choose the most appropriate response below.
  \begin{multipleChoice}
    \choice{The student is correct.}
    \choice[correct]{The student is incorrect.}
  \end{multipleChoice}
  \begin{problem}
    The mistake was in asserting that $\sqrt{1 - x^2} \ne 1 - x$.  To
    compute this antiderivative, the student should perform a
    $u$-substitution with $u = \answer{1 - x^2}$.  Upon making this
    substitution, the correct antiderivative of $y = x \sqrt{1 - x^2}$
    is $\answer{(-1/3)(1 - x^2)^{3/2}} + C$.
  \end{problem}
\end{problem}



\begin{problem}
  A student is asked to compute $\int 3x^2 \cos(x) \d x$ on a midterm and provides the following solution:
  \[
    \int 3x^2 \cos(x) \d x = x^3 \sin(x) + C.
  \]
  Determine if the student is correct.
  If the student is not correct, choose the most appropriate response below.
  \begin{multipleChoice}
    \choice{The student is correct}
    \choice{The student is incorrect; $\int 3x^2 \cos(x) \d x = -x^3 \sin(x) + C$.}
    \choice[correct]{The student is incorrect.}
  \end{multipleChoice}
  \begin{problem}
    The mistake was in attempting to split the integral over the
    product.  If the student were correct, then using the definition
    of the antiderivative, $\dd{x}(x^3\sin(x) + C) = 3x^2 \cos(x)$.
    However,
    \[
    \dd{x}(x^3 \sin(x) + C) = \answer{3x^2 \sin(x) + x^3 \cos(x)}
    \]
  \end{problem}
  \begin{hint}
    To verify an antiderivative, differentiate and compare.
  \end{hint}
\end{problem}



\begin{problem} Calculate:
\[
\int x^2 +1 \d x = \answer{x^3/3 + x+ C}
\]
\begin{hint}
  Use $C$ for the constant. 
\end{hint}
\end{problem}

\begin{problem} Calculate:
\[
\int x^3 +6 x + 1 \d x = \answer{x^4/4 + 3x^2 + x+ C}
\]
\begin{hint}
  Use $C$ for the constant. 
\end{hint}
\end{problem}


\begin{problem} Calculate:
\[
\int 3x^4 + \frac{x}{2} \d x = \answer{3x^5/5 + x^2/4 + C}
\]
\begin{hint}
  Use $C$ for the constant. 
\end{hint}
\end{problem}



\begin{problem} Calculate:
\[
\int \pi + \sin(x) \d x = \answer{\pi x - \cos(x)+ C}
\]
\begin{hint}
  Use $C$ for the constant. 
\end{hint}
\end{problem}


\begin{problem} Calculate:
\[
\int 1+ 2x + e^x \d x = \answer{x + x^2 + e^x+C}
\]
\begin{hint}
  Use $C$ for the constant. 
\end{hint}
\end{problem}

\begin{problem} Calculate:
\[
\int \frac{5}{x^5} + \frac{4}{x^4} \d x = \answer{\frac{-5}{4 x^4}-\frac{4}{3 x^3}+C}
\]
\begin{hint}
  Use $C$ for the constant. 
\end{hint}
\end{problem}


\begin{problem} Calculate:
\[
\int \frac{7}{x^6} + \frac{x^6}{7} \d x = \answer{\frac{x^7}{49}-\frac{7}{5 x^5}+C}
\]
\begin{hint}
  Use $C$ for the constant. 
\end{hint}
\end{problem}


\begin{problem} Calculate:
\[
\int \cos(x) - 2x^2 \d x = \answer{\sin(x) - 2x^3/3+C}
\]
\begin{hint}
  Use $C$ for the constant. 
\end{hint}
\end{problem}

\begin{problem} Calculate:
\[
\int e^x - 1 - x \d x = \answer{e^x-x-x^2/2+C}
\]
\begin{hint}
  Use $C$ for the constant. 
\end{hint}
\end{problem}



\begin{problem} Calculate:
\[
\int 16 \d u = \answer{16u+C}
\]
\begin{hint}
  Use $C$ for the constant. 
\end{hint}
\end{problem}


\begin{problem} Calculate:
\[
\int  \d \theta = \answer{\theta+C}
\]
\begin{hint}
  Use $C$ for the constant. 
\end{hint}
\end{problem}

\begin{problem} Calculate:
\[
\int \pi \d \theta = \answer{\pi\theta +C}
\]
\begin{hint}
  Use $C$ for the constant. 
\end{hint}
\end{problem}


\begin{problem} Calculate:
\[
\int e \d t = \answer{e t + C}
\]
\begin{hint}
  Use $C$ for the constant. 
\end{hint}
\end{problem}



\begin{problem} Calculate:
\[
\int -1 \d u = \answer{-u+C}
\]
\begin{hint}
  Use $C$ for the constant. 
\end{hint}
\end{problem}


\begin{problem} Calculate:
\[
\int 7 \d z = \answer{7z+C}
\]
\begin{hint}
  Use $C$ for the constant. 
\end{hint}
\end{problem}


\begin{problem} Calculate:
\[
\int 0 \d x = \answer{0+C}
\]
\begin{hint}
  Use $C$ for the constant. 
\end{hint}
\end{problem}

\begin{problem} Calculate:
\[
\int -3 \d \theta = \answer{-3\theta+C}
\]
\begin{hint}
  Use $C$ for the constant. 
\end{hint}
\end{problem}

\begin{problem} Calculate:
\[
\int  \d t = \answer{t+C}
\]
\begin{hint}
  Use $C$ for the constant. 
\end{hint}
\end{problem}



\begin{problem} Calculate:
\[
\int_0^{1/2} \frac{1}{\sqrt{1-x^2}} \d x =\answer{\frac{\pi }{6}}
\]
\end{problem}


\begin{problem} Calculate:
\[
\int_0^1 \frac{1}{x^2 + 1} \d x = \answer{\frac{\pi }{4}}
\]
\end{problem}



\begin{problem} Calculate:
\[
\int_0^1 \frac{1}{x + 1} \d x = \answer{\ln(2)}
\]
\end{problem}


\begin{problem} Calculate:
\[
\int_0^{1/2} \frac{4}{\sqrt{1 - x^2}} \d x = \answer{\frac{2\pi }{3}}
\]
\end{problem}


\begin{problem} Calculate:
\[
\int_0^{1/2} \frac{-1}{\sqrt{1 - x^2}} \d x = \answer{\frac{-\pi }{6}}
\]
\end{problem}


\begin{problem} Calculate:
\[
\int_2^3 \sec^2 \theta  \d \theta = \answer{\sin(1) \sec(2) \sec(3)}
\]
\end{problem}



\begin{problem} Calculate:
\[
\int_2^3 \sec(\theta)\cdot \tan(\theta)  \d \theta = \answer{\sec(3)-\sec(2)}
\]
\end{problem}



\begin{problem} Calculate:
\[
\int_2^3 3\sec^2\theta  \d \theta = \answer{3 \sin (1) \sec (2) \sec (3)}
\]
\end{problem}


\begin{problem} Calculate:
\[
\int_0^1 e^{-x/2} \d x = \answer{2-\frac{2}{\sqrt{e}}}
\]
\end{problem}



\begin{problem} Calculate:
\[
\int_0^1 2e^{4x} \d x = \answer{\frac{1}{2} (e^4-1)}
\]
\end{problem}




\begin{problem} Calculate:
\[
\int_0^3 \cos(3x) \d x = \answer{\frac{\sin (9)}{3}}
\]
\end{problem}


\begin{problem} Calculate:
\[
\int_0^{1/2} \cos(\pi x) \d x = \answer{\frac{1}{\pi }}
\]
\end{problem}




\begin{problem} Calculate:
\[
\int_1^2 3\sin(3x) \d x = \answer{\cos (3)-\cos (6)}
\]
\end{problem}



\begin{problem} Calculate:
\[
\int_1^4 \pi \sin(- x) \d x = \answer{\pi  (\cos (4)-\cos (1))}
\]
\end{problem}



\begin{problem} Calculate:
\[
\int_3^5 \frac{\sin(\pi x)}{\pi} \d x = \answer{0}
\]
\end{problem}



\begin{problem} Calculate:
\[
\int_1^2 \cos(7 x) \d x= \answer{\frac{1}{7} (\sin (14)-\sin (7))}
\]
\end{problem}



\begin{problem} Calculate:
\[
\int_3^4 \frac{\cos(2 x)}{3} \d x=\answer{\frac{1}{6} (\sin (8)-\sin (6))}
\]
\end{problem}


\begin{problem} Calculate:
\[
\int_0^{1/2} \pi \cos(x/\pi) \d x = \answer{\pi ^2 \sin (\frac{1}{2 \pi })}
\]
\end{problem}


\begin{problem} Calculate:
\[
\int_0^1 \frac{\sin(x/\pi)}{\pi} \d x = \answer{1-\cos \left(\frac{1}{\pi }\right)}
\]
\end{problem}



\begin{problem} Calculate:
\[
\int_0^2 \pi\sin(x/\pi) \d x =\answer{2 \pi ^2 \sin ^2\left(\frac{1}{\pi }\right)}
\]
\end{problem}



\begin{problem} Calculate:
\[
\int_0^1 \frac{3}{(2x+1)} \d x = \answer{\frac{3 \ln (3)}{2}}
\]
\end{problem}


\begin{problem} Calculate:
\[
\int_0^1 \frac{1}{5-3x} \d x \answer{\frac{\ln(5/2)}{3}}
\]
\end{problem}

\begin{problem} Calculate:
\[
\int_1^2 \frac{x^3}{1 + x^4} \d x = \answer{\frac{1}{4} \ln(\frac{17}{2})}
\]
\end{problem}



\begin{problem} Calculate:
\[
\int_0^1 \frac{2+ e^{2x}}{4x + e^{2x}} \d x = \answer{\frac{1}{2} \ln(4+e^2)}
\]
\end{problem}



\begin{problem} Calculate:
\[
\int_0^1 \frac{1}{1 + x} \d x =\answer{\ln(2)}
\]
\end{problem}


\begin{problem} Calculate:
\[
\int_3^6 \frac{x^2+4x^3}{x^3+3x^4} \d x =\answer{\frac{1}{3} \ln(\frac{76}{5})}
\]
\end{problem}



\begin{problem} Calculate:
\[
\int_2^5 \frac{1}{-3-3x} \d x =\answer{\frac{-\ln(2)}{3}}
\]
\end{problem}





\end{document}
