\documentclass{ximera}

%\usepackage{todonotes}
%\usepackage{mathtools} %% Required for wide table Curl and Greens
%\usepackage{cuted} %% Required for wide table Curl and Greens
\newcommand{\todo}{}

\usepackage{esint} % for \oiint
\ifxake%%https://math.meta.stackexchange.com/questions/9973/how-do-you-render-a-closed-surface-double-integral
\renewcommand{\oiint}{{\large\bigcirc}\kern-1.56em\iint}
\fi


\graphicspath{
  {./}
  {ximeraTutorial/}
  {basicPhilosophy/}
  {functionsOfSeveralVariables/}
  {normalVectors/}
  {lagrangeMultipliers/}
  {vectorFields/}
  {greensTheorem/}
  {shapeOfThingsToCome/}
  {dotProducts/}
  {partialDerivativesAndTheGradientVector/}
  {../productAndQuotientRules/exercises/}
  {../normalVectors/exercisesParametricPlots/}
  {../continuityOfFunctionsOfSeveralVariables/exercises/}
  {../partialDerivativesAndTheGradientVector/exercises/}
  {../directionalDerivativeAndChainRule/exercises/}
  {../commonCoordinates/exercisesCylindricalCoordinates/}
  {../commonCoordinates/exercisesSphericalCoordinates/}
  {../greensTheorem/exercisesCurlAndLineIntegrals/}
  {../greensTheorem/exercisesDivergenceAndLineIntegrals/}
  {../shapeOfThingsToCome/exercisesDivergenceTheorem/}
  {../greensTheorem/}
  {../shapeOfThingsToCome/}
  {../separableDifferentialEquations/exercises/}
  {vectorFields/}
}

\newcommand{\mooculus}{\textsf{\textbf{MOOC}\textnormal{\textsf{ULUS}}}}

\usepackage{tkz-euclide}\usepackage{tikz}
\usepackage{tikz-cd}
\usetikzlibrary{arrows}
\tikzset{>=stealth,commutative diagrams/.cd,
  arrow style=tikz,diagrams={>=stealth}} %% cool arrow head
\tikzset{shorten <>/.style={ shorten >=#1, shorten <=#1 } } %% allows shorter vectors

\usetikzlibrary{backgrounds} %% for boxes around graphs
\usetikzlibrary{shapes,positioning}  %% Clouds and stars
\usetikzlibrary{matrix} %% for matrix
\usepgfplotslibrary{polar} %% for polar plots
\usepgfplotslibrary{fillbetween} %% to shade area between curves in TikZ
\usetkzobj{all}
\usepackage[makeroom]{cancel} %% for strike outs
%\usepackage{mathtools} %% for pretty underbrace % Breaks Ximera
%\usepackage{multicol}
\usepackage{pgffor} %% required for integral for loops



%% http://tex.stackexchange.com/questions/66490/drawing-a-tikz-arc-specifying-the-center
%% Draws beach ball
\tikzset{pics/carc/.style args={#1:#2:#3}{code={\draw[pic actions] (#1:#3) arc(#1:#2:#3);}}}



\usepackage{array}
\setlength{\extrarowheight}{+.1cm}
\newdimen\digitwidth
\settowidth\digitwidth{9}
\def\divrule#1#2{
\noalign{\moveright#1\digitwidth
\vbox{\hrule width#2\digitwidth}}}





\newcommand{\RR}{\mathbb R}
\newcommand{\R}{\mathbb R}
\newcommand{\N}{\mathbb N}
\newcommand{\Z}{\mathbb Z}

\newcommand{\sagemath}{\textsf{SageMath}}


%\renewcommand{\d}{\,d\!}
\renewcommand{\d}{\mathop{}\!d}
\newcommand{\dd}[2][]{\frac{\d #1}{\d #2}}
\newcommand{\pp}[2][]{\frac{\partial #1}{\partial #2}}
\renewcommand{\l}{\ell}
\newcommand{\ddx}{\frac{d}{\d x}}

\newcommand{\zeroOverZero}{\ensuremath{\boldsymbol{\tfrac{0}{0}}}}
\newcommand{\inftyOverInfty}{\ensuremath{\boldsymbol{\tfrac{\infty}{\infty}}}}
\newcommand{\zeroOverInfty}{\ensuremath{\boldsymbol{\tfrac{0}{\infty}}}}
\newcommand{\zeroTimesInfty}{\ensuremath{\small\boldsymbol{0\cdot \infty}}}
\newcommand{\inftyMinusInfty}{\ensuremath{\small\boldsymbol{\infty - \infty}}}
\newcommand{\oneToInfty}{\ensuremath{\boldsymbol{1^\infty}}}
\newcommand{\zeroToZero}{\ensuremath{\boldsymbol{0^0}}}
\newcommand{\inftyToZero}{\ensuremath{\boldsymbol{\infty^0}}}



\newcommand{\numOverZero}{\ensuremath{\boldsymbol{\tfrac{\#}{0}}}}
\newcommand{\dfn}{\textbf}
%\newcommand{\unit}{\,\mathrm}
\newcommand{\unit}{\mathop{}\!\mathrm}
\newcommand{\eval}[1]{\bigg[ #1 \bigg]}
\newcommand{\seq}[1]{\left( #1 \right)}
\renewcommand{\epsilon}{\varepsilon}
\renewcommand{\phi}{\varphi}


\renewcommand{\iff}{\Leftrightarrow}

\DeclareMathOperator{\arccot}{arccot}
\DeclareMathOperator{\arcsec}{arcsec}
\DeclareMathOperator{\arccsc}{arccsc}
\DeclareMathOperator{\si}{Si}
\DeclareMathOperator{\scal}{scal}
\DeclareMathOperator{\sign}{sign}


%% \newcommand{\tightoverset}[2]{% for arrow vec
%%   \mathop{#2}\limits^{\vbox to -.5ex{\kern-0.75ex\hbox{$#1$}\vss}}}
\newcommand{\arrowvec}[1]{{\overset{\rightharpoonup}{#1}}}
%\renewcommand{\vec}[1]{\arrowvec{\mathbf{#1}}}
\renewcommand{\vec}[1]{{\overset{\boldsymbol{\rightharpoonup}}{\mathbf{#1}}}\hspace{0in}}

\newcommand{\point}[1]{\left(#1\right)} %this allows \vector{ to be changed to \vector{ with a quick find and replace
\newcommand{\pt}[1]{\mathbf{#1}} %this allows \vec{ to be changed to \vec{ with a quick find and replace
\newcommand{\Lim}[2]{\lim_{\point{#1} \to \point{#2}}} %Bart, I changed this to point since I want to use it.  It runs through both of the exercise and exerciseE files in limits section, which is why it was in each document to start with.

\DeclareMathOperator{\proj}{\mathbf{proj}}
\newcommand{\veci}{{\boldsymbol{\hat{\imath}}}}
\newcommand{\vecj}{{\boldsymbol{\hat{\jmath}}}}
\newcommand{\veck}{{\boldsymbol{\hat{k}}}}
\newcommand{\vecl}{\vec{\boldsymbol{\l}}}
\newcommand{\uvec}[1]{\mathbf{\hat{#1}}}
\newcommand{\utan}{\mathbf{\hat{t}}}
\newcommand{\unormal}{\mathbf{\hat{n}}}
\newcommand{\ubinormal}{\mathbf{\hat{b}}}

\newcommand{\dotp}{\bullet}
\newcommand{\cross}{\boldsymbol\times}
\newcommand{\grad}{\boldsymbol\nabla}
\newcommand{\divergence}{\grad\dotp}
\newcommand{\curl}{\grad\cross}
%\DeclareMathOperator{\divergence}{divergence}
%\DeclareMathOperator{\curl}[1]{\grad\cross #1}
\newcommand{\lto}{\mathop{\longrightarrow\,}\limits}

\renewcommand{\bar}{\overline}

\colorlet{textColor}{black}
\colorlet{background}{white}
\colorlet{penColor}{blue!50!black} % Color of a curve in a plot
\colorlet{penColor2}{red!50!black}% Color of a curve in a plot
\colorlet{penColor3}{red!50!blue} % Color of a curve in a plot
\colorlet{penColor4}{green!50!black} % Color of a curve in a plot
\colorlet{penColor5}{orange!80!black} % Color of a curve in a plot
\colorlet{penColor6}{yellow!70!black} % Color of a curve in a plot
\colorlet{fill1}{penColor!20} % Color of fill in a plot
\colorlet{fill2}{penColor2!20} % Color of fill in a plot
\colorlet{fillp}{fill1} % Color of positive area
\colorlet{filln}{penColor2!20} % Color of negative area
\colorlet{fill3}{penColor3!20} % Fill
\colorlet{fill4}{penColor4!20} % Fill
\colorlet{fill5}{penColor5!20} % Fill
\colorlet{gridColor}{gray!50} % Color of grid in a plot

\newcommand{\surfaceColor}{violet}
\newcommand{\surfaceColorTwo}{redyellow}
\newcommand{\sliceColor}{greenyellow}




\pgfmathdeclarefunction{gauss}{2}{% gives gaussian
  \pgfmathparse{1/(#2*sqrt(2*pi))*exp(-((x-#1)^2)/(2*#2^2))}%
}


%%%%%%%%%%%%%
%% Vectors
%%%%%%%%%%%%%

%% Simple horiz vectors
\renewcommand{\vector}[1]{\left\langle #1\right\rangle}


%% %% Complex Horiz Vectors with angle brackets
%% \makeatletter
%% \renewcommand{\vector}[2][ , ]{\left\langle%
%%   \def\nextitem{\def\nextitem{#1}}%
%%   \@for \el:=#2\do{\nextitem\el}\right\rangle%
%% }
%% \makeatother

%% %% Vertical Vectors
%% \def\vector#1{\begin{bmatrix}\vecListA#1,,\end{bmatrix}}
%% \def\vecListA#1,{\if,#1,\else #1\cr \expandafter \vecListA \fi}

%%%%%%%%%%%%%
%% End of vectors
%%%%%%%%%%%%%

%\newcommand{\fullwidth}{}
%\newcommand{\normalwidth}{}



%% makes a snazzy t-chart for evaluating functions
%\newenvironment{tchart}{\rowcolors{2}{}{background!90!textColor}\array}{\endarray}

%%This is to help with formatting on future title pages.
\newenvironment{sectionOutcomes}{}{}



%% Flowchart stuff
%\tikzstyle{startstop} = [rectangle, rounded corners, minimum width=3cm, minimum height=1cm,text centered, draw=black]
%\tikzstyle{question} = [rectangle, minimum width=3cm, minimum height=1cm, text centered, draw=black]
%\tikzstyle{decision} = [trapezium, trapezium left angle=70, trapezium right angle=110, minimum width=3cm, minimum height=1cm, text centered, draw=black]
%\tikzstyle{question} = [rectangle, rounded corners, minimum width=3cm, minimum height=1cm,text centered, draw=black]
%\tikzstyle{process} = [rectangle, minimum width=3cm, minimum height=1cm, text centered, draw=black]
%\tikzstyle{decision} = [trapezium, trapezium left angle=70, trapezium right angle=110, minimum width=3cm, minimum height=1cm, text centered, draw=black]


\title[Dig-In:]{Integration by substitution}

\begin{document}
\begin{abstract}
\end{abstract}
\maketitle


Computing antiderivatives is not as easy as computing derivatives. One
issue is that the chain rule can be difficult to ``undo.'' Sometimes 
it is helpful to transform the integral in question via substitution. 


\begin{theorem}[Integral Substitution Formula] 
If $u(x)$ is differentiable on the interval $[a,b]$ and $f(x)$ is
differentiable on the interval $[u(a),u(b)]$, then
\[
\int_a^b f'(u(x)) u'(x) \d x =\int_{u(a)}^{u(b)} f'(u) \d u.
\]
\end{theorem}
\begin{proof} First we recognize the chain rule
\[
\int_a^b f'(u(x)) u'(x) \d x = \int_a^b (f\circ u)'(x) \d x.
\]
Next we apply the Fundamental Theorem of Calculus. 
\begin{align*} 
\int_a^b (f\circ u)'(x) \d x &= \eval{f(u(x))}_a^b \\
&= \eval{f(x)}_{u(a)}^{u(b)}\\ 
&= \int_{g(a)}^{g(b)} f'(u) \d u.
\end{align*}
\end{proof}


There are several different ways to think about substitution. The
first is directly using the formula
\[
\int_a^b f'(u(x)) u'(x) \d x = \int_{u(a)}^{u(b)} f'(u) \d u.
\]
\begin{example}
Compute
\[
\int_1^3 x\cos(x^2)\d x.
\]


A little thought reveals that if $x\cos(x^2)$ is the derivative of
some function, then it must have come from an application of the chain
rule. Here we have $x$ on the ``outside,'' which is the derivative of
$x^2$ on the ``inside,'' 
\[
\int \underbrace{x}_{\text{outside}}\cos(\underbrace{x^2}_{\text{inside}})\d x.
\]
Set $u(x) = x^2$ so $u'(x) = 2x$ and now it must be that $f(u) =
\frac{\cos(u)}{2}$. Now we see
\begin{align*}
\int_1^3 x\cos(x^2)\d x &= \int_1^9 \frac{\cos(u)}{2}\d u\\
&= \eval{\frac{\sin(u)}{2}}_1^9 \\
&= \frac{\sin(9) -\sin(1)}{2}.
\end{align*}
\end{example}

Sometimes we frame the solution in a different way. Let's do the same
example again, this time we'll think in terms of differentials.

\begin{example}
Compute
\[
\int_1^3 x\cos(x^2)\d x.
\]


Here we will set $u=x^2$. Now $du = 2x \d x$, we are thinking in terms
of differentials. Now we see
\[
\int_{u(1)}^{u(3)} \frac{\cos(u)}{2}\d u = \int_1^3\frac{\cos(x^2)}{2}2x\d x.
\]
At this point, we can continue as we did before and write
\[
\int_1^3 x\cos(x^2)\d x= \frac{\sin(9) -\sin(1)}{2}.
\]
\end{example}

Finally, sometimes we simply want to deal with the antiderivative on
its own, we'll repeat the example one more time demonstrating this.

\begin{example}
Compute
\[
\int_1^3 x\cos(x^2)\d x.
\]


Here we start as we did before, setting $u=x^2$. Now $du = 2x \d x$,
again thinking in terms of differentials. Now we see
\[
\int  \frac{\cos(u)}{2}\d u = \int \frac{\cos(x^2)}{2}2x\d x.
\]
Hence 
\[
\int x\cos(x^2)\d x = \frac{\sin(u)}{2} = \frac{\sin(x^2)}{2}.
\]
Now we see
\begin{align*}
\int_1^3 x\cos(x^2)\d x &=\eval{\frac{\sin(x^2)}{2}}_1^3\\
&= \frac{\sin(9) -\sin(1)}{2}.
\end{align*}
\end{example}

With some experience, it is not hard to see which function is $f(x)$
and which is $u(x)$, let's see another example.
\begin{example}
Compute
\[
\int x^4(x^5+1)^{99} \d x.
\]


Here we set $u = x^5+1$ so $du = 5x^4 \d x$, and $f(u) = \frac{u^{99}}{5}$. Now
\begin{align*}
\int x^4(x^5+1)^{99} \d x &= \int \frac{u^{99}}{5} \d u\\
&= \frac{u^{100}}{500}.
\end{align*}
Recalling that $u = x^5+1$, we have our final answer
\[
\int x^4(x^5+1)^{99} \d x= \frac{(x^5+1)^{100}}{500}+C.
\]
\end{example}


Our next example is a bit different.

\begin{example}
Compute
\[
\int_{2}^{3} \frac{1}{x\ln(x)} \d x.
\]


Let $u=\ln(x)$ so $du=\frac{1}{x}\d x$. Write
\begin{align*}
\int_{2}^{3} \frac{1}{x\ln(x)} \d x = \int_{\ln(2)}^{\ln(3)} \frac{1}{u} \d u\\
&= \eval{\ln(u)}_{\ln(2)}^{\ln(3)}\\
& = \ln(\ln(3)) - \ln(\ln(2)).
\end{align*}
\end{example}


On the other hand our next example is much harder.

\begin{example} Compute
\[
\int x^3\sqrt{1-x^2}\d x.
\]


Here it is not apparent that the chain rule is involved. However, if
it was involved, perhaps a good guess for $u$ would be
\[
u = 1-x^2
\]
in this case
\[
du = -2x \d x.
\]
Now consider our indefinite integral
\[
\int x^3\sqrt{1-x^2}\d x,
\]
immediately we can substitute. Write
\[
\int x^3\sqrt{1-x^2}\d x = \int -\frac{x^2\sqrt{u}}{2}\d u.
\]
However, we cannot continue until each $x$ is replaced. We know however that 
\begin{align*}
u &= 1-x^2 \\
u -1 &= -x^2\\
1- u &= x^2
\end{align*}
so now we may write
\[
\int x^3\sqrt{1-x^2}\d x = \int -\frac{(1-u)\sqrt{u}}{2}\d u.
\]
At this point, we are close to being done. Write
\begin{align*}
\int -\frac{(1-u)\sqrt{u}}{2}\d u &= \int \left(\frac{u\sqrt{u}}{2} - \frac{\sqrt{u}}{2}\right) \d u \\
&= \int \frac{u^{3/2}}{2} \d u - \int \frac{\sqrt{u}}{2} \d u \\
&= \frac{u^{5/2}}{5} - \frac{u^{3/2}}{3}.
\end{align*}
Now recall that $u = 1-x^2$. Hence our final answer is
\[
\int x^3\sqrt{1-x^2}\d x = \frac{(1-x^2)^{5/2}}{5} - \frac{(1-x^2)^{3/2}}{3}+C.
\]
\end{example}

To summarize, if we suspect that a given function is the derivative of
another via the chain rule, we let $u$ denote a likely candidate for
the inner function, then translate the given function so that it is
written entirely in terms of $u$, with no $x$ remaining in the
expression. If we can integrate this new function of $u$, then the
antiderivative of the original function is obtained by replacing $u$
by the equivalent expression in $x$.


\end{document}
