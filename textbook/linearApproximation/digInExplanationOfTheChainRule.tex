\documentclass{ximera}

%\usepackage{todonotes}
%\usepackage{mathtools} %% Required for wide table Curl and Greens
%\usepackage{cuted} %% Required for wide table Curl and Greens
\newcommand{\todo}{}

\usepackage{esint} % for \oiint
\ifxake%%https://math.meta.stackexchange.com/questions/9973/how-do-you-render-a-closed-surface-double-integral
\renewcommand{\oiint}{{\large\bigcirc}\kern-1.56em\iint}
\fi


\graphicspath{
  {./}
  {ximeraTutorial/}
  {basicPhilosophy/}
  {functionsOfSeveralVariables/}
  {normalVectors/}
  {lagrangeMultipliers/}
  {vectorFields/}
  {greensTheorem/}
  {shapeOfThingsToCome/}
  {dotProducts/}
  {partialDerivativesAndTheGradientVector/}
  {../productAndQuotientRules/exercises/}
  {../normalVectors/exercisesParametricPlots/}
  {../continuityOfFunctionsOfSeveralVariables/exercises/}
  {../partialDerivativesAndTheGradientVector/exercises/}
  {../directionalDerivativeAndChainRule/exercises/}
  {../commonCoordinates/exercisesCylindricalCoordinates/}
  {../commonCoordinates/exercisesSphericalCoordinates/}
  {../greensTheorem/exercisesCurlAndLineIntegrals/}
  {../greensTheorem/exercisesDivergenceAndLineIntegrals/}
  {../shapeOfThingsToCome/exercisesDivergenceTheorem/}
  {../greensTheorem/}
  {../shapeOfThingsToCome/}
  {../separableDifferentialEquations/exercises/}
  {vectorFields/}
}

\newcommand{\mooculus}{\textsf{\textbf{MOOC}\textnormal{\textsf{ULUS}}}}

\usepackage{tkz-euclide}\usepackage{tikz}
\usepackage{tikz-cd}
\usetikzlibrary{arrows}
\tikzset{>=stealth,commutative diagrams/.cd,
  arrow style=tikz,diagrams={>=stealth}} %% cool arrow head
\tikzset{shorten <>/.style={ shorten >=#1, shorten <=#1 } } %% allows shorter vectors

\usetikzlibrary{backgrounds} %% for boxes around graphs
\usetikzlibrary{shapes,positioning}  %% Clouds and stars
\usetikzlibrary{matrix} %% for matrix
\usepgfplotslibrary{polar} %% for polar plots
\usepgfplotslibrary{fillbetween} %% to shade area between curves in TikZ
\usetkzobj{all}
\usepackage[makeroom]{cancel} %% for strike outs
%\usepackage{mathtools} %% for pretty underbrace % Breaks Ximera
%\usepackage{multicol}
\usepackage{pgffor} %% required for integral for loops



%% http://tex.stackexchange.com/questions/66490/drawing-a-tikz-arc-specifying-the-center
%% Draws beach ball
\tikzset{pics/carc/.style args={#1:#2:#3}{code={\draw[pic actions] (#1:#3) arc(#1:#2:#3);}}}



\usepackage{array}
\setlength{\extrarowheight}{+.1cm}
\newdimen\digitwidth
\settowidth\digitwidth{9}
\def\divrule#1#2{
\noalign{\moveright#1\digitwidth
\vbox{\hrule width#2\digitwidth}}}





\newcommand{\RR}{\mathbb R}
\newcommand{\R}{\mathbb R}
\newcommand{\N}{\mathbb N}
\newcommand{\Z}{\mathbb Z}

\newcommand{\sagemath}{\textsf{SageMath}}


%\renewcommand{\d}{\,d\!}
\renewcommand{\d}{\mathop{}\!d}
\newcommand{\dd}[2][]{\frac{\d #1}{\d #2}}
\newcommand{\pp}[2][]{\frac{\partial #1}{\partial #2}}
\renewcommand{\l}{\ell}
\newcommand{\ddx}{\frac{d}{\d x}}

\newcommand{\zeroOverZero}{\ensuremath{\boldsymbol{\tfrac{0}{0}}}}
\newcommand{\inftyOverInfty}{\ensuremath{\boldsymbol{\tfrac{\infty}{\infty}}}}
\newcommand{\zeroOverInfty}{\ensuremath{\boldsymbol{\tfrac{0}{\infty}}}}
\newcommand{\zeroTimesInfty}{\ensuremath{\small\boldsymbol{0\cdot \infty}}}
\newcommand{\inftyMinusInfty}{\ensuremath{\small\boldsymbol{\infty - \infty}}}
\newcommand{\oneToInfty}{\ensuremath{\boldsymbol{1^\infty}}}
\newcommand{\zeroToZero}{\ensuremath{\boldsymbol{0^0}}}
\newcommand{\inftyToZero}{\ensuremath{\boldsymbol{\infty^0}}}



\newcommand{\numOverZero}{\ensuremath{\boldsymbol{\tfrac{\#}{0}}}}
\newcommand{\dfn}{\textbf}
%\newcommand{\unit}{\,\mathrm}
\newcommand{\unit}{\mathop{}\!\mathrm}
\newcommand{\eval}[1]{\bigg[ #1 \bigg]}
\newcommand{\seq}[1]{\left( #1 \right)}
\renewcommand{\epsilon}{\varepsilon}
\renewcommand{\phi}{\varphi}


\renewcommand{\iff}{\Leftrightarrow}

\DeclareMathOperator{\arccot}{arccot}
\DeclareMathOperator{\arcsec}{arcsec}
\DeclareMathOperator{\arccsc}{arccsc}
\DeclareMathOperator{\si}{Si}
\DeclareMathOperator{\scal}{scal}
\DeclareMathOperator{\sign}{sign}


%% \newcommand{\tightoverset}[2]{% for arrow vec
%%   \mathop{#2}\limits^{\vbox to -.5ex{\kern-0.75ex\hbox{$#1$}\vss}}}
\newcommand{\arrowvec}[1]{{\overset{\rightharpoonup}{#1}}}
%\renewcommand{\vec}[1]{\arrowvec{\mathbf{#1}}}
\renewcommand{\vec}[1]{{\overset{\boldsymbol{\rightharpoonup}}{\mathbf{#1}}}\hspace{0in}}

\newcommand{\point}[1]{\left(#1\right)} %this allows \vector{ to be changed to \vector{ with a quick find and replace
\newcommand{\pt}[1]{\mathbf{#1}} %this allows \vec{ to be changed to \vec{ with a quick find and replace
\newcommand{\Lim}[2]{\lim_{\point{#1} \to \point{#2}}} %Bart, I changed this to point since I want to use it.  It runs through both of the exercise and exerciseE files in limits section, which is why it was in each document to start with.

\DeclareMathOperator{\proj}{\mathbf{proj}}
\newcommand{\veci}{{\boldsymbol{\hat{\imath}}}}
\newcommand{\vecj}{{\boldsymbol{\hat{\jmath}}}}
\newcommand{\veck}{{\boldsymbol{\hat{k}}}}
\newcommand{\vecl}{\vec{\boldsymbol{\l}}}
\newcommand{\uvec}[1]{\mathbf{\hat{#1}}}
\newcommand{\utan}{\mathbf{\hat{t}}}
\newcommand{\unormal}{\mathbf{\hat{n}}}
\newcommand{\ubinormal}{\mathbf{\hat{b}}}

\newcommand{\dotp}{\bullet}
\newcommand{\cross}{\boldsymbol\times}
\newcommand{\grad}{\boldsymbol\nabla}
\newcommand{\divergence}{\grad\dotp}
\newcommand{\curl}{\grad\cross}
%\DeclareMathOperator{\divergence}{divergence}
%\DeclareMathOperator{\curl}[1]{\grad\cross #1}
\newcommand{\lto}{\mathop{\longrightarrow\,}\limits}

\renewcommand{\bar}{\overline}

\colorlet{textColor}{black}
\colorlet{background}{white}
\colorlet{penColor}{blue!50!black} % Color of a curve in a plot
\colorlet{penColor2}{red!50!black}% Color of a curve in a plot
\colorlet{penColor3}{red!50!blue} % Color of a curve in a plot
\colorlet{penColor4}{green!50!black} % Color of a curve in a plot
\colorlet{penColor5}{orange!80!black} % Color of a curve in a plot
\colorlet{penColor6}{yellow!70!black} % Color of a curve in a plot
\colorlet{fill1}{penColor!20} % Color of fill in a plot
\colorlet{fill2}{penColor2!20} % Color of fill in a plot
\colorlet{fillp}{fill1} % Color of positive area
\colorlet{filln}{penColor2!20} % Color of negative area
\colorlet{fill3}{penColor3!20} % Fill
\colorlet{fill4}{penColor4!20} % Fill
\colorlet{fill5}{penColor5!20} % Fill
\colorlet{gridColor}{gray!50} % Color of grid in a plot

\newcommand{\surfaceColor}{violet}
\newcommand{\surfaceColorTwo}{redyellow}
\newcommand{\sliceColor}{greenyellow}




\pgfmathdeclarefunction{gauss}{2}{% gives gaussian
  \pgfmathparse{1/(#2*sqrt(2*pi))*exp(-((x-#1)^2)/(2*#2^2))}%
}


%%%%%%%%%%%%%
%% Vectors
%%%%%%%%%%%%%

%% Simple horiz vectors
\renewcommand{\vector}[1]{\left\langle #1\right\rangle}


%% %% Complex Horiz Vectors with angle brackets
%% \makeatletter
%% \renewcommand{\vector}[2][ , ]{\left\langle%
%%   \def\nextitem{\def\nextitem{#1}}%
%%   \@for \el:=#2\do{\nextitem\el}\right\rangle%
%% }
%% \makeatother

%% %% Vertical Vectors
%% \def\vector#1{\begin{bmatrix}\vecListA#1,,\end{bmatrix}}
%% \def\vecListA#1,{\if,#1,\else #1\cr \expandafter \vecListA \fi}

%%%%%%%%%%%%%
%% End of vectors
%%%%%%%%%%%%%

%\newcommand{\fullwidth}{}
%\newcommand{\normalwidth}{}



%% makes a snazzy t-chart for evaluating functions
%\newenvironment{tchart}{\rowcolors{2}{}{background!90!textColor}\array}{\endarray}

%%This is to help with formatting on future title pages.
\newenvironment{sectionOutcomes}{}{}



%% Flowchart stuff
%\tikzstyle{startstop} = [rectangle, rounded corners, minimum width=3cm, minimum height=1cm,text centered, draw=black]
%\tikzstyle{question} = [rectangle, minimum width=3cm, minimum height=1cm, text centered, draw=black]
%\tikzstyle{decision} = [trapezium, trapezium left angle=70, trapezium right angle=110, minimum width=3cm, minimum height=1cm, text centered, draw=black]
%\tikzstyle{question} = [rectangle, rounded corners, minimum width=3cm, minimum height=1cm,text centered, draw=black]
%\tikzstyle{process} = [rectangle, minimum width=3cm, minimum height=1cm, text centered, draw=black]
%\tikzstyle{decision} = [trapezium, trapezium left angle=70, trapezium right angle=110, minimum width=3cm, minimum height=1cm, text centered, draw=black]



\title[Dig-In:]{Explanation of the chain rule}

\begin{document}
\begin{abstract}
  We give a rigorous explanation of the chain rule.
\end{abstract}
\maketitle



\begin{explanation}
Here we give a somewhat unrigoroius explanation, but it will serve our
purposes.

\begin{image}
\begin{tikzpicture}
	\begin{axis}[
            axis lines=none,
            clip=false,
          ]          
          \addplot [->,textColor] plot coordinates {(0,0) (-2,-4)}; %% x axis
          \addplot [->,textColor] plot coordinates {(0,0) (0,6)}; %% y axis
          \addplot [->,textColor] plot coordinates {(0,0) (6,0)}; %% g(x) axis

          \addplot [dashed, textColor] plot coordinates {(-.7,-1.4) (1.4,-1.4)};
          \addplot [dashed, textColor] plot coordinates {(1.4,-1.4) (2.1,0)};
          \addplot [dashed, textColor] plot coordinates {(2.1,0) (2.1,4.1)};
          
          \addplot [dashed, textColor] plot coordinates {(2.6,-2.6) (3.5,0)};
          \addplot [dashed, textColor] plot coordinates {(3.5,0) (3.5,4.1)};

          \addplot [dashed, very thick, textColor] plot coordinates {(1.4,-1.4) (.8,-2.6)};
          \addplot [dashed, very thick, textColor] plot coordinates {(2.1,4.1) (3.5,4.1)};

          \addplot [very thick, penColor5] plot coordinates {(.8,-2.6) (2.6,-2.6)};
          \addplot [very thick, penColor4] plot coordinates {(3.5,4.1) (3.5,5.5)};

          \addplot [very thick,penColor,domain=(0:4)] {2+x};
          \addplot [very thick,penColor2,domain=(0:4)] {-x};

          \node at (axis cs:3.5,4.8) [anchor=west,penColor4] {$f'(g(a)){\color{penColor5}g'(a)h}$};
          \node at (axis cs:1.7,-2.6) [anchor=north,penColor5] {$g'(a)h$};
          
          \addplot[color=penColor2,fill=penColor2,only marks,mark=*] coordinates{(1.4,-1.4)};  %% closed hole          
          \addplot[color=penColor,fill=penColor,only marks,mark=*] coordinates{(2.1,4.1)};  %% closed hole          

          \node at (axis cs:1,-2.1) [anchor=south,yslant=0,xslant=0,rotate=53] {$\overbrace{\hspace{.36in}}^{h}$};
          \node at (axis cs:7,0) [anchor=east] {$g(x)$};
          \node at (axis cs:0,6.7) [anchor=north] {$y$};
          \node at (axis cs:-2.15,-4) [anchor=north] {$x$};
          \node at (axis cs:-.7,-1.4) [anchor=east] {$a$};
        \end{axis}
\end{tikzpicture}
%% \caption{A geometric interpretation of the chain rule. Increasing $a$
%%   by a ``small amount'' $h$, increases $f(g(a))$ by $f'(g(a))g'(a)h$. Hence, 
%% \[
%% \frac{\Delta y}{\Delta x}\approx \frac{f'(g(a))g'(a)h}{h} =
%% f'(g(a))g'(a).
%% \]} 
\end{image}

\end{explanation}



Previously we gave a picture to explain the chain rule. Now we give
the sheer, raw, unadulterated explanation.


\begin{theorem}[Chain Rule]\index{chain rule}\index{derivative rules!chain}
If $f(x)$ and $g(x)$ are differentiable, then
\[
\ddx f(g(x)) = f'(g(x))g'(x).
\]
\begin{explanation}
Let $g_0$ be some $x$-value and consider the following:
\[
f'(g_0) = \lim_{h\to 0}\frac{f(g_0+h)-f(g_0)}{h}.
\]

Set $h = g-g_0$ and we have
\[
f'(g_0) = \lim_{g\to g_0} \frac{f(g)-f(g_0)}{g-g_0}.
\]
At this point, we might like to set $g=g(x+h)$ and $g_0=g(x)$;
however, we cannot as we cannot be sure that
\[
g(x+h) - g(x) \ne 0\qquad\text{when $h\ne 0$.}
\]
To overcome this difficulty, let $E(g)$ be the ``error term'' that
gives the difference between the slope of the secant line from
$f(g_0)$ to $f(g)$ and $f'(g_0)$,
\[
E(g) = \frac{f(g)-f(g_0)}{g-g_0} - f'(g_0).
\]
In particular, $E(g)(g-g_0)$ is the difference between $f(g)$ and the
tangent line of $f(x)$ at $x=g$, see the figure below:

\begin{image}
\begin{tikzpicture}
	\begin{axis}[
            clip=false,
            domain=0:2, range=0:6,ymax=4,ymin=0,
            axis lines =left, xlabel=$x$, ylabel=$y$,
            every axis y label/.style={at=(current axis.above origin),anchor=south},
            every axis x label/.style={at=(current axis.right of origin),anchor=west},
            xtick={1,1.666}, ytick={1,3},
            xticklabels={$g_0$,$g$}, yticklabels={$f(g_0)$,$f(g)$},
            axis on top,
          ]         
	  \addplot [textColor,dashed] plot coordinates {(1,0) (1,1)};
          \addplot [textColor,dashed] plot coordinates {(0,1) (1,1)};
          \addplot [textColor,dashed] plot coordinates {(0,3) (1.666,3)};
          \addplot [textColor,dashed] plot coordinates {(1.666,0) (1.666,1)};

          \addplot [textColor,dashed,very thick] plot coordinates {(1,1) (1.666,1)};
          \node at (axis cs:1.333,1) [anchor=north] {$\underbrace{\hspace{.75in}}_{g-g_0}$};

          \addplot [penColor4,very thick] plot coordinates {(1.666,1) (1.666,1.666)};
          \addplot [penColor5,very thick] plot coordinates {(1.666,1.666) (1.666,3)};

          \node at (axis cs:1.666,1.333) [anchor=west] {$f'(g_0)(g-g_0)$};
          \node at (axis cs:1.666,2.333) [anchor=west] {$E(g)(g-g_0)$};

          \addplot [very thick,penColor, smooth,domain=(0:7/4)] {-1/(x-2)};
          \addplot [very thick,penColor2, smooth,domain=(0:2)] {x};

          \addplot[color=penColor,fill=penColor,only marks,mark=*] coordinates{(1.666,3)};  %% closed hole          
          \addplot[color=penColor,fill=penColor,only marks,mark=*] coordinates{(1,1)};  %% closed hole          
        \end{axis}
\end{tikzpicture}
\end{image}

Hence we see that
\[
f(g)-f(g_0) = \left(f'(g_0) + E(g)\right)(g-g_0),
\]
and so
\[
\frac{f(g)-f(g_0)}{g-g_0} = f'(g_0) + E(g).
\]
Combining this with the fact that 
\[
f'(g_0) = \lim_{g\to g_0} \frac{f(g)-f(g_0)}{g-g_0}
\]
we have that
\begin{align*}
f'(g_0) &= \lim_{g\to g_0}\frac{f(g)-f(g_0)}{g-g_0} \\
&= \lim_{g\to g_0}f'(g_0) + E(g)\\
&= f'(g_0) + \lim_{g\to g_0} E(g),
\end{align*}
and hence it follows that $\lim_{g\to g_0} E(g) = 0$. At this point,
we may return to the ``well-worn path.'' Starting with
\[
f(g)-f(g_0) = \left(f'(g_0) + E(g)\right)(g-g_0),
\]
divide both sides by $h$ and set
$g=g(x+h)$ and $g_0=g(x)$
\[
\frac{f(g(x+h))-f(g(x))}{h} = \left(f'(g(x)) +
E(g(x+h))\right)\frac{g(x+h)-g(x)}{h}.
\]
Taking the limit as $h$ approaches $0$, we see 
\begin{align*}
\lim_{h\to 0}\frac{f(g(x+h))-f(g(x))}{h} &= \lim_{h\to 0}\left(f'(g(x))
+ E(g(x+h))\right)\frac{g(x+h)-g(x)}{h}\\
&= \lim_{h\to 0}\left(f'(g(x))
+ E(g(x+h))\right)\lim_{h\to 0}\frac{g(x+h)-g(x)}{h}\\
&= f'(g(x))g'(x).
\end{align*}
Hence, $\ddx f(g(x))= f'(g(x))g'(x)$.
\end{explanation}
\end{theorem}


\end{document}
