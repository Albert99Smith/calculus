\documentclass{ximera}

%\usepackage{todonotes}
%\usepackage{mathtools} %% Required for wide table Curl and Greens
%\usepackage{cuted} %% Required for wide table Curl and Greens
\newcommand{\todo}{}

\usepackage{esint} % for \oiint
\ifxake%%https://math.meta.stackexchange.com/questions/9973/how-do-you-render-a-closed-surface-double-integral
\renewcommand{\oiint}{{\large\bigcirc}\kern-1.56em\iint}
\fi


\graphicspath{
  {./}
  {ximeraTutorial/}
  {basicPhilosophy/}
  {functionsOfSeveralVariables/}
  {normalVectors/}
  {lagrangeMultipliers/}
  {vectorFields/}
  {greensTheorem/}
  {shapeOfThingsToCome/}
  {dotProducts/}
  {partialDerivativesAndTheGradientVector/}
  {../productAndQuotientRules/exercises/}
  {../normalVectors/exercisesParametricPlots/}
  {../continuityOfFunctionsOfSeveralVariables/exercises/}
  {../partialDerivativesAndTheGradientVector/exercises/}
  {../directionalDerivativeAndChainRule/exercises/}
  {../commonCoordinates/exercisesCylindricalCoordinates/}
  {../commonCoordinates/exercisesSphericalCoordinates/}
  {../greensTheorem/exercisesCurlAndLineIntegrals/}
  {../greensTheorem/exercisesDivergenceAndLineIntegrals/}
  {../shapeOfThingsToCome/exercisesDivergenceTheorem/}
  {../greensTheorem/}
  {../shapeOfThingsToCome/}
  {../separableDifferentialEquations/exercises/}
  {vectorFields/}
}

\newcommand{\mooculus}{\textsf{\textbf{MOOC}\textnormal{\textsf{ULUS}}}}

\usepackage{tkz-euclide}\usepackage{tikz}
\usepackage{tikz-cd}
\usetikzlibrary{arrows}
\tikzset{>=stealth,commutative diagrams/.cd,
  arrow style=tikz,diagrams={>=stealth}} %% cool arrow head
\tikzset{shorten <>/.style={ shorten >=#1, shorten <=#1 } } %% allows shorter vectors

\usetikzlibrary{backgrounds} %% for boxes around graphs
\usetikzlibrary{shapes,positioning}  %% Clouds and stars
\usetikzlibrary{matrix} %% for matrix
\usepgfplotslibrary{polar} %% for polar plots
\usepgfplotslibrary{fillbetween} %% to shade area between curves in TikZ
\usetkzobj{all}
\usepackage[makeroom]{cancel} %% for strike outs
%\usepackage{mathtools} %% for pretty underbrace % Breaks Ximera
%\usepackage{multicol}
\usepackage{pgffor} %% required for integral for loops



%% http://tex.stackexchange.com/questions/66490/drawing-a-tikz-arc-specifying-the-center
%% Draws beach ball
\tikzset{pics/carc/.style args={#1:#2:#3}{code={\draw[pic actions] (#1:#3) arc(#1:#2:#3);}}}



\usepackage{array}
\setlength{\extrarowheight}{+.1cm}
\newdimen\digitwidth
\settowidth\digitwidth{9}
\def\divrule#1#2{
\noalign{\moveright#1\digitwidth
\vbox{\hrule width#2\digitwidth}}}





\newcommand{\RR}{\mathbb R}
\newcommand{\R}{\mathbb R}
\newcommand{\N}{\mathbb N}
\newcommand{\Z}{\mathbb Z}

\newcommand{\sagemath}{\textsf{SageMath}}


%\renewcommand{\d}{\,d\!}
\renewcommand{\d}{\mathop{}\!d}
\newcommand{\dd}[2][]{\frac{\d #1}{\d #2}}
\newcommand{\pp}[2][]{\frac{\partial #1}{\partial #2}}
\renewcommand{\l}{\ell}
\newcommand{\ddx}{\frac{d}{\d x}}

\newcommand{\zeroOverZero}{\ensuremath{\boldsymbol{\tfrac{0}{0}}}}
\newcommand{\inftyOverInfty}{\ensuremath{\boldsymbol{\tfrac{\infty}{\infty}}}}
\newcommand{\zeroOverInfty}{\ensuremath{\boldsymbol{\tfrac{0}{\infty}}}}
\newcommand{\zeroTimesInfty}{\ensuremath{\small\boldsymbol{0\cdot \infty}}}
\newcommand{\inftyMinusInfty}{\ensuremath{\small\boldsymbol{\infty - \infty}}}
\newcommand{\oneToInfty}{\ensuremath{\boldsymbol{1^\infty}}}
\newcommand{\zeroToZero}{\ensuremath{\boldsymbol{0^0}}}
\newcommand{\inftyToZero}{\ensuremath{\boldsymbol{\infty^0}}}



\newcommand{\numOverZero}{\ensuremath{\boldsymbol{\tfrac{\#}{0}}}}
\newcommand{\dfn}{\textbf}
%\newcommand{\unit}{\,\mathrm}
\newcommand{\unit}{\mathop{}\!\mathrm}
\newcommand{\eval}[1]{\bigg[ #1 \bigg]}
\newcommand{\seq}[1]{\left( #1 \right)}
\renewcommand{\epsilon}{\varepsilon}
\renewcommand{\phi}{\varphi}


\renewcommand{\iff}{\Leftrightarrow}

\DeclareMathOperator{\arccot}{arccot}
\DeclareMathOperator{\arcsec}{arcsec}
\DeclareMathOperator{\arccsc}{arccsc}
\DeclareMathOperator{\si}{Si}
\DeclareMathOperator{\scal}{scal}
\DeclareMathOperator{\sign}{sign}


%% \newcommand{\tightoverset}[2]{% for arrow vec
%%   \mathop{#2}\limits^{\vbox to -.5ex{\kern-0.75ex\hbox{$#1$}\vss}}}
\newcommand{\arrowvec}[1]{{\overset{\rightharpoonup}{#1}}}
%\renewcommand{\vec}[1]{\arrowvec{\mathbf{#1}}}
\renewcommand{\vec}[1]{{\overset{\boldsymbol{\rightharpoonup}}{\mathbf{#1}}}\hspace{0in}}

\newcommand{\point}[1]{\left(#1\right)} %this allows \vector{ to be changed to \vector{ with a quick find and replace
\newcommand{\pt}[1]{\mathbf{#1}} %this allows \vec{ to be changed to \vec{ with a quick find and replace
\newcommand{\Lim}[2]{\lim_{\point{#1} \to \point{#2}}} %Bart, I changed this to point since I want to use it.  It runs through both of the exercise and exerciseE files in limits section, which is why it was in each document to start with.

\DeclareMathOperator{\proj}{\mathbf{proj}}
\newcommand{\veci}{{\boldsymbol{\hat{\imath}}}}
\newcommand{\vecj}{{\boldsymbol{\hat{\jmath}}}}
\newcommand{\veck}{{\boldsymbol{\hat{k}}}}
\newcommand{\vecl}{\vec{\boldsymbol{\l}}}
\newcommand{\uvec}[1]{\mathbf{\hat{#1}}}
\newcommand{\utan}{\mathbf{\hat{t}}}
\newcommand{\unormal}{\mathbf{\hat{n}}}
\newcommand{\ubinormal}{\mathbf{\hat{b}}}

\newcommand{\dotp}{\bullet}
\newcommand{\cross}{\boldsymbol\times}
\newcommand{\grad}{\boldsymbol\nabla}
\newcommand{\divergence}{\grad\dotp}
\newcommand{\curl}{\grad\cross}
%\DeclareMathOperator{\divergence}{divergence}
%\DeclareMathOperator{\curl}[1]{\grad\cross #1}
\newcommand{\lto}{\mathop{\longrightarrow\,}\limits}

\renewcommand{\bar}{\overline}

\colorlet{textColor}{black}
\colorlet{background}{white}
\colorlet{penColor}{blue!50!black} % Color of a curve in a plot
\colorlet{penColor2}{red!50!black}% Color of a curve in a plot
\colorlet{penColor3}{red!50!blue} % Color of a curve in a plot
\colorlet{penColor4}{green!50!black} % Color of a curve in a plot
\colorlet{penColor5}{orange!80!black} % Color of a curve in a plot
\colorlet{penColor6}{yellow!70!black} % Color of a curve in a plot
\colorlet{fill1}{penColor!20} % Color of fill in a plot
\colorlet{fill2}{penColor2!20} % Color of fill in a plot
\colorlet{fillp}{fill1} % Color of positive area
\colorlet{filln}{penColor2!20} % Color of negative area
\colorlet{fill3}{penColor3!20} % Fill
\colorlet{fill4}{penColor4!20} % Fill
\colorlet{fill5}{penColor5!20} % Fill
\colorlet{gridColor}{gray!50} % Color of grid in a plot

\newcommand{\surfaceColor}{violet}
\newcommand{\surfaceColorTwo}{redyellow}
\newcommand{\sliceColor}{greenyellow}




\pgfmathdeclarefunction{gauss}{2}{% gives gaussian
  \pgfmathparse{1/(#2*sqrt(2*pi))*exp(-((x-#1)^2)/(2*#2^2))}%
}


%%%%%%%%%%%%%
%% Vectors
%%%%%%%%%%%%%

%% Simple horiz vectors
\renewcommand{\vector}[1]{\left\langle #1\right\rangle}


%% %% Complex Horiz Vectors with angle brackets
%% \makeatletter
%% \renewcommand{\vector}[2][ , ]{\left\langle%
%%   \def\nextitem{\def\nextitem{#1}}%
%%   \@for \el:=#2\do{\nextitem\el}\right\rangle%
%% }
%% \makeatother

%% %% Vertical Vectors
%% \def\vector#1{\begin{bmatrix}\vecListA#1,,\end{bmatrix}}
%% \def\vecListA#1,{\if,#1,\else #1\cr \expandafter \vecListA \fi}

%%%%%%%%%%%%%
%% End of vectors
%%%%%%%%%%%%%

%\newcommand{\fullwidth}{}
%\newcommand{\normalwidth}{}



%% makes a snazzy t-chart for evaluating functions
%\newenvironment{tchart}{\rowcolors{2}{}{background!90!textColor}\array}{\endarray}

%%This is to help with formatting on future title pages.
\newenvironment{sectionOutcomes}{}{}



%% Flowchart stuff
%\tikzstyle{startstop} = [rectangle, rounded corners, minimum width=3cm, minimum height=1cm,text centered, draw=black]
%\tikzstyle{question} = [rectangle, minimum width=3cm, minimum height=1cm, text centered, draw=black]
%\tikzstyle{decision} = [trapezium, trapezium left angle=70, trapezium right angle=110, minimum width=3cm, minimum height=1cm, text centered, draw=black]
%\tikzstyle{question} = [rectangle, rounded corners, minimum width=3cm, minimum height=1cm,text centered, draw=black]
%\tikzstyle{process} = [rectangle, minimum width=3cm, minimum height=1cm, text centered, draw=black]
%\tikzstyle{decision} = [trapezium, trapezium left angle=70, trapezium right angle=110, minimum width=3cm, minimum height=1cm, text centered, draw=black]


\outcome{Know the graphs and properties of ``famous'' functions.}
\outcome{Understand the properties of trigonometric functions.}
\outcome{Evaluate expressions and solve equations involving
          trigonometric functions and inverse trigonometric functions.}

\title[Dig-In:]{Trigonometric functions}

\begin{document}
\begin{abstract}
  We review trigonometric functions.
\end{abstract}
\maketitle



\section{What are trigonometric functions?}

\begin{definition}
  A \dfn{trigonometric function} is a function that relates an angle
  of a right triangle to a ratio of its sides.
\end{definition}


The basic trigonometric functions are sine and cosine. A convienient
way to think about sine and cosine is via the unit circle.
\begin{image}
\begin{tikzpicture}
	\begin{axis}[
            xmin=-1.1,xmax=1.1,ymin=-1.1,ymax=1.1,
            axis lines=center,
            width=5in,
            ticks=none,
            clip=false,
            unit vector ratio*=1 1 1,
            xlabel=$x$, ylabel=$y$,
            every axis y label/.style={at=(current axis.above origin),anchor=south},
            every axis x label/.style={at=(current axis.right of origin),anchor=west},
          ]        
          \addplot [dashed, smooth, domain=(0:360)] ({cos(x)},{sin(x)}); %% unit circle

          \addplot [textColor] plot coordinates {(0,0) (.766,.643)}; %% 40 degrees

          \addplot [ultra thick,penColor] plot coordinates {(.766,0) (.766,.643)}; %% 40 degrees
          \addplot [ultra thick,penColor2] plot coordinates {(0,0) (.766,0)}; %% 40 degrees
          
          %\addplot [ultra thick,penColor3] plot coordinates {(1,0) (1,.839)}; %% 40 degrees          

          \addplot [textColor,smooth, domain=(0:40)] ({.15*cos(x)},{.15*sin(x)});
          %\addplot [very thick,penColor] plot coordinates {(0,0) (.766,.643)}; %% sector
          %\addplot [very thick,penColor] plot coordinates {(0,0) (1,0)}; %% sector
          %\addplot [very thick, penColor, smooth, domain=(0:40)] ({cos(x)},{sin(x)}); %% sector
          \node at (axis cs:.15,.07) [anchor=west] {$\theta$};
          \node[penColor, rotate=-90] at (axis cs:.84,.322) {$\sin(\theta)$};
          \node[penColor2] at (axis cs:.383,0) [anchor=north] {$\cos(\theta)$};
          %\node[penColor3, rotate=-90] at (axis cs:1.06,.322) {$\tan(\theta)$};
        \end{axis}
\end{tikzpicture}
\end{image}


\begin{definition}
  The trigonometric functions are:
  \[
  \begin{aligned}
  \cos(\theta) &= \frac{\text{adj}}{\text{hyp}}\\
  \sec(\theta) &= \frac{1}{\cos(\theta)}\\
  \tan(\theta) &= \frac{\sin(\theta)}{\cos(\theta)}\qquad
  \end{aligned}
  \qquad
  \begin{aligned}
  \sin(\theta) &= \frac{\text{opp}}{\text{hyp}}\\
  \csc(\theta) &= \frac{1}{\sin(\theta)}\\
  \cot(\theta) &= \frac{\cos(\theta)}{\sin(\theta)}    
  \end{aligned}
  \]
  
\end{definition}
ADD DOMAINS, convert to whatever.



id plots of sine/cos


  
\section{Connections to inverse functions}


The trigonometric functions frequently arise in problems, and often we
are interested in finding specific angles, say $\theta$ such that
\[
\sin(\theta) = .7
\]
Hence we want to be able to invert functions like $\sin(\theta)$ and
$\cos(\theta)$. Since these functions are not one-to-one, meaning there are
are infinitely many angles with $\sin(\theta) = .7$, it is impossible
to find a true inverse function for $\sin(\theta)$. Nevertheless, it
is useful to have something like an inverse to the sine, however
imperfect. The usual approach is to pick out some collection of angles
that produce all possible values of the sine exactly once. If we
``discard'' all other angles, the resulting function has a proper
inverse.
\begin{image}
\begin{tikzpicture}
	\begin{axis}[
            xmin=-6.75,xmax=6.75,ymin=-1.5,ymax=1.5,
            axis lines=center,
            xtick={-6.28, -4.71, -3.14, -1.57, 0, 1.57, 3.142, 4.71, 6.28},
            xticklabels={$-2\pi$,$-3\pi/2$,$-\pi$, $-\pi/2$, $0$, $\pi/2$, $\pi$, $3\pi/2$, $2\pi$},
            ytick={-1,1},
            %ticks=none,
            width=9in,
            height=2in,
            unit vector ratio*=1 1 1,
            xlabel=$\theta$, ylabel=$y$,
            every axis y label/.style={at=(current axis.above origin),anchor=south},
            every axis x label/.style={at=(current axis.right of origin),anchor=west},
          ]        
          \addplot [very thick, penColor!20!background, samples=100,smooth, domain=(-6.75:-1.57)] {sin(deg(x))};
          \addplot [very thick, penColor!20!background, samples=100,smooth, domain=(1.57:6.75)] {sin(deg(x))};
          \addplot [very thick, penColor, samples=100,smooth, domain=(-1.57:1.57)] {sin(deg(x))};
          
          \addplot[color=penColor,fill=penColor,only marks,mark=*] coordinates{(-1.57,-1)};  %% closed hole          
          \addplot[color=penColor,fill=penColor,only marks,mark=*] coordinates{(1.57,1)};  %% closed hole          
          \node at (axis cs:3.14,.75) [penColor] {$\sin(\theta)$};
        \end{axis}
\end{tikzpicture}
%% \caption{The function $\sin(\theta)$ takes on all values between $-1$
%%   and $1$ exactly once on the interval $[-\pi/2,\pi/2]$. If we
%%   restrict $\sin(\theta)$ to this interval, then this restricted
%%   function has an inverse.}
%% \label{figure:sin-restricted}
%% \end{figure*}
\end{image}

In a similar fashion, we need to restrict cosine to be able to take an inverse.

\begin{image}
\begin{tikzpicture}
	\begin{axis}[
            xmin=-6.75,xmax=6.75,ymin=-1.5,ymax=1.5,
            axis lines=center,
            xtick={-6.28, -4.71, -3.14, -1.57, 0, 1.57, 3.142, 4.71, 6.28},
            xticklabels={$-2\pi$,$-3\pi/2$,$-\pi$, $-\pi/2$, $0$, $\pi/2$, $\pi$, $3\pi/2$, $2\pi$},
            ytick={-1,1},
            %ticks=none,
            width=9in,
            height=2in,
            unit vector ratio*=1 1 1,
            xlabel=$\theta$, ylabel=$y$,
            every axis y label/.style={at=(current axis.above origin),anchor=south},
            every axis x label/.style={at=(current axis.right of origin),anchor=west},
          ]        
          \addplot [very thick, penColor2!20!background, samples=100,smooth, domain=(-6.75:0)] {cos(deg(x))};
          \addplot [very thick, penColor2!20!background, samples=100,smooth, domain=(3.14:6.75)] {cos(deg(x))};
          \addplot [very thick, penColor2, samples=100,smooth, domain=(0:3.14)] {cos(deg(x))};
          
          \addplot[color=penColor2,fill=penColor2,only marks,mark=*] coordinates{(0,1)};  %% closed hole          
          \addplot[color=penColor2,fill=penColor2,only marks,mark=*] coordinates{(pi,-1)};  %% closed hole          
          \node at (axis cs:-1.57,.75) [penColor2] {$\cos(\theta)$};
        \end{axis}
\end{tikzpicture}
%% \caption{The function $\cos(\theta)$ takes on all values between $-1$
%%   and $1$ exactly once on the interval $[0,\pi]$. If we restrict
%%   $\cos(\theta)$ to this interval, then this restricted function has
%%   an inverse.}
%% \label{figure:cos-restricted}
%% \end{figure*}
\end{image}

By examining both sine and cosine on restricted domains, we can now produce functions arcsine and arccosine:


\begin{tabular}{lll}\index{arcsine}\index{arccosine}
\begin{tikzpicture}
	\begin{axis}[
            xmin=-1.5,xmax=1.5,ymin=-1.75,ymax=1.75,
            axis lines=center,
            ytick={-1.57, 0, 1.57},
            yticklabels={$-\pi/2$, $0$, $\pi/2$},
            xtick={-1,1},
            unit vector ratio*=1 1 1,
            xlabel=$y$, ylabel=$\theta$,
            every axis y label/.style={at=(current axis.above origin),anchor=south},
            every axis x label/.style={at=(current axis.right of origin),anchor=west},
          ]        
          \addplot [very thick, penColor4, samples=100,smooth, domain=(-1:1)] {asin(x)*pi/180};
                    
          \addplot[color=penColor4,fill=penColor4,only marks,mark=*] coordinates{(-1,-pi/2)};  %% closed hole          
          \addplot[color=penColor4,fill=penColor4,only marks,mark=*] coordinates{(1,pi/2)};  %% closed hole          
        \end{axis}
\end{tikzpicture}

&

\hspace{1in}

&

\begin{tikzpicture}
	\begin{axis}[
            xmin=-1.5,xmax=1.5,ymin=-.25,ymax=3.25,
            axis lines=center,
            ytick={0, 1.57,3.14},
            yticklabels={$0$, $\pi/2$,$\pi$},
            xtick={-1,1},
            unit vector ratio*=1 1 1,
            xlabel=$y$, ylabel=$\theta$,
            every axis y label/.style={at=(current axis.above origin),anchor=south},
            every axis x label/.style={at=(current axis.right of origin),anchor=west},
          ]        
          \addplot [very thick, penColor5, samples=100,smooth, domain=(-1:1)] {acos(x)*pi/180};
                    
          \addplot[color=penColor5,fill=penColor5,only marks,mark=*] coordinates{(1,0)};  %% closed hole          
          \addplot[color=penColor5,fill=penColor5,only marks,mark=*] coordinates{(-1,pi)};  %% closed hole          
        \end{axis}
\end{tikzpicture} \\
\begin{minipage}{2.5in}
Here we see a plot of $\arcsin(y)$, the inverse function of
$\sin(\theta)$ when it is restricted to the interval $[-\pi/2,\pi/2]$.
\end{minipage}

& 

& 
\begin{minipage}{2.5in}
Here we see a plot of $\arccos(y)$, the inverse function of
$\cos(\theta)$ when it is restricted to the interval $[0,\pi]$.
\end{minipage}
\end{tabular}



Recall that a function and its inverse undo each other in either
order, for example, 

\[
\sqrt[3]{x^3}=x\qquad \text{and}\qquad \left(\sqrt[3]{x}\right)^3=x. 
\]
Since arcsine is the inverse of sine restricted to the interval
$[-\pi/2,\pi/2]$, this does not work with sine and arcsine, for
example
\[
\arcsin(\sin(\pi))=0.
\]
Moreover, there is a similar situation for cosine and arccosine as
\[
\arccos(\cos(2\pi))=0.
\]
Once you get a feel for how $\arcsin(y)$ and $\arccos(y)$ behave, let's examine tangent.

\begin{image}
\begin{tikzpicture}
	\begin{axis}[
            xmin=-6.75,xmax=6.75,ymin=-3,ymax=3,
            axis lines=center,
            width=9in,
            height=3.5in,
            xtick={-6.28, -4.71, -3.14, -1.57, 0, 1.57, 3.142, 4.71, 6.28},
            xticklabels={$-2\pi$,$-3\pi/2$,$-\pi$, $-\pi/2$, $0$, $\pi/2$, $\pi$, $3\pi/2$, $2\pi$},       
            unit vector ratio*=1 1 1,
            xlabel=$\theta$, ylabel=$y$,
            every axis y label/.style={at=(current axis.above origin),anchor=south},
            every axis x label/.style={at=(current axis.right of origin),anchor=west},
          ]        
          \addplot [very thick, penColor3, samples=100,smooth, domain=(-1.55:1.55)] {tan(deg(x))};
          \addplot [very thick, penColor3!30!background, samples=100,smooth, domain=(-4.69:-1.59)] {tan(deg(x))};
          \addplot [very thick, penColor3!30!background, samples=100,smooth, domain=(-6.75:-4.73)] {tan(deg(x))};
          \addplot [very thick, penColor3!30!background, samples=100,smooth, domain=(1.59:4.69)] {tan(deg(x))};
          \addplot [very thick, penColor3!30!background, samples=100,smooth, domain=(4.73:6.75)] {tan(deg(x))};
          
          \addplot [textColor,dashed] plot coordinates {(-4.71,-3) (-4.71,3)};
          \addplot [textColor,dashed] plot coordinates {(-1.57,-3) (-1.57,3)};
          \addplot [textColor,dashed] plot coordinates {(1.57,-3) (1.57,3)};
          \addplot [textColor,dashed] plot coordinates {(4.71,-3) (4.71,3)};
          
          \node at (axis cs:.4,1.25) [penColor3] {$\tan(\theta)$};    
        \end{axis}
\end{tikzpicture}
%% \caption{The function $\tan(\theta)$ takes on all values in $\R$
%%   exactly once on the open interval $(-\pi/2,\pi/2)$. If we restrict
%%   $\tan(\theta)$ to this interval, then this restricted function has
%%   an inverse.}
%% \end{figure*}
%% \end{fullwidth}
\end{image}
  
Again, only working on a restricted domain of tangent, we can produce an inverse function, arctangent. 

\begin{image}
\begin{tikzpicture}
	\begin{axis}[
            xmin=-6,xmax=6,ymin=-2,ymax=2,
            axis lines=center,
            ytick={0, -1.57,1.57},
            width=9in,
            height=2.5in,
            yticklabels={$0$, $-\pi/2$,$\pi/2$},
            xtick={0},
            unit vector ratio*=1 1 1,
            xlabel=$y$, ylabel=$\theta$,
            every axis y label/.style={at=(current axis.above origin),anchor=south},
            every axis x label/.style={at=(current axis.right of origin),anchor=west},
          ]        
          \addplot [very thick, penColor3!20!penColor2, samples=100,smooth, domain=(-6:6)] {atan(x)*pi/180};
          \addplot [textColor,dashed] plot coordinates {(-6,-1.57) (6,-1.57)};
          \addplot [textColor,dashed] plot coordinates {(-6,1.57) (6,1.57)};
        \end{axis}
\end{tikzpicture}
%% \caption{Here we see a plot of $\arctan(y)$, the inverse function of
%% $\tan(\theta)$ when it is restricted to the interval $(-\pi/2,\pi/2)$.}
%% \end{figure*}
%% \index{arctangent}
%% \end{fullwidth}
\end{image}

We leave it to you, the reader, to investigate the functions
arcsecant, arccosecant, and arccotangent.




\end{document}
