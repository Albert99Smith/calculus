\documentclass{ximera}

%\usepackage{todonotes}
%\usepackage{mathtools} %% Required for wide table Curl and Greens
%\usepackage{cuted} %% Required for wide table Curl and Greens
\newcommand{\todo}{}

\usepackage{esint} % for \oiint
\ifxake%%https://math.meta.stackexchange.com/questions/9973/how-do-you-render-a-closed-surface-double-integral
\renewcommand{\oiint}{{\large\bigcirc}\kern-1.56em\iint}
\fi


\graphicspath{
  {./}
  {ximeraTutorial/}
  {basicPhilosophy/}
  {functionsOfSeveralVariables/}
  {normalVectors/}
  {lagrangeMultipliers/}
  {vectorFields/}
  {greensTheorem/}
  {shapeOfThingsToCome/}
  {dotProducts/}
  {partialDerivativesAndTheGradientVector/}
  {../productAndQuotientRules/exercises/}
  {../normalVectors/exercisesParametricPlots/}
  {../continuityOfFunctionsOfSeveralVariables/exercises/}
  {../partialDerivativesAndTheGradientVector/exercises/}
  {../directionalDerivativeAndChainRule/exercises/}
  {../commonCoordinates/exercisesCylindricalCoordinates/}
  {../commonCoordinates/exercisesSphericalCoordinates/}
  {../greensTheorem/exercisesCurlAndLineIntegrals/}
  {../greensTheorem/exercisesDivergenceAndLineIntegrals/}
  {../shapeOfThingsToCome/exercisesDivergenceTheorem/}
  {../greensTheorem/}
  {../shapeOfThingsToCome/}
  {../separableDifferentialEquations/exercises/}
  {vectorFields/}
}

\newcommand{\mooculus}{\textsf{\textbf{MOOC}\textnormal{\textsf{ULUS}}}}

\usepackage{tkz-euclide}\usepackage{tikz}
\usepackage{tikz-cd}
\usetikzlibrary{arrows}
\tikzset{>=stealth,commutative diagrams/.cd,
  arrow style=tikz,diagrams={>=stealth}} %% cool arrow head
\tikzset{shorten <>/.style={ shorten >=#1, shorten <=#1 } } %% allows shorter vectors

\usetikzlibrary{backgrounds} %% for boxes around graphs
\usetikzlibrary{shapes,positioning}  %% Clouds and stars
\usetikzlibrary{matrix} %% for matrix
\usepgfplotslibrary{polar} %% for polar plots
\usepgfplotslibrary{fillbetween} %% to shade area between curves in TikZ
\usetkzobj{all}
\usepackage[makeroom]{cancel} %% for strike outs
%\usepackage{mathtools} %% for pretty underbrace % Breaks Ximera
%\usepackage{multicol}
\usepackage{pgffor} %% required for integral for loops



%% http://tex.stackexchange.com/questions/66490/drawing-a-tikz-arc-specifying-the-center
%% Draws beach ball
\tikzset{pics/carc/.style args={#1:#2:#3}{code={\draw[pic actions] (#1:#3) arc(#1:#2:#3);}}}



\usepackage{array}
\setlength{\extrarowheight}{+.1cm}
\newdimen\digitwidth
\settowidth\digitwidth{9}
\def\divrule#1#2{
\noalign{\moveright#1\digitwidth
\vbox{\hrule width#2\digitwidth}}}





\newcommand{\RR}{\mathbb R}
\newcommand{\R}{\mathbb R}
\newcommand{\N}{\mathbb N}
\newcommand{\Z}{\mathbb Z}

\newcommand{\sagemath}{\textsf{SageMath}}


%\renewcommand{\d}{\,d\!}
\renewcommand{\d}{\mathop{}\!d}
\newcommand{\dd}[2][]{\frac{\d #1}{\d #2}}
\newcommand{\pp}[2][]{\frac{\partial #1}{\partial #2}}
\renewcommand{\l}{\ell}
\newcommand{\ddx}{\frac{d}{\d x}}

\newcommand{\zeroOverZero}{\ensuremath{\boldsymbol{\tfrac{0}{0}}}}
\newcommand{\inftyOverInfty}{\ensuremath{\boldsymbol{\tfrac{\infty}{\infty}}}}
\newcommand{\zeroOverInfty}{\ensuremath{\boldsymbol{\tfrac{0}{\infty}}}}
\newcommand{\zeroTimesInfty}{\ensuremath{\small\boldsymbol{0\cdot \infty}}}
\newcommand{\inftyMinusInfty}{\ensuremath{\small\boldsymbol{\infty - \infty}}}
\newcommand{\oneToInfty}{\ensuremath{\boldsymbol{1^\infty}}}
\newcommand{\zeroToZero}{\ensuremath{\boldsymbol{0^0}}}
\newcommand{\inftyToZero}{\ensuremath{\boldsymbol{\infty^0}}}



\newcommand{\numOverZero}{\ensuremath{\boldsymbol{\tfrac{\#}{0}}}}
\newcommand{\dfn}{\textbf}
%\newcommand{\unit}{\,\mathrm}
\newcommand{\unit}{\mathop{}\!\mathrm}
\newcommand{\eval}[1]{\bigg[ #1 \bigg]}
\newcommand{\seq}[1]{\left( #1 \right)}
\renewcommand{\epsilon}{\varepsilon}
\renewcommand{\phi}{\varphi}


\renewcommand{\iff}{\Leftrightarrow}

\DeclareMathOperator{\arccot}{arccot}
\DeclareMathOperator{\arcsec}{arcsec}
\DeclareMathOperator{\arccsc}{arccsc}
\DeclareMathOperator{\si}{Si}
\DeclareMathOperator{\scal}{scal}
\DeclareMathOperator{\sign}{sign}


%% \newcommand{\tightoverset}[2]{% for arrow vec
%%   \mathop{#2}\limits^{\vbox to -.5ex{\kern-0.75ex\hbox{$#1$}\vss}}}
\newcommand{\arrowvec}[1]{{\overset{\rightharpoonup}{#1}}}
%\renewcommand{\vec}[1]{\arrowvec{\mathbf{#1}}}
\renewcommand{\vec}[1]{{\overset{\boldsymbol{\rightharpoonup}}{\mathbf{#1}}}\hspace{0in}}

\newcommand{\point}[1]{\left(#1\right)} %this allows \vector{ to be changed to \vector{ with a quick find and replace
\newcommand{\pt}[1]{\mathbf{#1}} %this allows \vec{ to be changed to \vec{ with a quick find and replace
\newcommand{\Lim}[2]{\lim_{\point{#1} \to \point{#2}}} %Bart, I changed this to point since I want to use it.  It runs through both of the exercise and exerciseE files in limits section, which is why it was in each document to start with.

\DeclareMathOperator{\proj}{\mathbf{proj}}
\newcommand{\veci}{{\boldsymbol{\hat{\imath}}}}
\newcommand{\vecj}{{\boldsymbol{\hat{\jmath}}}}
\newcommand{\veck}{{\boldsymbol{\hat{k}}}}
\newcommand{\vecl}{\vec{\boldsymbol{\l}}}
\newcommand{\uvec}[1]{\mathbf{\hat{#1}}}
\newcommand{\utan}{\mathbf{\hat{t}}}
\newcommand{\unormal}{\mathbf{\hat{n}}}
\newcommand{\ubinormal}{\mathbf{\hat{b}}}

\newcommand{\dotp}{\bullet}
\newcommand{\cross}{\boldsymbol\times}
\newcommand{\grad}{\boldsymbol\nabla}
\newcommand{\divergence}{\grad\dotp}
\newcommand{\curl}{\grad\cross}
%\DeclareMathOperator{\divergence}{divergence}
%\DeclareMathOperator{\curl}[1]{\grad\cross #1}
\newcommand{\lto}{\mathop{\longrightarrow\,}\limits}

\renewcommand{\bar}{\overline}

\colorlet{textColor}{black}
\colorlet{background}{white}
\colorlet{penColor}{blue!50!black} % Color of a curve in a plot
\colorlet{penColor2}{red!50!black}% Color of a curve in a plot
\colorlet{penColor3}{red!50!blue} % Color of a curve in a plot
\colorlet{penColor4}{green!50!black} % Color of a curve in a plot
\colorlet{penColor5}{orange!80!black} % Color of a curve in a plot
\colorlet{penColor6}{yellow!70!black} % Color of a curve in a plot
\colorlet{fill1}{penColor!20} % Color of fill in a plot
\colorlet{fill2}{penColor2!20} % Color of fill in a plot
\colorlet{fillp}{fill1} % Color of positive area
\colorlet{filln}{penColor2!20} % Color of negative area
\colorlet{fill3}{penColor3!20} % Fill
\colorlet{fill4}{penColor4!20} % Fill
\colorlet{fill5}{penColor5!20} % Fill
\colorlet{gridColor}{gray!50} % Color of grid in a plot

\newcommand{\surfaceColor}{violet}
\newcommand{\surfaceColorTwo}{redyellow}
\newcommand{\sliceColor}{greenyellow}




\pgfmathdeclarefunction{gauss}{2}{% gives gaussian
  \pgfmathparse{1/(#2*sqrt(2*pi))*exp(-((x-#1)^2)/(2*#2^2))}%
}


%%%%%%%%%%%%%
%% Vectors
%%%%%%%%%%%%%

%% Simple horiz vectors
\renewcommand{\vector}[1]{\left\langle #1\right\rangle}


%% %% Complex Horiz Vectors with angle brackets
%% \makeatletter
%% \renewcommand{\vector}[2][ , ]{\left\langle%
%%   \def\nextitem{\def\nextitem{#1}}%
%%   \@for \el:=#2\do{\nextitem\el}\right\rangle%
%% }
%% \makeatother

%% %% Vertical Vectors
%% \def\vector#1{\begin{bmatrix}\vecListA#1,,\end{bmatrix}}
%% \def\vecListA#1,{\if,#1,\else #1\cr \expandafter \vecListA \fi}

%%%%%%%%%%%%%
%% End of vectors
%%%%%%%%%%%%%

%\newcommand{\fullwidth}{}
%\newcommand{\normalwidth}{}



%% makes a snazzy t-chart for evaluating functions
%\newenvironment{tchart}{\rowcolors{2}{}{background!90!textColor}\array}{\endarray}

%%This is to help with formatting on future title pages.
\newenvironment{sectionOutcomes}{}{}



%% Flowchart stuff
%\tikzstyle{startstop} = [rectangle, rounded corners, minimum width=3cm, minimum height=1cm,text centered, draw=black]
%\tikzstyle{question} = [rectangle, minimum width=3cm, minimum height=1cm, text centered, draw=black]
%\tikzstyle{decision} = [trapezium, trapezium left angle=70, trapezium right angle=110, minimum width=3cm, minimum height=1cm, text centered, draw=black]
%\tikzstyle{question} = [rectangle, rounded corners, minimum width=3cm, minimum height=1cm,text centered, draw=black]
%\tikzstyle{process} = [rectangle, minimum width=3cm, minimum height=1cm, text centered, draw=black]
%\tikzstyle{decision} = [trapezium, trapezium left angle=70, trapezium right angle=110, minimum width=3cm, minimum height=1cm, text centered, draw=black]


\outcome{}

\title[Dig-In:]{Polynomial functions}


\begin{document}
\begin{abstract}
  Polynomials are some of our favorite functions. 
\end{abstract}
\maketitle


The functions you are most familiar with are probably polynomial
functions.

\section{What are polynomial functions?}

\begin{definition}
  A \dfn{polynomial function} in the variable $x$ is a function
  which can be written in the form
  \[
  f(x) = a_nx^n + a_{n-1}x^{n-1} + \dots + a_1 x + a_0
  \]
  where the $a_i$'s are all constants (called the \dfn{coefficients})
  and $n$ is a whole number (called the \dfn{degree} when $n\ne
  0$). The domain of a polynomial function is $(-\infty,\infty)$.
\end{definition}

\begin{question}
  Which of the following are polynomial functions?
  \begin{selectAll}
    \choice[correct]{$f(x) = 0$}
    \choice[correct]{$f(x) = -9$}
    \choice[correct]{$f(x) = 3x+1$}
    \choice{$f(x) = x^{1/2}-x +8$}
    \choice[correct]{$f(x) = (x+1)(x-1)+e^x - e^x $}
    \choice{$f(x) = \frac{x^2 - 3x + 2}{x-2}$}
    \choice[correct]{$f(x) = x^7-32x^6-\pi x^3+45/84$}
  \end{selectAll}
\end{question}

The phrase above ``in the variable $x$'' can actually change.
\[
y^2-4y +1
\]
is a polynomial in $y$, and
\[
\sin^2(x) + \sin(x) -3 
\]
is a polynomial in $\sin(x)$.


\section{What can the graphs look like?}

Fun fact:

\begin{theorem}[The Fundamental Theorem of Algebra]
  Every polynomial of the form
  \[
  a_n x^n + a_{n-1} x^{n-1} + \dots + a_1 x + a_0
  \]
  where the $a_i$'s are real (or even complex!) numbers has exactly
  $n$ (possibly repeated) complex roots.
\end{theorem}

Remember, a \dfn{root} is where a polynomial is zero. The theorem
above is a deep fact of mathematics. The great mathematician Gauss
%(spelled Gau\ss\ for fancy people)
proved the theorem in 1799 for his
doctoral thesis. 

The upshot as far as we are concerned is that when we plot a
polynomial of degree $n$, its graph will cross the $x$-axis at most
$n$ times.

\begin{example}
  Here we see the the graphs of four polynomial functions.
  \begin{image}
    \begin{tabular}{cc}
      \begin{tikzpicture}
        \begin{axis}[
          domain=-2:2,
          xmin=-2, xmax=2,
          ymin=-2, ymax=2,
          width=2.5in,
          axis lines =middle, xlabel=$x$, ylabel=$y$,
          every axis y label/.style={at=(current axis.above origin),anchor=south},
          every axis x label/.style={at=(current axis.right of origin),anchor=west},
          ]
	  \addplot [very thick, penColor, smooth] {5*x^5-5*x^4-5*x^3+5*x^2+.5*x -1};
          \node at (axis cs:1.2, 1 ) [penColor,anchor=west] {$A$};
        \end{axis}
      \end{tikzpicture}
      &
      \begin{tikzpicture}
        \begin{axis}[
          domain=-2:2,
          xmin=-2, xmax=2,
          ymin=-2, ymax=2,
          width=2.5in,
          axis lines =middle, xlabel=$x$, ylabel=$y$,
          every axis y label/.style={at=(current axis.above origin),anchor=south},
          every axis x label/.style={at=(current axis.right of origin),anchor=west},
          ]
	  \addplot [very thick, penColor2, smooth] {-5*x^5+5*x^4+5*x^3-4.25*x^2-.3*x +.5};
          \node at (axis cs:1.2, 1 ) [penColor2,anchor=west] {$B$};
        \end{axis}
      \end{tikzpicture}\\
      \begin{tikzpicture}
        \begin{axis}[
          domain=-2:2,
          xmin=-2, xmax=2,
          ymin=-2, ymax=2,
          width=2.5in,
          axis lines =middle, xlabel=$x$, ylabel=$y$,
          every axis y label/.style={at=(current axis.above origin),anchor=south},
          every axis x label/.style={at=(current axis.right of origin),anchor=west},
          ]
	  \addplot [very thick, penColor3, smooth] {5*x^6-5*x^5-5*x^4+5*x^3+x^2 -.5};
          \node at (axis cs:1.2, 1 ) [penColor3,anchor=west] {$C$};
        \end{axis}
      \end{tikzpicture}
      &
      \begin{tikzpicture}
        \begin{axis}[
          domain=-2:2,
          xmin=-2, xmax=2,
          ymin=-2, ymax=2,
          width=2.5in,
          axis lines =middle, xlabel=$x$, ylabel=$y$,
          every axis y label/.style={at=(current axis.above origin),anchor=south},
          every axis x label/.style={at=(current axis.right of origin),anchor=west},
          ]
	  \addplot [very thick, penColor4, smooth,samples=100] {-5*x^6+5*x^5+5*x^4-5*x^3-x^2+1.5*x+1};
          \node at (axis cs:1.2, 1 ) [penColor4,anchor=west] {$D$};
        \end{axis}
      \end{tikzpicture}
    \end{tabular}
  \end{image}
  For each of the curves, determine if the polynomial has
  \textbf{even} or \textbf{odd} degree, and if the leading coefficient
  (the one next to the highest power of $x$) of the polynomial is
  \textbf{positive} or \textbf{negative}.
  \begin{explanation}\hfil
    \begin{itemize}
    \item Curve $A$ is defined by an
      \wordChoice{\choice{even}\choice[correct]{odd}} degree
      polynomial with a \wordChoice{\choice[correct]{positive}\choice{negative}}
      leading term.
    \item Curve $B$ is defined by an
      \wordChoice{\choice{even}\choice[correct]{odd}} degree
      polynomial with a
      \wordChoice{\choice{positive}\choice[correct]{negative}} leading
      term.
    \item Curve $C$ is defined by an
      \wordChoice{\choice[correct]{even}\choice{odd}} degree
      polynomial with a \wordChoice{\choice[correct]{positive}\choice{negative}}
      leading term.
    \item Curve $A$ is defined by an
      \wordChoice{\choice[correct]{even}\choice{odd}} degree
      polynomial with a \wordChoice{\choice{positive}\choice[correct]{negative}}
      leading term.
    \end{itemize}
  \end{explanation}
\end{example}


%% \section{Connections to inverse functions}

%% Polynomials only have inverse functions if they are
%% one-to-one. Pratically this means that polynomials of \textit{even}
%% degree are \textit{never} invertible since they are never
%% one-to-one. Odd degree polynomials may or may not be invertible. This
%% is worth repeating in the language of algebra, and we will write it as
%% a warning.

%% \begin{warning}
%%   Consider $f(x) = x^2$. Suppose we tried to find the inverse of
%%   $x^2$. In this case we plug $f^{-1}(x)$ into $f$ and write
%%   \begin{align*}
%%     f(f^{-1}(x)) &= \left(f^{-1}(x)\right)^2\\
%%     x &= \left(f^{-1}(x)\right)^2
%%   \end{align*}
%% at this point you may be tempted to say ``$f^{-1}(x) = \sqrt{x}$'' but
%% this is not correct. The correct next line is:
%% \[
%% f^{-1}(x) = \pm \sqrt{x},
%% \]
%% but this \textbf{is not a function} from the real numbers to the real
%% numbers.
%% \end{warning}


%% $x^\text{even}$ vs $x^\text{odd}$ and simple nth roots



\end{document}
