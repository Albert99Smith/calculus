\documentclass{ximera}

%\usepackage{todonotes}
%\usepackage{mathtools} %% Required for wide table Curl and Greens
%\usepackage{cuted} %% Required for wide table Curl and Greens
\newcommand{\todo}{}

\usepackage{esint} % for \oiint
\ifxake%%https://math.meta.stackexchange.com/questions/9973/how-do-you-render-a-closed-surface-double-integral
\renewcommand{\oiint}{{\large\bigcirc}\kern-1.56em\iint}
\fi


\graphicspath{
  {./}
  {ximeraTutorial/}
  {basicPhilosophy/}
  {functionsOfSeveralVariables/}
  {normalVectors/}
  {lagrangeMultipliers/}
  {vectorFields/}
  {greensTheorem/}
  {shapeOfThingsToCome/}
  {dotProducts/}
  {partialDerivativesAndTheGradientVector/}
  {../productAndQuotientRules/exercises/}
  {../normalVectors/exercisesParametricPlots/}
  {../continuityOfFunctionsOfSeveralVariables/exercises/}
  {../partialDerivativesAndTheGradientVector/exercises/}
  {../directionalDerivativeAndChainRule/exercises/}
  {../commonCoordinates/exercisesCylindricalCoordinates/}
  {../commonCoordinates/exercisesSphericalCoordinates/}
  {../greensTheorem/exercisesCurlAndLineIntegrals/}
  {../greensTheorem/exercisesDivergenceAndLineIntegrals/}
  {../shapeOfThingsToCome/exercisesDivergenceTheorem/}
  {../greensTheorem/}
  {../shapeOfThingsToCome/}
  {../separableDifferentialEquations/exercises/}
  {vectorFields/}
}

\newcommand{\mooculus}{\textsf{\textbf{MOOC}\textnormal{\textsf{ULUS}}}}

\usepackage{tkz-euclide}\usepackage{tikz}
\usepackage{tikz-cd}
\usetikzlibrary{arrows}
\tikzset{>=stealth,commutative diagrams/.cd,
  arrow style=tikz,diagrams={>=stealth}} %% cool arrow head
\tikzset{shorten <>/.style={ shorten >=#1, shorten <=#1 } } %% allows shorter vectors

\usetikzlibrary{backgrounds} %% for boxes around graphs
\usetikzlibrary{shapes,positioning}  %% Clouds and stars
\usetikzlibrary{matrix} %% for matrix
\usepgfplotslibrary{polar} %% for polar plots
\usepgfplotslibrary{fillbetween} %% to shade area between curves in TikZ
\usetkzobj{all}
\usepackage[makeroom]{cancel} %% for strike outs
%\usepackage{mathtools} %% for pretty underbrace % Breaks Ximera
%\usepackage{multicol}
\usepackage{pgffor} %% required for integral for loops



%% http://tex.stackexchange.com/questions/66490/drawing-a-tikz-arc-specifying-the-center
%% Draws beach ball
\tikzset{pics/carc/.style args={#1:#2:#3}{code={\draw[pic actions] (#1:#3) arc(#1:#2:#3);}}}



\usepackage{array}
\setlength{\extrarowheight}{+.1cm}
\newdimen\digitwidth
\settowidth\digitwidth{9}
\def\divrule#1#2{
\noalign{\moveright#1\digitwidth
\vbox{\hrule width#2\digitwidth}}}





\newcommand{\RR}{\mathbb R}
\newcommand{\R}{\mathbb R}
\newcommand{\N}{\mathbb N}
\newcommand{\Z}{\mathbb Z}

\newcommand{\sagemath}{\textsf{SageMath}}


%\renewcommand{\d}{\,d\!}
\renewcommand{\d}{\mathop{}\!d}
\newcommand{\dd}[2][]{\frac{\d #1}{\d #2}}
\newcommand{\pp}[2][]{\frac{\partial #1}{\partial #2}}
\renewcommand{\l}{\ell}
\newcommand{\ddx}{\frac{d}{\d x}}

\newcommand{\zeroOverZero}{\ensuremath{\boldsymbol{\tfrac{0}{0}}}}
\newcommand{\inftyOverInfty}{\ensuremath{\boldsymbol{\tfrac{\infty}{\infty}}}}
\newcommand{\zeroOverInfty}{\ensuremath{\boldsymbol{\tfrac{0}{\infty}}}}
\newcommand{\zeroTimesInfty}{\ensuremath{\small\boldsymbol{0\cdot \infty}}}
\newcommand{\inftyMinusInfty}{\ensuremath{\small\boldsymbol{\infty - \infty}}}
\newcommand{\oneToInfty}{\ensuremath{\boldsymbol{1^\infty}}}
\newcommand{\zeroToZero}{\ensuremath{\boldsymbol{0^0}}}
\newcommand{\inftyToZero}{\ensuremath{\boldsymbol{\infty^0}}}



\newcommand{\numOverZero}{\ensuremath{\boldsymbol{\tfrac{\#}{0}}}}
\newcommand{\dfn}{\textbf}
%\newcommand{\unit}{\,\mathrm}
\newcommand{\unit}{\mathop{}\!\mathrm}
\newcommand{\eval}[1]{\bigg[ #1 \bigg]}
\newcommand{\seq}[1]{\left( #1 \right)}
\renewcommand{\epsilon}{\varepsilon}
\renewcommand{\phi}{\varphi}


\renewcommand{\iff}{\Leftrightarrow}

\DeclareMathOperator{\arccot}{arccot}
\DeclareMathOperator{\arcsec}{arcsec}
\DeclareMathOperator{\arccsc}{arccsc}
\DeclareMathOperator{\si}{Si}
\DeclareMathOperator{\scal}{scal}
\DeclareMathOperator{\sign}{sign}


%% \newcommand{\tightoverset}[2]{% for arrow vec
%%   \mathop{#2}\limits^{\vbox to -.5ex{\kern-0.75ex\hbox{$#1$}\vss}}}
\newcommand{\arrowvec}[1]{{\overset{\rightharpoonup}{#1}}}
%\renewcommand{\vec}[1]{\arrowvec{\mathbf{#1}}}
\renewcommand{\vec}[1]{{\overset{\boldsymbol{\rightharpoonup}}{\mathbf{#1}}}\hspace{0in}}

\newcommand{\point}[1]{\left(#1\right)} %this allows \vector{ to be changed to \vector{ with a quick find and replace
\newcommand{\pt}[1]{\mathbf{#1}} %this allows \vec{ to be changed to \vec{ with a quick find and replace
\newcommand{\Lim}[2]{\lim_{\point{#1} \to \point{#2}}} %Bart, I changed this to point since I want to use it.  It runs through both of the exercise and exerciseE files in limits section, which is why it was in each document to start with.

\DeclareMathOperator{\proj}{\mathbf{proj}}
\newcommand{\veci}{{\boldsymbol{\hat{\imath}}}}
\newcommand{\vecj}{{\boldsymbol{\hat{\jmath}}}}
\newcommand{\veck}{{\boldsymbol{\hat{k}}}}
\newcommand{\vecl}{\vec{\boldsymbol{\l}}}
\newcommand{\uvec}[1]{\mathbf{\hat{#1}}}
\newcommand{\utan}{\mathbf{\hat{t}}}
\newcommand{\unormal}{\mathbf{\hat{n}}}
\newcommand{\ubinormal}{\mathbf{\hat{b}}}

\newcommand{\dotp}{\bullet}
\newcommand{\cross}{\boldsymbol\times}
\newcommand{\grad}{\boldsymbol\nabla}
\newcommand{\divergence}{\grad\dotp}
\newcommand{\curl}{\grad\cross}
%\DeclareMathOperator{\divergence}{divergence}
%\DeclareMathOperator{\curl}[1]{\grad\cross #1}
\newcommand{\lto}{\mathop{\longrightarrow\,}\limits}

\renewcommand{\bar}{\overline}

\colorlet{textColor}{black}
\colorlet{background}{white}
\colorlet{penColor}{blue!50!black} % Color of a curve in a plot
\colorlet{penColor2}{red!50!black}% Color of a curve in a plot
\colorlet{penColor3}{red!50!blue} % Color of a curve in a plot
\colorlet{penColor4}{green!50!black} % Color of a curve in a plot
\colorlet{penColor5}{orange!80!black} % Color of a curve in a plot
\colorlet{penColor6}{yellow!70!black} % Color of a curve in a plot
\colorlet{fill1}{penColor!20} % Color of fill in a plot
\colorlet{fill2}{penColor2!20} % Color of fill in a plot
\colorlet{fillp}{fill1} % Color of positive area
\colorlet{filln}{penColor2!20} % Color of negative area
\colorlet{fill3}{penColor3!20} % Fill
\colorlet{fill4}{penColor4!20} % Fill
\colorlet{fill5}{penColor5!20} % Fill
\colorlet{gridColor}{gray!50} % Color of grid in a plot

\newcommand{\surfaceColor}{violet}
\newcommand{\surfaceColorTwo}{redyellow}
\newcommand{\sliceColor}{greenyellow}




\pgfmathdeclarefunction{gauss}{2}{% gives gaussian
  \pgfmathparse{1/(#2*sqrt(2*pi))*exp(-((x-#1)^2)/(2*#2^2))}%
}


%%%%%%%%%%%%%
%% Vectors
%%%%%%%%%%%%%

%% Simple horiz vectors
\renewcommand{\vector}[1]{\left\langle #1\right\rangle}


%% %% Complex Horiz Vectors with angle brackets
%% \makeatletter
%% \renewcommand{\vector}[2][ , ]{\left\langle%
%%   \def\nextitem{\def\nextitem{#1}}%
%%   \@for \el:=#2\do{\nextitem\el}\right\rangle%
%% }
%% \makeatother

%% %% Vertical Vectors
%% \def\vector#1{\begin{bmatrix}\vecListA#1,,\end{bmatrix}}
%% \def\vecListA#1,{\if,#1,\else #1\cr \expandafter \vecListA \fi}

%%%%%%%%%%%%%
%% End of vectors
%%%%%%%%%%%%%

%\newcommand{\fullwidth}{}
%\newcommand{\normalwidth}{}



%% makes a snazzy t-chart for evaluating functions
%\newenvironment{tchart}{\rowcolors{2}{}{background!90!textColor}\array}{\endarray}

%%This is to help with formatting on future title pages.
\newenvironment{sectionOutcomes}{}{}



%% Flowchart stuff
%\tikzstyle{startstop} = [rectangle, rounded corners, minimum width=3cm, minimum height=1cm,text centered, draw=black]
%\tikzstyle{question} = [rectangle, minimum width=3cm, minimum height=1cm, text centered, draw=black]
%\tikzstyle{decision} = [trapezium, trapezium left angle=70, trapezium right angle=110, minimum width=3cm, minimum height=1cm, text centered, draw=black]
%\tikzstyle{question} = [rectangle, rounded corners, minimum width=3cm, minimum height=1cm,text centered, draw=black]
%\tikzstyle{process} = [rectangle, minimum width=3cm, minimum height=1cm, text centered, draw=black]
%\tikzstyle{decision} = [trapezium, trapezium left angle=70, trapezium right angle=110, minimum width=3cm, minimum height=1cm, text centered, draw=black]


\title[Dig-In:]{Applied related rates}

\begin{document}
\begin{abstract}
\end{abstract}
\maketitle


Suppose we have two variables $x$ and $y$ which are both changing with
respect to time.  A \textit{related rates} problem is a problem where
we know one rate at a given instant, and wish to find the other.  If
$y$ is written in terms of $x$, and we are given $\dd[x]{t}$, then it
is easy to find $\dd[y]{t}$ using the chain rule:
\[
\dd[y]{t}=y'(x(t))\cdot x'(t).
\]
In many cases, particularly the interesting ones, our functions will
be related in some other way. Nevertheless, in each case we'll use the
same strategy:

REWRITE

\begin{example}
You are inflating a spherical balloon at the rate of 7 cm${}^3$/sec.  How
fast is its radius increasing when the radius is 4 cm?
\end{example}

\begin{solution}
To start, \textbf{draw a picture}.

\begin{tikzpicture}
%\draw[penColor!50!background,very thick] (0,0) ellipse (2 and 1);
\draw[very thick,penColor!20!background] (2,0) arc (0:180:2 and .7);% top half of ellipse
\draw [penColor, very thick] (0,0) circle [radius=2];
\draw[penColor2, dashed, very thick] (0,0) -- (2,0);
\node [below,penColor2] at (1,0) {$r=4$ cm};
\draw[very thick,penColor] (-2,0) arc (180:360:2 and .7);% bottom half of ellipse
\node [penColor,left] at (-1.5,1.42) {$\dd[V]{t} = 7$ cm$^3$/sec};
\node [penColor, right] at (1.5,-1.42) {$V = \frac{4\pi r^3}{3}$ cm$^3$};
\end{tikzpicture}

Next we need to \textbf{find an equation}.  Thinking of the variables
$r$ and $V$ as functions of time, they are related by the equation
\[
V(t)=\frac{4\pi (r(t))^3}{3}.
\]

Now we need to \textbf{differentiate the equation}.  Taking the
derivative of both sides gives 
\[
\dd[V]{t}=4\pi (r(t))^2\cdot r'(t).
\]  
Finally we \textbf{evaluate the equation at the desired values}. Set
$r(t)= 4$ cm and $\dd[V]{t}$ = 7 cm$^3$/sec. Write 
\begin{align*}
7 &=4\pi 4^2r'(t),\\
r'(t) &=7/(64\pi)~\text{cm/sec}.
\end{align*}
\end{solution}

\begin{example} Water is poured into a conical container at the rate of 10
cm${}^3$/sec.  The cone points directly down, and it has a height of
30 cm and a base radius of 10 cm.  How fast is the water level rising
when the water is 4 cm deep?
\end{example}

\begin{solution}
To start, \textbf{draw a picture}.

\begin{tikzpicture}
\draw[penColor,very thick] (0,4) ellipse (4 and 1);
\draw[very thick,penColor!20!background] (2,2) arc (0:180:2 and .5);% top half of ellipse
\draw[very thick,penColor] (-2,2) arc (180:360:2 and .5);% bottom half of ellipse
\draw[penColor, very thick] (3.97,3.85) -- (0,0);
\draw[penColor, very thick] (-3.97,3.85) -- (0,0);
\draw[penColor, very thick] (0,4) -- (4,4);
\draw[penColor!50!background, very thick] (0,2) -- (2,2);
\draw[->,line width=0.4cm, penColor!20!background] (0,6) -- (0,4.25);
\draw[dashed, penColor2, very thick] (2.1,0) -- (2.1,2);
\draw[dashed, penColor, very thick] (-4.1,0) -- (-4.1,4);
\node[right, penColor] at (.4,5.6) {$\dd[V]{t} = 10$ cm$^3$/sec};
\node[below, penColor] at (2,4) {$10$ cm};
\node[above, penColor] at (1,2) {$r$ cm};
\node[right, penColor2] at (2.1,1) {$h(t) = 4$ cm};
\node[left, penColor] at (-4.1,2) {$30$ cm};
\end{tikzpicture}

Note, no attempt was made to draw this picture to scale, rather we
want all of the relevant information to be available to the
mathematician.

Now we need to \textbf{find an equation}. The formula for the volume of a cone tells us that 
\[
V = \frac{\pi}{3} r^2 h.
\]

Now we must \textbf{differentiate the equation}. We should use implicit differentiation, and treat each of the variables as functions of $t$. Write
\begin{equation}\label{equation:cone/water}
\dd[V]{t} = \frac{\pi}{3}\left(2rh \dd[r]{t} + r^2 \dd[h]{t}\right).
\end{equation}

At this point we \textbf{evaluate the equation at the desired values}.
At first something seems to be wrong, we do not know $\dd[r]{t}$.
However, the dimensions of the cone of water must have the same
proportions as those of the container.  That is, because of similar
triangles, 
\[
\frac{r}{h}=\frac{10}{30} \qquad\text{so}\qquad r={h/3}.
\]  
In particular, we see that when $h = 4$, $r=4/3$ and 
\[
\dd[r]{t} = \frac{1}{3}\cdot \dd[h]{t}.
\]
Now we can \textbf{evaluate the equation at the desired
  values}. Starting with Equation~\ref{equation:cone/water}, we plug
in $\dd[V]{t} = 10$, $r = 4/3$, $\dd[r]{t} = \frac{1}{3}\cdot \dd[h]{t}$
and $h=4$. Write
\begin{align*}
10 &= \frac{\pi}{3}\left(2\cdot \frac{4}{3}\cdot 4 \cdot\frac{1}{3}\cdot\dd[h]{t} + \left(\frac{4}{3}\right)^2 \dd[h]{t}\right)\\
10 &= \frac{\pi}{3}\left(\frac{32}{9}\dd[h]{t} + \frac{16}{9} \dd[h]{t}\right)\\
10 &= \frac{16\pi}{9}\dd[h]{t}\\
\frac{90}{16\pi} &= \dd[h]{t}.
\end{align*}
Thus, $\dd[h]{t}=\frac{90}{16\pi}$ cm/sec.
\end{solution}



\end{document}
