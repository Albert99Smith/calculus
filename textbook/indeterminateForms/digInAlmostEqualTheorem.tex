\documentclass{ximera}

%\usepackage{todonotes}
%\usepackage{mathtools} %% Required for wide table Curl and Greens
%\usepackage{cuted} %% Required for wide table Curl and Greens
\newcommand{\todo}{}

\usepackage{esint} % for \oiint
\ifxake%%https://math.meta.stackexchange.com/questions/9973/how-do-you-render-a-closed-surface-double-integral
\renewcommand{\oiint}{{\large\bigcirc}\kern-1.56em\iint}
\fi


\graphicspath{
  {./}
  {ximeraTutorial/}
  {basicPhilosophy/}
  {functionsOfSeveralVariables/}
  {normalVectors/}
  {lagrangeMultipliers/}
  {vectorFields/}
  {greensTheorem/}
  {shapeOfThingsToCome/}
  {dotProducts/}
  {partialDerivativesAndTheGradientVector/}
  {../productAndQuotientRules/exercises/}
  {../normalVectors/exercisesParametricPlots/}
  {../continuityOfFunctionsOfSeveralVariables/exercises/}
  {../partialDerivativesAndTheGradientVector/exercises/}
  {../directionalDerivativeAndChainRule/exercises/}
  {../commonCoordinates/exercisesCylindricalCoordinates/}
  {../commonCoordinates/exercisesSphericalCoordinates/}
  {../greensTheorem/exercisesCurlAndLineIntegrals/}
  {../greensTheorem/exercisesDivergenceAndLineIntegrals/}
  {../shapeOfThingsToCome/exercisesDivergenceTheorem/}
  {../greensTheorem/}
  {../shapeOfThingsToCome/}
  {../separableDifferentialEquations/exercises/}
  {vectorFields/}
}

\newcommand{\mooculus}{\textsf{\textbf{MOOC}\textnormal{\textsf{ULUS}}}}

\usepackage{tkz-euclide}\usepackage{tikz}
\usepackage{tikz-cd}
\usetikzlibrary{arrows}
\tikzset{>=stealth,commutative diagrams/.cd,
  arrow style=tikz,diagrams={>=stealth}} %% cool arrow head
\tikzset{shorten <>/.style={ shorten >=#1, shorten <=#1 } } %% allows shorter vectors

\usetikzlibrary{backgrounds} %% for boxes around graphs
\usetikzlibrary{shapes,positioning}  %% Clouds and stars
\usetikzlibrary{matrix} %% for matrix
\usepgfplotslibrary{polar} %% for polar plots
\usepgfplotslibrary{fillbetween} %% to shade area between curves in TikZ
\usetkzobj{all}
\usepackage[makeroom]{cancel} %% for strike outs
%\usepackage{mathtools} %% for pretty underbrace % Breaks Ximera
%\usepackage{multicol}
\usepackage{pgffor} %% required for integral for loops



%% http://tex.stackexchange.com/questions/66490/drawing-a-tikz-arc-specifying-the-center
%% Draws beach ball
\tikzset{pics/carc/.style args={#1:#2:#3}{code={\draw[pic actions] (#1:#3) arc(#1:#2:#3);}}}



\usepackage{array}
\setlength{\extrarowheight}{+.1cm}
\newdimen\digitwidth
\settowidth\digitwidth{9}
\def\divrule#1#2{
\noalign{\moveright#1\digitwidth
\vbox{\hrule width#2\digitwidth}}}





\newcommand{\RR}{\mathbb R}
\newcommand{\R}{\mathbb R}
\newcommand{\N}{\mathbb N}
\newcommand{\Z}{\mathbb Z}

\newcommand{\sagemath}{\textsf{SageMath}}


%\renewcommand{\d}{\,d\!}
\renewcommand{\d}{\mathop{}\!d}
\newcommand{\dd}[2][]{\frac{\d #1}{\d #2}}
\newcommand{\pp}[2][]{\frac{\partial #1}{\partial #2}}
\renewcommand{\l}{\ell}
\newcommand{\ddx}{\frac{d}{\d x}}

\newcommand{\zeroOverZero}{\ensuremath{\boldsymbol{\tfrac{0}{0}}}}
\newcommand{\inftyOverInfty}{\ensuremath{\boldsymbol{\tfrac{\infty}{\infty}}}}
\newcommand{\zeroOverInfty}{\ensuremath{\boldsymbol{\tfrac{0}{\infty}}}}
\newcommand{\zeroTimesInfty}{\ensuremath{\small\boldsymbol{0\cdot \infty}}}
\newcommand{\inftyMinusInfty}{\ensuremath{\small\boldsymbol{\infty - \infty}}}
\newcommand{\oneToInfty}{\ensuremath{\boldsymbol{1^\infty}}}
\newcommand{\zeroToZero}{\ensuremath{\boldsymbol{0^0}}}
\newcommand{\inftyToZero}{\ensuremath{\boldsymbol{\infty^0}}}



\newcommand{\numOverZero}{\ensuremath{\boldsymbol{\tfrac{\#}{0}}}}
\newcommand{\dfn}{\textbf}
%\newcommand{\unit}{\,\mathrm}
\newcommand{\unit}{\mathop{}\!\mathrm}
\newcommand{\eval}[1]{\bigg[ #1 \bigg]}
\newcommand{\seq}[1]{\left( #1 \right)}
\renewcommand{\epsilon}{\varepsilon}
\renewcommand{\phi}{\varphi}


\renewcommand{\iff}{\Leftrightarrow}

\DeclareMathOperator{\arccot}{arccot}
\DeclareMathOperator{\arcsec}{arcsec}
\DeclareMathOperator{\arccsc}{arccsc}
\DeclareMathOperator{\si}{Si}
\DeclareMathOperator{\scal}{scal}
\DeclareMathOperator{\sign}{sign}


%% \newcommand{\tightoverset}[2]{% for arrow vec
%%   \mathop{#2}\limits^{\vbox to -.5ex{\kern-0.75ex\hbox{$#1$}\vss}}}
\newcommand{\arrowvec}[1]{{\overset{\rightharpoonup}{#1}}}
%\renewcommand{\vec}[1]{\arrowvec{\mathbf{#1}}}
\renewcommand{\vec}[1]{{\overset{\boldsymbol{\rightharpoonup}}{\mathbf{#1}}}\hspace{0in}}

\newcommand{\point}[1]{\left(#1\right)} %this allows \vector{ to be changed to \vector{ with a quick find and replace
\newcommand{\pt}[1]{\mathbf{#1}} %this allows \vec{ to be changed to \vec{ with a quick find and replace
\newcommand{\Lim}[2]{\lim_{\point{#1} \to \point{#2}}} %Bart, I changed this to point since I want to use it.  It runs through both of the exercise and exerciseE files in limits section, which is why it was in each document to start with.

\DeclareMathOperator{\proj}{\mathbf{proj}}
\newcommand{\veci}{{\boldsymbol{\hat{\imath}}}}
\newcommand{\vecj}{{\boldsymbol{\hat{\jmath}}}}
\newcommand{\veck}{{\boldsymbol{\hat{k}}}}
\newcommand{\vecl}{\vec{\boldsymbol{\l}}}
\newcommand{\uvec}[1]{\mathbf{\hat{#1}}}
\newcommand{\utan}{\mathbf{\hat{t}}}
\newcommand{\unormal}{\mathbf{\hat{n}}}
\newcommand{\ubinormal}{\mathbf{\hat{b}}}

\newcommand{\dotp}{\bullet}
\newcommand{\cross}{\boldsymbol\times}
\newcommand{\grad}{\boldsymbol\nabla}
\newcommand{\divergence}{\grad\dotp}
\newcommand{\curl}{\grad\cross}
%\DeclareMathOperator{\divergence}{divergence}
%\DeclareMathOperator{\curl}[1]{\grad\cross #1}
\newcommand{\lto}{\mathop{\longrightarrow\,}\limits}

\renewcommand{\bar}{\overline}

\colorlet{textColor}{black}
\colorlet{background}{white}
\colorlet{penColor}{blue!50!black} % Color of a curve in a plot
\colorlet{penColor2}{red!50!black}% Color of a curve in a plot
\colorlet{penColor3}{red!50!blue} % Color of a curve in a plot
\colorlet{penColor4}{green!50!black} % Color of a curve in a plot
\colorlet{penColor5}{orange!80!black} % Color of a curve in a plot
\colorlet{penColor6}{yellow!70!black} % Color of a curve in a plot
\colorlet{fill1}{penColor!20} % Color of fill in a plot
\colorlet{fill2}{penColor2!20} % Color of fill in a plot
\colorlet{fillp}{fill1} % Color of positive area
\colorlet{filln}{penColor2!20} % Color of negative area
\colorlet{fill3}{penColor3!20} % Fill
\colorlet{fill4}{penColor4!20} % Fill
\colorlet{fill5}{penColor5!20} % Fill
\colorlet{gridColor}{gray!50} % Color of grid in a plot

\newcommand{\surfaceColor}{violet}
\newcommand{\surfaceColorTwo}{redyellow}
\newcommand{\sliceColor}{greenyellow}




\pgfmathdeclarefunction{gauss}{2}{% gives gaussian
  \pgfmathparse{1/(#2*sqrt(2*pi))*exp(-((x-#1)^2)/(2*#2^2))}%
}


%%%%%%%%%%%%%
%% Vectors
%%%%%%%%%%%%%

%% Simple horiz vectors
\renewcommand{\vector}[1]{\left\langle #1\right\rangle}


%% %% Complex Horiz Vectors with angle brackets
%% \makeatletter
%% \renewcommand{\vector}[2][ , ]{\left\langle%
%%   \def\nextitem{\def\nextitem{#1}}%
%%   \@for \el:=#2\do{\nextitem\el}\right\rangle%
%% }
%% \makeatother

%% %% Vertical Vectors
%% \def\vector#1{\begin{bmatrix}\vecListA#1,,\end{bmatrix}}
%% \def\vecListA#1,{\if,#1,\else #1\cr \expandafter \vecListA \fi}

%%%%%%%%%%%%%
%% End of vectors
%%%%%%%%%%%%%

%\newcommand{\fullwidth}{}
%\newcommand{\normalwidth}{}



%% makes a snazzy t-chart for evaluating functions
%\newenvironment{tchart}{\rowcolors{2}{}{background!90!textColor}\array}{\endarray}

%%This is to help with formatting on future title pages.
\newenvironment{sectionOutcomes}{}{}



%% Flowchart stuff
%\tikzstyle{startstop} = [rectangle, rounded corners, minimum width=3cm, minimum height=1cm,text centered, draw=black]
%\tikzstyle{question} = [rectangle, minimum width=3cm, minimum height=1cm, text centered, draw=black]
%\tikzstyle{decision} = [trapezium, trapezium left angle=70, trapezium right angle=110, minimum width=3cm, minimum height=1cm, text centered, draw=black]
%\tikzstyle{question} = [rectangle, rounded corners, minimum width=3cm, minimum height=1cm,text centered, draw=black]
%\tikzstyle{process} = [rectangle, minimum width=3cm, minimum height=1cm, text centered, draw=black]
%\tikzstyle{decision} = [trapezium, trapezium left angle=70, trapezium right angle=110, minimum width=3cm, minimum height=1cm, text centered, draw=black]


\title[Dig-In:]{The Almost Equal Theorem}

\begin{document}
\begin{abstract}
  We want to solve limits that cannot be solved directly using the
  Limit Laws.
\end{abstract}

\maketitle

%This chapter will consist of three dig in's 1. The Almost Equal Theorem 2. Limits of the form non-zero over zero 3. Determinate and indeterminate forms 

In the last section, we were interested in the limits we could compute
using continuity and the limit laws. What about the limits that cannot
be directly computed using these methods?  Is it possible to determine
these limits?  Often times, it is.  We will use two important tools to
find these limits.  The first is the following theorem:

\begin{theorem}[Almost Equal Theorem]
Let $f$ and $g$ be two functions such that $f(x)=g(x)$ for every $x$
in some open interval $I$ except for at a finite set of points in $I$.
Then,
\[
\lim_{x\to a} f(x)= \lim_{x\to a} g(x),
\]
provided at least one of these limits exists.  We call $f$ and $g$
\dfn{almost equal functions} in the context of considering the limit
as $x$ approaches $a$.
\end{theorem}

Here's the upshot: If two functions agree at every point near $a$,
then they have the same limit at $a$.
\begin{image}
      \begin{tikzpicture}
	\begin{axis}[
            domain=-2:4,
            ymin=-3,
            ymax=3,
            width=2.5in,
            axis lines =middle, xlabel=$x$, ylabel=$y$,
            every axis y label/.style={at=(current axis.above origin),anchor=south},
            every axis x label/.style={at=(current axis.right of origin),anchor=west},
            xtick={-2,...,4},
            ytick={-3,...,3},
          ]
	  \addplot [very thick, penColor, smooth] {-.5*x^3 + 1.6*x^2-.2*x -1};
          \addplot[color=penColor,fill=background,only marks,mark=*] coordinates{(2,1)};  %% open hole
          \addplot[color=penColor,fill=penColor,only marks,mark=*] coordinates{(2,-1.5)};  %% closed hole
        \end{axis}
      \end{tikzpicture}\qquad
      \begin{tikzpicture}
	\begin{axis}[
            domain=-2:4,
            ymin=-3,
            ymax=3,
            width=2.5in,
            axis lines =middle, xlabel=$x$, ylabel=$y$,
            every axis y label/.style={at=(current axis.above origin),anchor=south},
            every axis x label/.style={at=(current axis.right of origin),anchor=west},
            xtick={-2,...,4},
            ytick={-3,...,3},
          ]
	  \addplot [very thick, penColor, smooth] {-.5*x^3 + 1.6*x^2-.2*x -1};
          \addplot[color=penColor,fill=background,only marks,mark=*] coordinates{(.4,-.86)};  %% open hole
          \addplot[color=penColor,fill=penColor,only marks,mark=*] coordinates{(.4,2)};  %% closed hole
        \end{axis}
      \end{tikzpicture}
\end{image}
\begin{question}
  Are the two functions that produce the graphs above almost equal on
  the interval $(-1,3)$?
    \begin{multipleChoice}
      \choice[correct]{Yes}
      \choice{No}
    \end{multipleChoice}
  \begin{feedback}
    Since these functions are equal on the open interval $(-1,3)$,
    except for at two disconnected points, they are almost equal on
    the interval.
  \end{feedback}
\end{question}
This should make sense. We already know that the limit does not
``see'' what is happening exactly at $a$, it only sees what is
happening nearer and nearer to $a$.

Let's consider an example that we looked at in the previous section:

\begin{example}
  Compute:
  \[
  \lim_{x\to 2}\frac{x^2-3x+2}{x-2}
  \]
  \begin{explanation}
    In the previous section, we decided that we could not find this
    limit using the Quotient Law because the denominator is equal to 0
    at $x=2$.  Can we use the Almost Equal Thoerem?

    Well note that if we assume $x\ne 2$, then we can write
    \[
    \frac{x^2-3x+2}{x-2} = \frac{(x-2)(x-1)}{(x-2)} = x-1.
    \]
    Since this relation is true provided $x\ne 2$, we have that
    \[
    \frac{x^2-3x+2}{x-2} \text{ is almost equal to }x-1.
    \]
    \begin{image}
      \begin{tikzpicture}
        \begin{axis}[
            domain=-2:4,
            width=2.5in,
            axis lines =middle, xlabel=$x$, ylabel=$y$,
            every axis y label/.style={at=(current axis.above origin),anchor=south},
            every axis x label/.style={at=(current axis.right of origin),anchor=west},
            xtick={-2,...,4},
            ytick={-3,...,3},
          ]
	  \addplot [very thick, penColor, smooth] {x-1};
        \end{axis}
        \node [penColor] at (2,-.75) {$y= x-1$};
      \end{tikzpicture}
      \qquad
      \begin{tikzpicture}
	\begin{axis}[
            domain=-2:4,
            width=2.5in,
            axis lines =middle, xlabel=$x$, ylabel=$y$,
            every axis y label/.style={at=(current axis.above origin),anchor=south},
            every axis x label/.style={at=(current axis.right of origin),anchor=west},
            xtick={-2,...,4},
            ytick={-3,...,3},
          ]
	  \addplot [very thick, penColor, smooth] {x-1};
          \addplot[color=penColor,fill=background,only marks,mark=*] coordinates{(2,1)};  %% open hole
        \end{axis}
        \node [penColor] at (2,-.5) {$y=\frac{x^2-3x+2}{x-2}$};
      \end{tikzpicture}
    \end{image}
    This means
    \[
    \lim_{x\to 2}\frac{x^2-3x+2}{x-2} = \lim_{x\to 2} (x-1).
    \]
    Ah! $\lim_{x\to 2} (x-1) =1$. Hence
    \[
    \lim_{x\to 2}\frac{x^2-3x+2}{x-2} = \lim_{x\to 2} (x-1) = 1.
    \]
  \end{explanation}
\end{example} 

This trick does not always work.  For example, consider
\[
\lim_{x\to 3}\frac{x^2-3x+2}{x-3}.
\]
First note that if we plug $3$ into the denominator of this fraction,
the denominator will equal $0$.  Therefore, we cannot use the Quotient
Law to solve this problem.  Factoring will give us
$\frac{x^2-3x+2}{x-3}=\frac{(x-2)(x-1)}{x-3}$ and nothing cancels.  We
seem to be stuck.  We need to develop some new tools and come back to
this example.

In order to give us a concrete way of talking about the differences
between these two examples, we are going to introduce a new idea, the
\textit{form} of a limit.

\begin{definition}
  A limit
  \[
  \lim_{x\to a} \frac{f(x)}{g(x)}
  \]
  is said to be of the form \zeroOverZero\ if
  \[
  \lim_{x\to a} f(x) = 0\qquad\text{and}\qquad \lim_{x\to a} g(x) =0.
  \]
\end{definition}

\begin{question}
  Which of the following limits are of the form \zeroOverZero?
  \begin{multipleChoice}
    \choice[correct]{$\lim_{x\to 0}\frac{\sin(x)}{x}$}
    \choice{$\lim_{x\to 0}\frac{\cos(x)}{x}$}
    \choice{$\lim_{x\to 0}\frac{x^2-3x+2}{x-2}$}
    \choice[correct]{$\lim_{x\to 2}\frac{x^2-3x+2}{x-2}$}
    \choice{$\lim_{x\to 3}\frac{x^2-3x+2}{x-3}$}
  \end{multipleChoice}
\end{question}

Someone might say, ``So what if a limit is of the form
\zeroOverZero. It's just zero right?'' Wrong. Limits of the form
\zeroOverZero\ can in fact equal any number and beyond.

%% \begin{definition}
%%   Consider
%%   \[
%%   \lim_{x\to a} f(x).
%%   \]
%%   The \dfn{form} of this limit is a symbol which is not a number that
%%   results when $a$ is plugged directly into $f(x)$ and the result is
%%   simplified as much as possible.  Because a form is not a number, we
%%   will not say that a limit equals its form.  We will use the symbol
%%   $\sim$ to say that a limit has a certain form.
%% \end{definition} 

Consider the first example.
\[
\lim_{x\to 2}\frac{x^2-3x+2}{x-2}.
\]
Since
\[
\lim_{x\to 2}\left(x^2-3x+2\right) = 0\qquad\text{and}\qquad \lim_{x\to
  2}\left(x-2\right) = 0
\]
we see that $\lim_{x\to 2}\frac{x^2-3x+2}{x-2}$ is of the form
\zeroOverZero.

Knowing this form is informative.  Since we know that both the
numerator and the denominator are approaching $0$ as $x$ approaches
$2$, we suspect that there might be a common factor in both the top
and the bottom which are going to $0$.  If we could find that factor
and cancel it, we would have a function which is almost equal to the
original function and whose limit we may be able to find using the
Limit Laws.  In fact, this is exactly what happened in this example
when we factored out the $(x-2)$ term.

For now, lets consider some more examples of the form \zeroOverZero.

\begin{example}
  Compute:
  \[
  \lim_{x\to1}\frac{x-1}{x^2+2x-3}
  \]
  \begin{explanation}
    First note that
    \[
    \lim_{x\to1}\left(x^2+2x-3\right)=0 \qquad\text{and}\qquad  \lim_{x\to1}\left(x^2+2x-3\right) = 0
    \]
    Hence this limit is of the form \zeroOverZero.  This tells us we
    can likely find an almost equal function if we can cancel a factor
    going to $0$ out of the numerator and denominator.  Since $(x-1)$
    is a factor going to $0$ in the numerator, let's see if we can
    factor a $(x-1)$ out of the denominator as well.
    \begin{align*}
      \lim_{x\to1}\frac{x-1}{x^2+2x-3}&=\lim_{x\to1}\frac{x-1}{(x-1)(x+3)} \\
      &=\lim_{x\to1}\frac{1}{x+3)}\\
      &=\frac{1}{4}.
    \end{align*}
  \end{explanation}
\end{example}

Now consider again
\[
\lim_{x\to 3}\frac{x^2-3x+2}{x-3} 
\]
Since
\[
\lim_{x\to3}\left(x^2-3x+2\right) = 2 \qquad\text{and}\qquad  \lim_{x\to3}\left(x-3\right) = 0
\]
This limit is not of the form \zeroOverZero. It is of the form
\numOverZero, which immediately tells us that there cannot be a factor
$(x-3)$ which can be factored out of the denominator.  In fact, it
tells us more than that.  We will look at how to solve limits with a
form \numOverZero in the next dig in.


%% \begin{example}
%%   Compute:
%%   \[
%%   \lim_{x\to1}\frac{x^2+2x-4}{x-2}
%%   \]
%% \begin{explanation}
%%   We begin by finding the form of this
%%   limit.
%%   \[
%%   \lim_{x\to1}\frac{x^2+2x-4}{x-2} \sim\frac{(1)^2+2(1)-4}{(1)-2} \sim \frac{-1}{-1}.
%%   \]
%%   This is a number, not an indeterminate form! We have discovered
%%   that this is a limit for which we could have and should have used
%%   the continuity of rational functions on their domains to compute
%%   this limit.  All we have to do is change our $\sim$ symbols to $=$
%%   symbols to complete the problem.
%%   \[
%%   \lim_{x\to1}\frac{x^2+2x-4}{x-2} = \frac{(1)^2+2(1)-4}{(1)-2} =
%%   \frac{-1}{-1} =1.
%%   \]
%% \end{explanation}
%% \end{example}

\begin{example}
  Compute:
  \[
  \lim_{x\to 1} \frac{\frac{1}{x+1}-\frac{3}{x+5}}{x-1}
  \]
\begin{explanation}
  We begin by finding the form of this limit by looking at the limits
  of the numerator and denominator separately.
  \[
  \lim_{x\to 1}\left(\frac{1}{x+1}-\frac{3}{x+5}\right)=0\qquad\text{and}\qquad\lim_{x\to 1}\left(x-1\right)=0.
  \]
This tells us that this limit is of the form \zeroOverZero\ and we can
probably factor a term going to $0$ out of both the numerator and
denominator.  We suspect from looking at the denominator that this
term is $(x-1)$.  Unfortunately, it is not immediately obvious how to
factor an $(x-1)$ out of the numerator.  In order to simplify the
numerator, we will find a common denominator for the fractions in the
numerator.
\begin{align*}
\lim_{x\to 1} \frac{\frac{1}{x+1}-\frac{3}{x+5}}{x-1}  &= \lim_{x\to 1} \frac{\frac{1}{x+1} \cdot \frac{x+5}{x+5}-\frac{3}{x+5} \cdot \frac{x+1}{x+1}}{x-1}\\
&= \lim_{x\to 1} \frac{\frac{(x+5)-3(x+1)}{(x+1)(x+5)}}{x-1}\\
&= \lim_{x\to 1}\frac{(x+5)-3(x+1)}{(x+1)(x+5)} \cdot \frac{1}{x-1}\\
&= \lim_{x\to 1}\frac{(x+5)-3(x+1)}{(x+1)(x+5)(x-1)}
\end{align*}

Now we will multiply out the numerator.  Note that we do not want to
multiply out the denominator because we already have an $(x-1)$
factored out of the denominator and that was the goal.

\begin{align*}
\lim_{x\to 1}\frac{(x+5)-3(x+1)}{(x+1)(x+5)(x-1)} &= \lim_{x\to 1}\frac{x+5-3x-3}{(x+1)(x+5)(x-1)} \\
&= \lim_{x\to 1}\frac{-2x+2}{(x+1)(x+5)(x-1)}\\
&= \lim_{x\to 1}\frac{-2(x-1)}{(x+1)(x+5)(x-1)}\\
&= \lim_{x\to 1}\frac{-2}{(x+1)(x+5)}.
\end{align*}
  
Now we finally have an almost equal function that is continuous at
$x=1$. Hence

\[
\lim_{x\to 1} \frac{\frac{1}{x+1}-\frac{3}{x+5}}{x-1}=\lim_{x\to 1}\frac{-2}{(x+1)(x+5)} = \frac{-2}{((1)+1)((1)+5)} = \frac{-1}{6}
\]
\end{explanation}
\end{example}

And one more example:

\begin{example}
  Compute:
  \[
  \lim_{x\to-1} \frac{\sqrt{x+5}-2}{x+1}
  \]

\begin{explanation} 
  Note that 
  \[
  \lim_{x\to-1} \left(\sqrt{x+5}-2\right)=0\qquad\text{and}\qquad\lim_{x\to -1} \left(x+1\right) =0.
  \]
  This tells us this limit is of the form \zeroOverZero\ and that we
  can probably factor a term going to $0$ out of both the numerator
  and denominator.  We suspect from looking at the denominator that
  this term is $(x+1)$.  Unfortunately, it is not immediately obvious
  how to factor an $(x+1)$ out of the numerator.
 
  We will use an algebraic technique called \dfn{multiplying by the
    conjugate}.  This technique is useful when you are trying to
  simplify an expression that looks like
  \[
  \sqrt{\text{something} \pm \text{something else}}.
  \]
  It takes advantage of the difference of squares rule that
  \[
  a^2-b^2=(a-b)(a+b).
  \]
  In our case, we will use $a=\sqrt{x+5}$ and $b=2$.  Then, we have
  the $(a-b)$ term.  If we introduce the $(a+b)$ term by multiplying
  both the numerator and denonimator by $(\sqrt{x+5}+2)$, we will be
  able to simpify our result using the differences of squares.
 
\begin{align*}
\lim_{x\to-1} \frac{\sqrt{x+5}-2}{x+1}&=
\lim_{x\to-1} \frac{(\sqrt{x+5}-2)}{(x+1)} \cdot \frac{(\sqrt{x+5}+2)}{(\sqrt{x+5}+2)} \\
&=\lim_{x\to-1} \frac{(\sqrt{x+5})^2-(2)^2}{(x+1)(\sqrt{x+5}+2)} \\
&=\lim_{x\to-1} \frac{x+5-4}{(x+1)(\sqrt{x+5}+2)} \\
&=\lim_{x\to-1} \frac{(x+1)}{(x+1)(\sqrt{x+5}+2)} \\
&=\lim_{x\to-1} \frac{1}{\sqrt{x+5}+2}\\
&= \frac{1}{\sqrt{-1+5}+2}\\
&=\frac{1}{4}.
\end{align*}
\end{explanation}
\end{example}

All the examples in this section use the Almost Equal Theorem to
compute limits of the form \zeroOverZero.  When you come across a
limit of the form \zeroOverZero, you should try to use various algebra
techniques to come up with an almost equal function whose limit you
can evaluate.

Notice that we solved multiple examples of limits of the form
\zeroOverZero and we got different answers each time.  This tells us
that just knowing the form of the limit is \zeroOverZero is not enough
to compute the limit.

\begin{definition}
Forms which do not give us enough information to know the answer to
the limit from just looking at the form are called \dfn{indeterminate
  forms}.

There are some forms which we will see in future sections which will
be able to tell us the answer to the limit from just looking at the
form. These forms are called \dfn{determinate forms}.
\end{definition}  

The form \zeroOverZero\ is an indeterminate form.




\end{document}
