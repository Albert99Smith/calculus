\documentclass{ximera}
\usepackage[makeroom]{cancel}
% I suspect this package is supposed to go somewhere else and not just at the top of this document

%\usepackage{todonotes}
%\usepackage{mathtools} %% Required for wide table Curl and Greens
%\usepackage{cuted} %% Required for wide table Curl and Greens
\newcommand{\todo}{}

\usepackage{esint} % for \oiint
\ifxake%%https://math.meta.stackexchange.com/questions/9973/how-do-you-render-a-closed-surface-double-integral
\renewcommand{\oiint}{{\large\bigcirc}\kern-1.56em\iint}
\fi


\graphicspath{
  {./}
  {ximeraTutorial/}
  {basicPhilosophy/}
  {functionsOfSeveralVariables/}
  {normalVectors/}
  {lagrangeMultipliers/}
  {vectorFields/}
  {greensTheorem/}
  {shapeOfThingsToCome/}
  {dotProducts/}
  {partialDerivativesAndTheGradientVector/}
  {../productAndQuotientRules/exercises/}
  {../normalVectors/exercisesParametricPlots/}
  {../continuityOfFunctionsOfSeveralVariables/exercises/}
  {../partialDerivativesAndTheGradientVector/exercises/}
  {../directionalDerivativeAndChainRule/exercises/}
  {../commonCoordinates/exercisesCylindricalCoordinates/}
  {../commonCoordinates/exercisesSphericalCoordinates/}
  {../greensTheorem/exercisesCurlAndLineIntegrals/}
  {../greensTheorem/exercisesDivergenceAndLineIntegrals/}
  {../shapeOfThingsToCome/exercisesDivergenceTheorem/}
  {../greensTheorem/}
  {../shapeOfThingsToCome/}
  {../separableDifferentialEquations/exercises/}
  {vectorFields/}
}

\newcommand{\mooculus}{\textsf{\textbf{MOOC}\textnormal{\textsf{ULUS}}}}

\usepackage{tkz-euclide}\usepackage{tikz}
\usepackage{tikz-cd}
\usetikzlibrary{arrows}
\tikzset{>=stealth,commutative diagrams/.cd,
  arrow style=tikz,diagrams={>=stealth}} %% cool arrow head
\tikzset{shorten <>/.style={ shorten >=#1, shorten <=#1 } } %% allows shorter vectors

\usetikzlibrary{backgrounds} %% for boxes around graphs
\usetikzlibrary{shapes,positioning}  %% Clouds and stars
\usetikzlibrary{matrix} %% for matrix
\usepgfplotslibrary{polar} %% for polar plots
\usepgfplotslibrary{fillbetween} %% to shade area between curves in TikZ
\usetkzobj{all}
\usepackage[makeroom]{cancel} %% for strike outs
%\usepackage{mathtools} %% for pretty underbrace % Breaks Ximera
%\usepackage{multicol}
\usepackage{pgffor} %% required for integral for loops



%% http://tex.stackexchange.com/questions/66490/drawing-a-tikz-arc-specifying-the-center
%% Draws beach ball
\tikzset{pics/carc/.style args={#1:#2:#3}{code={\draw[pic actions] (#1:#3) arc(#1:#2:#3);}}}



\usepackage{array}
\setlength{\extrarowheight}{+.1cm}
\newdimen\digitwidth
\settowidth\digitwidth{9}
\def\divrule#1#2{
\noalign{\moveright#1\digitwidth
\vbox{\hrule width#2\digitwidth}}}





\newcommand{\RR}{\mathbb R}
\newcommand{\R}{\mathbb R}
\newcommand{\N}{\mathbb N}
\newcommand{\Z}{\mathbb Z}

\newcommand{\sagemath}{\textsf{SageMath}}


%\renewcommand{\d}{\,d\!}
\renewcommand{\d}{\mathop{}\!d}
\newcommand{\dd}[2][]{\frac{\d #1}{\d #2}}
\newcommand{\pp}[2][]{\frac{\partial #1}{\partial #2}}
\renewcommand{\l}{\ell}
\newcommand{\ddx}{\frac{d}{\d x}}

\newcommand{\zeroOverZero}{\ensuremath{\boldsymbol{\tfrac{0}{0}}}}
\newcommand{\inftyOverInfty}{\ensuremath{\boldsymbol{\tfrac{\infty}{\infty}}}}
\newcommand{\zeroOverInfty}{\ensuremath{\boldsymbol{\tfrac{0}{\infty}}}}
\newcommand{\zeroTimesInfty}{\ensuremath{\small\boldsymbol{0\cdot \infty}}}
\newcommand{\inftyMinusInfty}{\ensuremath{\small\boldsymbol{\infty - \infty}}}
\newcommand{\oneToInfty}{\ensuremath{\boldsymbol{1^\infty}}}
\newcommand{\zeroToZero}{\ensuremath{\boldsymbol{0^0}}}
\newcommand{\inftyToZero}{\ensuremath{\boldsymbol{\infty^0}}}



\newcommand{\numOverZero}{\ensuremath{\boldsymbol{\tfrac{\#}{0}}}}
\newcommand{\dfn}{\textbf}
%\newcommand{\unit}{\,\mathrm}
\newcommand{\unit}{\mathop{}\!\mathrm}
\newcommand{\eval}[1]{\bigg[ #1 \bigg]}
\newcommand{\seq}[1]{\left( #1 \right)}
\renewcommand{\epsilon}{\varepsilon}
\renewcommand{\phi}{\varphi}


\renewcommand{\iff}{\Leftrightarrow}

\DeclareMathOperator{\arccot}{arccot}
\DeclareMathOperator{\arcsec}{arcsec}
\DeclareMathOperator{\arccsc}{arccsc}
\DeclareMathOperator{\si}{Si}
\DeclareMathOperator{\scal}{scal}
\DeclareMathOperator{\sign}{sign}


%% \newcommand{\tightoverset}[2]{% for arrow vec
%%   \mathop{#2}\limits^{\vbox to -.5ex{\kern-0.75ex\hbox{$#1$}\vss}}}
\newcommand{\arrowvec}[1]{{\overset{\rightharpoonup}{#1}}}
%\renewcommand{\vec}[1]{\arrowvec{\mathbf{#1}}}
\renewcommand{\vec}[1]{{\overset{\boldsymbol{\rightharpoonup}}{\mathbf{#1}}}\hspace{0in}}

\newcommand{\point}[1]{\left(#1\right)} %this allows \vector{ to be changed to \vector{ with a quick find and replace
\newcommand{\pt}[1]{\mathbf{#1}} %this allows \vec{ to be changed to \vec{ with a quick find and replace
\newcommand{\Lim}[2]{\lim_{\point{#1} \to \point{#2}}} %Bart, I changed this to point since I want to use it.  It runs through both of the exercise and exerciseE files in limits section, which is why it was in each document to start with.

\DeclareMathOperator{\proj}{\mathbf{proj}}
\newcommand{\veci}{{\boldsymbol{\hat{\imath}}}}
\newcommand{\vecj}{{\boldsymbol{\hat{\jmath}}}}
\newcommand{\veck}{{\boldsymbol{\hat{k}}}}
\newcommand{\vecl}{\vec{\boldsymbol{\l}}}
\newcommand{\uvec}[1]{\mathbf{\hat{#1}}}
\newcommand{\utan}{\mathbf{\hat{t}}}
\newcommand{\unormal}{\mathbf{\hat{n}}}
\newcommand{\ubinormal}{\mathbf{\hat{b}}}

\newcommand{\dotp}{\bullet}
\newcommand{\cross}{\boldsymbol\times}
\newcommand{\grad}{\boldsymbol\nabla}
\newcommand{\divergence}{\grad\dotp}
\newcommand{\curl}{\grad\cross}
%\DeclareMathOperator{\divergence}{divergence}
%\DeclareMathOperator{\curl}[1]{\grad\cross #1}
\newcommand{\lto}{\mathop{\longrightarrow\,}\limits}

\renewcommand{\bar}{\overline}

\colorlet{textColor}{black}
\colorlet{background}{white}
\colorlet{penColor}{blue!50!black} % Color of a curve in a plot
\colorlet{penColor2}{red!50!black}% Color of a curve in a plot
\colorlet{penColor3}{red!50!blue} % Color of a curve in a plot
\colorlet{penColor4}{green!50!black} % Color of a curve in a plot
\colorlet{penColor5}{orange!80!black} % Color of a curve in a plot
\colorlet{penColor6}{yellow!70!black} % Color of a curve in a plot
\colorlet{fill1}{penColor!20} % Color of fill in a plot
\colorlet{fill2}{penColor2!20} % Color of fill in a plot
\colorlet{fillp}{fill1} % Color of positive area
\colorlet{filln}{penColor2!20} % Color of negative area
\colorlet{fill3}{penColor3!20} % Fill
\colorlet{fill4}{penColor4!20} % Fill
\colorlet{fill5}{penColor5!20} % Fill
\colorlet{gridColor}{gray!50} % Color of grid in a plot

\newcommand{\surfaceColor}{violet}
\newcommand{\surfaceColorTwo}{redyellow}
\newcommand{\sliceColor}{greenyellow}




\pgfmathdeclarefunction{gauss}{2}{% gives gaussian
  \pgfmathparse{1/(#2*sqrt(2*pi))*exp(-((x-#1)^2)/(2*#2^2))}%
}


%%%%%%%%%%%%%
%% Vectors
%%%%%%%%%%%%%

%% Simple horiz vectors
\renewcommand{\vector}[1]{\left\langle #1\right\rangle}


%% %% Complex Horiz Vectors with angle brackets
%% \makeatletter
%% \renewcommand{\vector}[2][ , ]{\left\langle%
%%   \def\nextitem{\def\nextitem{#1}}%
%%   \@for \el:=#2\do{\nextitem\el}\right\rangle%
%% }
%% \makeatother

%% %% Vertical Vectors
%% \def\vector#1{\begin{bmatrix}\vecListA#1,,\end{bmatrix}}
%% \def\vecListA#1,{\if,#1,\else #1\cr \expandafter \vecListA \fi}

%%%%%%%%%%%%%
%% End of vectors
%%%%%%%%%%%%%

%\newcommand{\fullwidth}{}
%\newcommand{\normalwidth}{}



%% makes a snazzy t-chart for evaluating functions
%\newenvironment{tchart}{\rowcolors{2}{}{background!90!textColor}\array}{\endarray}

%%This is to help with formatting on future title pages.
\newenvironment{sectionOutcomes}{}{}



%% Flowchart stuff
%\tikzstyle{startstop} = [rectangle, rounded corners, minimum width=3cm, minimum height=1cm,text centered, draw=black]
%\tikzstyle{question} = [rectangle, minimum width=3cm, minimum height=1cm, text centered, draw=black]
%\tikzstyle{decision} = [trapezium, trapezium left angle=70, trapezium right angle=110, minimum width=3cm, minimum height=1cm, text centered, draw=black]
%\tikzstyle{question} = [rectangle, rounded corners, minimum width=3cm, minimum height=1cm,text centered, draw=black]
%\tikzstyle{process} = [rectangle, minimum width=3cm, minimum height=1cm, text centered, draw=black]
%\tikzstyle{decision} = [trapezium, trapezium left angle=70, trapezium right angle=110, minimum width=3cm, minimum height=1cm, text centered, draw=black]


\title{Limits of the form nonzero over zero}

\begin{document}
\begin{abstract}
  We want to solve limits that have the form nonzero over zero.
\end{abstract}

\maketitle

%This chapter will consist of three dig in's 1. The Almost Equal Theorem 2. Limits of the form non-zero over zero 3. Determinate and indeterminate forms 

In this section, we will consider limits of the form $\frac{non-zero}{zero}$ which we will write from now on as $\frac{\#}{0}$.  What can we say about limits which have this form?  Let's start by investigating a simple example.  Let's consider $f(x)=\frac{1}{x}$ and $\lim_{x\to 0^+} \frac{1}{x}  \sim \frac{\#}{0}$.  In order to understand what is happening, let's plug in some numbers that get closer and closer to 0 from the right side:

$f(1) = \frac{1}{1}=1$\\ \vspace{.1in}
$f\left(\frac{1}{10}\right) = \frac{1}{\frac{1}{10}}=10$\\ \vspace{.1in}
$f\left(\frac{1}{100}\right) = \frac{1}{\frac{1}{100}}=100$\\ \vspace{.1in}
$f\left(\frac{1}{1000}\right) = \frac{1}{\frac{1}{1000}}=1000$\\ 

We can see that the closer to zero the number we put in for $x$, the larger the $f(x)$ value is.  In fact, you can make the $f(x)$- value as large as you want if you put in an $x$-value close enough to zero.  We can also see this as we look at the graph of the function.

%\begin{figure}
\begin{tikzpicture}
	\begin{axis}[
            domain=-1:1,
            ymax=100,
            samples=100,
            axis lines =middle, xlabel=$x$, ylabel=$y$,
            every axis y label/.style={at=(current axis.above origin),anchor=south},
            every axis x label/.style={at=(current axis.right of origin),anchor=west}
          ]
	  \addplot [very thick, penColor, smooth, domain=(-1:-.1)] {1/x};
          \addplot [very thick, penColor, smooth, domain=(.01:1)] {1/x};
          \addplot [textColor, dashed] plot coordinates {(0,0) (0,100)};
        \end{axis}
\end{tikzpicture}
%\caption{A plot of $f(x)=\protect\frac{1}{x^2}$.}
%\label{plot:1/x^2}
%\end{figure}

\begin{definition}\label{def:inflimit}\index{limit!infinite}\index{infinite limit}
If $f(x)$ grows arbitrarily large as $x$ approaches $a$, we write
\[
\lim_{x\to a} f(x) = \infty
\]
and say that the limit of $f(x)$ \textbf{approaches infinity} as $x$
goes to $a$.

If $|f(x)|$ grows arbitrarily large as $x$ approaches $a$ and $f(x)$ is
negative, we write
\[
\lim_{x\to a} f(x) = -\infty
\]
and say that the limit of $f(x)$ \textbf{approaches negative infinity}
as $x$ goes to $a$.
\end{definition}

We might want to ask, what happens to $\lim_{x\to 0^-} \frac{1}{x}  \sim \frac{\#}{0}$?  Considering points which are less than zero and approaching zero, we see:

$f(-1) = \frac{1}{-1}=-1$\\ \vspace{.1in}
$f\left(-\frac{1}{10}\right) = \frac{1}{-\frac{1}{10}}=-10$\\ \vspace{.1in}
$f\left(-\frac{1}{100}\right) = \frac{1}{-\frac{1}{100}}=-100$\\ \vspace{.1in}
$f\left(-\frac{1}{1000}\right) = \frac{1}{-\frac{1}{1000}}=-1000$\\ 

It appears that $\lim_{x\to 0^-} \frac{1}{x} = -\infty$.  

What if we consider the general limit as $x$ approaches zero, $\lim_{x\to 0} \frac{1}{x}$?  Then, we have that the limit from the left side does not equal the limit from the right side.  Therefore, the limit does not exist and we write $\lim_{x\to 0} \frac{1}{x} = DNE$.

\begin{theorem}[Limits of the Form Non-Zero Over Zero]
Let $\lim_{x\to a} f(x) \sim \frac{\#}{0}$.  Then, the $\lim_{x\to a^+} f(x) = \pm \infty$ and $\lim_{x\to a^-} f(x) = \pm \infty$, depending on the sign of $f(x)$.  If the limit from the left and the limit from the right agree, then $\lim_{x\to a} f(x)$ is said to have the same limit.  If the limit from the left and the limit from the right do not agree, then $\lim_{x\to a} f(x) = DNE$.  Special note: Since $\infty$ is not a number, none of these limits exist.  Writing that a limit equals $\infty$ or $-\infty$ is a more specific answer than just saying the limit generally does not exist.
\end{theorem}

\begin{example}
Find $\lim_{x\to -1} \frac{1}{(x+1)^2} = \answer[given]{\infty}$
\begin{explanation}
First, we look at the form of this limit.  We see that $\lim_{x\to -1} \frac{1}{(x+1)^2} \sim \frac{\#}{0}$.  This tells us that we need to consider the left and right hand limits separately.  For each of the left and right hand limits, we need to determine whether the limit will equal $+\infty$ or $-\infty$.  We will do this by looking at the sign of the nearby numbers.  When we write $a^+$, we will mean that we are thinking about numbers just a little bit bigger than $a$.  When we write, $a^-$, we will mean that we are thinking about numbers just a little bit smaller than $a$.\\
\begin{equation}
\lim_{x\to -1^+} \frac{1}{(x+1)^2} = \frac{1}{((-1)^+ +1)^2}
 = \frac{1}{(0^+)^2}
 = \frac{1}{(0^+)}
 =\frac{+}{+} = +
\end{equation}

Therefore, $\lim_{x\to -1^+} \frac{1}{(x+1)^2} = \infty$.

Note that in the computation above, we are only trying to determine the sign of the solution.  Otherwise, this reasoning would not make sense.  Let's look at the left sided limit now.
\begin{equation}
\lim_{x\to -1^-} \frac{1}{(x+1)^2} = \frac{1}{((-1)^- +1)^2}
 = \frac{1}{(0^-)^2}
 = \frac{1}{(0^+)}
 =\frac{+}{+} = +
\end{equation}

Therefore, $\lim_{x\to -1^-} \frac{1}{(x+1)^2} = \infty$.

Since both sides agree, we can conclude that $\lim_{x\to -1} \frac{1}{(x+1)^2} = \infty$.  Looking at the graph of this function, we see that this answer makes sense.\\
\begin{tikzpicture}
	\begin{axis}[
            domain=-2:1,
            ymax=100,
            samples=100,
            axis lines =middle, xlabel=$x$, ylabel=$y$,
            every axis y label/.style={at=(current axis.above origin),anchor=south},
            every axis x label/.style={at=(current axis.right of origin),anchor=west}
          ]
	  \addplot [very thick, penColor, smooth, domain=(-2:-1.1)] {1/(x+1)^2};
          \addplot [very thick, penColor, smooth, domain=(-.9:1)] {1/(x+1)^2};
          \addplot [textColor, dashed] plot coordinates {(-1,0) (-1,100)};
        \end{axis}
\end{tikzpicture}

\end{explanation}
\end{example}



\begin{example}
Find $\lim_{x\to 2} \frac{x^2-9x+14}{x^2-5x+6} = \answer[given]{5}$ and  $\lim_{x\to 3} \frac{x^2-9x+14}{x^2-5x+6}= \answer[given]{DNE} $

\begin{explanation}
Start by factoring both the numerator and the denominator:
\[
\frac{x^2-9x+14}{x^2-5x+6} = \frac{(x-2)(x-7)}{(x-2)(x-3)}
\]
This will allow us to quickly see the forms of the limits.  Let's first look at $\lim_{x\to 2} \frac{(x-2)(x-7)}{(x-2)(x-3)} \sim \frac{0}{0}$.  Recall that $\frac{0}{0}$ is an indeterminate form, and therefore we cannot tell the answer from just looking at the form.  We can easily see from our factored expression how to come up with an almost equal function which was can use to find this limit.  We just cancel the $(x-2)$ term from the numerator and denominator of this fraction.
\begin{align*}
\lim_{x\to 2} \frac{(x-2)(x-7)}{(x-2)(x-3)} &= \lim_{x\to 2} \frac{(x-7)}{(x-3)}\\
&= \frac{-5}{-1}\\
&=5.
\end{align*}

Now consider $\lim_{x\to 3} \frac{(x-2)(x-7)}{(x-2)(x-3)} \sim \frac{-4}{0} \sim \frac{\#}{0}$.   We can still use the almost equal function $\frac{(x-7)}{(x-3)}$ to find this limit if we wish, since we know these two functions are equal on all points near 3.  Since this limit has the form $\frac{\#}{0}$, we need to look at both the right and left limits and consider the sign of the solution to determine whether each limit is $+\infty$ or $-\infty$.

\begin{align*}
\lim_{x\to 3^+} \frac{(x-2)(x-7)}{(x-2)(x-3)} &= \lim_{x\to 3^+} \frac{(x-7)}{(x-3)}\\
&= \lim_{x\to 3^+}\frac{-4}{3^+ -3}\\
&= \lim_{x\to 3^+}\frac{-4}{0^+}\\
&= \lim_{x\to 3^+}\frac{-}{+} = -\\.
\end{align*}

Therefore, $\lim_{x\to 3^+} \frac{(x-2)(x-7)}{(x-2)(x-3)} = \infty$.

\begin{align*}
\lim_{x\to 3^-} \frac{(x-2)(x-7)}{(x-2)(x-3)} &= \lim_{x\to 3^-} \frac{(x-7)}{(x-3)}\\
&= \lim_{x\to 3^-}\frac{-4}{3^- -3}\\
&= \lim_{x\to 3^-}\frac{-4}{0^-}\\
&= \lim_{x\to 3^+}\frac{-}{-} = +\\.
\end{align*}

Therefore, $\lim_{x\to 3^-} \frac{(x-2)(x-7)}{(x-2)(x-3)} = -\infty$.

Since the limits from both sides do not agree, we say $\lim_{x\to 3} \frac{(x-2)(x-7)}{(x-2)(x-3)} = DNE$.

Looking at the graph of this function, we can see that our answers makes sense.

\begin{tikzpicture}
	\begin{axis}[
            domain=1:4,
            ymax=50,
            ymin=-50,
            samples=100,
            axis lines =middle, xlabel=$x$, ylabel=$y$,
            every axis y label/.style={at=(current axis.above origin),anchor=south},
            every axis x label/.style={at=(current axis.right of origin),anchor=west}
          ]
	  \addplot [very thick, penColor, smooth, domain=(3.02:4)] {(x-7)/(x-3)};
          \addplot [very thick, penColor, smooth, domain=(1:2.98)] {(x-7)/(x-3)};
          \addplot [textColor, dashed] plot coordinates {(3,-50) (3,50)};
          \addplot[color=penColor,fill=background,only marks,mark=*] coordinates{(2,5)};  %% open hole
        \end{axis}
\end{tikzpicture}

\end{explanation}
\end{example}


\end{document}

%\begin{exercises}

\noindent Compute the limits. If a limit does not exist, explain why.
%\twocol
\begin{exercise}
$\lim_{x\to 1-} \frac{1}{x^2-1}$
\begin{answer}
  $-\infty$
\end{answer}
\end{exercise}

\begin{exercise}
$\lim_{x\to 4-} \frac{3}{x^2-2}$
\begin{answer}
  $3/14$
\end{answer}
\end{exercise}

\begin{exercise}
$\lim_{x\to -1+} \frac{1+2x}{x^3-1}$
\begin{answer}
  $1/2$
\end{answer}
\end{exercise}

\begin{exercise}
$\lim_{x\to 3+} \frac{x-9}{x^2-6x+9}$
\begin{answer}
  $-\infty$
\end{answer}
\end{exercise}

\begin{exercise}
$\lim_{x\to 5} \frac{1}{(x-5)^4}$
\begin{answer}
  $\infty$
\end{answer}
\end{exercise}


\begin{exercise}
$\lim_{x\to -2} \frac{1}{(x^2+3x+2)^2}$
\begin{answer}
  $\infty$
\end{answer}
\end{exercise}


\begin{exercise}
$\lim_{x\to 0} \frac{1}{\frac{x}{x^5}-\cos(x)}$
\begin{answer}
  0
\end{answer}
\end{exercise}

\begin{exercise}
$\lim_{x\to 0+} \frac{x-11}{\sin(x)}$
\begin{answer}
  $-\infty$
\end{answer}
\end{exercise}


\endtwocol


\begin{exercise}
Find the vertical asymptotes of
\[
f(x) = \frac{x-3}{x^2+2x-3}.
\]
\begin{answer}
  $x = 1$ and $x = -3$
\end{answer}
\end{exercise}


\begin{exercise}
Find the vertical asymptotes of
\[
f(x) = \frac{x^2-x-6}{x+4}.
\]
\begin{answer}
  $x = -4$
\end{answer}
\end{exercise}
\end{exercises}






\section{Limits at Infinity}


Consider the function:
\[
f(x) = \frac{6x-9}{x-1}
\]
\begin{figure}[0in]
\begin{tikzpicture}
	\begin{axis}[
            domain=1:4,
            ymax=20,
            ymin=-10,
            samples=100,
            axis lines =middle, xlabel=$x$, ylabel=$y$,
            every axis y label/.style={at=(current axis.above origin),anchor=south},
            every axis x label/.style={at=(current axis.right of origin),anchor=west}
          ]
	  \addplot [very thick, penColor, smooth, domain=(0:.9)] {(6*x-9)/(x-1)};
          \addplot [very thick, penColor, smooth, domain=(1.1:3)] {(6*x-9)/(x-1)};
          \addplot [textColor, dashed] plot coordinates {(1,-10) (1,20)};
        \end{axis}
\end{tikzpicture}
\caption{A plot of $f(x)=\protect\frac{6x-9}{x-1}$.}
\label{plot:(6x-9)/(x-1)}
\end{figure}
As $x$ approaches infinity, it seems like $f(x)$ approaches a specific
value. This is a \textit{limit at infinity}.

\begin{definition}\label{def:limitAtInfty}\index{limit!at infinity}
If $f(x)$ becomes arbitrarily close to a specific value $L$ by making
$x$ sufficiently large, we write
\[
\lim_{x\to \infty} f(x) = L
\]
and we say, the \textbf{limit at infinity} of $f(x)$ is $L$.  

If $f(x)$ becomes
arbitrarily close to a specific value $L$ by making $x$ sufficiently
large and negative, we write
\[
\lim_{x\to -\infty} f(x) = L
\]
and we say, the \textbf{limit at negative infinity} of $f(x)$ is $L$.  
\end{definition}

\begin{example} Compute
\[
\lim_{x\to\infty} \frac{6x-9}{x-1}.
\]
\end{example}


\begin{solution}
Write
\begin{align*}
\lim_{x\to\infty}\frac{6x-9}{x-1} &= \lim_{x\to\infty}\frac{6x-9}{x-1} \frac{1/x}{1/x}\\
&=\lim_{x\to\infty}\frac{\frac{6x}{x} - \frac{9}{x}}{\frac{x}{x} - \frac{1}{x}}\\
&= \lim_{x\to\infty} \frac{6}{1}\\
&= 6.
\end{align*}
\end{solution}

Sometimes one must be careful, consider this example.

\begin{example}
Compute
\[
\lim_{x\to -\infty} \frac{x+1}{\sqrt{x^2}}
\]
\end{example}

\begin{solution}
In this case we multiply the numerator and denominator by $-1/x$,
which is a positive number as since $x\to -\infty$, $x$ is a negative
number.
\begin{align*}
\lim_{x\to -\infty} \frac{x+1}{\sqrt{x^2}} &= \lim_{x\to -\infty} \frac{x+1}{\sqrt{x^2}} \cdot \frac{-1/x}{-1/x}\\
&= \lim_{x\to -\infty} \frac{-1-1/x}{\sqrt{x^2/x^2}}\\
&= -1.
\end{align*}
\end{solution}


Here is a somewhat different example of a limit at infinity.

\begin{example}
Compute
\[
\lim_{x\to \infty} \frac{\sin(7x)}{x}+4.
\]
\end{example}

\begin{figure}[0in]
\begin{tikzpicture}
	\begin{axis}[
            domain=2:20,
            ymax=5,
            ymin=3,
            samples=100,
            axis lines =middle, xlabel=$x$, ylabel=$y$,
            every axis y label/.style={at=(current axis.above origin),anchor=south},
            every axis x label/.style={at=(current axis.right of origin),anchor=west}
          ]
	  \addplot [very thick, penColor, smooth] {(1/x) * sin(deg(7*x))+4};
        \end{axis}
\end{tikzpicture}
\caption{A plot of $f(x)=\frac{\sin(7x)}{x}+4$.}
\label{plot:sin7x/x+4}
\end{figure}

\begin{solution}
We can bound our function
\[
-1/x + 4 \le \frac{\sin(7x)}{x}+4 \le 1/x + 4.
\]
Since 
\[
\lim_{x\to \infty} -1/x + 4 = 4 = \lim_{x\to \infty}1/x + 4
\] 
we conclude by the Squeeze Theorem, Theorem~\ref{theorem:squeeze},
$\lim_{x\to\infty}\frac{\sin(7x)}{x}+4 = 4$.
\end{solution}






\begin{definition}\label{def:horiz asymptote}\index{asymptote!horizontal}\index{horizontal asymptote}
If  
\[
\lim_{x\to \infty} f(x) = L \qquad\text{or}\qquad \lim_{x\to -\infty} f(x) = L,
\]
then the line $y=L$ is a \textbf{horizontal asymptote} of $f(x)$.
\end{definition}

\begin{example} 
Give the horizontal asymptotes of
\[
f(x) = \frac{6x-9}{x-1}
\]
\end{example}

\begin{solution}
From our previous work, we see that $\lim_{x\to \infty} f(x) = 6$, and
upon further inspection, we see that $\lim_{x\to -\infty} f(x) =
6$. Hence the horizontal asymptote of $f(x)$ is the line $y=6$.
\end{solution}


It is a common misconception that a function cannot cross an
asymptote. As the next example shows, a function can cross an
asymptote, and in this case this occurs an infinite number of times!

\begin{example}
Give a horizontal asymptote of
\[
f(x) = \frac{\sin(7x)}{x}+4.
\]
\end{example}

\begin{solution}
Again from previous work, we see that $\lim_{x\to \infty} f(x) =
4$. Hence $y=4$ is a horizontal asymptote of $f(x)$.
\end{solution}


We conclude with an infinite limit at infinity.

\begin{example}
Compute
\[
\lim_{x\to \infty} \ln(x)
\]
\end{example}
\begin{figure}[0in]
\begin{tikzpicture}
	\begin{axis}[
            domain=0:20,
            ymax=4,
            ymin=-1,
            samples=100,
            axis lines =middle, xlabel=$x$, ylabel=$y$,
            every axis y label/.style={at=(current axis.above origin),anchor=south},
            every axis x label/.style={at=(current axis.right of origin),anchor=west}
          ]
	  \addplot [very thick, penColor, smooth] {ln(x)};
        \end{axis}
\end{tikzpicture}
\caption{A plot of $f(x)=\ln(x)$.}
\label{plot:lnx}
\end{figure}

\begin{solution}
The function $\ln(x)$ grows very slowly, and seems like it may have a
horizontal asymptote, see Figure~\ref{plot:lnx}. However, if we
consider the definition of the natural log
\[
\ln(x) = y \qquad \Leftrightarrow\qquad e^y =x
\]
Since we need to raise $e$ to higher and higher values to obtain
larger numbers, we see that $\ln(x)$ is unbounded, and hence
$\lim_{x\to\infty}\ln(x)=\infty$.
\end{solution}


\begin{exercises}

\noindent Compute the limits.
\twocol
\begin{exercise}
$\lim_{x\to \infty} \frac{1}{x}$
\begin{answer}
$0$
\end{answer}
\end{exercise}

\begin{exercise}
$\lim_{x\to \infty} \frac{-x}{\sqrt{4+x^2}}$
\begin{answer}
$-1$
\end{answer}
\end{exercise}

\begin{exercise}
$\lim_{x\to \infty} \frac{2x^2-x+1}{4x^2-3x-1}$
\begin{answer}
$\frac{1}{2}$
\end{answer}
\end{exercise}

\begin{exercise}
$\lim_{x\to -\infty} \frac{3x+7}{\sqrt{x^2}}$
\begin{answer}
$-3$
\end{answer}
\end{exercise}

\begin{exercise}
$\lim_{x\to -\infty} \frac{2x+7}{\sqrt{x^2+2x-1}}$
\begin{answer}
$-2$
\end{answer}
\end{exercise}

\begin{exercise}
$\lim_{x\to -\infty} \frac{x^3-4}{3x^2+4x-1}$
\begin{answer}
$-\infty$
\end{answer}
\end{exercise}


\begin{exercise}
$\lim_{x\to \infty} \left(\frac{4}{x}+\pi\right)$
\begin{answer}
$\pi$
\end{answer}
\end{exercise}

\begin{exercise}
$\lim_{x\to \infty} \frac{\cos(x)}{\ln(x)}$
\begin{answer}
$0$
\end{answer}
\end{exercise}

\begin{exercise}
$\lim_{x\to \infty} \frac{\sin\left(x^7\right)}{\sqrt{x}}$
\begin{answer}
$0$
\end{answer}
\end{exercise}

\begin{exercise}
$\lim_{x\to \infty} \left(17 + \frac{32}{x} - \frac{\left(\sin(x/2)\right)^2}{x^3}\right)$
\begin{answer}
$17$
\end{answer}
\end{exercise}

\endtwocol

\begin{exercise}
Suppose a population of feral cats on a certain college campus $t$
years from now is approximated by
\[
p(t) = \frac{1000}{5+ 2e^{-0.1 t}}.
\]
Approximately how many feral cats are on campus 10 years from now? 50
years from now? 100 years from now? 1000 years from now? What do you
notice about the prediction---is this realistic?
\begin{answer}
After 10 years, $\approx 174$ cats; after 50 years, $\approx 199$
cats; after 100 years, $\approx 200$ cats; after 1000 years, $\approx
200$ cats; in the sense that the population of cats cannot grow
indefinitely this is somewhat realistic.
\end{answer}
\end{exercise}

\begin{exercise}
The amplitude of an oscillating spring is given by
\[
a(t) = \frac{\sin(t)}{t}.
\]
What happens to the amplitude of the oscillation over a long period of
time?
\begin{answer}
The amplitude goes to zero. 
\end{answer}
\end{exercise}
\end{exercises}


\end{document}
