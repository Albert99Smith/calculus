\documentclass{ximera}

%\usepackage{todonotes}
%\usepackage{mathtools} %% Required for wide table Curl and Greens
%\usepackage{cuted} %% Required for wide table Curl and Greens
\newcommand{\todo}{}

\usepackage{esint} % for \oiint
\ifxake%%https://math.meta.stackexchange.com/questions/9973/how-do-you-render-a-closed-surface-double-integral
\renewcommand{\oiint}{{\large\bigcirc}\kern-1.56em\iint}
\fi


\graphicspath{
  {./}
  {ximeraTutorial/}
  {basicPhilosophy/}
  {functionsOfSeveralVariables/}
  {normalVectors/}
  {lagrangeMultipliers/}
  {vectorFields/}
  {greensTheorem/}
  {shapeOfThingsToCome/}
  {dotProducts/}
  {partialDerivativesAndTheGradientVector/}
  {../productAndQuotientRules/exercises/}
  {../normalVectors/exercisesParametricPlots/}
  {../continuityOfFunctionsOfSeveralVariables/exercises/}
  {../partialDerivativesAndTheGradientVector/exercises/}
  {../directionalDerivativeAndChainRule/exercises/}
  {../commonCoordinates/exercisesCylindricalCoordinates/}
  {../commonCoordinates/exercisesSphericalCoordinates/}
  {../greensTheorem/exercisesCurlAndLineIntegrals/}
  {../greensTheorem/exercisesDivergenceAndLineIntegrals/}
  {../shapeOfThingsToCome/exercisesDivergenceTheorem/}
  {../greensTheorem/}
  {../shapeOfThingsToCome/}
  {../separableDifferentialEquations/exercises/}
  {vectorFields/}
}

\newcommand{\mooculus}{\textsf{\textbf{MOOC}\textnormal{\textsf{ULUS}}}}

\usepackage{tkz-euclide}\usepackage{tikz}
\usepackage{tikz-cd}
\usetikzlibrary{arrows}
\tikzset{>=stealth,commutative diagrams/.cd,
  arrow style=tikz,diagrams={>=stealth}} %% cool arrow head
\tikzset{shorten <>/.style={ shorten >=#1, shorten <=#1 } } %% allows shorter vectors

\usetikzlibrary{backgrounds} %% for boxes around graphs
\usetikzlibrary{shapes,positioning}  %% Clouds and stars
\usetikzlibrary{matrix} %% for matrix
\usepgfplotslibrary{polar} %% for polar plots
\usepgfplotslibrary{fillbetween} %% to shade area between curves in TikZ
\usetkzobj{all}
\usepackage[makeroom]{cancel} %% for strike outs
%\usepackage{mathtools} %% for pretty underbrace % Breaks Ximera
%\usepackage{multicol}
\usepackage{pgffor} %% required for integral for loops



%% http://tex.stackexchange.com/questions/66490/drawing-a-tikz-arc-specifying-the-center
%% Draws beach ball
\tikzset{pics/carc/.style args={#1:#2:#3}{code={\draw[pic actions] (#1:#3) arc(#1:#2:#3);}}}



\usepackage{array}
\setlength{\extrarowheight}{+.1cm}
\newdimen\digitwidth
\settowidth\digitwidth{9}
\def\divrule#1#2{
\noalign{\moveright#1\digitwidth
\vbox{\hrule width#2\digitwidth}}}





\newcommand{\RR}{\mathbb R}
\newcommand{\R}{\mathbb R}
\newcommand{\N}{\mathbb N}
\newcommand{\Z}{\mathbb Z}

\newcommand{\sagemath}{\textsf{SageMath}}


%\renewcommand{\d}{\,d\!}
\renewcommand{\d}{\mathop{}\!d}
\newcommand{\dd}[2][]{\frac{\d #1}{\d #2}}
\newcommand{\pp}[2][]{\frac{\partial #1}{\partial #2}}
\renewcommand{\l}{\ell}
\newcommand{\ddx}{\frac{d}{\d x}}

\newcommand{\zeroOverZero}{\ensuremath{\boldsymbol{\tfrac{0}{0}}}}
\newcommand{\inftyOverInfty}{\ensuremath{\boldsymbol{\tfrac{\infty}{\infty}}}}
\newcommand{\zeroOverInfty}{\ensuremath{\boldsymbol{\tfrac{0}{\infty}}}}
\newcommand{\zeroTimesInfty}{\ensuremath{\small\boldsymbol{0\cdot \infty}}}
\newcommand{\inftyMinusInfty}{\ensuremath{\small\boldsymbol{\infty - \infty}}}
\newcommand{\oneToInfty}{\ensuremath{\boldsymbol{1^\infty}}}
\newcommand{\zeroToZero}{\ensuremath{\boldsymbol{0^0}}}
\newcommand{\inftyToZero}{\ensuremath{\boldsymbol{\infty^0}}}



\newcommand{\numOverZero}{\ensuremath{\boldsymbol{\tfrac{\#}{0}}}}
\newcommand{\dfn}{\textbf}
%\newcommand{\unit}{\,\mathrm}
\newcommand{\unit}{\mathop{}\!\mathrm}
\newcommand{\eval}[1]{\bigg[ #1 \bigg]}
\newcommand{\seq}[1]{\left( #1 \right)}
\renewcommand{\epsilon}{\varepsilon}
\renewcommand{\phi}{\varphi}


\renewcommand{\iff}{\Leftrightarrow}

\DeclareMathOperator{\arccot}{arccot}
\DeclareMathOperator{\arcsec}{arcsec}
\DeclareMathOperator{\arccsc}{arccsc}
\DeclareMathOperator{\si}{Si}
\DeclareMathOperator{\scal}{scal}
\DeclareMathOperator{\sign}{sign}


%% \newcommand{\tightoverset}[2]{% for arrow vec
%%   \mathop{#2}\limits^{\vbox to -.5ex{\kern-0.75ex\hbox{$#1$}\vss}}}
\newcommand{\arrowvec}[1]{{\overset{\rightharpoonup}{#1}}}
%\renewcommand{\vec}[1]{\arrowvec{\mathbf{#1}}}
\renewcommand{\vec}[1]{{\overset{\boldsymbol{\rightharpoonup}}{\mathbf{#1}}}\hspace{0in}}

\newcommand{\point}[1]{\left(#1\right)} %this allows \vector{ to be changed to \vector{ with a quick find and replace
\newcommand{\pt}[1]{\mathbf{#1}} %this allows \vec{ to be changed to \vec{ with a quick find and replace
\newcommand{\Lim}[2]{\lim_{\point{#1} \to \point{#2}}} %Bart, I changed this to point since I want to use it.  It runs through both of the exercise and exerciseE files in limits section, which is why it was in each document to start with.

\DeclareMathOperator{\proj}{\mathbf{proj}}
\newcommand{\veci}{{\boldsymbol{\hat{\imath}}}}
\newcommand{\vecj}{{\boldsymbol{\hat{\jmath}}}}
\newcommand{\veck}{{\boldsymbol{\hat{k}}}}
\newcommand{\vecl}{\vec{\boldsymbol{\l}}}
\newcommand{\uvec}[1]{\mathbf{\hat{#1}}}
\newcommand{\utan}{\mathbf{\hat{t}}}
\newcommand{\unormal}{\mathbf{\hat{n}}}
\newcommand{\ubinormal}{\mathbf{\hat{b}}}

\newcommand{\dotp}{\bullet}
\newcommand{\cross}{\boldsymbol\times}
\newcommand{\grad}{\boldsymbol\nabla}
\newcommand{\divergence}{\grad\dotp}
\newcommand{\curl}{\grad\cross}
%\DeclareMathOperator{\divergence}{divergence}
%\DeclareMathOperator{\curl}[1]{\grad\cross #1}
\newcommand{\lto}{\mathop{\longrightarrow\,}\limits}

\renewcommand{\bar}{\overline}

\colorlet{textColor}{black}
\colorlet{background}{white}
\colorlet{penColor}{blue!50!black} % Color of a curve in a plot
\colorlet{penColor2}{red!50!black}% Color of a curve in a plot
\colorlet{penColor3}{red!50!blue} % Color of a curve in a plot
\colorlet{penColor4}{green!50!black} % Color of a curve in a plot
\colorlet{penColor5}{orange!80!black} % Color of a curve in a plot
\colorlet{penColor6}{yellow!70!black} % Color of a curve in a plot
\colorlet{fill1}{penColor!20} % Color of fill in a plot
\colorlet{fill2}{penColor2!20} % Color of fill in a plot
\colorlet{fillp}{fill1} % Color of positive area
\colorlet{filln}{penColor2!20} % Color of negative area
\colorlet{fill3}{penColor3!20} % Fill
\colorlet{fill4}{penColor4!20} % Fill
\colorlet{fill5}{penColor5!20} % Fill
\colorlet{gridColor}{gray!50} % Color of grid in a plot

\newcommand{\surfaceColor}{violet}
\newcommand{\surfaceColorTwo}{redyellow}
\newcommand{\sliceColor}{greenyellow}




\pgfmathdeclarefunction{gauss}{2}{% gives gaussian
  \pgfmathparse{1/(#2*sqrt(2*pi))*exp(-((x-#1)^2)/(2*#2^2))}%
}


%%%%%%%%%%%%%
%% Vectors
%%%%%%%%%%%%%

%% Simple horiz vectors
\renewcommand{\vector}[1]{\left\langle #1\right\rangle}


%% %% Complex Horiz Vectors with angle brackets
%% \makeatletter
%% \renewcommand{\vector}[2][ , ]{\left\langle%
%%   \def\nextitem{\def\nextitem{#1}}%
%%   \@for \el:=#2\do{\nextitem\el}\right\rangle%
%% }
%% \makeatother

%% %% Vertical Vectors
%% \def\vector#1{\begin{bmatrix}\vecListA#1,,\end{bmatrix}}
%% \def\vecListA#1,{\if,#1,\else #1\cr \expandafter \vecListA \fi}

%%%%%%%%%%%%%
%% End of vectors
%%%%%%%%%%%%%

%\newcommand{\fullwidth}{}
%\newcommand{\normalwidth}{}



%% makes a snazzy t-chart for evaluating functions
%\newenvironment{tchart}{\rowcolors{2}{}{background!90!textColor}\array}{\endarray}

%%This is to help with formatting on future title pages.
\newenvironment{sectionOutcomes}{}{}



%% Flowchart stuff
%\tikzstyle{startstop} = [rectangle, rounded corners, minimum width=3cm, minimum height=1cm,text centered, draw=black]
%\tikzstyle{question} = [rectangle, minimum width=3cm, minimum height=1cm, text centered, draw=black]
%\tikzstyle{decision} = [trapezium, trapezium left angle=70, trapezium right angle=110, minimum width=3cm, minimum height=1cm, text centered, draw=black]
%\tikzstyle{question} = [rectangle, rounded corners, minimum width=3cm, minimum height=1cm,text centered, draw=black]
%\tikzstyle{process} = [rectangle, minimum width=3cm, minimum height=1cm, text centered, draw=black]
%\tikzstyle{decision} = [trapezium, trapezium left angle=70, trapezium right angle=110, minimum width=3cm, minimum height=1cm, text centered, draw=black]


\title[Dig-In:]{The precise definition of a limit}

\begin{document}
\begin{abstract}
  We give a mathematically precise definition of a limit.
\end{abstract}
\maketitle

Recall that intuitively, the \textit{limit} of $f(x)$ as $x$
approaches $a$ is $L$, written
\[
\lim_{x\to a} f(x) = L,
\]
if the value $f(x)$ can be made as close as one wishes to $L$ for
all $x$ sufficiently close, but not equal to, $a$.  This leads us to a
precise definition of a \textit{limit}.

\section{The definition of a limit}


%% \marginnote[1in]{Equivalently, $\lim_{x\to a}f(x)=L$, if for
%%   every $\epsilon>0$ there is a $\delta > 0$ so that whenever $x\ne a$
%%   and $a- \delta < x < a+ \delta$, we have $L-\epsilon<
%%   f(x)<L+\epsilon$.}

\begin{definition}\label{def:limit}\index{limit!definition}
The \dfn{limit} of $f(x)$ as $x$ goes to $a$ is $L$,
\[
\lim_{x\to a} f(x) = L,
\] 
if for every $\epsilon>0$ there is a $\delta > 0$ so that whenever
\[
0 < |x-a| < \delta, \qquad\text{we have} \qquad |f(x)-L|<\epsilon.
\] 
If no such value of $L$ can be found, then we say that the \dfn{limit
  does not exist}.
\end{definition}

In the figure below, we see a geometric interpretation of this
definition.

\begin{image}
\begin{tikzpicture}
	\begin{axis}[
            domain=0:2, 
            axis lines =left, xlabel=$x$, ylabel=$y$,
            every axis y label/.style={at=(current axis.above origin),anchor=south},
            every axis x label/.style={at=(current axis.right of origin),anchor=west},
            xtick={0.7,1,1.3}, ytick={3,4,5},
            xticklabels={$a-\delta$,$a$,$a+\delta$}, yticklabels={$L-\epsilon$,$L$,$L+\epsilon$},
            axis on top,
          ]          
          \addplot [color=textColor, fill=fill2, smooth, domain=(0:1.570)] {5} \closedcycle;
          \addplot [color=textColor, dashed, fill=fill1, smooth, domain=(0:1.3)] {4.537} \closedcycle;
          \addplot [color=textColor, dashed, fill=fill2, domain=(0:.7)] {3.283} \closedcycle;       
          \addplot [textColor, very thick, smooth, domain=(0:1)] {4};
          \addplot [color=textColor, fill=background, smooth, domain=(0:0.607)] {3} \closedcycle;
	  \addplot [draw=none, fill=background, smooth] {x*(x-2)^2+3*x} \closedcycle;
          \addplot [fill=fill1, draw=none, domain=.7:1.3] {x*(x-2)^2+3*x} \closedcycle;
          \addplot [textColor, very thick] plot coordinates {(1,0) (1,4)};
          \addplot [textColor] plot coordinates {(.7,0) (.7,3.283)};
          \addplot [textColor] plot coordinates {(1.3,0) (1.3,4.537)};
	  \addplot [very thick,penColor, smooth] {x*(x-2)^2+3*x};
        \end{axis}
\end{tikzpicture}
%% \caption{A geometric interpretation of the
%%   $(\epsilon,\delta)$-criterion for limits.  If $0<|x-a|<\delta$, then we have that $a
%%   -\delta < x < a+\delta$. In our diagram, we see that for all such
%%   $x$ we are sure to have $L - \epsilon< f(x) < L+\epsilon$, and hence
%%   $|f(x) - L|<\epsilon$.}
%% \label{figure:epsilon-delta}
\end{image}


Now we are going to get our hands dirty, and really use the definition
of a limit.


\begin{example} Show that $\lim_{x\to 2} x^2=4$. 
\begin{explanation}
We want to show that for any given $\epsilon>0$, we can find a
$\delta>0$ such that
\[
|x^2 -4|<\epsilon
\]
whenever $0<|x - 2|<\delta$.
\begin{image}
  \begin{tikzpicture}
	\begin{axis}[
            domain=1:3, 
            axis lines =left, xlabel=$x$, ylabel=$y$,
            every axis y label/.style={at=(current axis.above origin),anchor=south},
            every axis x label/.style={at=(current axis.right of origin),anchor=west},
            xtick={1.8,2,2.2}, ytick={3,4,5},
            xticklabels={$2-\delta$,$2$,$2+\delta$}, yticklabels={$4-\epsilon$,$4$,$4+\epsilon$},
            axis on top,
          ]          
          \addplot [color=textColor, fill=fill2, smooth, domain=(1:2.236)] {5} \closedcycle;
          \addplot [color=textColor, dashed, fill=fill1, domain=(1:2.2)] {4.84} \closedcycle;       
          \addplot [color=textColor, dashed, fill=fill2, domain=(1:1.8)] {3.24} \closedcycle;       
          \addplot [textColor, very thick, smooth, domain=(1:2)] {4};
          \addplot [color=textColor, fill=background, smooth, domain=(1:1.8)] {3} \closedcycle;
	  \addplot [draw=none, fill=background, smooth] {x^2} \closedcycle;
          \addplot [fill=fill1, draw=none, domain=1.8:2.2] {x^2} \closedcycle;
          \addplot [textColor, very thick] plot coordinates {(2,0) (2,4)};
          \addplot [textColor] plot coordinates {(1.8,0) (1.8,3.24)};
          \addplot [textColor] plot coordinates {(2.2,0) (2.2,4.84)};
	  \addplot [very thick,penColor, smooth] {x^2};
        \end{axis}
\end{tikzpicture}
%% label{plot:x^2 lim dfn}
%% \caption{The $(\epsilon,\delta)$-criterion for $\lim_{x\to 2}
%%   x^2=4$. Here $\delta= \min\left(\dfrac{\epsilon}{5},1\right)$.}
\end{image}
Start by factoring the left-hand side of
the inequality above
\[
|x+2||x-2|<\epsilon.
\]
Since we are going to assume that $0<|x - 2|<\delta$, we will focus on
the factor $|x+2|$. Since $x$ is assumed to be close to $2$, suppose that $x\in[1,3]$. In this case
\[
|x+2| \le 3+2 = 5,
\]
and so we want
\begin{align*}
5\cdot |x-2| &< \epsilon\\
|x-2| &< \frac{\epsilon}{5}
\end{align*}
Recall, we assumed that $x\in[1,3]$, which is equivalent to
$|x-2|\le 1$. Hence we must set $\delta = \min\left(\dfrac{\epsilon}{5},1\right)$.
\end{explanation}
\end{example}

When dealing with limits of polynomials, the general strategy is
always the same. Let $p(x)$ be a polynomial. If showing
\[
\lim_{x\to a} p(x) = L,
\]
one must first factor out $|x-a|$ from $|p(x) - L|$. Next bound $x\in
[a-1,a+1]$ and estimate the largest possible value of
\[
\left|\frac{p(x) -L}{x-a}\right|
\]
for $x\in[a-1,a+1]$, call this estimation $M$. Finally, one must set
$\delta = \min\left(\frac{\epsilon}{M}, 1\right)$.

\section{Tolerance problems}



\end{document}
