\documentclass{ximera}

%\usepackage{todonotes}
%\usepackage{mathtools} %% Required for wide table Curl and Greens
%\usepackage{cuted} %% Required for wide table Curl and Greens
\newcommand{\todo}{}

\usepackage{esint} % for \oiint
\ifxake%%https://math.meta.stackexchange.com/questions/9973/how-do-you-render-a-closed-surface-double-integral
\renewcommand{\oiint}{{\large\bigcirc}\kern-1.56em\iint}
\fi


\graphicspath{
  {./}
  {ximeraTutorial/}
  {basicPhilosophy/}
  {functionsOfSeveralVariables/}
  {normalVectors/}
  {lagrangeMultipliers/}
  {vectorFields/}
  {greensTheorem/}
  {shapeOfThingsToCome/}
  {dotProducts/}
  {partialDerivativesAndTheGradientVector/}
  {../productAndQuotientRules/exercises/}
  {../normalVectors/exercisesParametricPlots/}
  {../continuityOfFunctionsOfSeveralVariables/exercises/}
  {../partialDerivativesAndTheGradientVector/exercises/}
  {../directionalDerivativeAndChainRule/exercises/}
  {../commonCoordinates/exercisesCylindricalCoordinates/}
  {../commonCoordinates/exercisesSphericalCoordinates/}
  {../greensTheorem/exercisesCurlAndLineIntegrals/}
  {../greensTheorem/exercisesDivergenceAndLineIntegrals/}
  {../shapeOfThingsToCome/exercisesDivergenceTheorem/}
  {../greensTheorem/}
  {../shapeOfThingsToCome/}
  {../separableDifferentialEquations/exercises/}
  {vectorFields/}
}

\newcommand{\mooculus}{\textsf{\textbf{MOOC}\textnormal{\textsf{ULUS}}}}

\usepackage{tkz-euclide}\usepackage{tikz}
\usepackage{tikz-cd}
\usetikzlibrary{arrows}
\tikzset{>=stealth,commutative diagrams/.cd,
  arrow style=tikz,diagrams={>=stealth}} %% cool arrow head
\tikzset{shorten <>/.style={ shorten >=#1, shorten <=#1 } } %% allows shorter vectors

\usetikzlibrary{backgrounds} %% for boxes around graphs
\usetikzlibrary{shapes,positioning}  %% Clouds and stars
\usetikzlibrary{matrix} %% for matrix
\usepgfplotslibrary{polar} %% for polar plots
\usepgfplotslibrary{fillbetween} %% to shade area between curves in TikZ
\usetkzobj{all}
\usepackage[makeroom]{cancel} %% for strike outs
%\usepackage{mathtools} %% for pretty underbrace % Breaks Ximera
%\usepackage{multicol}
\usepackage{pgffor} %% required for integral for loops



%% http://tex.stackexchange.com/questions/66490/drawing-a-tikz-arc-specifying-the-center
%% Draws beach ball
\tikzset{pics/carc/.style args={#1:#2:#3}{code={\draw[pic actions] (#1:#3) arc(#1:#2:#3);}}}



\usepackage{array}
\setlength{\extrarowheight}{+.1cm}
\newdimen\digitwidth
\settowidth\digitwidth{9}
\def\divrule#1#2{
\noalign{\moveright#1\digitwidth
\vbox{\hrule width#2\digitwidth}}}





\newcommand{\RR}{\mathbb R}
\newcommand{\R}{\mathbb R}
\newcommand{\N}{\mathbb N}
\newcommand{\Z}{\mathbb Z}

\newcommand{\sagemath}{\textsf{SageMath}}


%\renewcommand{\d}{\,d\!}
\renewcommand{\d}{\mathop{}\!d}
\newcommand{\dd}[2][]{\frac{\d #1}{\d #2}}
\newcommand{\pp}[2][]{\frac{\partial #1}{\partial #2}}
\renewcommand{\l}{\ell}
\newcommand{\ddx}{\frac{d}{\d x}}

\newcommand{\zeroOverZero}{\ensuremath{\boldsymbol{\tfrac{0}{0}}}}
\newcommand{\inftyOverInfty}{\ensuremath{\boldsymbol{\tfrac{\infty}{\infty}}}}
\newcommand{\zeroOverInfty}{\ensuremath{\boldsymbol{\tfrac{0}{\infty}}}}
\newcommand{\zeroTimesInfty}{\ensuremath{\small\boldsymbol{0\cdot \infty}}}
\newcommand{\inftyMinusInfty}{\ensuremath{\small\boldsymbol{\infty - \infty}}}
\newcommand{\oneToInfty}{\ensuremath{\boldsymbol{1^\infty}}}
\newcommand{\zeroToZero}{\ensuremath{\boldsymbol{0^0}}}
\newcommand{\inftyToZero}{\ensuremath{\boldsymbol{\infty^0}}}



\newcommand{\numOverZero}{\ensuremath{\boldsymbol{\tfrac{\#}{0}}}}
\newcommand{\dfn}{\textbf}
%\newcommand{\unit}{\,\mathrm}
\newcommand{\unit}{\mathop{}\!\mathrm}
\newcommand{\eval}[1]{\bigg[ #1 \bigg]}
\newcommand{\seq}[1]{\left( #1 \right)}
\renewcommand{\epsilon}{\varepsilon}
\renewcommand{\phi}{\varphi}


\renewcommand{\iff}{\Leftrightarrow}

\DeclareMathOperator{\arccot}{arccot}
\DeclareMathOperator{\arcsec}{arcsec}
\DeclareMathOperator{\arccsc}{arccsc}
\DeclareMathOperator{\si}{Si}
\DeclareMathOperator{\scal}{scal}
\DeclareMathOperator{\sign}{sign}


%% \newcommand{\tightoverset}[2]{% for arrow vec
%%   \mathop{#2}\limits^{\vbox to -.5ex{\kern-0.75ex\hbox{$#1$}\vss}}}
\newcommand{\arrowvec}[1]{{\overset{\rightharpoonup}{#1}}}
%\renewcommand{\vec}[1]{\arrowvec{\mathbf{#1}}}
\renewcommand{\vec}[1]{{\overset{\boldsymbol{\rightharpoonup}}{\mathbf{#1}}}\hspace{0in}}

\newcommand{\point}[1]{\left(#1\right)} %this allows \vector{ to be changed to \vector{ with a quick find and replace
\newcommand{\pt}[1]{\mathbf{#1}} %this allows \vec{ to be changed to \vec{ with a quick find and replace
\newcommand{\Lim}[2]{\lim_{\point{#1} \to \point{#2}}} %Bart, I changed this to point since I want to use it.  It runs through both of the exercise and exerciseE files in limits section, which is why it was in each document to start with.

\DeclareMathOperator{\proj}{\mathbf{proj}}
\newcommand{\veci}{{\boldsymbol{\hat{\imath}}}}
\newcommand{\vecj}{{\boldsymbol{\hat{\jmath}}}}
\newcommand{\veck}{{\boldsymbol{\hat{k}}}}
\newcommand{\vecl}{\vec{\boldsymbol{\l}}}
\newcommand{\uvec}[1]{\mathbf{\hat{#1}}}
\newcommand{\utan}{\mathbf{\hat{t}}}
\newcommand{\unormal}{\mathbf{\hat{n}}}
\newcommand{\ubinormal}{\mathbf{\hat{b}}}

\newcommand{\dotp}{\bullet}
\newcommand{\cross}{\boldsymbol\times}
\newcommand{\grad}{\boldsymbol\nabla}
\newcommand{\divergence}{\grad\dotp}
\newcommand{\curl}{\grad\cross}
%\DeclareMathOperator{\divergence}{divergence}
%\DeclareMathOperator{\curl}[1]{\grad\cross #1}
\newcommand{\lto}{\mathop{\longrightarrow\,}\limits}

\renewcommand{\bar}{\overline}

\colorlet{textColor}{black}
\colorlet{background}{white}
\colorlet{penColor}{blue!50!black} % Color of a curve in a plot
\colorlet{penColor2}{red!50!black}% Color of a curve in a plot
\colorlet{penColor3}{red!50!blue} % Color of a curve in a plot
\colorlet{penColor4}{green!50!black} % Color of a curve in a plot
\colorlet{penColor5}{orange!80!black} % Color of a curve in a plot
\colorlet{penColor6}{yellow!70!black} % Color of a curve in a plot
\colorlet{fill1}{penColor!20} % Color of fill in a plot
\colorlet{fill2}{penColor2!20} % Color of fill in a plot
\colorlet{fillp}{fill1} % Color of positive area
\colorlet{filln}{penColor2!20} % Color of negative area
\colorlet{fill3}{penColor3!20} % Fill
\colorlet{fill4}{penColor4!20} % Fill
\colorlet{fill5}{penColor5!20} % Fill
\colorlet{gridColor}{gray!50} % Color of grid in a plot

\newcommand{\surfaceColor}{violet}
\newcommand{\surfaceColorTwo}{redyellow}
\newcommand{\sliceColor}{greenyellow}




\pgfmathdeclarefunction{gauss}{2}{% gives gaussian
  \pgfmathparse{1/(#2*sqrt(2*pi))*exp(-((x-#1)^2)/(2*#2^2))}%
}


%%%%%%%%%%%%%
%% Vectors
%%%%%%%%%%%%%

%% Simple horiz vectors
\renewcommand{\vector}[1]{\left\langle #1\right\rangle}


%% %% Complex Horiz Vectors with angle brackets
%% \makeatletter
%% \renewcommand{\vector}[2][ , ]{\left\langle%
%%   \def\nextitem{\def\nextitem{#1}}%
%%   \@for \el:=#2\do{\nextitem\el}\right\rangle%
%% }
%% \makeatother

%% %% Vertical Vectors
%% \def\vector#1{\begin{bmatrix}\vecListA#1,,\end{bmatrix}}
%% \def\vecListA#1,{\if,#1,\else #1\cr \expandafter \vecListA \fi}

%%%%%%%%%%%%%
%% End of vectors
%%%%%%%%%%%%%

%\newcommand{\fullwidth}{}
%\newcommand{\normalwidth}{}



%% makes a snazzy t-chart for evaluating functions
%\newenvironment{tchart}{\rowcolors{2}{}{background!90!textColor}\array}{\endarray}

%%This is to help with formatting on future title pages.
\newenvironment{sectionOutcomes}{}{}



%% Flowchart stuff
%\tikzstyle{startstop} = [rectangle, rounded corners, minimum width=3cm, minimum height=1cm,text centered, draw=black]
%\tikzstyle{question} = [rectangle, minimum width=3cm, minimum height=1cm, text centered, draw=black]
%\tikzstyle{decision} = [trapezium, trapezium left angle=70, trapezium right angle=110, minimum width=3cm, minimum height=1cm, text centered, draw=black]
%\tikzstyle{question} = [rectangle, rounded corners, minimum width=3cm, minimum height=1cm,text centered, draw=black]
%\tikzstyle{process} = [rectangle, minimum width=3cm, minimum height=1cm, text centered, draw=black]
%\tikzstyle{decision} = [trapezium, trapezium left angle=70, trapezium right angle=110, minimum width=3cm, minimum height=1cm, text centered, draw=black]


\title[Dig-In:]{Approximating area with rectangles}

\begin{document}
\begin{abstract}
  We introduce the basic idea of using rectangles to approximate the
  area under a curve.
\end{abstract}
\maketitle

\section{Limits and areas}

Here is the idea. We want to compute the area between the curve and
the horizontal axis.

Consider the function $f(x) = x^2 +1$ and suppose we wanted to know
the area between the curve $y=f(x)$ and the $x$-axis.
\begin{image}
  \begin{tikzpicture}
	\begin{axis}[
            domain=-1.2:1.2, xmin =-1.2,xmax=1.2,ymax=2.2,ymin=-.2,
            axis lines=center, xlabel=$x$, ylabel=$y$,
            every axis y label/.style={at=(current axis.above origin),anchor=south},
            every axis x label/.style={at=(current axis.right of origin),anchor=west},
            %xtick={1,1.25,1.5,1.75,2}, ytickmin=4, ytickmax=1,
            axis on top,
          ]
          \addplot [draw=none,fill=fillp,domain=-1:1, smooth] {x^2+1} \closedcycle;
          \addplot [very thick,penColor, smooth] {x^2+1};
%          \addplot [color=penColor,fill=penColor,only marks,mark=*] coordinates{(1,1)};  %% closed hole         
%          \addplot [color=penColor,fill=penColor,only marks,mark=*] coordinates{(1.25,1.06)};  %% closed hole       
%          \addplot [color=penColor,fill=penColor,only marks,mark=*] coordinates{(1.5,1.25)};  %% closed hole  
%          \addplot [color=penColor,fill=penColor,only marks,mark=*] coordinates{(1.75,1.56)};  %% closed hole       
          
          %\node at (axis cs:1,1) [textColor,above] {$f(x_0^*)$};
          %\node at (axis cs:1.25,1.06) [textColor,above] {$f(x_1^*)$};
          %\node at (axis cs:1.5,1.3) [textColor,above] {$f(x_2^*)$};
          %\node at (axis cs:1.75,1.6) [textColor,above] {$f(x_3^*)$};
        \end{axis}
\end{tikzpicture}
\end{image}
One way to do this would be to approximate the area with rectangles.
\begin{image}
  \begin{tikzpicture}
	\begin{axis}[
            domain=-1.2:1.2, xmin =-1.2,xmax=1.2,ymax=2.2,ymin=-.2,
            axis lines=center, xlabel=$x$, ylabel=$y$,
            every axis y label/.style={at=(current axis.above origin),anchor=south},
            every axis x label/.style={at=(current axis.right of origin),anchor=west},
            %xtick={1,1.25,1.5,1.75,2}, ytickmin=4, ytickmax=1,
            axis on top,
          ]
          \foreach \rectnumber in {1,2,...,8}
                   {
                     \addplot [draw=penColor,fill=fillp]
                     plot coordinates {({-1+(\rectnumber - 1) * 2/8},{(-1+\rectnumber * 2/8)^2+1} ) ({-1+(\rectnumber) * 2/8},{(-1+\rectnumber * 2/8)^2+1 })} \closedcycle;
                     }
          \addplot [very thick,penColor, smooth] {x^2+1};
%          \addplot [color=penColor,fill=penColor,only marks,mark=*] coordinates{(1,1)};  %% closed hole         
%          \addplot [color=penColor,fill=penColor,only marks,mark=*] coordinates{(1.25,1.06)};  %% closed hole       
%          \addplot [color=penColor,fill=penColor,only marks,mark=*] coordinates{(1.5,1.25)};  %% closed hole  
%          \addplot [color=penColor,fill=penColor,only marks,mark=*] coordinates{(1.75,1.56)};  %% closed hole       
          
          %\node at (axis cs:1,1) [textColor,above] {$f(x_0^*)$};
          %\node at (axis cs:1.25,1.06) [textColor,above] {$f(x_1^*)$};
          %\node at (axis cs:1.5,1.3) [textColor,above] {$f(x_2^*)$};
          %\node at (axis cs:1.75,1.6) [textColor,above] {$f(x_3^*)$};
        \end{axis}
\end{tikzpicture}
\end{image}
We could find the area exaclty if we could compute the limit
as the width of the rectangles goes to zero (and hence the number of
rectangles will often go to infinity). We'll take this very slowly.




\section{But which set of rectangles?}

If we are going to try and actually use many small rectangles to
compute the area under a curve, we should decide on exactly
\textit{which} rectangles we want to use. Here are three options that
we consider:

\paragraph{Rectangles defined by a left endpoint}

We can set the rectangles up so that the left endpoint touches the
curve.

\begin{image}
  \begin{tikzpicture}
	\begin{axis}[
            domain=-1.2:1.2, xmin =-1.2,xmax=1.2,ymax=2.2,ymin=-.2,
            axis lines=center, xlabel=$x$, ylabel=$y$,
            every axis y label/.style={at=(current axis.above origin),anchor=south},
            every axis x label/.style={at=(current axis.right of origin),anchor=west},
            %xtick={1,1.25,1.5,1.75,2}, ytickmin=4, ytickmax=1,
            axis on top,
          ]
          \foreach \rectnumber in {1,2,...,8}
                   {
                     \addplot [draw=penColor,fill=fillp]
                     plot coordinates {({-1+(\rectnumber - 1) * 2/8},{(-1+(\rectnumber-1) * 2/8)^2+1} ) ({-1+(\rectnumber) * 2/8},{(-1+(\rectnumber-1) * 2/8)^2+1 })} \closedcycle;
                     }
          \addplot [very thick,penColor, smooth] {x^2+1};
%          \addplot [color=penColor,fill=penColor,only marks,mark=*] coordinates{(1,1)};  %% closed hole         
%          \addplot [color=penColor,fill=penColor,only marks,mark=*] coordinates{(1.25,1.06)};  %% closed hole       
%          \addplot [color=penColor,fill=penColor,only marks,mark=*] coordinates{(1.5,1.25)};  %% closed hole  
%          \addplot [color=penColor,fill=penColor,only marks,mark=*] coordinates{(1.75,1.56)};  %% closed hole       
          
          %\node at (axis cs:1,1) [textColor,above] {$f(x_0^*)$};
          %\node at (axis cs:1.25,1.06) [textColor,above] {$f(x_1^*)$};
          %\node at (axis cs:1.5,1.3) [textColor,above] {$f(x_2^*)$};
          %\node at (axis cs:1.75,1.6) [textColor,above] {$f(x_3^*)$};
        \end{axis}
\end{tikzpicture}
\end{image}

In the graph above, every rectangle is touching the curve at its
left-endpoint.


\paragraph{Rectangles defined by a right endpoint}

We can set the rectangles up so that the right endpoint touches the
curve.

\begin{image}
  \begin{tikzpicture}
	\begin{axis}[
            domain=-1.2:1.2, xmin =-1.2,xmax=1.2,ymax=2.2,ymin=-.2,
            axis lines=center, xlabel=$x$, ylabel=$y$,
            every axis y label/.style={at=(current axis.above origin),anchor=south},
            every axis x label/.style={at=(current axis.right of origin),anchor=west},
            %xtick={1,1.25,1.5,1.75,2}, ytickmin=4, ytickmax=1,
            axis on top,
          ]
          \foreach \rectnumber in {1,2,...,8}
                   {
                     \addplot [draw=penColor,fill=fillp]
                     plot coordinates {({-1+(\rectnumber - 1) * 2/8},{(-1+\rectnumber * 2/8)^2+1} ) ({-1+(\rectnumber) * 2/8},{(-1+\rectnumber * 2/8)^2+1 })} \closedcycle;
                     }
          \addplot [very thick,penColor, smooth] {x^2+1};
%          \addplot [color=penColor,fill=penColor,only marks,mark=*] coordinates{(1,1)};  %% closed hole         
%          \addplot [color=penColor,fill=penColor,only marks,mark=*] coordinates{(1.25,1.06)};  %% closed hole       
%          \addplot [color=penColor,fill=penColor,only marks,mark=*] coordinates{(1.5,1.25)};  %% closed hole  
%          \addplot [color=penColor,fill=penColor,only marks,mark=*] coordinates{(1.75,1.56)};  %% closed hole       
          
          %\node at (axis cs:1,1) [textColor,above] {$f(x_0^*)$};
          %\node at (axis cs:1.25,1.06) [textColor,above] {$f(x_1^*)$};
          %\node at (axis cs:1.5,1.3) [textColor,above] {$f(x_2^*)$};
          %\node at (axis cs:1.75,1.6) [textColor,above] {$f(x_3^*)$};
        \end{axis}
\end{tikzpicture}
\end{image}


In the graph above, every rectangle is touching the curve at its
right-endpoint.

\paragraph{Rectangles defined by a midpoint}


\begin{image}
  \begin{tikzpicture}
	\begin{axis}[
            domain=-1.2:1.2, xmin =-1.2,xmax=1.2,ymax=2.2,ymin=-.2,
            axis lines=center, xlabel=$x$, ylabel=$y$,
            every axis y label/.style={at=(current axis.above origin),anchor=south},
            every axis x label/.style={at=(current axis.right of origin),anchor=west},
            %xtick={1,1.25,1.5,1.75,2}, ytickmin=4, ytickmax=1,
            axis on top,
          ]
          \foreach \rectnumber in {1,2,...,8}
                   {
                     \addplot [draw=penColor,fill=fillp]
                     plot coordinates {({-1+(\rectnumber - 1) * 2/8},{(-1+(\rectnumber-.5) * 2/8)^2+1} ) ({-1+(\rectnumber) * 2/8},{(-1+(\rectnumber -.5)* 2/8)^2+1 })} \closedcycle;
                   }
          \addplot [very thick,penColor, smooth] {x^2+1};
%          \addplot [color=penColor,fill=penColor,only marks,mark=*] coordinates{(1,1)};  %% closed hole         
%          \addplot [color=penColor,fill=penColor,only marks,mark=*] coordinates{(1.25,1.06)};  %% closed hole       
%          \addplot [color=penColor,fill=penColor,only marks,mark=*] coordinates{(1.5,1.25)};  %% closed hole  
%          \addplot [color=penColor,fill=penColor,only marks,mark=*] coordinates{(1.75,1.56)};  %% closed hole       
          
          %\node at (axis cs:1,1) [textColor,above] {$f(x_0^*)$};
          %\node at (axis cs:1.25,1.06) [textColor,above] {$f(x_1^*)$};
          %\node at (axis cs:1.5,1.3) [textColor,above] {$f(x_2^*)$};
          %\node at (axis cs:1.75,1.6) [textColor,above] {$f(x_3^*)$};
        \end{axis}
\end{tikzpicture}
\end{image}


In the graph above, every rectangle is touching the curve at its
midpoint.

\section{Computing these approximations}

Once we know how to identify our rectangles, we can compute some
approximate areas. To help us out we need the following definitions:

\begin{definition}
  When approximating an area with rectangles, a \dfn{grid point} is a
  point $x$ that determines the edge of a rectangle.
\begin{image}
  \begin{tikzpicture}
	\begin{axis}[
            domain=-1.2:1.2, xmin =-1.09545,xmax=1.09545,ymax=2.2,ymin=-0.2,
            clip=false,
            axis lines=center, xlabel=$x$, ylabel=$y$,
            every axis y label/.style={at=(current axis.above origin),anchor=south},
            every axis x label/.style={at=(current axis.right of origin),anchor=west},
            %xtick={1,1.25,1.5,1.75,2}, ytickmin=4, ytickmax=1,
            axis on top,
          ]
          \foreach \rectnumber in {1,2,...,8}
          {
          \addplot [draw=penColor,fill=fillp]
          plot coordinates {({-1+(\rectnumber -1)*2/8},{(-1+(\rectnumber -.5)*2/8)^2+1}) ({-1+(\rectnumber)*2/8},{(-1+(\rectnumber -.5)*2/8)^2+1})}
          \closedcycle;
%          \pgfmathparse{-1+(\rectnumber -1)*2/8}\let\a\pgfmathresult
%          \draw [->,very thick] (axis cs:0,-.5) -- (axis cs:{\aa},-0.5);
          }
          \addplot [very thick,penColor, smooth] {x^2+1};

          
%          \addplot [color=penColor,fill=penColor,only marks,mark=*] coordinates{(1,1)};  %% closed hole         
%          \addplot [color=penColor,fill=penColor,only marks,mark=*] coordinates{(1.25,1.06)};  %% closed hole       
%          \addplot [color=penColor,fill=penColor,only marks,mark=*] coordinates{(1.5,1.25)};  %% closed hole  
%          \addplot [color=penColor,fill=penColor,only marks,mark=*] coordinates{(1.75,1.56)};  %% closed hole       
          
          %\node at (axis cs:1,1) [textColor,above] {$f(x_0^*)$};
          %\node at (axis cs:1.25,1.06) [textColor,above] {$f(x_1^*)$};
          %\node at (axis cs:1.5,1.3) [textColor,above] {$f(x_2^*)$};
          %\node at (axis cs:1.75,1.6) [textColor,above] {$f(x_3^*)$};
        \end{axis}
\end{tikzpicture}
\end{image}
Note, if we are approximating the area between a curve and the
horizontal axis for $a\le x\le b$ with $n$ rectangles, then it is
always the case that
\[
x_0=a\qquad\text{and}\qquad x_n = b.
\]
\end{definition}

\begin{question}
  If we are approximating the area between a curve and the horizontal
  axis with $11$ rectangles, how many grid points will we have?
  \begin{hint}
    You can draw it!
  \end{hint}
  \begin{prompt}
    We'll have $\answer{12}$ grid points.
  \end{prompt}
\end{question}



\paragraph{Rectangles defined by a left endpoint}

\begin{image}
  \begin{tikzpicture}
	\begin{axis}[
            domain=-1.2:1.2, xmin =-1.2,xmax=1.2,ymax=2.2,ymin=-.2,
            axis lines=center, xlabel=$x$, ylabel=$y$,
            every axis y label/.style={at=(current axis.above origin),anchor=south},
            every axis x label/.style={at=(current axis.right of origin),anchor=west},
            %xtick={1,1.25,1.5,1.75,2}, ytickmin=4, ytickmax=1,
            axis on top,
          ]
          \foreach \rectnumber in {1,2,...,8}
                   {
                     \addplot [draw=penColor,fill=fillp]
                     plot coordinates {({-1+(\rectnumber - 1) * 2/8},{(-1+(\rectnumber-1) * 2/8)^2+1} ) ({-1+(\rectnumber) * 2/8},{(-1+(\rectnumber-1) * 2/8)^2+1 })} \closedcycle;
                   }
          \addplot [very thick,penColor, smooth] {x^2+1};
%          \addplot [color=penColor,fill=penColor,only marks,mark=*] coordinates{(1,1)};  %% closed hole         
%          \addplot [color=penColor,fill=penColor,only marks,mark=*] coordinates{(1.25,1.06)};  %% closed hole       
%          \addplot [color=penColor,fill=penColor,only marks,mark=*] coordinates{(1.5,1.25)};  %% closed hole  
%          \addplot [color=penColor,fill=penColor,only marks,mark=*] coordinates{(1.75,1.56)};  %% closed hole       
          
          %\node at (axis cs:1,1) [textColor,above] {$f(x_0^*)$};
          %\node at (axis cs:1.25,1.06) [textColor,above] {$f(x_1^*)$};
          %\node at (axis cs:1.5,1.3) [textColor,above] {$f(x_2^*)$};
          %\node at (axis cs:1.75,1.6) [textColor,above] {$f(x_3^*)$};
        \end{axis}
\end{tikzpicture}
\end{image}


\[
\begin{array}{c|c|c|c}
  k &  x_k  & x^*_k & f(x^*_k) \\ \hline
  0 & -1    & \text{NA} & \text{NA}  \\
  1 & -0.75 & -1    &   2      \\
  2 & -0.5  & -0.75 & 1.5625   \\
  3 & -0.25 & -0.5  & 1.25     \\
  4 &  0    & -0.25 & 1.0625   \\
  5 &  0.25 &  0    & 1        \\
  6 &  0.5  &  0.25 & 1.0625   \\
  7 &  0.75 &  0.5  & 1.25     \\
  8 &  1    &  0.75 & 1.5625   
\end{array}
\]

\paragraph{Rectangles defined by a right endpoint}


\paragraph{Rectangles defined by a midpoint}

\section{Riemann sums}



\end{document}
