\documentclass{ximera}

%\usepackage{todonotes}
%\usepackage{mathtools} %% Required for wide table Curl and Greens
%\usepackage{cuted} %% Required for wide table Curl and Greens
\newcommand{\todo}{}

\usepackage{esint} % for \oiint
\ifxake%%https://math.meta.stackexchange.com/questions/9973/how-do-you-render-a-closed-surface-double-integral
\renewcommand{\oiint}{{\large\bigcirc}\kern-1.56em\iint}
\fi


\graphicspath{
  {./}
  {ximeraTutorial/}
  {basicPhilosophy/}
  {functionsOfSeveralVariables/}
  {normalVectors/}
  {lagrangeMultipliers/}
  {vectorFields/}
  {greensTheorem/}
  {shapeOfThingsToCome/}
  {dotProducts/}
  {partialDerivativesAndTheGradientVector/}
  {../productAndQuotientRules/exercises/}
  {../normalVectors/exercisesParametricPlots/}
  {../continuityOfFunctionsOfSeveralVariables/exercises/}
  {../partialDerivativesAndTheGradientVector/exercises/}
  {../directionalDerivativeAndChainRule/exercises/}
  {../commonCoordinates/exercisesCylindricalCoordinates/}
  {../commonCoordinates/exercisesSphericalCoordinates/}
  {../greensTheorem/exercisesCurlAndLineIntegrals/}
  {../greensTheorem/exercisesDivergenceAndLineIntegrals/}
  {../shapeOfThingsToCome/exercisesDivergenceTheorem/}
  {../greensTheorem/}
  {../shapeOfThingsToCome/}
  {../separableDifferentialEquations/exercises/}
  {vectorFields/}
}

\newcommand{\mooculus}{\textsf{\textbf{MOOC}\textnormal{\textsf{ULUS}}}}

\usepackage{tkz-euclide}\usepackage{tikz}
\usepackage{tikz-cd}
\usetikzlibrary{arrows}
\tikzset{>=stealth,commutative diagrams/.cd,
  arrow style=tikz,diagrams={>=stealth}} %% cool arrow head
\tikzset{shorten <>/.style={ shorten >=#1, shorten <=#1 } } %% allows shorter vectors

\usetikzlibrary{backgrounds} %% for boxes around graphs
\usetikzlibrary{shapes,positioning}  %% Clouds and stars
\usetikzlibrary{matrix} %% for matrix
\usepgfplotslibrary{polar} %% for polar plots
\usepgfplotslibrary{fillbetween} %% to shade area between curves in TikZ
\usetkzobj{all}
\usepackage[makeroom]{cancel} %% for strike outs
%\usepackage{mathtools} %% for pretty underbrace % Breaks Ximera
%\usepackage{multicol}
\usepackage{pgffor} %% required for integral for loops



%% http://tex.stackexchange.com/questions/66490/drawing-a-tikz-arc-specifying-the-center
%% Draws beach ball
\tikzset{pics/carc/.style args={#1:#2:#3}{code={\draw[pic actions] (#1:#3) arc(#1:#2:#3);}}}



\usepackage{array}
\setlength{\extrarowheight}{+.1cm}
\newdimen\digitwidth
\settowidth\digitwidth{9}
\def\divrule#1#2{
\noalign{\moveright#1\digitwidth
\vbox{\hrule width#2\digitwidth}}}





\newcommand{\RR}{\mathbb R}
\newcommand{\R}{\mathbb R}
\newcommand{\N}{\mathbb N}
\newcommand{\Z}{\mathbb Z}

\newcommand{\sagemath}{\textsf{SageMath}}


%\renewcommand{\d}{\,d\!}
\renewcommand{\d}{\mathop{}\!d}
\newcommand{\dd}[2][]{\frac{\d #1}{\d #2}}
\newcommand{\pp}[2][]{\frac{\partial #1}{\partial #2}}
\renewcommand{\l}{\ell}
\newcommand{\ddx}{\frac{d}{\d x}}

\newcommand{\zeroOverZero}{\ensuremath{\boldsymbol{\tfrac{0}{0}}}}
\newcommand{\inftyOverInfty}{\ensuremath{\boldsymbol{\tfrac{\infty}{\infty}}}}
\newcommand{\zeroOverInfty}{\ensuremath{\boldsymbol{\tfrac{0}{\infty}}}}
\newcommand{\zeroTimesInfty}{\ensuremath{\small\boldsymbol{0\cdot \infty}}}
\newcommand{\inftyMinusInfty}{\ensuremath{\small\boldsymbol{\infty - \infty}}}
\newcommand{\oneToInfty}{\ensuremath{\boldsymbol{1^\infty}}}
\newcommand{\zeroToZero}{\ensuremath{\boldsymbol{0^0}}}
\newcommand{\inftyToZero}{\ensuremath{\boldsymbol{\infty^0}}}



\newcommand{\numOverZero}{\ensuremath{\boldsymbol{\tfrac{\#}{0}}}}
\newcommand{\dfn}{\textbf}
%\newcommand{\unit}{\,\mathrm}
\newcommand{\unit}{\mathop{}\!\mathrm}
\newcommand{\eval}[1]{\bigg[ #1 \bigg]}
\newcommand{\seq}[1]{\left( #1 \right)}
\renewcommand{\epsilon}{\varepsilon}
\renewcommand{\phi}{\varphi}


\renewcommand{\iff}{\Leftrightarrow}

\DeclareMathOperator{\arccot}{arccot}
\DeclareMathOperator{\arcsec}{arcsec}
\DeclareMathOperator{\arccsc}{arccsc}
\DeclareMathOperator{\si}{Si}
\DeclareMathOperator{\scal}{scal}
\DeclareMathOperator{\sign}{sign}


%% \newcommand{\tightoverset}[2]{% for arrow vec
%%   \mathop{#2}\limits^{\vbox to -.5ex{\kern-0.75ex\hbox{$#1$}\vss}}}
\newcommand{\arrowvec}[1]{{\overset{\rightharpoonup}{#1}}}
%\renewcommand{\vec}[1]{\arrowvec{\mathbf{#1}}}
\renewcommand{\vec}[1]{{\overset{\boldsymbol{\rightharpoonup}}{\mathbf{#1}}}\hspace{0in}}

\newcommand{\point}[1]{\left(#1\right)} %this allows \vector{ to be changed to \vector{ with a quick find and replace
\newcommand{\pt}[1]{\mathbf{#1}} %this allows \vec{ to be changed to \vec{ with a quick find and replace
\newcommand{\Lim}[2]{\lim_{\point{#1} \to \point{#2}}} %Bart, I changed this to point since I want to use it.  It runs through both of the exercise and exerciseE files in limits section, which is why it was in each document to start with.

\DeclareMathOperator{\proj}{\mathbf{proj}}
\newcommand{\veci}{{\boldsymbol{\hat{\imath}}}}
\newcommand{\vecj}{{\boldsymbol{\hat{\jmath}}}}
\newcommand{\veck}{{\boldsymbol{\hat{k}}}}
\newcommand{\vecl}{\vec{\boldsymbol{\l}}}
\newcommand{\uvec}[1]{\mathbf{\hat{#1}}}
\newcommand{\utan}{\mathbf{\hat{t}}}
\newcommand{\unormal}{\mathbf{\hat{n}}}
\newcommand{\ubinormal}{\mathbf{\hat{b}}}

\newcommand{\dotp}{\bullet}
\newcommand{\cross}{\boldsymbol\times}
\newcommand{\grad}{\boldsymbol\nabla}
\newcommand{\divergence}{\grad\dotp}
\newcommand{\curl}{\grad\cross}
%\DeclareMathOperator{\divergence}{divergence}
%\DeclareMathOperator{\curl}[1]{\grad\cross #1}
\newcommand{\lto}{\mathop{\longrightarrow\,}\limits}

\renewcommand{\bar}{\overline}

\colorlet{textColor}{black}
\colorlet{background}{white}
\colorlet{penColor}{blue!50!black} % Color of a curve in a plot
\colorlet{penColor2}{red!50!black}% Color of a curve in a plot
\colorlet{penColor3}{red!50!blue} % Color of a curve in a plot
\colorlet{penColor4}{green!50!black} % Color of a curve in a plot
\colorlet{penColor5}{orange!80!black} % Color of a curve in a plot
\colorlet{penColor6}{yellow!70!black} % Color of a curve in a plot
\colorlet{fill1}{penColor!20} % Color of fill in a plot
\colorlet{fill2}{penColor2!20} % Color of fill in a plot
\colorlet{fillp}{fill1} % Color of positive area
\colorlet{filln}{penColor2!20} % Color of negative area
\colorlet{fill3}{penColor3!20} % Fill
\colorlet{fill4}{penColor4!20} % Fill
\colorlet{fill5}{penColor5!20} % Fill
\colorlet{gridColor}{gray!50} % Color of grid in a plot

\newcommand{\surfaceColor}{violet}
\newcommand{\surfaceColorTwo}{redyellow}
\newcommand{\sliceColor}{greenyellow}




\pgfmathdeclarefunction{gauss}{2}{% gives gaussian
  \pgfmathparse{1/(#2*sqrt(2*pi))*exp(-((x-#1)^2)/(2*#2^2))}%
}


%%%%%%%%%%%%%
%% Vectors
%%%%%%%%%%%%%

%% Simple horiz vectors
\renewcommand{\vector}[1]{\left\langle #1\right\rangle}


%% %% Complex Horiz Vectors with angle brackets
%% \makeatletter
%% \renewcommand{\vector}[2][ , ]{\left\langle%
%%   \def\nextitem{\def\nextitem{#1}}%
%%   \@for \el:=#2\do{\nextitem\el}\right\rangle%
%% }
%% \makeatother

%% %% Vertical Vectors
%% \def\vector#1{\begin{bmatrix}\vecListA#1,,\end{bmatrix}}
%% \def\vecListA#1,{\if,#1,\else #1\cr \expandafter \vecListA \fi}

%%%%%%%%%%%%%
%% End of vectors
%%%%%%%%%%%%%

%\newcommand{\fullwidth}{}
%\newcommand{\normalwidth}{}



%% makes a snazzy t-chart for evaluating functions
%\newenvironment{tchart}{\rowcolors{2}{}{background!90!textColor}\array}{\endarray}

%%This is to help with formatting on future title pages.
\newenvironment{sectionOutcomes}{}{}



%% Flowchart stuff
%\tikzstyle{startstop} = [rectangle, rounded corners, minimum width=3cm, minimum height=1cm,text centered, draw=black]
%\tikzstyle{question} = [rectangle, minimum width=3cm, minimum height=1cm, text centered, draw=black]
%\tikzstyle{decision} = [trapezium, trapezium left angle=70, trapezium right angle=110, minimum width=3cm, minimum height=1cm, text centered, draw=black]
%\tikzstyle{question} = [rectangle, rounded corners, minimum width=3cm, minimum height=1cm,text centered, draw=black]
%\tikzstyle{process} = [rectangle, minimum width=3cm, minimum height=1cm, text centered, draw=black]
%\tikzstyle{decision} = [trapezium, trapezium left angle=70, trapezium right angle=110, minimum width=3cm, minimum height=1cm, text centered, draw=black]


\outcome{}

\title[Dig-In:]{Sigma Notation}

\begin{document}
\begin{abstract}
	Sigma notation is a convenient way to express a sum of many terms
\end{abstract}

\begin{definition}
	Let $f$ be a function, and $m \leq n$ integers.  Then we can write the sum 
	
	\[ f(m)+f(m+1)+f(m+2)+...+f(n-1)+f(n) = \displaystyle\sum_{k=m}^{k=n} f(k)\] 
	
	We read this as ``The sum of $f$ of $k$ from $k$ equals $m$ to $k$ equals $n$''
	
	\end{definition}

\begin{example}
  Write out the terms of this sum:
  \[
  \displaystyle\sum_{k=2}^{k=5} \sin(k)
  \]
  \begin{explanation}
  \begin{prompt}
    \[
    \sin(\answer{2}) + \sin(3) + \answer{\sin(4)} + \sin(5)
    \]
  \end{prompt}

  \end{explanation}
\end{example}

\begin{example}
$\displaystyle\sum_{k=3}^{k=4} \frac{1}{1+k} = \frac{1}{4}+\frac{1}{5}$
\end{example}

\begin{question}
	$\displaystyle\sum_{k=1}^{k=4} k  = \answer{10}$
	\begin{hint}
		$\displaystyle\sum_{k=1}^{k=4} k  = 1+2+3+4=10$
	\end{hint}
\end{question}

The variable $k$ in $\displaystyle\sum_{k=m}^{k=n}$ is called the ``index of summation'', or just ``the index''.  It can be any variable we like, but the letters $i,j,k$ are used traditionally.

\begin{question}
$\displaystyle\sum_{i=2}^{i=3} i^2  = \answer{13}$
\begin{hint}
	$\displaystyle\sum_{i=2}^{i=3} i^2  = 2^2+3^2 = 4+9=13$
\end{hint}
\end{question}

\begin{question}
	$1+\frac{1}{2} + \frac{1}{3} + \frac{1}{4} = \displaystyle\sum_{j=1}^{j= \answer{4}} \answer{1/j}$
	\begin{hint}
		There are $4$ terms, so since we start counting at $j=1$, we must go up to $j=4$.
	\end{hint}
	\begin{hint}
		$1+\frac{1}{2} + \frac{1}{3} + \frac{1}{4} = \displaystyle\sum_{j=1}^{j= 4} \frac{1}{j}$
	\end{hint}
\end{question}

\begin{question}
	$\displaystyle\sum_{i=5}^{i=5} i^3 = \answer{125}$
	\begin{hint}
		This is kind of funny, but in this case we just have one term, namely $5^3 = 125$
	\end{hint}
\end{question}

\begin{question}

	$\displaystyle\sum_{i=2}^{i=6} k = \answer{20}$

	 \begin{hint}
	 	$\displaystyle\sum_{i=2}^{i=6} k = 2+3+4+5+6 = 20$
	 \end{hint}
	
	$\displaystyle\sum_{j=0}^{j=4} (j+2) = \answer{20}$
	 \begin{hint}
	 	$\displaystyle\sum_{i=2}^{i=6} k = 2+3+4+5+6 = 20$ again!
	 \end{hint}
	
	\begin{feedback}
Did you notice how these two expressions had all the same terms?  Both are just shorthands for the sum $2+3+4+5+6$.  This is called ``reindexing'' a sum.
	\end{feedback}
\end{question}

\begin{question}
	Reindexing the sum $\displaystyle\sum_{j=4}^{j=7} \sin(j-2)$ to start at $k=1$, we have $\displaystyle\sum_{j=4}^{j=7} \sin(j-2) = \displaystyle\sum_{k=1}^{k=\answer{4}} \answer{\sin(k+1)}$
		\begin{hint}
			$\displaystyle\sum_{j=4}^{j=7} \sin(j-2) = \sin(2)+\sin(3) + \sin(4)+\sin(5) = \displaystyle\sum_{k=1}^{k=4} \sin(k+1)$
		\end{hint}
\end{question}

\begin{question}
	The sum $\sin(4+\frac{3}{n}) + \sin(4+\frac{6}{n})+\sin(4+\frac{9}{n})+...$ has $n$ terms. In sigma notation, this sum can be expressed as $\displaystyle\sum_{k=1}^{k=n} \answer{ \sin(4+\frac{3k}{n})}$
		\begin{hint}
			The $k^{\textrm{th}}$ term is of the form $4+\frac{3k}{n}$, so the sum is $\displaystyle\sum_{k=1}^{k=n} \sin(4+\frac{3k}{n})$
		\end{hint}
\end{question}

\begin{question}
	Fix a number $n$.  Then $\displaystyle\sum_{k=1}^{k=n} 1 = \answer{n}$
	 \begin{hint}
	 	By definition, $\displaystyle\sum_{k=1}^{k=n} 1$ is the sum of $n$ ones, which is just $n$
	 \end{hint}
\end{question}

\begin{question}
	If $\displaystyle\sum_{k=1}^{k=n} f(k) = n^2$, then $f(j) = \answer{2j-1}$
		\begin{hint}
			To find $f(j)$, we could think of this as $\displaystyle\sum_{k=1}^{k=j} f(j) - \displaystyle\sum_{k=1}^{k=j-1} f(j)$
		\end{hint}
		\begin{hint}
			So $f(j) = j^2 - (j-1)^2 = j^2- (j^2-2j+1) = 2j-1$
		\end{hint}
	\begin{feedback}
This is kind of cool.  It says that the sum of the first $n$ odd number is $n^2$.  Test it and see!  Can you find a geometric interpretation of this?  If you are interested by this, talk to your TA!
	\end{feedback}
\end{question}

\begin{question}
	Which of the following equations could possibly make any sense at all?  Mark all that apply.
	
	\begin{multipleChoice}
	\choice{$\displaystyle\sum_{j=1}^{j=n} f(j) = j^3$}
	\choice[correct]{$\displaystyle\sum_{j=1}^{j=n} f(n) = nf(n)$}
	\choice[correct]{$\displaystyle\sum_{j=1}^{j=n} f(j) = n^3$}
	\end{multipleChoice}
	
	\begin{hint}
		\begin{itemize}
		\item $\displaystyle\sum_{j=1}^{j=n} f(j) = j^3$ cannot make any sense, since one one side $j$ is telling us the index of a term we are summing, and on the other side it is a fixed number.  These two meanings of $j$ cannot coexist.
		\item $\displaystyle\sum_{j=1}^{j=n} f(n) = nf(n)$ not only makes sense, it is universally true!  It is okay that $n$ appears in all three parts of the expression, since it is just a fixed number
		\item $\displaystyle\sum_{j=1}^{j=n} f(j) = n^3$ is also fine.  As a bonus challenge, can you find the function $f$ which makes this true?
		\end{itemize}
	\end{hint}
\end{question}


\end{document}
