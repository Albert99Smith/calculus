\documentclass{ximera}

%\usepackage{todonotes}
%\usepackage{mathtools} %% Required for wide table Curl and Greens
%\usepackage{cuted} %% Required for wide table Curl and Greens
\newcommand{\todo}{}

\usepackage{esint} % for \oiint
\ifxake%%https://math.meta.stackexchange.com/questions/9973/how-do-you-render-a-closed-surface-double-integral
\renewcommand{\oiint}{{\large\bigcirc}\kern-1.56em\iint}
\fi


\graphicspath{
  {./}
  {ximeraTutorial/}
  {basicPhilosophy/}
  {functionsOfSeveralVariables/}
  {normalVectors/}
  {lagrangeMultipliers/}
  {vectorFields/}
  {greensTheorem/}
  {shapeOfThingsToCome/}
  {dotProducts/}
  {partialDerivativesAndTheGradientVector/}
  {../productAndQuotientRules/exercises/}
  {../normalVectors/exercisesParametricPlots/}
  {../continuityOfFunctionsOfSeveralVariables/exercises/}
  {../partialDerivativesAndTheGradientVector/exercises/}
  {../directionalDerivativeAndChainRule/exercises/}
  {../commonCoordinates/exercisesCylindricalCoordinates/}
  {../commonCoordinates/exercisesSphericalCoordinates/}
  {../greensTheorem/exercisesCurlAndLineIntegrals/}
  {../greensTheorem/exercisesDivergenceAndLineIntegrals/}
  {../shapeOfThingsToCome/exercisesDivergenceTheorem/}
  {../greensTheorem/}
  {../shapeOfThingsToCome/}
  {../separableDifferentialEquations/exercises/}
  {vectorFields/}
}

\newcommand{\mooculus}{\textsf{\textbf{MOOC}\textnormal{\textsf{ULUS}}}}

\usepackage{tkz-euclide}\usepackage{tikz}
\usepackage{tikz-cd}
\usetikzlibrary{arrows}
\tikzset{>=stealth,commutative diagrams/.cd,
  arrow style=tikz,diagrams={>=stealth}} %% cool arrow head
\tikzset{shorten <>/.style={ shorten >=#1, shorten <=#1 } } %% allows shorter vectors

\usetikzlibrary{backgrounds} %% for boxes around graphs
\usetikzlibrary{shapes,positioning}  %% Clouds and stars
\usetikzlibrary{matrix} %% for matrix
\usepgfplotslibrary{polar} %% for polar plots
\usepgfplotslibrary{fillbetween} %% to shade area between curves in TikZ
\usetkzobj{all}
\usepackage[makeroom]{cancel} %% for strike outs
%\usepackage{mathtools} %% for pretty underbrace % Breaks Ximera
%\usepackage{multicol}
\usepackage{pgffor} %% required for integral for loops



%% http://tex.stackexchange.com/questions/66490/drawing-a-tikz-arc-specifying-the-center
%% Draws beach ball
\tikzset{pics/carc/.style args={#1:#2:#3}{code={\draw[pic actions] (#1:#3) arc(#1:#2:#3);}}}



\usepackage{array}
\setlength{\extrarowheight}{+.1cm}
\newdimen\digitwidth
\settowidth\digitwidth{9}
\def\divrule#1#2{
\noalign{\moveright#1\digitwidth
\vbox{\hrule width#2\digitwidth}}}





\newcommand{\RR}{\mathbb R}
\newcommand{\R}{\mathbb R}
\newcommand{\N}{\mathbb N}
\newcommand{\Z}{\mathbb Z}

\newcommand{\sagemath}{\textsf{SageMath}}


%\renewcommand{\d}{\,d\!}
\renewcommand{\d}{\mathop{}\!d}
\newcommand{\dd}[2][]{\frac{\d #1}{\d #2}}
\newcommand{\pp}[2][]{\frac{\partial #1}{\partial #2}}
\renewcommand{\l}{\ell}
\newcommand{\ddx}{\frac{d}{\d x}}

\newcommand{\zeroOverZero}{\ensuremath{\boldsymbol{\tfrac{0}{0}}}}
\newcommand{\inftyOverInfty}{\ensuremath{\boldsymbol{\tfrac{\infty}{\infty}}}}
\newcommand{\zeroOverInfty}{\ensuremath{\boldsymbol{\tfrac{0}{\infty}}}}
\newcommand{\zeroTimesInfty}{\ensuremath{\small\boldsymbol{0\cdot \infty}}}
\newcommand{\inftyMinusInfty}{\ensuremath{\small\boldsymbol{\infty - \infty}}}
\newcommand{\oneToInfty}{\ensuremath{\boldsymbol{1^\infty}}}
\newcommand{\zeroToZero}{\ensuremath{\boldsymbol{0^0}}}
\newcommand{\inftyToZero}{\ensuremath{\boldsymbol{\infty^0}}}



\newcommand{\numOverZero}{\ensuremath{\boldsymbol{\tfrac{\#}{0}}}}
\newcommand{\dfn}{\textbf}
%\newcommand{\unit}{\,\mathrm}
\newcommand{\unit}{\mathop{}\!\mathrm}
\newcommand{\eval}[1]{\bigg[ #1 \bigg]}
\newcommand{\seq}[1]{\left( #1 \right)}
\renewcommand{\epsilon}{\varepsilon}
\renewcommand{\phi}{\varphi}


\renewcommand{\iff}{\Leftrightarrow}

\DeclareMathOperator{\arccot}{arccot}
\DeclareMathOperator{\arcsec}{arcsec}
\DeclareMathOperator{\arccsc}{arccsc}
\DeclareMathOperator{\si}{Si}
\DeclareMathOperator{\scal}{scal}
\DeclareMathOperator{\sign}{sign}


%% \newcommand{\tightoverset}[2]{% for arrow vec
%%   \mathop{#2}\limits^{\vbox to -.5ex{\kern-0.75ex\hbox{$#1$}\vss}}}
\newcommand{\arrowvec}[1]{{\overset{\rightharpoonup}{#1}}}
%\renewcommand{\vec}[1]{\arrowvec{\mathbf{#1}}}
\renewcommand{\vec}[1]{{\overset{\boldsymbol{\rightharpoonup}}{\mathbf{#1}}}\hspace{0in}}

\newcommand{\point}[1]{\left(#1\right)} %this allows \vector{ to be changed to \vector{ with a quick find and replace
\newcommand{\pt}[1]{\mathbf{#1}} %this allows \vec{ to be changed to \vec{ with a quick find and replace
\newcommand{\Lim}[2]{\lim_{\point{#1} \to \point{#2}}} %Bart, I changed this to point since I want to use it.  It runs through both of the exercise and exerciseE files in limits section, which is why it was in each document to start with.

\DeclareMathOperator{\proj}{\mathbf{proj}}
\newcommand{\veci}{{\boldsymbol{\hat{\imath}}}}
\newcommand{\vecj}{{\boldsymbol{\hat{\jmath}}}}
\newcommand{\veck}{{\boldsymbol{\hat{k}}}}
\newcommand{\vecl}{\vec{\boldsymbol{\l}}}
\newcommand{\uvec}[1]{\mathbf{\hat{#1}}}
\newcommand{\utan}{\mathbf{\hat{t}}}
\newcommand{\unormal}{\mathbf{\hat{n}}}
\newcommand{\ubinormal}{\mathbf{\hat{b}}}

\newcommand{\dotp}{\bullet}
\newcommand{\cross}{\boldsymbol\times}
\newcommand{\grad}{\boldsymbol\nabla}
\newcommand{\divergence}{\grad\dotp}
\newcommand{\curl}{\grad\cross}
%\DeclareMathOperator{\divergence}{divergence}
%\DeclareMathOperator{\curl}[1]{\grad\cross #1}
\newcommand{\lto}{\mathop{\longrightarrow\,}\limits}

\renewcommand{\bar}{\overline}

\colorlet{textColor}{black}
\colorlet{background}{white}
\colorlet{penColor}{blue!50!black} % Color of a curve in a plot
\colorlet{penColor2}{red!50!black}% Color of a curve in a plot
\colorlet{penColor3}{red!50!blue} % Color of a curve in a plot
\colorlet{penColor4}{green!50!black} % Color of a curve in a plot
\colorlet{penColor5}{orange!80!black} % Color of a curve in a plot
\colorlet{penColor6}{yellow!70!black} % Color of a curve in a plot
\colorlet{fill1}{penColor!20} % Color of fill in a plot
\colorlet{fill2}{penColor2!20} % Color of fill in a plot
\colorlet{fillp}{fill1} % Color of positive area
\colorlet{filln}{penColor2!20} % Color of negative area
\colorlet{fill3}{penColor3!20} % Fill
\colorlet{fill4}{penColor4!20} % Fill
\colorlet{fill5}{penColor5!20} % Fill
\colorlet{gridColor}{gray!50} % Color of grid in a plot

\newcommand{\surfaceColor}{violet}
\newcommand{\surfaceColorTwo}{redyellow}
\newcommand{\sliceColor}{greenyellow}




\pgfmathdeclarefunction{gauss}{2}{% gives gaussian
  \pgfmathparse{1/(#2*sqrt(2*pi))*exp(-((x-#1)^2)/(2*#2^2))}%
}


%%%%%%%%%%%%%
%% Vectors
%%%%%%%%%%%%%

%% Simple horiz vectors
\renewcommand{\vector}[1]{\left\langle #1\right\rangle}


%% %% Complex Horiz Vectors with angle brackets
%% \makeatletter
%% \renewcommand{\vector}[2][ , ]{\left\langle%
%%   \def\nextitem{\def\nextitem{#1}}%
%%   \@for \el:=#2\do{\nextitem\el}\right\rangle%
%% }
%% \makeatother

%% %% Vertical Vectors
%% \def\vector#1{\begin{bmatrix}\vecListA#1,,\end{bmatrix}}
%% \def\vecListA#1,{\if,#1,\else #1\cr \expandafter \vecListA \fi}

%%%%%%%%%%%%%
%% End of vectors
%%%%%%%%%%%%%

%\newcommand{\fullwidth}{}
%\newcommand{\normalwidth}{}



%% makes a snazzy t-chart for evaluating functions
%\newenvironment{tchart}{\rowcolors{2}{}{background!90!textColor}\array}{\endarray}

%%This is to help with formatting on future title pages.
\newenvironment{sectionOutcomes}{}{}



%% Flowchart stuff
%\tikzstyle{startstop} = [rectangle, rounded corners, minimum width=3cm, minimum height=1cm,text centered, draw=black]
%\tikzstyle{question} = [rectangle, minimum width=3cm, minimum height=1cm, text centered, draw=black]
%\tikzstyle{decision} = [trapezium, trapezium left angle=70, trapezium right angle=110, minimum width=3cm, minimum height=1cm, text centered, draw=black]
%\tikzstyle{question} = [rectangle, rounded corners, minimum width=3cm, minimum height=1cm,text centered, draw=black]
%\tikzstyle{process} = [rectangle, minimum width=3cm, minimum height=1cm, text centered, draw=black]
%\tikzstyle{decision} = [trapezium, trapezium left angle=70, trapezium right angle=110, minimum width=3cm, minimum height=1cm, text centered, draw=black]


\outcome{}

\title[Dig-In:]{Sigma Rules}

\begin{document}
\begin{abstract}
	We have formulas for some simple sums
\end{abstract}

We have already figured out that $\sum_{k=1}^{k=n} 1 = n$ in the last section.  This should make sense.  In words it just says ``The sum of $n$ ones is $n$".

We can generalize this slightly:

\begin{question}
	$\sum_{k=a}^{k=b} c = \answer{(b-a+1)c}$
	\begin{hint}
		How many terms are there?  Be careful!  Try some examples with small values of $a$ and $b$
	\end{hint}
	\begin{hint}
		There are $b-a+1$ terms.  For example, if $a=4$ and $b=5$, then there are $5-4+1 = 2$ terms.
	\end{hint}
	\begin{hint}
		So we are summing $b-a+1$ copies of $c$, which is $(b-a+1)c$
	\end{hint}
\end{question}

There is something worth highlighting about this example.

In $\sum_{k=1}^{k=n} c = c + c+ c+ ... + c$, no individual term involves $n$, but the number of terms does depend on $n$. 

The formula $\sum_{k=1}^{k=n} c = cn$ lets us rewrite this sum as a single function of $n$.

This idea, that we can sometimes simplify a sum of a variable number of terms into an explicit function of the number of terms, is very powerful.

We now develop a similar formula for $\sum_{k=1}^{k=n} k$

\begin{question}
	$\sum_{k=1}^{k=1} k = \answer{1}$
		\begin{hint}
			$\sum_{k=1}^{k=1} k = 1$
		\end{hint}
	
	$\sum_{k=1}^{k=2} k = \answer{3}$
		\begin{hint}
			$\sum_{k=1}^{k=2} k = 1+2 = 3$
		\end{hint}
	
	$\sum_{k=1}^{k=3} k = \answer{6}$
		\begin{hint}
			$\sum_{k=1}^{k=3} k = 1+2+3 = 6$
		\end{hint}
	
	$\sum_{k=1}^{k=4} k = \answer{10}$
		\begin{hint}
			$\sum_{k=1}^{k=4} k = 1+2+3+4 = 10$
		\end{hint}
	
\end{question}  

\begin{question}
	We are going to find a formula for $\sum_{k=1}^{n} k$ in terms of $n$.
	
	Write  .
	
	Then $S(n) = 1+2+3+\dots+(n-2)+(n-1)+n$.
	
	But we can also write $S(n) = n+(n-1)+(n-2)...+3+2+1$, since the order in which we add numbers does not affect the sum.
	
	Adding these two expressions for $S(n)$ together we have $2S(n) = (1+n)+[2+(n-1)]+[3+(n-2)]+...+[3+(n-2)]+...+[(n-2)+3]+[(n-1)+2]+(n+1)$.
		
	So $S(n) = \answer{\frac{n(n+1)}{2}}$ 
	 \begin{hint}
	 	 Each term simplifies to $n+1$, and there are $n$ terms, so the sum is $n(n+1)$.  Dividing both sides of the equation by $2$, we have $S(n) = \frac{n(n+1)}{2}$
	 \end{hint}
	 
	 Check this formula by plugging in $n=1,2,3,4$ and see that it agrees with the answers you computed ``by hand'' above.
	 
	 \begin{multipleChoice}
	 	\choice[correct]{On my honor, I did check that the formula works for these values}
		\choice{I did not check that the formula works for these values}
	 \end{multipleChoice}
	 
	 You might also enjoy the following visualization of the formula:
	 
	 %BADBAD picture of n x n+1 array decomposed into two "steps" representing the sum 1+2+3+...+n
\end{question}

\begin{example}
	Find a formula for $\sum_{k=1}^{n} k$ in terms of $n$.
	\begin{explanation}
          We may write
          \[
          S(n) = 1+2+3+\dots+(n-2)+(n-1)+n.
          \]
	  But we can also write
          \[
          S(n) = n+(n-1)+(n-2)+\dots+3+2+1,
          \]
          since the order in which we add numbers does not affect the sum. Adding these two expressions for $S(n)$ together we have
          \[
          2S(n) = (1+n)+[2+(n-1)]+[3+(n-2)]+...+[3+(n-2)]+\dots+[(n-2)+3]+[(n-1)+2]+\answer[given]{(n+1)}.
          \]
          Examine each term above. Each of these terms simplifies to
          $\answer[given]{n_1}$ and there are $\answer[given]{n}$
          terms total, so $S(n) = \answer{\frac{n(n+1)}{2}}$
	 \begin{hint}
	 	 Each term simplifies to $n+1$, and there are $n$ terms, so the sum is $n(n+1)$.  Dividing both sides of the equation by $2$, we have $S(n) = \frac{n(n+1)}{2}$
	 \end{hint}
	 
	 Check this formula by plugging in $n=1,2,3,4$ and see that it agrees with the answers you computed ``by hand'' above.
	 
	 \begin{multipleChoice}
	 	\choice[correct]{On my honor, I did check that the formula works for these values}
		\choice{I did not check that the formula works for these values}
	 \end{multipleChoice}
	 
	 You might also enjoy the following visualization of the formula:
	 
	 %BADBAD picture of n x n+1 array decomposed into two "steps" representing the sum 1+2+3+...+n
\end{question}




\begin{question}
	$\sum_{k=1}^{k=n-2} k = \answer{ \frac{(n-2)(n-2+1)}{2} }$
	\begin{hint}
		\sum_{k=1}^{k=n-2} k = \frac{(n-2)[(n-2)+1]}{2} 
	\end{hint}
\end{question}

\begin{question}
	$\sum_{k=4}^{k=n} k = \answer{ \frac{n(n+1)}{2} - 6}$
		\begin{hint}
			You could do this in many ways, for instance by reindexing the sum to start at $k=1$, or by thinking of this as $\sum_{k=1}^{k=n} k -\sum_{k=1}^{k=3} k $.
		\end{hint}
		\begin{hint}
			While the second way is probably easier, let us proceed by reindexing to get more practice with that technique.
			
			$\sum_{k=4}^{k=n} k  =\sum_{j=1}^{j=n-3} (j+3)$
		\end{hint}
		\begin{hint}
			\begin{align*}
				\sum_{k=4}^{k=n} k  &= \sum_{j=1}^{j=n-3} (j+3)\\
				&= \sum_{j=1}^{j=n-3} j + \sum_{j=1}^{j=n-3} 3\\
				&= \frac{(n-3)[(n-3)+1]}{2} + 3(n-3)\\
			\end{align*}
		\end{hint}
\end{question}

\begin{question}
	$\sum_{k=1}^{k=n} k+7  = \answer{ \frac{n(n+1)}{2} +7n}$
	\begin{hint}
		Breaking this into two sums, the first is $\frac{n(n+1)}{2}$ and the second is $7n$, so the answer is $\frac{n(n+1)}{2} +7n$.
	\end{hint}
\end{question}

\textbf{You should memorize the following formulas}. The first two we already understand, but deriving the last two is outside the scope of this class.  If you are curious about the last two, you may enjoy this or this.

\begin{theorem}
\begin{itemize}
	\item \sum_{k=1}^{k=n} 1 = n
	\item \sum_{k=1}^{k=n} k = \frac{n(n+1)}{2}
	\item \sum_{k=1}^{k=n} k^2 = \frac{n(n+1)(2n+1)}{6}
	\item \sum_{k=1}^{k=n} k^3 = \[\frac{n(n+1)}{2}\]^2
\end{theorem}

An interesting consequence of these identities is that, for instance, $(1+2+3+4+5)^2 = 1^3+2^3+3^3+4^3+5^3$, which should be really surprising!

\begin{question}
	$\sum_{k=1}^{k=8} (3k^2+2k+7) = \answer{740}$
	\begin{hint}
		\begin{align*}
			\sum_{k=1}^{k=8} (3k^2+2k+7) &= 3 \sum_{k=1}^{k=8} k^2 + 2 \sum_{k=1}^{k=8} k + 7   \sum_{k=1}^{k=8} 1\\
			&= 3 \frac{8(8+1)(2\cdot 8+1)}{6} + 2 \frac{8(8+1)}{2} + 7(8)\\
			&= 740
		\end{align*}
	\end{hint}
\end{question}

\begin{question}
	$\sum_{k=9}^{k=200} k^3 = \answer{404008704}$
	
	\begin{hint}
		 $\sum_{k=9}^{k=200} k^3 =  \sum_{k=1}^{k=200} k^3 -  \sum_{k=1}^{k=8} k^3  $
	\end{hint}
	\begin{hint}
		$= \frac{200*201*401}{6}-\frac{8(8+1)(2\cdot 8+1)}{6} = 404008704$
	\end{hint}
\end{question}

