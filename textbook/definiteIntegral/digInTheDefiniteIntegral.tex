\documentclass{ximera}

%\usepackage{todonotes}
%\usepackage{mathtools} %% Required for wide table Curl and Greens
%\usepackage{cuted} %% Required for wide table Curl and Greens
\newcommand{\todo}{}

\usepackage{esint} % for \oiint
\ifxake%%https://math.meta.stackexchange.com/questions/9973/how-do-you-render-a-closed-surface-double-integral
\renewcommand{\oiint}{{\large\bigcirc}\kern-1.56em\iint}
\fi


\graphicspath{
  {./}
  {ximeraTutorial/}
  {basicPhilosophy/}
  {functionsOfSeveralVariables/}
  {normalVectors/}
  {lagrangeMultipliers/}
  {vectorFields/}
  {greensTheorem/}
  {shapeOfThingsToCome/}
  {dotProducts/}
  {partialDerivativesAndTheGradientVector/}
  {../productAndQuotientRules/exercises/}
  {../normalVectors/exercisesParametricPlots/}
  {../continuityOfFunctionsOfSeveralVariables/exercises/}
  {../partialDerivativesAndTheGradientVector/exercises/}
  {../directionalDerivativeAndChainRule/exercises/}
  {../commonCoordinates/exercisesCylindricalCoordinates/}
  {../commonCoordinates/exercisesSphericalCoordinates/}
  {../greensTheorem/exercisesCurlAndLineIntegrals/}
  {../greensTheorem/exercisesDivergenceAndLineIntegrals/}
  {../shapeOfThingsToCome/exercisesDivergenceTheorem/}
  {../greensTheorem/}
  {../shapeOfThingsToCome/}
  {../separableDifferentialEquations/exercises/}
  {vectorFields/}
}

\newcommand{\mooculus}{\textsf{\textbf{MOOC}\textnormal{\textsf{ULUS}}}}

\usepackage{tkz-euclide}\usepackage{tikz}
\usepackage{tikz-cd}
\usetikzlibrary{arrows}
\tikzset{>=stealth,commutative diagrams/.cd,
  arrow style=tikz,diagrams={>=stealth}} %% cool arrow head
\tikzset{shorten <>/.style={ shorten >=#1, shorten <=#1 } } %% allows shorter vectors

\usetikzlibrary{backgrounds} %% for boxes around graphs
\usetikzlibrary{shapes,positioning}  %% Clouds and stars
\usetikzlibrary{matrix} %% for matrix
\usepgfplotslibrary{polar} %% for polar plots
\usepgfplotslibrary{fillbetween} %% to shade area between curves in TikZ
\usetkzobj{all}
\usepackage[makeroom]{cancel} %% for strike outs
%\usepackage{mathtools} %% for pretty underbrace % Breaks Ximera
%\usepackage{multicol}
\usepackage{pgffor} %% required for integral for loops



%% http://tex.stackexchange.com/questions/66490/drawing-a-tikz-arc-specifying-the-center
%% Draws beach ball
\tikzset{pics/carc/.style args={#1:#2:#3}{code={\draw[pic actions] (#1:#3) arc(#1:#2:#3);}}}



\usepackage{array}
\setlength{\extrarowheight}{+.1cm}
\newdimen\digitwidth
\settowidth\digitwidth{9}
\def\divrule#1#2{
\noalign{\moveright#1\digitwidth
\vbox{\hrule width#2\digitwidth}}}





\newcommand{\RR}{\mathbb R}
\newcommand{\R}{\mathbb R}
\newcommand{\N}{\mathbb N}
\newcommand{\Z}{\mathbb Z}

\newcommand{\sagemath}{\textsf{SageMath}}


%\renewcommand{\d}{\,d\!}
\renewcommand{\d}{\mathop{}\!d}
\newcommand{\dd}[2][]{\frac{\d #1}{\d #2}}
\newcommand{\pp}[2][]{\frac{\partial #1}{\partial #2}}
\renewcommand{\l}{\ell}
\newcommand{\ddx}{\frac{d}{\d x}}

\newcommand{\zeroOverZero}{\ensuremath{\boldsymbol{\tfrac{0}{0}}}}
\newcommand{\inftyOverInfty}{\ensuremath{\boldsymbol{\tfrac{\infty}{\infty}}}}
\newcommand{\zeroOverInfty}{\ensuremath{\boldsymbol{\tfrac{0}{\infty}}}}
\newcommand{\zeroTimesInfty}{\ensuremath{\small\boldsymbol{0\cdot \infty}}}
\newcommand{\inftyMinusInfty}{\ensuremath{\small\boldsymbol{\infty - \infty}}}
\newcommand{\oneToInfty}{\ensuremath{\boldsymbol{1^\infty}}}
\newcommand{\zeroToZero}{\ensuremath{\boldsymbol{0^0}}}
\newcommand{\inftyToZero}{\ensuremath{\boldsymbol{\infty^0}}}



\newcommand{\numOverZero}{\ensuremath{\boldsymbol{\tfrac{\#}{0}}}}
\newcommand{\dfn}{\textbf}
%\newcommand{\unit}{\,\mathrm}
\newcommand{\unit}{\mathop{}\!\mathrm}
\newcommand{\eval}[1]{\bigg[ #1 \bigg]}
\newcommand{\seq}[1]{\left( #1 \right)}
\renewcommand{\epsilon}{\varepsilon}
\renewcommand{\phi}{\varphi}


\renewcommand{\iff}{\Leftrightarrow}

\DeclareMathOperator{\arccot}{arccot}
\DeclareMathOperator{\arcsec}{arcsec}
\DeclareMathOperator{\arccsc}{arccsc}
\DeclareMathOperator{\si}{Si}
\DeclareMathOperator{\scal}{scal}
\DeclareMathOperator{\sign}{sign}


%% \newcommand{\tightoverset}[2]{% for arrow vec
%%   \mathop{#2}\limits^{\vbox to -.5ex{\kern-0.75ex\hbox{$#1$}\vss}}}
\newcommand{\arrowvec}[1]{{\overset{\rightharpoonup}{#1}}}
%\renewcommand{\vec}[1]{\arrowvec{\mathbf{#1}}}
\renewcommand{\vec}[1]{{\overset{\boldsymbol{\rightharpoonup}}{\mathbf{#1}}}\hspace{0in}}

\newcommand{\point}[1]{\left(#1\right)} %this allows \vector{ to be changed to \vector{ with a quick find and replace
\newcommand{\pt}[1]{\mathbf{#1}} %this allows \vec{ to be changed to \vec{ with a quick find and replace
\newcommand{\Lim}[2]{\lim_{\point{#1} \to \point{#2}}} %Bart, I changed this to point since I want to use it.  It runs through both of the exercise and exerciseE files in limits section, which is why it was in each document to start with.

\DeclareMathOperator{\proj}{\mathbf{proj}}
\newcommand{\veci}{{\boldsymbol{\hat{\imath}}}}
\newcommand{\vecj}{{\boldsymbol{\hat{\jmath}}}}
\newcommand{\veck}{{\boldsymbol{\hat{k}}}}
\newcommand{\vecl}{\vec{\boldsymbol{\l}}}
\newcommand{\uvec}[1]{\mathbf{\hat{#1}}}
\newcommand{\utan}{\mathbf{\hat{t}}}
\newcommand{\unormal}{\mathbf{\hat{n}}}
\newcommand{\ubinormal}{\mathbf{\hat{b}}}

\newcommand{\dotp}{\bullet}
\newcommand{\cross}{\boldsymbol\times}
\newcommand{\grad}{\boldsymbol\nabla}
\newcommand{\divergence}{\grad\dotp}
\newcommand{\curl}{\grad\cross}
%\DeclareMathOperator{\divergence}{divergence}
%\DeclareMathOperator{\curl}[1]{\grad\cross #1}
\newcommand{\lto}{\mathop{\longrightarrow\,}\limits}

\renewcommand{\bar}{\overline}

\colorlet{textColor}{black}
\colorlet{background}{white}
\colorlet{penColor}{blue!50!black} % Color of a curve in a plot
\colorlet{penColor2}{red!50!black}% Color of a curve in a plot
\colorlet{penColor3}{red!50!blue} % Color of a curve in a plot
\colorlet{penColor4}{green!50!black} % Color of a curve in a plot
\colorlet{penColor5}{orange!80!black} % Color of a curve in a plot
\colorlet{penColor6}{yellow!70!black} % Color of a curve in a plot
\colorlet{fill1}{penColor!20} % Color of fill in a plot
\colorlet{fill2}{penColor2!20} % Color of fill in a plot
\colorlet{fillp}{fill1} % Color of positive area
\colorlet{filln}{penColor2!20} % Color of negative area
\colorlet{fill3}{penColor3!20} % Fill
\colorlet{fill4}{penColor4!20} % Fill
\colorlet{fill5}{penColor5!20} % Fill
\colorlet{gridColor}{gray!50} % Color of grid in a plot

\newcommand{\surfaceColor}{violet}
\newcommand{\surfaceColorTwo}{redyellow}
\newcommand{\sliceColor}{greenyellow}




\pgfmathdeclarefunction{gauss}{2}{% gives gaussian
  \pgfmathparse{1/(#2*sqrt(2*pi))*exp(-((x-#1)^2)/(2*#2^2))}%
}


%%%%%%%%%%%%%
%% Vectors
%%%%%%%%%%%%%

%% Simple horiz vectors
\renewcommand{\vector}[1]{\left\langle #1\right\rangle}


%% %% Complex Horiz Vectors with angle brackets
%% \makeatletter
%% \renewcommand{\vector}[2][ , ]{\left\langle%
%%   \def\nextitem{\def\nextitem{#1}}%
%%   \@for \el:=#2\do{\nextitem\el}\right\rangle%
%% }
%% \makeatother

%% %% Vertical Vectors
%% \def\vector#1{\begin{bmatrix}\vecListA#1,,\end{bmatrix}}
%% \def\vecListA#1,{\if,#1,\else #1\cr \expandafter \vecListA \fi}

%%%%%%%%%%%%%
%% End of vectors
%%%%%%%%%%%%%

%\newcommand{\fullwidth}{}
%\newcommand{\normalwidth}{}



%% makes a snazzy t-chart for evaluating functions
%\newenvironment{tchart}{\rowcolors{2}{}{background!90!textColor}\array}{\endarray}

%%This is to help with formatting on future title pages.
\newenvironment{sectionOutcomes}{}{}



%% Flowchart stuff
%\tikzstyle{startstop} = [rectangle, rounded corners, minimum width=3cm, minimum height=1cm,text centered, draw=black]
%\tikzstyle{question} = [rectangle, minimum width=3cm, minimum height=1cm, text centered, draw=black]
%\tikzstyle{decision} = [trapezium, trapezium left angle=70, trapezium right angle=110, minimum width=3cm, minimum height=1cm, text centered, draw=black]
%\tikzstyle{question} = [rectangle, rounded corners, minimum width=3cm, minimum height=1cm,text centered, draw=black]
%\tikzstyle{process} = [rectangle, minimum width=3cm, minimum height=1cm, text centered, draw=black]
%\tikzstyle{decision} = [trapezium, trapezium left angle=70, trapezium right angle=110, minimum width=3cm, minimum height=1cm, text centered, draw=black]


\outcome{}

\title[Dig-In:]{The definite integral}

\begin{document}
\begin{abstract}
  Definite integrals compute signed area.
\end{abstract}
\maketitle


Definite integrals, often simply called integrals, compute signed area. 

\begin{definition}\index{integral}\index{definite integral}
The \dfn{definite integral}
\[
\int_a^b f(x) \d x
\]
computes the signed area between $y=f(x)$ and the $x$-axis on the
interval $[a,b]$.
\begin{itemize}
  \item If the region is above the $x$-axis, then the area has
    positive sign.
  \item If the region is below the $x$-axis, then the area has
    negative sign.
\end{itemize}
Note, when working with signed area, ``positive'' and ``negative''
area cancel each other out.
\end{definition}

\begin{question}
Consider the following graph of $y=f(x)$:
\begin{image}
  \begin{tikzpicture}
    \begin{axis}[
        width=6in,
        height=3in,
        xmin=-.5, xmax=5.5,ymin=-1.2,ymax=1.2,domain=0:6,
        axis lines =center, xlabel=$x$, ylabel=$y$,
        every axis y label/.style={at=(current axis.above origin),anchor=south},
        every axis x label/.style={at=(current axis.right of origin),anchor=west},
        axis on top,
    ] 
      \addplot [draw=none, fill=fillp,domain=0:1] {x} \closedcycle;
      \addplot [draw=none, fill=fillp,domain=1:5] {1.5-x/2} \closedcycle;
      
      \addplot [penColor,very thick,domain=0:1] {x};
      \addplot [penColor,very thick,domain=1:5] {1.5-x/2};
  \end{axis}
  \end{tikzpicture}
\end{image}
Compute:
\begin{enumerate}
\item $\int_0^3 f(x) \d x \begin{prompt}= \answer{1.5}\end{prompt}$
\item $\int_3^5 f(x) \d x \begin{prompt}= \answer{-1}\end{prompt}$
\item $\int_0^5 f(x) \d x \begin{prompt}= \answer{0.5}\end{prompt}$
\item $\int_0^3 5\cdot f(x) \d x \begin{prompt}= \answer{7.5}\end{prompt}$
\item $\int_1^1 5\cdot f(x) \d x \begin{prompt}= \answer{0}\end{prompt}$
\end{enumerate}
\begin{hint}
  Use the formula for the area of a triangle.
\end{hint}
\begin{hint}
  Remember, we are dealing with ``signed'' area here:
  \begin{image}
  \begin{tikzpicture}
    \begin{axis}[
        width=6in,
        height=3in,
        xmin=-.5, xmax=5.5,ymin=-1.2,ymax=1.2,domain=0:6,
        axis lines =center, xlabel=$x$, ylabel=$y$,
        every axis y label/.style={at=(current axis.above origin),anchor=south},
        every axis x label/.style={at=(current axis.right of origin),anchor=west},
        axis on top,
    ] 
      \addplot [draw=none, fill=fillp,domain=0:1] {x} \closedcycle;
      \addplot [draw=none, fill=fillp,domain=1:3] {1.5-x/2} \closedcycle;
      \addplot [draw=none, fill=filln,domain=3:5] {1.5-x/2} \closedcycle;
      
      \addplot [penColor,very thick,domain=0:1] {x};
      \addplot [penColor,very thick,domain=1:5] {1.5-x/2};
      \node at (axis cs:1.1,.4) [textColor] {\scalebox{2}{$\boldsymbol+$}};
      \node at (axis cs:4.5,-.4) [textColor] {\scalebox{2}{$\boldsymbol-$}};
  \end{axis}
  \end{tikzpicture}
\end{image}
\end{hint}
\end{question}

\begin{theorem}[Properties of the definite integral]
Let $f$ and $g$ be defined on a closed interval $[a,b]$ that contains the
value $c$, and let $k$ be a constant. The following
hold:
\begin{enumerate}
\item $\int_a^a f(x)\d x = 0$
\item $\int_a^b f(x)\d x + \int_b^c f(x)\d x = \int_a^cf(x)\d x$
\item $\int_a^bf(x)\d x = -\int_b^a f(x)\d x$
\item $\int_a^b f(x)\pm g(x)\d x = \int_a^bf(x)\d x \pm \int_a^bg(x)\d x$
\item $\int_a^bk\cdot f(x)\d x = k\cdot\int_a^bf(x)\d x$
\end{enumerate}
\end{theorem}

\end{document}







\begin{solution}
The definite integral $\int_0^3 x \d x$ measures signed area of the
shaded region shown in figure~\ref{figure:intfirst}. Since this region
is a triangle, we can use the formula for the area of the triangle to
compute
\[
\int_0^3 x \d x = \frac{1}{2} 3\cdot 3 = 9/2.
\]
\end{solution}



\begin{example} Compute
\[
\int_{-1}^3 \lfloor x \rfloor \d x.
\]
\end{example}

\begin{marginfigure}[0in]
\begin{tikzpicture}
  \begin{axis}[
      domain=-1:3,
      axis lines =middle, xlabel=$x$, ylabel=$y$,
      every axis y label/.style={at=(current axis.above origin),anchor=south},
      every axis x label/.style={at=(current axis.right of origin),anchor=west},
      clip=false,
      axis on top,
    ]
    \addplot [draw=none, fill=filln, domain=(-1:0)] {-1} \closedcycle;
    \addplot [draw=none, fill=fillp, domain=(0:1)] {0} \closedcycle;
    \addplot [draw=none, fill=fillp, domain=(1:2)] {1} \closedcycle;
    \addplot [draw=none, fill=fillp, domain=(2:3)] {2} \closedcycle;
    \addplot [very thick, penColor, domain=(-1:0)] {-1};
    \addplot [very thick, penColor, domain=(0:1)] {0};
    \addplot [very thick, penColor, domain=(1:2)] {1};
    \addplot [very thick, penColor, domain=(2:3)] {2};
    \addplot[color=penColor,fill=penColor,only marks,mark=*] coordinates{(-1,-1)};  %% closed hole          
    \addplot[color=penColor,fill=penColor,only marks,mark=*] coordinates{(0,0)};  %% closed hole          
    \addplot[color=penColor,fill=penColor,only marks,mark=*] coordinates{(1,1)};  %% closed hole          
    \addplot[color=penColor,fill=penColor,only marks,mark=*] coordinates{(2,2)};  %% closed hole  
    \addplot[color=penColor,fill=penColor,only marks,mark=*] coordinates{(3,3)};  %% closed hole                  
    \addplot[color=penColor,fill=background,only marks,mark=*] coordinates{(0,-1)};  %% open hole
    \addplot[color=penColor,fill=background,only marks,mark=*] coordinates{(1,0)};  %% open hole
    \addplot[color=penColor,fill=background,only marks,mark=*] coordinates{(2,1)};  %% open hole
    \addplot[color=penColor,fill=background,only marks,mark=*] coordinates{(3,2)};  %% open hole

    \node at (axis cs:-.5,-.5) [textColor] {\scalebox{2}{$\boldsymbol-$}};
    \node at (axis cs:2,.5) [textColor] {\scalebox{2}{$\boldsymbol+$}};
  \end{axis}
\end{tikzpicture}
\caption{The integral $\int_{-1}^3 \lfloor x\rfloor \d x$ measures the
  shaded area. Area above the $x$-axis has positive sign and the area below the $x$-axis has negative sign.}
\label{plot:int-greatist-integer}
\end{marginfigure}

\begin{solution}
The definite integral $\int_{-1}^3 \lfloor x\rfloor \d x$ measures
signed area of the shaded region shown in
figure~\ref{plot:int-greatist-integer}. We see that 
\[
\int_{-1}^3 \lfloor x\rfloor \d x = 
\int_{-1}^0 \lfloor x\rfloor \d x + \int_{0}^1 \lfloor x\rfloor \d x + \int_{1}^2 \lfloor x\rfloor \d x + \int_{2}^3 \lfloor x\rfloor \d x.
\]
So computing each of these areas separately
\begin{align*}
\int_{-1}^3 \lfloor x\rfloor \d x &= -1 + 0 + 1+2 \\
&= 2.
\end{align*}
\end{solution}

Our previous examples hopefully give us enough insight that this next
theorem is unsurprising.

\begin{mainTheorem}[Properties of Definite Integrals]
\begin{enumerate}
\item $\int_a^b k \d x= kb-ka$, where $k$ is a constant.
\item $\int_a^b \left( f(x) + g(x) \right) \d x = \int_a^b f(x) \d x + \int_a^b
  g(x) \d x$.
\item $\int_a^b k \cdot f(x) \d x = k \int_a^b f(x) \d x$.
\end{enumerate}
\end{mainTheorem}
Each of these properties follows from the notion that definite
integrals compute signed area.



\noindent For the following exercises consider the plot of $f(x)$
shown in Figure~\ref{figure:intExerPlot1}.
\begin{marginfigure}
\begin{tikzpicture}
  \begin{axis}[
      xmin=-2.5, xmax=2.5,ymin=-1,ymax=1,domain=-2.2:2.2,
      axis lines =center, xlabel=$x$, ylabel=$y$,
      every axis y label/.style={at=(current axis.above origin),anchor=south},
      every axis x label/.style={at=(current axis.right of origin),anchor=west},
      axis on top,
    ] 
    \addplot [penColor,very thick,smooth] {sin(deg(x))*sin(deg(x^2/1.3))};
  \end{axis}
\end{tikzpicture}
\caption{A plot of $f(x)$.}
\label{figure:intExerPlot1}
\end{marginfigure}
\twocol
\begin{exercise}
Is $\int_1^2 f(x) \d x$ \\
positive, negative, or zero?
\begin{answer}
positive
\end{answer}
\end{exercise}

\begin{exercise}
Is $\int_{-1}^0 f(x) \d x$ \\
positive, negative, or zero?
\begin{answer}
negative
\end{answer}
\end{exercise}

\begin{exercise}
Is $\int_{-1}^1 f(x) \d x$ \\
positive, negative, or zero?
\begin{answer}
zero
\end{answer}
\end{exercise}

\begin{exercise}
Is $\int_{-1}^2 f(x) \d x$ \\
positive, negative, or zero?
\begin{answer}
positive
\end{answer}
\end{exercise}

\endtwocol

\noindent For the following exercises use the plot of $g(x)$ shown in
Figure~\ref{plot:intExerPlot2} to compute the integrals below.
\begin{marginfigure}[0in]
\begin{tikzpicture}
  \begin{axis}[
      domain=-2:3,
      axis lines =middle, xlabel=$x$, ylabel=$y$,
      every axis y label/.style={at=(current axis.above origin),anchor=south},
      every axis x label/.style={at=(current axis.right of origin),anchor=west},
      clip=false,
      grid=both,
      grid style={dashed, gridColor},
    ]
    \addplot [penColor, very thick] plot coordinates {
      (-2,1) (-1,1) (0,-2) (2,2) (3,0)};
  \end{axis}
\end{tikzpicture}
\caption{A plot of $g(x)$.}
\label{plot:intExerPlot2}
\end{marginfigure}

\twocol

\begin{exercise}
$\int_{-2}^{-1} g(x) \d x$
\begin{answer}
$1$
\end{answer}
\end{exercise}

\begin{exercise}
$\int_1^3 g(x) \d x$
\begin{answer}
$2$
\end{answer}
\end{exercise}

\begin{exercise}
$\int_0^3 g(x) \d x$
\begin{answer}
$1$
\end{answer}
\end{exercise}

\begin{exercise}
$\int_{-1}^3 g(x) \d x$
\begin{answer}
$1/2$
\end{answer}
\end{exercise}






\end{document}
