\documentclass{ximera}

%\usepackage{todonotes}
%\usepackage{mathtools} %% Required for wide table Curl and Greens
%\usepackage{cuted} %% Required for wide table Curl and Greens
\newcommand{\todo}{}

\usepackage{esint} % for \oiint
\ifxake%%https://math.meta.stackexchange.com/questions/9973/how-do-you-render-a-closed-surface-double-integral
\renewcommand{\oiint}{{\large\bigcirc}\kern-1.56em\iint}
\fi


\graphicspath{
  {./}
  {ximeraTutorial/}
  {basicPhilosophy/}
  {functionsOfSeveralVariables/}
  {normalVectors/}
  {lagrangeMultipliers/}
  {vectorFields/}
  {greensTheorem/}
  {shapeOfThingsToCome/}
  {dotProducts/}
  {partialDerivativesAndTheGradientVector/}
  {../productAndQuotientRules/exercises/}
  {../normalVectors/exercisesParametricPlots/}
  {../continuityOfFunctionsOfSeveralVariables/exercises/}
  {../partialDerivativesAndTheGradientVector/exercises/}
  {../directionalDerivativeAndChainRule/exercises/}
  {../commonCoordinates/exercisesCylindricalCoordinates/}
  {../commonCoordinates/exercisesSphericalCoordinates/}
  {../greensTheorem/exercisesCurlAndLineIntegrals/}
  {../greensTheorem/exercisesDivergenceAndLineIntegrals/}
  {../shapeOfThingsToCome/exercisesDivergenceTheorem/}
  {../greensTheorem/}
  {../shapeOfThingsToCome/}
  {../separableDifferentialEquations/exercises/}
  {vectorFields/}
}

\newcommand{\mooculus}{\textsf{\textbf{MOOC}\textnormal{\textsf{ULUS}}}}

\usepackage{tkz-euclide}\usepackage{tikz}
\usepackage{tikz-cd}
\usetikzlibrary{arrows}
\tikzset{>=stealth,commutative diagrams/.cd,
  arrow style=tikz,diagrams={>=stealth}} %% cool arrow head
\tikzset{shorten <>/.style={ shorten >=#1, shorten <=#1 } } %% allows shorter vectors

\usetikzlibrary{backgrounds} %% for boxes around graphs
\usetikzlibrary{shapes,positioning}  %% Clouds and stars
\usetikzlibrary{matrix} %% for matrix
\usepgfplotslibrary{polar} %% for polar plots
\usepgfplotslibrary{fillbetween} %% to shade area between curves in TikZ
\usetkzobj{all}
\usepackage[makeroom]{cancel} %% for strike outs
%\usepackage{mathtools} %% for pretty underbrace % Breaks Ximera
%\usepackage{multicol}
\usepackage{pgffor} %% required for integral for loops



%% http://tex.stackexchange.com/questions/66490/drawing-a-tikz-arc-specifying-the-center
%% Draws beach ball
\tikzset{pics/carc/.style args={#1:#2:#3}{code={\draw[pic actions] (#1:#3) arc(#1:#2:#3);}}}



\usepackage{array}
\setlength{\extrarowheight}{+.1cm}
\newdimen\digitwidth
\settowidth\digitwidth{9}
\def\divrule#1#2{
\noalign{\moveright#1\digitwidth
\vbox{\hrule width#2\digitwidth}}}





\newcommand{\RR}{\mathbb R}
\newcommand{\R}{\mathbb R}
\newcommand{\N}{\mathbb N}
\newcommand{\Z}{\mathbb Z}

\newcommand{\sagemath}{\textsf{SageMath}}


%\renewcommand{\d}{\,d\!}
\renewcommand{\d}{\mathop{}\!d}
\newcommand{\dd}[2][]{\frac{\d #1}{\d #2}}
\newcommand{\pp}[2][]{\frac{\partial #1}{\partial #2}}
\renewcommand{\l}{\ell}
\newcommand{\ddx}{\frac{d}{\d x}}

\newcommand{\zeroOverZero}{\ensuremath{\boldsymbol{\tfrac{0}{0}}}}
\newcommand{\inftyOverInfty}{\ensuremath{\boldsymbol{\tfrac{\infty}{\infty}}}}
\newcommand{\zeroOverInfty}{\ensuremath{\boldsymbol{\tfrac{0}{\infty}}}}
\newcommand{\zeroTimesInfty}{\ensuremath{\small\boldsymbol{0\cdot \infty}}}
\newcommand{\inftyMinusInfty}{\ensuremath{\small\boldsymbol{\infty - \infty}}}
\newcommand{\oneToInfty}{\ensuremath{\boldsymbol{1^\infty}}}
\newcommand{\zeroToZero}{\ensuremath{\boldsymbol{0^0}}}
\newcommand{\inftyToZero}{\ensuremath{\boldsymbol{\infty^0}}}



\newcommand{\numOverZero}{\ensuremath{\boldsymbol{\tfrac{\#}{0}}}}
\newcommand{\dfn}{\textbf}
%\newcommand{\unit}{\,\mathrm}
\newcommand{\unit}{\mathop{}\!\mathrm}
\newcommand{\eval}[1]{\bigg[ #1 \bigg]}
\newcommand{\seq}[1]{\left( #1 \right)}
\renewcommand{\epsilon}{\varepsilon}
\renewcommand{\phi}{\varphi}


\renewcommand{\iff}{\Leftrightarrow}

\DeclareMathOperator{\arccot}{arccot}
\DeclareMathOperator{\arcsec}{arcsec}
\DeclareMathOperator{\arccsc}{arccsc}
\DeclareMathOperator{\si}{Si}
\DeclareMathOperator{\scal}{scal}
\DeclareMathOperator{\sign}{sign}


%% \newcommand{\tightoverset}[2]{% for arrow vec
%%   \mathop{#2}\limits^{\vbox to -.5ex{\kern-0.75ex\hbox{$#1$}\vss}}}
\newcommand{\arrowvec}[1]{{\overset{\rightharpoonup}{#1}}}
%\renewcommand{\vec}[1]{\arrowvec{\mathbf{#1}}}
\renewcommand{\vec}[1]{{\overset{\boldsymbol{\rightharpoonup}}{\mathbf{#1}}}\hspace{0in}}

\newcommand{\point}[1]{\left(#1\right)} %this allows \vector{ to be changed to \vector{ with a quick find and replace
\newcommand{\pt}[1]{\mathbf{#1}} %this allows \vec{ to be changed to \vec{ with a quick find and replace
\newcommand{\Lim}[2]{\lim_{\point{#1} \to \point{#2}}} %Bart, I changed this to point since I want to use it.  It runs through both of the exercise and exerciseE files in limits section, which is why it was in each document to start with.

\DeclareMathOperator{\proj}{\mathbf{proj}}
\newcommand{\veci}{{\boldsymbol{\hat{\imath}}}}
\newcommand{\vecj}{{\boldsymbol{\hat{\jmath}}}}
\newcommand{\veck}{{\boldsymbol{\hat{k}}}}
\newcommand{\vecl}{\vec{\boldsymbol{\l}}}
\newcommand{\uvec}[1]{\mathbf{\hat{#1}}}
\newcommand{\utan}{\mathbf{\hat{t}}}
\newcommand{\unormal}{\mathbf{\hat{n}}}
\newcommand{\ubinormal}{\mathbf{\hat{b}}}

\newcommand{\dotp}{\bullet}
\newcommand{\cross}{\boldsymbol\times}
\newcommand{\grad}{\boldsymbol\nabla}
\newcommand{\divergence}{\grad\dotp}
\newcommand{\curl}{\grad\cross}
%\DeclareMathOperator{\divergence}{divergence}
%\DeclareMathOperator{\curl}[1]{\grad\cross #1}
\newcommand{\lto}{\mathop{\longrightarrow\,}\limits}

\renewcommand{\bar}{\overline}

\colorlet{textColor}{black}
\colorlet{background}{white}
\colorlet{penColor}{blue!50!black} % Color of a curve in a plot
\colorlet{penColor2}{red!50!black}% Color of a curve in a plot
\colorlet{penColor3}{red!50!blue} % Color of a curve in a plot
\colorlet{penColor4}{green!50!black} % Color of a curve in a plot
\colorlet{penColor5}{orange!80!black} % Color of a curve in a plot
\colorlet{penColor6}{yellow!70!black} % Color of a curve in a plot
\colorlet{fill1}{penColor!20} % Color of fill in a plot
\colorlet{fill2}{penColor2!20} % Color of fill in a plot
\colorlet{fillp}{fill1} % Color of positive area
\colorlet{filln}{penColor2!20} % Color of negative area
\colorlet{fill3}{penColor3!20} % Fill
\colorlet{fill4}{penColor4!20} % Fill
\colorlet{fill5}{penColor5!20} % Fill
\colorlet{gridColor}{gray!50} % Color of grid in a plot

\newcommand{\surfaceColor}{violet}
\newcommand{\surfaceColorTwo}{redyellow}
\newcommand{\sliceColor}{greenyellow}




\pgfmathdeclarefunction{gauss}{2}{% gives gaussian
  \pgfmathparse{1/(#2*sqrt(2*pi))*exp(-((x-#1)^2)/(2*#2^2))}%
}


%%%%%%%%%%%%%
%% Vectors
%%%%%%%%%%%%%

%% Simple horiz vectors
\renewcommand{\vector}[1]{\left\langle #1\right\rangle}


%% %% Complex Horiz Vectors with angle brackets
%% \makeatletter
%% \renewcommand{\vector}[2][ , ]{\left\langle%
%%   \def\nextitem{\def\nextitem{#1}}%
%%   \@for \el:=#2\do{\nextitem\el}\right\rangle%
%% }
%% \makeatother

%% %% Vertical Vectors
%% \def\vector#1{\begin{bmatrix}\vecListA#1,,\end{bmatrix}}
%% \def\vecListA#1,{\if,#1,\else #1\cr \expandafter \vecListA \fi}

%%%%%%%%%%%%%
%% End of vectors
%%%%%%%%%%%%%

%\newcommand{\fullwidth}{}
%\newcommand{\normalwidth}{}



%% makes a snazzy t-chart for evaluating functions
%\newenvironment{tchart}{\rowcolors{2}{}{background!90!textColor}\array}{\endarray}

%%This is to help with formatting on future title pages.
\newenvironment{sectionOutcomes}{}{}



%% Flowchart stuff
%\tikzstyle{startstop} = [rectangle, rounded corners, minimum width=3cm, minimum height=1cm,text centered, draw=black]
%\tikzstyle{question} = [rectangle, minimum width=3cm, minimum height=1cm, text centered, draw=black]
%\tikzstyle{decision} = [trapezium, trapezium left angle=70, trapezium right angle=110, minimum width=3cm, minimum height=1cm, text centered, draw=black]
%\tikzstyle{question} = [rectangle, rounded corners, minimum width=3cm, minimum height=1cm,text centered, draw=black]
%\tikzstyle{process} = [rectangle, minimum width=3cm, minimum height=1cm, text centered, draw=black]
%\tikzstyle{decision} = [trapezium, trapezium left angle=70, trapezium right angle=110, minimum width=3cm, minimum height=1cm, text centered, draw=black]


\title[Dig-In]{Maximums and minimums}



\outcome{Define a critical point.}
\outcome{Find critical points.}
\outcome{Define absolute maximum and absolute minimum.}
\outcome{Find the absolute max or min of a continuous function on a closed interval.}
\outcome{Define local maximum and local minimum.}
\outcome{Compare and contrast local and absolute maxima and minima.}
\outcome{Identify situations in which an absolute maximum or minimum is guaranteed.}
\outcome{Classify critical points.}
\outcome{State the First Derivative Test.}
\outcome{Apply the First Derivative Test.}
\outcome{State the Second Derivative Test.}
\outcome{Apply the Second Derivative Test.}
\outcome{Define inflection points.}
\outcome{Find inflection points.}
  


\begin{document}
\begin{abstract}
We use derivatives to help locate extrema.  
\end{abstract}
\maketitle


Whether we are interested in a function as a purely mathematical
object or in connection with some application to the real world, it is
often useful to know what the graph of the function looks like. We can
obtain a good picture of the graph using certain crucial information
provided by derivatives of the function.

\section{Extrema}

Local \textit{extrema} on a function are points on the graph where the
$y$ coordinate is larger (or smaller) than all other $y$ coordinates
on the graph at points ``close to'' $(x,y)$. 

\begin{definition}\hfil\index{maximum/minimum!local}
\begin{enumerate}
\item A function $f$ has a \dfn{local maximum} at $x=a$ if there is an
  open interval with $f(a)\ge f(x)$ for every $x$ in that interval.
\item A function $f$ has a \dfn{local minimum} at $x=a$ if there is an
  open interval with $f(a)\le f(x)$ for every $x$ in that interval.
\end{enumerate}
A \dfn{local extremum}\index{extremum!local} is either a local
maximum or a local minimum.
\end{definition}

\begin{question}
  True or false: ``All global extrema are also local extrema.''
  \begin{multipleChoice}
    \choice[correct]{true}
    \choice{false}
  \end{multipleChoice}
  \begin{feedback}
    All global extrema are local extrema.
  \end{feedback}
\end{question}

Local maximum and minimum points are quite distinctive on the graph of
a function, and are therefore useful in understanding the shape of the
graph. In many applied problems we want to find the largest or
smallest value that a function achieves (for example, we might want
to find the minimum cost at which some task can be performed) and so
identifying maximum and minimum points will be useful for applied
problems as well.



\section{Critical points}


If $(x,f(x))$ is a point where $f(x)$ reaches a local maximum or
minimum, and if the derivative of $f$ exists at $x$, then the graph
has a tangent line and the tangent line must be horizontal. This is
important enough to state as a theorem, though we will not prove it.

\begin{theorem}[Fermat's Theorem]\index{Fermat's Theorem}\label{theorem:fermat}
If $f$ has a local extremum at $x=a$ and $f$ is differentiable
at $a$, then $f'(a)=0$.
\end{theorem}
\begin{question}
  Does Fermat's Theorem say that if $f'(a) = 0$, then $f$ has a local
  extrema at $x=a$?
  \begin{multipleChoice}
    \choice{yes}
    \choice[correct]{no}
  \end{multipleChoice}
  \begin{feedback}
    Consider $f(x) = x^3$, $f'(0) = 0$, but $f$ does not have a local
    maximum or minimum at $x=0$.
  \end{feedback}
\end{question}


Fermat's Theorem says that the only points at which a function can
have a local maximum or minimum are points at which the derivative is
zero, consider the plots of $f(x) = x^3-4x^2+3x$ and $f'(x) =
3x^2-8x+3$,
\begin{image}
\begin{tikzpicture}
	\begin{axis}[
            domain=-3:3,
            ymax=3,
            ymin=-3,
            %samples=100,
            axis lines =middle, xlabel=$x$, ylabel=$y$,
            every axis y label/.style={at=(current axis.above origin),anchor=south},
            every axis x label/.style={at=(current axis.right of origin),anchor=west}
          ]
          \addplot [dashed, textColor, smooth] plot coordinates {(.451,0) (.451,.631)}; %% {.451};
          \addplot [dashed, textColor, smooth] plot coordinates {(2.215,-2.113) (2.215,0)}; %% axis{2.215};
          \addplot [very thick, penColor2, smooth] {3*x^2-8*x+3};
          \addplot [very thick, penColor, smooth] {x^3-4*x^2+3*x};
          \node at (axis cs:2.5,-2) [anchor=west] {\color{penColor}$f(x)$};  
          \node at (axis cs:.2,2) [anchor=west] {\color{penColor2}$f'(x)$};
          \addplot[color=penColor2,fill=penColor2,only marks,mark=*] coordinates{(.451,0)};  %% closed hole
          \addplot[color=penColor2,fill=penColor2,only marks,mark=*] coordinates{(2.215,0)};  %% closed hole
          \addplot[color=penColor,fill=penColor,only marks,mark=*] coordinates{(.451,.631)};  %% closed hole
          \addplot[color=penColor,fill=penColor,only marks,mark=*] coordinates{(2.215,-2.113)};  %% closed hole
        \end{axis}
\end{tikzpicture}
%% \caption{A plot of $f(x) = x^3-4x^2+3x$ and $f'(x) = 3x^2-8x+3$.}
%% \label{figure:x^3-4x^2+3x}
\end{image}
or the derivative is undefined, as in the plot of $f(x) = x^{2/3}$ and $f'(x) = \frac{2}{3x^{1/3}}$:
\begin{image}
\begin{tikzpicture}
	\begin{axis}[
            domain=-3:3,
            ymax=2,
            ymin=-2,
            axis lines =middle, xlabel=$x$, ylabel=$y$,
            every axis y label/.style={at=(current axis.above origin),anchor=south},
            every axis x label/.style={at=(current axis.right of origin),anchor=west}
          ]
          \addplot [very thick, penColor2, samples=100, smooth,domain=(-3:-.01)] {-(2/3)*abs(x)^(-1/3)};
          \addplot [very thick, penColor2, samples=100, smooth,domain=(.01:3)] {(2/3)*abs(x)^(-1/3)};
          \addplot [very thick, penColor, smooth,domain=(-3:-.01)] {abs(x)^(2/3)};
          \addplot [very thick, penColor, smooth,domain=(.01:3)] {x^(2/3)};         
          \node at (axis cs:-2,1.7) [anchor=west] {\color{penColor}$f(x)$};  
          \node at (axis cs:2,.7) [anchor=west] {\color{penColor2}$f'(x)$};
        \end{axis}
\end{tikzpicture}
%% \caption{A plot of $f(x) = x^{2/3}$ and $f'(x) = \frac{2}{3x^{1/3}}$.}
%% \label{figure:x^{2/3}}
\end{image}
This brings us to our next definition.

\begin{definition}\index{critical point}
  A function has a \dfn{critical point} at $x=a$ if 
  \[
  f'(a) = 0\qquad\text{or}\qquad \text{$f'(a)$ does not exist.}
  \]
\end{definition}

\begin{warning} 
When looking for local maximum and minimum points, you are likely to
make two sorts of mistakes: 
\begin{itemize}
\item You may forget that a maximum or minimum can occur where the
  derivative does not exist, and so forget to check whether the
  derivative exists everywhere. 
\item You might assume that any place that the derivative is zero is a
  local maximum or minimum point, but this is not true, consider the
  plots of $f(x) = x^3$ and $f'(x) = 3x^2$.
\begin{image}
\begin{tikzpicture}
	\begin{axis}[
            domain=-3:3,
            ymax=3,
            ymin=-3,
            axis lines =middle, xlabel=$x$, ylabel=$y$,
            every axis y label/.style={at=(current axis.above origin),anchor=south},
            every axis x label/.style={at=(current axis.right of origin),anchor=west}
          ]
          \addplot [very thick, penColor2, smooth] {3*x^2};
          \addplot [very thick, penColor, smooth] {x^3};         
          \node at (axis cs:1,.9) [anchor=west] {\color{penColor}$f(x)$};  
          \node at (axis cs:-.5,1) [anchor=west] {\color{penColor2}$f'(x)$};
        \end{axis}
\end{tikzpicture}
%% \caption{A plot of $f(x) = x^3$ and $f'(x) = 3x^2$. While $f'(0)=0$,
%%   there is neither a maximum nor minimum at $(0,f(0))$.}
%% \label{figure:x^3}
\end{image}
While $f'(0)=0$, there is neither a maximum nor minimum at $(0,f(0))$.
\end{itemize}
\end{warning}



Since the derivative is zero or undefined at both local maximum and
local minimum points, we need a way to determine which, if either,
actually occurs. The most elementary approach is to test directly
whether the $y$ coordinates near the potential maximum or minimum are
above or below the $y$ coordinate at the point of interest. 

It is not always easy to compute the value of a function at a
particular point. The task is made easier by the availability of
calculators and computers, but they have their own drawbacks---they do
not always allow us to distinguish between values that are very close
together. Nevertheless, because this method is conceptually simple and
sometimes easy to perform, you should always consider it.




\begin{example}
Find all local maximum and minimum points for the function 
$f(x)=x^3-x$. 
\begin{explanation} 
Write
\[
\ddx f(x)=\answer[given]{3x^2-1}.
\] 
This is defined everywhere and is zero at $x=\pm \sqrt{3}/3$. Looking
first at $x=\sqrt{3}/3$, we see that 
\[
f(\sqrt{3}/3)=\answer[given]{-2\sqrt{3}/9}.
\] 
Now we test two points on either side of $x=\sqrt{3}/3$, making sure
that neither is farther away than the nearest critical point; since
$\sqrt{3}<3$, $\sqrt{3}/3<1$ and we can use $x=0$ and $x=1$. Since
\[
f(0)=0>-2\sqrt{3}/9\qquad\text{and}\qquad f(1)=0>-2\sqrt{3}/9,
\] 
there must be a local minimum at $x=\answer[given]{\sqrt{3}/3}$.

For $x=-\sqrt{3}/3$, we see that $f(-\sqrt{3}/3)=2\sqrt{3}/9$. This
time we can use $x=0$ and $x=-1$, and we find that $f(-1)=f(0)=0<
2\sqrt{3}/9$, so there must be a local maximum at
$x=\answer[given]{-\sqrt{3}/3}$, see the plot below:
\begin{image}
\begin{tikzpicture}
	\begin{axis}[
            domain=-2:2,
            ymax=2,
            ymin=-2,
            %samples=100,
            axis lines =middle, xlabel=$x$, ylabel=$y$,
            every axis y label/.style={at=(current axis.above origin),anchor=south},
            every axis x label/.style={at=(current axis.right of origin),anchor=west}
          ]
          \addplot [dashed, textColor, smooth] plot coordinates {(-.577,0) (-.577,.385)}; %% {.451};
          \addplot [dashed, textColor, smooth] plot coordinates {(.577,-.385) (.577,0)}; %% axis{2.215};

          \addplot [very thick, penColor2, smooth] {3*x^2-1};
          \addplot [very thick, penColor, smooth] {x^3-x};

          \node at (axis cs:1,1) [anchor=west] {\color{penColor}$f$};  
          \node at (axis cs:-.75,1) [anchor=west] {\color{penColor2}$f'$};

          \addplot[color=penColor2,fill=penColor2,only marks,mark=*] coordinates{(-.577,0)};  %% closed hole
          \addplot[color=penColor2,fill=penColor2,only marks,mark=*] coordinates{(.577,0)};  %% closed hole
          \addplot[color=penColor,fill=penColor,only marks,mark=*] coordinates{(-.577,.385)};  %% closed hole
          \addplot[color=penColor,fill=penColor,only marks,mark=*] coordinates{(.577,-.385)};  %% closed hole
        \end{axis}
\end{tikzpicture}
%%\caption{A plot of $f(x) = x^3-x$ and $f'(x) = 3x^2-1$.}
%%\label{figure:x^3-x}
\end{image}
\end{explanation}
\end{example}






\section{The first derivative test}

The method of the previous section for deciding whether there is a
local maximum or minimum at a critical point by testing ``near-by''
points is not always convenient. Instead, since we have already had to
compute the derivative to find the critical points, we can use
information about the derivative to decide. Recall that
\begin{itemize}
\item If $f'(x) >0$ on an interval, then $f$ is increasing on that interval.
\item If $f'(x) <0$ on an interval, then $f$ is decreasing on that interval.
\end{itemize}

So how exactly does the derivative tell us whether there is a maximum,
minimum, or neither at a point? Use the \textit{first derivative test}.

\begin{theorem}[First Derivative Test]\index{first derivative test}\label{T:fdt}\hfil
Suppose that $f$ is continuous on an interval, and that $f'(a)=0$ for
some value of $a$ in that interval.
\begin{itemize}
\item If $f'(x)>0$ to the left of $a$ and $f'(x)<0$ to the right of
  $a$, then $f(a)$ is a local maximum.
\item If $f'(x)<0$ to the left of $a$ and $f'(x)>0$ to the right of
  $a$, then $f(a)$ is a local minimum.
\item If $f'(x)$ has the same sign to the left and right of $a$,
  then $f(a)$ is not a local extremum.
\end{itemize}
\end{theorem}

\begin{example}\label{E:localextrema}
Consider the function 
\[
f(x) = \frac{x^4}{4}+\frac{x^3}{3}-x^2
\]
Find the intervals on which $f(x)$ is increasing and decreasing and
identify the local extrema of $f(x)$.


\begin{explanation}
Start by computing
\[
\ddx f(x) = \answer[given]{x^3+x^2-2x}.
\]
Now we need to find when this function is positive and when it is
negative. To do this, solve 
\[
f'(x) = \answer[given]{x^3+x^2-2x} =0.
\]
Factor $f'(x)$
\begin{align*}
f'(x) &= \answer[given]{x^3+x^2-2x} \\
&=x(\answer[given]{x^2+x-2})\\
&=x(x+2)\answer[given]{(x-1)}.
\end{align*}
So the critical points (when $f'(x)=0$) are when $x=-2$, $x=0$, and
$x=1$. Now we can check points \textbf{between} the critical points to find
when $f'(x)$ is increasing and decreasing:
\[
f'(-3)=\answer[given]{-12} \qquad f'(.5)=\answer[given]{-0.625} \qquad f'(-1)=\answer[given]{2} \qquad f'(2)=\answer[given]{8}
\]
From this we can make a sign table:

\begin{image}
\begin{tikzpicture}
	\begin{axis}[
            trim axis left,
            scale only axis,
            domain=-3:3,
            ymax=2,
            ymin=-2,
            axis lines=none,
            height=3cm, %% Hard coded height! 
            width=\textwidth, %% width
          ]
          \addplot [draw=none, fill=fill1, domain=(-3:-2)] {2} \closedcycle;
          \addplot [draw=none, fill=fill2, domain=(-2:0)] {2} \closedcycle;
          \addplot [draw=none, fill=fill1, domain=(0:1)] {2} \closedcycle;
          \addplot [draw=none, fill=fill2, domain=(1:3)] {2} \closedcycle;
          
          \addplot [->,textColor] plot coordinates {(-3,0) (3,0)}; %% axis{0};
          
          \addplot [dashed, textColor] plot coordinates {(-2,0) (-2,2)};
          \addplot [dashed, textColor] plot coordinates {(0,0) (0,2)};
          \addplot [dashed, textColor] plot coordinates {(1,0) (1,2)};
          
          \node at (axis cs:-2,0) [anchor=north,textColor] {\footnotesize$-2$};
          \node at (axis cs:0,0) [anchor=north,textColor] {\footnotesize$0$};
          \node at (axis cs:1,0) [anchor=north,textColor] {\footnotesize$1$};

          \node at (axis cs:-2.5,1) [textColor] {\footnotesize$f'(x)<0$};
          \node at (axis cs:.5,1) [textColor] {\footnotesize$f'(x)<0$};
          \node at (axis cs:-1,1) [textColor] {\footnotesize$f'(x)>0$};
          \node at (axis cs:2,1) [textColor] {\footnotesize$f'(x)>0$};

          \node at (axis cs:-2.5,-.5) [anchor=north,textColor] {\footnotesize Decreasing};
          \node at (axis cs:.5,-.5) [anchor=north,textColor] {\footnotesize Decreasing};
          \node at (axis cs:-1,-.5) [anchor=north,textColor] {\footnotesize Increasing};
          \node at (axis cs:2,-.5) [anchor=north,textColor] {\footnotesize Increasing};

        \end{axis}
\end{tikzpicture}
\end{image}

Hence $f$ is increasing on $(-2,0)\cup(1,\infty)$ and $f(x)$ is
decreasing on $(-\infty,-2)\cup(0,1)$. Moreover, from the first
derivative test, the local maximum is at $x=0$ while the local minima
are at $x=-2$ and $x=1$, see the graphs of of $f(x) =x^4/4 + x^3/3
-x^2$ and $f'(x) = x^3 + x^2 -2x$.
\begin{image}
\begin{tikzpicture}
	\begin{axis}[
            domain=-4:4,
            ymax=5,
            ymin=-5,
            %samples=100,
            axis lines =middle, xlabel=$x$, ylabel=$y$,
            every axis y label/.style={at=(current axis.above origin),anchor=south},
            every axis x label/.style={at=(current axis.right of origin),anchor=west}
          ]
          \addplot [dashed, textColor, smooth] plot coordinates {(-2,0) (-2,-2.667)}; %% {.451};
          \addplot [dashed, textColor, smooth] plot coordinates {(1,0) (1,-.4167)}; %% axis{2.215};

          \addplot [very thick, penColor, smooth] {(x^4)/4 + (x^3)/3 -x^2};
          \addplot [very thick, penColor2, smooth] {x^3 + x^2 -2*x};

          \node at (axis cs:-1.3,-2) [anchor=west] {\color{penColor}$f(x)$};  
          \node at (axis cs:-2.1,2) [anchor=west] {\color{penColor2}$f'(x)$};

          \addplot[color=penColor2,fill=penColor2,only marks,mark=*] coordinates{(-2,0)};  %% closed hole
          \addplot[color=penColor2,fill=penColor2,only marks,mark=*] coordinates{(1,0)};  %% closed hole
          \addplot[color=penColor2,fill=penColor3,only marks,mark=*] coordinates{(0,0)};  %% closed hole
          \addplot[color=penColor,fill=penColor,only marks,mark=*] coordinates{(-2,.-2.667)};  %% closed hole
          \addplot[color=penColor,fill=penColor,only marks,mark=*] coordinates{(1,-.4167)};  %% closed hole
        \end{axis}
\end{tikzpicture}
\end{image}
\end{explanation}
\end{example}


Hence we have seen that if $f'$ is zero and increasing at a point,
then $f$ has a local minimum at the point. If $f'$ is zero and
decreasing at a point then $f$ has a local maximum at the
point. Thus, we see that we can gain information about $f$ by
studying how $f'$ changes. This leads us to our next section.








\section{Inflection points}


If we are trying to understand the shape of the graph of a function,
knowing where it is concave up and concave down helps us to get a more
accurate picture. It is worth summarizing what we have seen already in
to a single theorem.

\begin{theorem}[Test for Concavity]\index{concavity test}
Suppose that $f''(x)$ exists on an interval.
\begin{enumerate}
\item If $f''(x)>0$ on an interval, then $f$ is concave up on that interval.
\item If $f''(x)<0$ on an interval, then $f$ is concave down on that interval.
\end{enumerate}
\end{theorem}


Of particular interest are points at which the concavity changes from
up to down or down to up. 

\begin{definition}\index{inflection point}
If $f$ is continuous and its concavity changes either from up to down
or down to up at $x=a$, then $f$ has an \dfn{inflection point} at
$x=a$.
\end{definition}

It is instructive to see some examples and nonexamples of inflection
points.
\begin{center}
\begin{tabular}{cccc}
\begin{tikzpicture}
	\begin{axis}[
            domain=0:2,
            ymax=2,
            height=7cm,
            width=2in,
            ymin=-1,
            axis lines=none,
            clip=false,
          ]
          \addplot [very thick, penColor, smooth, domain=(0:1)] {(x-1)^2+1};
          \addplot [very thick, penColor, smooth, domain=(1:2)] {-(x-1)^2+1};
          \addplot[color=penColor,fill=penColor,only marks,mark=*] coordinates{(1,1)};
          \node at (axis cs:1,-.5) [text width=2in] {This is an inflection point. The concavity changes from concave up to concave down.};
        \end{axis}
\end{tikzpicture}

&

\begin{tikzpicture}
	\begin{axis}[
            height=7cm,
            width=2in,
            domain=0:2,
            ymax=1,
            ymin=-2,
            axis lines=none,
            clip=false,
          ]
          \addplot [very thick, penColor2, smooth] {-(x-1)^2+.75};
          \addplot[color=penColor2,fill=penColor2,only marks,mark=*] coordinates{(1,.75)};
          \node at (axis cs:1,-1) [text width=2in] {This is \textbf{not} an inflection point. The curve is concave down on either side of the point.};
        \end{axis}
\end{tikzpicture} 

&

\begin{tikzpicture}
	\begin{axis}[
            height=7cm,
            domain=0:2,
            width=2in,
            ymax=2,
            ymin=-1,
            samples=100,
            axis lines=none,
            clip=false,
          ]
          \addplot [very thick, penColor, smooth,domain=(1:2)] {sqrt(x-1)+1};
          \addplot [very thick, penColor, smooth,domain=(0:1)] {-sqrt(abs(1-x))+1};
          \addplot[color=penColor,fill=penColor,only marks,mark=*] coordinates{(1,1)};
          \node at (axis cs:1,-.5) [text width=2in] {This is an inflection point. The concavity changes from concave up to concave down.};
        \end{axis}
\end{tikzpicture}

&

\begin{tikzpicture}
	\begin{axis}[
            height=7cm,
            width=2in,
            domain=0:2,
            ymax=2,
            ymin=-1,
            axis lines=none,
            clip=false,
          ]
          \addplot [very thick, penColor2, smooth,domain=(1:2)] {sqrt(x-1)+.5};
          \addplot [very thick, penColor2, smooth,domain=(0:1)] {sqrt(abs(1-x))+.5};
          \addplot[color=penColor2,fill=penColor2,only marks,mark=*] coordinates{(1,.5)};
          \node at (axis cs:1,-.5) [text width=2in] {This is \textbf{not} an inflection point. The curve is concave down on either side of the point.};
        \end{axis}
\end{tikzpicture}
\end{tabular}
\end{center}

We identify inflection points by first finding where $f''(x)$ is zero
or undefined and then checking to see whether $f''(x)$ does in fact go
from positive to negative or negative to positive at these points.

\begin{warning}
Even if $f''(a) = 0$, the point determined by $x=a$ might \textbf{not}
be an inflection point.
\end{warning}


\begin{example}
Describe the concavity of $f(x)=x^3-x$. 

\begin{explanation}
To start, compute the first and second derivative of $f(x)$ with
respect to $x$,
\[
f'(x)=\answer[given]{3x^2-1}\qquad\text{and}\qquad f''(x)=\answer[given]{6x}.
\]
Since $f''(0)=0$, there is potentially an inflection point at
zero. Since $f''(x)>0$ when $x>0$ and $f''(x)<0$ when $x<0$ the
concavity does change from down to up at zero---there is an inflection
point at $x=0$. The curve is concave down for all $x<0$ and concave up
for all $x>0$, see the graphs of $f(x) = x^3-x$ and $f''(x) = 6x$.
\begin{image}
\begin{tikzpicture}
	\begin{axis}[
            domain=-3:3,
            ymax=3,
            ymin=-3,
            axis lines =middle, xlabel=$x$, ylabel=$y$,
            every axis y label/.style={at=(current axis.above origin),anchor=south},
            every axis x label/.style={at=(current axis.right of origin),anchor=west}
          ]
          \addplot [very thick, penColor, smooth] {x^3-x};
          \addplot [very thick, penColor4, smooth] {6*x};         
          \node at (axis cs:-.75,.6) [anchor=west] {\color{penColor}$f(x)$};  
          \node at (axis cs:.2,1) [anchor=west] {\color{penColor4}$f''(x)$};
          \addplot[color=penColor4!50!penColor,fill=penColor4!50!penColor,only marks,mark=*] coordinates{(0,0)};  %% closed hole
        \end{axis}
\end{tikzpicture}
%% \caption{A plot of $f(x) = x^3-x$ and $f''(x) = 6x$. We can see that
%%   the concavity change at $x=0$.}
%% \label{figure:3x^2-1}
%% \end{marginfigure}
\end{image}
\end{explanation}
\end{example}


Note that we need to compute and analyze the second derivative to
understand concavity, so we may as well try to use the second
derivative test for maxima and minima. If for some reason this fails
we can then try one of the other tests.

\section{The second derivative test}


Recall the first derivative test:
\begin{itemize}
\item If $f'(x)>0$ to the left of $a$ and $f'(x)<0$ to the right of
  $a$, then $f(a)$ is a local maximum.
\item If $f'(x)<0$ to the left of $a$ and $f'(x)>0$ to the right of
  $a$, then $f(a)$ is a local minimum.
\end{itemize}

If $f'$ changes from positive to negative it is decreasing. In this
case, $f''$ might be negative, and if in fact $f''$ is negative
then $f'$ is definitely decreasing, so there is a local maximum at
the point in question. On the other hand, if $f'$ changes from
negative to positive it is increasing. Again, this means that
$f''$ might be positive, and if in fact $f''$ is positive then
$f'$ is definitely increasing, so there is a local minimum at the
point in question. We summarize this as the \textit{second derivative
  test}.

\begin{theorem}[Second Derivative Test]\index{second derivative test}\label{T:sdt}
Suppose that $f''(x)$ is continuous on an open interval and that
$f'(a)=0$ for some value of $a$ in that interval.
\begin{itemize}
\item If $f''(a) <0$, then $f$ has a local maximum at $a$.
\item If $f''(a) >0$, then $f$ has a local minimum at $a$.
\item If $f''(a) =0$, then the test is inconclusive. In this case,
  $f$ may or may not have a local extremum at $x=a$.
\end{itemize}
\end{theorem}


The second derivative test is often the easiest way to identify local
maximum and minimum points. Sometimes the test fails and sometimes
the second derivative is quite difficult to evaluate. In such cases we
must fall back on one of the previous tests.

\begin{example}
Once again, consider the function 
\[
f(x) = \frac{x^4}{4}+\frac{x^3}{3}-x^2
\]
Use the second derivative test, to locate the
local extrema of $f$.

\begin{explanation}
Start by computing
\[
f'(x) = \answer[given]{x^3 + x^2 -2x} \qquad\text{and}\qquad f''(x) = \answer[given]{3x^2 + 2x-2}.
\] 
Using the same technique as we used before, we find that 
\[
f'(-2) = \answer[given]{0},\qquad f'(0) = \answer[given]{0}, \qquad f'(1) = \answer[given]{0}. 
\]
Now we'll attempt to use the second derivative test,
\[
f''(-2) = \answer[given]{6}, \qquad f''(0) =\answer[given]{ -2}, \qquad f''(1) = \answer[given]{3}.
\]
Hence we see that $f$ has a local minimum at $x=-2$, a local
maximum at $x=0$, and a local minimum at $x=1$, see below for a plot
of $f(x) =x^4/4 + x^3/3 -x^2$ and $f''(x) = 3x^2 + 2x -2$.

\begin{image}
\begin{tikzpicture}
	\begin{axis}[
            domain=-4:4,
            ymax=7,
            ymin=-4,
            %samples=100,
            axis lines =middle, xlabel=$x$, ylabel=$y$,
            every axis y label/.style={at=(current axis.above origin),anchor=south},
            every axis x label/.style={at=(current axis.right of origin),anchor=west}
          ]
          \addplot [dashed, textColor, smooth] plot coordinates {(-2,-2.667) (-2,6)}; %% {.451};
          \addplot [dashed, textColor, smooth] plot coordinates {(1,0) (1,3)}; %% axis{2.215};

          \addplot [very thick, penColor, smooth] {(x^4)/4 + (x^3)/3 -x^2};
          \addplot [very thick, penColor4, smooth] {3*x^2 + 2*x -2};

          \node at (axis cs:-1.7,-2.7) [anchor=west] {\color{penColor}$f(x)$};  
          \node at (axis cs:-1.5,2) [anchor=west] {\color{penColor4}$f''(x)$};

          \addplot[color=penColor4,fill=penColor4,only marks,mark=*] coordinates{(-2,6)};  %% closed hole
          \addplot[color=penColor4,fill=penColor4,only marks,mark=*] coordinates{(1,3)};  %% closed hole
          \addplot[color=penColor4,fill=penColor4,only marks,mark=*] coordinates{(0,-2)};  %% closed hole
          \addplot[color=penColor,fill=penColor,only marks,mark=*] coordinates{(0,0)};  %% closed hole
          \addplot[color=penColor,fill=penColor,only marks,mark=*] coordinates{(-2,.-2.667)};  %% closed hole
          \addplot[color=penColor,fill=penColor,only marks,mark=*] coordinates{(1,-.4167)};  %% closed hole
        \end{axis}
\end{tikzpicture}
%% \caption{A plot of $f(x) =x^4/4 + x^3/3 -x^2$ and $f''(x) = 3x^2 + 2x -2$.}
%% \label{figure:SDT(x^4)/4 + (x^3)/3 -x^2}
\end{image}

\end{explanation}
\end{example}





\begin{question}
  If $f''(a)=0$, what does the second derivative test tell us?
  \begin{multipleChoice}
    \choice{The function has a local extrema at $x=a$.}
    \choice{The function does not have a local extrema at $x=a$.}
    \choice[correct]{It gives no information on whether $x=a$ is a local extremum.} 
  \end{multipleChoice}
  
\end{question}


\end{document}
