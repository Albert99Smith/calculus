\documentclass{ximera}

%\usepackage{todonotes}
%\usepackage{mathtools} %% Required for wide table Curl and Greens
%\usepackage{cuted} %% Required for wide table Curl and Greens
\newcommand{\todo}{}

\usepackage{esint} % for \oiint
\ifxake%%https://math.meta.stackexchange.com/questions/9973/how-do-you-render-a-closed-surface-double-integral
\renewcommand{\oiint}{{\large\bigcirc}\kern-1.56em\iint}
\fi


\graphicspath{
  {./}
  {ximeraTutorial/}
  {basicPhilosophy/}
  {functionsOfSeveralVariables/}
  {normalVectors/}
  {lagrangeMultipliers/}
  {vectorFields/}
  {greensTheorem/}
  {shapeOfThingsToCome/}
  {dotProducts/}
  {partialDerivativesAndTheGradientVector/}
  {../productAndQuotientRules/exercises/}
  {../normalVectors/exercisesParametricPlots/}
  {../continuityOfFunctionsOfSeveralVariables/exercises/}
  {../partialDerivativesAndTheGradientVector/exercises/}
  {../directionalDerivativeAndChainRule/exercises/}
  {../commonCoordinates/exercisesCylindricalCoordinates/}
  {../commonCoordinates/exercisesSphericalCoordinates/}
  {../greensTheorem/exercisesCurlAndLineIntegrals/}
  {../greensTheorem/exercisesDivergenceAndLineIntegrals/}
  {../shapeOfThingsToCome/exercisesDivergenceTheorem/}
  {../greensTheorem/}
  {../shapeOfThingsToCome/}
  {../separableDifferentialEquations/exercises/}
  {vectorFields/}
}

\newcommand{\mooculus}{\textsf{\textbf{MOOC}\textnormal{\textsf{ULUS}}}}

\usepackage{tkz-euclide}\usepackage{tikz}
\usepackage{tikz-cd}
\usetikzlibrary{arrows}
\tikzset{>=stealth,commutative diagrams/.cd,
  arrow style=tikz,diagrams={>=stealth}} %% cool arrow head
\tikzset{shorten <>/.style={ shorten >=#1, shorten <=#1 } } %% allows shorter vectors

\usetikzlibrary{backgrounds} %% for boxes around graphs
\usetikzlibrary{shapes,positioning}  %% Clouds and stars
\usetikzlibrary{matrix} %% for matrix
\usepgfplotslibrary{polar} %% for polar plots
\usepgfplotslibrary{fillbetween} %% to shade area between curves in TikZ
\usetkzobj{all}
\usepackage[makeroom]{cancel} %% for strike outs
%\usepackage{mathtools} %% for pretty underbrace % Breaks Ximera
%\usepackage{multicol}
\usepackage{pgffor} %% required for integral for loops



%% http://tex.stackexchange.com/questions/66490/drawing-a-tikz-arc-specifying-the-center
%% Draws beach ball
\tikzset{pics/carc/.style args={#1:#2:#3}{code={\draw[pic actions] (#1:#3) arc(#1:#2:#3);}}}



\usepackage{array}
\setlength{\extrarowheight}{+.1cm}
\newdimen\digitwidth
\settowidth\digitwidth{9}
\def\divrule#1#2{
\noalign{\moveright#1\digitwidth
\vbox{\hrule width#2\digitwidth}}}





\newcommand{\RR}{\mathbb R}
\newcommand{\R}{\mathbb R}
\newcommand{\N}{\mathbb N}
\newcommand{\Z}{\mathbb Z}

\newcommand{\sagemath}{\textsf{SageMath}}


%\renewcommand{\d}{\,d\!}
\renewcommand{\d}{\mathop{}\!d}
\newcommand{\dd}[2][]{\frac{\d #1}{\d #2}}
\newcommand{\pp}[2][]{\frac{\partial #1}{\partial #2}}
\renewcommand{\l}{\ell}
\newcommand{\ddx}{\frac{d}{\d x}}

\newcommand{\zeroOverZero}{\ensuremath{\boldsymbol{\tfrac{0}{0}}}}
\newcommand{\inftyOverInfty}{\ensuremath{\boldsymbol{\tfrac{\infty}{\infty}}}}
\newcommand{\zeroOverInfty}{\ensuremath{\boldsymbol{\tfrac{0}{\infty}}}}
\newcommand{\zeroTimesInfty}{\ensuremath{\small\boldsymbol{0\cdot \infty}}}
\newcommand{\inftyMinusInfty}{\ensuremath{\small\boldsymbol{\infty - \infty}}}
\newcommand{\oneToInfty}{\ensuremath{\boldsymbol{1^\infty}}}
\newcommand{\zeroToZero}{\ensuremath{\boldsymbol{0^0}}}
\newcommand{\inftyToZero}{\ensuremath{\boldsymbol{\infty^0}}}



\newcommand{\numOverZero}{\ensuremath{\boldsymbol{\tfrac{\#}{0}}}}
\newcommand{\dfn}{\textbf}
%\newcommand{\unit}{\,\mathrm}
\newcommand{\unit}{\mathop{}\!\mathrm}
\newcommand{\eval}[1]{\bigg[ #1 \bigg]}
\newcommand{\seq}[1]{\left( #1 \right)}
\renewcommand{\epsilon}{\varepsilon}
\renewcommand{\phi}{\varphi}


\renewcommand{\iff}{\Leftrightarrow}

\DeclareMathOperator{\arccot}{arccot}
\DeclareMathOperator{\arcsec}{arcsec}
\DeclareMathOperator{\arccsc}{arccsc}
\DeclareMathOperator{\si}{Si}
\DeclareMathOperator{\scal}{scal}
\DeclareMathOperator{\sign}{sign}


%% \newcommand{\tightoverset}[2]{% for arrow vec
%%   \mathop{#2}\limits^{\vbox to -.5ex{\kern-0.75ex\hbox{$#1$}\vss}}}
\newcommand{\arrowvec}[1]{{\overset{\rightharpoonup}{#1}}}
%\renewcommand{\vec}[1]{\arrowvec{\mathbf{#1}}}
\renewcommand{\vec}[1]{{\overset{\boldsymbol{\rightharpoonup}}{\mathbf{#1}}}\hspace{0in}}

\newcommand{\point}[1]{\left(#1\right)} %this allows \vector{ to be changed to \vector{ with a quick find and replace
\newcommand{\pt}[1]{\mathbf{#1}} %this allows \vec{ to be changed to \vec{ with a quick find and replace
\newcommand{\Lim}[2]{\lim_{\point{#1} \to \point{#2}}} %Bart, I changed this to point since I want to use it.  It runs through both of the exercise and exerciseE files in limits section, which is why it was in each document to start with.

\DeclareMathOperator{\proj}{\mathbf{proj}}
\newcommand{\veci}{{\boldsymbol{\hat{\imath}}}}
\newcommand{\vecj}{{\boldsymbol{\hat{\jmath}}}}
\newcommand{\veck}{{\boldsymbol{\hat{k}}}}
\newcommand{\vecl}{\vec{\boldsymbol{\l}}}
\newcommand{\uvec}[1]{\mathbf{\hat{#1}}}
\newcommand{\utan}{\mathbf{\hat{t}}}
\newcommand{\unormal}{\mathbf{\hat{n}}}
\newcommand{\ubinormal}{\mathbf{\hat{b}}}

\newcommand{\dotp}{\bullet}
\newcommand{\cross}{\boldsymbol\times}
\newcommand{\grad}{\boldsymbol\nabla}
\newcommand{\divergence}{\grad\dotp}
\newcommand{\curl}{\grad\cross}
%\DeclareMathOperator{\divergence}{divergence}
%\DeclareMathOperator{\curl}[1]{\grad\cross #1}
\newcommand{\lto}{\mathop{\longrightarrow\,}\limits}

\renewcommand{\bar}{\overline}

\colorlet{textColor}{black}
\colorlet{background}{white}
\colorlet{penColor}{blue!50!black} % Color of a curve in a plot
\colorlet{penColor2}{red!50!black}% Color of a curve in a plot
\colorlet{penColor3}{red!50!blue} % Color of a curve in a plot
\colorlet{penColor4}{green!50!black} % Color of a curve in a plot
\colorlet{penColor5}{orange!80!black} % Color of a curve in a plot
\colorlet{penColor6}{yellow!70!black} % Color of a curve in a plot
\colorlet{fill1}{penColor!20} % Color of fill in a plot
\colorlet{fill2}{penColor2!20} % Color of fill in a plot
\colorlet{fillp}{fill1} % Color of positive area
\colorlet{filln}{penColor2!20} % Color of negative area
\colorlet{fill3}{penColor3!20} % Fill
\colorlet{fill4}{penColor4!20} % Fill
\colorlet{fill5}{penColor5!20} % Fill
\colorlet{gridColor}{gray!50} % Color of grid in a plot

\newcommand{\surfaceColor}{violet}
\newcommand{\surfaceColorTwo}{redyellow}
\newcommand{\sliceColor}{greenyellow}




\pgfmathdeclarefunction{gauss}{2}{% gives gaussian
  \pgfmathparse{1/(#2*sqrt(2*pi))*exp(-((x-#1)^2)/(2*#2^2))}%
}


%%%%%%%%%%%%%
%% Vectors
%%%%%%%%%%%%%

%% Simple horiz vectors
\renewcommand{\vector}[1]{\left\langle #1\right\rangle}


%% %% Complex Horiz Vectors with angle brackets
%% \makeatletter
%% \renewcommand{\vector}[2][ , ]{\left\langle%
%%   \def\nextitem{\def\nextitem{#1}}%
%%   \@for \el:=#2\do{\nextitem\el}\right\rangle%
%% }
%% \makeatother

%% %% Vertical Vectors
%% \def\vector#1{\begin{bmatrix}\vecListA#1,,\end{bmatrix}}
%% \def\vecListA#1,{\if,#1,\else #1\cr \expandafter \vecListA \fi}

%%%%%%%%%%%%%
%% End of vectors
%%%%%%%%%%%%%

%\newcommand{\fullwidth}{}
%\newcommand{\normalwidth}{}



%% makes a snazzy t-chart for evaluating functions
%\newenvironment{tchart}{\rowcolors{2}{}{background!90!textColor}\array}{\endarray}

%%This is to help with formatting on future title pages.
\newenvironment{sectionOutcomes}{}{}



%% Flowchart stuff
%\tikzstyle{startstop} = [rectangle, rounded corners, minimum width=3cm, minimum height=1cm,text centered, draw=black]
%\tikzstyle{question} = [rectangle, minimum width=3cm, minimum height=1cm, text centered, draw=black]
%\tikzstyle{decision} = [trapezium, trapezium left angle=70, trapezium right angle=110, minimum width=3cm, minimum height=1cm, text centered, draw=black]
%\tikzstyle{question} = [rectangle, rounded corners, minimum width=3cm, minimum height=1cm,text centered, draw=black]
%\tikzstyle{process} = [rectangle, minimum width=3cm, minimum height=1cm, text centered, draw=black]
%\tikzstyle{decision} = [trapezium, trapezium left angle=70, trapezium right angle=110, minimum width=3cm, minimum height=1cm, text centered, draw=black]


\begin{document}
\begin{abstract}
  The rate that accumulated area under a curve grows is described
  identically by that curve.
\end{abstract}
\maketitle

While the definite integral computes a signed area, which is a fixed
number, there is a way to turn it into a function.
\begin{definition}
Given a function $f(x)$, an \dfn{accumulation function} for
$f(x)$ is given by
\[
F(x) = \int_a^x f(t) \d t.
\]
\end{definition}

One thing that you might note is that an accumulation function seems
to have two variables $x$ and $t$. Let's see if we can explain
this. Consider the following plot:

\begin{tikzpicture}
	\begin{axis}[
            domain=0:6, ymax=2.2,xmax=6,
            axis lines =left, xlabel=$t$, ylabel=$y$,
            every axis y label/.style={at=(current axis.above origin),anchor=south},
            every axis x label/.style={at=(current axis.right of origin),anchor=west},
            xtick={1,5}, ytick={.203,1.679},
            xticklabels={$a$,$x$}, yticklabels={$f(a)$,$f(x)$},
            axis on top,
          ]
          \addplot [draw=none,fill=fillp,domain=(1:5)] {sin(deg((x - 4)/2)) + 1.2} \closedcycle;
          \addplot [very thick,penColor, smooth,domain=(0:6)] {sin(deg((x - 4)/2)) + 1.2};

          \addplot [textColor,dashed] plot coordinates {(0,1.679) (5,1.679)};
          \addplot [textColor,dashed] plot coordinates {(0,.203) (1,.203)};
          %\addplot [textColor,dashed] plot coordinates {(5,0) (5,1.679)};
          \addplot [textColor] plot coordinates {(1,0) (1,.203)};

          \addplot [color=penColor,fill=penColor,only marks,mark=*] coordinates{(1,.203)};  %% closed hole         
          \addplot [color=penColor,fill=penColor,only marks,mark=*] coordinates{(5,1.679)};  %% closed hole       
          \node at (axis cs:3.4,.3) [textColor] {$F(x) = \int_a^x f(t) \d t$};
          \node at (axis cs:3.4,1.1) [penColor] {$f(t)$};
        \end{axis}
\end{tikzpicture}


An accumulation function $F(x)$ is measuring the signed area in the
region $[a,x]$ between $f(t)$ and the $t$-axis. Hence $t$ is playing
the role of a ``place-holder'' and represents numbers where we are
evaluating $f(t)$. On the other hand, $x$ is the specific number that
we are using to bound the region that will determine the area between
$f(t)$ and the $t$-axis.


\begin{example} 
Consider the following accumulation function for $f(x) = x^3$.
\[
F(x) = \int_{-1}^x t^3 \d t.
\]
Considering the interval $[-1,1]$, where is $F(x)$ increasing? Where
is $F(x)$ decreasing? When does $F(x)$ have local extrema?
\end{example}

\begin{marginfigure}[-3in]
\begin{tikzpicture}
  \begin{axis}[
      xmin=-1.2, xmax=1,ymin=-1,ymax=1,domain=-1:1,
      axis lines =center, xlabel=$t$, ylabel=$y$,
      every axis y label/.style={at=(current axis.above origin),anchor=south},
      every axis x label/.style={at=(current axis.right of origin),anchor=west},
      xtick={-1,.8}, 
      xticklabels={$-1$,$x$}, 
      axis on top,
    ] 
    \addplot [draw=none, fill=fillp,domain=0:.8] {x^3} \closedcycle;
    \addplot [draw=none, fill=filln,domain=-1:0] {x^3} \closedcycle;
    \addplot [penColor,very thick,domain=-1.2:1,] {x^3};
    
    \addplot [textColor] plot coordinates {(-1,0) (-1,-1)};

    \node at (axis cs:.67,.15) [textColor] {\scalebox{2}{$\boldsymbol+$}};
    \node at (axis cs:-.85,-.3) [textColor] {\scalebox{2}{$\boldsymbol-$}};
  \end{axis}
\end{tikzpicture}
\caption{The integral $\int_{-1}^x t^3 \d t$ measures the shaded area.}
\label{figure:accumulationeg}
\end{marginfigure}

\begin{solution}
We can see a plot of $f(t)$ along with the signed area measured by the
accumulation function in Figure~\ref{figure:accumulationeg}. The
accumulation function starts off at zero, and then is decreasing as it
accumulates negatively signed area. However when $x>0$, $F(x)$ starts
to accumulate positively signed area, and hence is increasing. Thus
$F(x)$ is increasing on $(0,1)$, decreasing on $(-1,0)$ and hence has
a local minimum at $(0,0)$.
\end{solution}

Working with the accumulation function leads us to a question, what is  
\[
\int_a^x f(x) \d x
\]
when $x< a$? The general convention is that 
\[
\int_a^b f(x) \d x = -\int_b^a f(x) \d x. 
\]
With this in mind, let's consider one more example.


\begin{example} 
Consider the following accumulation function for $f(x) = x^3$.
\[
F(x) = \int_{-1}^x t^3 \d t.
\]
Where is $F(x)$ increasing? Where is $F(x)$ decreasing? When does
$F(x)$ have local extrema?
\end{example}

\begin{marginfigure}
\begin{tikzpicture}
  \begin{axis}[
      xmin=-3, xmax=1,ymin=-10,ymax=1,domain=-3:1,
      axis lines =center, xlabel=$t$, ylabel=$y$,
      every axis y label/.style={at=(current axis.above origin),anchor=south},
      every axis x label/.style={at=(current axis.right of origin),anchor=west},
      xtick={-2,-1}, 
      xticklabels={$x$,$-1$}, 
      axis on top,
    ] 
    \addplot [draw=none, fill=fillp,domain=-2:-1] {x^3} \closedcycle;
    \addplot [penColor,very thick] {x^3};

    \addplot [textColor] plot coordinates {(-1,0) (-1,-1)};
    
    \node at (axis cs:-1.5,-1.5) [textColor] {\scalebox{2}{$\boldsymbol+$}};
  \end{axis}
\end{tikzpicture}
\caption{The integral $\int_{-1}^x t^3 \d t$ measures the shaded
  area. Note, since $x<-1$, the area has positive sign.}
\label{figure:accumulationegreal}
\end{marginfigure}

\begin{solution}
From our previous example, we know that $F(x)$ is increasing on
$(0,1)$. Since $f(t)$ continues to be positive at $t=1$ and beyond,
  $F(x)$ is increasing on $(0,\infty)$. On the other hand, we know
  from our previous example that $F(x)$ is decreasing on $(-1,0)$. For
  values to the left of $t=-1$, $F(x)$ is still decreasing, as less
  and less positively signed area is accumulated. Hence $F(x)$ is
  increasing on $(0,\infty)$, decreasing on $(-\infty,0)$ and hence
  has an absolute minimum at $(0,0)$.
\end{solution}

The key point to take from these examples is that an accumulation function
\[
\int_a^x f(t) \d t
\]
is increasing precisely when $f(t)$ is positive and is decreasing
precisely when $f(t)$ is negative. In short, it seems that $f(x)$ is
behaving in a similar fashion to $F'(x)$.






\ddx \int_a^x f(t) \d t = f(x).
\]
\end{theorem}



Let $f(x)$ be continuous on the real numbers and consider
\[
  F(x) = \int_a^x f(t)\d t.
\]
From our previous work we know that $F(x)$ is increasing when $f(x)$
is positive and $F(x)$ is decreasing when $f(x)$ is negative. Moreover,
with careful observation, we can even see that $F(x)$ is concave up
when $f'(x)$ is positive and that $F(x)$ is concave down when $f'(x)$
is negative. Thinking about what we have learned about the
relationship of a function to its first and second derivatives, it is
not too hard to guess that there must be a connection between $F'(x)$
and the function $f(x)$. This is a good guess, check out our next
theorem:


\begin{mainTheorem}[Fundamental Theorem of Calculus---Version I]
\index{fundamental theorem of calculus---version 1}
\label{thm:fundamental_theorem_I}\hfil

\noindent Suppose that $f(x)$ is continuous on the real numbers and let
\[
  F(x)=\int_a^x f(t)\d t.
\]
Then $F'(x)=f(x)$.
\end{mainTheorem}

\begin{proof}
Using the limit definition of the derivative we'll compute $F'(x)$.
Write
\begin{align*}
F'(x) &= \lim_{h\to 0}\frac{F(x+h)-F(x)}{h}\\ 
&=\lim_{h\to 0}\frac{1}{h}\left( \int_a^{x+h} f(t)\d t - \int_a^x f(t)\d t\right).
\end{align*}
Recall that if the limits of integration are swapped, then the sign
of the integral is swapped, so we have
\[
F'(x) =\lim_{h\to 0} \frac{1}{h}\left( \int_a^{x+h} f(t)\d t + \int_x^a f(t)\d t\right)
\]
At this point, we can combine the integrals, as we are just ``connecting'' adjacent signed areas to find
\begin{equation}\label{ftc:eqn1}
F'(x)=\lim_{h\to 0} \frac{1}{h}\int_x^{x+h} f(t)\d t.
\end{equation}
Since $f(x)$ is continuous on the interval $[x,x+h]$, and $h$ is
approaching zero, there is an $\epsilon$ that goes to zero as $h$ goes
to zero such that
\[
f(x)-\epsilon < f(x^*) < f(x) + \epsilon \qquad \text{for all }x^*\in[x,x+h],
\]
see Figure~\ref{F:fun diagram}. This means that 
\[
(f(x) - \epsilon)h < \int_x^{x+h} f(t)\d t < (f(x) + \epsilon)h
\]
Dividing all sides by $h$ we find
\[
f(x) - \epsilon < \frac{1}{h}\int_x^{x+h} f(t)\d t < f(x) + \epsilon.
\]
Comparing this to Equation~\ref{ftc:eqn1}, and taking the limit as $h$
goes to zero (remembering that this also means that $\epsilon$ goes to
zero) we see that $F'(x) = f(x)$.
\end{proof}

\begin{marginfigure}[-6in]
\begin{tikzpicture}
  \begin{axis}[
      xmin=0, xmax=2,ymin=0,ymax=2.3,domain=0:2,
      axis lines =center, xlabel=$x$, ylabel=$y$,
      every axis y label/.style={at=(current axis.above origin),anchor=south},
      every axis x label/.style={at=(current axis.right of origin),anchor=west},
      axis on top,
      xtick={.5,1.3,1.453,1.7}, 
      ytickmin=4, ytickmax=1,
      xticklabels={$a$,$x$,$x^*$,$x+h$}, 
    ] 
    \addplot [draw=none, fill=fill1,domain=1.3:1.7] {1+sin(deg(x))*sin(deg(x^2/1.3))} \closedcycle;
    \addplot [textColor,dashed] plot coordinates {(1.453,0) (1.453,1.99)};
    \addplot [penColor,very thick,smooth] {1+sin(deg(x))*sin(deg(x^2/1.3))};
    
    \node at (axis cs:.9,1.7) [penColor] {$f(x)$};
  \end{axis}
\end{tikzpicture}
\caption{Here we see $f(x)$ along with $a$, $x$, $x^*$ and $x+h$.}
\label{F:fun diagram}
\end{marginfigure}


The Fundamental Theorem of Calculus says that an accumulation function
of $f(x)$ is an antiderivative of $f(x)$.  Because of the close
relationship between an integral and an antiderivative, the integral
sign is also used to mean ``antiderivative.'' You can tell which is
intended by whether the limits of integration are included. Hence
\[
  \int_a^b f(x)\d x
\] 
is a definite integral, because it has a definite value---the signed
area between $f(x)$ and the $x$-axis.  On the other hand, we use
\[
  \int f(x)\d x
\]
to denote the antiderivative of $f(x)$, also called an
\textit{indefinite integral}\index{indefinite integral}.
This is evaluated as
\[
  \int f(x)\d x = F(x)+C.
\]
Where $F'(x) = f(x)$ and the constant $C$ indicates that there are
really an infinite number of antiderivatives. We do not need to add
this $C$ to compute definite integrals, but in other circumstances we
will need to remember that the $C$ is there, so it is best to get into
the habit of writing the $C$.

\end{document}
