\documentclass{ximera}

%\usepackage{todonotes}
%\usepackage{mathtools} %% Required for wide table Curl and Greens
%\usepackage{cuted} %% Required for wide table Curl and Greens
\newcommand{\todo}{}

\usepackage{esint} % for \oiint
\ifxake%%https://math.meta.stackexchange.com/questions/9973/how-do-you-render-a-closed-surface-double-integral
\renewcommand{\oiint}{{\large\bigcirc}\kern-1.56em\iint}
\fi


\graphicspath{
  {./}
  {ximeraTutorial/}
  {basicPhilosophy/}
  {functionsOfSeveralVariables/}
  {normalVectors/}
  {lagrangeMultipliers/}
  {vectorFields/}
  {greensTheorem/}
  {shapeOfThingsToCome/}
  {dotProducts/}
  {partialDerivativesAndTheGradientVector/}
  {../productAndQuotientRules/exercises/}
  {../normalVectors/exercisesParametricPlots/}
  {../continuityOfFunctionsOfSeveralVariables/exercises/}
  {../partialDerivativesAndTheGradientVector/exercises/}
  {../directionalDerivativeAndChainRule/exercises/}
  {../commonCoordinates/exercisesCylindricalCoordinates/}
  {../commonCoordinates/exercisesSphericalCoordinates/}
  {../greensTheorem/exercisesCurlAndLineIntegrals/}
  {../greensTheorem/exercisesDivergenceAndLineIntegrals/}
  {../shapeOfThingsToCome/exercisesDivergenceTheorem/}
  {../greensTheorem/}
  {../shapeOfThingsToCome/}
  {../separableDifferentialEquations/exercises/}
  {vectorFields/}
}

\newcommand{\mooculus}{\textsf{\textbf{MOOC}\textnormal{\textsf{ULUS}}}}

\usepackage{tkz-euclide}\usepackage{tikz}
\usepackage{tikz-cd}
\usetikzlibrary{arrows}
\tikzset{>=stealth,commutative diagrams/.cd,
  arrow style=tikz,diagrams={>=stealth}} %% cool arrow head
\tikzset{shorten <>/.style={ shorten >=#1, shorten <=#1 } } %% allows shorter vectors

\usetikzlibrary{backgrounds} %% for boxes around graphs
\usetikzlibrary{shapes,positioning}  %% Clouds and stars
\usetikzlibrary{matrix} %% for matrix
\usepgfplotslibrary{polar} %% for polar plots
\usepgfplotslibrary{fillbetween} %% to shade area between curves in TikZ
\usetkzobj{all}
\usepackage[makeroom]{cancel} %% for strike outs
%\usepackage{mathtools} %% for pretty underbrace % Breaks Ximera
%\usepackage{multicol}
\usepackage{pgffor} %% required for integral for loops



%% http://tex.stackexchange.com/questions/66490/drawing-a-tikz-arc-specifying-the-center
%% Draws beach ball
\tikzset{pics/carc/.style args={#1:#2:#3}{code={\draw[pic actions] (#1:#3) arc(#1:#2:#3);}}}



\usepackage{array}
\setlength{\extrarowheight}{+.1cm}
\newdimen\digitwidth
\settowidth\digitwidth{9}
\def\divrule#1#2{
\noalign{\moveright#1\digitwidth
\vbox{\hrule width#2\digitwidth}}}





\newcommand{\RR}{\mathbb R}
\newcommand{\R}{\mathbb R}
\newcommand{\N}{\mathbb N}
\newcommand{\Z}{\mathbb Z}

\newcommand{\sagemath}{\textsf{SageMath}}


%\renewcommand{\d}{\,d\!}
\renewcommand{\d}{\mathop{}\!d}
\newcommand{\dd}[2][]{\frac{\d #1}{\d #2}}
\newcommand{\pp}[2][]{\frac{\partial #1}{\partial #2}}
\renewcommand{\l}{\ell}
\newcommand{\ddx}{\frac{d}{\d x}}

\newcommand{\zeroOverZero}{\ensuremath{\boldsymbol{\tfrac{0}{0}}}}
\newcommand{\inftyOverInfty}{\ensuremath{\boldsymbol{\tfrac{\infty}{\infty}}}}
\newcommand{\zeroOverInfty}{\ensuremath{\boldsymbol{\tfrac{0}{\infty}}}}
\newcommand{\zeroTimesInfty}{\ensuremath{\small\boldsymbol{0\cdot \infty}}}
\newcommand{\inftyMinusInfty}{\ensuremath{\small\boldsymbol{\infty - \infty}}}
\newcommand{\oneToInfty}{\ensuremath{\boldsymbol{1^\infty}}}
\newcommand{\zeroToZero}{\ensuremath{\boldsymbol{0^0}}}
\newcommand{\inftyToZero}{\ensuremath{\boldsymbol{\infty^0}}}



\newcommand{\numOverZero}{\ensuremath{\boldsymbol{\tfrac{\#}{0}}}}
\newcommand{\dfn}{\textbf}
%\newcommand{\unit}{\,\mathrm}
\newcommand{\unit}{\mathop{}\!\mathrm}
\newcommand{\eval}[1]{\bigg[ #1 \bigg]}
\newcommand{\seq}[1]{\left( #1 \right)}
\renewcommand{\epsilon}{\varepsilon}
\renewcommand{\phi}{\varphi}


\renewcommand{\iff}{\Leftrightarrow}

\DeclareMathOperator{\arccot}{arccot}
\DeclareMathOperator{\arcsec}{arcsec}
\DeclareMathOperator{\arccsc}{arccsc}
\DeclareMathOperator{\si}{Si}
\DeclareMathOperator{\scal}{scal}
\DeclareMathOperator{\sign}{sign}


%% \newcommand{\tightoverset}[2]{% for arrow vec
%%   \mathop{#2}\limits^{\vbox to -.5ex{\kern-0.75ex\hbox{$#1$}\vss}}}
\newcommand{\arrowvec}[1]{{\overset{\rightharpoonup}{#1}}}
%\renewcommand{\vec}[1]{\arrowvec{\mathbf{#1}}}
\renewcommand{\vec}[1]{{\overset{\boldsymbol{\rightharpoonup}}{\mathbf{#1}}}\hspace{0in}}

\newcommand{\point}[1]{\left(#1\right)} %this allows \vector{ to be changed to \vector{ with a quick find and replace
\newcommand{\pt}[1]{\mathbf{#1}} %this allows \vec{ to be changed to \vec{ with a quick find and replace
\newcommand{\Lim}[2]{\lim_{\point{#1} \to \point{#2}}} %Bart, I changed this to point since I want to use it.  It runs through both of the exercise and exerciseE files in limits section, which is why it was in each document to start with.

\DeclareMathOperator{\proj}{\mathbf{proj}}
\newcommand{\veci}{{\boldsymbol{\hat{\imath}}}}
\newcommand{\vecj}{{\boldsymbol{\hat{\jmath}}}}
\newcommand{\veck}{{\boldsymbol{\hat{k}}}}
\newcommand{\vecl}{\vec{\boldsymbol{\l}}}
\newcommand{\uvec}[1]{\mathbf{\hat{#1}}}
\newcommand{\utan}{\mathbf{\hat{t}}}
\newcommand{\unormal}{\mathbf{\hat{n}}}
\newcommand{\ubinormal}{\mathbf{\hat{b}}}

\newcommand{\dotp}{\bullet}
\newcommand{\cross}{\boldsymbol\times}
\newcommand{\grad}{\boldsymbol\nabla}
\newcommand{\divergence}{\grad\dotp}
\newcommand{\curl}{\grad\cross}
%\DeclareMathOperator{\divergence}{divergence}
%\DeclareMathOperator{\curl}[1]{\grad\cross #1}
\newcommand{\lto}{\mathop{\longrightarrow\,}\limits}

\renewcommand{\bar}{\overline}

\colorlet{textColor}{black}
\colorlet{background}{white}
\colorlet{penColor}{blue!50!black} % Color of a curve in a plot
\colorlet{penColor2}{red!50!black}% Color of a curve in a plot
\colorlet{penColor3}{red!50!blue} % Color of a curve in a plot
\colorlet{penColor4}{green!50!black} % Color of a curve in a plot
\colorlet{penColor5}{orange!80!black} % Color of a curve in a plot
\colorlet{penColor6}{yellow!70!black} % Color of a curve in a plot
\colorlet{fill1}{penColor!20} % Color of fill in a plot
\colorlet{fill2}{penColor2!20} % Color of fill in a plot
\colorlet{fillp}{fill1} % Color of positive area
\colorlet{filln}{penColor2!20} % Color of negative area
\colorlet{fill3}{penColor3!20} % Fill
\colorlet{fill4}{penColor4!20} % Fill
\colorlet{fill5}{penColor5!20} % Fill
\colorlet{gridColor}{gray!50} % Color of grid in a plot

\newcommand{\surfaceColor}{violet}
\newcommand{\surfaceColorTwo}{redyellow}
\newcommand{\sliceColor}{greenyellow}




\pgfmathdeclarefunction{gauss}{2}{% gives gaussian
  \pgfmathparse{1/(#2*sqrt(2*pi))*exp(-((x-#1)^2)/(2*#2^2))}%
}


%%%%%%%%%%%%%
%% Vectors
%%%%%%%%%%%%%

%% Simple horiz vectors
\renewcommand{\vector}[1]{\left\langle #1\right\rangle}


%% %% Complex Horiz Vectors with angle brackets
%% \makeatletter
%% \renewcommand{\vector}[2][ , ]{\left\langle%
%%   \def\nextitem{\def\nextitem{#1}}%
%%   \@for \el:=#2\do{\nextitem\el}\right\rangle%
%% }
%% \makeatother

%% %% Vertical Vectors
%% \def\vector#1{\begin{bmatrix}\vecListA#1,,\end{bmatrix}}
%% \def\vecListA#1,{\if,#1,\else #1\cr \expandafter \vecListA \fi}

%%%%%%%%%%%%%
%% End of vectors
%%%%%%%%%%%%%

%\newcommand{\fullwidth}{}
%\newcommand{\normalwidth}{}



%% makes a snazzy t-chart for evaluating functions
%\newenvironment{tchart}{\rowcolors{2}{}{background!90!textColor}\array}{\endarray}

%%This is to help with formatting on future title pages.
\newenvironment{sectionOutcomes}{}{}



%% Flowchart stuff
%\tikzstyle{startstop} = [rectangle, rounded corners, minimum width=3cm, minimum height=1cm,text centered, draw=black]
%\tikzstyle{question} = [rectangle, minimum width=3cm, minimum height=1cm, text centered, draw=black]
%\tikzstyle{decision} = [trapezium, trapezium left angle=70, trapezium right angle=110, minimum width=3cm, minimum height=1cm, text centered, draw=black]
%\tikzstyle{question} = [rectangle, rounded corners, minimum width=3cm, minimum height=1cm,text centered, draw=black]
%\tikzstyle{process} = [rectangle, minimum width=3cm, minimum height=1cm, text centered, draw=black]
%\tikzstyle{decision} = [trapezium, trapezium left angle=70, trapezium right angle=110, minimum width=3cm, minimum height=1cm, text centered, draw=black]


\title[Dig-In:]{Derivatives of inverse functions}

\begin{document}
\begin{abstract}
\end{abstract}
\maketitle


\section{The derivative of the natural logarithm}

Geometrically, there is a close relationship between the plots of
$e^x$ and $\ln(x)$, they are reflections of each other over the line
$y=x$:
\begin{image}
\begin{tikzpicture}
	\begin{axis}[
            xmin=-6,xmax=6,ymin=-6,ymax=6,
            axis lines=center,
            xlabel=$x$, ylabel=$y$,
            every axis y label/.style={at=(current axis.above origin),anchor=south},
            every axis x label/.style={at=(current axis.right of origin),anchor=west},
          ]        
          \addplot [very thick, penColor, smooth, domain=(-6:6)] {e^x};
          \addplot [very thick, penColor2, samples=100, smooth, domain=(.002:6)] {ln(x)};
          \addplot [dashed, textColor, domain=(-6:6)] {x};
          \node at (axis cs:-2,1) [penColor] {$e^x$};
          \node at (axis cs:1,-2) [penColor2] {$\ln(x)$};
        \end{axis}
\end{tikzpicture}
%% \caption{A plot of $e^x$ and $\ln(x)$. Since they are inverse
%%   functions, they are reflections of each other across the line $y=x$.}
%% \label{plot:e^x lnx}
\end{image}
One may suspect that we can use the fact that $\ddx e^x = e^x$, to
deduce the derivative of $\ln(x)$.  We will use implicit
differentiation to exploit this relationship computationally.

\begin{theorem}[The Derivative of the Natural Logrithm]\index{derivative!of the natural logarithm}
\[
\ddx \ln(x) = \frac{1}{x}.
\]
\begin{explanation}
Recall
\[
\ln(x) = y \qquad\Leftrightarrow\qquad e^y = x.
\]
Hence
\begin{align*}
e^y &= x\\
\ddx e^y &= \ddx x &\text{Differentiate both sides.}\\
e^y \dd[y]{x} &= 1 &\text{Implicit differentiation.}\\
\dd[y]{x} &= \frac{1}{e^y} = \frac{1}{x}.
\end{align*}
Since $y=\ln(x)$, $\ddx \ln(x) = \frac{1}{x}$.
\end{explanation}
\end{theorem}

From this fact, we can deduce another:

\begin{theorem}[The derivative of an exponential function]
  \[
  \ddx a^x = a^x\cdot \ln(a).
  \]
  \begin{explanation}
    Here we need to be slightly sneaky. Note
    \[
    a^x = e^{\ln(a^x)} = e^{x\ln(a)}.
    \]
    So we may write
    \begin{align*}
      \ddx a^x &= \ddx e^{x\ln(a)}\\
      &= e^{x\ln(a)} \cdot \ln(a)\\
      &= a^x\cdot \ln(a).
    \end{align*}
  \end{explanation}
\end{theorem}





\section{The derivatives of inverse trigonometric functions}

What is the derivative of the arcsine? Since this is an inverse
function, we can find its derivative by using implicit
differentiation and the Inverse Function Theorem, Theorem~\ref{theorem:IFT}.


\begin{theorem}[The Derivative of arcsin(\textit{y})]\index{derivative!of arcsine}
\[
\dd{y} \arcsin(y) = \frac{1}{\sqrt{1-y^2}}.
\]
\begin{explanation} 
To start, note that the Inverse Function Theorem,
Theorem~\ref{theorem:IFT} assures us that this derivative actually
exists.  Recall
\[
\arcsin(y) = \theta \qquad\Rightarrow\qquad \sin(\theta) = y.
\]
Hence
\begin{align*}
\sin(\theta) &= y\\
\dd{y} \sin(\theta) &= \dd{y} y \\
\cos(\theta) \frac{d\theta}{dy} &= 1 \\
\frac{d\theta}{dy} &= \frac{1}{\cos(\theta)}.
\end{align*}
At this point, we would like $\cos(\theta)$ written in terms of $y$. Since
\[
\cos^2(\theta)+\sin^2(\theta) =1
\]
and $\sin(\theta) = y$, we may write
\begin{align*}
\cos^2(\theta)+y^2 &=1\\
\cos^2(\theta) &=1-y^2\\
\cos(\theta) &= \pm \sqrt{1-y^2}.
\end{align*}
Since $\theta=\arcsin(y)$ we know that $-\pi/2\le \theta\le \pi/2$, and the cosine of
an angle in this interval is always positive. Thus
$\cos(\theta)=\sqrt{1-y^2}$ and 
\[
\dd{y} \arcsin(y) = \frac{1}{\sqrt{1-y^2}}.
\]
\end{explanation}
\end{theorem}


We can do something similar with arccosine. 

\begin{theorem}[The Derivative of arccos(\textit{y})]\index{derivative!of arccosine}
\[
\dd{y} \arccos(y) = \frac{-1}{\sqrt{1-y^2}}.
\]
\begin{explanation} 
To start, note that the Inverse Function Theorem,
Theorem~\ref{theorem:IFT} assures us that this derivative actually
exists.  Recall
\[
\arccos(y) = \theta \qquad\Rightarrow\qquad \cos(\theta) = y.
\]
Hence
\begin{align*}
\cos(\theta) &= y\\
\dd{y} \cos(\theta) &= \dd{y} y \\
-\sin(\theta) \frac{d\theta}{dy} &= 1 \\
\frac{d\theta}{dy} &= \frac{-1}{\sin(\theta)}.
\end{align*}
At this point, we would like $\sin(\theta)$ written in terms of $y$. Since
\[
\cos^2(\theta)+\sin^2(\theta) =1
\]
and $\cos(\theta) = y$, we may write
\begin{align*}
y^2+\sin^2(\theta) &=1\\
\sin^2(\theta) &=1-y^2\\
\sin(\theta) &= \pm \sqrt{1-y^2}.
\end{align*}
Since $\theta=\arccos(y)$ we know that $0\le \theta\le \pi$, and the sine of
an angle in this interval is always positive. Thus
$\sin(\theta)=\sqrt{1-y^2}$ and 
\[
\dd{y} \arccos(y) = \frac{-1}{\sqrt{1-y^2}}.
\]
\end{explanation}
\end{theorem}


Finally, let's look at arctangent.

\begin{theorem}[The Derivative of arctan(\textit{y})]\index{derivative!of arctangent}
\[
\dd{y} \arctan(y) = \frac{1}{1+y^2}.
\]
\begin{explanation} 
To start, note that the Inverse Function Theorem,
Theorem~\ref{theorem:IFT} assures us that this derivative actually
exists.  Recall
\[
\arctan(y) = \theta \qquad\Rightarrow\qquad \tan(\theta) = y.
\]
Hence
\begin{align*}
\tan(\theta) &= y\\
\dd{y} \tan(\theta) &= \dd{y} y \\
\sec^2(\theta) \frac{d\theta}{dy} &= 1 \\
\frac{d\theta}{dy} &= \frac{1}{\sec^2(\theta)}.
\end{align*}
At this point, we would like $\sec^2(\theta)$ written in terms of $y$. Recall
\[
\sec^2(\theta) = 1+\tan^2(\theta)
\]
and $\tan(\theta) = y$, we may write $\sec^2(\theta)=1+y^2$. Hence
\[
\dd{y} \arctan(y) = \frac{1}{1+y^2}.
\]
\end{explanation}
\end{theorem}


We leave it to you, the reader, to investigate the derivatives of
arcsecant, arccosecant, and arccotangent. However, as a gesture of
friendship, we now present you with a list of derivative formulas for
inverse trigonometric functions.

\begin{theorem}[The Derivatives of Inverse Trigonometric Functions] \hfil
\begin{itemize}
\item $\dd{y} \arcsin(y) = \frac{1}{\sqrt{1-y^2}}$.
\item $\dd{y} \arccos(y) = \frac{-1}{\sqrt{1-y^2}}$.
\item $\dd{y} \arctan(y) = \frac{1}{1+y^2}$.
\item $\dd{y} \arcsec(y) = \frac{1}{|y|\sqrt{y^2-1}}$ for $|y|>1$.
\item $\dd{y} \arccsc(y) = \frac{-1}{|y|\sqrt{y^2-1}}$ for $|y|>1$.
\item $\dd{y} \arccot(y) = \frac{-1}{1+y^2}$.
\end{itemize}
\end{theorem}








\end{document}
