\documentclass{ximera}

%\usepackage{todonotes}
%\usepackage{mathtools} %% Required for wide table Curl and Greens
%\usepackage{cuted} %% Required for wide table Curl and Greens
\newcommand{\todo}{}

\usepackage{esint} % for \oiint
\ifxake%%https://math.meta.stackexchange.com/questions/9973/how-do-you-render-a-closed-surface-double-integral
\renewcommand{\oiint}{{\large\bigcirc}\kern-1.56em\iint}
\fi


\graphicspath{
  {./}
  {ximeraTutorial/}
  {basicPhilosophy/}
  {functionsOfSeveralVariables/}
  {normalVectors/}
  {lagrangeMultipliers/}
  {vectorFields/}
  {greensTheorem/}
  {shapeOfThingsToCome/}
  {dotProducts/}
  {partialDerivativesAndTheGradientVector/}
  {../productAndQuotientRules/exercises/}
  {../normalVectors/exercisesParametricPlots/}
  {../continuityOfFunctionsOfSeveralVariables/exercises/}
  {../partialDerivativesAndTheGradientVector/exercises/}
  {../directionalDerivativeAndChainRule/exercises/}
  {../commonCoordinates/exercisesCylindricalCoordinates/}
  {../commonCoordinates/exercisesSphericalCoordinates/}
  {../greensTheorem/exercisesCurlAndLineIntegrals/}
  {../greensTheorem/exercisesDivergenceAndLineIntegrals/}
  {../shapeOfThingsToCome/exercisesDivergenceTheorem/}
  {../greensTheorem/}
  {../shapeOfThingsToCome/}
  {../separableDifferentialEquations/exercises/}
  {vectorFields/}
}

\newcommand{\mooculus}{\textsf{\textbf{MOOC}\textnormal{\textsf{ULUS}}}}

\usepackage{tkz-euclide}\usepackage{tikz}
\usepackage{tikz-cd}
\usetikzlibrary{arrows}
\tikzset{>=stealth,commutative diagrams/.cd,
  arrow style=tikz,diagrams={>=stealth}} %% cool arrow head
\tikzset{shorten <>/.style={ shorten >=#1, shorten <=#1 } } %% allows shorter vectors

\usetikzlibrary{backgrounds} %% for boxes around graphs
\usetikzlibrary{shapes,positioning}  %% Clouds and stars
\usetikzlibrary{matrix} %% for matrix
\usepgfplotslibrary{polar} %% for polar plots
\usepgfplotslibrary{fillbetween} %% to shade area between curves in TikZ
\usetkzobj{all}
\usepackage[makeroom]{cancel} %% for strike outs
%\usepackage{mathtools} %% for pretty underbrace % Breaks Ximera
%\usepackage{multicol}
\usepackage{pgffor} %% required for integral for loops



%% http://tex.stackexchange.com/questions/66490/drawing-a-tikz-arc-specifying-the-center
%% Draws beach ball
\tikzset{pics/carc/.style args={#1:#2:#3}{code={\draw[pic actions] (#1:#3) arc(#1:#2:#3);}}}



\usepackage{array}
\setlength{\extrarowheight}{+.1cm}
\newdimen\digitwidth
\settowidth\digitwidth{9}
\def\divrule#1#2{
\noalign{\moveright#1\digitwidth
\vbox{\hrule width#2\digitwidth}}}





\newcommand{\RR}{\mathbb R}
\newcommand{\R}{\mathbb R}
\newcommand{\N}{\mathbb N}
\newcommand{\Z}{\mathbb Z}

\newcommand{\sagemath}{\textsf{SageMath}}


%\renewcommand{\d}{\,d\!}
\renewcommand{\d}{\mathop{}\!d}
\newcommand{\dd}[2][]{\frac{\d #1}{\d #2}}
\newcommand{\pp}[2][]{\frac{\partial #1}{\partial #2}}
\renewcommand{\l}{\ell}
\newcommand{\ddx}{\frac{d}{\d x}}

\newcommand{\zeroOverZero}{\ensuremath{\boldsymbol{\tfrac{0}{0}}}}
\newcommand{\inftyOverInfty}{\ensuremath{\boldsymbol{\tfrac{\infty}{\infty}}}}
\newcommand{\zeroOverInfty}{\ensuremath{\boldsymbol{\tfrac{0}{\infty}}}}
\newcommand{\zeroTimesInfty}{\ensuremath{\small\boldsymbol{0\cdot \infty}}}
\newcommand{\inftyMinusInfty}{\ensuremath{\small\boldsymbol{\infty - \infty}}}
\newcommand{\oneToInfty}{\ensuremath{\boldsymbol{1^\infty}}}
\newcommand{\zeroToZero}{\ensuremath{\boldsymbol{0^0}}}
\newcommand{\inftyToZero}{\ensuremath{\boldsymbol{\infty^0}}}



\newcommand{\numOverZero}{\ensuremath{\boldsymbol{\tfrac{\#}{0}}}}
\newcommand{\dfn}{\textbf}
%\newcommand{\unit}{\,\mathrm}
\newcommand{\unit}{\mathop{}\!\mathrm}
\newcommand{\eval}[1]{\bigg[ #1 \bigg]}
\newcommand{\seq}[1]{\left( #1 \right)}
\renewcommand{\epsilon}{\varepsilon}
\renewcommand{\phi}{\varphi}


\renewcommand{\iff}{\Leftrightarrow}

\DeclareMathOperator{\arccot}{arccot}
\DeclareMathOperator{\arcsec}{arcsec}
\DeclareMathOperator{\arccsc}{arccsc}
\DeclareMathOperator{\si}{Si}
\DeclareMathOperator{\scal}{scal}
\DeclareMathOperator{\sign}{sign}


%% \newcommand{\tightoverset}[2]{% for arrow vec
%%   \mathop{#2}\limits^{\vbox to -.5ex{\kern-0.75ex\hbox{$#1$}\vss}}}
\newcommand{\arrowvec}[1]{{\overset{\rightharpoonup}{#1}}}
%\renewcommand{\vec}[1]{\arrowvec{\mathbf{#1}}}
\renewcommand{\vec}[1]{{\overset{\boldsymbol{\rightharpoonup}}{\mathbf{#1}}}\hspace{0in}}

\newcommand{\point}[1]{\left(#1\right)} %this allows \vector{ to be changed to \vector{ with a quick find and replace
\newcommand{\pt}[1]{\mathbf{#1}} %this allows \vec{ to be changed to \vec{ with a quick find and replace
\newcommand{\Lim}[2]{\lim_{\point{#1} \to \point{#2}}} %Bart, I changed this to point since I want to use it.  It runs through both of the exercise and exerciseE files in limits section, which is why it was in each document to start with.

\DeclareMathOperator{\proj}{\mathbf{proj}}
\newcommand{\veci}{{\boldsymbol{\hat{\imath}}}}
\newcommand{\vecj}{{\boldsymbol{\hat{\jmath}}}}
\newcommand{\veck}{{\boldsymbol{\hat{k}}}}
\newcommand{\vecl}{\vec{\boldsymbol{\l}}}
\newcommand{\uvec}[1]{\mathbf{\hat{#1}}}
\newcommand{\utan}{\mathbf{\hat{t}}}
\newcommand{\unormal}{\mathbf{\hat{n}}}
\newcommand{\ubinormal}{\mathbf{\hat{b}}}

\newcommand{\dotp}{\bullet}
\newcommand{\cross}{\boldsymbol\times}
\newcommand{\grad}{\boldsymbol\nabla}
\newcommand{\divergence}{\grad\dotp}
\newcommand{\curl}{\grad\cross}
%\DeclareMathOperator{\divergence}{divergence}
%\DeclareMathOperator{\curl}[1]{\grad\cross #1}
\newcommand{\lto}{\mathop{\longrightarrow\,}\limits}

\renewcommand{\bar}{\overline}

\colorlet{textColor}{black}
\colorlet{background}{white}
\colorlet{penColor}{blue!50!black} % Color of a curve in a plot
\colorlet{penColor2}{red!50!black}% Color of a curve in a plot
\colorlet{penColor3}{red!50!blue} % Color of a curve in a plot
\colorlet{penColor4}{green!50!black} % Color of a curve in a plot
\colorlet{penColor5}{orange!80!black} % Color of a curve in a plot
\colorlet{penColor6}{yellow!70!black} % Color of a curve in a plot
\colorlet{fill1}{penColor!20} % Color of fill in a plot
\colorlet{fill2}{penColor2!20} % Color of fill in a plot
\colorlet{fillp}{fill1} % Color of positive area
\colorlet{filln}{penColor2!20} % Color of negative area
\colorlet{fill3}{penColor3!20} % Fill
\colorlet{fill4}{penColor4!20} % Fill
\colorlet{fill5}{penColor5!20} % Fill
\colorlet{gridColor}{gray!50} % Color of grid in a plot

\newcommand{\surfaceColor}{violet}
\newcommand{\surfaceColorTwo}{redyellow}
\newcommand{\sliceColor}{greenyellow}




\pgfmathdeclarefunction{gauss}{2}{% gives gaussian
  \pgfmathparse{1/(#2*sqrt(2*pi))*exp(-((x-#1)^2)/(2*#2^2))}%
}


%%%%%%%%%%%%%
%% Vectors
%%%%%%%%%%%%%

%% Simple horiz vectors
\renewcommand{\vector}[1]{\left\langle #1\right\rangle}


%% %% Complex Horiz Vectors with angle brackets
%% \makeatletter
%% \renewcommand{\vector}[2][ , ]{\left\langle%
%%   \def\nextitem{\def\nextitem{#1}}%
%%   \@for \el:=#2\do{\nextitem\el}\right\rangle%
%% }
%% \makeatother

%% %% Vertical Vectors
%% \def\vector#1{\begin{bmatrix}\vecListA#1,,\end{bmatrix}}
%% \def\vecListA#1,{\if,#1,\else #1\cr \expandafter \vecListA \fi}

%%%%%%%%%%%%%
%% End of vectors
%%%%%%%%%%%%%

%\newcommand{\fullwidth}{}
%\newcommand{\normalwidth}{}



%% makes a snazzy t-chart for evaluating functions
%\newenvironment{tchart}{\rowcolors{2}{}{background!90!textColor}\array}{\endarray}

%%This is to help with formatting on future title pages.
\newenvironment{sectionOutcomes}{}{}



%% Flowchart stuff
%\tikzstyle{startstop} = [rectangle, rounded corners, minimum width=3cm, minimum height=1cm,text centered, draw=black]
%\tikzstyle{question} = [rectangle, minimum width=3cm, minimum height=1cm, text centered, draw=black]
%\tikzstyle{decision} = [trapezium, trapezium left angle=70, trapezium right angle=110, minimum width=3cm, minimum height=1cm, text centered, draw=black]
%\tikzstyle{question} = [rectangle, rounded corners, minimum width=3cm, minimum height=1cm,text centered, draw=black]
%\tikzstyle{process} = [rectangle, minimum width=3cm, minimum height=1cm, text centered, draw=black]
%\tikzstyle{decision} = [trapezium, trapezium left angle=70, trapezium right angle=110, minimum width=3cm, minimum height=1cm, text centered, draw=black]


\outcome{}

\title[Dig-In:]{Approximating Antiderivatives}

\begin{document}
	For some functions, we might be able to guess an antiderivative symbolically.  For instance, because $2x\cos(x^2)$ looks like "a chain rule has been applied", you can guess one antiderivative is $\antiderivativeanswer{sin(x^2)}$.
	
	Other functions may be more difficult to guess an antiderivative symbolically.  Some might even be \textit{impossible}.  It turns out that the function $f(x) = e^{-x^2}$, for instance, has an antiderivative, but you cannot write down a formula for it in terms of elementary functions.  See \href{https://en.wikipedia.org/wiki/Nonelementary_integral}{the wikipedia entry on non elementary integrals} for more details. This particular antiderivative ends up being \href{https://en.wikipedia.org/wiki/Normal_distribution}{incredibly important in statistics}.  
	
	This brings us to the general question of this section:  if we do not know how to get an antiderivative symbolically, can we compute it analytically?  As always in calculus we will
	\begin{itemize}
		\item Approximate the values of the antiderivative
		\item Improve the approximation in a systematic way
		\item In the limit, improving the approximation will yield an exact answer
	\end{itemize}	
	
	In the following sequence of questions, we will approximate the value of the antiderivative of $e^{x^2}$ at $x=1$.
	
	\begin{question}
		Let $f(x) = e^{-x^2}$.  Let $F(x)$ be the antiderivative of $f(x)$ with $F(0)=0$.  One way to approximate $F(1)$ would be to use the tangent line approximation to $F$ at $x=0$.  Using this we have $F(1) \approx \answer{1}$
		\begin{hint}
			$F'(0) = f(0) = e^0 = 1$
		\end{hint}
		\begin{hint}
			We have a run of $1$, so (since the slope is $1$) we get a rise of $1$
		\end{hint}
		\begin{hint}
			Thus $F(1) \approx 0+1 = 1$
		\end{hint}
	\end{question}
	
	\begin{question}
		One way to improve our approximation is to \textbf{linearly approximate repeatedly}.  Instead of approximating $F$ by a single linear function on $[0,1]$, we can approximate it by a piecewise linear function. 
		
		Using the tangent line approximation to $F$ at $x=0$, $F(\frac{1}{2}) \approx \answer{1/2}$
			\begin{hint}
				$F(\frac{1}{2}) \approx F(0)+F'(0)(\frac{1}{2} - 0) = 0+1(\frac{1}{2}-0) = \frac{1}{2}$
			\end{hint}
		
		Now using the tangent line approximation to $F$ at $x=\frac{1}{2}$, $F(1) \approx \answer{\frac{1}{2}(1+e^{-1/4})}$
			\begin{hint}
				$F(1) \approx F(\frac{1}{2}) + F'(\frac{1}{2})(1 - \frac{1}{2}) \approx \frac{1}{2} + e^{-\frac{1}{4}}(\frac{1}{2})$
			\end{hint}
	\end{question}
	
	\begin{question}
		Using the same idea, what approximation do we get to $F(1)$ if we use $3$ subintervals?
		
		$F(1) \approx \answer{\frac{1}{3}(1+e^{-\frac{1}{9}}+e^{-\frac{4}{9}})}$
			\begin{hint}
				\begin{itemize}
				\itemize Over the interval $[0,\frac{1}{3}]$, $F$ should rise approximately $\frac{1}{3} F'(0) $
				\itemize  Over the interval $[\frac{1}{3},\frac{2}{3}]$, $F$ should rise approximately $\frac{1}{3} F'(\frac{1}{3}) $
				\itemize  Over the interval $[\frac{2}{3}1]$, $F$ should rise approximately $\frac{1}{3} F'(\frac{2}{3}) $
				\end{itemize}
				
				So all together, $F$ should rise approximately $\frac{1}{3} F'(0) + \frac{1}{3}F'(\frac{1}{3})+\frac{1}{3} F'(\frac{2}{3})$
			\end{hint}
			\begin{hint}
				$F'(x) = f(x)$ by definition of an antiderivative, so this sum is equal to $\frac{1}{3} (f(0)+f(\frac{1}{3})+f(\frac{2}{3})) = \frac{1}{3}(1+e^{-\frac{1}{9}}+e^{-\frac{4}{9}})$
			\end{hint}
	\end{question}
	
	\begin{question}
		Using the same idea, what approximation do we get to $F(1)$ if we use $n$ subintervals?
		
		\[
		F(1) \approx \sum_{k=1}^{k=n} \answer{\frac{1}{n} e^{-\frac{(k-1)^2}{n^2}}}
		\]
		
		\begin{hint}
			The $k^{th}$ subinterval is $[ \frac{k-1}{n},\frac{k}{n}]$, which has length $\frac{1}{n}$
		\end{hint}
		\begin{hint}
			The slope of $F$ on this subinterval can be approximated by the slope at the left endpoint.  So we are approximating the slope to be $F'(\frac{k-1}{n}) = f(\frac{k-1}{n}) = e^{-\frac{(k-1)^2}{n^2}}$
		\end{hint}
		\begin{hint}
			So the change in $F$ over the $k^{th}$ subinterval is approximately $\frac{1}{n} e^{-\frac{(k-1)^2}{n^2}}$.  Summing all of these should give an approximation for the total change in $F$ from $x=0$ to $x=1$.  Since $F(0)=0$, this total change approximates the value at $x=1$.
		\end{hint}
	\end{question}
	
	This procedure can be generalized to approximate an antiderivative of any continuous function.
	
	%We do not have an idea environment, but I think this would be nice.
	\begin{idea}
		If $f$ is a given continuous function, and we want to approximate an antiderivative of $F$, we can do so in the following way:
			\being{itemize}
				\itemize Pick a single value for $F$, say $f(a) = C$.  This chooses one of the antiderivatives (which are all constant shifts of each other)
				\itemize To approximate $F(b)$, take the interval $[a,b]$ and chop it into $n$ equal subintervals
				\itemize  The length of the each subinterval is $\frac{b-a}{n}$, and the $k^{th} $ subinterval is $[a+\frac{b-a}{n} (k-1), a+\frac{b-a}{n} k$
				\itemize We can approximate $F$ on the $k^{th}$ subinterval by a linear function whose slope is $F'(a+\frac{b-a}{n} (k-1)) =f(a+\frac{b-a}{n} (k-1))$.
				\itemize So the total change in $F$ over the $k^{th}$ subinterval is approximately $f(a+\frac{b-a}{n} (k-1)) \frac{b-a}{n}$
				\itemize The total change in $F$ over the whole interval $[a,b]$ is the sum of these  changes, i.e. $F(b) - F(a) \approx \sum_{k=1}^{n} f(a+\frac{b-a}{n} (k-1)) \frac{b-a}{n}$
				\itemize Thus $F(b) \approx F(a)+ \sum_{k=1}^{n} f(a+\frac{b-a}{n} (k-1)) \frac{b-a}{n} = C + \sum_{k=1}^{n} f(a+\frac{b-a}{n} (k-1)) \frac{b-a}{n}$
			\end{itemize}
	\end{idea}
	
	It is important that, rather than just memorizing the last line of the outline above, you really make sure you understand the whole process and can recreate this reasoning for yourself.  
	
	Let's get some practice dealing with the algebra of this construction:
	
	\begin{question}
		Let $F$ be the antiderivative of $f(x) = \log(x)$ which satisfies $F(2) = 3$.  Apply the idea above to approximate $F(6)$ using $4$ equally spaced subintervals.  $F(6) \approx \answer{3+ \log(2) + \log(3) + \log(4) + \log(5)}$
		\begin{hint}
			The subintervals are $[2,3]$, $[3,4]$, $[4,5]$, and $[5,6]$
		\end{hint}
		\begin{hint}
			Approximating $F$ by a line on each subinterval, we have that the slope on each subinterval is $\log(2)$, $\log(3)$, $\log(4)$,  and $\log(5)$ respectively.
		\end{hint}
		\begin{hint}
			Since each subinterval is of length $1$, and ``rise = slope $\times$ run'', we have that the total rise is approximately $\log(2) + \log(3) + \log(4) + \log(5)$ 
		\end{hint}
		\begin{hint}
			Since $F(2) = 3$, we get the approximation $F(6) \approx 3+ \log(2) + \log(3) + \log(4) + \log(5)$
		\end{hint}
	\end{question}
	
	\begin{question}
		Say you found a piece of paper with the following scratch work on it: $F'(x) =f(x)$,  $F(1) = 3$, some indecipherable work, and $F(b) \approx 3 + \sum_{k=1}^{n} \left[2+ ( 1+ \frac{7}{n}(k-1))^2 \right ] \frac{7}{n}$.
		
		This is enough information to figure out that $b = \answer{8}$   and $f = \answer{2+x^2}$
	\end{question}
\end{document}
