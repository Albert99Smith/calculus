\documentclass{ximera}

%\usepackage{todonotes}
%\usepackage{mathtools} %% Required for wide table Curl and Greens
%\usepackage{cuted} %% Required for wide table Curl and Greens
\newcommand{\todo}{}

\usepackage{esint} % for \oiint
\ifxake%%https://math.meta.stackexchange.com/questions/9973/how-do-you-render-a-closed-surface-double-integral
\renewcommand{\oiint}{{\large\bigcirc}\kern-1.56em\iint}
\fi


\graphicspath{
  {./}
  {ximeraTutorial/}
  {basicPhilosophy/}
  {functionsOfSeveralVariables/}
  {normalVectors/}
  {lagrangeMultipliers/}
  {vectorFields/}
  {greensTheorem/}
  {shapeOfThingsToCome/}
  {dotProducts/}
  {partialDerivativesAndTheGradientVector/}
  {../productAndQuotientRules/exercises/}
  {../normalVectors/exercisesParametricPlots/}
  {../continuityOfFunctionsOfSeveralVariables/exercises/}
  {../partialDerivativesAndTheGradientVector/exercises/}
  {../directionalDerivativeAndChainRule/exercises/}
  {../commonCoordinates/exercisesCylindricalCoordinates/}
  {../commonCoordinates/exercisesSphericalCoordinates/}
  {../greensTheorem/exercisesCurlAndLineIntegrals/}
  {../greensTheorem/exercisesDivergenceAndLineIntegrals/}
  {../shapeOfThingsToCome/exercisesDivergenceTheorem/}
  {../greensTheorem/}
  {../shapeOfThingsToCome/}
  {../separableDifferentialEquations/exercises/}
  {vectorFields/}
}

\newcommand{\mooculus}{\textsf{\textbf{MOOC}\textnormal{\textsf{ULUS}}}}

\usepackage{tkz-euclide}\usepackage{tikz}
\usepackage{tikz-cd}
\usetikzlibrary{arrows}
\tikzset{>=stealth,commutative diagrams/.cd,
  arrow style=tikz,diagrams={>=stealth}} %% cool arrow head
\tikzset{shorten <>/.style={ shorten >=#1, shorten <=#1 } } %% allows shorter vectors

\usetikzlibrary{backgrounds} %% for boxes around graphs
\usetikzlibrary{shapes,positioning}  %% Clouds and stars
\usetikzlibrary{matrix} %% for matrix
\usepgfplotslibrary{polar} %% for polar plots
\usepgfplotslibrary{fillbetween} %% to shade area between curves in TikZ
\usetkzobj{all}
\usepackage[makeroom]{cancel} %% for strike outs
%\usepackage{mathtools} %% for pretty underbrace % Breaks Ximera
%\usepackage{multicol}
\usepackage{pgffor} %% required for integral for loops



%% http://tex.stackexchange.com/questions/66490/drawing-a-tikz-arc-specifying-the-center
%% Draws beach ball
\tikzset{pics/carc/.style args={#1:#2:#3}{code={\draw[pic actions] (#1:#3) arc(#1:#2:#3);}}}



\usepackage{array}
\setlength{\extrarowheight}{+.1cm}
\newdimen\digitwidth
\settowidth\digitwidth{9}
\def\divrule#1#2{
\noalign{\moveright#1\digitwidth
\vbox{\hrule width#2\digitwidth}}}





\newcommand{\RR}{\mathbb R}
\newcommand{\R}{\mathbb R}
\newcommand{\N}{\mathbb N}
\newcommand{\Z}{\mathbb Z}

\newcommand{\sagemath}{\textsf{SageMath}}


%\renewcommand{\d}{\,d\!}
\renewcommand{\d}{\mathop{}\!d}
\newcommand{\dd}[2][]{\frac{\d #1}{\d #2}}
\newcommand{\pp}[2][]{\frac{\partial #1}{\partial #2}}
\renewcommand{\l}{\ell}
\newcommand{\ddx}{\frac{d}{\d x}}

\newcommand{\zeroOverZero}{\ensuremath{\boldsymbol{\tfrac{0}{0}}}}
\newcommand{\inftyOverInfty}{\ensuremath{\boldsymbol{\tfrac{\infty}{\infty}}}}
\newcommand{\zeroOverInfty}{\ensuremath{\boldsymbol{\tfrac{0}{\infty}}}}
\newcommand{\zeroTimesInfty}{\ensuremath{\small\boldsymbol{0\cdot \infty}}}
\newcommand{\inftyMinusInfty}{\ensuremath{\small\boldsymbol{\infty - \infty}}}
\newcommand{\oneToInfty}{\ensuremath{\boldsymbol{1^\infty}}}
\newcommand{\zeroToZero}{\ensuremath{\boldsymbol{0^0}}}
\newcommand{\inftyToZero}{\ensuremath{\boldsymbol{\infty^0}}}



\newcommand{\numOverZero}{\ensuremath{\boldsymbol{\tfrac{\#}{0}}}}
\newcommand{\dfn}{\textbf}
%\newcommand{\unit}{\,\mathrm}
\newcommand{\unit}{\mathop{}\!\mathrm}
\newcommand{\eval}[1]{\bigg[ #1 \bigg]}
\newcommand{\seq}[1]{\left( #1 \right)}
\renewcommand{\epsilon}{\varepsilon}
\renewcommand{\phi}{\varphi}


\renewcommand{\iff}{\Leftrightarrow}

\DeclareMathOperator{\arccot}{arccot}
\DeclareMathOperator{\arcsec}{arcsec}
\DeclareMathOperator{\arccsc}{arccsc}
\DeclareMathOperator{\si}{Si}
\DeclareMathOperator{\scal}{scal}
\DeclareMathOperator{\sign}{sign}


%% \newcommand{\tightoverset}[2]{% for arrow vec
%%   \mathop{#2}\limits^{\vbox to -.5ex{\kern-0.75ex\hbox{$#1$}\vss}}}
\newcommand{\arrowvec}[1]{{\overset{\rightharpoonup}{#1}}}
%\renewcommand{\vec}[1]{\arrowvec{\mathbf{#1}}}
\renewcommand{\vec}[1]{{\overset{\boldsymbol{\rightharpoonup}}{\mathbf{#1}}}\hspace{0in}}

\newcommand{\point}[1]{\left(#1\right)} %this allows \vector{ to be changed to \vector{ with a quick find and replace
\newcommand{\pt}[1]{\mathbf{#1}} %this allows \vec{ to be changed to \vec{ with a quick find and replace
\newcommand{\Lim}[2]{\lim_{\point{#1} \to \point{#2}}} %Bart, I changed this to point since I want to use it.  It runs through both of the exercise and exerciseE files in limits section, which is why it was in each document to start with.

\DeclareMathOperator{\proj}{\mathbf{proj}}
\newcommand{\veci}{{\boldsymbol{\hat{\imath}}}}
\newcommand{\vecj}{{\boldsymbol{\hat{\jmath}}}}
\newcommand{\veck}{{\boldsymbol{\hat{k}}}}
\newcommand{\vecl}{\vec{\boldsymbol{\l}}}
\newcommand{\uvec}[1]{\mathbf{\hat{#1}}}
\newcommand{\utan}{\mathbf{\hat{t}}}
\newcommand{\unormal}{\mathbf{\hat{n}}}
\newcommand{\ubinormal}{\mathbf{\hat{b}}}

\newcommand{\dotp}{\bullet}
\newcommand{\cross}{\boldsymbol\times}
\newcommand{\grad}{\boldsymbol\nabla}
\newcommand{\divergence}{\grad\dotp}
\newcommand{\curl}{\grad\cross}
%\DeclareMathOperator{\divergence}{divergence}
%\DeclareMathOperator{\curl}[1]{\grad\cross #1}
\newcommand{\lto}{\mathop{\longrightarrow\,}\limits}

\renewcommand{\bar}{\overline}

\colorlet{textColor}{black}
\colorlet{background}{white}
\colorlet{penColor}{blue!50!black} % Color of a curve in a plot
\colorlet{penColor2}{red!50!black}% Color of a curve in a plot
\colorlet{penColor3}{red!50!blue} % Color of a curve in a plot
\colorlet{penColor4}{green!50!black} % Color of a curve in a plot
\colorlet{penColor5}{orange!80!black} % Color of a curve in a plot
\colorlet{penColor6}{yellow!70!black} % Color of a curve in a plot
\colorlet{fill1}{penColor!20} % Color of fill in a plot
\colorlet{fill2}{penColor2!20} % Color of fill in a plot
\colorlet{fillp}{fill1} % Color of positive area
\colorlet{filln}{penColor2!20} % Color of negative area
\colorlet{fill3}{penColor3!20} % Fill
\colorlet{fill4}{penColor4!20} % Fill
\colorlet{fill5}{penColor5!20} % Fill
\colorlet{gridColor}{gray!50} % Color of grid in a plot

\newcommand{\surfaceColor}{violet}
\newcommand{\surfaceColorTwo}{redyellow}
\newcommand{\sliceColor}{greenyellow}




\pgfmathdeclarefunction{gauss}{2}{% gives gaussian
  \pgfmathparse{1/(#2*sqrt(2*pi))*exp(-((x-#1)^2)/(2*#2^2))}%
}


%%%%%%%%%%%%%
%% Vectors
%%%%%%%%%%%%%

%% Simple horiz vectors
\renewcommand{\vector}[1]{\left\langle #1\right\rangle}


%% %% Complex Horiz Vectors with angle brackets
%% \makeatletter
%% \renewcommand{\vector}[2][ , ]{\left\langle%
%%   \def\nextitem{\def\nextitem{#1}}%
%%   \@for \el:=#2\do{\nextitem\el}\right\rangle%
%% }
%% \makeatother

%% %% Vertical Vectors
%% \def\vector#1{\begin{bmatrix}\vecListA#1,,\end{bmatrix}}
%% \def\vecListA#1,{\if,#1,\else #1\cr \expandafter \vecListA \fi}

%%%%%%%%%%%%%
%% End of vectors
%%%%%%%%%%%%%

%\newcommand{\fullwidth}{}
%\newcommand{\normalwidth}{}



%% makes a snazzy t-chart for evaluating functions
%\newenvironment{tchart}{\rowcolors{2}{}{background!90!textColor}\array}{\endarray}

%%This is to help with formatting on future title pages.
\newenvironment{sectionOutcomes}{}{}



%% Flowchart stuff
%\tikzstyle{startstop} = [rectangle, rounded corners, minimum width=3cm, minimum height=1cm,text centered, draw=black]
%\tikzstyle{question} = [rectangle, minimum width=3cm, minimum height=1cm, text centered, draw=black]
%\tikzstyle{decision} = [trapezium, trapezium left angle=70, trapezium right angle=110, minimum width=3cm, minimum height=1cm, text centered, draw=black]
%\tikzstyle{question} = [rectangle, rounded corners, minimum width=3cm, minimum height=1cm,text centered, draw=black]
%\tikzstyle{process} = [rectangle, minimum width=3cm, minimum height=1cm, text centered, draw=black]
%\tikzstyle{decision} = [trapezium, trapezium left angle=70, trapezium right angle=110, minimum width=3cm, minimum height=1cm, text centered, draw=black]


\outcome{}

\title[Dig-In:]{Anti-Derivatives}

\begin{document}
	\begin{definition}
		We say that $F$ is an \textbf{antiderivative} of $f$ if the derivative of $F$ is $f$.
	\end{definition}
	
%	\begin{question}
%		BADBAD 
%		Want a question here where the student is presented with $f$ and has to sketch an antiderivative.
%	\end{question}
	
	\begin{question}
		An antiderivative of $f(x) = x^2$ is $F(x) = \antiderivativeanswer{x^2}$
			\begin{hint}
				You have to think backwards.  What function can you think of that, when differentiated, has an $x^2$ term in it?  Can you modify that function to give you the antiderivative you seek?
			\end{hint}
			\begin{hint}
				$\frac{d}{dx} x^3 = 3x^2$, so this is close.  What could one do to $x^3$ to get rid of that factor of $3$?
			\end{hint}
			\begin{hint}
				$\frac{d}{dx} \frac{1}{3} x^3 = x^2$, so $F(x) = \frac{1}{3} x^3$ is an antiderivative.  Note that there are other antiderivatives, like $F(x) = \frac{1}{3}x^3 + 7$, for instance.
			\end{hint}
	\end{question}
	
	\begin{question}
		%I do not think we have the ability to access multiple answers yet, even in a custom solution environment.
		
		What are two different antiderivatives of $f(x) = 0$?  $F_1(x) =  \antiderivativeanswer{0}$ and $F_2(x) = \antiderivativeanswer{0}$.
		
		\begin{hint}
			The derivative of any constant function is $0$, so you can just use two different constant functions.
		\end{hint}
	\end{question}
	
	\begin{question}
		Use the mean value theorem for derivatives to show that if $F$ is an antiderivative of $f(x) = 0$, then $F$ must be a constant function.  This should make intuitive sense, since basically the statement says "If the slope is always 0, then the graph is a horizontal line", but here you should provide a rigorous justification using the mean value theorem.
		
		\begin{hint}
			Showing that $F$ is constant is the same as showing that $F(a) = F(b)$ for every $a,b \in \mathbb{R}$.
		\end{hint}
		\begin{hint}
			If you assume to the contrary that $F(a) \neq F(b)$ for some $a,b \in \mathbb{R}$, then the slope of the secant line connecting them will be nonzero.  Can you use the mean value theorem for derivatives to get a contradiction?
 		\end{hint}
		
		\begin{free-response}
			Showing that $F$ is constant is the same as showing that $F(a) = F(b)$ for every $a,b \in \mathbb{R}$.
			
			Assume to the contrary that $F(a) \neq F(b)$ for some $a,b \in \mathbb{R}$.
			
			Then the slope of the secant line connecting $(a,F(a))$ to $(b,F(b))$ is $\frac{F(b)-F(a)}{b-a}$ which is nonzero.
			
			By the mean value theorem for derivatives, there is a point $c$ between $a$ and $b$ with $F'(c) = \frac{F(b) - F(a)}{b-a}$.
			
			But $F'(c) = f(c)$ since $F$ is an antiderivative of $f$, and $f(c) = 0$.
			
			Thus assuming $F$ was not constant has lead to a contradiction.
			
			Thus $F$ must be a constant function. 
		\end{free-response}
	\end{question}
	
	\begin{question}
		Using the previous exercise, show that if $F_1$ and $F_2$ are both antiderivatives of $f$ on an interval $(a,b)$, then $F_1 - F_2$ is constant on that interval. 
		\begin{free-response}
			Since $F_1$ and $F_2$ are both antiderivatives of $f$, we have
			
			$\frac{d}{dx}  \left[ F_1(x) - F_2(x)\right]  =f(x) - f(x) = 0$
			
			This $F_1 - F_2$ is an antiderivative of $0$, which by the previous exercise must be a constant function.
		\end{free-response}
	\end{question}
	
	It is worth stating the contents of the last question as a theorem:
	
	\begin{theorem}
		Let $F_1$ and $F_2 $ be antiderivatives of $f$ on an interval $(a,b)$.  Then $F_1(x) = F_2(x)+C$ for some constant $C$.
	\end{theorem}
	
	\begin{question}
		Let $f(x)$ be defined on $[-3,6]$.  Let $F(x)$ and $G(x)$ be antiderivatives of $f$.  Mark all that necessarily  are true.
		\begin{multiple-response}
			\choice[correct]{ F(1) - G(1) = F(2)  - G(2)}
			\choice[correct]{F(3) - F(0) = G(3) - G(0)}
			\choice{F(1) - G(0) = G(1) - F(0)}
			\choice{F(1) - F(0) = F(2) - F(1)} 
		\end{multiple-response}
	\end{question}
	
	\begin{question}
		$\frac{d}{dx} $
	\end{question}
	
	\begin{question}
		Let $f(x)$ be defined on $(-7,0) \cup (3,9)$. Let  $F$ and $G$ be antiderivatives of $f$.  Mark all that are necessarily true:
		\begin{multiple-response}
			\choice[correct]{F(-2) - F(-5) = G(-2) - G(-5)}
			\choice{F(5) - F(-2) = G(5) - G(-2) }
			\chocie[correct]{F(7) - G(7) = F(5) - G(5) }
			\choice{There is a constant $C$ such that $F(x) = G(x)+C$}
		\end{multiple-response}
	\end{question}
	
	As the last exercise shows, if $f$ is defined on a disjoint set of intervals, its antiderivatives can differ by a different constant on each piece.  
	For example, in the following picture $F$ and $G$ are both antiderivatives of $f$, but they have a different constant on each interval.
		
	\begin{theorem}
		Let $f(x)$ be defined on $(a_1,b_1) \cup (a_2,b_2) \cup ... \cup (a_n,b_n)$ with all the intervals $(a_i,b_i)$ disjoint.  If $F$ and $G$ are antiderivatives of $f$, 
		then there are constants $C_1,C_2,...,C_n$ such that
		
		$G(x) = \begin{cases}
			F(x)+C_1 \text{on $(a_1,b_1)$}\\
			F(x)+C_2 \text{on $(a_2,b_2)$}\\
			\vdots\\
			F(x)+C_n \text{on $(a_n,b_n)$}\\
		\end{cases}$
	\end{theorem}
	
	\begin{example}
		We know that $\frac{d}{dx} \log( |x |) =  \frac{1}{x}$ on its domain.   Since its domain consists of the two disjoint intervals $(-\infty,0) \cup (0,\infty)$, we have that any antiderivative of $\frac{1}{x}$ can be written 
		\[
		F(x) = \begin{cases}
			\log(x)+C_1 if $x >0$\\
			\log(-x)+C_2 if $x<0$
		\end{cases}
		\]
	\end{example}
	
	\begin{question}
		Let $a$ be a positive number.  Show that $\log(ax)$ and $\log(x)$ are both antiderivatives of $\frac{1}{x}$ on $(0,\infty)$.  Conclude that $\log(ax)$ and $\log(x)$ must differ by a constant.  What constant do they differ by?  Is your discovery already familiar to you?
		
		\begin{free-response}
			\begin{align*}
				\frac{d}{dx} \log(ax) &= a\frac{1}{ax}\\
				&=\frac{1}{x}
				&=\frac{d}{dx} \log(x)
			\end{align*}
			
			Thus $\log(ax)$ and $\log(x)$are both antiderivatives of $\frac{1}{x}$ on $(0,\infty)$.
			
			Thus they must differ by a constant, so $\log(ax) = \log(x)+C$ for some constant $C$.
			
			Plugging in $x=1$, we have $\log(a) = \log(1)+C$, so $\log(a) = C$.
			
			Thus we can conclude that $\log(ax) = \log(x)+\log(a)$.
			
			This is one of the laws of logarithms!  It is kind of neat that we can derive this law of logarithms in this way.
			
		\end{free-response}
	\end{question}
	
	

\end{document}