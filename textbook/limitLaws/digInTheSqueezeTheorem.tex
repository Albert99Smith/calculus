\documentclass{ximera}

%\usepackage{todonotes}
%\usepackage{mathtools} %% Required for wide table Curl and Greens
%\usepackage{cuted} %% Required for wide table Curl and Greens
\newcommand{\todo}{}

\usepackage{esint} % for \oiint
\ifxake%%https://math.meta.stackexchange.com/questions/9973/how-do-you-render-a-closed-surface-double-integral
\renewcommand{\oiint}{{\large\bigcirc}\kern-1.56em\iint}
\fi


\graphicspath{
  {./}
  {ximeraTutorial/}
  {basicPhilosophy/}
  {functionsOfSeveralVariables/}
  {normalVectors/}
  {lagrangeMultipliers/}
  {vectorFields/}
  {greensTheorem/}
  {shapeOfThingsToCome/}
  {dotProducts/}
  {partialDerivativesAndTheGradientVector/}
  {../productAndQuotientRules/exercises/}
  {../normalVectors/exercisesParametricPlots/}
  {../continuityOfFunctionsOfSeveralVariables/exercises/}
  {../partialDerivativesAndTheGradientVector/exercises/}
  {../directionalDerivativeAndChainRule/exercises/}
  {../commonCoordinates/exercisesCylindricalCoordinates/}
  {../commonCoordinates/exercisesSphericalCoordinates/}
  {../greensTheorem/exercisesCurlAndLineIntegrals/}
  {../greensTheorem/exercisesDivergenceAndLineIntegrals/}
  {../shapeOfThingsToCome/exercisesDivergenceTheorem/}
  {../greensTheorem/}
  {../shapeOfThingsToCome/}
  {../separableDifferentialEquations/exercises/}
  {vectorFields/}
}

\newcommand{\mooculus}{\textsf{\textbf{MOOC}\textnormal{\textsf{ULUS}}}}

\usepackage{tkz-euclide}\usepackage{tikz}
\usepackage{tikz-cd}
\usetikzlibrary{arrows}
\tikzset{>=stealth,commutative diagrams/.cd,
  arrow style=tikz,diagrams={>=stealth}} %% cool arrow head
\tikzset{shorten <>/.style={ shorten >=#1, shorten <=#1 } } %% allows shorter vectors

\usetikzlibrary{backgrounds} %% for boxes around graphs
\usetikzlibrary{shapes,positioning}  %% Clouds and stars
\usetikzlibrary{matrix} %% for matrix
\usepgfplotslibrary{polar} %% for polar plots
\usepgfplotslibrary{fillbetween} %% to shade area between curves in TikZ
\usetkzobj{all}
\usepackage[makeroom]{cancel} %% for strike outs
%\usepackage{mathtools} %% for pretty underbrace % Breaks Ximera
%\usepackage{multicol}
\usepackage{pgffor} %% required for integral for loops



%% http://tex.stackexchange.com/questions/66490/drawing-a-tikz-arc-specifying-the-center
%% Draws beach ball
\tikzset{pics/carc/.style args={#1:#2:#3}{code={\draw[pic actions] (#1:#3) arc(#1:#2:#3);}}}



\usepackage{array}
\setlength{\extrarowheight}{+.1cm}
\newdimen\digitwidth
\settowidth\digitwidth{9}
\def\divrule#1#2{
\noalign{\moveright#1\digitwidth
\vbox{\hrule width#2\digitwidth}}}





\newcommand{\RR}{\mathbb R}
\newcommand{\R}{\mathbb R}
\newcommand{\N}{\mathbb N}
\newcommand{\Z}{\mathbb Z}

\newcommand{\sagemath}{\textsf{SageMath}}


%\renewcommand{\d}{\,d\!}
\renewcommand{\d}{\mathop{}\!d}
\newcommand{\dd}[2][]{\frac{\d #1}{\d #2}}
\newcommand{\pp}[2][]{\frac{\partial #1}{\partial #2}}
\renewcommand{\l}{\ell}
\newcommand{\ddx}{\frac{d}{\d x}}

\newcommand{\zeroOverZero}{\ensuremath{\boldsymbol{\tfrac{0}{0}}}}
\newcommand{\inftyOverInfty}{\ensuremath{\boldsymbol{\tfrac{\infty}{\infty}}}}
\newcommand{\zeroOverInfty}{\ensuremath{\boldsymbol{\tfrac{0}{\infty}}}}
\newcommand{\zeroTimesInfty}{\ensuremath{\small\boldsymbol{0\cdot \infty}}}
\newcommand{\inftyMinusInfty}{\ensuremath{\small\boldsymbol{\infty - \infty}}}
\newcommand{\oneToInfty}{\ensuremath{\boldsymbol{1^\infty}}}
\newcommand{\zeroToZero}{\ensuremath{\boldsymbol{0^0}}}
\newcommand{\inftyToZero}{\ensuremath{\boldsymbol{\infty^0}}}



\newcommand{\numOverZero}{\ensuremath{\boldsymbol{\tfrac{\#}{0}}}}
\newcommand{\dfn}{\textbf}
%\newcommand{\unit}{\,\mathrm}
\newcommand{\unit}{\mathop{}\!\mathrm}
\newcommand{\eval}[1]{\bigg[ #1 \bigg]}
\newcommand{\seq}[1]{\left( #1 \right)}
\renewcommand{\epsilon}{\varepsilon}
\renewcommand{\phi}{\varphi}


\renewcommand{\iff}{\Leftrightarrow}

\DeclareMathOperator{\arccot}{arccot}
\DeclareMathOperator{\arcsec}{arcsec}
\DeclareMathOperator{\arccsc}{arccsc}
\DeclareMathOperator{\si}{Si}
\DeclareMathOperator{\scal}{scal}
\DeclareMathOperator{\sign}{sign}


%% \newcommand{\tightoverset}[2]{% for arrow vec
%%   \mathop{#2}\limits^{\vbox to -.5ex{\kern-0.75ex\hbox{$#1$}\vss}}}
\newcommand{\arrowvec}[1]{{\overset{\rightharpoonup}{#1}}}
%\renewcommand{\vec}[1]{\arrowvec{\mathbf{#1}}}
\renewcommand{\vec}[1]{{\overset{\boldsymbol{\rightharpoonup}}{\mathbf{#1}}}\hspace{0in}}

\newcommand{\point}[1]{\left(#1\right)} %this allows \vector{ to be changed to \vector{ with a quick find and replace
\newcommand{\pt}[1]{\mathbf{#1}} %this allows \vec{ to be changed to \vec{ with a quick find and replace
\newcommand{\Lim}[2]{\lim_{\point{#1} \to \point{#2}}} %Bart, I changed this to point since I want to use it.  It runs through both of the exercise and exerciseE files in limits section, which is why it was in each document to start with.

\DeclareMathOperator{\proj}{\mathbf{proj}}
\newcommand{\veci}{{\boldsymbol{\hat{\imath}}}}
\newcommand{\vecj}{{\boldsymbol{\hat{\jmath}}}}
\newcommand{\veck}{{\boldsymbol{\hat{k}}}}
\newcommand{\vecl}{\vec{\boldsymbol{\l}}}
\newcommand{\uvec}[1]{\mathbf{\hat{#1}}}
\newcommand{\utan}{\mathbf{\hat{t}}}
\newcommand{\unormal}{\mathbf{\hat{n}}}
\newcommand{\ubinormal}{\mathbf{\hat{b}}}

\newcommand{\dotp}{\bullet}
\newcommand{\cross}{\boldsymbol\times}
\newcommand{\grad}{\boldsymbol\nabla}
\newcommand{\divergence}{\grad\dotp}
\newcommand{\curl}{\grad\cross}
%\DeclareMathOperator{\divergence}{divergence}
%\DeclareMathOperator{\curl}[1]{\grad\cross #1}
\newcommand{\lto}{\mathop{\longrightarrow\,}\limits}

\renewcommand{\bar}{\overline}

\colorlet{textColor}{black}
\colorlet{background}{white}
\colorlet{penColor}{blue!50!black} % Color of a curve in a plot
\colorlet{penColor2}{red!50!black}% Color of a curve in a plot
\colorlet{penColor3}{red!50!blue} % Color of a curve in a plot
\colorlet{penColor4}{green!50!black} % Color of a curve in a plot
\colorlet{penColor5}{orange!80!black} % Color of a curve in a plot
\colorlet{penColor6}{yellow!70!black} % Color of a curve in a plot
\colorlet{fill1}{penColor!20} % Color of fill in a plot
\colorlet{fill2}{penColor2!20} % Color of fill in a plot
\colorlet{fillp}{fill1} % Color of positive area
\colorlet{filln}{penColor2!20} % Color of negative area
\colorlet{fill3}{penColor3!20} % Fill
\colorlet{fill4}{penColor4!20} % Fill
\colorlet{fill5}{penColor5!20} % Fill
\colorlet{gridColor}{gray!50} % Color of grid in a plot

\newcommand{\surfaceColor}{violet}
\newcommand{\surfaceColorTwo}{redyellow}
\newcommand{\sliceColor}{greenyellow}




\pgfmathdeclarefunction{gauss}{2}{% gives gaussian
  \pgfmathparse{1/(#2*sqrt(2*pi))*exp(-((x-#1)^2)/(2*#2^2))}%
}


%%%%%%%%%%%%%
%% Vectors
%%%%%%%%%%%%%

%% Simple horiz vectors
\renewcommand{\vector}[1]{\left\langle #1\right\rangle}


%% %% Complex Horiz Vectors with angle brackets
%% \makeatletter
%% \renewcommand{\vector}[2][ , ]{\left\langle%
%%   \def\nextitem{\def\nextitem{#1}}%
%%   \@for \el:=#2\do{\nextitem\el}\right\rangle%
%% }
%% \makeatother

%% %% Vertical Vectors
%% \def\vector#1{\begin{bmatrix}\vecListA#1,,\end{bmatrix}}
%% \def\vecListA#1,{\if,#1,\else #1\cr \expandafter \vecListA \fi}

%%%%%%%%%%%%%
%% End of vectors
%%%%%%%%%%%%%

%\newcommand{\fullwidth}{}
%\newcommand{\normalwidth}{}



%% makes a snazzy t-chart for evaluating functions
%\newenvironment{tchart}{\rowcolors{2}{}{background!90!textColor}\array}{\endarray}

%%This is to help with formatting on future title pages.
\newenvironment{sectionOutcomes}{}{}



%% Flowchart stuff
%\tikzstyle{startstop} = [rectangle, rounded corners, minimum width=3cm, minimum height=1cm,text centered, draw=black]
%\tikzstyle{question} = [rectangle, minimum width=3cm, minimum height=1cm, text centered, draw=black]
%\tikzstyle{decision} = [trapezium, trapezium left angle=70, trapezium right angle=110, minimum width=3cm, minimum height=1cm, text centered, draw=black]
%\tikzstyle{question} = [rectangle, rounded corners, minimum width=3cm, minimum height=1cm,text centered, draw=black]
%\tikzstyle{process} = [rectangle, minimum width=3cm, minimum height=1cm, text centered, draw=black]
%\tikzstyle{decision} = [trapezium, trapezium left angle=70, trapezium right angle=110, minimum width=3cm, minimum height=1cm, text centered, draw=black]


\outcome{Understand the Squeeze Theorem and how it can be used to find limit values.}
\outcome{Calculate limits using the Squeeze Theorem.}

\title[Dig-In:]{The squeeze theorem}

\begin{document}
\begin{abstract}
The Squeeze theorem allows us to exchange difficult functions for easy functions.
\end{abstract}
\maketitle

In mathematics, sometimes we can study complex functions by exchanging
them for simplier functions. The \textit{Squeeze Theorem} tells us one
situation where this is possible.

\begin{theorem}[Squeeze Theorem]
  Suppose that
  \[
  g(x) \le f(x) \le h(x)
  \]
  for all $x$ close to $a$ but not necessarily equal to $a$. If
  \[
  \lim_{x\to a} g(x) = L = \lim_{x\to a} h(x),
  \] 
  then $\lim_{x\to a} f(x) = L$.
\end{theorem}

\begin{question}
  I'm thinking of a function $f$. I know that for all $x$
  \[
  0 \le f(x) \le x^2.
  \]
  What is $\lim_{x\to 0} f(x)$?
  \begin{multipleChoice}
    \choice{$f(x)$}
    \choice{$f(0)$}
    \choice[correct]{$0$}
    \choice{impossible to say}
  \end{multipleChoice}
\end{question}



\begin{example}
Consider the function
\[
f(x) = 
\begin{cases}
\sqrt[5]{x}\sin\left(\frac{1}{x}\right) & \text{if $x \ne 0$,}\\
0 & \text{if $x = 0$,}
\end{cases}
\]
\begin{image}
\begin{tikzpicture}
	\begin{axis}[
            domain=-.2:.2,    
            samples=500,
            axis lines =middle, xlabel=$x$, ylabel=$y$,
            yticklabels = {}, 
            every axis y label/.style={at=(current axis.above origin),anchor=south},
            every axis x label/.style={at=(current axis.right of origin),anchor=west},
            clip=false,
          ]
	  \addplot [very thick, penColor, smooth, domain=(-.2:-.02)] {abs(x)^(1/5)*sin(deg(1/x))};
          \addplot [very thick, penColor, smooth, domain=(.02:.2)] {x^(1/5)*sin(deg(1/x))};
	  \addplot [color=penColor, fill=penColor, very thick, smooth,domain=(-.02:.02)] {abs(x)^(1/5)} \closedcycle;
          \addplot [color=penColor, fill=penColor, very thick, smooth,domain=(-.02:.02)] {-abs(x)^(1/5)} \closedcycle;
        \end{axis}
\end{tikzpicture}
%% \caption[A continuous function.]{A plot of
%% \[
%% f(x)=
%% \begin{cases}
%% \sqrt[5]{x}\sin\left(\frac{1}{x}\right) & \text{if $x \ne 0$,}\\
%%  0 & \text{if $x = 0$.}
%% \end{cases}
%% \]
%% }
\end{image}
Is this function continuous at $x=0$?
\begin{explanation}
We must show that $\lim_{x\to 0} f(x) = \answer[given]{0}$. Note
\[
-|\sqrt[5]{x}|\le f(x) \le |\sqrt[5]{x}|.
\]
Since
\[
\lim_{x\to 0} -|\sqrt[5]{x}| = \answer[given]{0} = \lim_{x\to 0}|\sqrt[5]{x}|,
\]
we see by the Squeeze Theorem, Theorem, that
$\lim_{x\to 0} f(x) = \answer[given]{0}$. Hence $f(x)$ is continuous.

Here we see how the informal definition of continuity being that you
can ``draw it'' without ``lifting your pencil'' differs from the
formal definition.
\end{explanation}
\end{example}




\begin{example}
Compute:
\[
\lim_{\theta\to 0} \frac{\sin(\theta)}{\theta}
\]
\begin{explanation}
To compute this limit, use the Squeeze Theorem. First note that we
only need to examine $\theta\in \left(\frac{-\pi}{2}, \frac{\pi}{2}\right)$
and for the present time, we'll assume that $\theta$ is positive---consider
the diagrams below:

\begin{image}
\begin{tabular}{ccc}
\begin{tikzpicture}
	\begin{axis}[
            xmin=-.1,xmax=1.1,ymin=-.1,ymax=1.1,
            axis lines=center,
            ticks=none,
            unit vector ratio*=1 1 1,
            xlabel=$x$, ylabel=$y$,
            every axis y label/.style={at=(current axis.above origin),anchor=south},
            every axis x label/.style={at=(current axis.right of origin),anchor=west},
          ]        
          \addplot [very thick, penColor2, smooth, domain=(-.2:.2+pi/2)] ({cos(deg(x))},{sin(deg(x))});
          \addplot [textColor] plot coordinates {(0,0) (1,.839)}; %% 40 degrees
          \addplot [very thick, penColor] plot coordinates {(.766,0) (.766,.643)}; %% 40 degrees
          \addplot [textColor] plot coordinates {(1,0) (1,.839)}; %% 40 degrees
          \addplot [very thick,penColor,fill=fill1] plot coordinates {(0,0) (.766,.643)}\closedcycle; %% triangle
          \addplot [textColor,smooth, domain=(0:40)] ({.15*cos(x)},{.15*sin(x)});
          \node at (axis cs:.15,.07) [anchor=west] {$\theta$};
          \node at (axis cs:.766,.322) [anchor=east] {$\sin(\theta)$};
          \node at (axis cs:.383,0) [anchor=north] {$\cos(\theta)$};
        \end{axis}
\end{tikzpicture} & 
\begin{tikzpicture}
	\begin{axis}[
            xmin=-.1,xmax=1.1,ymin=-.1,ymax=1.1,
            axis lines=center,
            ticks=none,
            unit vector ratio*=1 1 1,
            xlabel=$x$, ylabel=$y$,
            every axis y label/.style={at=(current axis.above origin),anchor=south},
            every axis x label/.style={at=(current axis.right of origin),anchor=west},
          ]        
          \addplot [draw=none,fill=fill1] plot coordinates {(0,0) (.766,.643)}\closedcycle; %% sector
          \addplot [draw=none, fill=fill1, samples=100, domain=(0:40)] ({cos(x)},{sin(x)})\closedcycle; %% sector 
          \addplot [very thick, penColor2, smooth, domain=(-.2:.2+pi/2)] ({cos(deg(x))},{sin(deg(x))});
          \addplot [textColor] plot coordinates {(0,0) (1,.839)}; %% 40 degrees
          \addplot [textColor] plot coordinates {(.766,0) (.766,.643)}; %% 40 degrees
          \addplot [textColor] plot coordinates {(1,0) (1,.839)}; %% 40 degrees          
          \addplot [textColor,smooth, domain=(0:40)] ({.15*cos(x)},{.15*sin(x)});
          \addplot [very thick,penColor] plot coordinates {(0,0) (.766,.643)}; %% sector
          \addplot [very thick,penColor] plot coordinates {(0,0) (1,0)}; %% sector
          \addplot [very thick, penColor, smooth, domain=(0:40)] ({cos(x)},{sin(x)}); %% sector
          \node at (axis cs:.15,.07) [anchor=west] {$\theta$};
          \node at (axis cs:.5,0) [anchor=north] {$1$};
        \end{axis}
\end{tikzpicture} & 
\begin{tikzpicture}
	\begin{axis}[clip=false,
            xmin=-.1,xmax=1.1,ymin=-.1,ymax=1.1,
            axis lines=center,
            ticks=none,
            unit vector ratio*=1 1 1,
            xlabel=$x$, ylabel=$y$,
            every axis y label/.style={at=(current axis.above origin),anchor=south},
            every axis x label/.style={at=(current axis.right of origin),anchor=west},
          ]        
          \addplot [very thick,penColor,fill=fill1] plot coordinates {(0,0) (1,.839)}\closedcycle; %% triangle          
          \addplot [very thick, penColor2, smooth, domain=(-.1:1.671)] ({cos(deg(x))},{sin(deg(x))});
          \addplot [very thick, penColor] plot coordinates {(0,0) (1,.839)}; %% 40 degrees
          \addplot [textColor] plot coordinates {(.766,0) (.766,.643)}; %% 40 degrees
          \addplot [very thick, penColor] plot coordinates {(1,0) (1,.839)}; %% 40 degrees          
          \addplot [textColor,smooth, domain=(0:40)] ({.15*cos(x)},{.15*sin(x)});
          \node at (axis cs:.15,.07) [anchor=west] {$\theta$};
          \node at (axis cs:.5,0) [anchor=north] {$1$};
          \node at (axis cs:1,.42) [anchor=west] {$\tan(\theta)$};
        \end{axis}
\end{tikzpicture}\\
Triangle $A$ & Sector & Triangle $B$ \\
\end{tabular}
\end{image}


From our diagrams above we see that
\[
\text{Area of Triangle $A$} \le \text{Area of Sector} \le \text{Area of Triangle $B$}
\]
and computing these areas we find
\[
\frac{\cos(\theta)\sin(\theta)}{2} \le \frac{\theta}{2} \le \frac{\tan(\theta)}{2}.
\]
Multiplying through by $2$, and recalling that $\tan(\theta) =
\frac{\sin(\theta)}{\cos(\theta)}$ we obtain
\[
\cos(\theta)\sin(\theta) \le \theta \le \frac{\sin(\theta)}{\cos(\theta)}.
\]
Dividing through by $\sin(\theta)$ and taking the reciprocals
(reversing the inequalities), we find
\[
\cos(\theta) \le \frac{\sin(\theta)}{\theta} \le \frac{1}{\cos(\theta)}.
\]
Note, $\cos(-\theta) = \cos(\theta)$ and $\frac{\sin(-\theta)}{-\theta} =
\frac{\sin(\theta)}{\theta}$, so these inequalities hold for all $\theta\in
\left(\frac{-\pi}{2}, \frac{\pi}{2}\right)$.  Additionally, we know
\[
\lim_{\theta \to 0}\cos(\theta) = \answer[given]{1} = \lim_{\theta\to 0}\frac{1}{\cos(\theta)},
\]
and so we conclude by the Squeeze Theorem, $\lim_{\theta \to
  0}\frac{\sin(\theta)}{\theta} = \answer[given]{1}$.
\end{explanation}
\end{example}



\end{document}
