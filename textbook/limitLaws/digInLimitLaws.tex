\documentclass{ximera}

\outcome{Learn things.}

%\usepackage{todonotes}
%\usepackage{mathtools} %% Required for wide table Curl and Greens
%\usepackage{cuted} %% Required for wide table Curl and Greens
\newcommand{\todo}{}

\usepackage{esint} % for \oiint
\ifxake%%https://math.meta.stackexchange.com/questions/9973/how-do-you-render-a-closed-surface-double-integral
\renewcommand{\oiint}{{\large\bigcirc}\kern-1.56em\iint}
\fi


\graphicspath{
  {./}
  {ximeraTutorial/}
  {basicPhilosophy/}
  {functionsOfSeveralVariables/}
  {normalVectors/}
  {lagrangeMultipliers/}
  {vectorFields/}
  {greensTheorem/}
  {shapeOfThingsToCome/}
  {dotProducts/}
  {partialDerivativesAndTheGradientVector/}
  {../productAndQuotientRules/exercises/}
  {../normalVectors/exercisesParametricPlots/}
  {../continuityOfFunctionsOfSeveralVariables/exercises/}
  {../partialDerivativesAndTheGradientVector/exercises/}
  {../directionalDerivativeAndChainRule/exercises/}
  {../commonCoordinates/exercisesCylindricalCoordinates/}
  {../commonCoordinates/exercisesSphericalCoordinates/}
  {../greensTheorem/exercisesCurlAndLineIntegrals/}
  {../greensTheorem/exercisesDivergenceAndLineIntegrals/}
  {../shapeOfThingsToCome/exercisesDivergenceTheorem/}
  {../greensTheorem/}
  {../shapeOfThingsToCome/}
  {../separableDifferentialEquations/exercises/}
  {vectorFields/}
}

\newcommand{\mooculus}{\textsf{\textbf{MOOC}\textnormal{\textsf{ULUS}}}}

\usepackage{tkz-euclide}\usepackage{tikz}
\usepackage{tikz-cd}
\usetikzlibrary{arrows}
\tikzset{>=stealth,commutative diagrams/.cd,
  arrow style=tikz,diagrams={>=stealth}} %% cool arrow head
\tikzset{shorten <>/.style={ shorten >=#1, shorten <=#1 } } %% allows shorter vectors

\usetikzlibrary{backgrounds} %% for boxes around graphs
\usetikzlibrary{shapes,positioning}  %% Clouds and stars
\usetikzlibrary{matrix} %% for matrix
\usepgfplotslibrary{polar} %% for polar plots
\usepgfplotslibrary{fillbetween} %% to shade area between curves in TikZ
\usetkzobj{all}
\usepackage[makeroom]{cancel} %% for strike outs
%\usepackage{mathtools} %% for pretty underbrace % Breaks Ximera
%\usepackage{multicol}
\usepackage{pgffor} %% required for integral for loops



%% http://tex.stackexchange.com/questions/66490/drawing-a-tikz-arc-specifying-the-center
%% Draws beach ball
\tikzset{pics/carc/.style args={#1:#2:#3}{code={\draw[pic actions] (#1:#3) arc(#1:#2:#3);}}}



\usepackage{array}
\setlength{\extrarowheight}{+.1cm}
\newdimen\digitwidth
\settowidth\digitwidth{9}
\def\divrule#1#2{
\noalign{\moveright#1\digitwidth
\vbox{\hrule width#2\digitwidth}}}





\newcommand{\RR}{\mathbb R}
\newcommand{\R}{\mathbb R}
\newcommand{\N}{\mathbb N}
\newcommand{\Z}{\mathbb Z}

\newcommand{\sagemath}{\textsf{SageMath}}


%\renewcommand{\d}{\,d\!}
\renewcommand{\d}{\mathop{}\!d}
\newcommand{\dd}[2][]{\frac{\d #1}{\d #2}}
\newcommand{\pp}[2][]{\frac{\partial #1}{\partial #2}}
\renewcommand{\l}{\ell}
\newcommand{\ddx}{\frac{d}{\d x}}

\newcommand{\zeroOverZero}{\ensuremath{\boldsymbol{\tfrac{0}{0}}}}
\newcommand{\inftyOverInfty}{\ensuremath{\boldsymbol{\tfrac{\infty}{\infty}}}}
\newcommand{\zeroOverInfty}{\ensuremath{\boldsymbol{\tfrac{0}{\infty}}}}
\newcommand{\zeroTimesInfty}{\ensuremath{\small\boldsymbol{0\cdot \infty}}}
\newcommand{\inftyMinusInfty}{\ensuremath{\small\boldsymbol{\infty - \infty}}}
\newcommand{\oneToInfty}{\ensuremath{\boldsymbol{1^\infty}}}
\newcommand{\zeroToZero}{\ensuremath{\boldsymbol{0^0}}}
\newcommand{\inftyToZero}{\ensuremath{\boldsymbol{\infty^0}}}



\newcommand{\numOverZero}{\ensuremath{\boldsymbol{\tfrac{\#}{0}}}}
\newcommand{\dfn}{\textbf}
%\newcommand{\unit}{\,\mathrm}
\newcommand{\unit}{\mathop{}\!\mathrm}
\newcommand{\eval}[1]{\bigg[ #1 \bigg]}
\newcommand{\seq}[1]{\left( #1 \right)}
\renewcommand{\epsilon}{\varepsilon}
\renewcommand{\phi}{\varphi}


\renewcommand{\iff}{\Leftrightarrow}

\DeclareMathOperator{\arccot}{arccot}
\DeclareMathOperator{\arcsec}{arcsec}
\DeclareMathOperator{\arccsc}{arccsc}
\DeclareMathOperator{\si}{Si}
\DeclareMathOperator{\scal}{scal}
\DeclareMathOperator{\sign}{sign}


%% \newcommand{\tightoverset}[2]{% for arrow vec
%%   \mathop{#2}\limits^{\vbox to -.5ex{\kern-0.75ex\hbox{$#1$}\vss}}}
\newcommand{\arrowvec}[1]{{\overset{\rightharpoonup}{#1}}}
%\renewcommand{\vec}[1]{\arrowvec{\mathbf{#1}}}
\renewcommand{\vec}[1]{{\overset{\boldsymbol{\rightharpoonup}}{\mathbf{#1}}}\hspace{0in}}

\newcommand{\point}[1]{\left(#1\right)} %this allows \vector{ to be changed to \vector{ with a quick find and replace
\newcommand{\pt}[1]{\mathbf{#1}} %this allows \vec{ to be changed to \vec{ with a quick find and replace
\newcommand{\Lim}[2]{\lim_{\point{#1} \to \point{#2}}} %Bart, I changed this to point since I want to use it.  It runs through both of the exercise and exerciseE files in limits section, which is why it was in each document to start with.

\DeclareMathOperator{\proj}{\mathbf{proj}}
\newcommand{\veci}{{\boldsymbol{\hat{\imath}}}}
\newcommand{\vecj}{{\boldsymbol{\hat{\jmath}}}}
\newcommand{\veck}{{\boldsymbol{\hat{k}}}}
\newcommand{\vecl}{\vec{\boldsymbol{\l}}}
\newcommand{\uvec}[1]{\mathbf{\hat{#1}}}
\newcommand{\utan}{\mathbf{\hat{t}}}
\newcommand{\unormal}{\mathbf{\hat{n}}}
\newcommand{\ubinormal}{\mathbf{\hat{b}}}

\newcommand{\dotp}{\bullet}
\newcommand{\cross}{\boldsymbol\times}
\newcommand{\grad}{\boldsymbol\nabla}
\newcommand{\divergence}{\grad\dotp}
\newcommand{\curl}{\grad\cross}
%\DeclareMathOperator{\divergence}{divergence}
%\DeclareMathOperator{\curl}[1]{\grad\cross #1}
\newcommand{\lto}{\mathop{\longrightarrow\,}\limits}

\renewcommand{\bar}{\overline}

\colorlet{textColor}{black}
\colorlet{background}{white}
\colorlet{penColor}{blue!50!black} % Color of a curve in a plot
\colorlet{penColor2}{red!50!black}% Color of a curve in a plot
\colorlet{penColor3}{red!50!blue} % Color of a curve in a plot
\colorlet{penColor4}{green!50!black} % Color of a curve in a plot
\colorlet{penColor5}{orange!80!black} % Color of a curve in a plot
\colorlet{penColor6}{yellow!70!black} % Color of a curve in a plot
\colorlet{fill1}{penColor!20} % Color of fill in a plot
\colorlet{fill2}{penColor2!20} % Color of fill in a plot
\colorlet{fillp}{fill1} % Color of positive area
\colorlet{filln}{penColor2!20} % Color of negative area
\colorlet{fill3}{penColor3!20} % Fill
\colorlet{fill4}{penColor4!20} % Fill
\colorlet{fill5}{penColor5!20} % Fill
\colorlet{gridColor}{gray!50} % Color of grid in a plot

\newcommand{\surfaceColor}{violet}
\newcommand{\surfaceColorTwo}{redyellow}
\newcommand{\sliceColor}{greenyellow}




\pgfmathdeclarefunction{gauss}{2}{% gives gaussian
  \pgfmathparse{1/(#2*sqrt(2*pi))*exp(-((x-#1)^2)/(2*#2^2))}%
}


%%%%%%%%%%%%%
%% Vectors
%%%%%%%%%%%%%

%% Simple horiz vectors
\renewcommand{\vector}[1]{\left\langle #1\right\rangle}


%% %% Complex Horiz Vectors with angle brackets
%% \makeatletter
%% \renewcommand{\vector}[2][ , ]{\left\langle%
%%   \def\nextitem{\def\nextitem{#1}}%
%%   \@for \el:=#2\do{\nextitem\el}\right\rangle%
%% }
%% \makeatother

%% %% Vertical Vectors
%% \def\vector#1{\begin{bmatrix}\vecListA#1,,\end{bmatrix}}
%% \def\vecListA#1,{\if,#1,\else #1\cr \expandafter \vecListA \fi}

%%%%%%%%%%%%%
%% End of vectors
%%%%%%%%%%%%%

%\newcommand{\fullwidth}{}
%\newcommand{\normalwidth}{}



%% makes a snazzy t-chart for evaluating functions
%\newenvironment{tchart}{\rowcolors{2}{}{background!90!textColor}\array}{\endarray}

%%This is to help with formatting on future title pages.
\newenvironment{sectionOutcomes}{}{}



%% Flowchart stuff
%\tikzstyle{startstop} = [rectangle, rounded corners, minimum width=3cm, minimum height=1cm,text centered, draw=black]
%\tikzstyle{question} = [rectangle, minimum width=3cm, minimum height=1cm, text centered, draw=black]
%\tikzstyle{decision} = [trapezium, trapezium left angle=70, trapezium right angle=110, minimum width=3cm, minimum height=1cm, text centered, draw=black]
%\tikzstyle{question} = [rectangle, rounded corners, minimum width=3cm, minimum height=1cm,text centered, draw=black]
%\tikzstyle{process} = [rectangle, minimum width=3cm, minimum height=1cm, text centered, draw=black]
%\tikzstyle{decision} = [trapezium, trapezium left angle=70, trapezium right angle=110, minimum width=3cm, minimum height=1cm, text centered, draw=black]

\title[Dig-In:]{Limit laws}
\begin{document}
\begin{abstract}
We give basic laws for working with limits. 
\end{abstract}
\maketitle

In this section, we present a handful of tools to compute many limits
without explicitly working with the definition of limit.  We learned
previously that just estimating limits by plugging in nearby values
does not always give the correct answer.  We need some theorems that
tell us how to quickly find limits, at least in the most common cases.
Each of these could be proved directly as we did in the previous
section. We start by listing a set of basic limits. 


\begin{theorem}[Limit Laws: Basic Limits]
  Here are some basic limits:
  \begin{description}
  \item[\textbf{Limit of a Constant}] $\lim_{x\to a} k = k$ where $k$
    is a constant..
  \item[\textbf{Limit of $\boldsymbol{x}$}] $\lim_{x\to a}x =a$.
  \item[\textbf{Limit of $\boldsymbol{\sin(x)}$}] $\lim_{x\to a}
    \sin(x) = \sin(a)$.
  \item[\textbf{Limit of $\boldsymbol{b^x}$}] $\lim_{x\to a} b^x =
    b^a$ provided $b\ne 0$.
  \item[\textbf{Limit of $\boldsymbol{\log_b(x)}$}] $\lim_{x\to a}
    \log_b(x) = \log_b(a)$ provided $a$ and $b$ are both greater than
    zero.
  \end{description}
\end{theorem}

\begin{question}
  For the ``Limit of $b^x$'' why is it necessary that $b\ne 0$?
  \begin{hint}
    Suppose that $b=0$. Compute:
    \[
    0^1
    \begin{prompt}
      = \answer[given]{0}
    \end{prompt}
    \qquad
    0^{0.1}
    \begin{prompt}
      = \answer[given]{0}
    \end{prompt}
    \qquad
    0^{0.01}
    \begin{prompt}
      = \answer[given]{0}
    \end{prompt}
    \]
    However, we should compare this to
    \[
    0^0
    \begin{prompt}
      = \answer[given]{1}
    \end{prompt}
    \]
  \end{hint}
  \begin{freeResponse}[given]
    If $b=0$, then $0^x = 0$ for all $x$ except $x=0$. On the other
    hand, $0^0 = 1$.
  \end{freeResponse}
\end{question}

Now we will give some basic rules describing how operations interact
with limits.

\begin{theorem}[Limit Laws: Operations]\index{limit laws}\label{theorem:limit-laws}
Suppose that $\lim_{x\to a}f(x)=L$, $\lim_{x\to a}g(x)=M$.
\begin{description}
%\item[\textbf{Constant Multiple Law}] $\lim_{x\to a} kf(x) = k\lim_{x\to a}f(x)=kL$.
\item[\textbf{Sum/Difference Law}] $\lim_{x\to a} (f(x) \pm g(x)) =
  \lim_{x\to a}f(x) \pm \lim_{x\to a}g(x)=L \pm M$.
\item[\textbf{Product Law}] $\lim_{x\to a} (f(x)g(x)) = \lim_{x\to
  a}f(x)\cdot\lim_{x\to a}g(x)=LM$.
\item[\textbf{Quotient Law}] $\lim_{x\to a} \frac{f(x)}{g(x)} =
  \frac{\lim_{x\to a}f(x)}{\lim_{x\to a}g(x)}=\frac{L}{M}$, if
  $M\ne0$.
\end{description}
\label{thm:limit laws}
\end{theorem}
\begin{question}
  Suppose $k$ is a constant, which limit laws are used in the
  computations below?
  \begin{explanation}%%BADBAD Need drop-down
    \begin{align*}
      \lim_{x\to a} k\cdot f(x) &= \lim_{x\to a} k \cdot \lim_{x\to a} f(x) && \answer[given]{Product Law}\\
      &= k \cdot L && \answer[given]{Limit of a Constant}
    \end{align*}
  \end{explanation}  
\end{question}


Finally, we give basic rules for how limits interact with composition
of functions.

\begin{theorem}[Limit Laws: Composition]\index{limit laws}\label{theorem:limit-laws}
Suppose that $\lim_{x\to a}f(x)=L$, $\lim_{x\to a}g(x)=M$, $k$ is some
constant, $b$ is a nonzero constant, and $n$ is a positive integer.
\begin{description}
\item[\textbf{Power Law}] $\lim_{x\to a} f(x)^n = \left(\lim_{x\to a}f(x)\right)^n=L^n$. 
\item[\textbf{Root Law}] $\lim_{x\to a} \sqrt[n]{f(x)}= \sqrt[n]{\lim_{x\to  a}f(x)}=\sqrt[n]{L}$
  provided that if $n$ is even, then $f(x)\ge 0$ near $a$.
\item[\textbf{General Composition Law}] If $\lim_{x\to a}g(x)=M$ and
  $\lim_{x\to M}f(x) = f(M)$, then $\lim_{x\to a} f(g(x)) = f(M)$.
\end{description}
\end{theorem}


\begin{example}
Find $\lim_{x\to1}(x-2)$.
\begin{explanation}
We do this ``by the book'' and explicitly use our limit laws. First write
\[
\lim_{x\to1}(x-2) = \lim_{x\to1}(x)-\lim_{x\to1}(2)
\]
What limit law was used in this step? $\answer[given]{\text{Sum/Difference Law}}$\\
\[
=\lim_{x\to1}(x)-2
\]
What limit law was used in this step? $\answer[given]{\text{Limit of a Constant}}$\\
\[
=1-2
\]
What limit law was used in this step? $\answer[given]{\text{Limit of $x$}}$\\
\[
=-1.
\]
Hence we se that $\lim_{x\to1}(x-2) = 1$.
%%%BADBAD This should probably be drop down from above
%%How do I make text answers?
\end{explanation}
\end{example}

This first example might seem trivial, but remember how we tricked
ourselves in the previous section.  The limit laws allow us to know
that as long as we follow them, we are actually getting the right
answer for the limit.

\begin{example}
Compute $\lim_{x\to 1}\frac{x^2-3x+5}{x-2}$. 
\begin{explanation}
Using limit laws, 
\begin{align*}
\lim_{x\to 1}\frac{x^2-3x+5}{x-2}&=
\frac{\lim_{x\to 1}(x^2-3x+5)}{\lim_{x\to1}(x-2)}  && \text{(assuming $\lim_{x\to1}(x-2) \neq 0$)} \\
&=\frac{\lim_{x\to 1}(x^2)-\lim_{x\to1}(3x)+\lim_{x\to1}(5)}{\lim_{x\to1}(x)-\lim_{x\to1}(2)} \\
&=\frac{\left(\lim_{x\to 1}x\right)^2-3(\lim_{x\to1}x)+5}{\lim_{x\to1}(x)-2} \\
&=\frac{1^2-3\cdot1+5}{1-2} \\
&=\frac{1-3+5}{-1} = -3.
\end{align*}
\textit{On your own}: Go through each step in this example an determine which limit laws were used.  It is important to make sure you are only using real limit laws and not making up your own rules!

%Can we use color in the book.  For example,  \text{\ \ (assuming $\lim_{x\to1}(x-2) \neq 0$)}  should be a different color.
%% BADBAD For the print book, we *could* but probably don't want to, since printing would be hard.
%% however, it would be good to do this for the online version...
\end{explanation}
\end{example}


\end{document}
