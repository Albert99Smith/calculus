\documentclass{ximera}

\outcome{Learn things.}

%\usepackage{todonotes}
%\usepackage{mathtools} %% Required for wide table Curl and Greens
%\usepackage{cuted} %% Required for wide table Curl and Greens
\newcommand{\todo}{}

\usepackage{esint} % for \oiint
\ifxake%%https://math.meta.stackexchange.com/questions/9973/how-do-you-render-a-closed-surface-double-integral
\renewcommand{\oiint}{{\large\bigcirc}\kern-1.56em\iint}
\fi


\graphicspath{
  {./}
  {ximeraTutorial/}
  {basicPhilosophy/}
  {functionsOfSeveralVariables/}
  {normalVectors/}
  {lagrangeMultipliers/}
  {vectorFields/}
  {greensTheorem/}
  {shapeOfThingsToCome/}
  {dotProducts/}
  {partialDerivativesAndTheGradientVector/}
  {../productAndQuotientRules/exercises/}
  {../normalVectors/exercisesParametricPlots/}
  {../continuityOfFunctionsOfSeveralVariables/exercises/}
  {../partialDerivativesAndTheGradientVector/exercises/}
  {../directionalDerivativeAndChainRule/exercises/}
  {../commonCoordinates/exercisesCylindricalCoordinates/}
  {../commonCoordinates/exercisesSphericalCoordinates/}
  {../greensTheorem/exercisesCurlAndLineIntegrals/}
  {../greensTheorem/exercisesDivergenceAndLineIntegrals/}
  {../shapeOfThingsToCome/exercisesDivergenceTheorem/}
  {../greensTheorem/}
  {../shapeOfThingsToCome/}
  {../separableDifferentialEquations/exercises/}
  {vectorFields/}
}

\newcommand{\mooculus}{\textsf{\textbf{MOOC}\textnormal{\textsf{ULUS}}}}

\usepackage{tkz-euclide}\usepackage{tikz}
\usepackage{tikz-cd}
\usetikzlibrary{arrows}
\tikzset{>=stealth,commutative diagrams/.cd,
  arrow style=tikz,diagrams={>=stealth}} %% cool arrow head
\tikzset{shorten <>/.style={ shorten >=#1, shorten <=#1 } } %% allows shorter vectors

\usetikzlibrary{backgrounds} %% for boxes around graphs
\usetikzlibrary{shapes,positioning}  %% Clouds and stars
\usetikzlibrary{matrix} %% for matrix
\usepgfplotslibrary{polar} %% for polar plots
\usepgfplotslibrary{fillbetween} %% to shade area between curves in TikZ
\usetkzobj{all}
\usepackage[makeroom]{cancel} %% for strike outs
%\usepackage{mathtools} %% for pretty underbrace % Breaks Ximera
%\usepackage{multicol}
\usepackage{pgffor} %% required for integral for loops



%% http://tex.stackexchange.com/questions/66490/drawing-a-tikz-arc-specifying-the-center
%% Draws beach ball
\tikzset{pics/carc/.style args={#1:#2:#3}{code={\draw[pic actions] (#1:#3) arc(#1:#2:#3);}}}



\usepackage{array}
\setlength{\extrarowheight}{+.1cm}
\newdimen\digitwidth
\settowidth\digitwidth{9}
\def\divrule#1#2{
\noalign{\moveright#1\digitwidth
\vbox{\hrule width#2\digitwidth}}}





\newcommand{\RR}{\mathbb R}
\newcommand{\R}{\mathbb R}
\newcommand{\N}{\mathbb N}
\newcommand{\Z}{\mathbb Z}

\newcommand{\sagemath}{\textsf{SageMath}}


%\renewcommand{\d}{\,d\!}
\renewcommand{\d}{\mathop{}\!d}
\newcommand{\dd}[2][]{\frac{\d #1}{\d #2}}
\newcommand{\pp}[2][]{\frac{\partial #1}{\partial #2}}
\renewcommand{\l}{\ell}
\newcommand{\ddx}{\frac{d}{\d x}}

\newcommand{\zeroOverZero}{\ensuremath{\boldsymbol{\tfrac{0}{0}}}}
\newcommand{\inftyOverInfty}{\ensuremath{\boldsymbol{\tfrac{\infty}{\infty}}}}
\newcommand{\zeroOverInfty}{\ensuremath{\boldsymbol{\tfrac{0}{\infty}}}}
\newcommand{\zeroTimesInfty}{\ensuremath{\small\boldsymbol{0\cdot \infty}}}
\newcommand{\inftyMinusInfty}{\ensuremath{\small\boldsymbol{\infty - \infty}}}
\newcommand{\oneToInfty}{\ensuremath{\boldsymbol{1^\infty}}}
\newcommand{\zeroToZero}{\ensuremath{\boldsymbol{0^0}}}
\newcommand{\inftyToZero}{\ensuremath{\boldsymbol{\infty^0}}}



\newcommand{\numOverZero}{\ensuremath{\boldsymbol{\tfrac{\#}{0}}}}
\newcommand{\dfn}{\textbf}
%\newcommand{\unit}{\,\mathrm}
\newcommand{\unit}{\mathop{}\!\mathrm}
\newcommand{\eval}[1]{\bigg[ #1 \bigg]}
\newcommand{\seq}[1]{\left( #1 \right)}
\renewcommand{\epsilon}{\varepsilon}
\renewcommand{\phi}{\varphi}


\renewcommand{\iff}{\Leftrightarrow}

\DeclareMathOperator{\arccot}{arccot}
\DeclareMathOperator{\arcsec}{arcsec}
\DeclareMathOperator{\arccsc}{arccsc}
\DeclareMathOperator{\si}{Si}
\DeclareMathOperator{\scal}{scal}
\DeclareMathOperator{\sign}{sign}


%% \newcommand{\tightoverset}[2]{% for arrow vec
%%   \mathop{#2}\limits^{\vbox to -.5ex{\kern-0.75ex\hbox{$#1$}\vss}}}
\newcommand{\arrowvec}[1]{{\overset{\rightharpoonup}{#1}}}
%\renewcommand{\vec}[1]{\arrowvec{\mathbf{#1}}}
\renewcommand{\vec}[1]{{\overset{\boldsymbol{\rightharpoonup}}{\mathbf{#1}}}\hspace{0in}}

\newcommand{\point}[1]{\left(#1\right)} %this allows \vector{ to be changed to \vector{ with a quick find and replace
\newcommand{\pt}[1]{\mathbf{#1}} %this allows \vec{ to be changed to \vec{ with a quick find and replace
\newcommand{\Lim}[2]{\lim_{\point{#1} \to \point{#2}}} %Bart, I changed this to point since I want to use it.  It runs through both of the exercise and exerciseE files in limits section, which is why it was in each document to start with.

\DeclareMathOperator{\proj}{\mathbf{proj}}
\newcommand{\veci}{{\boldsymbol{\hat{\imath}}}}
\newcommand{\vecj}{{\boldsymbol{\hat{\jmath}}}}
\newcommand{\veck}{{\boldsymbol{\hat{k}}}}
\newcommand{\vecl}{\vec{\boldsymbol{\l}}}
\newcommand{\uvec}[1]{\mathbf{\hat{#1}}}
\newcommand{\utan}{\mathbf{\hat{t}}}
\newcommand{\unormal}{\mathbf{\hat{n}}}
\newcommand{\ubinormal}{\mathbf{\hat{b}}}

\newcommand{\dotp}{\bullet}
\newcommand{\cross}{\boldsymbol\times}
\newcommand{\grad}{\boldsymbol\nabla}
\newcommand{\divergence}{\grad\dotp}
\newcommand{\curl}{\grad\cross}
%\DeclareMathOperator{\divergence}{divergence}
%\DeclareMathOperator{\curl}[1]{\grad\cross #1}
\newcommand{\lto}{\mathop{\longrightarrow\,}\limits}

\renewcommand{\bar}{\overline}

\colorlet{textColor}{black}
\colorlet{background}{white}
\colorlet{penColor}{blue!50!black} % Color of a curve in a plot
\colorlet{penColor2}{red!50!black}% Color of a curve in a plot
\colorlet{penColor3}{red!50!blue} % Color of a curve in a plot
\colorlet{penColor4}{green!50!black} % Color of a curve in a plot
\colorlet{penColor5}{orange!80!black} % Color of a curve in a plot
\colorlet{penColor6}{yellow!70!black} % Color of a curve in a plot
\colorlet{fill1}{penColor!20} % Color of fill in a plot
\colorlet{fill2}{penColor2!20} % Color of fill in a plot
\colorlet{fillp}{fill1} % Color of positive area
\colorlet{filln}{penColor2!20} % Color of negative area
\colorlet{fill3}{penColor3!20} % Fill
\colorlet{fill4}{penColor4!20} % Fill
\colorlet{fill5}{penColor5!20} % Fill
\colorlet{gridColor}{gray!50} % Color of grid in a plot

\newcommand{\surfaceColor}{violet}
\newcommand{\surfaceColorTwo}{redyellow}
\newcommand{\sliceColor}{greenyellow}




\pgfmathdeclarefunction{gauss}{2}{% gives gaussian
  \pgfmathparse{1/(#2*sqrt(2*pi))*exp(-((x-#1)^2)/(2*#2^2))}%
}


%%%%%%%%%%%%%
%% Vectors
%%%%%%%%%%%%%

%% Simple horiz vectors
\renewcommand{\vector}[1]{\left\langle #1\right\rangle}


%% %% Complex Horiz Vectors with angle brackets
%% \makeatletter
%% \renewcommand{\vector}[2][ , ]{\left\langle%
%%   \def\nextitem{\def\nextitem{#1}}%
%%   \@for \el:=#2\do{\nextitem\el}\right\rangle%
%% }
%% \makeatother

%% %% Vertical Vectors
%% \def\vector#1{\begin{bmatrix}\vecListA#1,,\end{bmatrix}}
%% \def\vecListA#1,{\if,#1,\else #1\cr \expandafter \vecListA \fi}

%%%%%%%%%%%%%
%% End of vectors
%%%%%%%%%%%%%

%\newcommand{\fullwidth}{}
%\newcommand{\normalwidth}{}



%% makes a snazzy t-chart for evaluating functions
%\newenvironment{tchart}{\rowcolors{2}{}{background!90!textColor}\array}{\endarray}

%%This is to help with formatting on future title pages.
\newenvironment{sectionOutcomes}{}{}



%% Flowchart stuff
%\tikzstyle{startstop} = [rectangle, rounded corners, minimum width=3cm, minimum height=1cm,text centered, draw=black]
%\tikzstyle{question} = [rectangle, minimum width=3cm, minimum height=1cm, text centered, draw=black]
%\tikzstyle{decision} = [trapezium, trapezium left angle=70, trapezium right angle=110, minimum width=3cm, minimum height=1cm, text centered, draw=black]
%\tikzstyle{question} = [rectangle, rounded corners, minimum width=3cm, minimum height=1cm,text centered, draw=black]
%\tikzstyle{process} = [rectangle, minimum width=3cm, minimum height=1cm, text centered, draw=black]
%\tikzstyle{decision} = [trapezium, trapezium left angle=70, trapezium right angle=110, minimum width=3cm, minimum height=1cm, text centered, draw=black]

\title[Dig-In:]{Limit Laws}
\begin{document}
\begin{abstract}
We give basic laws for working with limits. 
\end{abstract}
\maketitle

%This section comes second.

In this section, we present a handful of rules called the Limit Laws that allow us to find limits of various combinations of functions.  

\begin{theorem}[Limit Laws]\index{limit laws}\label{theorem:limit-laws}
Suppose that $\lim_{x\to a}f(x)=L$, $\lim_{x\to a}g(x)=M$.
\begin{itemize}
%\item[\textbf{Constant Multiple Law}] $\lim_{x\to a} kf(x) = k\lim_{x\to a}f(x)=kL$.
\item \textbf{Sum/Difference Law} $\lim_{x\to a} (f(x) \pm g(x)) =
  \lim_{x\to a}f(x) \pm \lim_{x\to a}g(x)=L \pm M$.
\item \textbf{Product Law}  $\lim_{x\to a} (f(x)g(x)) = \lim_{x\to
  a}f(x)\cdot\lim_{x\to a}g(x)=LM$.
\item \textbf{Quotient Law}  $\lim_{x\to a} \frac{f(x)}{g(x)} =
  \frac{\lim_{x\to a}f(x)}{\lim_{x\to a}g(x)}=\frac{L}{M}$, if
  $M\ne0$.
\end{itemize}
\label{thm:limit laws}
\end{theorem}

In particular, if $f$ and $g$ are continuous functions on an interval
$I$, then
\begin{itemize}
\item $f\pm g$ is continuous on $I$.
\item $f\cdot g$ is continuous on $I$.
  $\frac{f}{g}$ is continuous on all points in $I$ except those
  where $g(x)=0$.
\end{itemize}

\begin{example}
$\lim_{x\to 1}(5x^2+3x-2) = \answer[given]{6}$
\end{example}
  \begin{explanation}
    \begin{align*}
      \lim_{x\to 1} (5x^2+3x-2) &= \lim_{x\to 1} 5x^2 + \lim_{x\to 1} 3x - \lim_{x\to 1}2 && \text{Sum/Difference Law}\\
      &= 5\lim_{x\to 1} x^2 + 3\lim_{x\to 1} x - \lim_{x\to 1}2 && \text{Product Law}\\
      &= 5(1)^2 + 3(1) - 2 =6 && \text{Continuity of $x^k$ and $k$}
    \end{align*}
  \end{explanation}  

We can generalize the example above to get the following theorems.

\begin{theorem}
  Polynomials, expressions of the form
  \[
  a_0+a_1x^1+a_2x^2+a_3x^3+\dots+a_nx^n
  \]
  where $n$ is a whole number and each $a_i$ is a real number, are
  continuous for all real numbers.
  \begin{proof}
  \end{proof}
\end{theorem}

\begin{theorem} 
Let $f(x)$ and $g(x)$ be polynomials.  Then, $h(x)=\frac{f(x)}{g(x)}$ is continuous for all real numbers except where $g(x)=0$.  That is, rational functions are continuous wherever they are defined.

\begin{proof}
      Let $a$ be a real number such that $g(a)\neq 0$.   Then, since $g(x)$ is continuous at $a$,  $\lim_{x\to a} g(x) \neq 0$.  Therefore, 
\begin{align*}
            \lim_{x \to a} h(x) &= \lim_{x\to a} \frac{f(x)}{g(x)} &&\\
             &= \frac{\lim_{x\to a} f(x)}{ \lim_{x\to a} g(x)} && \text{Quotient Law}\\
      &= \frac{f(a)}{g(a)}=h(a) && \text{Continuity of polynomials}
    \end{align*}
\end{proof}
\end{theorem}


Now, we give basic rules for how limits interact with composition
of functions.

\begin{theorem}[Limit Laws: Composition]If $\lim_{x\to a}g(x)=M$ and
  $\lim_{x\to M}f(x) = f(M)$, then $\lim_{x\to a} f(g(x)) = f(M)$.
\end{theorem}

Because the limit of a continuous function is the same as the function value, we can now pass limits inside continuous functions.  

\begin{theorem}[Continuity of Composite Functions]\index{continuity of famous functions}\label{theorem:continuity}
Let $\lim_{x\to a} g(x) = L$ and let $f(x)$ be continuous at the point $g(a)$.  Then, $\lim_{x\to a} f(g(x)) = f(\lim_{x\to a} g(x))$.  In particular, if $g(x)$ is continuous at $a$, then $f(g(x))$ is continuous at $a$.
\end{theorem}

\begin{example}
$\lim_{x \to 0} \sqrt{\cos(x)} = \answer[given]{1}$
\begin{explanation}
\begin{align*}
  \lim_{x \to 0} \sqrt{\cos(x)} &= 
  \sqrt{\lim_{x \to 0} (\cos(x))} \text{\ \ (by continuity of $x^k$, assuming $\lim_{x \to 0} \cos(x) >0$})\\
  &= \sqrt{\cos(0)} = \sqrt{1} = 1 \text{\ \ (by continuity of $\cos(x)$)}\\
\end{align*}
\end{explanation}
\end{example}

Many of the Limit Laws and theorems about continuity in this section might seem like they should be obvious.  You may be wondering why we spent an entire section on these theorems.  The answer is that these theorems will tell you exactly when it is easy to find the value of a limit, and exactly what to do in those cases.  The most important thing to learn from this section is whether the limit laws can be applied for a certain problem, and when we need to do something more interesting.  We will begin discussing those more interesting cases in the next section.  For now, we end this section with a
question:
 
\begin{question}
Can the following limits can be directly computed by theorems given in this section? If so, find the value of the limit.
\begin{align*}
    &\text{Limit} && \text{Computed?}  &&& \text{Answer} \\
   a)  &\lim_{x\to 2}\frac{x^2+3x+2}{x+2} && \answer[given]{yes} &&& \answer[given]{3}\\
   b)  &\lim_{x\to 2}\frac{x^2-3x+2}{x-2} && \answer[given]{no} &&& \answer[given]{}\\
   c)  &\lim_{x\to 0} x\sin(1/x) && \answer[given]{no} &&& \answer[given]{}\\
   d) &\lim_{x\to 0} \cot(x^3) && \answer[given]{no} &&& \answer[given]{}\\
   e) &\lim_{x\to 0}\sec^2(e^x-1) && \answer[given]{yes} &&& \answer[given]{1}\\
   f) &\lim_{x\to 1}{(x-1)\cdot \csc(\ln(x))}  && \answer[given]{no} &&& \answer[given]{}\\
  g) &\lim_{x\to 0} x\ln x  && \answer[given]{no} &&& \answer[given]{}\\
  h) &\lim_{x\to 0}\frac{2^x-1}{3^{x-1}}  && \answer[given]{yes} &&& \answer[given]{0}\\
   i) &\lim_{x\to 0}(1+x)^{1/x}  && \answer[given]{no} &&& \answer[given]{}
\end{align*}

    \begin{feedback} See explanations below:\\

    \begin{enumerate}
    \item $f(x)=\frac{x^2+3x+2}{x+2}$ is a rational function, and the denominator does not equal 0, so $f(x)$ is continuous.  Thus, to find this limit, it suffices to plug 2 into $f(x)$.
    
    \item $f(x)=\frac{x^2-3x+2}{x-2}$ is a rational function, but the denominator does equal 0 when $x=2$.  None of our current theorems address the situation when the bottom of a fraction approaches 0 as x approaches the limiting value.
     
    \item If we are trying to use Limit Laws to compute this limit, we would first have to use the Product Law to say that  
    $\lim_{x\to 0} x\sin(1/x)= \lim_{x\to 0} x \cdot \lim_{x\to 0} \sin(1/x).$  We are only allowed to use this law if both limits exist, so we check this first.  We know from continuity that $\lim_{x\to 0}x=0$, but in the first section on limits, we learned that $\lim_{x\to 0} \sin(1/x)$  does not exist.  Therefore, we cannot use the Product Law.
     
    \item Notice that $\cot(x^3) = \frac{\cos(x^3)}{\sin(x^3)}$.  If we are trying to use Limit Laws to compute this limit, we would now have to use the Quotient Law to say that $\lim_{x\to 0} \frac{\cos(x^3)}{\sin(x^3)}
    = \frac{\lim_{x\to 0} \cos(x^3)}{\lim_{x\to 0} \sin(x^3)}$.  We are only allowed to use this law if both limits exist and the bottom is not 0. We would need to check both limits to use this law.  We suspect that the limit on on the denominator might equal 0, so we check this limit first.\\
    $\lim_{x\to 0} \sin(x^3)\\
    = \sin(\lim_{x\to 0}x^3)\\
    =\sin(0)=0$ \\
    Therefore, we cannot use the Quotient Law.
      
    \item $\lim_{x\to 0} \sec^2(e^x-1) \\
    = \lim_{x\to 0} \frac{1}{\cos^2(e^x-1)} \\
   = \frac{ \lim_{x\to 0}1}{ \lim_{x\to 0}\cos^2(e^x-1)} \text{\ \ (provided both limits exist and the bottom is not 0)}$\\
   Let's check each limit separately.\\
   Numerator:\\
   $\lim_{x\to 0}1=1$\\
   Denominator:\\
   $\lim_{x\to 0}\cos^2(e^x-1)\\
   = \cos^2(\lim_{x\to 0}(e^x-1))\\
   = \cos^2(\lim_{x\to 0}(e^x)-\lim_{x\to 0}(1)))\\
   = \cos^2(1-1))\\
   = \cos^2(0)=1$\\
   Therefore, the limit in the denominator exists and does not equal 0.  We can use the Quotient Law.\\
   $\frac{ \lim_{x\to 0}1}{ \lim_{x\to 0}\cos^2(e^x-1)} = \frac{1}{1}=1\\ $
   
   \item $\lim_{x\to 1}{(x-1)\cdot \csc(\ln(x))} \\
   = \lim_{x\to 1}{(x-1)\cdot \lim_{x\to 1}\csc(\ln(x))} \text{\ \ (provided both limits exist)}$\\
   Let's check each limit separately.  \\
   $\lim_{x\to 1} (x-1) = \lim_{x\to 1} (x)-\lim_{x\to 1}(1)=1-1=0$\\
   
   $\lim_{x\to 1}\csc(\ln(x)) \\
   = \frac{1}{\sin(\ln(x))}\\
   = \frac{\lim_{x\to 1}1}{\lim_{x\to 1}\sin(\ln(x))} \text{\ \ (provided both limits exist and the bottom does not equal 0)}$\\
   The limit in the numerator definitely exists, so lets check the limit in the denominator.\\
    $\lim_{x\to 1}\sin(\ln(x)) \\
    = \sin(\lim_{x\to 1}\ln(x))\\
    = \sin(\ln(1))\\
    = \sin(0) = 0\\$
   Since the denominator is 0, we cannot apply the Quotient Law.  
     
   \item $\lim_{x\to 0} x\ln x\\
   = \lim_{x\to 0} x \cdot \lim_{x\to 0}\ln x \text{\ \ (provided both limits exist)}$\\
   We know $\lim_{x\to 0} x = 0$, but what about $\lim_{x\to 0}\ln x$?
   We do not know how to find $\lim_{x\to 0}\ln x$ using limit laws because 0 is not in the domain of $\ln x$.
   
   \item $\lim_{x\to 0}\frac{2^x-1}{3^{x-1}} \\
   = \frac{\lim_{x\to 0}(2^x-1)}{\lim_{x\to 0}(3^{x-1})} \text{\ \ (provided both limits exist and the bottom does not equal 0)}$\\
   Let's check each limit separately.\\
   Numerator:\\
   $\lim_{x\to 0}(2^x-1)\\
   =\lim_{x\to 0}(2^x)-\lim_{x\to 0}(1)\\
   =1-1=0$\\
   Denominator:\\
   $\lim_{x\to 0}(3^{x-1})\\
   =3^{\lim_{x\to 0}(x-1)}\\
   =3^{-1}=\frac{1}{3}$\\
   The limits in both the numerator and denominator exist and the limit in the denominator does not equal 0, so we can use the Quotient Law.  We get:\\
   $\frac{\lim_{x\to 0}(2^x-1)}{\lim_{x\to 0}(3^{x-1})}
   =\frac{0}{\frac{1}{3}}=0$.
   


   
   \item We do not have any limit laws for functions of the form $f(x)^{g(x)}$, so we cannot compute this limit.
    \end{enumerate}
    \end{feedback} 
    
\end{question}

\end{document}
%% \begin{example}
%% Find $\lim_{x\to1}(x-2)$.
%% \begin{explanation}
%% We do this ``by the book'' and explicitly use our limit laws. First write
%% \[
%% \lim_{x\to1}(x-2) = \lim_{x\to1}(x)-\lim_{x\to1}(2)
%% \]
%% What limit law was used in this step? $\answer[given]{\text{Sum/Difference Law}}$\\
%% \[
%% =\lim_{x\to1}(x)-2
%% \]
%% What limit law was used in this step? $\answer[given]{\text{Limit of a Constant}}$\\
%% \[
%% =1-2
%% \]
%% What limit law was used in this step? $\answer[given]{\text{Limit of $x$}}$\\
%% \[
%% =-1.
%% \]
%% Hence we se that $\lim_{x\to1}(x-2) = 1$.
%% %%%BADBAD This should probably be drop down from above
%% %%How do I make text answers?
%% \end{explanation}
%% \end{example}

%% This first example might seem trivial, but remember how we tricked
%% ourselves in the previous section.  The limit laws allow us to know
%% that as long as we follow them, we are actually getting the right
%% answer for the limit.

%% \begin{example}
%% Compute $\lim_{x\to 1}\frac{x^2-3x+5}{x-2}$. 
%% \begin{explanation}
%% Using limit laws, 
%% \begin{align*}
%% \lim_{x\to 1}\frac{x^2-3x+5}{x-2}&=
%% \frac{\lim_{x\to 1}(x^2-3x+5)}{\lim_{x\to1}(x-2)}  && \text{(assuming $\lim_{x\to1}(x-2) \neq 0$)} \\
%% &=\frac{\lim_{x\to 1}(x^2)-\lim_{x\to1}(3x)+\lim_{x\to1}(5)}{\lim_{x\to1}(x)-\lim_{x\to1}(2)} \\
%% &=\frac{\left(\lim_{x\to 1}x\right)^2-3(\lim_{x\to1}x)+5}{\lim_{x\to1}(x)-2} \\
%% &=\frac{1^2-3\cdot1+5}{1-2} \\
%% &=\frac{1-3+5}{-1} = -3.
%% \end{align*}
%% \textit{On your own}: Go through each step in this example an determine which limit laws were used.  It is important to make sure you are only using real limit laws and not making up your own rules!

%% %Can we use color in the book.  For example,  \text{\ \ (assuming $\lim_{x\to1}(x-2) \neq 0$)}  should be a different color.
%% %% BADBAD For the print book, we *could* but probably don't want to, since printing would be hard.
%% %% however, it would be good to do this for the online version...
%% \end{explanation}
%% \end{example}

%%\begin{theorem}[Limit Laws: Basic Limits]
%%  Here are some basic limits:
 %% \begin{description}
 %% \item[\textbf{Limit of a Constant}] $\lim_{x\to a} k = k$ where $k$
%%    is a constant..
%%  \item[\textbf{Limit of $\boldsymbol{x}$}] $\lim_{x\to a}x =a$.
%%  \item[\textbf{Limit of $\boldsymbol{\sin(x)}$}] $\lim_{x\to a}
%%    \sin(x) = \sin(a)$.
%%  \item[\textbf{Limit of $\boldsymbol{b^x}$}] $\lim_{x\to a} b^x =
%%    b^a$ provided $b\ne 0$.
%%  \item[\textbf{Limit of $\boldsymbol{\log_b(x)}$}] $\lim_{x\to a}
%%    \log_b(x) = \log_b(a)$ provided $a$ and $b$ are both greater than zero.
%%  \end{description}
%%\end{theorem}

%%\begin{question}
 %% For the ``Limit of $b^x$'' why is it necessary that $b\ne 0$?
 % \begin{hint}
    %Suppose that $b=0$. Compute:
    %\[
    %0^1
    %\begin{prompt}
    %  = \answer[given]{0}
    %\end{prompt}
    %\qquad
    %0^{0.1}
    %\begin{prompt}
      %= \answer[given]{0}
    %\end{prompt}
    %\qquad
    %0^{0.01}
    %\begin{prompt}
      %= \answer[given]{0}
    %\end{prompt}
    %\]
    %However, we should compare this to
    %\[
    %0^0
    %\begin{prompt}
     % = \answer[given]{1}
    %\end{prompt}
    %\]
  %\end{hint}
  %\begin{freeResponse}[given]
   % If $b=0$, then $0^x = 0$ for all $x$ except $x=0$. On the other
    %hand, $0^0 = 1$.
 % \end{freeResponse}
%\end{question}

%For some reason it will not compile if I comment this out
\begin{question}
  Which limit laws are used in the computations below?
 \begin{explanation}%%BADBAD Need drop-down
   \begin{align*}
    \lim_{x\to 0} \cos x &= \sin(x+\pi/2) && \answer[given]{\text{Definition of Cosine}}\\
     &= \sin(\pi/2) && \answer[given]{\text{General Composition Law}}
  \end{align*} \end{explanation}  
\end{question}

