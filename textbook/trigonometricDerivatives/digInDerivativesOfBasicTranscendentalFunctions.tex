\documentclass{ximera}

%\usepackage{todonotes}
%\usepackage{mathtools} %% Required for wide table Curl and Greens
%\usepackage{cuted} %% Required for wide table Curl and Greens
\newcommand{\todo}{}

\usepackage{esint} % for \oiint
\ifxake%%https://math.meta.stackexchange.com/questions/9973/how-do-you-render-a-closed-surface-double-integral
\renewcommand{\oiint}{{\large\bigcirc}\kern-1.56em\iint}
\fi


\graphicspath{
  {./}
  {ximeraTutorial/}
  {basicPhilosophy/}
  {functionsOfSeveralVariables/}
  {normalVectors/}
  {lagrangeMultipliers/}
  {vectorFields/}
  {greensTheorem/}
  {shapeOfThingsToCome/}
  {dotProducts/}
  {partialDerivativesAndTheGradientVector/}
  {../productAndQuotientRules/exercises/}
  {../normalVectors/exercisesParametricPlots/}
  {../continuityOfFunctionsOfSeveralVariables/exercises/}
  {../partialDerivativesAndTheGradientVector/exercises/}
  {../directionalDerivativeAndChainRule/exercises/}
  {../commonCoordinates/exercisesCylindricalCoordinates/}
  {../commonCoordinates/exercisesSphericalCoordinates/}
  {../greensTheorem/exercisesCurlAndLineIntegrals/}
  {../greensTheorem/exercisesDivergenceAndLineIntegrals/}
  {../shapeOfThingsToCome/exercisesDivergenceTheorem/}
  {../greensTheorem/}
  {../shapeOfThingsToCome/}
  {../separableDifferentialEquations/exercises/}
  {vectorFields/}
}

\newcommand{\mooculus}{\textsf{\textbf{MOOC}\textnormal{\textsf{ULUS}}}}

\usepackage{tkz-euclide}\usepackage{tikz}
\usepackage{tikz-cd}
\usetikzlibrary{arrows}
\tikzset{>=stealth,commutative diagrams/.cd,
  arrow style=tikz,diagrams={>=stealth}} %% cool arrow head
\tikzset{shorten <>/.style={ shorten >=#1, shorten <=#1 } } %% allows shorter vectors

\usetikzlibrary{backgrounds} %% for boxes around graphs
\usetikzlibrary{shapes,positioning}  %% Clouds and stars
\usetikzlibrary{matrix} %% for matrix
\usepgfplotslibrary{polar} %% for polar plots
\usepgfplotslibrary{fillbetween} %% to shade area between curves in TikZ
\usetkzobj{all}
\usepackage[makeroom]{cancel} %% for strike outs
%\usepackage{mathtools} %% for pretty underbrace % Breaks Ximera
%\usepackage{multicol}
\usepackage{pgffor} %% required for integral for loops



%% http://tex.stackexchange.com/questions/66490/drawing-a-tikz-arc-specifying-the-center
%% Draws beach ball
\tikzset{pics/carc/.style args={#1:#2:#3}{code={\draw[pic actions] (#1:#3) arc(#1:#2:#3);}}}



\usepackage{array}
\setlength{\extrarowheight}{+.1cm}
\newdimen\digitwidth
\settowidth\digitwidth{9}
\def\divrule#1#2{
\noalign{\moveright#1\digitwidth
\vbox{\hrule width#2\digitwidth}}}





\newcommand{\RR}{\mathbb R}
\newcommand{\R}{\mathbb R}
\newcommand{\N}{\mathbb N}
\newcommand{\Z}{\mathbb Z}

\newcommand{\sagemath}{\textsf{SageMath}}


%\renewcommand{\d}{\,d\!}
\renewcommand{\d}{\mathop{}\!d}
\newcommand{\dd}[2][]{\frac{\d #1}{\d #2}}
\newcommand{\pp}[2][]{\frac{\partial #1}{\partial #2}}
\renewcommand{\l}{\ell}
\newcommand{\ddx}{\frac{d}{\d x}}

\newcommand{\zeroOverZero}{\ensuremath{\boldsymbol{\tfrac{0}{0}}}}
\newcommand{\inftyOverInfty}{\ensuremath{\boldsymbol{\tfrac{\infty}{\infty}}}}
\newcommand{\zeroOverInfty}{\ensuremath{\boldsymbol{\tfrac{0}{\infty}}}}
\newcommand{\zeroTimesInfty}{\ensuremath{\small\boldsymbol{0\cdot \infty}}}
\newcommand{\inftyMinusInfty}{\ensuremath{\small\boldsymbol{\infty - \infty}}}
\newcommand{\oneToInfty}{\ensuremath{\boldsymbol{1^\infty}}}
\newcommand{\zeroToZero}{\ensuremath{\boldsymbol{0^0}}}
\newcommand{\inftyToZero}{\ensuremath{\boldsymbol{\infty^0}}}



\newcommand{\numOverZero}{\ensuremath{\boldsymbol{\tfrac{\#}{0}}}}
\newcommand{\dfn}{\textbf}
%\newcommand{\unit}{\,\mathrm}
\newcommand{\unit}{\mathop{}\!\mathrm}
\newcommand{\eval}[1]{\bigg[ #1 \bigg]}
\newcommand{\seq}[1]{\left( #1 \right)}
\renewcommand{\epsilon}{\varepsilon}
\renewcommand{\phi}{\varphi}


\renewcommand{\iff}{\Leftrightarrow}

\DeclareMathOperator{\arccot}{arccot}
\DeclareMathOperator{\arcsec}{arcsec}
\DeclareMathOperator{\arccsc}{arccsc}
\DeclareMathOperator{\si}{Si}
\DeclareMathOperator{\scal}{scal}
\DeclareMathOperator{\sign}{sign}


%% \newcommand{\tightoverset}[2]{% for arrow vec
%%   \mathop{#2}\limits^{\vbox to -.5ex{\kern-0.75ex\hbox{$#1$}\vss}}}
\newcommand{\arrowvec}[1]{{\overset{\rightharpoonup}{#1}}}
%\renewcommand{\vec}[1]{\arrowvec{\mathbf{#1}}}
\renewcommand{\vec}[1]{{\overset{\boldsymbol{\rightharpoonup}}{\mathbf{#1}}}\hspace{0in}}

\newcommand{\point}[1]{\left(#1\right)} %this allows \vector{ to be changed to \vector{ with a quick find and replace
\newcommand{\pt}[1]{\mathbf{#1}} %this allows \vec{ to be changed to \vec{ with a quick find and replace
\newcommand{\Lim}[2]{\lim_{\point{#1} \to \point{#2}}} %Bart, I changed this to point since I want to use it.  It runs through both of the exercise and exerciseE files in limits section, which is why it was in each document to start with.

\DeclareMathOperator{\proj}{\mathbf{proj}}
\newcommand{\veci}{{\boldsymbol{\hat{\imath}}}}
\newcommand{\vecj}{{\boldsymbol{\hat{\jmath}}}}
\newcommand{\veck}{{\boldsymbol{\hat{k}}}}
\newcommand{\vecl}{\vec{\boldsymbol{\l}}}
\newcommand{\uvec}[1]{\mathbf{\hat{#1}}}
\newcommand{\utan}{\mathbf{\hat{t}}}
\newcommand{\unormal}{\mathbf{\hat{n}}}
\newcommand{\ubinormal}{\mathbf{\hat{b}}}

\newcommand{\dotp}{\bullet}
\newcommand{\cross}{\boldsymbol\times}
\newcommand{\grad}{\boldsymbol\nabla}
\newcommand{\divergence}{\grad\dotp}
\newcommand{\curl}{\grad\cross}
%\DeclareMathOperator{\divergence}{divergence}
%\DeclareMathOperator{\curl}[1]{\grad\cross #1}
\newcommand{\lto}{\mathop{\longrightarrow\,}\limits}

\renewcommand{\bar}{\overline}

\colorlet{textColor}{black}
\colorlet{background}{white}
\colorlet{penColor}{blue!50!black} % Color of a curve in a plot
\colorlet{penColor2}{red!50!black}% Color of a curve in a plot
\colorlet{penColor3}{red!50!blue} % Color of a curve in a plot
\colorlet{penColor4}{green!50!black} % Color of a curve in a plot
\colorlet{penColor5}{orange!80!black} % Color of a curve in a plot
\colorlet{penColor6}{yellow!70!black} % Color of a curve in a plot
\colorlet{fill1}{penColor!20} % Color of fill in a plot
\colorlet{fill2}{penColor2!20} % Color of fill in a plot
\colorlet{fillp}{fill1} % Color of positive area
\colorlet{filln}{penColor2!20} % Color of negative area
\colorlet{fill3}{penColor3!20} % Fill
\colorlet{fill4}{penColor4!20} % Fill
\colorlet{fill5}{penColor5!20} % Fill
\colorlet{gridColor}{gray!50} % Color of grid in a plot

\newcommand{\surfaceColor}{violet}
\newcommand{\surfaceColorTwo}{redyellow}
\newcommand{\sliceColor}{greenyellow}




\pgfmathdeclarefunction{gauss}{2}{% gives gaussian
  \pgfmathparse{1/(#2*sqrt(2*pi))*exp(-((x-#1)^2)/(2*#2^2))}%
}


%%%%%%%%%%%%%
%% Vectors
%%%%%%%%%%%%%

%% Simple horiz vectors
\renewcommand{\vector}[1]{\left\langle #1\right\rangle}


%% %% Complex Horiz Vectors with angle brackets
%% \makeatletter
%% \renewcommand{\vector}[2][ , ]{\left\langle%
%%   \def\nextitem{\def\nextitem{#1}}%
%%   \@for \el:=#2\do{\nextitem\el}\right\rangle%
%% }
%% \makeatother

%% %% Vertical Vectors
%% \def\vector#1{\begin{bmatrix}\vecListA#1,,\end{bmatrix}}
%% \def\vecListA#1,{\if,#1,\else #1\cr \expandafter \vecListA \fi}

%%%%%%%%%%%%%
%% End of vectors
%%%%%%%%%%%%%

%\newcommand{\fullwidth}{}
%\newcommand{\normalwidth}{}



%% makes a snazzy t-chart for evaluating functions
%\newenvironment{tchart}{\rowcolors{2}{}{background!90!textColor}\array}{\endarray}

%%This is to help with formatting on future title pages.
\newenvironment{sectionOutcomes}{}{}



%% Flowchart stuff
%\tikzstyle{startstop} = [rectangle, rounded corners, minimum width=3cm, minimum height=1cm,text centered, draw=black]
%\tikzstyle{question} = [rectangle, minimum width=3cm, minimum height=1cm, text centered, draw=black]
%\tikzstyle{decision} = [trapezium, trapezium left angle=70, trapezium right angle=110, minimum width=3cm, minimum height=1cm, text centered, draw=black]
%\tikzstyle{question} = [rectangle, rounded corners, minimum width=3cm, minimum height=1cm,text centered, draw=black]
%\tikzstyle{process} = [rectangle, minimum width=3cm, minimum height=1cm, text centered, draw=black]
%\tikzstyle{decision} = [trapezium, trapezium left angle=70, trapezium right angle=110, minimum width=3cm, minimum height=1cm, text centered, draw=black]


\title[Dig-In]{Derivatives of basic transcendental functions}

\begin{document}
\begin{abstract}
\end{abstract}
\maketitle

\section{The Derivative of $\textit{e}^\textit{x}$}


We don't know anything about derivatives that allows us to compute the
derivatives of exponential functions without getting our hands
dirty. Let's do a little work with the definition of the derivative:
\begin{align*}
\ddx a^x &=\lim_{h\to 0} \frac{a^{x+h}-a^x}{h} \\
&=\lim_{h\to 0} \frac{a^xa^{h}-a^x}{h} \\
&=\lim_{h\to 0} a^x\frac{a^{h}-1}{h} \\
&=a^x\lim_{h\to 0} \frac{a^{h}-1}{h} \\
&=a^x \cdot \underbrace{\text{(constant)}}_{\lim_{h\to 0} \frac{a^{h}-1}{h}}
\end{align*}
There are two interesting things to note here: We are left with a
limit that involves $h$ but not $x$, which means that whatever $
\lim_{h\to 0} (a^h-1)/h$ is, we know that it is a number, that is, a
constant. This means that $a^x$ has a remarkable property: Its
derivative is a constant times itself. Unfortunately it is beyond the
scope of this text to compute the limit
\[
\lim_{h\to 0} \frac{a^h-1}{h}.
\]
However, we can look at some examples. Consider $(2^h-1)/h$ and $(3^h-1)/h$:
\[
\begin{tchart}{ll}
 h &     (2^h-1)/h\\ \hline
 -1 & .5  \\
-0.1 &  \approx0.6700 \\
-0.01 & \approx0.6910 \\
-0.001 & \approx0.6929 \\
-0.0001 & \approx0.6931 \\
-0.00001 & \approx0.6932 \\
\end{tchart}
\qquad
\begin{tchart}{ll}
 h &     (2^h-1)/h\\ \hline
 1 & 1  \\
 0.1 &  \approx0.7177 \\
 0.01 & \approx0.6956 \\
 0.001 & \approx0.6934 \\
 0.0001 & \approx0.6932 \\
 0.00001 & \approx0.6932 \\
\end{tchart}
\qquad\qquad
\begin{tchart}{ll}
 h &     (3^h-1)/h\\ \hline
-1 & \approx 0.6667  \\
-0.1 &  \approx1.0404  \\
-0.01 & \approx1.0926 \\
-0.001 & \approx1.0980 \\
-0.0001 & \approx1.0986 \\
-0.00001 & \approx1.0986 \\
\end{tchart}
\qquad
\begin{tchart}{ll}
 h &     (3^h-1)/h\\ \hline
 1 & 2  \\
 0.1 &  \approx1.1612 \\
 0.01 & \approx1.1047 \\
 0.001 & \approx1.0992 \\
 0.0001 & \approx1.0987 \\
 0.00001 & \approx1.0986 \\
\end{tchart}
\]

While these tables don't prove a pattern, it turns out that
\[
\lim_{h\to 0}\frac{2^h-1}{h} \approx .7 \qquad\text{and}\qquad \lim_{h\to 0} \frac{3^h-1}{h} \approx 1.1.
\]
Moreover, if you do more examples you will find that the limit varies
directly with the value of $a$: bigger $a$, bigger limit; smaller $a$,
smaller limit. As we can already see, some of these limits will be
less than $1$ and some larger than $1$. Somewhere between $a=2$ and $a=3$
the limit will be exactly $1$. This happens when 
\[
a = e = 2.718281828459045\dots.
\]
This brings us to our next definition.

\begin{definition}\index{Euler's number}
Euler's number is defined to be the number $e$ such that
\[
\lim_{h\to 0} \frac{e^h-1}{h} = 1.
\]
\end{definition}

Now we see that the function $e^x$ has a truly remarkable property:

\begin{theorem}[The Derivative of $\textit{e}^\textit{x}$]\index{ex@$e^x$}\index{derivative!of ex@of $e^x$}
\[
\ddx e^x = e^x.
\]
\end{theorem}
\begin{proof}  
From the limit definition of the derivative, write
\begin{align*}
\ddx e^x&=\lim_{h\to 0} \frac{e^{x+h}-e^x}{h} \\
&=\lim_{h\to 0} \frac{e^xe^{h}-e^x}{h} \\
&=\lim_{h\to 0} e^x\frac{e^{h}-1}{h} \\
&=e^x\lim_{h\to 0} \frac{e^{h}-1}{h} \\
&=e^x.
\end{align*}
\end{proof}



Hence $e^x$ is its own derivative. In other words, the slope of the
plot of $e^x$ is the same as its height, or the same as its second
coordinate: The function $f(x)=e^x$ goes through the point $ (a,e^a)$
and has slope $e^a$ there, no matter what $a$ is. 



\begin{example}
Compute:
\[
\ddx\left(8\sqrt{x} + 7e^x \right)
\]

Write:
\begin{align*}
\ddx\left(8\sqrt{x} + 7e^x \right) &= 8\ddx x^{1/2} + 7\ddx e^x\\
&= 4x^{-1/2} + 7 e^x.
\end{align*}
\end{example}



\section{The Derivative of $\textrm{sin}(\textit{x})$}

It is now time to visit our friend who concerns itself
periodically with triangles and circles.

\begin{theorem}[The Derivative of sin(\textit{x})]\index{derivative!of sine}\label{theorem:deriv sin}
\[
\ddx \sin(x) = \cos(x).
\]
\end{theorem}

%% Should be (re)moved
\begin{align*}
\lim_{h\to 0}\frac{\cos(h)-1}{h} &= \lim_{h\to 0}\left(\frac{\cos(h)-1}{h}\cdot\frac{\cos(h)+1}{\cos(h)+1}\right)\\
&=\lim_{h\to 0}\frac{\cos^2(h)-1}{h(\cos(h)+1)}\\
&=\lim_{h\to 0}\frac{-\sin^2(h)}{h(\cos(h)+1)}\\
&=-\lim_{h\to 0}\left(\frac{\sin(h)}{h}\cdot\frac{\sin(h)}{(\cos(h)+1)}\right)\\
&= -1 \cdot \frac{0}{2} = 0.
\end{align*}
%% end

\begin{proof}
Using the definition of the derivative, write
\begin{align*}
\ddx \sin(x) &= \lim_{h\to0} \frac{\sin(x+h)-\sin(x)}{h} \\
&= \lim_{h\to0} \frac{\sin(x)\cos(h)+\sin(h)\cos(x)-\sin(x)}{h}  & \text{Trig Identity.}\\
&= \lim_{h\to0} \left(\frac{\sin(x)\cos(h)-\sin(x)}{h} + \frac{\sin(h)\cos{x}}{h} \right)\\
&=\lim_{h\to0} \left(\sin (x)\frac{\cos(h) - 1}{h}+\cos(x)\frac{\sin(h)}{h}\right) \\
&=\sin(x) \cdot 0 + \cos(x) \cdot 1 = \cos x. & \text{See Example~\ref{example:sinx/x}.}
\end{align*}
\end{proof}

Consider the following geometric interpretation of the derivative of
$\sin(\theta)$.  

\begin{image}
\begin{tikzpicture}
	\begin{axis}[
            xmin=-.1,xmax=1.1,ymin=-.1,ymax=1.1,
            axis lines=center,
            ticks=none,
            width=5in,
            unit vector ratio*=1 1 1,
            xlabel=$x$, ylabel=$y$,
            every axis y label/.style={at=(current axis.above origin),anchor=south},
            every axis x label/.style={at=(current axis.right of origin),anchor=west},
          ]        
          \addplot [very thick, textColor!30!background, smooth, domain=(-.2:.2+pi/2)] ({cos(deg(x))},{sin(deg(x))});
          \addplot [textColor,very thick] plot coordinates {(0,0) (.766,.643)}; %% 40 degrees
          \addplot [textColor,very thick] plot coordinates {(0,0) (.766,0)}; %% bottom
          \addplot [very thick, penColor2!30!background] {(x-.766)*(-.766/.643)+.643};
          \addplot [textColor,dashed] plot coordinates {(0,0) (.766-.196,.643+1-.766)}; %% 40+16.98 degrees          

          %% \addplot [textColor!20!background] plot coordinates {(.766,.643) (1,.839)}; %% hyp
          %% \addplot [textColor!20!background] plot coordinates {(1,.643) (1,.839)}; %% side
          %% \addplot [textColor!20!background] plot coordinates {(.766,.643) (1,.643)}; %% bottom
          %% \addplot [textColor!20!background,smooth, domain=(0:40)] ({.05*cos(x)+.766},{.05*sin(x)+.643}); %% angle
          %% \node at (axis cs:.84,.670) [textColor!20!background] {\footnotesize$\theta$};
          
          %% \addplot [textColor!20!background] plot coordinates {(.766,.643) (.766,.839)}; %% side
          %% \addplot [textColor!20!background] plot coordinates {(.766,.839) (1,.839)}; %% bottom
          %% \addplot [textColor!20!background,smooth, domain=(180:220)] ({.05*cos(x)+1},{.05*sin(x)+.839}); %% angle
          %% \node at (axis cs:.926,.812) [textColor!20!background] {\footnotesize$\theta$};
          
          \draw[rotate around={30:(.5,.5)}] (.7,.7) rectangle (.25,.25);

          %\draw[textColor, rotate around={45:(.5,.5)}] (.5,.5) rectangle (.2,.2);

          \addplot [penColor4,very thick] plot coordinates {(.766,.643) (.766,.643+1-.766)}; %% side
          \addplot [textColor,very thick] plot coordinates {(.766,.643+1-.766) (.766-.196,.643+1-.766)}; %% top
          \addplot [textColor,smooth, domain=(90:130)] ({.05*cos(x)+.766},{.05*sin(x)+.643}); %% angle
          \addplot [very thick, textColor] plot coordinates {(.766-.196,.643+1-.766) (.766,.643)}; %% hyp
          \node at (axis cs:.739,.717) [textColor] {\footnotesize$\theta$};
          
          \node at (axis cs:.668,.877) [anchor=south] {\footnotesize$h\sin(\theta)$};
          \node at (axis cs:.766,.76) [anchor=west] {\footnotesize$h\cos(\theta)$};
          \node at (axis cs:.65,.78) [anchor=west] {\footnotesize$\approx h$};

          \addplot [very thick, penColor] plot coordinates {(.766,0) (.766,.643)}; %% sin theta          
          
          \addplot [textColor, smooth, domain=(0:40)] ({.15*cos(x)},{.15*sin(x)});
          \addplot [textColor, smooth, domain=(40:56.90)] ({.17*cos(x)},{.17*sin(x)});
          \addplot [textColor, smooth, domain=(40:56.90)] ({.185*cos(x)},{.185*sin(x)});
          \node at (axis cs:.15,.07) [anchor=west] {$\theta$};
          \node at (axis cs:.15,.17) {$h$};
          \node at (axis cs:.766,.322) [anchor=east] {$\sin(\theta)$};
          \node at (axis cs:.383,0) [anchor=north] {$\cos(\theta)$};
        \end{axis}
\end{tikzpicture}
\end{image}

Here we see that increasing $\theta$ by a ``small amount'' $h$,
increases $\sin(\theta)$ by approximately $h\cos(\theta)$. Hence,
\[
\frac{\Delta y}{\Delta \theta}\approx \frac{h\cos(\theta)}{h} =
\cos(\theta).
\]

With this said, the derivative of a function measures the slope of the
plot of a function.  If we examine the graphs of the sine and cosine
side by side, it should be that the latter appears to accurately
describe the slope of the former, and indeed this is true, see
Figure~\ref{figure:sin/cos}.
\begin{image}
\begin{tikzpicture}
	\begin{axis}[
            xmin=-6.75,xmax=6.75,ymin=-1.5,ymax=1.5,
            axis lines=center,
            xtick={-6.28, -4.71, -3.14, -1.57, 0, 1.57, 3.142, 4.71, 6.28},
            xticklabels={$-2\pi$,$-3\pi/2$,$-\pi$, $-\pi/2$, $0$, $\pi/2$, $\pi$, $3\pi/2$, $2\pi$},
            ytick={-1,1},
            %ticks=none,
            width=9in,
            height=2in,
            unit vector ratio*=1 1 1,
            xlabel=$x$, ylabel=$y$,
            every axis y label/.style={at=(current axis.above origin),anchor=south},
            every axis x label/.style={at=(current axis.right of origin),anchor=west},
          ]        
          \addplot [very thick, penColor, samples=100,smooth, domain=(-6.75:6.75)] {sin(deg(x))};
          \addplot [very thick, penColor2, samples=100,smooth, domain=(-6.75:6.75)] {cos(deg(x))};
          \node at (axis cs:3.14,.75) [penColor] {$f(x)$};
          \node at (axis cs:-1.57,.75) [penColor2] {$f'(x)$};
        \end{axis}
\end{tikzpicture}
%% \caption{Here we see a plot of $f(x)=\sin(x)$ and its derivative
%%   $f'(x)=\cos(x)$. One can readily see that $\cos(x)$ is positive when
%%   $\sin(x)$ is increasing, and that $\cos(x)$ is negative when
%%   $\sin(x)$ is decreasing.}
%% \label{figure:sin/cos}
\end{image}

%% Of course, now that we know the derivative of the sine, we can compute
%% derivatives of more complicated functions involving the sine.

\end{document}
