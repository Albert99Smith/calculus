\documentclass{ximera}

%\usepackage{todonotes}
%\usepackage{mathtools} %% Required for wide table Curl and Greens
%\usepackage{cuted} %% Required for wide table Curl and Greens
\newcommand{\todo}{}

\usepackage{esint} % for \oiint
\ifxake%%https://math.meta.stackexchange.com/questions/9973/how-do-you-render-a-closed-surface-double-integral
\renewcommand{\oiint}{{\large\bigcirc}\kern-1.56em\iint}
\fi


\graphicspath{
  {./}
  {ximeraTutorial/}
  {basicPhilosophy/}
  {functionsOfSeveralVariables/}
  {normalVectors/}
  {lagrangeMultipliers/}
  {vectorFields/}
  {greensTheorem/}
  {shapeOfThingsToCome/}
  {dotProducts/}
  {partialDerivativesAndTheGradientVector/}
  {../productAndQuotientRules/exercises/}
  {../normalVectors/exercisesParametricPlots/}
  {../continuityOfFunctionsOfSeveralVariables/exercises/}
  {../partialDerivativesAndTheGradientVector/exercises/}
  {../directionalDerivativeAndChainRule/exercises/}
  {../commonCoordinates/exercisesCylindricalCoordinates/}
  {../commonCoordinates/exercisesSphericalCoordinates/}
  {../greensTheorem/exercisesCurlAndLineIntegrals/}
  {../greensTheorem/exercisesDivergenceAndLineIntegrals/}
  {../shapeOfThingsToCome/exercisesDivergenceTheorem/}
  {../greensTheorem/}
  {../shapeOfThingsToCome/}
  {../separableDifferentialEquations/exercises/}
  {vectorFields/}
}

\newcommand{\mooculus}{\textsf{\textbf{MOOC}\textnormal{\textsf{ULUS}}}}

\usepackage{tkz-euclide}\usepackage{tikz}
\usepackage{tikz-cd}
\usetikzlibrary{arrows}
\tikzset{>=stealth,commutative diagrams/.cd,
  arrow style=tikz,diagrams={>=stealth}} %% cool arrow head
\tikzset{shorten <>/.style={ shorten >=#1, shorten <=#1 } } %% allows shorter vectors

\usetikzlibrary{backgrounds} %% for boxes around graphs
\usetikzlibrary{shapes,positioning}  %% Clouds and stars
\usetikzlibrary{matrix} %% for matrix
\usepgfplotslibrary{polar} %% for polar plots
\usepgfplotslibrary{fillbetween} %% to shade area between curves in TikZ
\usetkzobj{all}
\usepackage[makeroom]{cancel} %% for strike outs
%\usepackage{mathtools} %% for pretty underbrace % Breaks Ximera
%\usepackage{multicol}
\usepackage{pgffor} %% required for integral for loops



%% http://tex.stackexchange.com/questions/66490/drawing-a-tikz-arc-specifying-the-center
%% Draws beach ball
\tikzset{pics/carc/.style args={#1:#2:#3}{code={\draw[pic actions] (#1:#3) arc(#1:#2:#3);}}}



\usepackage{array}
\setlength{\extrarowheight}{+.1cm}
\newdimen\digitwidth
\settowidth\digitwidth{9}
\def\divrule#1#2{
\noalign{\moveright#1\digitwidth
\vbox{\hrule width#2\digitwidth}}}





\newcommand{\RR}{\mathbb R}
\newcommand{\R}{\mathbb R}
\newcommand{\N}{\mathbb N}
\newcommand{\Z}{\mathbb Z}

\newcommand{\sagemath}{\textsf{SageMath}}


%\renewcommand{\d}{\,d\!}
\renewcommand{\d}{\mathop{}\!d}
\newcommand{\dd}[2][]{\frac{\d #1}{\d #2}}
\newcommand{\pp}[2][]{\frac{\partial #1}{\partial #2}}
\renewcommand{\l}{\ell}
\newcommand{\ddx}{\frac{d}{\d x}}

\newcommand{\zeroOverZero}{\ensuremath{\boldsymbol{\tfrac{0}{0}}}}
\newcommand{\inftyOverInfty}{\ensuremath{\boldsymbol{\tfrac{\infty}{\infty}}}}
\newcommand{\zeroOverInfty}{\ensuremath{\boldsymbol{\tfrac{0}{\infty}}}}
\newcommand{\zeroTimesInfty}{\ensuremath{\small\boldsymbol{0\cdot \infty}}}
\newcommand{\inftyMinusInfty}{\ensuremath{\small\boldsymbol{\infty - \infty}}}
\newcommand{\oneToInfty}{\ensuremath{\boldsymbol{1^\infty}}}
\newcommand{\zeroToZero}{\ensuremath{\boldsymbol{0^0}}}
\newcommand{\inftyToZero}{\ensuremath{\boldsymbol{\infty^0}}}



\newcommand{\numOverZero}{\ensuremath{\boldsymbol{\tfrac{\#}{0}}}}
\newcommand{\dfn}{\textbf}
%\newcommand{\unit}{\,\mathrm}
\newcommand{\unit}{\mathop{}\!\mathrm}
\newcommand{\eval}[1]{\bigg[ #1 \bigg]}
\newcommand{\seq}[1]{\left( #1 \right)}
\renewcommand{\epsilon}{\varepsilon}
\renewcommand{\phi}{\varphi}


\renewcommand{\iff}{\Leftrightarrow}

\DeclareMathOperator{\arccot}{arccot}
\DeclareMathOperator{\arcsec}{arcsec}
\DeclareMathOperator{\arccsc}{arccsc}
\DeclareMathOperator{\si}{Si}
\DeclareMathOperator{\scal}{scal}
\DeclareMathOperator{\sign}{sign}


%% \newcommand{\tightoverset}[2]{% for arrow vec
%%   \mathop{#2}\limits^{\vbox to -.5ex{\kern-0.75ex\hbox{$#1$}\vss}}}
\newcommand{\arrowvec}[1]{{\overset{\rightharpoonup}{#1}}}
%\renewcommand{\vec}[1]{\arrowvec{\mathbf{#1}}}
\renewcommand{\vec}[1]{{\overset{\boldsymbol{\rightharpoonup}}{\mathbf{#1}}}\hspace{0in}}

\newcommand{\point}[1]{\left(#1\right)} %this allows \vector{ to be changed to \vector{ with a quick find and replace
\newcommand{\pt}[1]{\mathbf{#1}} %this allows \vec{ to be changed to \vec{ with a quick find and replace
\newcommand{\Lim}[2]{\lim_{\point{#1} \to \point{#2}}} %Bart, I changed this to point since I want to use it.  It runs through both of the exercise and exerciseE files in limits section, which is why it was in each document to start with.

\DeclareMathOperator{\proj}{\mathbf{proj}}
\newcommand{\veci}{{\boldsymbol{\hat{\imath}}}}
\newcommand{\vecj}{{\boldsymbol{\hat{\jmath}}}}
\newcommand{\veck}{{\boldsymbol{\hat{k}}}}
\newcommand{\vecl}{\vec{\boldsymbol{\l}}}
\newcommand{\uvec}[1]{\mathbf{\hat{#1}}}
\newcommand{\utan}{\mathbf{\hat{t}}}
\newcommand{\unormal}{\mathbf{\hat{n}}}
\newcommand{\ubinormal}{\mathbf{\hat{b}}}

\newcommand{\dotp}{\bullet}
\newcommand{\cross}{\boldsymbol\times}
\newcommand{\grad}{\boldsymbol\nabla}
\newcommand{\divergence}{\grad\dotp}
\newcommand{\curl}{\grad\cross}
%\DeclareMathOperator{\divergence}{divergence}
%\DeclareMathOperator{\curl}[1]{\grad\cross #1}
\newcommand{\lto}{\mathop{\longrightarrow\,}\limits}

\renewcommand{\bar}{\overline}

\colorlet{textColor}{black}
\colorlet{background}{white}
\colorlet{penColor}{blue!50!black} % Color of a curve in a plot
\colorlet{penColor2}{red!50!black}% Color of a curve in a plot
\colorlet{penColor3}{red!50!blue} % Color of a curve in a plot
\colorlet{penColor4}{green!50!black} % Color of a curve in a plot
\colorlet{penColor5}{orange!80!black} % Color of a curve in a plot
\colorlet{penColor6}{yellow!70!black} % Color of a curve in a plot
\colorlet{fill1}{penColor!20} % Color of fill in a plot
\colorlet{fill2}{penColor2!20} % Color of fill in a plot
\colorlet{fillp}{fill1} % Color of positive area
\colorlet{filln}{penColor2!20} % Color of negative area
\colorlet{fill3}{penColor3!20} % Fill
\colorlet{fill4}{penColor4!20} % Fill
\colorlet{fill5}{penColor5!20} % Fill
\colorlet{gridColor}{gray!50} % Color of grid in a plot

\newcommand{\surfaceColor}{violet}
\newcommand{\surfaceColorTwo}{redyellow}
\newcommand{\sliceColor}{greenyellow}




\pgfmathdeclarefunction{gauss}{2}{% gives gaussian
  \pgfmathparse{1/(#2*sqrt(2*pi))*exp(-((x-#1)^2)/(2*#2^2))}%
}


%%%%%%%%%%%%%
%% Vectors
%%%%%%%%%%%%%

%% Simple horiz vectors
\renewcommand{\vector}[1]{\left\langle #1\right\rangle}


%% %% Complex Horiz Vectors with angle brackets
%% \makeatletter
%% \renewcommand{\vector}[2][ , ]{\left\langle%
%%   \def\nextitem{\def\nextitem{#1}}%
%%   \@for \el:=#2\do{\nextitem\el}\right\rangle%
%% }
%% \makeatother

%% %% Vertical Vectors
%% \def\vector#1{\begin{bmatrix}\vecListA#1,,\end{bmatrix}}
%% \def\vecListA#1,{\if,#1,\else #1\cr \expandafter \vecListA \fi}

%%%%%%%%%%%%%
%% End of vectors
%%%%%%%%%%%%%

%\newcommand{\fullwidth}{}
%\newcommand{\normalwidth}{}



%% makes a snazzy t-chart for evaluating functions
%\newenvironment{tchart}{\rowcolors{2}{}{background!90!textColor}\array}{\endarray}

%%This is to help with formatting on future title pages.
\newenvironment{sectionOutcomes}{}{}



%% Flowchart stuff
%\tikzstyle{startstop} = [rectangle, rounded corners, minimum width=3cm, minimum height=1cm,text centered, draw=black]
%\tikzstyle{question} = [rectangle, minimum width=3cm, minimum height=1cm, text centered, draw=black]
%\tikzstyle{decision} = [trapezium, trapezium left angle=70, trapezium right angle=110, minimum width=3cm, minimum height=1cm, text centered, draw=black]
%\tikzstyle{question} = [rectangle, rounded corners, minimum width=3cm, minimum height=1cm,text centered, draw=black]
%\tikzstyle{process} = [rectangle, minimum width=3cm, minimum height=1cm, text centered, draw=black]
%\tikzstyle{decision} = [trapezium, trapezium left angle=70, trapezium right angle=110, minimum width=3cm, minimum height=1cm, text centered, draw=black]


\title{The Theorem}

\begin{document}
\begin{abstract}
\end{abstract}
\maketitle

In the previous activity, we gathered some strong intuitive evidence for a theorem which would let us compute limits
by using derivatives.

\todo{L'H\^opital's rule applies even when $\lim_{x\to a}f(x) =
  \pm \infty$ and $\lim_{x\to a}g(x) = \mp \infty$. See Example~\ref{example:xlnx infty}.}
  
\begin{theorem}[L'H\^opital's Rule]\index{l'H\^opital's Rule} 
Let $f(x)$ and $g(x)$ be functions that are differentiable near (but not necessarily at) $a$.  
Here $a$ is also allowed to be $\infty$ or $-\infty\$, and in this case we require that $f$ and $g$ 
be differentiable on some interval $(N,\infty)$ or $(-\infty,N)$.

If

\[
\lim_{x \to a} f(x) = \lim_{x \to a}g(x) = 0 \qquad \text{or} \pm \infty,
\]
and $\lim_{x \to a} \frac{f'(x)}{g'(x)}$ exists, and $g'(x) \neq 0$
for all $x$ on an interval containing $a$, then 
\[
\lim_{x \to a} \frac{f(x)}{g(x)} = \lim_{x \to a} \frac{f'(x)}{g'(x)}.
\]
\end{theorem}

Intuitively the theorem says "To compare the growth of two quantites which are diminishing in size, 
we can instead compare the rates at which they are diminishing".  The proof of this theorem is outside 
the scope of this course, but you may enjoying reading a sketch of the proof 
\href{here}{https://www.math.hmc.edu/calculus/tutorials/lhopital/sketch_proof.html}.

\begin{question}
	Select all of the limits to which l'H\^{o}pital's rule immediately applies.
	\begin{hint}
		It applies to $\displaystyle\lim_{x \to 0} \frac{x^{-1}}{x^{-1}}$ 
		since the numerator and denominator both go to $\pm \infty$, and the derivative of the denominator is $\frac{-1}{x^2}$,
		which has no roots.  Note that while $l'H\^opitals$ rule applies to this example, it would be \textbf{very silly} to use it.
	\end{hint}
	\begin{hint}
		It does not apply to $\displaystyle \lim_{x \to 3} \frac{x+1}{x+4}$ 
		since $\displaystyle\lim_{x \to 3} (x+1) = 4$, which is not $0$, $\infty$ or $-\infty$
	\end{hint}
	\begin{hint}
		It does not apply to $\displaystyle \lim_{x \to 3} \displaystyle} \lim_{x \to 0} \frac{x}{x\sin(x)}$ 
		since even though the numerator and denominator both go to $0$ as $x \to 0$, 
		the derivative of the denominator has zeros arbitrarily close to $x=0$ (try graphing $y=x\sin(x)$).
	\end{hint}
	\begin{hint}
		It applies to $\displaystyle\lim_{x \to 2} \frac{e^x-e^2}{x-2}$ 
		since the numerator and denominator both go to $0$, and the derivative of the denominator is the constant function $1$,
		which has no roots at all.  Note that, while it is fine to use l'H\^{0}pital's rule to solve this limit,
		it is better to recognise that this is the definition of the derivative of $e^x$ at $x=2$.
	\end{hint}
	
	\todo{Implement multiple select choice}
	\begin{multiple-choice}
		\choice[correct]{$\displaystyle\lim_{x \to 2} \frac{e^x-e^2}{x-2}$}
		\choice[correct]{$\displaystyle\lim_{x \to 0}\frac{x^{-1}}{x^{-1}}$}
		\choice[correct]{$\displaystyle \lim_{x \to \infty}\frac{\log(x)}{x}$}
		\choice{$\displaystyle \lim_{x \to 3} \frac{x+1}{x+4}$}
		\choice{$\displaystyle} \lim_{x \to 0} \frac{x}{x\sin(x)$}
	\end{multiple-choice}
\end{question}

\todo{Implement \infty answer type}

\begin{question}
	$\displaystyle \lim_{x \to \infty} \frac{e^x}{x}=\answer{\infty}$.
	
\end{question}

L'H\^{o}pital's rule allows us to investigate limits of
\textit{indeterminate form}.

\begin{definition}[List of Indeterminate Forms]\hfil
\begin{itemize}
\item[\textbf{0/0}] This refers to a limit of the form $\lim_{x\to a}
  \frac{f(x)}{g(x)}$ where $f(x)\to 0$ and $g(x)\to 0$ as $x\to a$.
\item[\textbf{$\pmb\infty$/$\pmb\infty$}] This refers to a limit of the form $\lim_{x\to a}
  \frac{f(x)}{g(x)}$ where $f(x)\to \infty$ and $g(x)\to \infty$ as $x\to a$.
\item[\textbf{0\,$\pmb{\cdot\infty}$}] This refers to a limit of the form $\lim_{x\to a}
  \left(f(x)\cdot g(x)\right)$ where $f(x)\to 0$ and $g(x)\to \infty$ as $x\to a$.
\item[\textbf{$\pmb\infty$--$\pmb\infty$}] This refers to a limit of the form $\lim_{x\to a}\left(
  f(x)-g(x)\right)$ where $f(x)\to \infty$ and $g(x)\to \infty$ as $x\to a$.

\item[\textbf{1$^{\pmb\infty}$}] This refers to a limit of the form $\lim_{x\to a}
  f(x)^{g(x)}$ where $f(x)\to 1$ and $g(x)\to \infty$ as $x\to a$.
\item[\textbf{0$^\text{0}$}] This refers to a limit of the form $\lim_{x\to a}
  f(x)^{g(x)}$ where $f(x)\to 0$ and $g(x)\to 0$ as $x\to a$.
\item[\textbf{$\pmb\infty^\text{0}$}] This refers to a limit of the form $\lim_{x\to a}
  f(x)^{g(x)}$ where $f(x)\to \infty$ and $g(x)\to 0$ as $x\to a$.
\end{itemize}
 
\end{definition}

In each of these cases, the value of the limit is \textbf{not} immediately
obvious.  For example, if we have a limit of the form $[0^0]$, then the base of the exponent is 
"driving the limit to $0$", while the exponent is "driving the limit to $1$".  
In this tug of war, it is not clear which of the two combatants will triumpf, 
or (if neither "wins") to what value they will reach a compromise. Hence, a careful analysis is required.

\todo{dropdown}

\begin{question}
    $\lim_{x \to 0} \left(\frac{1}{x}\right)^x$ is
    \begin{multiple-choice}[dropdown]
    	\choice[correct]{a $[\infty^0}]$ indeterminate form}
    	\choice{a $[0^0]$ indeterminate form}
    	\choice{a $\frac{0}{0}$ indeterminate form}
    	\choice{not an indeterminate form}
    \end{multiple-choice}
\end{question}

\begin{question}
    $\lim_{x \to 0} x^{left(\frac{1}{x}\right}$ is
    \begin{multiple-choice}[dropdown]
    	\choice{a $[\infty^0}]$ indeterminate form}
    	\choice{a $[0^0]$ indeterminate form}
    	\choice{a $\frac{0}{0}$ indeterminate form}
    	\choice[correct]{not an indeterminate form}
    \end{multiple-choice}
\end{question}

Our first example is the computation of a limit that was somewhat
difficult before, see Example~\ref{example:sinx/x}. Note, this is an
example of the indeterminate form $0/0$.

\begin{example}[0/0]\label{example:sinx/x-lhopital}
\todo{Silly example}
Compute
\[
\lim_{x\to 0} \frac{\sin(x)}{x}.
\]

Set $f(x) = \sin(x)$ and $g(x) = x$.  Since both $f(x)$ and $g(x)$ are
differentiable functions at $0$, and 
\[
\lim_{x \to 0} f(x) = \lim_{x \to 0}g(x) = 0,
\]
this situation is ripe for l'H\^opital's Rule. Now
\[
f'(x) = \cos(x) \qquad\text{and}\qquad g'(x) = 1.
\] 
L'H\^opital's rule tells us that 
\[
\lim_{x \to 0} \frac{\sin(x)}{x} = \lim_{x \to 0} \frac{\cos(x)}{1} = 1.
\]
\end{example}


\begin{image}
\begin{tikzpicture}
	\begin{axis}[
            xmin=-1.6,xmax=1.6,ymin=-1.5,ymax=1.5, axis lines=center,
            xtick={-1.57, 0, 1.57}, xticklabels={$-\pi/2$, $0$,
              $\pi/2$}, ytick={-1,1},
            %ticks=none,
            %width=3in,
            %height=2in,
            unit vector ratio*=1 1 1,
            xlabel=$x$, ylabel=$y$,
            every axis y label/.style={at=(current axis.above origin),anchor=south},
            every axis x label/.style={at=(current axis.right of origin),anchor=west},
          ]        
          \addplot [very thick, penColor, samples=100,smooth, domain=(-1.6:1.6)] {sin(deg(x))};
          \addplot [very thick, penColor2] {x};
          \node at (axis cs:1,.6) [penColor] {$f(x)$};
          \node at (axis cs:-1,-1.2) [penColor2] {$g(x)$};
        \end{axis}
\end{tikzpicture}
%\caption{A plot of $f(x)=\sin(x)$ and $g(x) = x$. Note how the tangent
%  lines for each curve are coincident at $x=0$.}
%\label{example:sinx and x}
\end{image}


From this example, we gain an intuitive feeling for why l'H\^opital's
rule is true: If two functions are both $0$ when $x=a$, and if their
tangent lines have the same slope, then the functions coincide as $x$
approaches $a$. See Figure~\ref{example:sinx and x}. 




Our next set of examples will run through the remaining indeterminate
forms one is likely to encounter.

\begin{example}[$\pmb\infty$/$\pmb\infty$] Compute 
\[
\lim_{x\to \pi/2+} \frac{\sec(x)}{\tan(x)}.
\]

Set $f(x) = \sec(x)$ and $g(x) = \tan(x)$. Both $f(x)$ and $g(x)$
are differentiable near $\pi/2$. Additionally,
\[
\lim_{x \to \pi/2+} f(x) = \lim_{x \to \pi/2+}g(x) = -\infty.
\]
This situation is ripe for l'H\^opital's Rule. Now 
\[
f'(x) = \sec(x)\tan(x) \qquad\text{and}\qquad g'(x) = \sec^2(x).
\]
L'H\^opital's rule tells us that 
\[
\lim_{x\to \pi/2+} \frac{\sec(x)}{\tan(x)} = \lim_{x\to \pi/2+}
\frac{\sec(x)\tan(x)}{\sec^2(x)} = \lim_{x\to \pi/2+} \sin(x) =
1.
\]
\end{example}



\begin{example}[0\,$\pmb{\cdot\infty}$]\label{example:xlnx infty} 
Compute 
\[
\lim_{x\to 0+} x\ln x.
\]

This doesn't appear to be suitable for l'H\^opital's Rule. As $x$
approaches zero, $\ln x$ goes to $-\infty$, so the product looks like
\[
(\text{something very small})\cdot (\text{something very large and
  negative}).
\] 
This product could be anything---a careful analysis is required.
Write
\[
x\ln x = \frac{\ln x}{x^{-1}}.
\]
Set $f(x) = \ln(x)$ and $g(x) = x^{-1}$.  Since both functions are differentiable near zero and 
\[
\lim_{x\to 0+} \ln(x) = -\infty\qquad\text{and}\qquad \lim_{x\to 0+} x^{-1} = \infty,
\]
we may apply l'H\^opital's rule. Write
\[
f'(x) = x^{-1}\qquad \text{and}\qquad g'(x) = -x^{-2},
\]
so
\[
\lim_{x\to 0+} x\ln x = \lim_{x\to 0+} \frac{\ln x}{x^{-1}} = \lim_{x\to 0+} \frac{x^{-1}}{-x^{-2}} =\lim_{x\to 0+} -x = 0.
\]
One way to interpret this is that since $\lim_{x\to 0^+}x\ln x = 0$,
the function $x$ approaches zero much faster than $\ln x$ approaches
$-\infty$.
\end{example}

\subsection*{Indeterminate Forms Involving Subtraction}

There are two basic cases here, we'll do an example of each.

\begin{example}[$\pmb\infty$--$\pmb\infty$]
Compute
\[
\lim_{x\to 0} \left(\cot(x) - \csc(x)\right).
\]

Here we simply need to write each term as a fraction,
\begin{align*}
\lim_{x\to 0} \left(\cot(x) - \csc(x)\right) &= \lim_{x\to 0} \left(\frac{\cos(x)}{\sin(x)} - \frac{1}{\sin(x)}\right)\\
&= \lim_{x\to 0} \frac{\cos(x)-1}{\sin(x)} 
\end{align*}
Setting $f(x) = \cos(x)-1$ and $g(x)=\sin(x)$, both functions are differentiable near zero and 
\[
\lim_{x\to 0}(\cos(x)-1)=\lim_{x\to 0}\sin(x) = 0.
\]
We may now apply l'H\^opital's rule. Write
\[
f'(x) = -\sin(x)\qquad \text{and}\qquad g'(x) = \cos(x),
\]
so
\[
\lim_{x\to 0} \left(\cot(x) - \csc(x)\right) = \lim_{x\to 0} \frac{\cos(x)-1}{\sin(x)} = \lim_{x\to 0} \frac{-\sin(x)}{\cos(x)} =0.
\]
\end{example}


Sometimes one must be slightly more clever. 

\begin{example}[$\pmb\infty$--$\pmb\infty$]
Compute
\[
\lim_{x\to\infty}\left(\sqrt{x^2+x}-x\right).
\]

Again, this doesn't appear to be suitable for l'H\^opital's Rule. A bit of algebraic manipulation will help. Write
\begin{align*}
\lim_{x\to\infty}\left(\sqrt{x^2+x}-x\right) &= \lim_{x\to\infty}\left(x\left(\sqrt{1+1/x}-1\right)\right)\\
&=\lim_{x\to\infty}\frac{\sqrt{1+1/x}-1}{x^{-1}}
\end{align*}
Now set $f(x) = \sqrt{1+1/x}-1$, $g(x) = x^{-1}$. Since both
  functions are differentiable for large values of $x$ and 
\[
\lim_{x\to\infty} (\sqrt{1+1/x}-1) = \lim_{x\to\infty}x^{-1} = 0, 
\]
we may apply l'H\^opital's rule. Write
\[
f'(x) = (1/2)(1+1/x)^{-1/2}\cdot(-x^{-2}) \qquad \text{and}\qquad g'(x) = -x^{-2}
\]
so
\begin{align*}
\lim_{x\to\infty}\left(\sqrt{x^2+x}-x\right) &= \lim_{x\to\infty}\frac{\sqrt{1+1/x}-1}{x^{-1}} \\
&= \lim_{x\to\infty}\frac{(1/2)(1+1/x)^{-1/2}\cdot(-x^{-2})}{-x^{-2}} \\
&= \lim_{x\to\infty} \frac{1}{2\sqrt{1+1/x}}\\
&= \frac{1}{2}.
\end{align*}
\end{example}


\subsection*{Exponential Indeterminate Forms}

There is a standard trick for dealing with the indeterminate forms
\[
1^\infty,\qquad 0^0,\qquad \infty^0.
\]
Given $u(x)$ and $v(x)$ such that
\[
\lim_{x\to a}u(x)^{v(x)}
\]
falls into one of the categories described above, rewrite as
\[
\lim_{x\to a}e^{v(x)\ln(u(x))}
\]
and then examine the limit of the exponent
\[
\lim_{x\to a} v(x)\ln(u(x)) = \lim_{x\to a} \frac{\ln(u(x))}{v(x)^{-1}}
\]
using l'H\^opital's rule.  Since these forms are all very similar, we
will only give a single example.


\begin{example}[1$^{\pmb\infty}$]
Compute
\[
\lim_{x\to \infty}\left(1 + \frac{1}{x}\right)^x.
\]


Write
\[
\lim_{x\to \infty}\left(1 + \frac{1}{x}\right)^x = \lim_{x\to \infty}e^{x\ln\left(1 + \frac{1}{x}\right)}.
\]
So now look at the limit of the exponent
\[
\lim_{x\to\infty} x\ln\left(1 + \frac{1}{x}\right) = \lim_{x\to\infty} \frac{\ln\left(1 + \frac{1}{x}\right)}{x^{-1}}.
\]
Setting $f(x) = \ln\left(1 + \frac{1}{x}\right)$ and $g(x) = x^{-1}$,
both functions are differentiable for large values of $x$ and
\[
\lim_{x\to \infty}\ln\left(1 + \frac{1}{x}\right)=\lim_{x\to \infty}x^{-1} = 0.
\]
We may now apply l'H\^opital's rule. Write
\[
f'(x) = \frac{-x^{-2}}{1 + \frac{1}{x}}\qquad\text{and}\qquad g'(x) = -x^{-2},
\]
so
\begin{align*}
\lim_{x\to\infty} \frac{\ln\left(1 + \frac{1}{x}\right)}{x^{-1}} &= \lim_{x\to\infty} \frac{\frac{-x^{-2}}{1 + \frac{1}{x}}}{-x^{-2}} \\
&=\lim_{x\to\infty} \frac{1}{1 + \frac{1}{x}}\\
&=1.
\end{align*}
Hence, 
\[
\lim_{x\to \infty}\left(1 + \frac{1}{x}\right)^x = \lim_{x\to \infty}e^{x\ln\left(1 + \frac{1}{x}\right)} =e^{1} = e.
\]
\end{example}


\end{document}
