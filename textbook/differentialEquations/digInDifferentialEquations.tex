\documentclass{ximera}

%\usepackage{todonotes}
%\usepackage{mathtools} %% Required for wide table Curl and Greens
%\usepackage{cuted} %% Required for wide table Curl and Greens
\newcommand{\todo}{}

\usepackage{esint} % for \oiint
\ifxake%%https://math.meta.stackexchange.com/questions/9973/how-do-you-render-a-closed-surface-double-integral
\renewcommand{\oiint}{{\large\bigcirc}\kern-1.56em\iint}
\fi


\graphicspath{
  {./}
  {ximeraTutorial/}
  {basicPhilosophy/}
  {functionsOfSeveralVariables/}
  {normalVectors/}
  {lagrangeMultipliers/}
  {vectorFields/}
  {greensTheorem/}
  {shapeOfThingsToCome/}
  {dotProducts/}
  {partialDerivativesAndTheGradientVector/}
  {../productAndQuotientRules/exercises/}
  {../normalVectors/exercisesParametricPlots/}
  {../continuityOfFunctionsOfSeveralVariables/exercises/}
  {../partialDerivativesAndTheGradientVector/exercises/}
  {../directionalDerivativeAndChainRule/exercises/}
  {../commonCoordinates/exercisesCylindricalCoordinates/}
  {../commonCoordinates/exercisesSphericalCoordinates/}
  {../greensTheorem/exercisesCurlAndLineIntegrals/}
  {../greensTheorem/exercisesDivergenceAndLineIntegrals/}
  {../shapeOfThingsToCome/exercisesDivergenceTheorem/}
  {../greensTheorem/}
  {../shapeOfThingsToCome/}
  {../separableDifferentialEquations/exercises/}
  {vectorFields/}
}

\newcommand{\mooculus}{\textsf{\textbf{MOOC}\textnormal{\textsf{ULUS}}}}

\usepackage{tkz-euclide}\usepackage{tikz}
\usepackage{tikz-cd}
\usetikzlibrary{arrows}
\tikzset{>=stealth,commutative diagrams/.cd,
  arrow style=tikz,diagrams={>=stealth}} %% cool arrow head
\tikzset{shorten <>/.style={ shorten >=#1, shorten <=#1 } } %% allows shorter vectors

\usetikzlibrary{backgrounds} %% for boxes around graphs
\usetikzlibrary{shapes,positioning}  %% Clouds and stars
\usetikzlibrary{matrix} %% for matrix
\usepgfplotslibrary{polar} %% for polar plots
\usepgfplotslibrary{fillbetween} %% to shade area between curves in TikZ
\usetkzobj{all}
\usepackage[makeroom]{cancel} %% for strike outs
%\usepackage{mathtools} %% for pretty underbrace % Breaks Ximera
%\usepackage{multicol}
\usepackage{pgffor} %% required for integral for loops



%% http://tex.stackexchange.com/questions/66490/drawing-a-tikz-arc-specifying-the-center
%% Draws beach ball
\tikzset{pics/carc/.style args={#1:#2:#3}{code={\draw[pic actions] (#1:#3) arc(#1:#2:#3);}}}



\usepackage{array}
\setlength{\extrarowheight}{+.1cm}
\newdimen\digitwidth
\settowidth\digitwidth{9}
\def\divrule#1#2{
\noalign{\moveright#1\digitwidth
\vbox{\hrule width#2\digitwidth}}}





\newcommand{\RR}{\mathbb R}
\newcommand{\R}{\mathbb R}
\newcommand{\N}{\mathbb N}
\newcommand{\Z}{\mathbb Z}

\newcommand{\sagemath}{\textsf{SageMath}}


%\renewcommand{\d}{\,d\!}
\renewcommand{\d}{\mathop{}\!d}
\newcommand{\dd}[2][]{\frac{\d #1}{\d #2}}
\newcommand{\pp}[2][]{\frac{\partial #1}{\partial #2}}
\renewcommand{\l}{\ell}
\newcommand{\ddx}{\frac{d}{\d x}}

\newcommand{\zeroOverZero}{\ensuremath{\boldsymbol{\tfrac{0}{0}}}}
\newcommand{\inftyOverInfty}{\ensuremath{\boldsymbol{\tfrac{\infty}{\infty}}}}
\newcommand{\zeroOverInfty}{\ensuremath{\boldsymbol{\tfrac{0}{\infty}}}}
\newcommand{\zeroTimesInfty}{\ensuremath{\small\boldsymbol{0\cdot \infty}}}
\newcommand{\inftyMinusInfty}{\ensuremath{\small\boldsymbol{\infty - \infty}}}
\newcommand{\oneToInfty}{\ensuremath{\boldsymbol{1^\infty}}}
\newcommand{\zeroToZero}{\ensuremath{\boldsymbol{0^0}}}
\newcommand{\inftyToZero}{\ensuremath{\boldsymbol{\infty^0}}}



\newcommand{\numOverZero}{\ensuremath{\boldsymbol{\tfrac{\#}{0}}}}
\newcommand{\dfn}{\textbf}
%\newcommand{\unit}{\,\mathrm}
\newcommand{\unit}{\mathop{}\!\mathrm}
\newcommand{\eval}[1]{\bigg[ #1 \bigg]}
\newcommand{\seq}[1]{\left( #1 \right)}
\renewcommand{\epsilon}{\varepsilon}
\renewcommand{\phi}{\varphi}


\renewcommand{\iff}{\Leftrightarrow}

\DeclareMathOperator{\arccot}{arccot}
\DeclareMathOperator{\arcsec}{arcsec}
\DeclareMathOperator{\arccsc}{arccsc}
\DeclareMathOperator{\si}{Si}
\DeclareMathOperator{\scal}{scal}
\DeclareMathOperator{\sign}{sign}


%% \newcommand{\tightoverset}[2]{% for arrow vec
%%   \mathop{#2}\limits^{\vbox to -.5ex{\kern-0.75ex\hbox{$#1$}\vss}}}
\newcommand{\arrowvec}[1]{{\overset{\rightharpoonup}{#1}}}
%\renewcommand{\vec}[1]{\arrowvec{\mathbf{#1}}}
\renewcommand{\vec}[1]{{\overset{\boldsymbol{\rightharpoonup}}{\mathbf{#1}}}\hspace{0in}}

\newcommand{\point}[1]{\left(#1\right)} %this allows \vector{ to be changed to \vector{ with a quick find and replace
\newcommand{\pt}[1]{\mathbf{#1}} %this allows \vec{ to be changed to \vec{ with a quick find and replace
\newcommand{\Lim}[2]{\lim_{\point{#1} \to \point{#2}}} %Bart, I changed this to point since I want to use it.  It runs through both of the exercise and exerciseE files in limits section, which is why it was in each document to start with.

\DeclareMathOperator{\proj}{\mathbf{proj}}
\newcommand{\veci}{{\boldsymbol{\hat{\imath}}}}
\newcommand{\vecj}{{\boldsymbol{\hat{\jmath}}}}
\newcommand{\veck}{{\boldsymbol{\hat{k}}}}
\newcommand{\vecl}{\vec{\boldsymbol{\l}}}
\newcommand{\uvec}[1]{\mathbf{\hat{#1}}}
\newcommand{\utan}{\mathbf{\hat{t}}}
\newcommand{\unormal}{\mathbf{\hat{n}}}
\newcommand{\ubinormal}{\mathbf{\hat{b}}}

\newcommand{\dotp}{\bullet}
\newcommand{\cross}{\boldsymbol\times}
\newcommand{\grad}{\boldsymbol\nabla}
\newcommand{\divergence}{\grad\dotp}
\newcommand{\curl}{\grad\cross}
%\DeclareMathOperator{\divergence}{divergence}
%\DeclareMathOperator{\curl}[1]{\grad\cross #1}
\newcommand{\lto}{\mathop{\longrightarrow\,}\limits}

\renewcommand{\bar}{\overline}

\colorlet{textColor}{black}
\colorlet{background}{white}
\colorlet{penColor}{blue!50!black} % Color of a curve in a plot
\colorlet{penColor2}{red!50!black}% Color of a curve in a plot
\colorlet{penColor3}{red!50!blue} % Color of a curve in a plot
\colorlet{penColor4}{green!50!black} % Color of a curve in a plot
\colorlet{penColor5}{orange!80!black} % Color of a curve in a plot
\colorlet{penColor6}{yellow!70!black} % Color of a curve in a plot
\colorlet{fill1}{penColor!20} % Color of fill in a plot
\colorlet{fill2}{penColor2!20} % Color of fill in a plot
\colorlet{fillp}{fill1} % Color of positive area
\colorlet{filln}{penColor2!20} % Color of negative area
\colorlet{fill3}{penColor3!20} % Fill
\colorlet{fill4}{penColor4!20} % Fill
\colorlet{fill5}{penColor5!20} % Fill
\colorlet{gridColor}{gray!50} % Color of grid in a plot

\newcommand{\surfaceColor}{violet}
\newcommand{\surfaceColorTwo}{redyellow}
\newcommand{\sliceColor}{greenyellow}




\pgfmathdeclarefunction{gauss}{2}{% gives gaussian
  \pgfmathparse{1/(#2*sqrt(2*pi))*exp(-((x-#1)^2)/(2*#2^2))}%
}


%%%%%%%%%%%%%
%% Vectors
%%%%%%%%%%%%%

%% Simple horiz vectors
\renewcommand{\vector}[1]{\left\langle #1\right\rangle}


%% %% Complex Horiz Vectors with angle brackets
%% \makeatletter
%% \renewcommand{\vector}[2][ , ]{\left\langle%
%%   \def\nextitem{\def\nextitem{#1}}%
%%   \@for \el:=#2\do{\nextitem\el}\right\rangle%
%% }
%% \makeatother

%% %% Vertical Vectors
%% \def\vector#1{\begin{bmatrix}\vecListA#1,,\end{bmatrix}}
%% \def\vecListA#1,{\if,#1,\else #1\cr \expandafter \vecListA \fi}

%%%%%%%%%%%%%
%% End of vectors
%%%%%%%%%%%%%

%\newcommand{\fullwidth}{}
%\newcommand{\normalwidth}{}



%% makes a snazzy t-chart for evaluating functions
%\newenvironment{tchart}{\rowcolors{2}{}{background!90!textColor}\array}{\endarray}

%%This is to help with formatting on future title pages.
\newenvironment{sectionOutcomes}{}{}



%% Flowchart stuff
%\tikzstyle{startstop} = [rectangle, rounded corners, minimum width=3cm, minimum height=1cm,text centered, draw=black]
%\tikzstyle{question} = [rectangle, minimum width=3cm, minimum height=1cm, text centered, draw=black]
%\tikzstyle{decision} = [trapezium, trapezium left angle=70, trapezium right angle=110, minimum width=3cm, minimum height=1cm, text centered, draw=black]
%\tikzstyle{question} = [rectangle, rounded corners, minimum width=3cm, minimum height=1cm,text centered, draw=black]
%\tikzstyle{process} = [rectangle, minimum width=3cm, minimum height=1cm, text centered, draw=black]
%\tikzstyle{decision} = [trapezium, trapezium left angle=70, trapezium right angle=110, minimum width=3cm, minimum height=1cm, text centered, draw=black]


\outcome{Define a differential equation.}
\outcome{Verify solutions to differential equations.}

\title[Dig-In:]{Differential equations}

\begin{document}
\begin{abstract}
  We study equations with that relate functions with their rates.
\end{abstract}
\maketitle

A \textit{differential equation}\index{differential equation} is
simply an equation with a derivative in it. Here is an example:
\[
a\cdot f''(x) + b\cdot f'(x) + c\cdot f(x) = g(x). 
\]
\begin{question}
  What is a differential equation?
  \begin{multipleChoice}
    \choice{An equation that you take the derivative of.}
    \choice[correct]{An equation that relates the rate of a function to other values.}
    \choice{It is a formula for the slope of a tangent line at a given point.}  
  \end{multipleChoice}
\end{question}

When a mathematician solves a differential equation, they are finding
\textit{functions} satisfying the equation.
\begin{question}
  Which of the following functions solve the differential equation
  \[
  f''(x) = -f(x)?
  \]
  \begin{selectAll}
    \choice{$e^x$}
    \choice[correct]{$\sin(x)$}
    \choice[correct]{$\cos(x)$}
  \end{selectAll}
\end{question}






\section{Exponential growth and decay}

A function $f$ exhibits \textit{exponential growth}\index{exponential
  growth} if its growth rate is proportional to its value. As a
differential equation, this means
\[
f'(x) = k f(x)\qquad\text{for some constant of proportionality $k$.}
\]
We claim that this differential equation is solved by
\[
f(x) = Ae^{kx},
\]
where $A$ and $k$ are constants.  Check it out, if $f(x) = Ae^{kx}$,
then
\begin{align*}
f'(x) &= Ak e^{kx}\\
&= k\left(Ae^{kx} \right)\\
&= k f(x).
\end{align*}


\begin{example}
A culture of yeast starts with $100$ cells. After $160$ minutes, there
are $350$ cells. Assuming that the growth rate of the yeast is
proportional to the number of yeast cells present, estimate when the
culture will have $1000$ cells.

\begin{explanation}
Since the growth rate of the yeast is proportional to the number of
yeast cells present, we have the following differential equation
\[
p'(t) = k p(t)
\]
where $p(t)$ is the population of the yeast culture at time $t$ with
$t$ measured in minutes. We know that this differential equation is
solved by the function
\[
p(t) = A e^{kt}
\]
where $A$ and $k$ are yet to be determined constants. Since
\[
100 = p(0) = Ae^{k\cdot 0}
\]
we see that $A = 100$. So 
\[
p(t) = 100 e^{kt}.
\]
Now we must find $k$. Since we know that 
\[
\answer[given]{350} = p(160) = 100e^{k\cdot 160}
\]
we need to solve for $k$. Write
\begin{align*}
350 &= 100 e^{k\cdot 160}\\
3.5 &= e^{k\cdot 160}\\
\ln(3.5) &= k\cdot 160\\
\ln(3.5)/160 &= k. 
\end{align*}
Hence
\[
p(t) = 100 e^{t\ln(3.5)/160} = 100 \cdot 3.5^{t/160}.
\]
To find out when the culture has 1000 cells, write
\begin{align*}
1000 &= 100 \cdot 3.5^{t/160}\\
10 &= 3.5^{t/160}\\
\ln(10) &= \frac{t\ln(3.5)}{160}\\
\frac{160\ln(10)}{\ln(3.5)} &= t.
\end{align*}
From this we find that after approximately $294$ minutes, there are
around $1000$ yeast cells present.
\end{explanation}
\end{example}

It is worth seeing an example of exponential decay as well. Consider
this: Living tissue contains two types of carbon, a stable isotope
carbon-12 and a radioactive (unstable) isotope carbon-14. While an
organism is alive, the ratio of one isotope of carbon to the other is
always constant. When the organism dies, the ratio changes as the
radioactive isotope decays.  This is the basis of radiocarbon dating.


\begin{example}
The half-life of carbon-14 (the time it takes for half of an amount of
carbon-14 to decay) is about 5730 years. Moreover, the rate of decay
of carbon-14 is proportional to the amount of carbon-14.

If we find a bone with $1/70$th of the amount of carbon-14 we would
expect to find in a living organism, approximately how old is the
bone?

\begin{explanation}
Since the rate of decay of carbon-14 is proportional to the amount of
carbon-14 present, we can model this situation with the differential
equation
\[
f'(t) = k f(t).
\]
We know that this differential equation is solved by the function
defined by
\[
f(t) = A e^{kt}
\]
where $A$ and $k$ are yet to be determined constants. Since the
half-life of carbon-14 is about $5730$ years we write
\[
\frac{1}{2} = e^{k 5730}.
\]
Solving this equation for $k$, gives
\[
k = \frac{-\ln(2)}{5730}.
\]
Since we currently have $1/70$th of the original amount of carbon-14
we write
\[
\answer[given]{\frac{1}{70}} = 1\cdot e^{\frac{-\ln(2)t}{5730}}.
\]
Solving this equation for $t$, we find $t \approx -35121$. This means
that the bone is approximately $35121$ years old.
\end{explanation}
\end{example}




\section{Infectious diseases}

There are many models for the spread of infectious diseases. Perhaps
the most basic is the following:
\[
\mathrm{infect}'(t) = k\cdot \mathrm{infect}(t)\cdot(P-\mathrm{infect}(t))
\]
where $k$ is a constant, $\mathrm{infect}(t)$ is the number of people
infected by the disease on day $t$, and $P$ is the size of the
population vulnerable to the disease.

What this is saying is that the rate that the infectious disease
spreads is proportional to the product of the infected by the
uninfected:
  \begin{image}
    \begin{tikzpicture}
    \node at (0,0) {
      $\underbrace{\mathrm{infect}'(t)} = \overbrace{k} \cdot \underbrace{\mathrm{infect}(t)\cdot(P-\mathrm{infect}(t))}$
    };
    \node at (-2.6,-.7) {\small{rate the disease spreads}};
    \node at (-.9,.7) {\small{is proportional to}};
    \node at (1.3,-.7) {\small{this product}};
    \end{tikzpicture}
  \end{image}

Why might this make a good model? We expect the rate that
disease is spreading to be largest when
\[
\mathrm{infect}(t) \approx P/2.
\]
The product
\[
\mathrm{infect}(t)\cdot (P-\mathrm{infect}(t))
\]
is largest when $\mathrm{infect}(t) = P/2$. Finally we add the constant
of proportionality as a scale factor.




\begin{example}
  Suppose your calculus class has had a freak outbreak of the
  \textit{math-philia}. Some facts: We have around $200$ students in
  our class, we are now on the $23$rd day of the outbreak, and
  currently $100$ students are infected. Using the differential
  equation
  \[
  \mathrm{infect}'(t) = k\cdot \mathrm{infect}(t)\cdot (P-\mathrm{infect}(t))
  \]
  we can model the spread of \textit{math-philia} by setting $k=0.001$.
  What is $\mathrm{infect}'(23)$?

  \begin{explanation}
    Here all we need to do is substitute all of the necessary
    information into the differential equation. We know
    \begin{align*}
    t &= \answer[given]{23},\\
    P &= \answer[given]{200},\\
    \mathrm{infect}(23) &= \answer[given]{100},\\
    k&= 0.001.
    \end{align*}
    So
    \begin{align*}
      \mathrm{infect}'(t) &= k\cdot \mathrm{infect}(t)\cdot (P-\mathrm{infect}(t))\\
      &= 0.001\cdot 100\cdot(200-100)\\
      &=\answer[given]{10}.
    \end{align*}
    Hence on day $23$, we expect the disease to be spreading at a rate
    of $\answer[given]{10}$ newly infected people per day.
  \end{explanation}
\end{example}


%% To actually solve this differential equation, you need more
%% information. In particular, you need to know $\mathrm{infect}(0)$.
%% Setting $\mathrm{infect}(0) = a$,
%% \[
%% \mathrm{infect}'(t) = k\cdot \mathrm{infect}(t)(P-\mathrm{infect}(t))
%% \]
%% is solved by
%% \[
%% \mathrm{infect}(t) = \frac{P\cdot a \cdot e^{P\cdot k\cdot t}}{P + a \cdot \left(e^{P\cdot k\cdot t}-1\right)}.
%% \]
\end{document}
