\documentclass{ximera}

%\usepackage{todonotes}
%\usepackage{mathtools} %% Required for wide table Curl and Greens
%\usepackage{cuted} %% Required for wide table Curl and Greens
\newcommand{\todo}{}

\usepackage{esint} % for \oiint
\ifxake%%https://math.meta.stackexchange.com/questions/9973/how-do-you-render-a-closed-surface-double-integral
\renewcommand{\oiint}{{\large\bigcirc}\kern-1.56em\iint}
\fi


\graphicspath{
  {./}
  {ximeraTutorial/}
  {basicPhilosophy/}
  {functionsOfSeveralVariables/}
  {normalVectors/}
  {lagrangeMultipliers/}
  {vectorFields/}
  {greensTheorem/}
  {shapeOfThingsToCome/}
  {dotProducts/}
  {partialDerivativesAndTheGradientVector/}
  {../productAndQuotientRules/exercises/}
  {../normalVectors/exercisesParametricPlots/}
  {../continuityOfFunctionsOfSeveralVariables/exercises/}
  {../partialDerivativesAndTheGradientVector/exercises/}
  {../directionalDerivativeAndChainRule/exercises/}
  {../commonCoordinates/exercisesCylindricalCoordinates/}
  {../commonCoordinates/exercisesSphericalCoordinates/}
  {../greensTheorem/exercisesCurlAndLineIntegrals/}
  {../greensTheorem/exercisesDivergenceAndLineIntegrals/}
  {../shapeOfThingsToCome/exercisesDivergenceTheorem/}
  {../greensTheorem/}
  {../shapeOfThingsToCome/}
  {../separableDifferentialEquations/exercises/}
  {vectorFields/}
}

\newcommand{\mooculus}{\textsf{\textbf{MOOC}\textnormal{\textsf{ULUS}}}}

\usepackage{tkz-euclide}\usepackage{tikz}
\usepackage{tikz-cd}
\usetikzlibrary{arrows}
\tikzset{>=stealth,commutative diagrams/.cd,
  arrow style=tikz,diagrams={>=stealth}} %% cool arrow head
\tikzset{shorten <>/.style={ shorten >=#1, shorten <=#1 } } %% allows shorter vectors

\usetikzlibrary{backgrounds} %% for boxes around graphs
\usetikzlibrary{shapes,positioning}  %% Clouds and stars
\usetikzlibrary{matrix} %% for matrix
\usepgfplotslibrary{polar} %% for polar plots
\usepgfplotslibrary{fillbetween} %% to shade area between curves in TikZ
\usetkzobj{all}
\usepackage[makeroom]{cancel} %% for strike outs
%\usepackage{mathtools} %% for pretty underbrace % Breaks Ximera
%\usepackage{multicol}
\usepackage{pgffor} %% required for integral for loops



%% http://tex.stackexchange.com/questions/66490/drawing-a-tikz-arc-specifying-the-center
%% Draws beach ball
\tikzset{pics/carc/.style args={#1:#2:#3}{code={\draw[pic actions] (#1:#3) arc(#1:#2:#3);}}}



\usepackage{array}
\setlength{\extrarowheight}{+.1cm}
\newdimen\digitwidth
\settowidth\digitwidth{9}
\def\divrule#1#2{
\noalign{\moveright#1\digitwidth
\vbox{\hrule width#2\digitwidth}}}





\newcommand{\RR}{\mathbb R}
\newcommand{\R}{\mathbb R}
\newcommand{\N}{\mathbb N}
\newcommand{\Z}{\mathbb Z}

\newcommand{\sagemath}{\textsf{SageMath}}


%\renewcommand{\d}{\,d\!}
\renewcommand{\d}{\mathop{}\!d}
\newcommand{\dd}[2][]{\frac{\d #1}{\d #2}}
\newcommand{\pp}[2][]{\frac{\partial #1}{\partial #2}}
\renewcommand{\l}{\ell}
\newcommand{\ddx}{\frac{d}{\d x}}

\newcommand{\zeroOverZero}{\ensuremath{\boldsymbol{\tfrac{0}{0}}}}
\newcommand{\inftyOverInfty}{\ensuremath{\boldsymbol{\tfrac{\infty}{\infty}}}}
\newcommand{\zeroOverInfty}{\ensuremath{\boldsymbol{\tfrac{0}{\infty}}}}
\newcommand{\zeroTimesInfty}{\ensuremath{\small\boldsymbol{0\cdot \infty}}}
\newcommand{\inftyMinusInfty}{\ensuremath{\small\boldsymbol{\infty - \infty}}}
\newcommand{\oneToInfty}{\ensuremath{\boldsymbol{1^\infty}}}
\newcommand{\zeroToZero}{\ensuremath{\boldsymbol{0^0}}}
\newcommand{\inftyToZero}{\ensuremath{\boldsymbol{\infty^0}}}



\newcommand{\numOverZero}{\ensuremath{\boldsymbol{\tfrac{\#}{0}}}}
\newcommand{\dfn}{\textbf}
%\newcommand{\unit}{\,\mathrm}
\newcommand{\unit}{\mathop{}\!\mathrm}
\newcommand{\eval}[1]{\bigg[ #1 \bigg]}
\newcommand{\seq}[1]{\left( #1 \right)}
\renewcommand{\epsilon}{\varepsilon}
\renewcommand{\phi}{\varphi}


\renewcommand{\iff}{\Leftrightarrow}

\DeclareMathOperator{\arccot}{arccot}
\DeclareMathOperator{\arcsec}{arcsec}
\DeclareMathOperator{\arccsc}{arccsc}
\DeclareMathOperator{\si}{Si}
\DeclareMathOperator{\scal}{scal}
\DeclareMathOperator{\sign}{sign}


%% \newcommand{\tightoverset}[2]{% for arrow vec
%%   \mathop{#2}\limits^{\vbox to -.5ex{\kern-0.75ex\hbox{$#1$}\vss}}}
\newcommand{\arrowvec}[1]{{\overset{\rightharpoonup}{#1}}}
%\renewcommand{\vec}[1]{\arrowvec{\mathbf{#1}}}
\renewcommand{\vec}[1]{{\overset{\boldsymbol{\rightharpoonup}}{\mathbf{#1}}}\hspace{0in}}

\newcommand{\point}[1]{\left(#1\right)} %this allows \vector{ to be changed to \vector{ with a quick find and replace
\newcommand{\pt}[1]{\mathbf{#1}} %this allows \vec{ to be changed to \vec{ with a quick find and replace
\newcommand{\Lim}[2]{\lim_{\point{#1} \to \point{#2}}} %Bart, I changed this to point since I want to use it.  It runs through both of the exercise and exerciseE files in limits section, which is why it was in each document to start with.

\DeclareMathOperator{\proj}{\mathbf{proj}}
\newcommand{\veci}{{\boldsymbol{\hat{\imath}}}}
\newcommand{\vecj}{{\boldsymbol{\hat{\jmath}}}}
\newcommand{\veck}{{\boldsymbol{\hat{k}}}}
\newcommand{\vecl}{\vec{\boldsymbol{\l}}}
\newcommand{\uvec}[1]{\mathbf{\hat{#1}}}
\newcommand{\utan}{\mathbf{\hat{t}}}
\newcommand{\unormal}{\mathbf{\hat{n}}}
\newcommand{\ubinormal}{\mathbf{\hat{b}}}

\newcommand{\dotp}{\bullet}
\newcommand{\cross}{\boldsymbol\times}
\newcommand{\grad}{\boldsymbol\nabla}
\newcommand{\divergence}{\grad\dotp}
\newcommand{\curl}{\grad\cross}
%\DeclareMathOperator{\divergence}{divergence}
%\DeclareMathOperator{\curl}[1]{\grad\cross #1}
\newcommand{\lto}{\mathop{\longrightarrow\,}\limits}

\renewcommand{\bar}{\overline}

\colorlet{textColor}{black}
\colorlet{background}{white}
\colorlet{penColor}{blue!50!black} % Color of a curve in a plot
\colorlet{penColor2}{red!50!black}% Color of a curve in a plot
\colorlet{penColor3}{red!50!blue} % Color of a curve in a plot
\colorlet{penColor4}{green!50!black} % Color of a curve in a plot
\colorlet{penColor5}{orange!80!black} % Color of a curve in a plot
\colorlet{penColor6}{yellow!70!black} % Color of a curve in a plot
\colorlet{fill1}{penColor!20} % Color of fill in a plot
\colorlet{fill2}{penColor2!20} % Color of fill in a plot
\colorlet{fillp}{fill1} % Color of positive area
\colorlet{filln}{penColor2!20} % Color of negative area
\colorlet{fill3}{penColor3!20} % Fill
\colorlet{fill4}{penColor4!20} % Fill
\colorlet{fill5}{penColor5!20} % Fill
\colorlet{gridColor}{gray!50} % Color of grid in a plot

\newcommand{\surfaceColor}{violet}
\newcommand{\surfaceColorTwo}{redyellow}
\newcommand{\sliceColor}{greenyellow}




\pgfmathdeclarefunction{gauss}{2}{% gives gaussian
  \pgfmathparse{1/(#2*sqrt(2*pi))*exp(-((x-#1)^2)/(2*#2^2))}%
}


%%%%%%%%%%%%%
%% Vectors
%%%%%%%%%%%%%

%% Simple horiz vectors
\renewcommand{\vector}[1]{\left\langle #1\right\rangle}


%% %% Complex Horiz Vectors with angle brackets
%% \makeatletter
%% \renewcommand{\vector}[2][ , ]{\left\langle%
%%   \def\nextitem{\def\nextitem{#1}}%
%%   \@for \el:=#2\do{\nextitem\el}\right\rangle%
%% }
%% \makeatother

%% %% Vertical Vectors
%% \def\vector#1{\begin{bmatrix}\vecListA#1,,\end{bmatrix}}
%% \def\vecListA#1,{\if,#1,\else #1\cr \expandafter \vecListA \fi}

%%%%%%%%%%%%%
%% End of vectors
%%%%%%%%%%%%%

%\newcommand{\fullwidth}{}
%\newcommand{\normalwidth}{}



%% makes a snazzy t-chart for evaluating functions
%\newenvironment{tchart}{\rowcolors{2}{}{background!90!textColor}\array}{\endarray}

%%This is to help with formatting on future title pages.
\newenvironment{sectionOutcomes}{}{}



%% Flowchart stuff
%\tikzstyle{startstop} = [rectangle, rounded corners, minimum width=3cm, minimum height=1cm,text centered, draw=black]
%\tikzstyle{question} = [rectangle, minimum width=3cm, minimum height=1cm, text centered, draw=black]
%\tikzstyle{decision} = [trapezium, trapezium left angle=70, trapezium right angle=110, minimum width=3cm, minimum height=1cm, text centered, draw=black]
%\tikzstyle{question} = [rectangle, rounded corners, minimum width=3cm, minimum height=1cm,text centered, draw=black]
%\tikzstyle{process} = [rectangle, minimum width=3cm, minimum height=1cm, text centered, draw=black]
%\tikzstyle{decision} = [trapezium, trapezium left angle=70, trapezium right angle=110, minimum width=3cm, minimum height=1cm, text centered, draw=black]


\outcome{}

\title[Dig-In:]{Differential equations}

\begin{document}
\begin{abstract}
  We study equations with that relate functions with their rates.
\end{abstract}
\maketitle

A \textit{differential equation}\index{differential equation} is
simply an equation with a derivative in it. Here is an example:
\[
f'(x) = k f(x).
\]
When a mathematician solves a differential equation, they are finding
a \textit{function} that satisfies the equation.

\section{Falling objects}

Recall that the acceleration due to gravity is about $-9.8$
m/s$^2$. Since the first derivative of the function giving the
velocity of an object gives the acceleration of the object and the
second derivative of a function giving the position of a falling
object gives the acceleration, we have the differential equations
\begin{align*}
v'(t) &=  -9.8,\\
p''(t) &=  -9.8.
\end{align*}
From these simple equation, we can derive equations for the velocity of
the object and for the position using antiderivatives.


\begin{example}
A ball is tossed into the air with an initial velocity of $15$
m/s. What is the velocity of the ball after 1 second? How about after
2 seconds?
\begin{explanation}
Knowing that the acceleration due to gravity is $-9.8$ m/s$^2$, we write
\[
v'(t) = -9.8.
\]
To solve this differential equation, take the antiderivative of both sides
\begin{align*}
\int v'(t) \d t &= \int -9.8 \d t\\
v(t) &= -9.8t + C.
\end{align*}
Here $C$ represents the initial velocity of the ball. Since it is
tossed up with an initial velocity of $15$ m/s, 
\[
15 = v(0) = -9.8\cdot 0 + C,
\]
and we see that $C=15$. Hence $v(t) = -9.8t + 15$. Now $v(1) = 5.2$
m/s, the ball is rising, and $v(2) = -4.6$ m/s, the ball is falling.
\end{explanation}

Now let's do a similar problem, but instead of finding the velocity,
we will find the position.

\begin{example}
A ball is tossed into the air with an initial velocity of $15$ m/s
from a height of 2 meters. When does the ball hit the ground?

\begin{explanation}
Knowing that the acceleration due to gravity is $-9.8$ m/s$^2$, we write
\[
p''(t) = -9.8.
\]
Start by taking the antiderivative of both sides of the equation
\begin{align*}
\int p''(t) \d t &= \int -9.8 \d t\\
p'(t) &= -9.8t + C.
\end{align*}
Here $C$ represents the initial velocity of the ball. Since it is
tossed up with an initial velocity of $15$ m/s, $C = 15$ and 
\[
p'(t) = -9.8t + 15.
\]
Now let's take the antiderivative again. 
\begin{align*}
\int p'(t) \d t &= \int -9.8t +15\d t\\
p(t) &= \frac{-9.8t^2}{2} + 15t + D.
\end{align*}
Since we know the initial height was $2$ meters, write
\[
2 = p(0) =  \frac{-9.8\cdot 0^2}{2} + 15\cdot 0 + D.
\]
Hence $p(t) = \frac{-9.8t^2}{2} + 15t + 2$. We need to know when the
ball hits the ground, this is when $p(t)=0$. Solving the equation
\[
\frac{-9.8t^2}{2} + 15t + 2 = 0
\]
we find two solutions $t\approx -0.1$ and $t\approx 3.2$. Discarding
the negative solution, we see the ball will hit the ground after
approximately $3.2$ seconds.
\end{explanation}
\end{example}

The power of calculus is that it frees us from rote memorization of
formulas and enables us to derive what we need.



\section{Exponential growth and decay}

A function $f(x)$ exhibits \textit{exponential
  growth}\index{exponential growth} if its growth rate is proportional
to its value. As a differential equation, this means
\[
f'(x) = k f(x)\qquad\text{for some constant of proportionality $k$.}
\]
We claim that this differential equation is solved by $f(x) = A
e^{kx}$, where $A$ and $k$ are constants.  Check it out, if $f(x) =
Ae^{kx}$, then
\begin{align*}
f'(x) &= Ak e^{kx}\\
&= k\left(Ae^{kx} \right)\\
&= k f(x).
\end{align*}

\begin{example}
A culture of yeast starts with 100 cells. After 160 minutes, there
are 350 cells. Assuming that the growth rate of the yeast is
proportional to the number of yeast cells present, estimate when the
culture will have 1000 cells.

\begin{explanation}
Since the growth rate of the yeast is proportional to the number of
yeast cells present, we have the following differential equation
\[
p'(t) = k p(t)
\]
where $p(t)$ is the population of the yeast culture and $t$ is time
measured in minutes. We know that this differential equation is solved
by the function
\[
p(t) = A e^{kt}
\]
where $A$ and $k$ are yet to be determined constants. Since
\[
100 = p(0) = Ae^{k\cdot 0}
\]
we see that $A = 100$. So 
\[
p(t) = 100 e^{kt}.
\]
Now we must find $k$. Since we know that 
\[
350 = p(160) = 100e^{k\cdot 160}
\]
we need to solve for $k$. Write
\begin{align*}
350 &= 100 e^{k\cdot 160}\\
3.5 &= e^{k\cdot 160}\\
\ln(3.5) &= k\cdot 160\\
\ln(3.5)/160 &= k. 
\end{align*}
Hence
\[
p(t) = 100 e^{t\ln(3.5)/160} = 100 \cdot 3.5^{t/160}.
\]
To find out when the culture has 1000 cells, write
\begin{align*}
1000 &= 100 \cdot 3.5^{t/160}\\
10 &= 3.5^{t/160}\\
\ln(10) &= \frac{t\ln(3.5)}{160}\\
\frac{160\ln(10)}{\ln(3.5)} &= t.
\end{align*}
From this we find that after approximately $294$ minutes, there are
around $1000$ yeast cells present.
\end{explanation}
\end{example}

It is worth seeing an example of exponential decay as well. Consider
this: Living tissue contains two types of carbon, a stable
isotope carbon-12 and a radioactive (unstable) isotope
carbon-14. While an organism is alive, the ratio of one isotope of
carbon to the other is always constant. When the organism dies, the
ratio changes as the radioactive isotope decays. This is the basis of
radiocarbon dating.


\begin{example}
The half-life of carbon-14 (the time it takes for half of an amount of
carbon-14 to decay) is about 5730 years. If the rate of decay is
proportional to the amount of carbon-14, and if we found a bone with
$1/70$th of the amount of carbon-14 we would expect to find in a living
organism, approximately how old is the bone?

\begin{explanation}
Since the rate of decay of carbon-14 is proportional to the amount of
carbon-14 present, we can model this situation with the differential
equation
\[
f'(t) = k f(t).
\]
We know that this differential equation is solved by the function
defined by
\[
f(t) = A e^{kt}
\]
where $A$ and $k$ are yet to be determined constants. Since the
half-life of carbon-14 is about $5730$ years we write
\[
\frac{1}{2} = e^{k 5730}.
\]
Solving this equation for $k$, gives
\[
k = \frac{-\ln(2)}{5730}.
\]
Since we currently have $1/70$th of the original amount of carbon-14
we write
\[
\frac{1}{70} = 1\cdot e^{\frac{-\ln(2)t}{5730}}.
\]
Solving this equation for $t$, we find $t \approx -35121$. This means
that the bone is approximately $35121$ years old.
\end{explanation}
\end{example}

\section{Formulas or none}

In science and mathematics, it is often easier to setup a differential
equation than it is to solve it. In this case, a numerical
solution is often ``good enough.''

Suppose you have set up the following differential equation
\[
f'(x) = \left(f(x)\right)^2 - 6f(x) + 8.
\]
While one can solve this differential equation, we cannot solve it
\textit{yet}. Supposing we needed a solution, we could try to find a
numerical solution using Euler's Method.

\begin{margintable}[-1in]
\[
\begin{tchart}{lll}
n & x_n & y_n \\ \hline
0 & 1   & 3.8 \\
1 & 1.2 & 3.73\\
2 & 1.4 & 3.63\\
3 & 1.6 & 3.51\\
4 & 1.8 & 3.37\\
5 & 2 & 3.19\\
6 & 2.2   & 3.00 \\
7 & 2.4 & 2.80\\
8 & 2.6 & 2.61\\
9 & 2.8 & 2.44\\
10 & 3 & 2.30
\end{tchart}
\]
\caption{Variation of Euler's Method for the differential equation
  $f'(x) = \left(f(x)\right)^2 - 6f(x) + 8$ with initial condition
  $f(1) = 3.8$.}
\label{table:diffeuler1}
\end{margintable}


\begin{example}\label{example:slopefield1}
Consider the differential equation
\[
f'(x) = \left(f(x)\right)^2 - 6f(x) + 8.
\]
Suppose you know that $f(1)= 3.8$. Rounding to two decimals at each
step, use Euler's Method with $h=0.2$ to approximate $f(3)$. 


\begin{marginfigure}[0in]
\begin{tikzpicture}
	\begin{axis}[
            xmin=1, xmax=3,ymin=0,ymax=5,
            axis lines =center, xlabel=$x$, ylabel=$y$,
            every axis y label/.style={at=(current axis.above origin),anchor=south},
            every axis x label/.style={at=(current axis.right of origin),anchor=west},
            xtick={1,1.2,1.4,1.6,1.8,2,2.2,2.4,2.6,2.8,3},
            axis on top,
          ]         
	  \addplot [penColor2, very thick] plot coordinates {
            (1,3.8) (1.2,3.73) (1.4,3.63) (1.6,3.51) (1.8,3.37) (2,3.19)
            (2.2,3) (2.4,2.80) (2.6,2.61) (2.8,2.44) (3,2.30)
          };
          \node at (axis cs:.3,2) [penColor] {$f(x)$};          
        \end{axis}
\end{tikzpicture}
\caption{Here we see our polygonal curve found via Euler's Method
  based on the differential equation $f'(x) = \left(f(x)\right)^2 -
  6f(x) + 8$, with initial value $f(1) =3.8$.  Choosing a smaller
  step-size $h$ would yield a better approximation.}
\label{figure:eulerDiffEQ}
\end{marginfigure}

\begin{explanation}
To solve this problem we'll use a variation on Euler's Method. We'll
make a table following this format
\[
\begin{tchart}{lll}
n & x_n     & y_n \\ \hline
0 & x_0     & y_0 \\
1 & x_0 + h & y_0+h\cdot \left(y_0^2-6y_0+8\right)\\
2 & x_1 + h & y_1+h\cdot \left(y_1^2-6y_1+8\right)\\
3 & x_2 + h & y_2+h\cdot \left(y_2^2-6y_2+8\right)\\
4 & x_3 + h & y_3+h\cdot \left(y_3^2-6y_3+8\right)\\
  \hdotsfor{3}
\end{tchart}
\]
At each step, we are simply making a linear approximation to
$f(x)$. Filling out this table, we produce
Table~\ref{table:diffeuler1}.  Hence our estimate for $f(3)$ is
$2.30$, see Figure~\ref{figure:eulerDiffEQ}.
\end{explanation}
\end{example}


Let's try this example again with a different initial condition.


\begin{example}\label{example:slopefield2}
Consider the differential equation
\[
f'(x) = \left(f(x)\right)^2 - 6f(x) + 8.
\]
Suppose you know that $f(1)= 4$. Rounding to two decimals at each
step, use Euler's Method with $h=0.2$ to approximate $f(3)$. 

\begin{explanation}
Again we'll use a variation on Euler's Method. Making the table as we
did before, see Table~\ref{table:diffeuler2}.  This time our solution
is simply the function $f(x) = 4$. Note, this does solve the
differential equation as, given
\begin{align*}
f'(x) &= \left(f(x)\right)^2 - 6f(x) + 8\\
0 &= \left(4\right)^2 - 6\cdot4 + 8.
\end{align*}
\end{explanation}
\end{example}

\begin{margintable}[0in]
\[
\begin{tchart}{lll}
n & x_n & y_n \\ \hline
0 & 1   & 4 \\
1 & 1.2 & 4\\
2 & 1.4 & 4\\
3 & 1.6 & 4\\
4 & 1.8 & 4\\
5 & 2 & 4\\
6 & 2.2 & 4 \\
7 & 2.4 & 4\\
8 & 2.6 & 4\\
9 & 2.8 & 4\\
10 & 3 & 4
\end{tchart}
\]
\caption{Variation of Euler's Method for the differential equation
  $f'(x) = \left(f(x)\right)^2 - 6f(x) + 8$ with initial condition
  $f(1) = 4$.}
\label{table:diffeuler2}
\end{margintable}

Finally, we'll try do the same example again with another initial
condition.

\begin{margintable}[0in]
\[
\begin{tchart}{lll}
n & x_n & y_n \\ \hline
0 & 1   & 2 \\
1 & 1.2 & 2\\
2 & 1.4 & 2\\
3 & 1.6 & 2\\
4 & 1.8 & 2\\
5 & 2 & 2\\
6 & 2.2 & 2 \\
7 & 2.4 & 2\\
8 & 2.6 & 2\\
9 & 2.8 & 2\\
10 & 3 & 2
\end{tchart}
\]
\caption{Variation of Euler's Method for the differential equation
  $f'(x) = \left(f(x)\right)^2 - 6f(x) + 8$ with initial condition
  $f(1) = 2$.}
\label{table:diffeuler3}
\end{margintable}

\begin{example}\label{example:slopefield3}
Consider the differential equation
\[
f'(x) = \left(f(x)\right)^2 - 6f(x) + 8.
\]
Suppose you know that $f(1)= 2$. Rounding to two decimals at each
step, use Euler's Method with $h=0.2$ to approximate $f(3)$. 

\begin{explanation}
Using the same variation on Euler's Method as before, see
Table~\ref{table:diffeuler3}.  This time our solution is simply the
function $f(x) = 2$. Note, this does solve the differential equation
as, given
\begin{align*}
f'(x) &= \left(f(x)\right)^2 - 6f(x) + 8\\
0 &= \left(2\right)^2 - 6\cdot2 + 8.
\end{align*}
\end{explanation}
\end{example}


From our examples above, we see that certain differential equations
can have very different solutions based on initial conditions. To
really see what is happening here, we should look at a \textit{slope
  field}.\index{slope field}

\begin{procedureForConstructingASlopeField}
It is usually easiest to construct a slope field using a computer
algebra system. Nevertheless, the general theory of constructing a
slope field must be understood before one can do this.  Suppose you
have a differential equation relating $f(x)$ and $f'(x)$.
\begin{itemize}
\item Choose a value for $dx$, this will be your step-size.
\item Plot points on an $(x,y)$-plane in increments of size $dx$. 
\item For each point plotted, assume this point is on the curve $f(x)$. 
\item Now use your differential equation to plot an arrow pointing in
  the direction of $(dx,dy)$ from the given point, where $dy =
  f'(x)dx$. This means one draws an arrow in the same direction as the
  arrow from $(x,y)$ to $(x + dx, y + dy)$.
\end{itemize}
\end{procedureForConstructingASlopeField}

Consider the differential equation
\[
f'(x) = \left(f(x)\right)^2 - 6f(x) + 8.
\]
if the step-size is $dx =1$, and we are at the point $(3,1)$ then we
should plot an arrow in the same direction as the arrow whose tail is
at $(3,1)$ and whose tip is at
\begin{align*}
(3+1,1 + f'(1))  &= (4, 1 +1 -6+8)\\
&= (4,4).
\end{align*}
Let's examine the slope field for $f(x)$:

{\def\length{sqrt(1+(y^2-6*y+8)^2)}
\begin{tikzpicture}
  \begin{axis}[
      xmin=0, xmax=5,ymin=0,ymax=5,domain=0:5,view={0}{90},
      axis lines =center, xlabel=$x$, ylabel=$y$,
      every axis y label/.style={at=(current axis.above origin),anchor=south},
      every axis x label/.style={at=(current axis.right of origin),anchor=west},
      axis on top,
    ] 
    \addplot3 [penColor, quiver={u={1/\length}, v={(y^2 -6*y + 8)/(\length)},scale arrows=.2},-stealth,samples=20] {0};
  \end{axis}
\end{tikzpicture}}

Every solution to the differential equation should follow the arrows
in the slope field.  Compare this slope field to the solutions found
in Example~\ref{example:slopefield1},
Example~\ref{example:slopefield2}, and
Example~\ref{example:slopefield3}. The slope field allows us to
examine each solution of the given differential equation
simultaneously---this often gives more insight into a problem than a
single solution.


\end{document}
