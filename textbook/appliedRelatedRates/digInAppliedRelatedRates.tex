\documentclass{ximera}

%\usepackage{todonotes}
%\usepackage{mathtools} %% Required for wide table Curl and Greens
%\usepackage{cuted} %% Required for wide table Curl and Greens
\newcommand{\todo}{}

\usepackage{esint} % for \oiint
\ifxake%%https://math.meta.stackexchange.com/questions/9973/how-do-you-render-a-closed-surface-double-integral
\renewcommand{\oiint}{{\large\bigcirc}\kern-1.56em\iint}
\fi


\graphicspath{
  {./}
  {ximeraTutorial/}
  {basicPhilosophy/}
  {functionsOfSeveralVariables/}
  {normalVectors/}
  {lagrangeMultipliers/}
  {vectorFields/}
  {greensTheorem/}
  {shapeOfThingsToCome/}
  {dotProducts/}
  {partialDerivativesAndTheGradientVector/}
  {../productAndQuotientRules/exercises/}
  {../normalVectors/exercisesParametricPlots/}
  {../continuityOfFunctionsOfSeveralVariables/exercises/}
  {../partialDerivativesAndTheGradientVector/exercises/}
  {../directionalDerivativeAndChainRule/exercises/}
  {../commonCoordinates/exercisesCylindricalCoordinates/}
  {../commonCoordinates/exercisesSphericalCoordinates/}
  {../greensTheorem/exercisesCurlAndLineIntegrals/}
  {../greensTheorem/exercisesDivergenceAndLineIntegrals/}
  {../shapeOfThingsToCome/exercisesDivergenceTheorem/}
  {../greensTheorem/}
  {../shapeOfThingsToCome/}
  {../separableDifferentialEquations/exercises/}
  {vectorFields/}
}

\newcommand{\mooculus}{\textsf{\textbf{MOOC}\textnormal{\textsf{ULUS}}}}

\usepackage{tkz-euclide}\usepackage{tikz}
\usepackage{tikz-cd}
\usetikzlibrary{arrows}
\tikzset{>=stealth,commutative diagrams/.cd,
  arrow style=tikz,diagrams={>=stealth}} %% cool arrow head
\tikzset{shorten <>/.style={ shorten >=#1, shorten <=#1 } } %% allows shorter vectors

\usetikzlibrary{backgrounds} %% for boxes around graphs
\usetikzlibrary{shapes,positioning}  %% Clouds and stars
\usetikzlibrary{matrix} %% for matrix
\usepgfplotslibrary{polar} %% for polar plots
\usepgfplotslibrary{fillbetween} %% to shade area between curves in TikZ
\usetkzobj{all}
\usepackage[makeroom]{cancel} %% for strike outs
%\usepackage{mathtools} %% for pretty underbrace % Breaks Ximera
%\usepackage{multicol}
\usepackage{pgffor} %% required for integral for loops



%% http://tex.stackexchange.com/questions/66490/drawing-a-tikz-arc-specifying-the-center
%% Draws beach ball
\tikzset{pics/carc/.style args={#1:#2:#3}{code={\draw[pic actions] (#1:#3) arc(#1:#2:#3);}}}



\usepackage{array}
\setlength{\extrarowheight}{+.1cm}
\newdimen\digitwidth
\settowidth\digitwidth{9}
\def\divrule#1#2{
\noalign{\moveright#1\digitwidth
\vbox{\hrule width#2\digitwidth}}}





\newcommand{\RR}{\mathbb R}
\newcommand{\R}{\mathbb R}
\newcommand{\N}{\mathbb N}
\newcommand{\Z}{\mathbb Z}

\newcommand{\sagemath}{\textsf{SageMath}}


%\renewcommand{\d}{\,d\!}
\renewcommand{\d}{\mathop{}\!d}
\newcommand{\dd}[2][]{\frac{\d #1}{\d #2}}
\newcommand{\pp}[2][]{\frac{\partial #1}{\partial #2}}
\renewcommand{\l}{\ell}
\newcommand{\ddx}{\frac{d}{\d x}}

\newcommand{\zeroOverZero}{\ensuremath{\boldsymbol{\tfrac{0}{0}}}}
\newcommand{\inftyOverInfty}{\ensuremath{\boldsymbol{\tfrac{\infty}{\infty}}}}
\newcommand{\zeroOverInfty}{\ensuremath{\boldsymbol{\tfrac{0}{\infty}}}}
\newcommand{\zeroTimesInfty}{\ensuremath{\small\boldsymbol{0\cdot \infty}}}
\newcommand{\inftyMinusInfty}{\ensuremath{\small\boldsymbol{\infty - \infty}}}
\newcommand{\oneToInfty}{\ensuremath{\boldsymbol{1^\infty}}}
\newcommand{\zeroToZero}{\ensuremath{\boldsymbol{0^0}}}
\newcommand{\inftyToZero}{\ensuremath{\boldsymbol{\infty^0}}}



\newcommand{\numOverZero}{\ensuremath{\boldsymbol{\tfrac{\#}{0}}}}
\newcommand{\dfn}{\textbf}
%\newcommand{\unit}{\,\mathrm}
\newcommand{\unit}{\mathop{}\!\mathrm}
\newcommand{\eval}[1]{\bigg[ #1 \bigg]}
\newcommand{\seq}[1]{\left( #1 \right)}
\renewcommand{\epsilon}{\varepsilon}
\renewcommand{\phi}{\varphi}


\renewcommand{\iff}{\Leftrightarrow}

\DeclareMathOperator{\arccot}{arccot}
\DeclareMathOperator{\arcsec}{arcsec}
\DeclareMathOperator{\arccsc}{arccsc}
\DeclareMathOperator{\si}{Si}
\DeclareMathOperator{\scal}{scal}
\DeclareMathOperator{\sign}{sign}


%% \newcommand{\tightoverset}[2]{% for arrow vec
%%   \mathop{#2}\limits^{\vbox to -.5ex{\kern-0.75ex\hbox{$#1$}\vss}}}
\newcommand{\arrowvec}[1]{{\overset{\rightharpoonup}{#1}}}
%\renewcommand{\vec}[1]{\arrowvec{\mathbf{#1}}}
\renewcommand{\vec}[1]{{\overset{\boldsymbol{\rightharpoonup}}{\mathbf{#1}}}\hspace{0in}}

\newcommand{\point}[1]{\left(#1\right)} %this allows \vector{ to be changed to \vector{ with a quick find and replace
\newcommand{\pt}[1]{\mathbf{#1}} %this allows \vec{ to be changed to \vec{ with a quick find and replace
\newcommand{\Lim}[2]{\lim_{\point{#1} \to \point{#2}}} %Bart, I changed this to point since I want to use it.  It runs through both of the exercise and exerciseE files in limits section, which is why it was in each document to start with.

\DeclareMathOperator{\proj}{\mathbf{proj}}
\newcommand{\veci}{{\boldsymbol{\hat{\imath}}}}
\newcommand{\vecj}{{\boldsymbol{\hat{\jmath}}}}
\newcommand{\veck}{{\boldsymbol{\hat{k}}}}
\newcommand{\vecl}{\vec{\boldsymbol{\l}}}
\newcommand{\uvec}[1]{\mathbf{\hat{#1}}}
\newcommand{\utan}{\mathbf{\hat{t}}}
\newcommand{\unormal}{\mathbf{\hat{n}}}
\newcommand{\ubinormal}{\mathbf{\hat{b}}}

\newcommand{\dotp}{\bullet}
\newcommand{\cross}{\boldsymbol\times}
\newcommand{\grad}{\boldsymbol\nabla}
\newcommand{\divergence}{\grad\dotp}
\newcommand{\curl}{\grad\cross}
%\DeclareMathOperator{\divergence}{divergence}
%\DeclareMathOperator{\curl}[1]{\grad\cross #1}
\newcommand{\lto}{\mathop{\longrightarrow\,}\limits}

\renewcommand{\bar}{\overline}

\colorlet{textColor}{black}
\colorlet{background}{white}
\colorlet{penColor}{blue!50!black} % Color of a curve in a plot
\colorlet{penColor2}{red!50!black}% Color of a curve in a plot
\colorlet{penColor3}{red!50!blue} % Color of a curve in a plot
\colorlet{penColor4}{green!50!black} % Color of a curve in a plot
\colorlet{penColor5}{orange!80!black} % Color of a curve in a plot
\colorlet{penColor6}{yellow!70!black} % Color of a curve in a plot
\colorlet{fill1}{penColor!20} % Color of fill in a plot
\colorlet{fill2}{penColor2!20} % Color of fill in a plot
\colorlet{fillp}{fill1} % Color of positive area
\colorlet{filln}{penColor2!20} % Color of negative area
\colorlet{fill3}{penColor3!20} % Fill
\colorlet{fill4}{penColor4!20} % Fill
\colorlet{fill5}{penColor5!20} % Fill
\colorlet{gridColor}{gray!50} % Color of grid in a plot

\newcommand{\surfaceColor}{violet}
\newcommand{\surfaceColorTwo}{redyellow}
\newcommand{\sliceColor}{greenyellow}




\pgfmathdeclarefunction{gauss}{2}{% gives gaussian
  \pgfmathparse{1/(#2*sqrt(2*pi))*exp(-((x-#1)^2)/(2*#2^2))}%
}


%%%%%%%%%%%%%
%% Vectors
%%%%%%%%%%%%%

%% Simple horiz vectors
\renewcommand{\vector}[1]{\left\langle #1\right\rangle}


%% %% Complex Horiz Vectors with angle brackets
%% \makeatletter
%% \renewcommand{\vector}[2][ , ]{\left\langle%
%%   \def\nextitem{\def\nextitem{#1}}%
%%   \@for \el:=#2\do{\nextitem\el}\right\rangle%
%% }
%% \makeatother

%% %% Vertical Vectors
%% \def\vector#1{\begin{bmatrix}\vecListA#1,,\end{bmatrix}}
%% \def\vecListA#1,{\if,#1,\else #1\cr \expandafter \vecListA \fi}

%%%%%%%%%%%%%
%% End of vectors
%%%%%%%%%%%%%

%\newcommand{\fullwidth}{}
%\newcommand{\normalwidth}{}



%% makes a snazzy t-chart for evaluating functions
%\newenvironment{tchart}{\rowcolors{2}{}{background!90!textColor}\array}{\endarray}

%%This is to help with formatting on future title pages.
\newenvironment{sectionOutcomes}{}{}



%% Flowchart stuff
%\tikzstyle{startstop} = [rectangle, rounded corners, minimum width=3cm, minimum height=1cm,text centered, draw=black]
%\tikzstyle{question} = [rectangle, minimum width=3cm, minimum height=1cm, text centered, draw=black]
%\tikzstyle{decision} = [trapezium, trapezium left angle=70, trapezium right angle=110, minimum width=3cm, minimum height=1cm, text centered, draw=black]
%\tikzstyle{question} = [rectangle, rounded corners, minimum width=3cm, minimum height=1cm,text centered, draw=black]
%\tikzstyle{process} = [rectangle, minimum width=3cm, minimum height=1cm, text centered, draw=black]
%\tikzstyle{decision} = [trapezium, trapezium left angle=70, trapezium right angle=110, minimum width=3cm, minimum height=1cm, text centered, draw=black]


\title[Dig-In:]{Applied related rates}

\begin{document}
\begin{abstract}
\end{abstract}
\maketitle

Now we are ready to work related rates problems in context. Just as
before, we are going to follow essentially the same plan of attack in
each problem.


\begin{description}
\item[\textbf{Draw a picture.}] If possible, draw a schematic picture with all the relevant information. 
\item[\textbf{Find equations.}] We want equations that relate all
  relevant functions.
\item[\textbf{Differentiate the equations.}] Here we will often use
  implicit differentiation.
\item[\textbf{Evaluate and solve.}] Evaluate
  each equation at all known desired values and solve for the relevant
  rate.
\end{description}



\section{Formulas}


\begin{example}
A plane is flying directly away from you at $500$ mph at an altitude
of $3$ miles.  How fast is the plane's distance from you increasing at
the moment when the plane is flying over a point on the ground $4$
miles from you?

\begin{explanation}
  To start, \textbf{draw a picture}.
  
  Next we need to \textbf{find equations}.
  
  Imagining that $p$ and $s$ are functions of time, we now
  \textbf{differentiate the equation}.

  Now we'll \textbf{evaluate and solve}.
\end{explanation}
\end{example}

\section{Right triangles}

\begin{example}
A plane is flying directly away from you at $500$ mph at an altitude
of $3$ miles.  How fast is the plane's distance from you increasing at
the moment when the plane is flying over a point on the ground $4$
miles from you?


\begin{explanation}
To start, \textbf{draw a picture}.
\begin{image}
\begin{tikzpicture}
\draw[penColor2, dashed, very thick] (0,0) -- (5,4);
%\draw[penColor, dashed, very thick] (0,0) -- (0,4);
\draw[penColor, dashed, very thick] (5,0) -- (5,4);
\draw[penColor, dashed, very thick] (0,0) -- (5,0);
\draw[->,penColor, very thick] (1,4) -- (6,4);
\draw [penColor, fill] (5,4) circle [radius=.07];
%\node [left,penColor] at (0,0) {\scalebox{3} \Ladiesroom};
%\node [right,penColor] at (6,4) {\scalebox{3}{\ding{40}}};
\node [right,penColor] at (5,2) {$3$ miles};
\node [above,penColor] at (3,4) {$p'(t) = 500$ mph};
\node [above,penColor] at (5,4) {$p(t)$};
\node [below,penColor] at (2.5,0) {$4$ miles};
\node [left,penColor2] at (2.4,2) {$s(t)$ miles};
\draw [penColor, fill] (0,0) circle [radius=.07];
\node [left,penColor] at (-.1,0) {You};
\end{tikzpicture}
\end{image}
Next we need to \textbf{find equations}. By the Pythagorean Theorem
we know that
\[
p^2+3^2=s^2.
\] 
Imagining that $p$ and $s$ are functions of time, we now
\textbf{differentiate the equation}. Write
\[
2\cdot p(t)\cdot p'(t)  = 2\cdot s(t) \cdot s'(t).
\] 
Now we'll \textbf{evaluate and solve}.  We
are interested in the time at which $p(t)=4$ and $p'(t) =
500$. Additionally, at this time we know that $4^2+9=s^2$, so
$s(t)=5$.  Putting together all the information we get
\[
2(4)(500)=2(5)s'(t),
\]
thus $s'(t)=400$ mph.
\end{explanation}
\end{example}



\begin{example}
A road running north to south crosses a road going east to west at the
point $P$.  Cyclist $A$ is riding north along the first road, and
cyclist $B$ is riding east along the second road.  At a particular
time, cyclist $A$ is $3$ kilometers to the north of $P$ and traveling
at $20$ km/hr, while cyclist $B$ is $4$ kilometers to the east of $P$
and traveling at $15$ km/hr.  How fast is the distance between the two
cyclists changing?


\begin{explanation}
We start the same way we always do, we \textbf{draw a picture}.
\begin{image}
\begin{tikzpicture}
\draw[->,penColor!50!background, very thick] (-1,0) -- (4,0);
\draw[->,penColor!50!background, very thick] (0,-1) -- (0,4);
\draw[->,penColor, very thick] (0,3) -- (0,4);
\draw[->,penColor, very thick] (3,0) -- (4,0);
\draw [penColor, fill] (0,0) circle [radius=.07];
\draw [penColor, fill] (3,0) circle [radius=.07];
\draw [penColor, fill] (0,3) circle [radius=.07];
\draw[dashed,penColor2, very thick] (3,0) -- (0,3);

%\node[penColor,rotate=90,right] at (.5,3) {\scalebox{-2} \Bicycle};
\node[penColor,right] at (0,.2) {$P$};
\node[penColor,left] at (-.3,3) {$a'(t) = 20$ km/hr};
\node[penColor,left] at (0,1.5) {$3$ km};
\node[penColor,below] at (1.5,0) {$4$ km};
\node[penColor,below] at (4,0) {$b'(t)= 15$ km/hr};
\node[penColor2,above] at (1.6,1.6) {$c(t)$};
%\node[penColor,right,above] at (3.5,0) {\scalebox{-2}[2] \Bicycle};
\end{tikzpicture}
\end{image}
Here $a(t)$ is the distance of cyclist $A$ north of $P$ at time $t$,
and $b(t)$ the distance of cyclist $B$ east of $P$ at time $t$, and
$c(t)$ is the distance from cyclist $A$ to cyclist $B$ at time $t$.

We must \textbf{find equations}.  By the Pythagorean Theorem,
\[
c(t)^2=a(t)^2+b(t)^2.
\] 
Now we can \textbf{differentiate the equation}. Taking derivatives we
get
\[
2c(t)c'(t)=2a(t)a'(t)+2b(t)b'(t).
\]
Now we can \textbf{evaluate and solve}.  We know that $a(t) = 3$,
$a'(t) = 20$, $b(t) = 4$ and $b'(t) = 15$. Hence by the Pythagorean
Theorem, $c(t) = 5$. So
\[
2\cdot 5 \cdot c'(t) = 2 \cdot 3\cdot 20 + 2 \cdot 4 \cdot 15
\]
solving for $c'(t)$ we find $c'(t) = 24$ km/hr.
\end{explanation}
\end{example}



\section{Angular rates}

\begin{example}
A swing consists of a board at the end of a $10$ ft long rope.  Think
of the board as a point $P$ at the end of the rope, and let $Q$ be the
point of attachment at the other end.  Suppose that the swing is
directly below $Q$ at time $t=0$, and is being pushed by someone who
walks at 6 ft/sec from left to right.  What is the angular speed of
the rope in deg/sec after 1 sec?


\begin{explanation}
To start, \textbf{draw a picture}.

\begin{tikzpicture}[scale=1.3]
\draw[penColor!50!background, very thick] (0,3) -- (0,-1);
\draw[penColor, very thick] (0,3) -- (2.12,-.12);
\draw [penColor!50!background, very thick] (-2.12,-.12) arc [radius=3, start angle=225, end angle= 315];
\draw [penColor2, very thick] (0,2.3) arc [radius=.7, start angle=270, end angle= 305];
\draw[->, penColor, very thick] (0,-.12) -- (2,-.12);

%\node[penColor] at (2.12,-.12) {\scalebox{3} \Ladiesroom};
\draw [penColor, fill] (2.12,-.12) circle [radius=.07];
\node[penColor,right] at (2.3,-.12) {$P$};
\node[penColor,right,above] at (0,3) {$Q$};
\node[penColor,right] at (1.06,1.5) {$10$ ft};
\node[penColor,above] at (1,-.12) {$\dd[x]{t} = 6$ ft/sec};
\node[penColor2,right,above] at (.3,2) {$\theta$};
\end{tikzpicture}

Now we must \textbf{find equations}. From the right triangle in our
picture, we see
\[
\sin(\theta)=x/10.
\]
We can now \textbf{differentiate the equation}. Taking derivatives we obtain 
\[
\cos(\theta)\cdot \theta'(t)=0.1 x'(t).
\]
Now we can \textbf{evaluate and solve}.  When
$t=1$ sec, the person was pushed by someone who walks $6$
ft/sec. Hence we have a $6-8-10$ right triangle, with $x'(t) = 6$, and
$\cos\theta=8/10$. Thus
\[
(8/10) \theta'(t) =6/10,
\]
and so  $\theta'(t)=6/8=3/4$ rad/sec, or approximately $43$ deg/sec.
\end{explanation} 
\end{example}

CLOCK HANDS



\section{Similar triangles}


\begin{example} Water is poured into a conical container at the rate of 10
cm${}^3$/sec.  The cone points directly down, and it has a height of
30 cm and a base radius of 10 cm.  How fast is the water level rising
when the water is 4 cm deep?

\begin{explanation}
To start, \textbf{draw a picture}.
\begin{image}
\begin{tikzpicture}
\draw[penColor,very thick] (0,4) ellipse (4 and 1);
\draw[very thick,penColor!20!background] (2,2) arc (0:180:2 and .5);% top half of ellipse
\draw[very thick,penColor] (-2,2) arc (180:360:2 and .5);% bottom half of ellipse
\draw[penColor, very thick] (3.97,3.85) -- (0,0);
\draw[penColor, very thick] (-3.97,3.85) -- (0,0);
\draw[penColor, very thick] (0,4) -- (4,4);
\draw[penColor!50!background, very thick] (0,2) -- (2,2);
\draw[->,line width=0.4cm, penColor!20!background] (0,6) -- (0,4.25);
\draw[dashed, penColor2, very thick] (2.1,0) -- (2.1,2);
\draw[dashed, penColor, very thick] (-4.1,0) -- (-4.1,4);
\node[right, penColor] at (.4,5.6) {$\dd[V]{t} = 10$ cm$^3$/sec};
\node[below, penColor] at (2,4) {$10$ cm};
\node[above, penColor] at (1,2) {$r$ cm};
\node[right, penColor2] at (2.1,1) {$h(t) = 4$ cm};
\node[left, penColor] at (-4.1,2) {$30$ cm};
\end{tikzpicture}
\end{image}
Note, no attempt was made to draw this picture to scale, rather we
want all of the relevant information to be available to the
mathematician.

Now we need to \textbf{find equations}. The formula for the volume of a cone tells us that 
\[
V = \frac{\pi}{3} r^2 h.
\]

Now we must \textbf{differentiate the equation}. We should use implicit differentiation, and treat each of the variables as functions of $t$. Write
\begin{equation}\label{equation:cone/water}
\dd[V]{t} = \frac{\pi}{3}\left(2rh \dd[r]{t} + r^2 \dd[h]{t}\right).
\end{equation}

At this point we \textbf{evaluate and solve}.
At first something seems to be wrong, we do not know $\dd[r]{t}$.
However, the dimensions of the cone of water must have the same
proportions as those of the container.  That is, because of similar
triangles, 
\[
\frac{r}{h}=\frac{10}{30} \qquad\text{so}\qquad r={h/3}.
\]  
In particular, we see that when $h = 4$, $r=4/3$ and 
\[
\dd[r]{t} = \frac{1}{3}\cdot \dd[h]{t}.
\]
Now we can \textbf{evaluate the equation at the desired
  values}. Starting with Equation~\ref{equation:cone/water}, we plug
in $\dd[V]{t} = 10$, $r = 4/3$, $\dd[r]{t} = \frac{1}{3}\cdot \dd[h]{t}$
and $h=4$. Write
\begin{align*}
10 &= \frac{\pi}{3}\left(2\cdot \frac{4}{3}\cdot 4 \cdot\frac{1}{3}\cdot\dd[h]{t} + \left(\frac{4}{3}\right)^2 \dd[h]{t}\right)\\
10 &= \frac{\pi}{3}\left(\frac{32}{9}\dd[h]{t} + \frac{16}{9} \dd[h]{t}\right)\\
10 &= \frac{16\pi}{9}\dd[h]{t}\\
\frac{90}{16\pi} &= \dd[h]{t}.
\end{align*}
Thus, $\dd[h]{t}=\frac{90}{16\pi}$ cm/sec.
\end{explanation}
\end{example}


Walking shadow





\end{document}
