\documentclass{ximera}

%\usepackage{todonotes}
%\usepackage{mathtools} %% Required for wide table Curl and Greens
%\usepackage{cuted} %% Required for wide table Curl and Greens
\newcommand{\todo}{}

\usepackage{esint} % for \oiint
\ifxake%%https://math.meta.stackexchange.com/questions/9973/how-do-you-render-a-closed-surface-double-integral
\renewcommand{\oiint}{{\large\bigcirc}\kern-1.56em\iint}
\fi


\graphicspath{
  {./}
  {ximeraTutorial/}
  {basicPhilosophy/}
  {functionsOfSeveralVariables/}
  {normalVectors/}
  {lagrangeMultipliers/}
  {vectorFields/}
  {greensTheorem/}
  {shapeOfThingsToCome/}
  {dotProducts/}
  {partialDerivativesAndTheGradientVector/}
  {../productAndQuotientRules/exercises/}
  {../normalVectors/exercisesParametricPlots/}
  {../continuityOfFunctionsOfSeveralVariables/exercises/}
  {../partialDerivativesAndTheGradientVector/exercises/}
  {../directionalDerivativeAndChainRule/exercises/}
  {../commonCoordinates/exercisesCylindricalCoordinates/}
  {../commonCoordinates/exercisesSphericalCoordinates/}
  {../greensTheorem/exercisesCurlAndLineIntegrals/}
  {../greensTheorem/exercisesDivergenceAndLineIntegrals/}
  {../shapeOfThingsToCome/exercisesDivergenceTheorem/}
  {../greensTheorem/}
  {../shapeOfThingsToCome/}
  {../separableDifferentialEquations/exercises/}
  {vectorFields/}
}

\newcommand{\mooculus}{\textsf{\textbf{MOOC}\textnormal{\textsf{ULUS}}}}

\usepackage{tkz-euclide}\usepackage{tikz}
\usepackage{tikz-cd}
\usetikzlibrary{arrows}
\tikzset{>=stealth,commutative diagrams/.cd,
  arrow style=tikz,diagrams={>=stealth}} %% cool arrow head
\tikzset{shorten <>/.style={ shorten >=#1, shorten <=#1 } } %% allows shorter vectors

\usetikzlibrary{backgrounds} %% for boxes around graphs
\usetikzlibrary{shapes,positioning}  %% Clouds and stars
\usetikzlibrary{matrix} %% for matrix
\usepgfplotslibrary{polar} %% for polar plots
\usepgfplotslibrary{fillbetween} %% to shade area between curves in TikZ
\usetkzobj{all}
\usepackage[makeroom]{cancel} %% for strike outs
%\usepackage{mathtools} %% for pretty underbrace % Breaks Ximera
%\usepackage{multicol}
\usepackage{pgffor} %% required for integral for loops



%% http://tex.stackexchange.com/questions/66490/drawing-a-tikz-arc-specifying-the-center
%% Draws beach ball
\tikzset{pics/carc/.style args={#1:#2:#3}{code={\draw[pic actions] (#1:#3) arc(#1:#2:#3);}}}



\usepackage{array}
\setlength{\extrarowheight}{+.1cm}
\newdimen\digitwidth
\settowidth\digitwidth{9}
\def\divrule#1#2{
\noalign{\moveright#1\digitwidth
\vbox{\hrule width#2\digitwidth}}}





\newcommand{\RR}{\mathbb R}
\newcommand{\R}{\mathbb R}
\newcommand{\N}{\mathbb N}
\newcommand{\Z}{\mathbb Z}

\newcommand{\sagemath}{\textsf{SageMath}}


%\renewcommand{\d}{\,d\!}
\renewcommand{\d}{\mathop{}\!d}
\newcommand{\dd}[2][]{\frac{\d #1}{\d #2}}
\newcommand{\pp}[2][]{\frac{\partial #1}{\partial #2}}
\renewcommand{\l}{\ell}
\newcommand{\ddx}{\frac{d}{\d x}}

\newcommand{\zeroOverZero}{\ensuremath{\boldsymbol{\tfrac{0}{0}}}}
\newcommand{\inftyOverInfty}{\ensuremath{\boldsymbol{\tfrac{\infty}{\infty}}}}
\newcommand{\zeroOverInfty}{\ensuremath{\boldsymbol{\tfrac{0}{\infty}}}}
\newcommand{\zeroTimesInfty}{\ensuremath{\small\boldsymbol{0\cdot \infty}}}
\newcommand{\inftyMinusInfty}{\ensuremath{\small\boldsymbol{\infty - \infty}}}
\newcommand{\oneToInfty}{\ensuremath{\boldsymbol{1^\infty}}}
\newcommand{\zeroToZero}{\ensuremath{\boldsymbol{0^0}}}
\newcommand{\inftyToZero}{\ensuremath{\boldsymbol{\infty^0}}}



\newcommand{\numOverZero}{\ensuremath{\boldsymbol{\tfrac{\#}{0}}}}
\newcommand{\dfn}{\textbf}
%\newcommand{\unit}{\,\mathrm}
\newcommand{\unit}{\mathop{}\!\mathrm}
\newcommand{\eval}[1]{\bigg[ #1 \bigg]}
\newcommand{\seq}[1]{\left( #1 \right)}
\renewcommand{\epsilon}{\varepsilon}
\renewcommand{\phi}{\varphi}


\renewcommand{\iff}{\Leftrightarrow}

\DeclareMathOperator{\arccot}{arccot}
\DeclareMathOperator{\arcsec}{arcsec}
\DeclareMathOperator{\arccsc}{arccsc}
\DeclareMathOperator{\si}{Si}
\DeclareMathOperator{\scal}{scal}
\DeclareMathOperator{\sign}{sign}


%% \newcommand{\tightoverset}[2]{% for arrow vec
%%   \mathop{#2}\limits^{\vbox to -.5ex{\kern-0.75ex\hbox{$#1$}\vss}}}
\newcommand{\arrowvec}[1]{{\overset{\rightharpoonup}{#1}}}
%\renewcommand{\vec}[1]{\arrowvec{\mathbf{#1}}}
\renewcommand{\vec}[1]{{\overset{\boldsymbol{\rightharpoonup}}{\mathbf{#1}}}\hspace{0in}}

\newcommand{\point}[1]{\left(#1\right)} %this allows \vector{ to be changed to \vector{ with a quick find and replace
\newcommand{\pt}[1]{\mathbf{#1}} %this allows \vec{ to be changed to \vec{ with a quick find and replace
\newcommand{\Lim}[2]{\lim_{\point{#1} \to \point{#2}}} %Bart, I changed this to point since I want to use it.  It runs through both of the exercise and exerciseE files in limits section, which is why it was in each document to start with.

\DeclareMathOperator{\proj}{\mathbf{proj}}
\newcommand{\veci}{{\boldsymbol{\hat{\imath}}}}
\newcommand{\vecj}{{\boldsymbol{\hat{\jmath}}}}
\newcommand{\veck}{{\boldsymbol{\hat{k}}}}
\newcommand{\vecl}{\vec{\boldsymbol{\l}}}
\newcommand{\uvec}[1]{\mathbf{\hat{#1}}}
\newcommand{\utan}{\mathbf{\hat{t}}}
\newcommand{\unormal}{\mathbf{\hat{n}}}
\newcommand{\ubinormal}{\mathbf{\hat{b}}}

\newcommand{\dotp}{\bullet}
\newcommand{\cross}{\boldsymbol\times}
\newcommand{\grad}{\boldsymbol\nabla}
\newcommand{\divergence}{\grad\dotp}
\newcommand{\curl}{\grad\cross}
%\DeclareMathOperator{\divergence}{divergence}
%\DeclareMathOperator{\curl}[1]{\grad\cross #1}
\newcommand{\lto}{\mathop{\longrightarrow\,}\limits}

\renewcommand{\bar}{\overline}

\colorlet{textColor}{black}
\colorlet{background}{white}
\colorlet{penColor}{blue!50!black} % Color of a curve in a plot
\colorlet{penColor2}{red!50!black}% Color of a curve in a plot
\colorlet{penColor3}{red!50!blue} % Color of a curve in a plot
\colorlet{penColor4}{green!50!black} % Color of a curve in a plot
\colorlet{penColor5}{orange!80!black} % Color of a curve in a plot
\colorlet{penColor6}{yellow!70!black} % Color of a curve in a plot
\colorlet{fill1}{penColor!20} % Color of fill in a plot
\colorlet{fill2}{penColor2!20} % Color of fill in a plot
\colorlet{fillp}{fill1} % Color of positive area
\colorlet{filln}{penColor2!20} % Color of negative area
\colorlet{fill3}{penColor3!20} % Fill
\colorlet{fill4}{penColor4!20} % Fill
\colorlet{fill5}{penColor5!20} % Fill
\colorlet{gridColor}{gray!50} % Color of grid in a plot

\newcommand{\surfaceColor}{violet}
\newcommand{\surfaceColorTwo}{redyellow}
\newcommand{\sliceColor}{greenyellow}




\pgfmathdeclarefunction{gauss}{2}{% gives gaussian
  \pgfmathparse{1/(#2*sqrt(2*pi))*exp(-((x-#1)^2)/(2*#2^2))}%
}


%%%%%%%%%%%%%
%% Vectors
%%%%%%%%%%%%%

%% Simple horiz vectors
\renewcommand{\vector}[1]{\left\langle #1\right\rangle}


%% %% Complex Horiz Vectors with angle brackets
%% \makeatletter
%% \renewcommand{\vector}[2][ , ]{\left\langle%
%%   \def\nextitem{\def\nextitem{#1}}%
%%   \@for \el:=#2\do{\nextitem\el}\right\rangle%
%% }
%% \makeatother

%% %% Vertical Vectors
%% \def\vector#1{\begin{bmatrix}\vecListA#1,,\end{bmatrix}}
%% \def\vecListA#1,{\if,#1,\else #1\cr \expandafter \vecListA \fi}

%%%%%%%%%%%%%
%% End of vectors
%%%%%%%%%%%%%

%\newcommand{\fullwidth}{}
%\newcommand{\normalwidth}{}



%% makes a snazzy t-chart for evaluating functions
%\newenvironment{tchart}{\rowcolors{2}{}{background!90!textColor}\array}{\endarray}

%%This is to help with formatting on future title pages.
\newenvironment{sectionOutcomes}{}{}



%% Flowchart stuff
%\tikzstyle{startstop} = [rectangle, rounded corners, minimum width=3cm, minimum height=1cm,text centered, draw=black]
%\tikzstyle{question} = [rectangle, minimum width=3cm, minimum height=1cm, text centered, draw=black]
%\tikzstyle{decision} = [trapezium, trapezium left angle=70, trapezium right angle=110, minimum width=3cm, minimum height=1cm, text centered, draw=black]
%\tikzstyle{question} = [rectangle, rounded corners, minimum width=3cm, minimum height=1cm,text centered, draw=black]
%\tikzstyle{process} = [rectangle, minimum width=3cm, minimum height=1cm, text centered, draw=black]
%\tikzstyle{decision} = [trapezium, trapezium left angle=70, trapezium right angle=110, minimum width=3cm, minimum height=1cm, text centered, draw=black]


\title[Dig-In:]{Continuity}

\begin{document}

\begin{abstract}
\end{abstract}
\maketitle



Informally, a function is continuous if you can ``draw it'' without
``lifting your pencil.'' We need a formal definition.

\begin{definition}
  A function $f$ is \dfn{continuous at a point} $a$ if $\lim_{x\to a}
  f(x) = f(a)$.
\end{definition}
\index{continuous}


\begin{example}
Find the discontinuities (the values for $x$ where a function is not
continuous) for the function described below:
\begin{image}
\begin{tikzpicture}
	\begin{axis}[
            domain=0:10,
            ymax=5,
            ymin=0,
            samples=100,
            axis lines =middle, xlabel=$x$, ylabel=$y$,
            every axis y label/.style={at=(current axis.above origin),anchor=south},
            every axis x label/.style={at=(current axis.right of origin),anchor=west}
          ]
	  \addplot [very thick, penColor, smooth, domain=(4:10)] {3 + sin(deg(x*2))/(x-1)};
          \addplot [very thick, penColor, smooth, domain=(0:4)] {1};
          \addplot[color=penColor,fill=background,only marks,mark=*] coordinates{(4,3.30)};  %% open hole
          \addplot[color=penColor,fill=background,only marks,mark=*] coordinates{(6,2.893)};  %% open hole
          \addplot[color=penColor,fill=penColor,only marks,mark=*] coordinates{(4,1)};  %% closed hole
          \addplot[color=penColor,fill=penColor,only marks,mark=*] coordinates{(6,2)};  %% closed hole
        \end{axis}
\end{tikzpicture}
%% \caption{A plot of a function with discontinuities at $x=4$ and $x=6$.}
%% \label{plot:discontinuous-function}
\end{image}



From Figure~\ref{plot:discontinuous-function} we see that $\lim_{x\to 4} f(x)$ does not exist as
\[
\lim_{x\to 4-}f(x) = 1\qquad\text{and}\qquad \lim_{x\to 4+}f(x) \approx 3.5
\]
Hence $\lim_{x\to 4} f(x) \ne f(4)$, and so $f(x)$ is not
continuous at $x=4$.

We also see that $\lim_{x\to 6} f(x) \approx 3$ while $f(6) =
2$. Hence $\lim_{x\to 6} f(x) \ne f(6)$, and so $f(x)$ is not
continuous at $x=6$.
\end{example}

Building from the definition of \textit{continuous at a point}, we can
now define what it means for a function to be \textit{continuous} on
an interval.

\begin{definition} A function $f$ is \textbf{continuous on an interval} if it is
continuous at every point in the interval.
\end{definition}

In particular, we should note that if a function is not defined on an
interval, then it \textbf{cannot} be continuous on that interval.
\begin{image}
\begin{tikzpicture}
	\begin{axis}[
            domain=-.2:.2,    
            samples=500,
            axis lines =middle, xlabel=$x$, ylabel=$y$,
            yticklabels = {}, 
            every axis y label/.style={at=(current axis.above origin),anchor=south},
            every axis x label/.style={at=(current axis.right of origin),anchor=west},
            clip=false,
          ]
	  \addplot [very thick, penColor, smooth, domain=(-.2:-.02)] {abs(x)^(1/5)*sin(deg(1/x))};
          \addplot [very thick, penColor, smooth, domain=(.02:.2)] {x^(1/5)*sin(deg(1/x))};
	  \addplot [color=penColor, fill=penColor, very thick, smooth,domain=(-.02:.02)] {abs(x)^(1/5)} \closedcycle;
          \addplot [color=penColor, fill=penColor, very thick, smooth,domain=(-.02:.02)] {-abs(x)^(1/5)} \closedcycle;
        \end{axis}
\end{tikzpicture}
%% \caption[A continuous function.]{A plot of
%% \[
%% f(x)=
%% \begin{cases}
%% \sqrt[5]{x}\sin\left(\frac{1}{x}\right) & \text{if $x \ne 0$,}\\
%%  0 & \text{if $x = 0$.}
%% \end{cases}
%% \]
%% }
%% \label{plot:sqrt[5]{x}sin 1/x}
\end{image}

\begin{example}
Consider the function
\[
f(x) = 
\begin{cases}
\sqrt[5]{x}\sin\left(\frac{1}{x}\right) & \text{if $x \ne 0$,}\\
0 & \text{if $x = 0$,}
\end{cases}
\]
see Figure~\ref{plot:sqrt[5]{x}sin 1/x}. Is this function continuous?


Considering $f(x)$, the only issue is when $x=0$. We must show that
$\lim_{x\to 0} f(x) = 0$. Note
\[
-|\sqrt[5]{x}|\le f(x) \le |\sqrt[5]{x}|.
\]
Since
\[
\lim_{x\to 0} -|\sqrt[5]{x}| = 0 = \lim_{x\to 0}|\sqrt[5]{x}|,
\]
we see by the Squeeze Theorem, Theorem~\ref{theorem:squeeze}, that
$\lim_{x\to 0} f(x) = 0$. Hence $f(x)$ is continuous.

Here we see how the informal definition of continuity being that you
can ``draw it'' without ``lifting your pencil'' differs from the
formal definition.
\end{example}

We close with a useful theorem about continuous functions:

\begin{theorem}[Intermediate Value Theorem]\label{theorem:IVT}
If $f(x)$ is a continuous function for all $x$ in the closed interval
$[a,b]$ and $d$ is between $f(a)$ and $f(b)$, then there is a number
$c$ in $[a, b]$ such that $f(c) = d$.
\end{theorem}

%% \marginnote[-1.2in]{The Intermediate Value Theorem is most frequently
%%   used when $d=0$.}  
%% \marginnote[-.7in]{For a nice proof of this theorem, see: Walk, Stephen
%%   M. \textit{The intermediate value theorem is NOT obvious---and I am
%%     going to prove it to you}. College Math. J. 42 (2011), no. 4,
%%   254--259.}
%% In Figure~\ref{figure:intermediate-value}, we see a geometric
%% interpretation of this theorem.
\todo{here we have a note on the proof, see the source}


\begin{image}
\begin{tikzpicture}
	\begin{axis}[
            domain=0:6, ymin=0, ymax=2.2,xmax=6,
            axis lines =left, xlabel=$x$, ylabel=$y$,
            every axis y label/.style={at=(current axis.above origin),anchor=south},
            every axis x label/.style={at=(current axis.right of origin),anchor=west},
            xtick={1,3.597,5}, ytick={.203,1,1.679},
            xticklabels={$a$,$c$,$b$}, yticklabels={$f(a)$,$f(c)=d$,$f(b)$},
            axis on top,
          ]
          \addplot [draw=none, fill=fill2, domain=(0:7)] {1.679} \closedcycle;
          \addplot [draw=none, fill=background, domain=(0:7)] {.203} \closedcycle;
          \addplot [textColor,dashed] plot coordinates {(0,1.679) (6,1.679)};
          \addplot [textColor,dashed] plot coordinates {(0,.203) (6,.203)};
          \addplot [textColor,dashed] plot coordinates {(5,0) (5,1.679)};
          \addplot [textColor,dashed] plot coordinates {(1,0) (1,.203)};
          \addplot [textColor,dashed] plot coordinates {(3.587,0) (3.597,1)};
          \addplot [penColor2,domain=(0:6)] {1};
          \addplot [very thick,penColor, smooth,domain=(0:2.5)] {sin(deg((x - 4)/2)) + 1.2};
          \addplot [very thick,penColor, smooth,domain=(4:6)] {sin(deg((x - 4)/2)) + 1.2};
          \addplot [very thick,dashed,penColor!50!background, smooth,domain=(2.5:4)] {sin(deg((x - 4)/2)) + 1.2}; 
          \addplot [color=penColor!50!background,fill=penColor!50!background,only marks,mark=*] coordinates{(3.587,1)};  %% closed hole          
          \addplot [color=penColor,fill=penColor,only marks,mark=*] coordinates{(1,.203)};  %% closed hole          
          \addplot [color=penColor,fill=penColor,only marks,mark=*] coordinates{(5,1.679)};  %% closed hole          
        \end{axis}
\end{tikzpicture}
%% \caption{A geometric interpretation of the Intermediate Value
%%   Theorem. The function $f(x)$ is continuous on the interval
%%   $[a,b]$. Since $d$ is in the interval $[f(a),f(b)]$, there exists a
%%   value $c$ in $[a,b]$ such that $f(c) = d$.}
%% \label{figure:intermediate-value}
\end{image}





\begin{example} 
Explain why the function $f(x) =x^3 + 3x^2+x-2$ has a root between $0$
and $1$.


By Theorem~\ref{theorem:limit-laws}, $\lim_{x\to a} f(x) = f(a)$, for
all real values of $a$, and hence $f$ is continuous.  Since $f(0)=-2$
and $f(1)=3$, and $0$ is between $-2$ and $3$, by the Intermediate
Value Theorem, Theorem~\ref{theorem:IVT}, there is a $c\in[0,1]$ such
that $f(c)=0$.
\end{example}

This example also points the way to a simple method for approximating
roots. 

\begin{example} 
Approximate a root of $f(x) =x^3 + 3x^2+x-2$ to one decimal place.


If we compute $f(0.1)$, $f(0.2)$, and so on, we find that $f(0.6)<0$
and $f(0.7)>0$, so by the Intermediate Value Theorem, $f$ has a root
between $0.6$ and $0.7$. Repeating the process with $f(0.61)$,
$f(0.62)$, and so on, we find that $f(0.61)<0$ and $f(0.62)>0$, so by
the Intermediate Value Theorem, Theorem~\ref{theorem:IVT}, $f(x)$ has
a root between $0.61$ and $0.62$, and the root is $0.6$ rounded to one
decimal place.
\end{example}
\end{document}
