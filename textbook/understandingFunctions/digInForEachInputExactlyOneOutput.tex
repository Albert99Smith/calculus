\documentclass{ximera}

%\usepackage{todonotes}
%\usepackage{mathtools} %% Required for wide table Curl and Greens
%\usepackage{cuted} %% Required for wide table Curl and Greens
\newcommand{\todo}{}

\usepackage{esint} % for \oiint
\ifxake%%https://math.meta.stackexchange.com/questions/9973/how-do-you-render-a-closed-surface-double-integral
\renewcommand{\oiint}{{\large\bigcirc}\kern-1.56em\iint}
\fi


\graphicspath{
  {./}
  {ximeraTutorial/}
  {basicPhilosophy/}
  {functionsOfSeveralVariables/}
  {normalVectors/}
  {lagrangeMultipliers/}
  {vectorFields/}
  {greensTheorem/}
  {shapeOfThingsToCome/}
  {dotProducts/}
  {partialDerivativesAndTheGradientVector/}
  {../productAndQuotientRules/exercises/}
  {../normalVectors/exercisesParametricPlots/}
  {../continuityOfFunctionsOfSeveralVariables/exercises/}
  {../partialDerivativesAndTheGradientVector/exercises/}
  {../directionalDerivativeAndChainRule/exercises/}
  {../commonCoordinates/exercisesCylindricalCoordinates/}
  {../commonCoordinates/exercisesSphericalCoordinates/}
  {../greensTheorem/exercisesCurlAndLineIntegrals/}
  {../greensTheorem/exercisesDivergenceAndLineIntegrals/}
  {../shapeOfThingsToCome/exercisesDivergenceTheorem/}
  {../greensTheorem/}
  {../shapeOfThingsToCome/}
  {../separableDifferentialEquations/exercises/}
  {vectorFields/}
}

\newcommand{\mooculus}{\textsf{\textbf{MOOC}\textnormal{\textsf{ULUS}}}}

\usepackage{tkz-euclide}\usepackage{tikz}
\usepackage{tikz-cd}
\usetikzlibrary{arrows}
\tikzset{>=stealth,commutative diagrams/.cd,
  arrow style=tikz,diagrams={>=stealth}} %% cool arrow head
\tikzset{shorten <>/.style={ shorten >=#1, shorten <=#1 } } %% allows shorter vectors

\usetikzlibrary{backgrounds} %% for boxes around graphs
\usetikzlibrary{shapes,positioning}  %% Clouds and stars
\usetikzlibrary{matrix} %% for matrix
\usepgfplotslibrary{polar} %% for polar plots
\usepgfplotslibrary{fillbetween} %% to shade area between curves in TikZ
\usetkzobj{all}
\usepackage[makeroom]{cancel} %% for strike outs
%\usepackage{mathtools} %% for pretty underbrace % Breaks Ximera
%\usepackage{multicol}
\usepackage{pgffor} %% required for integral for loops



%% http://tex.stackexchange.com/questions/66490/drawing-a-tikz-arc-specifying-the-center
%% Draws beach ball
\tikzset{pics/carc/.style args={#1:#2:#3}{code={\draw[pic actions] (#1:#3) arc(#1:#2:#3);}}}



\usepackage{array}
\setlength{\extrarowheight}{+.1cm}
\newdimen\digitwidth
\settowidth\digitwidth{9}
\def\divrule#1#2{
\noalign{\moveright#1\digitwidth
\vbox{\hrule width#2\digitwidth}}}





\newcommand{\RR}{\mathbb R}
\newcommand{\R}{\mathbb R}
\newcommand{\N}{\mathbb N}
\newcommand{\Z}{\mathbb Z}

\newcommand{\sagemath}{\textsf{SageMath}}


%\renewcommand{\d}{\,d\!}
\renewcommand{\d}{\mathop{}\!d}
\newcommand{\dd}[2][]{\frac{\d #1}{\d #2}}
\newcommand{\pp}[2][]{\frac{\partial #1}{\partial #2}}
\renewcommand{\l}{\ell}
\newcommand{\ddx}{\frac{d}{\d x}}

\newcommand{\zeroOverZero}{\ensuremath{\boldsymbol{\tfrac{0}{0}}}}
\newcommand{\inftyOverInfty}{\ensuremath{\boldsymbol{\tfrac{\infty}{\infty}}}}
\newcommand{\zeroOverInfty}{\ensuremath{\boldsymbol{\tfrac{0}{\infty}}}}
\newcommand{\zeroTimesInfty}{\ensuremath{\small\boldsymbol{0\cdot \infty}}}
\newcommand{\inftyMinusInfty}{\ensuremath{\small\boldsymbol{\infty - \infty}}}
\newcommand{\oneToInfty}{\ensuremath{\boldsymbol{1^\infty}}}
\newcommand{\zeroToZero}{\ensuremath{\boldsymbol{0^0}}}
\newcommand{\inftyToZero}{\ensuremath{\boldsymbol{\infty^0}}}



\newcommand{\numOverZero}{\ensuremath{\boldsymbol{\tfrac{\#}{0}}}}
\newcommand{\dfn}{\textbf}
%\newcommand{\unit}{\,\mathrm}
\newcommand{\unit}{\mathop{}\!\mathrm}
\newcommand{\eval}[1]{\bigg[ #1 \bigg]}
\newcommand{\seq}[1]{\left( #1 \right)}
\renewcommand{\epsilon}{\varepsilon}
\renewcommand{\phi}{\varphi}


\renewcommand{\iff}{\Leftrightarrow}

\DeclareMathOperator{\arccot}{arccot}
\DeclareMathOperator{\arcsec}{arcsec}
\DeclareMathOperator{\arccsc}{arccsc}
\DeclareMathOperator{\si}{Si}
\DeclareMathOperator{\scal}{scal}
\DeclareMathOperator{\sign}{sign}


%% \newcommand{\tightoverset}[2]{% for arrow vec
%%   \mathop{#2}\limits^{\vbox to -.5ex{\kern-0.75ex\hbox{$#1$}\vss}}}
\newcommand{\arrowvec}[1]{{\overset{\rightharpoonup}{#1}}}
%\renewcommand{\vec}[1]{\arrowvec{\mathbf{#1}}}
\renewcommand{\vec}[1]{{\overset{\boldsymbol{\rightharpoonup}}{\mathbf{#1}}}\hspace{0in}}

\newcommand{\point}[1]{\left(#1\right)} %this allows \vector{ to be changed to \vector{ with a quick find and replace
\newcommand{\pt}[1]{\mathbf{#1}} %this allows \vec{ to be changed to \vec{ with a quick find and replace
\newcommand{\Lim}[2]{\lim_{\point{#1} \to \point{#2}}} %Bart, I changed this to point since I want to use it.  It runs through both of the exercise and exerciseE files in limits section, which is why it was in each document to start with.

\DeclareMathOperator{\proj}{\mathbf{proj}}
\newcommand{\veci}{{\boldsymbol{\hat{\imath}}}}
\newcommand{\vecj}{{\boldsymbol{\hat{\jmath}}}}
\newcommand{\veck}{{\boldsymbol{\hat{k}}}}
\newcommand{\vecl}{\vec{\boldsymbol{\l}}}
\newcommand{\uvec}[1]{\mathbf{\hat{#1}}}
\newcommand{\utan}{\mathbf{\hat{t}}}
\newcommand{\unormal}{\mathbf{\hat{n}}}
\newcommand{\ubinormal}{\mathbf{\hat{b}}}

\newcommand{\dotp}{\bullet}
\newcommand{\cross}{\boldsymbol\times}
\newcommand{\grad}{\boldsymbol\nabla}
\newcommand{\divergence}{\grad\dotp}
\newcommand{\curl}{\grad\cross}
%\DeclareMathOperator{\divergence}{divergence}
%\DeclareMathOperator{\curl}[1]{\grad\cross #1}
\newcommand{\lto}{\mathop{\longrightarrow\,}\limits}

\renewcommand{\bar}{\overline}

\colorlet{textColor}{black}
\colorlet{background}{white}
\colorlet{penColor}{blue!50!black} % Color of a curve in a plot
\colorlet{penColor2}{red!50!black}% Color of a curve in a plot
\colorlet{penColor3}{red!50!blue} % Color of a curve in a plot
\colorlet{penColor4}{green!50!black} % Color of a curve in a plot
\colorlet{penColor5}{orange!80!black} % Color of a curve in a plot
\colorlet{penColor6}{yellow!70!black} % Color of a curve in a plot
\colorlet{fill1}{penColor!20} % Color of fill in a plot
\colorlet{fill2}{penColor2!20} % Color of fill in a plot
\colorlet{fillp}{fill1} % Color of positive area
\colorlet{filln}{penColor2!20} % Color of negative area
\colorlet{fill3}{penColor3!20} % Fill
\colorlet{fill4}{penColor4!20} % Fill
\colorlet{fill5}{penColor5!20} % Fill
\colorlet{gridColor}{gray!50} % Color of grid in a plot

\newcommand{\surfaceColor}{violet}
\newcommand{\surfaceColorTwo}{redyellow}
\newcommand{\sliceColor}{greenyellow}




\pgfmathdeclarefunction{gauss}{2}{% gives gaussian
  \pgfmathparse{1/(#2*sqrt(2*pi))*exp(-((x-#1)^2)/(2*#2^2))}%
}


%%%%%%%%%%%%%
%% Vectors
%%%%%%%%%%%%%

%% Simple horiz vectors
\renewcommand{\vector}[1]{\left\langle #1\right\rangle}


%% %% Complex Horiz Vectors with angle brackets
%% \makeatletter
%% \renewcommand{\vector}[2][ , ]{\left\langle%
%%   \def\nextitem{\def\nextitem{#1}}%
%%   \@for \el:=#2\do{\nextitem\el}\right\rangle%
%% }
%% \makeatother

%% %% Vertical Vectors
%% \def\vector#1{\begin{bmatrix}\vecListA#1,,\end{bmatrix}}
%% \def\vecListA#1,{\if,#1,\else #1\cr \expandafter \vecListA \fi}

%%%%%%%%%%%%%
%% End of vectors
%%%%%%%%%%%%%

%\newcommand{\fullwidth}{}
%\newcommand{\normalwidth}{}



%% makes a snazzy t-chart for evaluating functions
%\newenvironment{tchart}{\rowcolors{2}{}{background!90!textColor}\array}{\endarray}

%%This is to help with formatting on future title pages.
\newenvironment{sectionOutcomes}{}{}



%% Flowchart stuff
%\tikzstyle{startstop} = [rectangle, rounded corners, minimum width=3cm, minimum height=1cm,text centered, draw=black]
%\tikzstyle{question} = [rectangle, minimum width=3cm, minimum height=1cm, text centered, draw=black]
%\tikzstyle{decision} = [trapezium, trapezium left angle=70, trapezium right angle=110, minimum width=3cm, minimum height=1cm, text centered, draw=black]
%\tikzstyle{question} = [rectangle, rounded corners, minimum width=3cm, minimum height=1cm,text centered, draw=black]
%\tikzstyle{process} = [rectangle, minimum width=3cm, minimum height=1cm, text centered, draw=black]
%\tikzstyle{decision} = [trapezium, trapezium left angle=70, trapezium right angle=110, minimum width=3cm, minimum height=1cm, text centered, draw=black]


\title{For each input, exactly one output}
\begin{document}
\begin{abstract}
  We define the concept of a function.
\end{abstract}
\maketitle


Life is complex. Part of this complexity stems from the fact that
there are many relationships between seemingly unrelated events. Armed
with mathematics we seek to understand the world, and hence we need
tools for talking about these relationships.  In mathematics, we refer to these relationships as relations.  


A \textit{function} is a relation between sets of objects that can be
thought of as a ``mathematical machine.'' 
%% Insert picture 1
This means for each input,there is exactly one output. 
%% Insert picture 2

Something as simple as a dictionary could be thought of as a relation,
as it connects \textit{words} to \textit{definitions}. However, a
dictionary is not a function, as there are words with multiple
definitions. On the other hand, if each word only had a single
definition, then it would be a function.  Let's say this explicitly.

\begin{definition}\index{function}
A \dfn{function} is a relation between sets, where for each input,
there is exactly one output.
\end{definition}

Moreover, whenever we talk about functions, we should try to
explicitly state what type of things the inputs are and what type of
things the outputs are.  In calculus, functions often define a
relation from (a subset of) the real numbers to (a subset of) the real
numbers.


\begin{example}
Consider the function $f$ that maps from the real numbers to the real
numbers by taking a number and mapping it to its cube:
\begin{align*}
1 &\mapsto 1\\
-2 &\mapsto -8\\
1.5 &\mapsto 3.375
\end{align*}
and so on. This function can be described by the formula $f(x)=x^3$ or
by the plot shown in Figure~\ref{plot:fxn x^3}.
\end{example}

\begin{warning}
A function is a relation (such that for each input, there is exactly one
output) between sets and should not be confused with either its
formula or its plot.
\begin{itemize}
\item A formula merely describes the mapping using algebra.
\item A plot merely describes the mapping using pictures. 
\end{itemize}
\end{warning}

\begin{image}
\begin{tikzpicture}
	\begin{axis}[
            domain=-2:2,
            axis lines =middle, xlabel=$x$, ylabel=$y$,
            every axis y label/.style={at=(current axis.above origin),anchor=south},
            every axis x label/.style={at=(current axis.right of origin),anchor=west},
          ]
	  \addplot [very thick, penColor, smooth] {x^3};
        \end{axis}
\end{tikzpicture}
%% \caption{A plot of $f(x)=x^3$. Here we can see that for each input (a
%%   value on the $x$-axis), there is exactly one output (a value on the
%%   $y$-axis).}
%% \label{plot:fxn x^3}
\end{image}



\begin{example}
Consider the \textit{greatest integer function}, denoted by
\[
f(x) = \lfloor x \rfloor.
\]
This is the function that maps any real number $x$ to the greatest
integer less than or equal to $x$. See Figure~\ref{plot:greatest-integer fxn} for a plot of
this function. Some might be confused because here we have multiple
inputs that give the same output. However, this is not a problem. To
be a function, we merely need to check that for each input, there is exactly
one output, and this is satisfied.
\end{example}

\begin{image}
\begin{tikzpicture}
	\begin{axis}[
            domain=-2:4,
            axis lines =middle, xlabel=$x$, ylabel=$y$,
            every axis y label/.style={at=(current axis.above origin),anchor=south},
            every axis x label/.style={at=(current axis.right of origin),anchor=west},
            clip=false,
            %axis on top,
          ]
          \addplot [textColor, very thin, domain=(0:2.3)] {0}; % puts the axis back, axis on top clobbers our open holes
          \addplot [textColor, very thin] plot coordinates {(0,0) (0,2)}; % puts the axis back, axis on top clobbers our open holes
	  \addplot [very thick, penColor, domain=(-2:-1)] {-2};
          \addplot [very thick, penColor, domain=(-1:0)] {-1};
          \addplot [very thick, penColor, domain=(0:1)] {0};
          \addplot [very thick, penColor, domain=(1:2)] {1};
          \addplot [very thick, penColor, domain=(2:3)] {2};
          \addplot [very thick, penColor, domain=(3:4)] {3};
          \addplot[color=penColor,fill=penColor,only marks,mark=*] coordinates{(-2,-2)};  %% closed hole          
          \addplot[color=penColor,fill=penColor,only marks,mark=*] coordinates{(-1,-1)};  %% closed hole          
          \addplot[color=penColor,fill=penColor,only marks,mark=*] coordinates{(0,0)};  %% closed hole          
          \addplot[color=penColor,fill=penColor,only marks,mark=*] coordinates{(1,1)};  %% closed hole          
          \addplot[color=penColor,fill=penColor,only marks,mark=*] coordinates{(2,2)};  %% closed hole  
          \addplot[color=penColor,fill=penColor,only marks,mark=*] coordinates{(3,3)};  %% closed hole                  
          \addplot[color=penColor,fill=background,only marks,mark=*] coordinates{(-1,-2)};  %% open hole
          \addplot[color=penColor,fill=background,only marks,mark=*] coordinates{(0,-1)};  %% open hole
          \addplot[color=penColor,fill=background,only marks,mark=*] coordinates{(1,0)};  %% open hole
          \addplot[color=penColor,fill=background,only marks,mark=*] coordinates{(2,1)};  %% open hole
          \addplot[color=penColor,fill=background,only marks,mark=*] coordinates{(3,2)};  %% open hole
          \addplot[color=penColor,fill=background,only marks,mark=*] coordinates{(4,3)};  %% open hole
        \end{axis}
\end{tikzpicture}
%% \caption{A plot of $f(x)=\lfloor x\rfloor$. Here we can see that for each input (a
%%   value on the $x$-axis), there is exactly one output (a value on the
%%   $y$-axis).}
%% \label{plot:greatest-integer fxn}
\end{image}


Notice that both Figure 1 and Figure 2 pass the vertical line test.  A vertical line can be drawn through any part of the plot and the vertical line will only intersect the plot once.  This tells us the plot is a function.

Just to remind you, a function maps from one set to another. We call
the set a function is mapping from the \dfn{domain}\index{domain}
or \textit{source} and we call the set a function is mapping to the
\dfn{range}\index{range} or \textit{target}.  In our previous
examples the domain and range have both been the real numbers, denoted
by $\RR$. In our next examples we show that this is not always the
case.


\begin{example}
Consider the function that maps non-negative real numbers to their positive square root. This function is denoted by 
\[
f(x) = \sqrt{x}.
\]
The domain is $\{x: x\ge 0\}$ written in set notation or$[0,\infty )$ in interval notation.  In calculus, we'll most often use intercal notation.  The range is $(-\infty,\infty)$.
See Figure~\ref{plot:sqrt fxn} for a plot of $f(x) = \sqrt{x}$.
\end{example}

\begin{image}
\begin{tikzpicture}
	\begin{axis}[
            xmin=-8,xmax=8,
            ymin=-5,ymax=5,
            domain=0:8,
            axis lines =middle, xlabel=$x$, ylabel=$y$,
            every axis y label/.style={at=(current axis.above origin),anchor=south},
            every axis x label/.style={at=(current axis.right of origin),anchor=west},
          ]
	  \addplot [very thick, penColor, smooth,samples=100] {sqrt(x)};
        \end{axis}
\end{tikzpicture}
%% \caption{A plot of $f(x)=\sqrt{x}$. Here we can see that for each
%%   input (a non-negative value on the $x$-axis), there is exactly one
%%   output (a positive value on the $y$-axis).}
%% \label{plot:sqrt fxn}
\end{image}


Note: $\sqrt{x^2} = |x|$  Why?  Although $\sqrt{x^2}$ may appear to simplify to $x$, let's see what happens when we plug in values.
%%Insert picture 3
We see that $\sqrt{x^2}\ne x$ as $f(-2)=2$ and $f(-1)=1$.  Instead, we see that $\sqrt{x^2} = |x|$.  The domain of $f(x)=\sqrt{x^2}$is $(-\infty,\infty)$ and the range is $[0,\infty)$.  See figure for plot.
%%Insert picture 4


Finally, we will consider a function whose domain is all real numbers
except for a single point.

\begin{example}
Consider the function defined by 
\[
f(x) = \frac{x^2 - 3x + 2}{x-2} = \frac{(x-2)(x-1)}{(x-2)}
\]
This function may seem innocent enough; however, it is undefined at
$x=2$. Why? Recall that we cannot divide bt zero and therefore the function is undefined when the denominator, $x-2$ equals $0$.  We have $x-2 = 0$ when $x=2$.  Therefore, the domain is $(\-infty ,2)\cup (2,\infty )$  
See Figure~\ref{plot:point undfed fxn} for a plot of this function.
\end{example}

Like Devyn and Riley, you may be wondering, are $f(x) = \frac{x^2 - 3x + 2}{x-2} = \frac{(x-2)(x-1)}{(x-2)}$ and $g(x) = x-1$ the same function?  What if we compare plots?  See figures 4 and 5.  What if we compare domain?  We see a hole in $f(x)$ at $x=2$.  This is where $f(X)$ is undefined.  On the other hand, there is no hole in $g(x)$ and the domain is $(-\infty ,\infty )$.  Thus, these are not the same function.  What we can say is that $f(x)=g(x)$ or $f(x)=x-2$ when $x \ne 2$.  It is important to specify the domain when we have algeraically simplified. 

%%Insert margin picture 5 which is fgure 5 referenced above

\begin{image}
\begin{tikzpicture}
	\begin{axis}[
            domain=-2:4,
            axis lines =middle, xlabel=$x$, ylabel=$y$,
            every axis y label/.style={at=(current axis.above origin),anchor=south},
            every axis x label/.style={at=(current axis.right of origin),anchor=west},
            xtick={-2,...,4},
            ytick={-3,...,3},
          ]
	  \addplot [very thick, penColor, smooth] {x-1};
          \addplot[color=penColor,fill=background,only marks,mark=*] coordinates{(2,1)};  %% open hole
        \end{axis}
\end{tikzpicture}
%% \caption{A plot of $f(x)=\protect\frac{x^2 - 3x + 2}{x-2}$. Here we
%%   can see that for each input (any value on the $x$-axis except for
%%   $x=2$), there is exactly one output (a value on the $y$-axis).}
%% \label{plot:point undfed fxn}
\end{image}

%%BADBAD Insert picture 6


\end{document}
