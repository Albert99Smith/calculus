\section{A Note on Notation}

Mathematicians and teachers are sometimes sloppy regarding the
notational distinction between a function value and the function as a
whole, allowing $f(x)=x^2$, for example, to be taken as a statement
about the whole function.

Consider the following expressions: $f(a)$, $f(x_0)$, and $f(x)$.
Without any additional context, many mathematicians and teachers
interpret the first two as particular output values, because it is
customary to use the letter $a$ and the subscripted $x_0$ to denote
particular values, considered one at a time and conceived as ``fixed''
while reasoning through a problem.  The expression $f(x)$, on the
other hand, is ambiguous, for it sometimes denotes a particular output
value, yet other times denotes the function as a whole.

Some mathematicians occasionally rail at the use of $f(x)$ for the
function as a whole, while others are content that the meaning is
usually clear from the context.  When specifying a function, some
authors and computer algebra systems avoid the ambiguous
``specification formula'' $f(x)=x^2$ and instead use the notation $f:
x\mapsto x^2$, which can be read, ``$f$ maps $x$ to $x^2$.''

This distinction and the ``maps to'' notation are likely too subtle
when high school students are first learning function notation,
because students already have plenty of difficulty with simple uses of
the notation.  The distinction can be useful in calculus, however, and
it becomes necessary in upper-level undergraduate mathematics courses.
And it is important that teachers understand the distinction because
some of their students' difficulties will involve this issue.
