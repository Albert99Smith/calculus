\documentclass{ximera}

%\usepackage{todonotes}
%\usepackage{mathtools} %% Required for wide table Curl and Greens
%\usepackage{cuted} %% Required for wide table Curl and Greens
\newcommand{\todo}{}

\usepackage{esint} % for \oiint
\ifxake%%https://math.meta.stackexchange.com/questions/9973/how-do-you-render-a-closed-surface-double-integral
\renewcommand{\oiint}{{\large\bigcirc}\kern-1.56em\iint}
\fi


\graphicspath{
  {./}
  {ximeraTutorial/}
  {basicPhilosophy/}
  {functionsOfSeveralVariables/}
  {normalVectors/}
  {lagrangeMultipliers/}
  {vectorFields/}
  {greensTheorem/}
  {shapeOfThingsToCome/}
  {dotProducts/}
  {partialDerivativesAndTheGradientVector/}
  {../productAndQuotientRules/exercises/}
  {../normalVectors/exercisesParametricPlots/}
  {../continuityOfFunctionsOfSeveralVariables/exercises/}
  {../partialDerivativesAndTheGradientVector/exercises/}
  {../directionalDerivativeAndChainRule/exercises/}
  {../commonCoordinates/exercisesCylindricalCoordinates/}
  {../commonCoordinates/exercisesSphericalCoordinates/}
  {../greensTheorem/exercisesCurlAndLineIntegrals/}
  {../greensTheorem/exercisesDivergenceAndLineIntegrals/}
  {../shapeOfThingsToCome/exercisesDivergenceTheorem/}
  {../greensTheorem/}
  {../shapeOfThingsToCome/}
  {../separableDifferentialEquations/exercises/}
  {vectorFields/}
}

\newcommand{\mooculus}{\textsf{\textbf{MOOC}\textnormal{\textsf{ULUS}}}}

\usepackage{tkz-euclide}\usepackage{tikz}
\usepackage{tikz-cd}
\usetikzlibrary{arrows}
\tikzset{>=stealth,commutative diagrams/.cd,
  arrow style=tikz,diagrams={>=stealth}} %% cool arrow head
\tikzset{shorten <>/.style={ shorten >=#1, shorten <=#1 } } %% allows shorter vectors

\usetikzlibrary{backgrounds} %% for boxes around graphs
\usetikzlibrary{shapes,positioning}  %% Clouds and stars
\usetikzlibrary{matrix} %% for matrix
\usepgfplotslibrary{polar} %% for polar plots
\usepgfplotslibrary{fillbetween} %% to shade area between curves in TikZ
\usetkzobj{all}
\usepackage[makeroom]{cancel} %% for strike outs
%\usepackage{mathtools} %% for pretty underbrace % Breaks Ximera
%\usepackage{multicol}
\usepackage{pgffor} %% required for integral for loops



%% http://tex.stackexchange.com/questions/66490/drawing-a-tikz-arc-specifying-the-center
%% Draws beach ball
\tikzset{pics/carc/.style args={#1:#2:#3}{code={\draw[pic actions] (#1:#3) arc(#1:#2:#3);}}}



\usepackage{array}
\setlength{\extrarowheight}{+.1cm}
\newdimen\digitwidth
\settowidth\digitwidth{9}
\def\divrule#1#2{
\noalign{\moveright#1\digitwidth
\vbox{\hrule width#2\digitwidth}}}





\newcommand{\RR}{\mathbb R}
\newcommand{\R}{\mathbb R}
\newcommand{\N}{\mathbb N}
\newcommand{\Z}{\mathbb Z}

\newcommand{\sagemath}{\textsf{SageMath}}


%\renewcommand{\d}{\,d\!}
\renewcommand{\d}{\mathop{}\!d}
\newcommand{\dd}[2][]{\frac{\d #1}{\d #2}}
\newcommand{\pp}[2][]{\frac{\partial #1}{\partial #2}}
\renewcommand{\l}{\ell}
\newcommand{\ddx}{\frac{d}{\d x}}

\newcommand{\zeroOverZero}{\ensuremath{\boldsymbol{\tfrac{0}{0}}}}
\newcommand{\inftyOverInfty}{\ensuremath{\boldsymbol{\tfrac{\infty}{\infty}}}}
\newcommand{\zeroOverInfty}{\ensuremath{\boldsymbol{\tfrac{0}{\infty}}}}
\newcommand{\zeroTimesInfty}{\ensuremath{\small\boldsymbol{0\cdot \infty}}}
\newcommand{\inftyMinusInfty}{\ensuremath{\small\boldsymbol{\infty - \infty}}}
\newcommand{\oneToInfty}{\ensuremath{\boldsymbol{1^\infty}}}
\newcommand{\zeroToZero}{\ensuremath{\boldsymbol{0^0}}}
\newcommand{\inftyToZero}{\ensuremath{\boldsymbol{\infty^0}}}



\newcommand{\numOverZero}{\ensuremath{\boldsymbol{\tfrac{\#}{0}}}}
\newcommand{\dfn}{\textbf}
%\newcommand{\unit}{\,\mathrm}
\newcommand{\unit}{\mathop{}\!\mathrm}
\newcommand{\eval}[1]{\bigg[ #1 \bigg]}
\newcommand{\seq}[1]{\left( #1 \right)}
\renewcommand{\epsilon}{\varepsilon}
\renewcommand{\phi}{\varphi}


\renewcommand{\iff}{\Leftrightarrow}

\DeclareMathOperator{\arccot}{arccot}
\DeclareMathOperator{\arcsec}{arcsec}
\DeclareMathOperator{\arccsc}{arccsc}
\DeclareMathOperator{\si}{Si}
\DeclareMathOperator{\scal}{scal}
\DeclareMathOperator{\sign}{sign}


%% \newcommand{\tightoverset}[2]{% for arrow vec
%%   \mathop{#2}\limits^{\vbox to -.5ex{\kern-0.75ex\hbox{$#1$}\vss}}}
\newcommand{\arrowvec}[1]{{\overset{\rightharpoonup}{#1}}}
%\renewcommand{\vec}[1]{\arrowvec{\mathbf{#1}}}
\renewcommand{\vec}[1]{{\overset{\boldsymbol{\rightharpoonup}}{\mathbf{#1}}}\hspace{0in}}

\newcommand{\point}[1]{\left(#1\right)} %this allows \vector{ to be changed to \vector{ with a quick find and replace
\newcommand{\pt}[1]{\mathbf{#1}} %this allows \vec{ to be changed to \vec{ with a quick find and replace
\newcommand{\Lim}[2]{\lim_{\point{#1} \to \point{#2}}} %Bart, I changed this to point since I want to use it.  It runs through both of the exercise and exerciseE files in limits section, which is why it was in each document to start with.

\DeclareMathOperator{\proj}{\mathbf{proj}}
\newcommand{\veci}{{\boldsymbol{\hat{\imath}}}}
\newcommand{\vecj}{{\boldsymbol{\hat{\jmath}}}}
\newcommand{\veck}{{\boldsymbol{\hat{k}}}}
\newcommand{\vecl}{\vec{\boldsymbol{\l}}}
\newcommand{\uvec}[1]{\mathbf{\hat{#1}}}
\newcommand{\utan}{\mathbf{\hat{t}}}
\newcommand{\unormal}{\mathbf{\hat{n}}}
\newcommand{\ubinormal}{\mathbf{\hat{b}}}

\newcommand{\dotp}{\bullet}
\newcommand{\cross}{\boldsymbol\times}
\newcommand{\grad}{\boldsymbol\nabla}
\newcommand{\divergence}{\grad\dotp}
\newcommand{\curl}{\grad\cross}
%\DeclareMathOperator{\divergence}{divergence}
%\DeclareMathOperator{\curl}[1]{\grad\cross #1}
\newcommand{\lto}{\mathop{\longrightarrow\,}\limits}

\renewcommand{\bar}{\overline}

\colorlet{textColor}{black}
\colorlet{background}{white}
\colorlet{penColor}{blue!50!black} % Color of a curve in a plot
\colorlet{penColor2}{red!50!black}% Color of a curve in a plot
\colorlet{penColor3}{red!50!blue} % Color of a curve in a plot
\colorlet{penColor4}{green!50!black} % Color of a curve in a plot
\colorlet{penColor5}{orange!80!black} % Color of a curve in a plot
\colorlet{penColor6}{yellow!70!black} % Color of a curve in a plot
\colorlet{fill1}{penColor!20} % Color of fill in a plot
\colorlet{fill2}{penColor2!20} % Color of fill in a plot
\colorlet{fillp}{fill1} % Color of positive area
\colorlet{filln}{penColor2!20} % Color of negative area
\colorlet{fill3}{penColor3!20} % Fill
\colorlet{fill4}{penColor4!20} % Fill
\colorlet{fill5}{penColor5!20} % Fill
\colorlet{gridColor}{gray!50} % Color of grid in a plot

\newcommand{\surfaceColor}{violet}
\newcommand{\surfaceColorTwo}{redyellow}
\newcommand{\sliceColor}{greenyellow}




\pgfmathdeclarefunction{gauss}{2}{% gives gaussian
  \pgfmathparse{1/(#2*sqrt(2*pi))*exp(-((x-#1)^2)/(2*#2^2))}%
}


%%%%%%%%%%%%%
%% Vectors
%%%%%%%%%%%%%

%% Simple horiz vectors
\renewcommand{\vector}[1]{\left\langle #1\right\rangle}


%% %% Complex Horiz Vectors with angle brackets
%% \makeatletter
%% \renewcommand{\vector}[2][ , ]{\left\langle%
%%   \def\nextitem{\def\nextitem{#1}}%
%%   \@for \el:=#2\do{\nextitem\el}\right\rangle%
%% }
%% \makeatother

%% %% Vertical Vectors
%% \def\vector#1{\begin{bmatrix}\vecListA#1,,\end{bmatrix}}
%% \def\vecListA#1,{\if,#1,\else #1\cr \expandafter \vecListA \fi}

%%%%%%%%%%%%%
%% End of vectors
%%%%%%%%%%%%%

%\newcommand{\fullwidth}{}
%\newcommand{\normalwidth}{}



%% makes a snazzy t-chart for evaluating functions
%\newenvironment{tchart}{\rowcolors{2}{}{background!90!textColor}\array}{\endarray}

%%This is to help with formatting on future title pages.
\newenvironment{sectionOutcomes}{}{}



%% Flowchart stuff
%\tikzstyle{startstop} = [rectangle, rounded corners, minimum width=3cm, minimum height=1cm,text centered, draw=black]
%\tikzstyle{question} = [rectangle, minimum width=3cm, minimum height=1cm, text centered, draw=black]
%\tikzstyle{decision} = [trapezium, trapezium left angle=70, trapezium right angle=110, minimum width=3cm, minimum height=1cm, text centered, draw=black]
%\tikzstyle{question} = [rectangle, rounded corners, minimum width=3cm, minimum height=1cm,text centered, draw=black]
%\tikzstyle{process} = [rectangle, minimum width=3cm, minimum height=1cm, text centered, draw=black]
%\tikzstyle{decision} = [trapezium, trapezium left angle=70, trapezium right angle=110, minimum width=3cm, minimum height=1cm, text centered, draw=black]


\title{Inverses of functions}

\outcome{Find the domain and range of a function.}
\outcome{Determine if a function is one-to-one.}
\outcome{Perform basic operations and compositions on functions.}

\begin{document}
\begin{abstract}
  Here we ``undo'' functions.
\end{abstract}
\maketitle


If a function maps every ``input'' to exactly one ``output,'' an
inverse of that function maps every ``output'' to exactly one
``input.''  We need a more formal definition to actually say anything
with rigor.

\begin{definition}
  Let $f$ be a function with domain $A$ and range $B$:
  \begin{image}
    \begin{tikzpicture}
      \node[star,star points=7,star point ratio=2.5,draw] at (0,0) {$A$};
      \node[cloud, draw,cloud puffs=10,cloud puff arc=120, aspect=2, inner ysep=1em] at (5,0) {$B$};
      \node at (2.25,.3) {$f$};
      \draw[->] (1.5,0) to (3,0);
    \end{tikzpicture}
  \end{image}
  Let $g$ be a function with domain $B$ and range $A$:
  \begin{image}
    \begin{tikzpicture}
      \node[cloud, draw,cloud puffs=10,cloud puff arc=120, aspect=2, inner ysep=1em] at (-.5,0) {$B$};
      \node[star,star points=7,star point ratio=2.5,draw] at (4.5,0) {$A$};
      \node at (2.25,.3) {$g$};
      \draw[->] (1.5,0) to (3,0);
    \end{tikzpicture}
  \end{image}
  We say that $f$ and $g$ are \dfn{inverses} of each other if $f(g(b))
  = b$ for all $b$ in $B$, and also $g(f(a)) = a$ for all $a$ in $A$.
  Sometimes we write $g = f^{-1}$ in this case.
  \begin{image}
    \begin{tikzpicture}
      \node[cloud, draw,cloud puffs=10,cloud puff arc=120, aspect=2, inner ysep=1em] at (-.5,0) {$B$};
      \node[cloud, draw,cloud puffs=10,cloud puff arc=120, aspect=2, inner ysep=1em] at (5,0) {$B$};
      \node at (2.25,.3) {$f(f^{-1})$};
      \draw[->] (1.5,0) to (3,0);
    \end{tikzpicture}
  \end{image}
  and
  \begin{image}
    \begin{tikzpicture}
      \node[star,star points=7,star point ratio=2.5,draw] at (0,0) {$A$};
      \node[star,star points=7,star point ratio=2.5,draw] at (4.5,0) {$A$};
      \node at (2.25,.3) {$f^{-1}(f)$};
      \draw[->] (1.5,0) to (3,0);
    \end{tikzpicture}
  \end{image}
  So we could rephrase the conditions as
  \[
  f(f^{-1}(x)) = x\qquad\text{and}\qquad f^{-1}(f(x)) = x.
  \]
  \end{definition}
These two simple equations are somewhat more subtle than they
initially appear.

\begin{question}
  Let $f$ be a function.  If the point $(1,9)$ is on the graph of $f$,
  what point must be the the graph of $f^{-1}$?	
  \[
  \left( \answer[given]{9}, \answer[given]{1} \right)
  \]
  \begin{feedback}
    Since $f(1) = 9$, we must have $f^{-1}(f(1)) = 1$, so $f^{-1}(9) =
    1$.  Thus $(9,1)$ is on the graph of $f^{-1}$.  This is a general
    rule.  If $(a,b)$ is on the graph of $f$, and $(b,a)$ will be the
    graph of $f^{-1}$.
  \end{feedback}
\end{question}

\begin{warning}
  Keep a watchful eye:
  \begin{align*}
    f^{-1}(x) &= \text{the inverse function of $f(x)$.}\\
    f(x)^{-1} &= \text{$\frac{1}{f(x)}$.}
  \end{align*}
\end{warning}
\begin{question}
  Which of the following represents the inverse of the function
  $\sin(\theta)$ on the interval $[-\pi/2,\pi/2]$
  \begin{multipleChoice}
    \choice[correct]{$\sin^{-1}(\theta)$}
    \choice{$\sin(\theta)^{-1}$}
  \end{multipleChoice}
\end{question}


While this might sound somewhat esoteric, let's see
if we can ground this in some real-life contexts.

\begin{example}
  The the following function
  \[
  f(t) = \left(\frac{9}{5}\right) t + 32
  \]	
  is the function which takes Celsius measurements (its domain is
  $-\infty < t < \infty$) and converts them to Fahrenheit
  measurements of temperature.  What does the inverse of this function
  tell you? What is the inverse of this function?

  \begin{explanation}
    If $f$ converts Celsius measurements to Fahrenheit measurements of
    temperature, then $f^{-1}$ converts Fahrenheit measurements to
    Celsius measurements of temperature. First note that 
    \[
    f(f^{-1}(t)) = t \qquad \text{by the definition of inverse
      functions.}
    \]
    Now write out the left-hand side of the equation
    \[
    f(f^{-1}(t)) = \left(\frac{9}{5}\right) f^{-1}(t)+32\qquad\text{by the rule for $f$}
    \]
    and solve for $f^{-1}(t)$.
    \begin{align*}
      \left(\frac{9}{5}\right) f^{-1}(t)+32 &= t &&\text{by the rule for $f$}\\
      \left(\frac{9}{5}\right) f^{-1}(t)&= t -32\\
      f^{-1}(t) &= \left(\frac{5}{9}\right)(t - 32).
    \end{align*}
    So $f^{-1}(t) = \left(\frac{5}{9}\right)(t - 32)$ is the inverse
    function of $f$, which converts a Fahrenheit measurement back into
    a Celsius measurement and has a domain of $-\infty < t <\infty$.
    Finally note that
    \[
    f(f^{-1}(t)) = t,
    \]
    since we solved for $f^{-1}$ to ensure this is so, and that
    \begin{align*}
    f^{-1}(f(t)) &=\left(\frac{5}{9}\right)(f(t) - 32)\\
    &= \left(\frac{5}{9}\right)(f(t) - 32)\\
    \end{align*}
  \end{explanation}
\end{example}






\begin{question}
  Let $f$ be a function, and imagine that the points $(2,3)$ and
  $(7,3)$ are both on its graph.  Could $f$ have an inverse function?
  \begin{multipleChoice}
    \choice{Yes}
    \choice[correct]{No}
  \end{multipleChoice}
  \begin{feedback}
    The function $f$ could \textbf{not} have an inverse function.
    Imagine that it did.  Then $f^{-1}(f(2)) = 2$ and $f^{-1}(f(7)) =
    7$.  Then we have both $f^{-1}(3) = 2$ and $f^{-1}(3) = 7$.  Since
    a \textbf{function} cannot send the same input to two different
    outputs, $f$ must not have an inverse function.
  \end{feedback}
\end{question}

The last question highlights something interesting about inverse
functions.  If two different inputs into a function have the same
output, there is no hope of that function having an inverse, since a
function can only have one output.  This leads us to the following
definition:

\begin{definition}
A function is \dfn{one-to-one} if for every value in the range,
there is exactly one value in the domain.
\end{definition}


\begin{question}
Which of the following functions are one to one?  Select all that
apply.
\begin{selectAll}
\choice[correct]{$f(x) = x$}
\choice{$f(x) = x^2$}
\choice{$f(x) = x^3 - 4x$}
\choice[correct]{$f(x) = x^3+4$}
\end{selectAll}
\end{question}

You may recall that a plot gives $y$ as a function of $x$ if every
vertical line crosses the plot at most once, this is commonly known as
the \dfn{vertical line test}. At this point the function is one-to-one
if every horizontal line crosses the plot at most once, which is
commonly known as the \dfn{horizontal line test}. Below we give a plot
of $f(x)=-5x^2+30x+60$. While this plot passes the vertical line test,
and hence represents $y$ as a function of $x$, it does not pass the
horizontal line test, so the function is not one-to-one.
\begin{image}
\begin{tikzpicture}
	\begin{axis}[
            clip=false, domain=0:7.58, axis lines =middle, xlabel=$t$,
            ylabel=$h$, every axis y label/.style={at=(current
              axis.above origin),anchor=south}, every axis x
            label/.style={at=(current axis.right of
              origin),anchor=west}, ] \addplot [very thick, penColor,
            smooth] {-5*x^2 +30*x+60}; \addplot [very thick,
            penColor2] {80}; \addplot [very thick, penColor4] plot
          coordinates {(5,0) (5,110)};
        \end{axis}
\end{tikzpicture}
\end{image}


We can only find an inverse to a function when it is one-to-one,
otherwise we must restrict the domain. Let's see what this means in
our next example.

\begin{example}
Consider the function
\[
f(x) = x^2.
\]
Does $f$ have an inverse? If so what is it? If not, attempt to
restrict the domain of $f$ and find an inverse on the restricted
domain.
\end{example}


\begin{explanation}
In this case $f$ is not one-to-one. However, it is one-to-one on
the interval $[0,\infty)$. Hence we can find an inverse of $f(x)=x^2$
  on this interval, and it is our familiar function $\sqrt{x}$.  
\begin{image}
\begin{tikzpicture}
	\begin{axis}[
            domain=-2:2,
            xmin=-2, xmax=2,
            ymin=-2, ymax=2,
            axis lines =middle, xlabel=$x$, ylabel=$y$,
            every axis y label/.style={at=(current axis.above origin),anchor=south},
            every axis x label/.style={at=(current axis.right of origin),anchor=west},
          ]
	  \addplot [very thick, penColor, smooth] {x^2};
          \addplot [very thick, penColor2, smooth, samples=100,domain=0:2] {sqrt(x)};
          \addplot [dashed, textColor] {x};
          \node at (axis cs:-1.2,.55) [penColor,anchor=west] {$f(x)$};
          \node at (axis cs:1.4,1) [penColor2, anchor=west] {$f^{-1}(x)$};
        \end{axis}
\end{tikzpicture}
%% \caption{A plot of $f(x)=x^2$ and $f^{-1}(x) = \sqrt{x}$. While
%%   $f(x)=x^2$ is not one-to-one on $\RR$, it is one-to-one on
%%   $[0,\infty)$.}
%% \label{plot:fxn and inverse x^2}
\end{image}
\end{explanation}




Let's look at several examples.



\begin{example}
Consider the function
\[
f(x) = x^3.
\]
Does $f(x)$ have an inverse? If so what is it? If not, attempt to
restrict the domain of $f(x)$ and find an inverse on the restricted
domain.
\begin{explanation}
In this case $f(x)$ is one-to-one and $f^{-1}(x) = \sqrt[3]{x}$. For
your viewing pleasure we give a graph of $y=f(x)=x^3$ and
$y=f^{-1}(x)= \sqrt[3]{x}$. Note, the graph of $f^{-1}$ is the image
of $f$ after being flipped over the line $y=x$.
\begin{image}
\begin{tikzpicture}
	\begin{axis}[
            domain=-2:2,
            xmin=-2, xmax=2,
            ymin=-2, ymax=2,
            axis lines =middle, xlabel=$x$, ylabel=$y$,
            every axis y label/.style={at=(current axis.above origin),anchor=south},
            every axis x label/.style={at=(current axis.right of origin),anchor=west},
          ]
	  \addplot [very thick, penColor, smooth] {x^3};
          \addplot [very thick, penColor2, smooth, samples=100,domain=.01:2] {x^(1/3)};
          \addplot [very thick, penColor2, smooth, samples=100,domain=-2:-.01] {-abs(x)^(1/3)};
          \addplot [very thick, penColor2] plot coordinates {(.01,.215) (-.01,-.215)};
          \addplot [dashed, textColor] {x};
          \node at (axis cs:-1.2,-.42) [penColor,anchor=west] {$f(x)$};
          \node at (axis cs:1.2,.9) [penColor2, anchor=west] {$f^{-1}(x)$};
        \end{axis}
\end{tikzpicture}
%% \caption{A plot of $f(x)=x^3$ and $f^{-1}(x) = \sqrt[3]{x}$. Note
%%   $f^{-1}(x)$ is the image of $f(x)$ after being flipped over the line
%%   $y=x$.}
%% \label{plot:fxn and inverse x^3}
\end{image}
\end{explanation}
\end{example}


\end{document}

