\documentclass{ximera}

%\usepackage{todonotes}
%\usepackage{mathtools} %% Required for wide table Curl and Greens
%\usepackage{cuted} %% Required for wide table Curl and Greens
\newcommand{\todo}{}

\usepackage{esint} % for \oiint
\ifxake%%https://math.meta.stackexchange.com/questions/9973/how-do-you-render-a-closed-surface-double-integral
\renewcommand{\oiint}{{\large\bigcirc}\kern-1.56em\iint}
\fi


\graphicspath{
  {./}
  {ximeraTutorial/}
  {basicPhilosophy/}
  {functionsOfSeveralVariables/}
  {normalVectors/}
  {lagrangeMultipliers/}
  {vectorFields/}
  {greensTheorem/}
  {shapeOfThingsToCome/}
  {dotProducts/}
  {partialDerivativesAndTheGradientVector/}
  {../productAndQuotientRules/exercises/}
  {../normalVectors/exercisesParametricPlots/}
  {../continuityOfFunctionsOfSeveralVariables/exercises/}
  {../partialDerivativesAndTheGradientVector/exercises/}
  {../directionalDerivativeAndChainRule/exercises/}
  {../commonCoordinates/exercisesCylindricalCoordinates/}
  {../commonCoordinates/exercisesSphericalCoordinates/}
  {../greensTheorem/exercisesCurlAndLineIntegrals/}
  {../greensTheorem/exercisesDivergenceAndLineIntegrals/}
  {../shapeOfThingsToCome/exercisesDivergenceTheorem/}
  {../greensTheorem/}
  {../shapeOfThingsToCome/}
  {../separableDifferentialEquations/exercises/}
  {vectorFields/}
}

\newcommand{\mooculus}{\textsf{\textbf{MOOC}\textnormal{\textsf{ULUS}}}}

\usepackage{tkz-euclide}\usepackage{tikz}
\usepackage{tikz-cd}
\usetikzlibrary{arrows}
\tikzset{>=stealth,commutative diagrams/.cd,
  arrow style=tikz,diagrams={>=stealth}} %% cool arrow head
\tikzset{shorten <>/.style={ shorten >=#1, shorten <=#1 } } %% allows shorter vectors

\usetikzlibrary{backgrounds} %% for boxes around graphs
\usetikzlibrary{shapes,positioning}  %% Clouds and stars
\usetikzlibrary{matrix} %% for matrix
\usepgfplotslibrary{polar} %% for polar plots
\usepgfplotslibrary{fillbetween} %% to shade area between curves in TikZ
\usetkzobj{all}
\usepackage[makeroom]{cancel} %% for strike outs
%\usepackage{mathtools} %% for pretty underbrace % Breaks Ximera
%\usepackage{multicol}
\usepackage{pgffor} %% required for integral for loops



%% http://tex.stackexchange.com/questions/66490/drawing-a-tikz-arc-specifying-the-center
%% Draws beach ball
\tikzset{pics/carc/.style args={#1:#2:#3}{code={\draw[pic actions] (#1:#3) arc(#1:#2:#3);}}}



\usepackage{array}
\setlength{\extrarowheight}{+.1cm}
\newdimen\digitwidth
\settowidth\digitwidth{9}
\def\divrule#1#2{
\noalign{\moveright#1\digitwidth
\vbox{\hrule width#2\digitwidth}}}





\newcommand{\RR}{\mathbb R}
\newcommand{\R}{\mathbb R}
\newcommand{\N}{\mathbb N}
\newcommand{\Z}{\mathbb Z}

\newcommand{\sagemath}{\textsf{SageMath}}


%\renewcommand{\d}{\,d\!}
\renewcommand{\d}{\mathop{}\!d}
\newcommand{\dd}[2][]{\frac{\d #1}{\d #2}}
\newcommand{\pp}[2][]{\frac{\partial #1}{\partial #2}}
\renewcommand{\l}{\ell}
\newcommand{\ddx}{\frac{d}{\d x}}

\newcommand{\zeroOverZero}{\ensuremath{\boldsymbol{\tfrac{0}{0}}}}
\newcommand{\inftyOverInfty}{\ensuremath{\boldsymbol{\tfrac{\infty}{\infty}}}}
\newcommand{\zeroOverInfty}{\ensuremath{\boldsymbol{\tfrac{0}{\infty}}}}
\newcommand{\zeroTimesInfty}{\ensuremath{\small\boldsymbol{0\cdot \infty}}}
\newcommand{\inftyMinusInfty}{\ensuremath{\small\boldsymbol{\infty - \infty}}}
\newcommand{\oneToInfty}{\ensuremath{\boldsymbol{1^\infty}}}
\newcommand{\zeroToZero}{\ensuremath{\boldsymbol{0^0}}}
\newcommand{\inftyToZero}{\ensuremath{\boldsymbol{\infty^0}}}



\newcommand{\numOverZero}{\ensuremath{\boldsymbol{\tfrac{\#}{0}}}}
\newcommand{\dfn}{\textbf}
%\newcommand{\unit}{\,\mathrm}
\newcommand{\unit}{\mathop{}\!\mathrm}
\newcommand{\eval}[1]{\bigg[ #1 \bigg]}
\newcommand{\seq}[1]{\left( #1 \right)}
\renewcommand{\epsilon}{\varepsilon}
\renewcommand{\phi}{\varphi}


\renewcommand{\iff}{\Leftrightarrow}

\DeclareMathOperator{\arccot}{arccot}
\DeclareMathOperator{\arcsec}{arcsec}
\DeclareMathOperator{\arccsc}{arccsc}
\DeclareMathOperator{\si}{Si}
\DeclareMathOperator{\scal}{scal}
\DeclareMathOperator{\sign}{sign}


%% \newcommand{\tightoverset}[2]{% for arrow vec
%%   \mathop{#2}\limits^{\vbox to -.5ex{\kern-0.75ex\hbox{$#1$}\vss}}}
\newcommand{\arrowvec}[1]{{\overset{\rightharpoonup}{#1}}}
%\renewcommand{\vec}[1]{\arrowvec{\mathbf{#1}}}
\renewcommand{\vec}[1]{{\overset{\boldsymbol{\rightharpoonup}}{\mathbf{#1}}}\hspace{0in}}

\newcommand{\point}[1]{\left(#1\right)} %this allows \vector{ to be changed to \vector{ with a quick find and replace
\newcommand{\pt}[1]{\mathbf{#1}} %this allows \vec{ to be changed to \vec{ with a quick find and replace
\newcommand{\Lim}[2]{\lim_{\point{#1} \to \point{#2}}} %Bart, I changed this to point since I want to use it.  It runs through both of the exercise and exerciseE files in limits section, which is why it was in each document to start with.

\DeclareMathOperator{\proj}{\mathbf{proj}}
\newcommand{\veci}{{\boldsymbol{\hat{\imath}}}}
\newcommand{\vecj}{{\boldsymbol{\hat{\jmath}}}}
\newcommand{\veck}{{\boldsymbol{\hat{k}}}}
\newcommand{\vecl}{\vec{\boldsymbol{\l}}}
\newcommand{\uvec}[1]{\mathbf{\hat{#1}}}
\newcommand{\utan}{\mathbf{\hat{t}}}
\newcommand{\unormal}{\mathbf{\hat{n}}}
\newcommand{\ubinormal}{\mathbf{\hat{b}}}

\newcommand{\dotp}{\bullet}
\newcommand{\cross}{\boldsymbol\times}
\newcommand{\grad}{\boldsymbol\nabla}
\newcommand{\divergence}{\grad\dotp}
\newcommand{\curl}{\grad\cross}
%\DeclareMathOperator{\divergence}{divergence}
%\DeclareMathOperator{\curl}[1]{\grad\cross #1}
\newcommand{\lto}{\mathop{\longrightarrow\,}\limits}

\renewcommand{\bar}{\overline}

\colorlet{textColor}{black}
\colorlet{background}{white}
\colorlet{penColor}{blue!50!black} % Color of a curve in a plot
\colorlet{penColor2}{red!50!black}% Color of a curve in a plot
\colorlet{penColor3}{red!50!blue} % Color of a curve in a plot
\colorlet{penColor4}{green!50!black} % Color of a curve in a plot
\colorlet{penColor5}{orange!80!black} % Color of a curve in a plot
\colorlet{penColor6}{yellow!70!black} % Color of a curve in a plot
\colorlet{fill1}{penColor!20} % Color of fill in a plot
\colorlet{fill2}{penColor2!20} % Color of fill in a plot
\colorlet{fillp}{fill1} % Color of positive area
\colorlet{filln}{penColor2!20} % Color of negative area
\colorlet{fill3}{penColor3!20} % Fill
\colorlet{fill4}{penColor4!20} % Fill
\colorlet{fill5}{penColor5!20} % Fill
\colorlet{gridColor}{gray!50} % Color of grid in a plot

\newcommand{\surfaceColor}{violet}
\newcommand{\surfaceColorTwo}{redyellow}
\newcommand{\sliceColor}{greenyellow}




\pgfmathdeclarefunction{gauss}{2}{% gives gaussian
  \pgfmathparse{1/(#2*sqrt(2*pi))*exp(-((x-#1)^2)/(2*#2^2))}%
}


%%%%%%%%%%%%%
%% Vectors
%%%%%%%%%%%%%

%% Simple horiz vectors
\renewcommand{\vector}[1]{\left\langle #1\right\rangle}


%% %% Complex Horiz Vectors with angle brackets
%% \makeatletter
%% \renewcommand{\vector}[2][ , ]{\left\langle%
%%   \def\nextitem{\def\nextitem{#1}}%
%%   \@for \el:=#2\do{\nextitem\el}\right\rangle%
%% }
%% \makeatother

%% %% Vertical Vectors
%% \def\vector#1{\begin{bmatrix}\vecListA#1,,\end{bmatrix}}
%% \def\vecListA#1,{\if,#1,\else #1\cr \expandafter \vecListA \fi}

%%%%%%%%%%%%%
%% End of vectors
%%%%%%%%%%%%%

%\newcommand{\fullwidth}{}
%\newcommand{\normalwidth}{}



%% makes a snazzy t-chart for evaluating functions
%\newenvironment{tchart}{\rowcolors{2}{}{background!90!textColor}\array}{\endarray}

%%This is to help with formatting on future title pages.
\newenvironment{sectionOutcomes}{}{}



%% Flowchart stuff
%\tikzstyle{startstop} = [rectangle, rounded corners, minimum width=3cm, minimum height=1cm,text centered, draw=black]
%\tikzstyle{question} = [rectangle, minimum width=3cm, minimum height=1cm, text centered, draw=black]
%\tikzstyle{decision} = [trapezium, trapezium left angle=70, trapezium right angle=110, minimum width=3cm, minimum height=1cm, text centered, draw=black]
%\tikzstyle{question} = [rectangle, rounded corners, minimum width=3cm, minimum height=1cm,text centered, draw=black]
%\tikzstyle{process} = [rectangle, minimum width=3cm, minimum height=1cm, text centered, draw=black]
%\tikzstyle{decision} = [trapezium, trapezium left angle=70, trapezium right angle=110, minimum width=3cm, minimum height=1cm, text centered, draw=black]


\title[Dig-In:]{Horizontal asymptotes}

\outcome{Find horizontal asymptotes using limits.}
\outcome{Recognize that a curve can cross a horizontal asymptote.}
\outcome{Calculate the limit as $x$ approaches $\pm \infty$ of common functions algebraically.}
\outcome{Find the limit as $x$ approaches $\pm \infty$ from a graph.}
\outcome{Define a horizontal asymptote.}
\outcome{Compute limits at infinity of famous functions.}
\outcome{Identify horizontal asymptotes by looking at a graph.}

\begin{document}
\begin{abstract}
We explore functions that behave like horizontal lines as the input
grows without bound.
\end{abstract}
\maketitle



Consider the function:
\[
f(x) = \frac{6x-9}{x-1}
\]
\begin{image}
\begin{tikzpicture}
	\begin{axis}[
            domain=1:4,
            ymax=20,
            ymin=-10,
            samples=100,
            axis lines =middle, xlabel=$x$, ylabel=$y$,
            every axis y label/.style={at=(current axis.above origin),anchor=south},
            every axis x label/.style={at=(current axis.right of origin),anchor=west}
          ]
	  \addplot [very thick, penColor, smooth, domain=(0:.9)] {(6*x-9)/(x-1)};
          \addplot [very thick, penColor, smooth, domain=(1.1:3)] {(6*x-9)/(x-1)};
          \addplot [textColor, dashed] plot coordinates {(1,-10) (1,20)};
        \end{axis}
\end{tikzpicture}
%% \caption{A plot of $f(x)=\protect\frac{6x-9}{x-1}$.}
%% \label{plot:(6x-9)/(x-1)}
\end{image}
As $x$ approaches infinity, it seems like $f(x)$ approaches a specific
value. Such a limit is called a \textit{limit at infinity}.

\begin{definition}\label{def:limitAtInfty}\index{limit!at infinity}
If $f(x)$ becomes arbitrarily close to a specific value $L$ by making
$x$ sufficiently large, we write
\[
\lim_{x\to \infty} f(x) = L
\]
and we say, the \dfn{limit at infinity} of $f(x)$ is $L$.  

If $f(x)$ becomes arbitrarily close to a specific value $L$ by making
$x$ sufficiently large and negative, we write
\[
\lim_{x\to -\infty} f(x) = L
\]
and we say, the \dfn{limit at negative infinity} of $f(x)$ is $L$.  
\end{definition}

\begin{example} Compute
\[
\lim_{x\to\infty} \frac{6x-9}{x-1}.
\]
\begin{explanation}
Write
\begin{align*}
\lim_{x\to\infty}\frac{6x-9}{x-1} &= \lim_{x\to\infty}\frac{6x-9}{x-1}\cdot \frac{1/x}{1/x}\\
&=\lim_{x\to\infty}\frac{\frac{6x}{x} - \frac{9}{x}}{\frac{x}{x} - \frac{1}{x}}\\
&= \lim_{x\to\infty} \frac{6}{1}\\
&= 6.
\end{align*}
\end{explanation}
\end{example}

Sometimes one must be careful, consider this example.

\begin{example}
Compute
\[
\lim_{x\to -\infty} \frac{x+1}{\sqrt{x^2}}
\]
\begin{explanation}
In this case we multiply the numerator and denominator by $-1/x$,
which is a positive number as since $x\to -\infty$, $x$ is a negative
number.
\begin{align*}
\lim_{x\to -\infty} \frac{x+1}{\sqrt{x^2}} &= \lim_{x\to -\infty} \frac{x+1}{\sqrt{x^2}} \cdot \frac{-1/x}{-1/x}\\
&= \lim_{x\to -\infty} \frac{-1-1/x}{\sqrt{x^2/x^2}}\\
&= -1.
\end{align*}
\end{explanation}
\end{example}


Here is a somewhat different example of a limit at infinity.

\begin{example}
Compute
\[
\lim_{x\to \infty} \left(\frac{\sin(7x)}{x}+4\right).
\]

\begin{image}
\begin{tikzpicture}
	\begin{axis}[
            domain=2:20,
            ymax=5,
            ymin=3,
            samples=100,
            axis lines =middle, xlabel=$x$, ylabel=$y$,
            every axis y label/.style={at=(current axis.above origin),anchor=south},
            every axis x label/.style={at=(current axis.right of origin),anchor=west}
          ]
	  \addplot [very thick, penColor, smooth] {(1/x) * sin(deg(7*x))+4};
        \end{axis}
\end{tikzpicture}
%% \caption{A plot of $f(x)=\frac{\sin(7x)}{x}+4$.}
%% \label{plot:sin7x/x+4}
\end{image}

\begin{explanation}
We can bound our function
\[
-1/x + 4 \le \frac{\sin(7x)}{x}+4 \le 1/x + 4.
\]
Since 
\[
\lim_{x\to \infty} \left(-1/x + 4\right) = 4 = \lim_{x\to \infty}1/x + 4
\] 
we conclude by the Squeeze Theorem, Theorem~\ref{theorem:squeeze},
$\lim_{x\to\infty}\frac{\sin(7x)}{x}+4 = 4$.
\end{explanation}
\end{example}






\begin{definition}\label{def:horiz asymptote}\index{asymptote!horizontal}\index{horizontal asymptote}
If  
\[
\lim_{x\to \infty} f(x) = L \qquad\text{or}\qquad \lim_{x\to -\infty} f(x) = L,
\]
then the line $y=L$ is a \dfn{horizontal asymptote} of $f(x)$.
\end{definition}

\begin{example} 
Give the horizontal asymptotes of
\[
f(x) = \frac{6x-9}{x-1}
\]
\begin{explanation}
From our previous work, we see that $\lim_{x\to \infty} f(x) = 6$, and
upon further inspection, we see that $\lim_{x\to -\infty} f(x) =
6$. Hence the horizontal asymptote of $f(x)$ is the line $y=6$.
\end{explanation}
\end{example}


It is a common misconception that a function cannot cross an
asymptote. As the next example shows, a function can cross a horizontal
asymptote, and in the example this occurs an infinite number of times!

\begin{example}
Give a horizontal asymptote of
\[
f(x) = \frac{\sin(7x)}{x}+4.
\]
\begin{explanation}
Again from previous work, we see that $\lim_{x\to \infty} f(x) =
\answer[given]{4}$. Hence $y=\answer[given]{4}$ is a horizontal asymptote of $f(x)$.
\end{explanation}
\end{example}


We conclude with an infinite limit at infinity.

\begin{example}
Compute
\[
\lim_{x\to \infty} \ln(x)
\]
\begin{image}
\begin{tikzpicture}
	\begin{axis}[
            domain=0:20,
            ymax=4,
            ymin=-1,
            samples=100,
            axis lines =middle, xlabel=$x$, ylabel=$y$,
            every axis y label/.style={at=(current axis.above origin),anchor=south},
            every axis x label/.style={at=(current axis.right of origin),anchor=west}
          ]
	  \addplot [very thick, penColor, smooth] {ln(x)};
        \end{axis}
\end{tikzpicture}
%% \caption{A plot of $f(x)=\ln(x)$.}
%% \label{plot:lnx}
\end{image}
\begin{explanation}
The function $\ln(x)$ grows very slowly, and seems like it may have a
horizontal asymptote, see Figure~\ref{plot:lnx}. However, if we
consider the definition of the natural log as the inverse of the exponential function
\[
\ln(x) = y \qquad \Leftrightarrow\qquad e^y =x
\]
we see that we need to raise $e$ to higher and higher values to obtain
larger numbers.  This means that $\ln(x)$ is unbounded, and hence
$\lim_{x\to\infty}\ln(x)=\infty$.
\end{explanation}
\end{example}


\end{document}
