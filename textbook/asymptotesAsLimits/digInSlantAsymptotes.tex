\documentclass{ximera}

%\usepackage{todonotes}
%\usepackage{mathtools} %% Required for wide table Curl and Greens
%\usepackage{cuted} %% Required for wide table Curl and Greens
\newcommand{\todo}{}

\usepackage{esint} % for \oiint
\ifxake%%https://math.meta.stackexchange.com/questions/9973/how-do-you-render-a-closed-surface-double-integral
\renewcommand{\oiint}{{\large\bigcirc}\kern-1.56em\iint}
\fi


\graphicspath{
  {./}
  {ximeraTutorial/}
  {basicPhilosophy/}
  {functionsOfSeveralVariables/}
  {normalVectors/}
  {lagrangeMultipliers/}
  {vectorFields/}
  {greensTheorem/}
  {shapeOfThingsToCome/}
  {dotProducts/}
  {partialDerivativesAndTheGradientVector/}
  {../productAndQuotientRules/exercises/}
  {../normalVectors/exercisesParametricPlots/}
  {../continuityOfFunctionsOfSeveralVariables/exercises/}
  {../partialDerivativesAndTheGradientVector/exercises/}
  {../directionalDerivativeAndChainRule/exercises/}
  {../commonCoordinates/exercisesCylindricalCoordinates/}
  {../commonCoordinates/exercisesSphericalCoordinates/}
  {../greensTheorem/exercisesCurlAndLineIntegrals/}
  {../greensTheorem/exercisesDivergenceAndLineIntegrals/}
  {../shapeOfThingsToCome/exercisesDivergenceTheorem/}
  {../greensTheorem/}
  {../shapeOfThingsToCome/}
  {../separableDifferentialEquations/exercises/}
  {vectorFields/}
}

\newcommand{\mooculus}{\textsf{\textbf{MOOC}\textnormal{\textsf{ULUS}}}}

\usepackage{tkz-euclide}\usepackage{tikz}
\usepackage{tikz-cd}
\usetikzlibrary{arrows}
\tikzset{>=stealth,commutative diagrams/.cd,
  arrow style=tikz,diagrams={>=stealth}} %% cool arrow head
\tikzset{shorten <>/.style={ shorten >=#1, shorten <=#1 } } %% allows shorter vectors

\usetikzlibrary{backgrounds} %% for boxes around graphs
\usetikzlibrary{shapes,positioning}  %% Clouds and stars
\usetikzlibrary{matrix} %% for matrix
\usepgfplotslibrary{polar} %% for polar plots
\usepgfplotslibrary{fillbetween} %% to shade area between curves in TikZ
\usetkzobj{all}
\usepackage[makeroom]{cancel} %% for strike outs
%\usepackage{mathtools} %% for pretty underbrace % Breaks Ximera
%\usepackage{multicol}
\usepackage{pgffor} %% required for integral for loops



%% http://tex.stackexchange.com/questions/66490/drawing-a-tikz-arc-specifying-the-center
%% Draws beach ball
\tikzset{pics/carc/.style args={#1:#2:#3}{code={\draw[pic actions] (#1:#3) arc(#1:#2:#3);}}}



\usepackage{array}
\setlength{\extrarowheight}{+.1cm}
\newdimen\digitwidth
\settowidth\digitwidth{9}
\def\divrule#1#2{
\noalign{\moveright#1\digitwidth
\vbox{\hrule width#2\digitwidth}}}





\newcommand{\RR}{\mathbb R}
\newcommand{\R}{\mathbb R}
\newcommand{\N}{\mathbb N}
\newcommand{\Z}{\mathbb Z}

\newcommand{\sagemath}{\textsf{SageMath}}


%\renewcommand{\d}{\,d\!}
\renewcommand{\d}{\mathop{}\!d}
\newcommand{\dd}[2][]{\frac{\d #1}{\d #2}}
\newcommand{\pp}[2][]{\frac{\partial #1}{\partial #2}}
\renewcommand{\l}{\ell}
\newcommand{\ddx}{\frac{d}{\d x}}

\newcommand{\zeroOverZero}{\ensuremath{\boldsymbol{\tfrac{0}{0}}}}
\newcommand{\inftyOverInfty}{\ensuremath{\boldsymbol{\tfrac{\infty}{\infty}}}}
\newcommand{\zeroOverInfty}{\ensuremath{\boldsymbol{\tfrac{0}{\infty}}}}
\newcommand{\zeroTimesInfty}{\ensuremath{\small\boldsymbol{0\cdot \infty}}}
\newcommand{\inftyMinusInfty}{\ensuremath{\small\boldsymbol{\infty - \infty}}}
\newcommand{\oneToInfty}{\ensuremath{\boldsymbol{1^\infty}}}
\newcommand{\zeroToZero}{\ensuremath{\boldsymbol{0^0}}}
\newcommand{\inftyToZero}{\ensuremath{\boldsymbol{\infty^0}}}



\newcommand{\numOverZero}{\ensuremath{\boldsymbol{\tfrac{\#}{0}}}}
\newcommand{\dfn}{\textbf}
%\newcommand{\unit}{\,\mathrm}
\newcommand{\unit}{\mathop{}\!\mathrm}
\newcommand{\eval}[1]{\bigg[ #1 \bigg]}
\newcommand{\seq}[1]{\left( #1 \right)}
\renewcommand{\epsilon}{\varepsilon}
\renewcommand{\phi}{\varphi}


\renewcommand{\iff}{\Leftrightarrow}

\DeclareMathOperator{\arccot}{arccot}
\DeclareMathOperator{\arcsec}{arcsec}
\DeclareMathOperator{\arccsc}{arccsc}
\DeclareMathOperator{\si}{Si}
\DeclareMathOperator{\scal}{scal}
\DeclareMathOperator{\sign}{sign}


%% \newcommand{\tightoverset}[2]{% for arrow vec
%%   \mathop{#2}\limits^{\vbox to -.5ex{\kern-0.75ex\hbox{$#1$}\vss}}}
\newcommand{\arrowvec}[1]{{\overset{\rightharpoonup}{#1}}}
%\renewcommand{\vec}[1]{\arrowvec{\mathbf{#1}}}
\renewcommand{\vec}[1]{{\overset{\boldsymbol{\rightharpoonup}}{\mathbf{#1}}}\hspace{0in}}

\newcommand{\point}[1]{\left(#1\right)} %this allows \vector{ to be changed to \vector{ with a quick find and replace
\newcommand{\pt}[1]{\mathbf{#1}} %this allows \vec{ to be changed to \vec{ with a quick find and replace
\newcommand{\Lim}[2]{\lim_{\point{#1} \to \point{#2}}} %Bart, I changed this to point since I want to use it.  It runs through both of the exercise and exerciseE files in limits section, which is why it was in each document to start with.

\DeclareMathOperator{\proj}{\mathbf{proj}}
\newcommand{\veci}{{\boldsymbol{\hat{\imath}}}}
\newcommand{\vecj}{{\boldsymbol{\hat{\jmath}}}}
\newcommand{\veck}{{\boldsymbol{\hat{k}}}}
\newcommand{\vecl}{\vec{\boldsymbol{\l}}}
\newcommand{\uvec}[1]{\mathbf{\hat{#1}}}
\newcommand{\utan}{\mathbf{\hat{t}}}
\newcommand{\unormal}{\mathbf{\hat{n}}}
\newcommand{\ubinormal}{\mathbf{\hat{b}}}

\newcommand{\dotp}{\bullet}
\newcommand{\cross}{\boldsymbol\times}
\newcommand{\grad}{\boldsymbol\nabla}
\newcommand{\divergence}{\grad\dotp}
\newcommand{\curl}{\grad\cross}
%\DeclareMathOperator{\divergence}{divergence}
%\DeclareMathOperator{\curl}[1]{\grad\cross #1}
\newcommand{\lto}{\mathop{\longrightarrow\,}\limits}

\renewcommand{\bar}{\overline}

\colorlet{textColor}{black}
\colorlet{background}{white}
\colorlet{penColor}{blue!50!black} % Color of a curve in a plot
\colorlet{penColor2}{red!50!black}% Color of a curve in a plot
\colorlet{penColor3}{red!50!blue} % Color of a curve in a plot
\colorlet{penColor4}{green!50!black} % Color of a curve in a plot
\colorlet{penColor5}{orange!80!black} % Color of a curve in a plot
\colorlet{penColor6}{yellow!70!black} % Color of a curve in a plot
\colorlet{fill1}{penColor!20} % Color of fill in a plot
\colorlet{fill2}{penColor2!20} % Color of fill in a plot
\colorlet{fillp}{fill1} % Color of positive area
\colorlet{filln}{penColor2!20} % Color of negative area
\colorlet{fill3}{penColor3!20} % Fill
\colorlet{fill4}{penColor4!20} % Fill
\colorlet{fill5}{penColor5!20} % Fill
\colorlet{gridColor}{gray!50} % Color of grid in a plot

\newcommand{\surfaceColor}{violet}
\newcommand{\surfaceColorTwo}{redyellow}
\newcommand{\sliceColor}{greenyellow}




\pgfmathdeclarefunction{gauss}{2}{% gives gaussian
  \pgfmathparse{1/(#2*sqrt(2*pi))*exp(-((x-#1)^2)/(2*#2^2))}%
}


%%%%%%%%%%%%%
%% Vectors
%%%%%%%%%%%%%

%% Simple horiz vectors
\renewcommand{\vector}[1]{\left\langle #1\right\rangle}


%% %% Complex Horiz Vectors with angle brackets
%% \makeatletter
%% \renewcommand{\vector}[2][ , ]{\left\langle%
%%   \def\nextitem{\def\nextitem{#1}}%
%%   \@for \el:=#2\do{\nextitem\el}\right\rangle%
%% }
%% \makeatother

%% %% Vertical Vectors
%% \def\vector#1{\begin{bmatrix}\vecListA#1,,\end{bmatrix}}
%% \def\vecListA#1,{\if,#1,\else #1\cr \expandafter \vecListA \fi}

%%%%%%%%%%%%%
%% End of vectors
%%%%%%%%%%%%%

%\newcommand{\fullwidth}{}
%\newcommand{\normalwidth}{}



%% makes a snazzy t-chart for evaluating functions
%\newenvironment{tchart}{\rowcolors{2}{}{background!90!textColor}\array}{\endarray}

%%This is to help with formatting on future title pages.
\newenvironment{sectionOutcomes}{}{}



%% Flowchart stuff
%\tikzstyle{startstop} = [rectangle, rounded corners, minimum width=3cm, minimum height=1cm,text centered, draw=black]
%\tikzstyle{question} = [rectangle, minimum width=3cm, minimum height=1cm, text centered, draw=black]
%\tikzstyle{decision} = [trapezium, trapezium left angle=70, trapezium right angle=110, minimum width=3cm, minimum height=1cm, text centered, draw=black]
%\tikzstyle{question} = [rectangle, rounded corners, minimum width=3cm, minimum height=1cm,text centered, draw=black]
%\tikzstyle{process} = [rectangle, minimum width=3cm, minimum height=1cm, text centered, draw=black]
%\tikzstyle{decision} = [trapezium, trapezium left angle=70, trapezium right angle=110, minimum width=3cm, minimum height=1cm, text centered, draw=black]


\outcome{Define a slant asymptote.}
\outcome{Approximate a slant asymptote from the graph of a function.}
\outcome{Find slant asymptotes algebraically and graphically.}

\title[Dig-In:]{Slant asymptotes}

\begin{document}
\begin{abstract}
We explore functions that ``shoot to infinity'' at certain points in
their domain.
\end{abstract}
\maketitle

If we think of an asymptote as a ``line that a function resembles when
the input is large,'' then there are three types of asymptotes, just
as there are three types of lines:
\begin{align*}
  \text{Vertical Asymptotes} \qquad&\leftrightarrow\qquad \text{Vertical Lines}\\
  \text{Horizontal Aymptotes}\qquad&\leftrightarrow\qquad \text{Horizontal Lines} \\
  \text{Slant Asymptotes}\qquad&\leftrightarrow\qquad \text{Slant Lines} 
\end{align*}
Here we've made up a new term ``slant'' line, meaning a line whose
slope is neither zero, nor is it undefined. Let's do a quick review of
the different types of asymptotes:



\paragraph{Vertical asymptotes}

Recall, a function $f$ has a vertical asymptote at $x=a$ if at least
one of the following hold:
\begin{itemize}
\item $\lim_{x\to a} f(x) = \pm\infty$,
\item $\lim_{x\to a^+} f(x) = \pm\infty$,
\item $\lim_{x\to a^-} f(x) = \pm\infty$.
\end{itemize}
In this case, the asymptote is the vertical line
\[
x = a.
\]



\paragraph{Horizontal asymptotes}

We have also seen that a function $f$ has a horizontal asymptote if
\[
\lim_{x\to \infty} f(x) = L \qquad\text{or}\qquad \lim_{x\to -\infty}
f(x) = L,
\]
and in this case, the asymptote is the horizontal line
\[
\l(x) = L.
\]



\paragraph{Slant asymptotes}

On the other hand, a \textit{slant asymptote} is a somewhat different
beast.

\begin{definition}\index{slant asymptote}\index{asymptote!slant}
  If there is a nonhorizontal line $\l(x) = m\cdot x+b$ such that
  \[
  \lim_{x\to \infty}\left(f(x) - \l(x)\right) = 0 \qquad\text{or}\qquad \lim_{x\to -\infty} \left( f(x) - \l(x)\right) = 0,
  \]
  then $\l$ is a \dfn{slant asymptote} for $f$.
\end{definition}
\begin{question}
  Consider the graph of the following function. 
  \begin{image}
    \begin{tikzpicture}
      \begin{axis}[
          samples=100,
          axis lines =middle, xlabel=$x$, ylabel=$y$,
          every axis y label/.style={at=(current axis.above origin),anchor=south},
          every axis x label/.style={at=(current axis.right of origin),anchor=west},
        ]
	\addplot [very thick, penColor, smooth,domain=0:5] {x/2-1};
        \addplot [very thick, penColor, smooth,domain=-5:0] {sqrt(-x)};
        \addplot[color=penColor,fill=background,only marks,mark=*] coordinates{(0,-1)};  %% open hole
        \addplot[color=penColor,fill=penColor,only marks,] coordinates{(0,0)};  %% closed hole
      \end{axis}
    \end{tikzpicture}
  \end{image}
  What is the slant asymptote of this function?
  \begin{prompt}
    \[
    \l(x) = \answer{x/2 -1}
    \]
  \end{prompt}
\end{question}

To analytically find slant asymptotes, one must find the required
information to determine a line:
\begin{itemize}
\item The slope.
\item The $y$-intercept.
\end{itemize}
While there are several ways to do this, we will give a method that is
fairly general.


\begin{example}
  Find the slant asymptote of
  \[
f(x) = \frac{3x^2+x+2}{x+2}.
  \]
  \begin{explanation}
    We are looking to see if there is a line $\l$ such that
    \[
    \lim_{x\to \infty}\left(f(x) - \l(x)\right) = 0\qquad\text{or}\qquad \lim_{x\to -\infty} \left( f(x) - \l(x)\right) = 0.
    \]
    First, let's consider the limit as $x$ approaches positive infinity.  We will imagine that we have such a line
    \[
    \l(x) = m\cdot x + b
    \]
    and attempt to find the correct values for $m$ and $b$.  Let's look again at our limit.  We are assuming:
\[
    \lim_{x \to \infty} \left (\frac{3x^2 + x + 2}{x+2} - (mx + b) \right) = 0.
\]
    We know that $f(x)$ is continuous everywhere except at $x = -2$ and $\l(x)$ is continuous everywhere, so 
    we can apply our limit laws away from $x = -2$.  We're looking at large values of $x$, so this is no problem.  We use the fact that the sum of the limits is the limit of the sums.
    \begin{align*}
    \lim_{x \to \infty} \left ( \frac{3x^2 + x + 2}{x+2} \right ) - &\lim_{x\to\infty} (mx+b) = 0 \\
    \lim_{x \to \infty} \frac{3x^2 + x + 2}{x+2} = &\lim_{x \to \infty} (mx+b)
    \end{align*}
    
    We are assuming these two limits are equal.  Dividing by $x$ on the right hand side makes the limit equal to $m$:
    \begin{align*}
    m = \lim_{x \to \infty} \left(\frac{mx}{x} + \frac{b}{x}\right) = \lim_{x \to \infty} \left(\frac{mx+b}{x}\right).
    \end{align*}
    To find the value of $m$, then, we can divide the left hand side by $x$ and evaluate the limit.   We see the following. 
    \begin{align*}
      m &=\lim_{x\to\infty}\frac{\frac{3x^2+x+2}{x+2}}{x}\\
      &= \lim_{x\to\infty}\frac{\frac{3x^2+x+2}{x+2}}{x}\\
      &= \lim_{x\to\infty}\frac{3x^2+x+2}{x^2+2x}\\
      &= \lim_{x\to\infty}\left(\frac{3x^2+x+2}{x^2+2x}\cdot\frac{1/x^2}{1/x^2}\right)\\
      &= \lim_{x\to\infty}\frac{3+1/x+2/x^2}{1+2/x}\\
      &= \answer[given]{3}.
    \end{align*}
    So $m=3$.  We now know that
    \[
    \lim_{x \to \infty}\frac{3x^2 +x+2}{x+2} = \lim_{x \to \infty} (3x + b)
    \]
    for some value of $b$.  To find the $y$-intercept $b$, we use a similar method.  Notice that
    \[
    \lim_{x \to \infty}( 3x + b - 3x )= \lim_{x \to \infty} b = b,
    \]
    so if we subtract $3x$ from the right hand side, we are left with just $b$.  Since the two sides are equal, subtracting $3x$ from the left hand side and evaluating the limit will give us the value for $b$.  We write the following.
    \begin{align*}
      b &=\lim_{x\to\infty} \left(\frac{3x^2+x+2}{x+2} - 3x\right)\\
      &=\lim_{x\to\infty} \left( \frac{3x^2+x+2}{x+2} - \frac{3x^2 + 6x}{x+2}\right) \\
      &=\lim_{x\to\infty} \frac{3x^2+x+2-3x^2-6x}{x+2}\\
      &=\lim_{x\to\infty}  \frac{-5x+2}{x+2}\\
      &=\lim_{x\to\infty} \left(\frac{-5x+2}{x+2}\cdot\frac{1/x}{1/x}\right)\\
      &=\lim_{x\to\infty} \frac{-5+2/x}{1+2/x}\\
      &= \answer[given]{-5}.
    \end{align*}
    By this method, we have determined that 
    \[
    \lim_{x\to\infty}\left(\frac{3x^2+x+2}{x+2} - (3x-5) \right) = 0
    \]
   In other words, $\l(x) = \answer[given]{3x-5}$ is a slant asymptote for our function $f(x)$.  
   
   You should check that we get the same slant asymptote $\l(x) =
   \answer[given]{3x-5}$ when we take the limit to negative infinity
   as well.
  \end{explanation}
\end{example}





\end{document}
