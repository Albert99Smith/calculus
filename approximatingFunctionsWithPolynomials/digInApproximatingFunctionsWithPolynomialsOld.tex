\documentclass{ximera}

%\usepackage{todonotes}
%\usepackage{mathtools} %% Required for wide table Curl and Greens
%\usepackage{cuted} %% Required for wide table Curl and Greens
\newcommand{\todo}{}

\usepackage{esint} % for \oiint
\ifxake%%https://math.meta.stackexchange.com/questions/9973/how-do-you-render-a-closed-surface-double-integral
\renewcommand{\oiint}{{\large\bigcirc}\kern-1.56em\iint}
\fi


\graphicspath{
  {./}
  {ximeraTutorial/}
  {basicPhilosophy/}
  {functionsOfSeveralVariables/}
  {normalVectors/}
  {lagrangeMultipliers/}
  {vectorFields/}
  {greensTheorem/}
  {shapeOfThingsToCome/}
  {dotProducts/}
  {partialDerivativesAndTheGradientVector/}
  {../productAndQuotientRules/exercises/}
  {../normalVectors/exercisesParametricPlots/}
  {../continuityOfFunctionsOfSeveralVariables/exercises/}
  {../partialDerivativesAndTheGradientVector/exercises/}
  {../directionalDerivativeAndChainRule/exercises/}
  {../commonCoordinates/exercisesCylindricalCoordinates/}
  {../commonCoordinates/exercisesSphericalCoordinates/}
  {../greensTheorem/exercisesCurlAndLineIntegrals/}
  {../greensTheorem/exercisesDivergenceAndLineIntegrals/}
  {../shapeOfThingsToCome/exercisesDivergenceTheorem/}
  {../greensTheorem/}
  {../shapeOfThingsToCome/}
  {../separableDifferentialEquations/exercises/}
  {vectorFields/}
}

\newcommand{\mooculus}{\textsf{\textbf{MOOC}\textnormal{\textsf{ULUS}}}}

\usepackage{tkz-euclide}\usepackage{tikz}
\usepackage{tikz-cd}
\usetikzlibrary{arrows}
\tikzset{>=stealth,commutative diagrams/.cd,
  arrow style=tikz,diagrams={>=stealth}} %% cool arrow head
\tikzset{shorten <>/.style={ shorten >=#1, shorten <=#1 } } %% allows shorter vectors

\usetikzlibrary{backgrounds} %% for boxes around graphs
\usetikzlibrary{shapes,positioning}  %% Clouds and stars
\usetikzlibrary{matrix} %% for matrix
\usepgfplotslibrary{polar} %% for polar plots
\usepgfplotslibrary{fillbetween} %% to shade area between curves in TikZ
\usetkzobj{all}
\usepackage[makeroom]{cancel} %% for strike outs
%\usepackage{mathtools} %% for pretty underbrace % Breaks Ximera
%\usepackage{multicol}
\usepackage{pgffor} %% required for integral for loops



%% http://tex.stackexchange.com/questions/66490/drawing-a-tikz-arc-specifying-the-center
%% Draws beach ball
\tikzset{pics/carc/.style args={#1:#2:#3}{code={\draw[pic actions] (#1:#3) arc(#1:#2:#3);}}}



\usepackage{array}
\setlength{\extrarowheight}{+.1cm}
\newdimen\digitwidth
\settowidth\digitwidth{9}
\def\divrule#1#2{
\noalign{\moveright#1\digitwidth
\vbox{\hrule width#2\digitwidth}}}





\newcommand{\RR}{\mathbb R}
\newcommand{\R}{\mathbb R}
\newcommand{\N}{\mathbb N}
\newcommand{\Z}{\mathbb Z}

\newcommand{\sagemath}{\textsf{SageMath}}


%\renewcommand{\d}{\,d\!}
\renewcommand{\d}{\mathop{}\!d}
\newcommand{\dd}[2][]{\frac{\d #1}{\d #2}}
\newcommand{\pp}[2][]{\frac{\partial #1}{\partial #2}}
\renewcommand{\l}{\ell}
\newcommand{\ddx}{\frac{d}{\d x}}

\newcommand{\zeroOverZero}{\ensuremath{\boldsymbol{\tfrac{0}{0}}}}
\newcommand{\inftyOverInfty}{\ensuremath{\boldsymbol{\tfrac{\infty}{\infty}}}}
\newcommand{\zeroOverInfty}{\ensuremath{\boldsymbol{\tfrac{0}{\infty}}}}
\newcommand{\zeroTimesInfty}{\ensuremath{\small\boldsymbol{0\cdot \infty}}}
\newcommand{\inftyMinusInfty}{\ensuremath{\small\boldsymbol{\infty - \infty}}}
\newcommand{\oneToInfty}{\ensuremath{\boldsymbol{1^\infty}}}
\newcommand{\zeroToZero}{\ensuremath{\boldsymbol{0^0}}}
\newcommand{\inftyToZero}{\ensuremath{\boldsymbol{\infty^0}}}



\newcommand{\numOverZero}{\ensuremath{\boldsymbol{\tfrac{\#}{0}}}}
\newcommand{\dfn}{\textbf}
%\newcommand{\unit}{\,\mathrm}
\newcommand{\unit}{\mathop{}\!\mathrm}
\newcommand{\eval}[1]{\bigg[ #1 \bigg]}
\newcommand{\seq}[1]{\left( #1 \right)}
\renewcommand{\epsilon}{\varepsilon}
\renewcommand{\phi}{\varphi}


\renewcommand{\iff}{\Leftrightarrow}

\DeclareMathOperator{\arccot}{arccot}
\DeclareMathOperator{\arcsec}{arcsec}
\DeclareMathOperator{\arccsc}{arccsc}
\DeclareMathOperator{\si}{Si}
\DeclareMathOperator{\scal}{scal}
\DeclareMathOperator{\sign}{sign}


%% \newcommand{\tightoverset}[2]{% for arrow vec
%%   \mathop{#2}\limits^{\vbox to -.5ex{\kern-0.75ex\hbox{$#1$}\vss}}}
\newcommand{\arrowvec}[1]{{\overset{\rightharpoonup}{#1}}}
%\renewcommand{\vec}[1]{\arrowvec{\mathbf{#1}}}
\renewcommand{\vec}[1]{{\overset{\boldsymbol{\rightharpoonup}}{\mathbf{#1}}}\hspace{0in}}

\newcommand{\point}[1]{\left(#1\right)} %this allows \vector{ to be changed to \vector{ with a quick find and replace
\newcommand{\pt}[1]{\mathbf{#1}} %this allows \vec{ to be changed to \vec{ with a quick find and replace
\newcommand{\Lim}[2]{\lim_{\point{#1} \to \point{#2}}} %Bart, I changed this to point since I want to use it.  It runs through both of the exercise and exerciseE files in limits section, which is why it was in each document to start with.

\DeclareMathOperator{\proj}{\mathbf{proj}}
\newcommand{\veci}{{\boldsymbol{\hat{\imath}}}}
\newcommand{\vecj}{{\boldsymbol{\hat{\jmath}}}}
\newcommand{\veck}{{\boldsymbol{\hat{k}}}}
\newcommand{\vecl}{\vec{\boldsymbol{\l}}}
\newcommand{\uvec}[1]{\mathbf{\hat{#1}}}
\newcommand{\utan}{\mathbf{\hat{t}}}
\newcommand{\unormal}{\mathbf{\hat{n}}}
\newcommand{\ubinormal}{\mathbf{\hat{b}}}

\newcommand{\dotp}{\bullet}
\newcommand{\cross}{\boldsymbol\times}
\newcommand{\grad}{\boldsymbol\nabla}
\newcommand{\divergence}{\grad\dotp}
\newcommand{\curl}{\grad\cross}
%\DeclareMathOperator{\divergence}{divergence}
%\DeclareMathOperator{\curl}[1]{\grad\cross #1}
\newcommand{\lto}{\mathop{\longrightarrow\,}\limits}

\renewcommand{\bar}{\overline}

\colorlet{textColor}{black}
\colorlet{background}{white}
\colorlet{penColor}{blue!50!black} % Color of a curve in a plot
\colorlet{penColor2}{red!50!black}% Color of a curve in a plot
\colorlet{penColor3}{red!50!blue} % Color of a curve in a plot
\colorlet{penColor4}{green!50!black} % Color of a curve in a plot
\colorlet{penColor5}{orange!80!black} % Color of a curve in a plot
\colorlet{penColor6}{yellow!70!black} % Color of a curve in a plot
\colorlet{fill1}{penColor!20} % Color of fill in a plot
\colorlet{fill2}{penColor2!20} % Color of fill in a plot
\colorlet{fillp}{fill1} % Color of positive area
\colorlet{filln}{penColor2!20} % Color of negative area
\colorlet{fill3}{penColor3!20} % Fill
\colorlet{fill4}{penColor4!20} % Fill
\colorlet{fill5}{penColor5!20} % Fill
\colorlet{gridColor}{gray!50} % Color of grid in a plot

\newcommand{\surfaceColor}{violet}
\newcommand{\surfaceColorTwo}{redyellow}
\newcommand{\sliceColor}{greenyellow}




\pgfmathdeclarefunction{gauss}{2}{% gives gaussian
  \pgfmathparse{1/(#2*sqrt(2*pi))*exp(-((x-#1)^2)/(2*#2^2))}%
}


%%%%%%%%%%%%%
%% Vectors
%%%%%%%%%%%%%

%% Simple horiz vectors
\renewcommand{\vector}[1]{\left\langle #1\right\rangle}


%% %% Complex Horiz Vectors with angle brackets
%% \makeatletter
%% \renewcommand{\vector}[2][ , ]{\left\langle%
%%   \def\nextitem{\def\nextitem{#1}}%
%%   \@for \el:=#2\do{\nextitem\el}\right\rangle%
%% }
%% \makeatother

%% %% Vertical Vectors
%% \def\vector#1{\begin{bmatrix}\vecListA#1,,\end{bmatrix}}
%% \def\vecListA#1,{\if,#1,\else #1\cr \expandafter \vecListA \fi}

%%%%%%%%%%%%%
%% End of vectors
%%%%%%%%%%%%%

%\newcommand{\fullwidth}{}
%\newcommand{\normalwidth}{}



%% makes a snazzy t-chart for evaluating functions
%\newenvironment{tchart}{\rowcolors{2}{}{background!90!textColor}\array}{\endarray}

%%This is to help with formatting on future title pages.
\newenvironment{sectionOutcomes}{}{}



%% Flowchart stuff
%\tikzstyle{startstop} = [rectangle, rounded corners, minimum width=3cm, minimum height=1cm,text centered, draw=black]
%\tikzstyle{question} = [rectangle, minimum width=3cm, minimum height=1cm, text centered, draw=black]
%\tikzstyle{decision} = [trapezium, trapezium left angle=70, trapezium right angle=110, minimum width=3cm, minimum height=1cm, text centered, draw=black]
%\tikzstyle{question} = [rectangle, rounded corners, minimum width=3cm, minimum height=1cm,text centered, draw=black]
%\tikzstyle{process} = [rectangle, minimum width=3cm, minimum height=1cm, text centered, draw=black]
%\tikzstyle{decision} = [trapezium, trapezium left angle=70, trapezium right angle=110, minimum width=3cm, minimum height=1cm, text centered, draw=black]


\outcome{Compute Taylor polynomials.}
\outcome{Use Taylor's theorem to estimate the error of a Taylor polynomial.}
\outcome{Determine the maximum error between a function and a given
  Taylor polynomial.}

\title[Dig-In:]{Approximating functions with polynomials}

\begin{document}
\begin{abstract}
We can approximate smooth functions with polynomials.
\end{abstract}
\maketitle


\section{Polynomials can approximate some functions}

In our study of mathematics, we've found that some functions are
easier to work with than others. For instance, if you are doing
calculus, typically polynomials are ``easy'' to work with because they
are easy to differentiate and integrate. Other functions, like
\[
f(x) =
\begin{cases}
  \frac{\sin(x)}{x} &\text{if $x\ne 0$}\\
  1 &\text{if $x=0$}
\end{cases}
\]
are more difficult to work with. However, there are
polynomials that mimic the behavior of $f$ near zero.
\begin{image}
  \begin{tikzpicture}
    \begin{axis}[
        xmin=-6.75,xmax=6.75,ymin=-1.5,ymax=1.5,
        axis lines=center,
        xtick={-6.28, -4.71, -3.14, -1.57, 0, 1.57, 3.142, 4.71, 6.28},
        xticklabels={$-2\pi$,$-3\pi/2$,$-\pi$, $-\pi/2$, $0$, $\pi/2$, $\pi$, $3\pi/2$, $2\pi$},
        ytick={-1,1},
        %ticks=none,
        width=6in,
        height=3in,
        unit vector ratio*=1 1 1,
        xlabel=$x$, ylabel=$y$,
        every axis y label/.style={at=(current axis.above origin),anchor=south},
        every axis x label/.style={at=(current axis.right of origin),anchor=west},
      ]        
      \addplot [very thick, penColor, samples=100,smooth, domain=(-6.75:6.75)] {sin(deg(x))/x};
      \addplot [very thick, penColor2, samples=100,smooth, domain=(-6.75:6.75)] {1-x^2/6 + x^4/120};
      %\addplot[color=penColor,fill=white,only marks,mark=*] coordinates{(0,1)};  %% open hole          
    \end{axis}
  \end{tikzpicture}
\end{image}
Above we see a graph of $f$ along with the polynomial
\[
1-\frac{x^2}{6}+\frac{x^4}{120}.
\]
As we see, this polynomial approximates $f$ very well near zero.  There are times 
when we would much rather work with a polynomial than any other type of function. 
Many of the ideas we discuss in this section should remind you of our work with 
linear approximation.  In the case of linear approximation, we replaced a difficult 
function with a line, and used this line to approximate the function.  Now, we want to 
replace our complicated function with a polynomial. This leads us to a question.


\subsection{How do we produce approximating polynomials?}

Cutting straight to the point, the approximating polynomials we'll
discuss are called \textit{Taylor polynomials} and \textit{Maclaurin
  polynomials}.

\begin{definition}
  Let $f$ be a function whose first $n$ derivatives exist at $x=c$.
  \begin{itemize}
  \item The \dfn{Taylor polynomial} of degree $n$ of $f$ at $x=c$ is
    \begin{align*}
      p_n(x) = f(c) &+ f'(c)(x-c) \\
      &+ \frac{f''(c)}{2!}(x-c)^2 \\
      &+\frac{f'''(c)}{3!}(x-c)^3 \\
      &\ \vdots \\
      &+\frac{f^{(n)}(c)}{n!}(x-c)^n.
    \end{align*}
  \item A special case of the Taylor polynomial is the Maclaurin
    polynomial, where $c=0$. That is, the \dfn{Maclaurin polynomial}
    of degree $n$ of $f$ is
    \begin{align*}
      p_n(x) = f(0) &+ f'(0)x \\
      &+\frac{f''(0)}{2!}x^2\\
      &+\frac{f'''(0)}{3!}x^3\\
      &\ \vdots\\
      &+\frac{f^{(n)}(0)}{n!}x^n.
    \end{align*}
  \end{itemize}
  We say these polynomials have a \dfn{center} of $x=c$, and so
  Maclaurin polynomials are Taylor polynomials centered at
  zero.
\end{definition}

\begin{question}
  Consider $f(x) = e^x$. Which of the following is a Maclaurin
  polynomial for $e^x$?
  \begin{selectAll}
    \choice{$1+x+x^2+x^3+x^4$}
    \choice[correct]{$1+x$}
    \choice[correct]{$1+x+\frac{x^2}{2!} + \frac{x^3}{3!}+ \frac{x^4}{4!}$}
    \choice{$1-\frac{x^2}{2!} +\frac{x^4}{4!}$}
    \choice{$x-\frac{x^3}{3!}$}
    \choice[correct]{$1$}
  \end{selectAll}
  Check out a plot of $e^x$ along with the plots of the correct answers above:
  \begin{image}
    \begin{tikzpicture}
      \begin{axis}[
          xmin=-5,xmax=4,ymin=-1,ymax=10,
          axis lines=center,
          %ticks=none,
          width=6in,
          height=3in,
          %unit vector ratio*=1 1 1,
          xlabel=$x$, ylabel=$y$,
          every axis y label/.style={at=(current axis.above origin),anchor=south},
          every axis x label/.style={at=(current axis.right of origin),anchor=west},
        ]        
        \addplot [very thick, penColor, samples=100,smooth] {e^x};
        \addplot [very thick, penColor2, samples=100,smooth] {1};
        \addplot [very thick, penColor3, samples=100,smooth] {1+x};
        \addplot [very thick, penColor4, samples=100,smooth] {1+x+x^2/2+x^3/6+x^4/24};
      \end{axis}
    \end{tikzpicture}
  \end{image}
\end{question}

\begin{question}
  Consider $f(x) = \ln(x)$. Which of the following is a Taylor
  polynomial centered at $1$ for $f$?
    \begin{selectAll}
    \choice[correct]{$x-1$}
    \choice{$1+x$}
    \choice{$1+x+\frac{x^2}{2!} + \frac{x^3}{3!}+ \frac{x^4}{4!}$}
    \choice[correct]{$(x-1)-\frac{(x-1)^2}{2}+\frac{(x-1)^3}{3}$}
    \choice[correct]{$(x-1)-\frac{(x-1)^2}{2}+\frac{(x-1)^3}{3}-\frac{(x-1)^4}{4}$}
    \choice{$1$}
  \end{selectAll}
  Check out a plot of $\ln(x)$ along with the plots of the correct answers above.
  \begin{image}
    \begin{tikzpicture}
      \begin{axis}[
          xmin=-1,xmax=4,ymin=-5,ymax=5,
          axis lines=center,
          %ticks=none,
          width=6in,
          height=3in,
          %unit vector ratio*=1 1 1,
          xlabel=$x$, ylabel=$y$,
          every axis y label/.style={at=(current axis.above origin),anchor=south},
          every axis x label/.style={at=(current axis.right of origin),anchor=west},
        ]        
        \addplot [very thick, penColor, samples=100,smooth,domain=.01:4] {ln(x)};
        \addplot [very thick, penColor2, samples=100,smooth] {(x-1)};
        \addplot [very thick, penColor3, samples=100,smooth] {(x-1)-(1/2) *(x-1)^2+1/3* (x-1)^3};
        \addplot [very thick, penColor4, samples=100,smooth] {(x-1)-(1/2) *(x-1)^2+(1/3) * (x-1)^3-(1/4) * (x-1)^4};  
      \end{axis}
    \end{tikzpicture}
  \end{image}
\end{question}

\begin{example}
  Compute the Maclaurin polynomial of degree $4$ for $f(x) =
  \frac{1}{1-x}$.
  \begin{explanation}
    To do this, we set $f(x) =\frac{1}{1-x}$ and use the following formula.
    \[
    p_n(x) = f(0) + f'(0)x +\frac{f''(0)}{2!}x^2+\frac{f'''(0)}{3!}x^3+\frac{f^{(4)}(0)}{4!}x^4
    \]
    So we must compute the first four derivatives of $f$
    \begin{align*}
      f(x) &= \answer[given]{\frac{1}{1-x}},\\
      f'(x) &= \answer[given]{\frac{1}{(1-x)^2}},\\
      f''(x) &= \answer[given]{\frac{2}{(1-x)^3}},\\
      f'''(x) &= \answer[given]{\frac{2\cdot 3}{(1-x)^4}},\\
      f^{(4)}(x) &= \answer[given]{\frac{2\cdot 3\cdot 4}{(1-x)^5}},
    \end{align*}
    and evaluate each of these at $x=0$. Write with me.
    \begin{align*}
      f(0) &= \answer[given]{1},\\
      f'(0) &= \answer[given]{1},\\
      f''(0) &= \answer[given]{2},\\
      f'''(0) &=\answer[given]{2\cdot 3},\\
      f^{(4)}(0) &= \answer[given]{2\cdot 3\cdot 4}.
    \end{align*}
    Now, plugging this into our formula above we see
    \[
    p_4 = \answer[given]{1 + x + x^2 + x^3 + x^4}.
    \]
    (Notice that the factorials have canceled, here.)  Finally, let's see a graph of $p_4$ and $f$.
      \begin{image}
    \begin{tikzpicture}
      \begin{axis}[
          xmin=-1.5,xmax=1.5,ymin=-1,ymax=5,
          axis lines=center,
          %ticks=none,
          width=6in,
          height=3in,
          %unit vector ratio*=1 1 1,
          xlabel=$x$, ylabel=$y$,
          every axis y label/.style={at=(current axis.above origin),anchor=south},
          every axis x label/.style={at=(current axis.right of origin),anchor=west},
        ]        
        \addplot [very thick, penColor, samples=100,smooth,domain=-1.5:.99] {1/(1-x)};
        \addplot [very thick, penColor2, smooth,domain=-1.5:.99] {1+x+x^2+x^3+x^4};
      \end{axis}
    \end{tikzpicture}
  \end{image}
  \end{explanation}
\end{example}


\begin{example}
Find the $n$th Taylor polynomial of $y=\ln(x)$ centered at $x=1$.
\begin{explanation}
  To do this, we set $f(x) =\ln(x)$ and use the following formula:
    \[
    p_n(x) = f(0) + f'(0)x +\frac{f''(0)}{2!}x^2+\frac{f'''(0)}{3!}x^3+\cdots+\frac{f^{(n)}(0)}{n!}x^n
    \]
    So we must compute the first several derivatives of $f$, then look for a pattern 
    to tell us about the $n$-th derivative of $f$.
    \begin{align*}
      f(x) &= \answer[given]{\ln(x)},\\
      f'(x) &= \answer[given]{\frac{1}{x}},\\
      f''(x) &= \answer[given]{\frac{-1}{x^2}},\\
      f'''(x) &= \answer[given]{\frac{2}{x^3}},\\
      f^{(4)}(x) &= \answer[given]{\frac{-2\cdot 3}{x^4}},\\
      f^{(5)}(x) &= \answer[given]{\frac{2\cdot 3\cdot 4}{x^5}},\\
      &\vdots\\
      f^{(n)}(x) &= \frac{(-1)^{n-1}\cdot (n-1)!}{x^n},
    \end{align*}
    As usual, it's a good idea to check and make sure that the formula we found 
    for the $n$-th derivative works for $n=1$ through $n=5$.  Now, we evaluate 
    each of these derivatives at $x=1$. Write with me.
    \begin{align*}
      f(1) &= \answer[given]{0},\\
      f'(1) &= \answer[given]{1},\\
      f''(1) &= \answer[given]{-1},\\
      f'''(1) &= \answer[given]{2},\\
      f^{(4)}(1) &= \answer[given]{-6},\\
      f^{(5)}(1) &= \answer[given]{24},\\
      &\vdots\\
      f^{(n)}(1) &= (-1)^{n-1}\cdot (n-1)!
    \end{align*}
    Finally, plugging this into our formula above we see
    \[
    p_n = (x-1) -\frac{(x-1)^2}{2}+\cdots+(-1)^{n-1}\frac{(x-1)^n}{n}.
    \]
\end{explanation}
\end{example}

In the previous example, why did we choose to use $x=1$ for the center? 
There are several reasons.  We often like to use $x=0$ for the center of a 
Taylor polynomial (which is why such polynomials have a special name).  In 
the case of $f(x) = \ln(x)$, however, $f(0)$ is undefined.  So, we must choose 
a different center.  Since we have to evaluate the function at the center, any 
center point we choose should be a value for which we know and can easily 
compute the function value.  In the case of $f(x) = \ln(x)$, the value we know 
best is $f(1) = 0$, so $x=1$ is a good choice for the center.  Of course, in general we 
can choose any center we like, but this will give us a different polynomial (and 
may make our calculations more difficult.)

\begin{question}
  Using $p_6(x)$, what is the approximate value of $\ln(1.5)$?
  \begin{prompt}
    \[
    p_6(1.5) = \answer{.404688}
    \]
  \end{prompt}
  \begin{hint}
    Since $p_n(x)$ approximates $\ln(x)$ well near $x=1$, we approximate
    $\ln(1.5) \approx p_6(1.5)$:
    \begin{align*}
      p_6(1.5) &= (1.5-1)-\frac12(1.5-1)^2+\frac13(1.5-1)^3-\frac14(1.5-1)^4+\cdots \\
      &\cdots +\frac15(1.5-1)^5-\frac16(1.5-1)^6\\
      &=\frac{259}{640}\\
      &\approx 0.404688.
    \end{align*}
    This is a good approximation as a calculator shows that $\ln(1.5)
    \approx 0.4055.$
  \end{hint}
  \begin{question}
    Using $p_6(x)$, what is approximate value of $\ln(2)$?
    \begin{prompt} 
      \[
      p_6(2) = \answer{0.616667}
      \]
    \end{prompt}
    \begin{hint}
      We approximate $\ln(2)$ with $ p_6(2)$:
      \begin{align*}
        p_6(2) &= (2-1)-\frac12(2-1)^2+\frac13(2-1)^3-\frac14(2-1)^4+\cdots \\
        &\cdots +\frac15(2-1)^5-\frac16(2-1)^6\\
        &=	1-\frac12+\frac13-\frac14+\frac15-\frac16 \\
        &= \frac{37}{60}\\ 
        &\approx 0.616667.
      \end{align*}
      This approximation is not terribly impressive: a hand held
      calculator shows that $\ln(2) \approx 0.693147$. Surprisingly
      enough, even the $20$th degree Taylor polynomial fails to
      approximate $\ln(x)$ for $x>2$. We'll soon discuss why this is.
    \end{hint}
  \end{question}
\end{question}


You may be wondering, \textit{how} exactly Taylor polynomials and
Maclaurin polynomials approximate these functions. Here's the idea:
Suppose you have two functions $f$ and $g$ which are each as 
differentiable as we need them to be. If for some specific value
$x=c$ we have that
\begin{align*}
  f(c) &= g(c)\\
  f'(c) &= g'(c)\\
  f''(c) &= g''(c)\\
  f'''(c) &= g'''(c)\\
          &\vdots
\end{align*}
then it makes sense that 
\[
f(x) = g(x)
\]
for all $x$. The more derivatives we use, the better 
our approximation (usually) is.  In that sense, we are just working with a
better version of linear approximation -- we could call this polynomial approximation!  
The Taylor and Maclaurin polynomials are ``cooked up'' so
that their value and the value of their derivatives equals the value
of the related function at $x=c$.  Check it out: here we see the third
Maclaurin polynomial for $f(x) = \sin(x)$:
\[
p_3(x) = x - \frac{x^3}{3!} 
\]
we see
\begin{align*}
  p_3(0) = 0 &= f(0) = \sin(0)\\
  p_3'(0) = 1 &= f'(1) = \cos(0)\\
  p_3''(0) = 0 &= f''(0) = -\sin(0)\\
  p_3'''(0) = -1 &= f'''(0) = -\cos(0)\\
  p_3^{(4)}(0) = 0 &= f^{(4)}(0) = \sin(0)\\
  p_3^{(5)}(0) = 0 &\ne f^{(5)}(0) = \cos(0)
\end{align*}
Note that in the case of sine,
\[
p_3(x) = x - \frac{x^3}{3!} 
\]
shares the function's value at $x=0$ and shares the first $4$
derivatives, though the $5$th derivative is different. Let's see a
graph to help us understand what is going on.
      \begin{image}
    \begin{tikzpicture}
      \begin{axis}[
          xmin=-6.28,xmax=6.28,ymin=-1.5,ymax=1.5,
          axis lines=center,
          %ticks=none,
          width=6in,
          height=3in,
          %unit vector ratio*=1 1 1,
          xlabel=$x$, ylabel=$y$,
          every axis y label/.style={at=(current axis.above origin),anchor=south},
          every axis x label/.style={at=(current axis.right of origin),anchor=west},
        ]        
        \addplot [very thick, penColor,smooth] {sin(deg(x))};
        \addplot [very thick, penColor2, smooth] {x-x^3/6};
      \end{axis}
    \end{tikzpicture}
      \end{image}
      We can see that $p_3$ is a good approximation for $\sin(x)$ near
      $x=0$. Next, we quantify exactly how good our approximation is.

\section{Taylor's theorem}

Again, let's get to the point by stating Taylor's theorem (which is a
generalization of the mean value theorem):

\begin{theorem}[Taylor's Theorem]\index{Taylor's Theorem}
Let $f$ be a function whose $(n+1)$th derivative exists on an interval
$I$, let $c$ be in $I$, and let $p_n$ be the $n$th Taylor polynomial
for $f$ centered at $x=c$. Then, for each $x$ in $I$, there exists $b$
between $x$ and $c$ such that
\[
f(x) = p_n(x) + R_n(x),
\]
where $R_n(x) = \frac{f^{(n+1)}(b)}{(n+1)!}(x-c)^{(n+1)}$.
\begin{image}
  \begin{tikzpicture}
    \begin{axis}[
        xmin=5.28,xmax=9.42,ymin=-.5,ymax=2.5,
        domain=5.28:9.42,
        axis lines=center,
        %ticks=none,
        xtick={6.28,6.73,8.75},
        xticklabels={$c$,$b$,},
        ytickmin=1, ytickmax=0,
        width=6in,
        height=3in,
        %unit vector ratio*=1 1 1,
        xlabel=$x$, ylabel=$y$,
        every axis y label/.style={at=(current axis.above origin),anchor=south},
        every axis x label/.style={at=(current axis.right of origin),anchor=west},
      ]        
      \addplot [very thick, penColor,smooth] {1+sin(deg(x))};
      \addplot [very thick, penColor2, smooth] {1+(x-2*pi)-(x-2*pi)^3/6};
      \addplot [textColor,dashed] plot coordinates {(6.28,0) (6.28,1)};
      \addplot [textColor,dashed] plot coordinates {(8.75,0) (8.75,1.63)};
      \addplot [decoration={brace,mirror, raise=.1cm},decorate,thin]
      plot coordinates {(8.75,.97) (8.75,1.63)};
      \node at (axis cs: 8.97,1.25) {$R_n(x)$};
      \node at (axis cs: 8.5,2) {$f$};
      \node at (axis cs: 8.5,1.2) {$p_n$};
      
    \end{axis}
  \end{tikzpicture}   
\end{image}
Moreover,
\[
\left|R_n(x)\right| \leq \frac{M}{(n+1)!}|x-c|^{(n+1)}
\]
where $M$ is the maximum value of $|f^{(n+1)}|$ on $[c,x]$.
\end{theorem}

The first part of Taylor's theorem states that $f(x) = p_n(x) +
R_n(x)$, where $p_n(x)$ is the $n$th order Taylor polynomial and
$R_n(x)$ is the remainder, or error, in the Taylor approximation. The
second part gives bounds on how big that error can be. If the
$(n+1)$th derivative is large, the error may be large; if $x$ is far
from $c$, the error may also be large. However, the $(n+1)!$ term in
the denominator tends to ensure that the error gets smaller as $n$
increases. 


\begin{example}
  Approximate $f(x) = e^x$ with a $5$th degree Maclaurin polynomial at
  $x = 0$.  What is the maximum error of your approximation at $x=2$?
  Note, the answer to this question may not be unique.
  \begin{explanation}
    To do this, we set $f(x) =e^x$ and use the following formula.
    \[
    p_n(x) = f(0) + f'(0)x +\frac{f''(0)}{2!}x^2+\cdots +\frac{f^{(5)}(0)}{5!}x^5
    \]
    So, we must compute the first five derivatives of $f$
    \[
      f(x) = f'(x) = f''(x) = f'''(x) = f^{(4)}(x)  = f^{(5)}(x) = \answer[given]{e^x}
    \]
    and evaluate each of these at $x=0$. Write with me.
    \[
    f(0) = f'(0) = f''(0) = f'''(0) = f^{(4)}(0)  = f^{(5)}(0) = e^0 = \answer[given]{1}
    \]
    Now, plugging this into our formula above we see
    \[
    p_5 = \answer[given]{1 + x + \frac{x^2}{2!} + \frac{x^3}{3!} + \frac{x^4}{4!} + \frac{x^5}{5!}}.
    \]
    Finally, let's see a graph of $p_5$ and $f$:
      \begin{image}
    \begin{tikzpicture}
      \begin{axis}[
          xmin=-5,xmax=4,ymin=-1,ymax=10,
          axis lines=center,
          %ticks=none,
          width=6in,
          height=3in,
          %unit vector ratio*=1 1 1,
          xlabel=$x$, ylabel=$y$,
          every axis y label/.style={at=(current axis.above origin),anchor=south},
          every axis x label/.style={at=(current axis.right of origin),anchor=west},
        ]        
        \addplot [very thick, penColor, samples=100,smooth] {e^x};
        \addplot [very thick, penColor2, smooth] {1+x+x^2/2+x^3/6+x^4/24+x^5/120};
      \end{axis}
    \end{tikzpicture}
      \end{image}
      To estimate the error, we need to find a bound $M$ such that
      \[
      M >|f^{(6)}(x)|
      \]
      on the interval $[0,2]$. Since $f^{(6)}(x) = e^x$, the largest
      value will be $e^2$. Instead of computing this directly, note
      \[
      e^2 < 3^2< 9.
      \]
      Hence we can say for certainty that
      \[
      |p_n(2) - e^2| < \frac{9}{6!}\cdot(2)^6 = \frac{4}{5}.
      \]
      In other words, we know that the maximum error we can have when using 
      $p_5(x)$ to approximate $f(2)$ is $\frac{4}{5} = 0.8$.
  \end{explanation}
\end{example}


The following example computes error estimates for the approximations
of $\ln(1.5)$ and $\ln(2)$ made earlier.

\begin{example}
Find error bounds when approximating $\ln(1.5)$ with $p_6(x)$, the
Taylor polynomial of degree $6$ of $f(x)=\ln(x)$ centered at $x=1$.
\begin{explanation}
We start with the approximation of $\ln(1.5)$ with $p_6(1.5)$. Taylor's
theorem references an open interval $I$ that contains both $x$ and
$c$. The smaller the interval we use the better; it will give us a
more accurate (and smaller!) approximation of the error. We let
\[
I =(0.9,1.6),
\]
as this interval contains both $c=\answer[given]{1}$ and
$x=\answer[given]{1.5}$.  Now we ask ``How big can the $7$th
derivative of $y=\ln(x)$ be on the interval $(0.9,1.6)$?'' The seventh
derivative is $\answer[given]{720/x^7}$. The largest value it attains on $I$ is about
$1506$. Thus we can bound the error as:
\begin{align*}
|R_6(1.5)| &\leq \frac{1506}{7!}|1.5-1|^7\\
&\leq \frac{1506}{5040}\cdot\frac1{2^7}\\
&\approx 0.0023.
\end{align*}
Remember, of course, that this bound on the error is only accurate on our 
interval $I$.
\end{explanation}
\end{example}

  
We practice again. This time, we use Taylor's theorem to find $n$ that
guarantees our approximation is within a certain amount.

\begin{example}
  Find $n$ such that the $n$th Taylor polynomial of $f(x)=\cos(x)$ at
  $x=0$ approximates $\cos(2)$ to within $0.001$ of the actual
  answer.
\begin{explanation}
  Following Taylor's theorem, we need bounds on the size of the
  derivatives of $f(x)=\cos(x)$. In the case of this trigonometric
  function, this is easy. All derivatives of cosine are $\pm \sin(x)$
  or $\pm \cos(x)$. In all cases, these functions are never greater
  than $\answer[given]{1}$ in absolute value. We want the error to be
  less than $0.001$. To find the appropriate $n$, consider the
  following inequality:
  \[
  \frac{1}{(n+1)!}\cdot2^{(n+1)} \leq 0.001
  \]
We find an $n$ that satisfies this last inequality with trial-and-error. When $n=8$, we have
\[
\answer[given]{\frac{2^{8+1}}{(8+1)!}} \approx 0.0014;
\]
when $n=9$, we have
\[
\answer[given]{\frac{2^{9+1}}{(9+1)!}} \approx 0.000282 <0.001.
\]
Thus we want to approximate $\cos(2)$ with $p_9(2)$ in order to have 
the desired accuracy.
\end{explanation}
\end{example}


\section{Connections to differential equations}

Our final example gives a brief introduction for using Taylor
polynomials to solve differential equations.

\begin{example}
  A function $y=f(x)$ is unknown save for the following two facts.
  \begin{itemize}
  \item	$y(0) = f(0) = 1$, and
  \item	$y'= y^2$.
  \end{itemize}
  Find the degree $3$ Maclaurin polynomial $p_3(x)$ of $y=f(x)$.
  \begin{explanation}
    One might initially think that not enough information is given to
    find $p_3(x)$. However, note how the second fact above actually
    lets us know what $y'(0)$ is:
    \[
    y' = y^2 \Rightarrow y'(0) = y^2(0).
    \]
    Since $y(0) = 1$, we conclude that $y'(0) = 1$.
    
    Now we find information about $y''$. Starting with $y'=y^2$, take
    derivatives of both sides, \emph{with respect to $x$}. That means
    we must use implicit differentiation.
    \begin{align*}
      y' &= \answer[given]{y^2}\\
      \ddx(y') &= \ddx(y^2)\\
      y'' &= \answer[given]{2y}\cdot y'.
    \end{align*}
    Now evaluate both sides at $x=0$:
    \begin{align*}
      y''(0) &= 2y(0)\cdot y'(0)\\
      y''(0) &= \answer[given]{2}
    \end{align*}
    We repeat this once more to find $y'''(0)$. We again use implicit
    differentiation; this time the Product Rule is also required.
    \begin{align*}
      \ddx(y'') &= \ddx (2yy')\\
      y''' &= 2y'\cdot y' + 2y\cdot y''.
    \end{align*}
    Now evaluate both sides at $x=0$:
    \begin{align*}
      y'''(0) &= 2y'(0)^2 + 2y(0)y''(0)\\
      y'''(0) &= \answer[given]{6}
    \end{align*}
    In summary, we have:
    \[
    y(0) = \answer[given]{1}, \qquad y'(0) = \answer[given]{1},  \qquad y''(0) = \answer[given]{2}, \qquad y'''(0) = \answer[given]{6}.
    \]
    We can now form $p_3(x)$:
    \[
      p_3(x) = \answer[given]{1+x+x^2+x^3}.
    \]
    It turns out that the differential equation we started with,
    $y'=y^2$, where $y(0)=1$, can be solved without too much
    difficulty: $y = \frac{1}{1-x}$. This makes sense in regard to
    the previous examples.
  \end{explanation}
\end{example}


Taylor polynomials are very useful approximation in two basic
situations:
\begin{enumerate}
\item When $f$ is known, but perhaps ``hard'' to compute directly. For
  instance, we can define $y=\cos(x)$ as either the ratio of sides of a
  right triangle (``adjacent over hypotenuse'') or with the unit
  circle. However, neither of these provides a convenient way of
  computing $\cos(2)$. A Taylor polynomial of sufficiently high degree
  can provide a reasonable method of computing such values using only
  operations usually hard-wired into a computer ($+$, $-$, $\times$
  and $\div$). However, even though Taylor polynomials \emph{could} be
  used in calculators and computers to calculate values of
  trigonometric functions, in practice they generally aren't. Other
  more efficient and accurate methods have been developed, such as the
  CORDIC algorithm.
\item When $f$ is not known, but information about its derivatives
  is known. This occurs more often than one might think, especially in
  the study of differential equations.
\end{enumerate}




\end{document}
