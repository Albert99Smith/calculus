\documentclass{ximera}

%\usepackage{todonotes}
%\usepackage{mathtools} %% Required for wide table Curl and Greens
%\usepackage{cuted} %% Required for wide table Curl and Greens
\newcommand{\todo}{}

\usepackage{esint} % for \oiint
\ifxake%%https://math.meta.stackexchange.com/questions/9973/how-do-you-render-a-closed-surface-double-integral
\renewcommand{\oiint}{{\large\bigcirc}\kern-1.56em\iint}
\fi


\graphicspath{
  {./}
  {ximeraTutorial/}
  {basicPhilosophy/}
  {functionsOfSeveralVariables/}
  {normalVectors/}
  {lagrangeMultipliers/}
  {vectorFields/}
  {greensTheorem/}
  {shapeOfThingsToCome/}
  {dotProducts/}
  {partialDerivativesAndTheGradientVector/}
  {../productAndQuotientRules/exercises/}
  {../normalVectors/exercisesParametricPlots/}
  {../continuityOfFunctionsOfSeveralVariables/exercises/}
  {../partialDerivativesAndTheGradientVector/exercises/}
  {../directionalDerivativeAndChainRule/exercises/}
  {../commonCoordinates/exercisesCylindricalCoordinates/}
  {../commonCoordinates/exercisesSphericalCoordinates/}
  {../greensTheorem/exercisesCurlAndLineIntegrals/}
  {../greensTheorem/exercisesDivergenceAndLineIntegrals/}
  {../shapeOfThingsToCome/exercisesDivergenceTheorem/}
  {../greensTheorem/}
  {../shapeOfThingsToCome/}
  {../separableDifferentialEquations/exercises/}
  {vectorFields/}
}

\newcommand{\mooculus}{\textsf{\textbf{MOOC}\textnormal{\textsf{ULUS}}}}

\usepackage{tkz-euclide}\usepackage{tikz}
\usepackage{tikz-cd}
\usetikzlibrary{arrows}
\tikzset{>=stealth,commutative diagrams/.cd,
  arrow style=tikz,diagrams={>=stealth}} %% cool arrow head
\tikzset{shorten <>/.style={ shorten >=#1, shorten <=#1 } } %% allows shorter vectors

\usetikzlibrary{backgrounds} %% for boxes around graphs
\usetikzlibrary{shapes,positioning}  %% Clouds and stars
\usetikzlibrary{matrix} %% for matrix
\usepgfplotslibrary{polar} %% for polar plots
\usepgfplotslibrary{fillbetween} %% to shade area between curves in TikZ
\usetkzobj{all}
\usepackage[makeroom]{cancel} %% for strike outs
%\usepackage{mathtools} %% for pretty underbrace % Breaks Ximera
%\usepackage{multicol}
\usepackage{pgffor} %% required for integral for loops



%% http://tex.stackexchange.com/questions/66490/drawing-a-tikz-arc-specifying-the-center
%% Draws beach ball
\tikzset{pics/carc/.style args={#1:#2:#3}{code={\draw[pic actions] (#1:#3) arc(#1:#2:#3);}}}



\usepackage{array}
\setlength{\extrarowheight}{+.1cm}
\newdimen\digitwidth
\settowidth\digitwidth{9}
\def\divrule#1#2{
\noalign{\moveright#1\digitwidth
\vbox{\hrule width#2\digitwidth}}}





\newcommand{\RR}{\mathbb R}
\newcommand{\R}{\mathbb R}
\newcommand{\N}{\mathbb N}
\newcommand{\Z}{\mathbb Z}

\newcommand{\sagemath}{\textsf{SageMath}}


%\renewcommand{\d}{\,d\!}
\renewcommand{\d}{\mathop{}\!d}
\newcommand{\dd}[2][]{\frac{\d #1}{\d #2}}
\newcommand{\pp}[2][]{\frac{\partial #1}{\partial #2}}
\renewcommand{\l}{\ell}
\newcommand{\ddx}{\frac{d}{\d x}}

\newcommand{\zeroOverZero}{\ensuremath{\boldsymbol{\tfrac{0}{0}}}}
\newcommand{\inftyOverInfty}{\ensuremath{\boldsymbol{\tfrac{\infty}{\infty}}}}
\newcommand{\zeroOverInfty}{\ensuremath{\boldsymbol{\tfrac{0}{\infty}}}}
\newcommand{\zeroTimesInfty}{\ensuremath{\small\boldsymbol{0\cdot \infty}}}
\newcommand{\inftyMinusInfty}{\ensuremath{\small\boldsymbol{\infty - \infty}}}
\newcommand{\oneToInfty}{\ensuremath{\boldsymbol{1^\infty}}}
\newcommand{\zeroToZero}{\ensuremath{\boldsymbol{0^0}}}
\newcommand{\inftyToZero}{\ensuremath{\boldsymbol{\infty^0}}}



\newcommand{\numOverZero}{\ensuremath{\boldsymbol{\tfrac{\#}{0}}}}
\newcommand{\dfn}{\textbf}
%\newcommand{\unit}{\,\mathrm}
\newcommand{\unit}{\mathop{}\!\mathrm}
\newcommand{\eval}[1]{\bigg[ #1 \bigg]}
\newcommand{\seq}[1]{\left( #1 \right)}
\renewcommand{\epsilon}{\varepsilon}
\renewcommand{\phi}{\varphi}


\renewcommand{\iff}{\Leftrightarrow}

\DeclareMathOperator{\arccot}{arccot}
\DeclareMathOperator{\arcsec}{arcsec}
\DeclareMathOperator{\arccsc}{arccsc}
\DeclareMathOperator{\si}{Si}
\DeclareMathOperator{\scal}{scal}
\DeclareMathOperator{\sign}{sign}


%% \newcommand{\tightoverset}[2]{% for arrow vec
%%   \mathop{#2}\limits^{\vbox to -.5ex{\kern-0.75ex\hbox{$#1$}\vss}}}
\newcommand{\arrowvec}[1]{{\overset{\rightharpoonup}{#1}}}
%\renewcommand{\vec}[1]{\arrowvec{\mathbf{#1}}}
\renewcommand{\vec}[1]{{\overset{\boldsymbol{\rightharpoonup}}{\mathbf{#1}}}\hspace{0in}}

\newcommand{\point}[1]{\left(#1\right)} %this allows \vector{ to be changed to \vector{ with a quick find and replace
\newcommand{\pt}[1]{\mathbf{#1}} %this allows \vec{ to be changed to \vec{ with a quick find and replace
\newcommand{\Lim}[2]{\lim_{\point{#1} \to \point{#2}}} %Bart, I changed this to point since I want to use it.  It runs through both of the exercise and exerciseE files in limits section, which is why it was in each document to start with.

\DeclareMathOperator{\proj}{\mathbf{proj}}
\newcommand{\veci}{{\boldsymbol{\hat{\imath}}}}
\newcommand{\vecj}{{\boldsymbol{\hat{\jmath}}}}
\newcommand{\veck}{{\boldsymbol{\hat{k}}}}
\newcommand{\vecl}{\vec{\boldsymbol{\l}}}
\newcommand{\uvec}[1]{\mathbf{\hat{#1}}}
\newcommand{\utan}{\mathbf{\hat{t}}}
\newcommand{\unormal}{\mathbf{\hat{n}}}
\newcommand{\ubinormal}{\mathbf{\hat{b}}}

\newcommand{\dotp}{\bullet}
\newcommand{\cross}{\boldsymbol\times}
\newcommand{\grad}{\boldsymbol\nabla}
\newcommand{\divergence}{\grad\dotp}
\newcommand{\curl}{\grad\cross}
%\DeclareMathOperator{\divergence}{divergence}
%\DeclareMathOperator{\curl}[1]{\grad\cross #1}
\newcommand{\lto}{\mathop{\longrightarrow\,}\limits}

\renewcommand{\bar}{\overline}

\colorlet{textColor}{black}
\colorlet{background}{white}
\colorlet{penColor}{blue!50!black} % Color of a curve in a plot
\colorlet{penColor2}{red!50!black}% Color of a curve in a plot
\colorlet{penColor3}{red!50!blue} % Color of a curve in a plot
\colorlet{penColor4}{green!50!black} % Color of a curve in a plot
\colorlet{penColor5}{orange!80!black} % Color of a curve in a plot
\colorlet{penColor6}{yellow!70!black} % Color of a curve in a plot
\colorlet{fill1}{penColor!20} % Color of fill in a plot
\colorlet{fill2}{penColor2!20} % Color of fill in a plot
\colorlet{fillp}{fill1} % Color of positive area
\colorlet{filln}{penColor2!20} % Color of negative area
\colorlet{fill3}{penColor3!20} % Fill
\colorlet{fill4}{penColor4!20} % Fill
\colorlet{fill5}{penColor5!20} % Fill
\colorlet{gridColor}{gray!50} % Color of grid in a plot

\newcommand{\surfaceColor}{violet}
\newcommand{\surfaceColorTwo}{redyellow}
\newcommand{\sliceColor}{greenyellow}




\pgfmathdeclarefunction{gauss}{2}{% gives gaussian
  \pgfmathparse{1/(#2*sqrt(2*pi))*exp(-((x-#1)^2)/(2*#2^2))}%
}


%%%%%%%%%%%%%
%% Vectors
%%%%%%%%%%%%%

%% Simple horiz vectors
\renewcommand{\vector}[1]{\left\langle #1\right\rangle}


%% %% Complex Horiz Vectors with angle brackets
%% \makeatletter
%% \renewcommand{\vector}[2][ , ]{\left\langle%
%%   \def\nextitem{\def\nextitem{#1}}%
%%   \@for \el:=#2\do{\nextitem\el}\right\rangle%
%% }
%% \makeatother

%% %% Vertical Vectors
%% \def\vector#1{\begin{bmatrix}\vecListA#1,,\end{bmatrix}}
%% \def\vecListA#1,{\if,#1,\else #1\cr \expandafter \vecListA \fi}

%%%%%%%%%%%%%
%% End of vectors
%%%%%%%%%%%%%

%\newcommand{\fullwidth}{}
%\newcommand{\normalwidth}{}



%% makes a snazzy t-chart for evaluating functions
%\newenvironment{tchart}{\rowcolors{2}{}{background!90!textColor}\array}{\endarray}

%%This is to help with formatting on future title pages.
\newenvironment{sectionOutcomes}{}{}



%% Flowchart stuff
%\tikzstyle{startstop} = [rectangle, rounded corners, minimum width=3cm, minimum height=1cm,text centered, draw=black]
%\tikzstyle{question} = [rectangle, minimum width=3cm, minimum height=1cm, text centered, draw=black]
%\tikzstyle{decision} = [trapezium, trapezium left angle=70, trapezium right angle=110, minimum width=3cm, minimum height=1cm, text centered, draw=black]
%\tikzstyle{question} = [rectangle, rounded corners, minimum width=3cm, minimum height=1cm,text centered, draw=black]
%\tikzstyle{process} = [rectangle, minimum width=3cm, minimum height=1cm, text centered, draw=black]
%\tikzstyle{decision} = [trapezium, trapezium left angle=70, trapezium right angle=110, minimum width=3cm, minimum height=1cm, text centered, draw=black]


\author{Jim Talamo}
\license{Creative Commons 3.0 By-bC}


\outcome{Explore the idea that higher order Taylor Polynomials are lower order ones plus correction terms}
\outcome{Introduce an aspect of qualitative reasoning to Taylor polynomials}
\outcome{Show that the quantitative and qualitative approaches are consistent}

\begin{document}
\begin{exercise}
This exercise examines qualitative relationships between a function and its Taylor polynomials and explores how the higher order Taylor polynomials naturally ``correct'' the lower order ones in the context of a specific example.

Consider the function $f(x)=e^{2x}$.  Give the first, second, and third degree Taylor polynomials centered at $x=0$ below:

\begin{align*}
p_1(x) &= \answer{1+2x} \\
p_2(x) &= \answer{1+2x+2x^2} \\
p_3(x) &= \answer{1+2x+2x^2+\frac{4}{3}x^3}
\end{align*}

Note that each higher order polynomial the next lower order one plus a new term.  For instance:

\begin{image}
  \begin{tikzpicture}
        \node at (0,0) {
          $p_3(x)= \underbrace{1+2x+2x^2}+\underbrace{\frac{4}{3}x^3}$};
        \node at (.2,-.5) {\small{This is $p_2(x)$}}; 
        %%
          \node at (3,-.7) {\small{This is a new term, which}};
          \node at (3,-1) {\small{we call a ``correction term".}};
      \end{tikzpicture}
  \end{image}

This exercise explores what that new ``correction term" does in the context of this specific example.

\begin{exercise}
First, let's look at the graph of $f(x)=e^{2x}$ and $p_1(x)=1+2x$.

\begin{image}
\begin{tikzpicture}

\begin{axis}
	[
	domain=-3:3, ymax=9,xmax=2.3, ymin=-4.5, xmin=-2.3,
	axis lines=center, xlabel=$x$, ylabel=$y$,
	xtick={-2,-1,1,2},
	ytick={-4,-2,2,4,6,8},
	every axis y label/.style={at=(current axis.above origin),anchor=south},
	every axis x label/.style={at=(current axis.right of origin),anchor=west},
	axis on top,
	typeset ticklabels with strut,
	]

	\addplot [draw=penColor,very thick, smooth] {exp(2*x)};
	\addplot [draw=penColor2,very thick, smooth] {1+2*x};
	
	\node at (axis cs:1.7,8) [penColor] {$f(x)=e^{2x}$};
	\node at (axis cs:1.3,1.2) [penColor2] {$p_1(x)=1+2x$};
\end{axis}

\end{tikzpicture}
\end{image}

By looking at the graphs, if we use $p_1(x)$ to approximate $f(x)$ at a nonzero $x$-value $x=x_0$, we expect:
\begin{multipleChoice}
\choice{$p_1(x_0)>f(x_0)$}
\choice[correct]{$p_1(x_0)<f(x_0)$}
\end{multipleChoice}

Indeed, if we pick $x_0 = 1$:

\begin{align*}
f(x_0)=f(1) = e^2 &= \answer[tolerance=.001]{7.389} \textrm{ to $3$ decimal places}\\
p_1(x_0) = p_1(1) &= \answer{3}
\end{align*}

\end{exercise}

\begin{exercise}
Since $p_1(x)$ is always small than $e^{2x}$ when $x \neq 0$, the second degree Taylor polynomial will try to bridge the gap between the lower order polynomial $p_1(x)$ and $e^{2x}$.  

First, note that:

\begin{image}
  \begin{tikzpicture}
        \node at (0,0) {
          $p_2(x)= \underbrace{1+2x}+2x^2=p_1(x)+$ ``correction term''};
        \node at (-2,-.5) {\small{This is $p_1(x)$}}; 
              \end{tikzpicture}
  \end{image}

Since we established that the first degree Taylor polynomial is an underestimate for the function, this new ``correction term'':

\begin{multipleChoice}
\choice[correct]{should be positive for all $x$.}
\choice{should be positive for $x>0$ and negative for $x<0$.}
\choice{should be negative for $x>0$ and positive for $x<0$.}
\choice{should be negative for all $x$.}
\end{multipleChoice}

Now, from the second degree Taylor polynomial, the actual ``correction term" is $\answer{2x^2}$.  Notice that:
\begin{multipleChoice}
\choice[correct]{$2x^2$ is positive for all $x$.}
\choice{$2x^2$ is  positive for $x>0$ and negative for $x<0$.}
\choice{$2x^2$ is  negative for $x>0$ and positive for $x<0$.}
\choice{$2x^2$ is negative for all $x$.}
\end{multipleChoice}

Does this quantitative result agree with the previous qualitative result that precedes it?
\begin{multipleChoice}
\choice[correct]{Yes}
\choice{No}
\end{multipleChoice}
\end{exercise}

%%%%%%%%%%%%%%%%%
\begin{exercise}
Now, let's look at the graphs of $y=e^{2x}$ and $y=p_2(x)$:

\begin{image}
\begin{tikzpicture}

\begin{axis}
	[
	domain=-3:3, ymax=9,xmax=2.3, ymin=-.5, xmin=-2.3,
	axis lines=center, xlabel=$x$, ylabel=$y$,
	xtick={-2,-1,1,2},
	ytick={-4,-2,2,4,6,8},
	every axis y label/.style={at=(current axis.above origin),anchor=south},
	every axis x label/.style={at=(current axis.right of origin),anchor=west},
	axis on top,
	typeset ticklabels with strut,
	]

	\addplot [draw=penColor,very thick, smooth] {exp(2*x)};
	\addplot [draw=penColor2,very thick, smooth] {1+2*x+2*x^2};
	
	\node at (axis cs:.5,8) [penColor] {$y=e^{2x}$};
	\node at (axis cs:1.6,4) [penColor2] {$y=p_2(x)$};
\end{axis}

\end{tikzpicture}
\end{image}

By simply looking at the graphs above (i.e. WITHOUT examining the third degree polynomial for $e^{2x}$ centered at $x=0$), the ``correction term'' for the third degree Taylor polynomial:

\begin{multipleChoice}
\choice{should be positive for all $x$.}
\choice[correct]{should be positive for $x>0$ and negative for $x<0$.}
\choice{should be negative for $x>0$ and positive for $x<0$.}
\choice{should be negative for all $x$.}
\end{multipleChoice}

Now, from the third degree Taylor polynomial, the actual correction term is $\answer{\frac{4}{3}x^3}$.  Notice that this ``correction term'':
\begin{multipleChoice}
\choice{is positive for all $x$.}
\choice[correct]{is  positive for $x>0$ and negative for $x<0$.}
\choice{is  negative for $x>0$ and positive for $x<0$.}
\choice{ is negative for all $x$.}
\end{multipleChoice}

Does this quantitative result agree with the previous qualitative result that precedes it?
\begin{multipleChoice}
\choice[correct]{Yes}
\choice{No}
\end{multipleChoice}

%%%%%%%%%%%%%%%%%
\begin{exercise}
Now, let's look at the graphs of $y=e^{2x}$ and $y=p_3(x)$:

\begin{image}
\begin{tikzpicture}

\begin{axis}
	[
	domain=-3:3, ymax=9,xmax=2.3, ymin=-4.5, xmin=-2.3,
	axis lines=center, xlabel=$x$, ylabel=$y$,
	xtick={-2,-1,1,2},
	ytick={-4,-2,2,4,6,8},
	every axis y label/.style={at=(current axis.above origin),anchor=south},
	every axis x label/.style={at=(current axis.right of origin),anchor=west},
	axis on top,
	typeset ticklabels with strut,
	]

	\addplot [draw=penColor,very thick, smooth] {exp(2*x)};
	\addplot [draw=penColor2,very thick, smooth] {1+2*x+2*x^2+4/3*x^3};
	
	\node at (axis cs:.5,8) [penColor] {$y=e^{2x}$};
	\node at (axis cs:1.6,5) [penColor2] {$y=p_3(x)$};
\end{axis}

\end{tikzpicture}
\end{image}

By simply looking at the graphs above (i.e. WITHOUT examining the third degree polynomial for $e^{2x}$ centered at $x=0$), the ``correction term'' for the fourth degree Taylor polynomial:

\begin{multipleChoice}
\choice[correct]{should be positive for all $x$.}
\choice{should be positive for $x>0$ and negative for $x<0$.}
\choice{should be negative for $x>0$ and positive for $x<0$.}
\choice{should be negative for all $x$.}
\end{multipleChoice}

The actual fourth degree Taylor polynomial centered at $x=0$ for $f(x)=e^{2x}$ is:

\[
p_4(x) = 1+2x+2x^2+\frac{4}{3}x^3+\frac{2}{3}x^4
\]

so the actual correction term is $\answer{\frac{2}{3}x^4}$, which:

\begin{multipleChoice}
\choice[correct]{is positive for all $x$.}
\choice{is positive for $x>0$ and negative for $x<0$.}
\choice{is negative for $x>0$ and positive for $x<0$.}
\choice{ is negative for all $x$.}
\end{multipleChoice}

Does this quantitative result agree with the previous qualitative result that precedes it?
\begin{multipleChoice}
\choice[correct]{Yes}
\choice{No}
\end{multipleChoice}

We now make an important observation:

\begin{remark}
Note that each of the Taylor polynomials ``overcorrects'' the previous ones; indeed no higher order Taylor polynomial will agree with $e^{2x}$ for all $x$-values since $e^{2x}$ is not a polynomial.  Thus, each higher order Taylor polynomial will try to ``fix'' the overcorrection from the previous one, with the hope that the discrepancies become smaller as higher order Taylor polynomials are used.
\end{remark}

\end{exercise}
\end{exercise}
\end{exercise}
\end{document}
