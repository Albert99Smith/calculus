\documentclass{ximera}

%\usepackage{todonotes}
%\usepackage{mathtools} %% Required for wide table Curl and Greens
%\usepackage{cuted} %% Required for wide table Curl and Greens
\newcommand{\todo}{}

\usepackage{esint} % for \oiint
\ifxake%%https://math.meta.stackexchange.com/questions/9973/how-do-you-render-a-closed-surface-double-integral
\renewcommand{\oiint}{{\large\bigcirc}\kern-1.56em\iint}
\fi


\graphicspath{
  {./}
  {ximeraTutorial/}
  {basicPhilosophy/}
  {functionsOfSeveralVariables/}
  {normalVectors/}
  {lagrangeMultipliers/}
  {vectorFields/}
  {greensTheorem/}
  {shapeOfThingsToCome/}
  {dotProducts/}
  {partialDerivativesAndTheGradientVector/}
  {../productAndQuotientRules/exercises/}
  {../normalVectors/exercisesParametricPlots/}
  {../continuityOfFunctionsOfSeveralVariables/exercises/}
  {../partialDerivativesAndTheGradientVector/exercises/}
  {../directionalDerivativeAndChainRule/exercises/}
  {../commonCoordinates/exercisesCylindricalCoordinates/}
  {../commonCoordinates/exercisesSphericalCoordinates/}
  {../greensTheorem/exercisesCurlAndLineIntegrals/}
  {../greensTheorem/exercisesDivergenceAndLineIntegrals/}
  {../shapeOfThingsToCome/exercisesDivergenceTheorem/}
  {../greensTheorem/}
  {../shapeOfThingsToCome/}
  {../separableDifferentialEquations/exercises/}
  {vectorFields/}
}

\newcommand{\mooculus}{\textsf{\textbf{MOOC}\textnormal{\textsf{ULUS}}}}

\usepackage{tkz-euclide}\usepackage{tikz}
\usepackage{tikz-cd}
\usetikzlibrary{arrows}
\tikzset{>=stealth,commutative diagrams/.cd,
  arrow style=tikz,diagrams={>=stealth}} %% cool arrow head
\tikzset{shorten <>/.style={ shorten >=#1, shorten <=#1 } } %% allows shorter vectors

\usetikzlibrary{backgrounds} %% for boxes around graphs
\usetikzlibrary{shapes,positioning}  %% Clouds and stars
\usetikzlibrary{matrix} %% for matrix
\usepgfplotslibrary{polar} %% for polar plots
\usepgfplotslibrary{fillbetween} %% to shade area between curves in TikZ
\usetkzobj{all}
\usepackage[makeroom]{cancel} %% for strike outs
%\usepackage{mathtools} %% for pretty underbrace % Breaks Ximera
%\usepackage{multicol}
\usepackage{pgffor} %% required for integral for loops



%% http://tex.stackexchange.com/questions/66490/drawing-a-tikz-arc-specifying-the-center
%% Draws beach ball
\tikzset{pics/carc/.style args={#1:#2:#3}{code={\draw[pic actions] (#1:#3) arc(#1:#2:#3);}}}



\usepackage{array}
\setlength{\extrarowheight}{+.1cm}
\newdimen\digitwidth
\settowidth\digitwidth{9}
\def\divrule#1#2{
\noalign{\moveright#1\digitwidth
\vbox{\hrule width#2\digitwidth}}}





\newcommand{\RR}{\mathbb R}
\newcommand{\R}{\mathbb R}
\newcommand{\N}{\mathbb N}
\newcommand{\Z}{\mathbb Z}

\newcommand{\sagemath}{\textsf{SageMath}}


%\renewcommand{\d}{\,d\!}
\renewcommand{\d}{\mathop{}\!d}
\newcommand{\dd}[2][]{\frac{\d #1}{\d #2}}
\newcommand{\pp}[2][]{\frac{\partial #1}{\partial #2}}
\renewcommand{\l}{\ell}
\newcommand{\ddx}{\frac{d}{\d x}}

\newcommand{\zeroOverZero}{\ensuremath{\boldsymbol{\tfrac{0}{0}}}}
\newcommand{\inftyOverInfty}{\ensuremath{\boldsymbol{\tfrac{\infty}{\infty}}}}
\newcommand{\zeroOverInfty}{\ensuremath{\boldsymbol{\tfrac{0}{\infty}}}}
\newcommand{\zeroTimesInfty}{\ensuremath{\small\boldsymbol{0\cdot \infty}}}
\newcommand{\inftyMinusInfty}{\ensuremath{\small\boldsymbol{\infty - \infty}}}
\newcommand{\oneToInfty}{\ensuremath{\boldsymbol{1^\infty}}}
\newcommand{\zeroToZero}{\ensuremath{\boldsymbol{0^0}}}
\newcommand{\inftyToZero}{\ensuremath{\boldsymbol{\infty^0}}}



\newcommand{\numOverZero}{\ensuremath{\boldsymbol{\tfrac{\#}{0}}}}
\newcommand{\dfn}{\textbf}
%\newcommand{\unit}{\,\mathrm}
\newcommand{\unit}{\mathop{}\!\mathrm}
\newcommand{\eval}[1]{\bigg[ #1 \bigg]}
\newcommand{\seq}[1]{\left( #1 \right)}
\renewcommand{\epsilon}{\varepsilon}
\renewcommand{\phi}{\varphi}


\renewcommand{\iff}{\Leftrightarrow}

\DeclareMathOperator{\arccot}{arccot}
\DeclareMathOperator{\arcsec}{arcsec}
\DeclareMathOperator{\arccsc}{arccsc}
\DeclareMathOperator{\si}{Si}
\DeclareMathOperator{\scal}{scal}
\DeclareMathOperator{\sign}{sign}


%% \newcommand{\tightoverset}[2]{% for arrow vec
%%   \mathop{#2}\limits^{\vbox to -.5ex{\kern-0.75ex\hbox{$#1$}\vss}}}
\newcommand{\arrowvec}[1]{{\overset{\rightharpoonup}{#1}}}
%\renewcommand{\vec}[1]{\arrowvec{\mathbf{#1}}}
\renewcommand{\vec}[1]{{\overset{\boldsymbol{\rightharpoonup}}{\mathbf{#1}}}\hspace{0in}}

\newcommand{\point}[1]{\left(#1\right)} %this allows \vector{ to be changed to \vector{ with a quick find and replace
\newcommand{\pt}[1]{\mathbf{#1}} %this allows \vec{ to be changed to \vec{ with a quick find and replace
\newcommand{\Lim}[2]{\lim_{\point{#1} \to \point{#2}}} %Bart, I changed this to point since I want to use it.  It runs through both of the exercise and exerciseE files in limits section, which is why it was in each document to start with.

\DeclareMathOperator{\proj}{\mathbf{proj}}
\newcommand{\veci}{{\boldsymbol{\hat{\imath}}}}
\newcommand{\vecj}{{\boldsymbol{\hat{\jmath}}}}
\newcommand{\veck}{{\boldsymbol{\hat{k}}}}
\newcommand{\vecl}{\vec{\boldsymbol{\l}}}
\newcommand{\uvec}[1]{\mathbf{\hat{#1}}}
\newcommand{\utan}{\mathbf{\hat{t}}}
\newcommand{\unormal}{\mathbf{\hat{n}}}
\newcommand{\ubinormal}{\mathbf{\hat{b}}}

\newcommand{\dotp}{\bullet}
\newcommand{\cross}{\boldsymbol\times}
\newcommand{\grad}{\boldsymbol\nabla}
\newcommand{\divergence}{\grad\dotp}
\newcommand{\curl}{\grad\cross}
%\DeclareMathOperator{\divergence}{divergence}
%\DeclareMathOperator{\curl}[1]{\grad\cross #1}
\newcommand{\lto}{\mathop{\longrightarrow\,}\limits}

\renewcommand{\bar}{\overline}

\colorlet{textColor}{black}
\colorlet{background}{white}
\colorlet{penColor}{blue!50!black} % Color of a curve in a plot
\colorlet{penColor2}{red!50!black}% Color of a curve in a plot
\colorlet{penColor3}{red!50!blue} % Color of a curve in a plot
\colorlet{penColor4}{green!50!black} % Color of a curve in a plot
\colorlet{penColor5}{orange!80!black} % Color of a curve in a plot
\colorlet{penColor6}{yellow!70!black} % Color of a curve in a plot
\colorlet{fill1}{penColor!20} % Color of fill in a plot
\colorlet{fill2}{penColor2!20} % Color of fill in a plot
\colorlet{fillp}{fill1} % Color of positive area
\colorlet{filln}{penColor2!20} % Color of negative area
\colorlet{fill3}{penColor3!20} % Fill
\colorlet{fill4}{penColor4!20} % Fill
\colorlet{fill5}{penColor5!20} % Fill
\colorlet{gridColor}{gray!50} % Color of grid in a plot

\newcommand{\surfaceColor}{violet}
\newcommand{\surfaceColorTwo}{redyellow}
\newcommand{\sliceColor}{greenyellow}




\pgfmathdeclarefunction{gauss}{2}{% gives gaussian
  \pgfmathparse{1/(#2*sqrt(2*pi))*exp(-((x-#1)^2)/(2*#2^2))}%
}


%%%%%%%%%%%%%
%% Vectors
%%%%%%%%%%%%%

%% Simple horiz vectors
\renewcommand{\vector}[1]{\left\langle #1\right\rangle}


%% %% Complex Horiz Vectors with angle brackets
%% \makeatletter
%% \renewcommand{\vector}[2][ , ]{\left\langle%
%%   \def\nextitem{\def\nextitem{#1}}%
%%   \@for \el:=#2\do{\nextitem\el}\right\rangle%
%% }
%% \makeatother

%% %% Vertical Vectors
%% \def\vector#1{\begin{bmatrix}\vecListA#1,,\end{bmatrix}}
%% \def\vecListA#1,{\if,#1,\else #1\cr \expandafter \vecListA \fi}

%%%%%%%%%%%%%
%% End of vectors
%%%%%%%%%%%%%

%\newcommand{\fullwidth}{}
%\newcommand{\normalwidth}{}



%% makes a snazzy t-chart for evaluating functions
%\newenvironment{tchart}{\rowcolors{2}{}{background!90!textColor}\array}{\endarray}

%%This is to help with formatting on future title pages.
\newenvironment{sectionOutcomes}{}{}



%% Flowchart stuff
%\tikzstyle{startstop} = [rectangle, rounded corners, minimum width=3cm, minimum height=1cm,text centered, draw=black]
%\tikzstyle{question} = [rectangle, minimum width=3cm, minimum height=1cm, text centered, draw=black]
%\tikzstyle{decision} = [trapezium, trapezium left angle=70, trapezium right angle=110, minimum width=3cm, minimum height=1cm, text centered, draw=black]
%\tikzstyle{question} = [rectangle, rounded corners, minimum width=3cm, minimum height=1cm,text centered, draw=black]
%\tikzstyle{process} = [rectangle, minimum width=3cm, minimum height=1cm, text centered, draw=black]
%\tikzstyle{decision} = [trapezium, trapezium left angle=70, trapezium right angle=110, minimum width=3cm, minimum height=1cm, text centered, draw=black]


\title[Dig-in:]{Slopes of tangent lines via limits} %% I don't know about this title

\begin{document}
\begin{abstract}
We compute the derivative by computing the limits of average growth %% I don't know about this absract
rates.
\end{abstract}
\maketitle


Suppose that $f(x)$ is a function.  It is often useful to know the
rate at which $f(x)$ changes. To give you a feeling why this is true,
consider the following:
\begin{itemize}
\item If $s(t)$ represents the displacement (position relative to an
  origin) of an object with respect to time, the rate of change gives
  the velocity of the object.
\item If $v(t)$ represents the velocity of an object with respect to
  time, the rate of change gives the acceleration of the object.
\item If $R(x)$ represents the revenue generated by selling $x$
  objects, the rate of change gives us the \textit{marginal revenue},
  meaning the additional revenue generated by selling one additional
  unit. Note, there is an implicit assumption that $x$ is quite large
  compared to $1$.
\item If $C(x)$ represents the cost to produce $x$ objects, the rate
  of change gives us the \textit{marginal cost}, meaning the
  additional cost generated by selling one additional unit. Again,
  there is an implicit assumption that $x$ is quite large compared to
  $1$.
\item If $P(x)$ represents the profit gained by selling $x$ objects,
  the rate of change gives us the \textit{marginal profit}, meaning
  the additional cost generated by selling one additional unit. Again,
  there is an implicit assumption that $x$ is quite large compared to
  $1$.
\item The rate of change of a function can help us approximate a
  complicated function with a simple function.
\item The rate of change of a function can be used to help us solve
  equations that we would not be able to solve via other methods.
\end{itemize}

\begin{xarmaBoost}
What other examples can you find where one is interested in a ``rate
of change?''
\begin{freeResponse}
\end{freeResponse}
\end{xarmaBoost}

%% This next part feels to me like I've driving the so-called
%% MOOCulus-Mobile over a curb! Should it be so abrupt? Or is it good
%% as is?


You've been computing average rates of change for a while now, the
computation is simply
\[
\frac{\text{change in the function}}{\text{change in the input to the
    function}}.
\]
However, the question remains: Given a function that represents an
amount, how exactly does one find the function that will give the
instantaneous rate of change? Recall that the instantaneous rate of change
of a line is the slope of the line.  Hence the instantaneous rate of
change of a function is the slope of the tangent line. For now,
consider the following informal definition of a \textit{tangent line}:
\begin{quote}\index{tangent line}
Given a function $f(x)$, if one can ``zoom in''
on $f(x)$ sufficiently so that $f(x)$ seems to be a straight line,
then that line is the \dfn{tangent line} to $f(x)$ at the point
determined by $x$.
\end{quote}
We illustrate this informal definition with the following diagram:
\begin{image}
\begin{tikzpicture}
  \colorlet{penColor}{blue!50!black}
  \colorlet{penColor2}{red!50!black}
  \colorlet{textColor}{black}
  \colorlet{background}{white}
	\begin{axis}[
            domain=0:6, range=0:7,
            ymin=-.2,ymax=7,
            width=\textwidth,
            height=7cm, %% Hard coded height! Moreover this effects the aspect ratio of the zoom--sort of BAD
            axis lines=none,
          ]   
          \addplot [draw=none, fill=textColor!10!background] plot coordinates {(.8,1.6) (2.834,5)} \closedcycle; %% zoom fill
          \addplot [draw=none, fill=textColor!10!background] plot coordinates {(2.834,5) (4.166,5)} \closedcycle; %% zoom fill
          \addplot [draw=none, fill=background] plot coordinates {(1.2,1.6) (4.166,5)} \closedcycle; %% zoom fill
          \addplot [draw=none, fill=background] plot coordinates {(.8,1.6) (1.2,1.6)} \closedcycle; %% zoom fill

          \addplot [draw=none, fill=textColor!10!background] plot coordinates {(3.3,3.6) (5.334,5)} \closedcycle; %% zoom fill
          \addplot [draw=none, fill=textColor!10!background] plot coordinates {(5.334,5) (6.666,5)} \closedcycle; %% zoom fill
          \addplot [draw=none, fill=background] plot coordinates {(3.7,3.6) (6.666,5)} \closedcycle; %% zoom fill
          \addplot [draw=none, fill=background] plot coordinates {(3.3,3.6) (3.7,3.6)} \closedcycle; %% zoom fill
          
          \addplot [draw=none, fill=textColor!10!background] plot coordinates {(3.7,2.4) (6.666,1)} \closedcycle; %% zoom fill
          \addplot [draw=none, fill=textColor!10!background] plot coordinates {(3.3,2.4) (3.7,2.4)} \closedcycle; %% zoom fill
          \addplot [draw=none, fill=background] plot coordinates {(3.3,2.4) (5.334,1)} \closedcycle; %% zoom fill          
          \addplot [draw=none, fill=background] plot coordinates {(5.334,1) (6.666,1)} \closedcycle; %% zoom fill
          

          \addplot [draw=none, fill=textColor!10!background] plot coordinates {(.8,.4) (2.834,1)} \closedcycle; %% zoom fill
          \addplot [draw=none, fill=textColor!10!background] plot coordinates {(2.834,1) (4.166,1)} \closedcycle; %% zoom fill
          \addplot [draw=none, fill=background] plot coordinates {(1.2,.4) (4.166,1)} \closedcycle; %% zoom fill
          \addplot [draw=none, fill=background] plot coordinates {(.8,.4) (1.2,.4)} \closedcycle; %% zoom fill

          \addplot[very thick,penColor, smooth,domain=(0:1.833)] {-1/(x-2)};
          \addplot[very thick,penColor, smooth,domain=(2.834:4.166)] {3.333/(2.050-.3*x)-0.333}; %% 2.5 to 4.333
          %\addplot[very thick,penColor, smooth,domain=(5.334:6.666)] {11.11/(1.540-.09*x)-8.109}; %% 5 to 6.833
          \addplot[very thick,penColor, smooth,domain=(5.334:6.666)] {x-3}; %% 5 to 6.833
          
          \addplot[color=penColor,fill=penColor,only marks,mark=*] coordinates{(1,1)};  %% point to be zoomed
          \addplot[color=penColor,fill=penColor,only marks,mark=*] coordinates{(3.5,3)};  %% zoomed pt 1
          \addplot[color=penColor,fill=penColor,only marks,mark=*] coordinates{(6,3)};  %% zoomed pt 2

          \addplot [->,textColor] plot coordinates {(0,0) (0,6)}; %% axis
          \addplot [->,textColor] plot coordinates {(0,0) (2,0)}; %% axis
          
          \addplot [textColor!50!background] plot coordinates {(.8,.4) (.8,1.6)}; %% box around pt
          \addplot [textColor!50!background] plot coordinates {(1.2,.4) (1.2,1.6)}; %% box around pt
          \addplot [textColor!50!background] plot coordinates {(.8,1.6) (1.2,1.6)}; %% box around pt
          \addplot [textColor!50!background] plot coordinates {(.8,.4) (1.2,.4)}; %% box around pt
          
          \addplot [textColor!50!background] plot coordinates {(2.834,1) (2.834,5)}; %% zoomed box 1
          \addplot [textColor!50!background] plot coordinates {(4.166,1) (4.166,5)}; %% zoomed box 1
          \addplot [textColor!50!background] plot coordinates {(2.834,1) (4.166,1)}; %% zoomed box 1
          \addplot [textColor!50!background] plot coordinates {(2.834,5) (4.166,5)}; %% zoomed box 1

          \addplot [textColor] plot coordinates {(3.3,2.4) (3.3,3.6)}; %% box around zoomed pt
          \addplot [textColor] plot coordinates {(3.7,2.4) (3.7,3.6)}; %% box around zoomed pt
          \addplot [textColor] plot coordinates {(3.3,3.6) (3.7,3.6)}; %% box around zoomed pt
          \addplot [textColor] plot coordinates {(3.3,2.4) (3.7,2.4)}; %% box around zoomed pt

          \addplot [textColor] plot coordinates {(5.334,1) (5.334,5)}; %% zoomed box 2
          \addplot [textColor] plot coordinates {(6.666,1) (6.666,5)}; %% zoomed box 2
          \addplot [textColor] plot coordinates {(5.334,1) (6.666,1)}; %% zoomed box 2
          \addplot [textColor] plot coordinates {(5.334,5) (6.666,5)}; %% zoomed box 2

          \node at (axis cs:2.2,0) [anchor=east] {$x$};
          \node at (axis cs:0,6.6) [anchor=north] {$y$};
        \end{axis}
\end{tikzpicture}
%% \caption{Given a function $f(x)$, if one can ``zoom in''
%% on $f(x)$ sufficiently so that $f(x)$ seems to be a straight line,
%% then that line is the \textbf{tangent line} to $f(x)$ at the point
%% determined by $x$.}
%% \label{figure:informal-tangent}
\end{image}



The \textit{derivative} of a function $f(x)$ at $x$, is the instantaneous
rate of change, and hence is the slope of the tangent line at $x$. To
find the slope of this line, we consider \textit{secant} lines, lines
that locally intersect the curve at two points.  The slope of any
secant line that passes through the points $(x,f(x))$ and $(x+h,
f(x+h))$ is given by
\[
\frac{\Delta y}{\Delta x}=\frac{f(x+h) -f(x)}{(x+h)-x} =
\frac{f(x+h)-f(x)}{h}.
\]
%see Figure~\ref{figure:limit-dfn}. 
This leads to the \textit{limit definition of the derivative}:

%\begin{definitionOfTheDerivative}\index{limit!definition of the derivative}\index{derivative!limit definition}
\begin{definition}
The \dfn{derivative} of $f(x)$ is the function
\[
\ddx f(x) = \lim_{h\to 0} \frac{f(x+h) - f(x)}{h}.
\]
If this limit does not exist for a given value of $x$, then $f(x)$ is
not \dfn{differentiable} at $x$.
\end{definition}
%\end{definitionOfTheDerivative}
%\begin{marginfigure}[-1.75in]
\begin{image}
\begin{tikzpicture}
  \colorlet{penColor}{blue!50!black}
  \colorlet{penColor2}{red!50!black}
  \colorlet{textColor}{black}
  \colorlet{background}{white}  
	\begin{axis}[
            domain=0:2, range=0:6,ymax=6,ymin=0,
            axis lines =left, xlabel=$x$, ylabel=$y$,
            every axis y label/.style={at=(current axis.above origin),anchor=south},
            every axis x label/.style={at=(current axis.right of origin),anchor=west},
            xtick={1,1.666}, ytick={1,3},
            xticklabels={$x$,$x+h$}, yticklabels={$f(x)$,$f(x+h)$},
            axis on top,
          ]         
          \addplot [penColor2!15!background, domain=(0:2)] {-3.348+4.348*x};
          \addplot [penColor2!32!background, domain=(0:2)] {-2.704+3.704*x};
          \addplot [penColor2!49!background, domain=(0:2)] {-1.994+2.994*x};         
          \addplot [penColor2!66!background, domain=(0:2)] {-1.326+2.326*x}; 
          \addplot [penColor2!83!background, domain=(0:2)] {-0.666+1.666*x};
	  \addplot [textColor,dashed] plot coordinates {(1,0) (1,1)};
          \addplot [textColor,dashed] plot coordinates {(0,1) (1,1)};
          \addplot [textColor,dashed] plot coordinates {(0,3) (1.666,3)};
          \addplot [textColor,dashed] plot coordinates {(1.666,0) (1.666,3)};
          \addplot [very thick,penColor, smooth,domain=(0:1.833)] {-1/(x-2)};
          \addplot[color=penColor,fill=penColor,only marks,mark=*] coordinates{(1.666,3)};  %% closed hole          
          \addplot[color=penColor,fill=penColor,only marks,mark=*] coordinates{(1,1)};  %% closed hole          
          \addplot [very thick,penColor2, smooth,domain=(0:2)] {x};
        \end{axis}
\end{tikzpicture}
\end{image}
%% \caption{Tangent lines can be found as the limit of secant lines. The slope of the tangent line is given by
%% $\lim_{h\to 0} \frac{f(x+h) - f(x)}{h}.$}
%%  \label{figure:limit-dfn}
%% \end{marginfigure}

%\break



\begin{question} 

%% How do we write this question? I want to get both standard dfns of
%% the derivative out here, along with negations of the numerator and
%% denominator. How is a good questions constructed here?

Consider the following limits:
\begin{enumerate}
\item $\lim_{x\to a} \frac{f(x)-f(a)}{x-a}$\\
\item $\lim_{x\to a} \frac{f(a)-f(x)}{a-x}$\\
\item FIX THIS
\end{enumerate}
Can you explain why each of the limits above are also equivalent
definitions of the derivative?
\end{question}

\begin{definition}\index{derivative!notation}
There are several different notations for the derivative, we'll mainly
use
\[
\ddx f(x) = f'(x).
\]
If one is working with a function of a variable other than $x$, say $t$ we write
\[
\dd{t} f(t) = f'(t).
\]
However, if $y = f(x)$, $\dd[y]{x}$, $\dot{y}$, and $D_x f(x)$ are
also used.
\end{definition}

Now we will give a number of examples, starting with a basic example.

\begin{example}
Compute 
\[
\ddx (x^3 + 1).
\]
Start by writing out the limit definition of the derivative where
$f(x) = x^3+1$.
\begin{hint}
\[
\ddx f(x) = \lim_{h\to 0}\frac{(x+h)^3 + 1 - (x^3 +1)}{h}
\]
\end{hint}
Now expand the numerator of the fraction.
\begin{hint}
\[
\ddx f(x) = \lim_{h\to 0}\frac{x^3+3x^2h+3xh^2 + h^3 + 1 - x^3 -1}{h}
\]
\end{hint}
Now combine like-terms.
\begin{hint}
\[
\ddx f(x) = \lim_{h\to 0}\frac{3x^2h+3xh^2 + h^3}{h}
\]
\end{hint}
Factor an $h$ from every term in the numerator.
\begin{hint}
\[
\ddx f(x) = \lim_{h\to 0}\frac{h(3x^2+3xh + h^2)}{h}
\]
\end{hint}
Cancel $h$ from the numerator and denominator.
\begin{hint}
\[
\ddx f(x) = \lim_{h\to 0} \left(3x^2+3xh + h^2\right)
\]
\end{hint}
Take the limit as $h$ goes to $0$. 
\begin{hint}
\[
\ddx f(x) = 3x^2
\]
\end{hint}
%% Using the definition of the derivative,
%% \begin{align*}
%% \ddx f(x) &= \lim_{h\to 0}\frac{(x+h)^3 + 1 - (x^3 +1)}{h}\\
%% &= \lim_{h\to 0}\frac{x^3+3x^2h+3xh^2 + h^3 + 1 - x^3 -1}{h}\\
%% &= \lim_{h\to 0}\frac{3x^2h+3xh^2 + h^3}{h}\\
%% &= \lim_{h\to 0}(3x^2+3xh + h^2)\\
%% &= 3x^2.
%% \end{align*}
For your viewing pleasure, we have supplied a plot of both $f(x)$ and
$f'(x)$:
\begin{image}
\begin{tikzpicture}
  \colorlet{background}{white}
  \colorlet{textColor}{black}
  \colorlet{penColor}{blue!50!black}
  \colorlet{penColor2}{red!50!black}
	\begin{axis}[
            domain=-3:3,
            ymax=4,
            ymin=-4,
            %samples=100,
            axis lines =middle, xlabel=$x$, ylabel=$y$,
            every axis y label/.style={at=(current axis.above origin),anchor=south},
            every axis x label/.style={at=(current axis.right of origin),anchor=west}
          ]
          \addplot [very thick, penColor2, smooth,domain=(-3:3)] {3*x^2};
          \addplot [very thick, penColor, smooth,domain=(-3:3)] {x^3+1};
          \node at (axis cs:1,1.8) [anchor=west] {\color{penColor}$f(x)$};  
          \node at (axis cs:-1,3.3) [anchor=west] {\color{penColor2}$f'(x)$};
        \end{axis}
\end{tikzpicture}
%\caption{A plot of $f(x) = x^3+1$ and $f'(x) = 3x^2$.}
%\label{figure:x^3+1}
\end{image}
\end{example}




\end{document}

%% I've stopped here for now. 





Next we will consider the derivative a function that is not continuous
on $\RR$.


\begin{example}
Compute
\[
\dd t \frac{1}{t}.
\]
\end{example}

\begin{solution}
Using the definition of the derivative,
\begin{align*}
\dd{t}\frac{1}{t}&=\lim_{ h\to0}\frac{\frac{1}{t+ h} - \frac{1}{t}}{h} \\
&=\lim_{h\to0}\frac{\frac{t}{t(t+ h)} - \frac{t+ h}{t(t+ h)}}{h} \\
&=\lim_{h\to0}\frac{\frac{t-(t+ h)}{t(t+ h)}}{h} \\
&=\lim_{h\to0}\frac{t-t- h}{t(t+ h) h} \\
&=\lim_{h\to0}\frac{- h}{t(t+ h) h} \\
&=\lim_{h\to0}\frac{-1}{t(t+ h)}\\
&=\frac{-1}{t^2}.
\end{align*}
This function is differentiable at all real numbers except for $t=0$, see Figure~\ref{figure:plot1/x}.
\end{solution}
\begin{marginfigure}
\begin{tikzpicture}
	\begin{axis}[
            domain=-3:3,
            ymax=4,
            ymin=-4,
            samples=100,
            axis lines =middle, xlabel=$t$, ylabel=$y$,
            every axis y label/.style={at=(current axis.above origin),anchor=south},
            every axis x label/.style={at=(current axis.right of origin),anchor=west}
          ]
          \addplot [very thick, penColor2, smooth,domain=(-3:-.1)] {-1/x^2};
          \addplot [very thick, penColor2, smooth,domain=(.1:3)] {-1/x^2};
	  \addplot [very thick, penColor, smooth,domain=(-3:-.1)] {1/x};
          \addplot [very thick, penColor, smooth,domain=(.1:3)] {1/x};
          \node at (axis cs:1,1.3) [anchor=west] {\color{penColor}$f(t)$}; 
          \node at (axis cs:1,-1.1) [anchor=west] {\color{penColor2}$f'(t)$};
        \end{axis}
\end{tikzpicture}
\caption{A plot of $f(t) = \frac{1}{t}$ and $f'(t) = \frac{-1}{t^2}$.}
\label{figure:plot1/x}
\end{marginfigure}

Let us now answer the following question: MOOCulus MOBILE




%% Perhaps a split should be made here!!



As you may have guessed, there is some connection to continuity and
differentiability. 



\begin{mainTheorem}[Differentiability Implies Continuity]\label{theorem:diff-cont}
If $f(x)$ is a differentiable function at $x = a$, then $f(x)$ is
continuous at $x=a$.
\end{mainTheorem}

\begin{proof}
We want to show that $f(x)$ is continuous at $x=a$, hence we must show that 
\[
\lim_{x\to a} f(x) = f(a).
\]
Consider
\begin{align*}
\lim_{x\to a} \left(f(x) - f(a)\right) &= \lim_{x\to a} \left((x-a)\frac{f(x) - f(a)}{x-a}\right) &\text{Multiply and divide by $(x-a)$.} \\
&= \lim_{h\to 0} h \cdot \frac{f(a+h) - f(a)}{h} &\text{Set $x = a+h$.} \\
&= \left(\lim_{h\to 0} h\right) \left(\lim_{h\to 0}\frac{f(a+h) - f(a)}{h}\right) &\text{Limit Law.} \\
&= 0\cdot f'(a) = 0.
\end{align*}
Since 
\[
\lim_{x\to a}\left(f(x) - f(a)\right) = 0 
\]
we see that $\lim_{x\to a} f(x) = f(a)$, and so $f(x)$ is continuous.
\end{proof}

This theorem is often written as its contrapositive:
\begin{quote}
If $f(x)$ is not continuous at $x=a$, then $f(x)$ is not
differentiable at $x=a$.
\end{quote}


Let's see a function that is continuous whose derivative does not
exist everywhere.


\begin{example}
Compute 
\[
\ddx |x|.
\]
\end{example}
\begin{marginfigure}
\begin{tikzpicture}
	\begin{axis}[
            domain=-3:3,
            ymax=3,
            ymin=-2,
            samples=100,
            axis lines =middle, xlabel=$x$, ylabel=$y$,
            every axis y label/.style={at=(current axis.above origin),anchor=south},
            every axis x label/.style={at=(current axis.right of origin),anchor=west}
          ]
          \addplot [very thick, penColor2, smooth,domain=(0:3)] {1};
          \addplot [very thick, penColor2, smooth,domain=(-3:0)] {-1};
          \addplot [very thick, penColor, smooth] {abs(x)};
          \node at (axis cs:1,1.7) [anchor=west] {\color{penColor}$f(t)$}; 
          \node at (axis cs:-1,-1.5) [anchor=south] {\color{penColor2}$f'(t)$};
          \addplot[color=penColor2,fill=background,only marks,mark=*] coordinates{(0,1)};  %% open hole
          \addplot[color=penColor2,fill=background,only marks,mark=*] coordinates{(0,-1)};  %% open hole
        \end{axis}
\end{tikzpicture}
\caption[A plot of $f(x) = |x|$ and its derivative.]{A plot of $f(x) = |x|$ and \[
f'(x) = \begin{cases}
1 &\text{if $x>0$,}\\
-1 &\text{if $x<0$.}
\end{cases}\]
}
\label{figure:plot-abs}
\end{marginfigure}
\begin{solution}
Using the definition of the derivative,
\[
\ddx |x| = \lim_{h\to0}\frac{|x+h| -|x|}{h}.
\]
If $x$ is positive we may assume that $x$ is larger than $h$, as we are
taking the limit as $h$ goes to $0$,
\begin{align*}
\lim_{h\to0}\frac{|x+h| -|x|}{h} &= \lim_{h\to0}\frac{x+h -x}{h}\\
&= \lim_{h\to0}\frac{h}{h}\\
&= 1.
\end{align*}
If $x$ is negative we may assume that $|x|$ is larger than $h$, as we are taking
the limit as $h$ goes to $0$,
\begin{align*}
\lim_{h\to0}\frac{|x+h| -|x|}{h} &= \lim_{h\to0}\frac{-x-h +x}{h}\\
&= \lim_{h\to0}\frac{-h}{h}\\
&= -1.
\end{align*}
However we still have one case left, when $x=0$. In this situation, we
must consider the one-sided limits:
\[
\lim_{h\to0+}\frac{|x+h| -|x|}{h}\qquad\text{and}\qquad \lim_{h\to0-}\frac{|x+h| -|x|}{h}.
\]
In the first case, 
\begin{align*}
\lim_{h\to0+}\frac{|x+h| -|x|}{h} &= \lim_{h\to 0+}\frac{0+h - 0}{h}\\
&= \lim_{h\to 0+}\frac{h}{h}\\
&=1.
\end{align*}
On the other hand
\begin{align*}
\lim_{h\to0-}\frac{|x+h| -|x|}{h} &= \lim_{h\to 0-}\frac{|0+h| - 0}{h}\\
&= \lim_{h\to 0-}\frac{|h|}{h}\\
&=-1.
\end{align*}
Hence we see that the derivative is
\[
f'(x) = 
\begin{cases}
1 &\text{if $x>0$,}\\
-1 &\text{if $x<0$.}
\end{cases}
\]
Note this function is undefined at $0$, see Figure~\ref{figure:plot-abs}. 
\end{solution}


Thus from Theorem~\ref{theorem:diff-cont}, we see that all
differentiable functions on $\RR$ are continuous on $\RR$. Nevertheless
as the previous example shows, there are continuous functions on $\RR$
that are not differentiable on $\RR$.






\end{document}
