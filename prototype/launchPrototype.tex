\documentclass{ximera}



%% The launch should somehow present a ``mystery'' for the students to
%% solve.  Moreover, it should lead them to the right path. Ideally if we
%% had a bright student working on the launch, they might even develop
%% the techniques from the lesson to solve teh problem. 
%% By the end of the lesson, the mystery is solved!


\title{The derivative: understanding change}

\begin{document}
\begin{abstract}
The derivative helps us understand change.
\end{abstract}
\maketitle

%% Here the idea is to be truthful, enthuastic, and narrow. We want a
%% SINGLE example. So for this prototype we could talk about all kinds
%% of rates of change, but lets refrain from this.


Let me tell you about something that is really awesome: \textit{The
  Global Positioning System.} Seriously, this is a modern ``Wonder of
the World.'' With this system, the secrets of navigating our beautiful
blue sphere are no longer limited to an elite few. Instead we have
devices small enough to fit in our pockets, that cost as little as a
few meals, that can tell us where we are, and even \textit{how fast we
  are going.} Think about this, a mere 2000 years ago, a GPS in the
right hands would have been poweful enough to change the face of
history!


Let me ask you something:

\begin{question}
Assuming a GPS device knows your position at any given time, how
exactly does it compute your velcity?
\begin{freeResponse}
\end{freeResponse}
\end{question}


%% What exactly should we be computing the position of? Maybe hava a
%% video too? I was going to use a rocket, but GPS doesn't work with
%% rockets... Maybe we should not use GPS but RADAR?


Suppose $s(t) = t^2 + 5$ over the time interval $I$ where
\[
I = 
\begin{cases}
[2+h,2]  & \text{if $-1<h<0$}, \\ %% note this is MORE correct than std books
[2,2+h]  & \text{if $0<h<1$}. %% part of the question need to be "WHY" this interval is so complex
\end{cases}
\]


COMPARE WITH CM...


\begin{question}
This method described above may be fine for a robot that is computing
velocity from position using raw data. However, we are human beings
and not robots. What would really help us out is a forumla. Given a
formula for position, how do we find a formula for velocity?
\end{question}











%% REPLACE WITH XARMA BOOST
\begin{xarmaBoost}
Write down at least \textbf{five} questions for this lecture. After
you have your questions, label them as ``Level 1,'' ``Level 2,'' or ``Level 3'' where:
\begin{description}
\item[Level 1] Means you know the answer, or know exactly how to do this problem.
\item[Level 2] Means you think you know how to do the problem, or will soon learn how to do the problem.
\item[Level 3] Means you have no idea how to do the problem. 
\end{description}
  \begin{freeResponse}
  \end{freeResponse}
\end{xarmaBoost}

\end{document}
