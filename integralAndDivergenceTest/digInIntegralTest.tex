\documentclass{ximera}

%\usepackage{todonotes}
%\usepackage{mathtools} %% Required for wide table Curl and Greens
%\usepackage{cuted} %% Required for wide table Curl and Greens
\newcommand{\todo}{}

\usepackage{esint} % for \oiint
\ifxake%%https://math.meta.stackexchange.com/questions/9973/how-do-you-render-a-closed-surface-double-integral
\renewcommand{\oiint}{{\large\bigcirc}\kern-1.56em\iint}
\fi


\graphicspath{
  {./}
  {ximeraTutorial/}
  {basicPhilosophy/}
  {functionsOfSeveralVariables/}
  {normalVectors/}
  {lagrangeMultipliers/}
  {vectorFields/}
  {greensTheorem/}
  {shapeOfThingsToCome/}
  {dotProducts/}
  {partialDerivativesAndTheGradientVector/}
  {../productAndQuotientRules/exercises/}
  {../normalVectors/exercisesParametricPlots/}
  {../continuityOfFunctionsOfSeveralVariables/exercises/}
  {../partialDerivativesAndTheGradientVector/exercises/}
  {../directionalDerivativeAndChainRule/exercises/}
  {../commonCoordinates/exercisesCylindricalCoordinates/}
  {../commonCoordinates/exercisesSphericalCoordinates/}
  {../greensTheorem/exercisesCurlAndLineIntegrals/}
  {../greensTheorem/exercisesDivergenceAndLineIntegrals/}
  {../shapeOfThingsToCome/exercisesDivergenceTheorem/}
  {../greensTheorem/}
  {../shapeOfThingsToCome/}
  {../separableDifferentialEquations/exercises/}
  {vectorFields/}
}

\newcommand{\mooculus}{\textsf{\textbf{MOOC}\textnormal{\textsf{ULUS}}}}

\usepackage{tkz-euclide}\usepackage{tikz}
\usepackage{tikz-cd}
\usetikzlibrary{arrows}
\tikzset{>=stealth,commutative diagrams/.cd,
  arrow style=tikz,diagrams={>=stealth}} %% cool arrow head
\tikzset{shorten <>/.style={ shorten >=#1, shorten <=#1 } } %% allows shorter vectors

\usetikzlibrary{backgrounds} %% for boxes around graphs
\usetikzlibrary{shapes,positioning}  %% Clouds and stars
\usetikzlibrary{matrix} %% for matrix
\usepgfplotslibrary{polar} %% for polar plots
\usepgfplotslibrary{fillbetween} %% to shade area between curves in TikZ
\usetkzobj{all}
\usepackage[makeroom]{cancel} %% for strike outs
%\usepackage{mathtools} %% for pretty underbrace % Breaks Ximera
%\usepackage{multicol}
\usepackage{pgffor} %% required for integral for loops



%% http://tex.stackexchange.com/questions/66490/drawing-a-tikz-arc-specifying-the-center
%% Draws beach ball
\tikzset{pics/carc/.style args={#1:#2:#3}{code={\draw[pic actions] (#1:#3) arc(#1:#2:#3);}}}



\usepackage{array}
\setlength{\extrarowheight}{+.1cm}
\newdimen\digitwidth
\settowidth\digitwidth{9}
\def\divrule#1#2{
\noalign{\moveright#1\digitwidth
\vbox{\hrule width#2\digitwidth}}}





\newcommand{\RR}{\mathbb R}
\newcommand{\R}{\mathbb R}
\newcommand{\N}{\mathbb N}
\newcommand{\Z}{\mathbb Z}

\newcommand{\sagemath}{\textsf{SageMath}}


%\renewcommand{\d}{\,d\!}
\renewcommand{\d}{\mathop{}\!d}
\newcommand{\dd}[2][]{\frac{\d #1}{\d #2}}
\newcommand{\pp}[2][]{\frac{\partial #1}{\partial #2}}
\renewcommand{\l}{\ell}
\newcommand{\ddx}{\frac{d}{\d x}}

\newcommand{\zeroOverZero}{\ensuremath{\boldsymbol{\tfrac{0}{0}}}}
\newcommand{\inftyOverInfty}{\ensuremath{\boldsymbol{\tfrac{\infty}{\infty}}}}
\newcommand{\zeroOverInfty}{\ensuremath{\boldsymbol{\tfrac{0}{\infty}}}}
\newcommand{\zeroTimesInfty}{\ensuremath{\small\boldsymbol{0\cdot \infty}}}
\newcommand{\inftyMinusInfty}{\ensuremath{\small\boldsymbol{\infty - \infty}}}
\newcommand{\oneToInfty}{\ensuremath{\boldsymbol{1^\infty}}}
\newcommand{\zeroToZero}{\ensuremath{\boldsymbol{0^0}}}
\newcommand{\inftyToZero}{\ensuremath{\boldsymbol{\infty^0}}}



\newcommand{\numOverZero}{\ensuremath{\boldsymbol{\tfrac{\#}{0}}}}
\newcommand{\dfn}{\textbf}
%\newcommand{\unit}{\,\mathrm}
\newcommand{\unit}{\mathop{}\!\mathrm}
\newcommand{\eval}[1]{\bigg[ #1 \bigg]}
\newcommand{\seq}[1]{\left( #1 \right)}
\renewcommand{\epsilon}{\varepsilon}
\renewcommand{\phi}{\varphi}


\renewcommand{\iff}{\Leftrightarrow}

\DeclareMathOperator{\arccot}{arccot}
\DeclareMathOperator{\arcsec}{arcsec}
\DeclareMathOperator{\arccsc}{arccsc}
\DeclareMathOperator{\si}{Si}
\DeclareMathOperator{\scal}{scal}
\DeclareMathOperator{\sign}{sign}


%% \newcommand{\tightoverset}[2]{% for arrow vec
%%   \mathop{#2}\limits^{\vbox to -.5ex{\kern-0.75ex\hbox{$#1$}\vss}}}
\newcommand{\arrowvec}[1]{{\overset{\rightharpoonup}{#1}}}
%\renewcommand{\vec}[1]{\arrowvec{\mathbf{#1}}}
\renewcommand{\vec}[1]{{\overset{\boldsymbol{\rightharpoonup}}{\mathbf{#1}}}\hspace{0in}}

\newcommand{\point}[1]{\left(#1\right)} %this allows \vector{ to be changed to \vector{ with a quick find and replace
\newcommand{\pt}[1]{\mathbf{#1}} %this allows \vec{ to be changed to \vec{ with a quick find and replace
\newcommand{\Lim}[2]{\lim_{\point{#1} \to \point{#2}}} %Bart, I changed this to point since I want to use it.  It runs through both of the exercise and exerciseE files in limits section, which is why it was in each document to start with.

\DeclareMathOperator{\proj}{\mathbf{proj}}
\newcommand{\veci}{{\boldsymbol{\hat{\imath}}}}
\newcommand{\vecj}{{\boldsymbol{\hat{\jmath}}}}
\newcommand{\veck}{{\boldsymbol{\hat{k}}}}
\newcommand{\vecl}{\vec{\boldsymbol{\l}}}
\newcommand{\uvec}[1]{\mathbf{\hat{#1}}}
\newcommand{\utan}{\mathbf{\hat{t}}}
\newcommand{\unormal}{\mathbf{\hat{n}}}
\newcommand{\ubinormal}{\mathbf{\hat{b}}}

\newcommand{\dotp}{\bullet}
\newcommand{\cross}{\boldsymbol\times}
\newcommand{\grad}{\boldsymbol\nabla}
\newcommand{\divergence}{\grad\dotp}
\newcommand{\curl}{\grad\cross}
%\DeclareMathOperator{\divergence}{divergence}
%\DeclareMathOperator{\curl}[1]{\grad\cross #1}
\newcommand{\lto}{\mathop{\longrightarrow\,}\limits}

\renewcommand{\bar}{\overline}

\colorlet{textColor}{black}
\colorlet{background}{white}
\colorlet{penColor}{blue!50!black} % Color of a curve in a plot
\colorlet{penColor2}{red!50!black}% Color of a curve in a plot
\colorlet{penColor3}{red!50!blue} % Color of a curve in a plot
\colorlet{penColor4}{green!50!black} % Color of a curve in a plot
\colorlet{penColor5}{orange!80!black} % Color of a curve in a plot
\colorlet{penColor6}{yellow!70!black} % Color of a curve in a plot
\colorlet{fill1}{penColor!20} % Color of fill in a plot
\colorlet{fill2}{penColor2!20} % Color of fill in a plot
\colorlet{fillp}{fill1} % Color of positive area
\colorlet{filln}{penColor2!20} % Color of negative area
\colorlet{fill3}{penColor3!20} % Fill
\colorlet{fill4}{penColor4!20} % Fill
\colorlet{fill5}{penColor5!20} % Fill
\colorlet{gridColor}{gray!50} % Color of grid in a plot

\newcommand{\surfaceColor}{violet}
\newcommand{\surfaceColorTwo}{redyellow}
\newcommand{\sliceColor}{greenyellow}




\pgfmathdeclarefunction{gauss}{2}{% gives gaussian
  \pgfmathparse{1/(#2*sqrt(2*pi))*exp(-((x-#1)^2)/(2*#2^2))}%
}


%%%%%%%%%%%%%
%% Vectors
%%%%%%%%%%%%%

%% Simple horiz vectors
\renewcommand{\vector}[1]{\left\langle #1\right\rangle}


%% %% Complex Horiz Vectors with angle brackets
%% \makeatletter
%% \renewcommand{\vector}[2][ , ]{\left\langle%
%%   \def\nextitem{\def\nextitem{#1}}%
%%   \@for \el:=#2\do{\nextitem\el}\right\rangle%
%% }
%% \makeatother

%% %% Vertical Vectors
%% \def\vector#1{\begin{bmatrix}\vecListA#1,,\end{bmatrix}}
%% \def\vecListA#1,{\if,#1,\else #1\cr \expandafter \vecListA \fi}

%%%%%%%%%%%%%
%% End of vectors
%%%%%%%%%%%%%

%\newcommand{\fullwidth}{}
%\newcommand{\normalwidth}{}



%% makes a snazzy t-chart for evaluating functions
%\newenvironment{tchart}{\rowcolors{2}{}{background!90!textColor}\array}{\endarray}

%%This is to help with formatting on future title pages.
\newenvironment{sectionOutcomes}{}{}



%% Flowchart stuff
%\tikzstyle{startstop} = [rectangle, rounded corners, minimum width=3cm, minimum height=1cm,text centered, draw=black]
%\tikzstyle{question} = [rectangle, minimum width=3cm, minimum height=1cm, text centered, draw=black]
%\tikzstyle{decision} = [trapezium, trapezium left angle=70, trapezium right angle=110, minimum width=3cm, minimum height=1cm, text centered, draw=black]
%\tikzstyle{question} = [rectangle, rounded corners, minimum width=3cm, minimum height=1cm,text centered, draw=black]
%\tikzstyle{process} = [rectangle, minimum width=3cm, minimum height=1cm, text centered, draw=black]
%\tikzstyle{decision} = [trapezium, trapezium left angle=70, trapezium right angle=110, minimum width=3cm, minimum height=1cm, text centered, draw=black]

%
%\author{Jim Talamo and ???}
\outcome{Use the integral test to  determine that a series diverges.}
\outcome{Use integrals to estimate the value of a series within an error.}

\title[Dig-In:]{The integral test}

\begin{document}
\begin{abstract}
Certain infinite series can be studied using improper integrals.
\end{abstract}
\maketitle

In order to study the convergence of a series $\sum_{k=k_0}^{\infty} a_k$, our first attempt to determine whether the series converges is to form the sequence of partial sums $\{s_n\}_{n=k_0}$ since we know that the series $\sum_{k=k_0}^{\infty} a_k$ converges if and only if $\lim_{n \to \infty} s_n$ exists.  In the case of geometric or telescoping series, we were able to find an explicit formula for $s_n$, and analyze $\lim_{n \to \infty} s_n$ by explicit computation.  However, we cannot always find such an explicit formula, and when this is the case, we try to use properties of the terms in the sequence $\{a_n\}$ to determine whether $\lim_{n \to \infty} s_n$ exists.  Our first result was the divergence test, which states

\begin{quote}
If $\lim_{n \to \infty} a_n \neq 0$, then $\sum_{k=k_0} a_k$ diverges.
\end{quote}

However, there are still some divergent series that the divergence test does not pick out!  We begin this section with such an example that shows how there is a connection between certain special types of series and improper integrals.

\begin{model}
Consider the series $\sum_{k=1}^{\infty} \frac{1}{k}$.  This series is called the \emph{harmonic series} and will be important in the coming material.  Notice that this series \wordChoice{\choice{is}\choice[correct]{is not}} geometric. It \wordChoice{\choice{is}\choice[correct]{is not}} telescoping, and $\lim_{n \to \infty} \frac{1}{n} = \answer{0}$, so the divergence test \wordChoice{\choice{is}\choice[correct]{is not}} conclusive.  So what should we do? 

We have seen that we can graph a sequence as a collection of points in
the plane.  We consider the \dfn{harmonic sequence} where $a_n = 1/n$, and write out the ordered list that represents it.
\[
1, \, \frac{1}{2}, \, \frac{1}{3}, \, \frac{1}{4}, \, \frac{1}{5}, \,\dots . 
\]
If we plot the harmonic sequence, it looks like this.
\begin{image}
\begin{tikzpicture}
	\begin{axis}[
            domain=0:6,xmin=0,xmax=6,ymin=0,ymax=1.5,
            width=4in,
            height=2in,
            xtick={1,2,...,5},
            ytick={1,.5,.333,.25,.2},
            yticklabels={},%$a_1 = 10$,$a_2=30$,$a_3=90$,$a_4=270$,$a_5=810$},
            axis lines =middle, xlabel=$n$, ylabel=$a$,
            every axis y label/.style={at=(current axis.above origin),anchor=south},
            every axis x label/.style={at=(current axis.right of origin),anchor=west},
            clip=false,
            axis on top,
          ]
          \addplot[color=penColor,fill=penColor,only marks,mark=*] coordinates{(1,1)};  %% closed hole          
          \addplot[color=penColor,fill=penColor,only marks,mark=*] coordinates{(2,1/2)};  %% closed hole          
          \addplot[color=penColor,fill=penColor,only marks,mark=*] coordinates{(3,1/3)};  %% closed hole          
          \addplot[color=penColor,fill=penColor,only marks,mark=*] coordinates{(4,1/4)};  %% closed hole          
          \addplot[color=penColor,fill=penColor,only marks,mark=*] coordinates{(5,1/5)};  %% closed hole  
        \end{axis}
\end{tikzpicture}
\end{image}

As it turns out, there is a nice way to visualize the sum $\sum_{k=1}^\infty \frac{1}{k}$ too!  One such way is to to make rectangles whose areas are equal to the terms in the sequence.
\begin{image}
\begin{tikzpicture}
	\begin{axis}[
            domain=0:6,xmin=0,xmax=6,ymin=0,ymax=1.5,
            width=4in,
            height=2in,
            xtick={1,2,...,5},
            ytick={1,.5,.333,.25,.2},
            yticklabels={},%$a_1 = 10$,$a_2=30$,$a_3=90$,$a_4=270$,$a_5=810$},
            axis lines =middle, xlabel=$n$, ylabel=$a$,
            every axis y label/.style={at=(current axis.above origin),anchor=south},
            every axis x label/.style={at=(current axis.right of origin),anchor=west},
            axis on top,
          ]
          \addplot[color=penColor,fill=penColor,only marks,mark=*] coordinates{(1,1)};  %% closed hole
		\addplot [draw=penColor, fill = fillp] plot coordinates {(1,0) (2,0) (2, 1) (1,1) (1, 0)};          
          
	\addplot[color=penColor,fill=penColor,only marks,mark=*] coordinates{(2,.5)};  %% closed hole
		\addplot [draw=penColor, fill = fillp] plot coordinates {(2,0) (3,0) (3, .5) (2,.5) (2, 0)};          
          
          \addplot[color=penColor,fill=penColor,only marks,mark=*] coordinates{(3,.333)};  %% closed hole
          	\addplot [draw=penColor, fill = fillp] plot coordinates {(3,0) (4,0) (4, .333) (3,.333) (3, 0)};          

          
          \addplot[color=penColor,fill=penColor,only marks,mark=*] coordinates{(4,.25)};  %% closed hole        
          	\addplot [draw=penColor, fill = fillp] plot coordinates {(4,0) (5,0) (5, .25) (4,.25) (4, 0)};          

		  
          \addplot[color=penColor,fill=penColor,only marks,mark=*] coordinates{(5,.2)};  %% closed hole 
          	\addplot [draw=penColor, fill = fillp] plot coordinates {(5,0) (6,0) (6, .2) (5,.2) (5, 0)};          
           %\addplot [draw=penColor,very thick,smooth, domain=.5:6] {1/x};
        \end{axis}
\end{tikzpicture}
\end{image}

Note that the height of the $k$-th rectangle is precisely $\frac{1}{k}$ and the width of all of the rectangles is $1$, so the area of the $k$-th rectangle is $\frac{1}{k}$.  

Now, in order to conclude whether $\sum_{k=1}^{\infty} a_k$ converges, we must analyze $\lim_{n \to \infty} s_n$.  Note that $s_n$ has a nice visual interpretation as the sum of the areas of the first $n$ rectangles, but since we don't have an explicit formula for $s_n$ we can try to establish that 

\begin{itemize}
\item $\{s_n\}$ is bounded and monotonic, and hence $\lim_{n \to \infty} s_n$ exists.
\item $\{s_n\}$ is unbounded so $\lim_{n \to \infty} s_n$ does not exist.
\end{itemize}

How can we establish this?  The previous image might remind you of a Riemann Sum and for good reason. 
This technique lets us visually compare the sum of an infinite series to the
value of an improper integral.  For instance, if we add a plot of
$1/x$ to our picture above
\begin{image}
\begin{tikzpicture}
	\begin{axis}[
            domain=0:6,xmin=0,xmax=6,ymin=0,ymax=1.5,
            width=4in,
            height=2in,
            xtick={1,2,...,5},
            ytick={1,.5,.333,.25,.2},
            yticklabels={},%$a_1 = 10$,$a_2=30$,$a_3=90$,$a_4=270$,$a_5=810$},
            axis lines =middle, xlabel={}, ylabel={},
            every axis y label/.style={at=(current axis.above origin),anchor=south},
            every axis x label/.style={at=(current axis.right of origin),anchor=west},
            axis on top,
          ]
          \addplot[color=penColor,fill=penColor,only marks,mark=*] coordinates{(1,1)};  %% closed hole
		\addplot [draw=penColor, fill = fillp] plot coordinates {(1,0) (2,0) (2, 1) (1,1) (1, 0)};          
          
	\addplot[color=penColor,fill=penColor,only marks,mark=*] coordinates{(2,.5)};  %% closed hole
		\addplot [draw=penColor, fill = fillp] plot coordinates {(2,0) (3,0) (3, .5) (2,.5) (2, 0)};          
          
          \addplot[color=penColor,fill=penColor,only marks,mark=*] coordinates{(3,.333)};  %% closed hole
          	\addplot [draw=penColor, fill = fillp] plot coordinates {(3,0) (4,0) (4, .333) (3,.333) (3, 0)};          

          
          \addplot[color=penColor,fill=penColor,only marks,mark=*] coordinates{(4,.25)};  %% closed hole        
          	\addplot [draw=penColor, fill = fillp] plot coordinates {(4,0) (5,0) (5, .25) (4,.25) (4, 0)};          

		  
          \addplot[color=penColor,fill=penColor,only marks,mark=*] coordinates{(5,.2)};  %% closed hole 
          	\addplot [draw=penColor, fill = fillp] plot coordinates {(5,0) (6,0) (6, .2) (5,.2) (5, 0)};          
           \addplot [draw=penColor,very thick,smooth, domain=.5:6] {1/x};
        \end{axis}
\end{tikzpicture}
\end{image}
we see that
\[
\int_1^n \frac{1}{x} \d x < \sum_{k=1}^n \frac{1}{k}.
\]

Notice that the sum on the righthand side is simply $s_n$.  Since $\int_1^{\infty} \frac{1}{x} \d x$ is an improper integral, so we need to determine whether $\lim_{n \to \infty} \left[ \int_1^n \frac{1}{x} \d x \right]$ exists.  Notice

\[
\lim_{n \to \infty} \left[ \int_1^n \frac{1}{x} \d x \right] = \lim_{n \to \infty} \eval{\ln(x)}_1^n = \infty. 
\]

This means that $\{s_n\}$ is not bounded and hence $\lim_{n \to \infty} s_n = \infty$ and $\sum_{k=1}^{\infty} \frac{1}{k}$ must diverge.  

\begin{remark}
To argue why $\{s_n\}$ is not bounded a little more formally, note that if $\{s_n\}$ were bounded, then \emph{all} of the terms in the sequence $\{s_n\}$ would be less than some number $M$.  Since $\lim_{n \to \infty} \left[ \int_1^n \frac{1}{x} \d x \right] =  \infty$, this means that, for some value of $N$ after which all values of $n>N$, $ \int_1^n \frac{1}{x} \d x > M$, and since $\int_1^n \frac{1}{x} \d x < s_n$ for all $n$, we may conclude that $s_n >M$ for all $n\geq N$ and hence $\{s_n\}$ is unbounded.
\end{remark}

\end{model}

Now, let's take a step back and see what we really needed in the previous example.

\begin{itemize}
\item We needed to find a function for which the area under the curve over any particular interval $[n,n+1]$ was less than the area of the rectangle whose height is $a_k$ to establish a lower bound for each $s_n$.  Note that we can always do this if $f(x)$ is eventually positive and \wordChoice{\choice[correct]{decreasing}\choice{increasing}} since we may view each $a_k$ as the area of the rectangle that coincides with $f(x)$ at its \wordChoice{\choice[correct]{lefthand}\choice{righthand}} endpoint.
\item We needed the function to be ``eventually continuous'' so the improper integral $\int_{a}^{\infty} f(x) \d x$ can be computed as the limit of a single definite integral.
\end{itemize}

By ``eventually'' above, we really mean that $f(x)$ should be continuous, positive, and decreasing on some interval $[a,\infty)$ for some $a>0$; it doesn't need to happen right away, but it should hold for \emph{all} real large enough $x$-values.  This leads us to an interesting observation.
\begin{quote}
  Let $f(x)$ be an eventually continuous, positive, and decreasing function with
    $a_k = f(k)$.  If $\int_1^\infty f(x) \d x$ diverges, so does
    $\sum_{k=1}^\infty a_k$.
\end{quote}

That's a pretty good observation, but we can do even better. 

\begin{model}
Consider the
sequence $\{a_n\}_{n=1}$ where $a_n = \frac{1}{n^2}$.  We can write out a few terms

\[
1, \, \frac{1}{4}, \, \frac{1}{9}, \, \frac{1}{16}, \, \frac{1}{25}, \,\dots 
\]
and try the same idea of visualizing the series $\sum_{k=1}^{\infty} \frac{1}{k^2}$ as an area.


%\[
%\sum_{k=1}^\infty a_k
%\]
%diverges, then so does
%\[
%\sum_{k=1}^\infty a_{k+1}.
%\]
%Why?  The second sum is simply the first sum missing $a_1$ and no single number can
%be responsible for this sum diverging to infinity. This fact can be summed up in the following theorem:
%
%\begin{theorem}[Infinite Nature of Series]
%The convergence or divergence of a series remains unchanged by the addition or
%subtraction of any finite number of terms. That is:
%\begin{itemize}
%\item a divergent series will remain divergent with the addition or
%  subtraction of any finite number of terms.
%\item a convergent series will remain convergent with the addition or
%  subtraction of any finite number of terms. (Of course, the
%  \emph{sum} will likely change.)
%\end{itemize}
%\end{theorem}
%Back to the task at hand! Let's graph $a_{k+1}$ using our rectangle technique.
\begin{image}
\begin{tikzpicture}
	\begin{axis}[
            domain=0:6,xmin=0,xmax=6,ymin=0,ymax=1.5,
            width=4in,
            height=2in,
            xtick={1,2,...,5},
            ytick={1,.5,.333,.25,.2},
            yticklabels={},%$a_1 = 10$,$a_2=30$,$a_3=90$,$a_4=270$,$a_5=810$},
            axis lines =middle, xlabel=$n$, ylabel=$a_n$,
            every axis y label/.style={at=(current axis.above origin),anchor=south},
            every axis x label/.style={at=(current axis.right of origin),anchor=west},
            axis on top,
          ]
          \addplot[color=penColor,fill=penColor,only marks,mark=*] coordinates{(0,1)};  %% closed hole
		\addplot [draw=penColor, fill = fillp] plot coordinates {(0,0) (1,0) (1, 1)(0,1) };   
		
          \addplot[color=penColor,fill=penColor,only marks,mark=*] coordinates{(1,1/4)};  %% closed hole
		\addplot [draw=penColor, fill = fillp] plot coordinates {(1,0) (2,0) (2, 1/4) (1,1/4) (1, 0)};          
          
	\addplot[color=penColor,fill=penColor,only marks,mark=*] coordinates{(2,1/9)};  %% closed hole
		\addplot [draw=penColor, fill = fillp] plot coordinates {(2,0) (3,0) (3, 1/9) (2,1/9) (2, 0)};          
          
          \addplot[color=penColor,fill=penColor,only marks,mark=*] coordinates{(3,1/16)};  %% closed hole
          	\addplot [draw=penColor, fill = fillp] plot coordinates {(3,0) (4,0) (4, 1/16) (3,1/16) (3, 0)};          

          
          \addplot[color=penColor,fill=penColor,only marks,mark=*] coordinates{(4,1/25)};  %% closed hole        
          	\addplot [draw=penColor, fill = fillp] plot coordinates {(4,0) (5,0) (5, 1/25) (4,1/25) (4, 0)};          

		  
          \addplot[color=penColor,fill=penColor,only marks,mark=*] coordinates{(5,1/36)};  %% closed hole 
          	\addplot [draw=penColor, fill = fillp] plot coordinates {(5,0) (6,0) (6, 1/36) (5,1/36) (5, 0)};          
           %\addplot [draw=penColor,very thick,smooth, domain=.5:6] {1/x};
        \end{axis}
\end{tikzpicture}
\end{image}
The shaded area above \textbf{exactly} represents to the sum
\[
\sum_{k=1}^\infty\frac{1}{k^2}.
\]
We again visually compare the sum of an infinite series to the
value of an improper integral by adding a plot of
$1/x^2$ to our picture above.
\begin{image}
\begin{tikzpicture}
	\begin{axis}[
            domain=0:6,xmin=0,xmax=6,ymin=0,ymax=1.5,
            width=4in,
            height=2in,
            xtick={1,2,...,5},
            ytick={1,.5,.333,.25,.2},
            yticklabels={},%$a_1 = 10$,$a_2=30$,$a_3=90$,$a_4=270$,$a_5=810$},
            axis lines =middle, xlabel={}, ylabel={},
            every axis y label/.style={at=(current axis.above origin),anchor=south},
            every axis x label/.style={at=(current axis.right of origin),anchor=west},
            axis on top,
          ]
          
                    \addplot[color=penColor,fill=penColor,only marks,mark=*] coordinates{(0,1)};  %% closed hole
		\addplot [draw=penColor, fill = fillp] plot coordinates {(0,0) (1,0) (1, 1)(0,1) };   
		
		
          \addplot[color=penColor,fill=penColor,only marks,mark=*] coordinates{(1,1/4)};  %% closed hole
		\addplot [draw=penColor, fill = fillp] plot coordinates {(1,0) (2,0) (2, 1/4) (1,1/4) (1, 0)};          
          
	\addplot[color=penColor,fill=penColor,only marks,mark=*] coordinates{(2,1/9)};  %% closed hole
		\addplot [draw=penColor, fill = fillp] plot coordinates {(2,0) (3,0) (3, 1/9) (2,1/9) (2, 0)};          
          
          \addplot[color=penColor,fill=penColor,only marks,mark=*] coordinates{(3,1/16)};  %% closed hole
          	\addplot [draw=penColor, fill = fillp] plot coordinates {(3,0) (4,0) (4, 1/16) (3,1/16) (3, 0)};          

          
          \addplot[color=penColor,fill=penColor,only marks,mark=*] coordinates{(4,1/25)};  %% closed hole        
          	\addplot [draw=penColor, fill = fillp] plot coordinates {(4,0) (5,0) (5, 1/25) (4,1/25) (4, 0)};          

		  
          \addplot[color=penColor,fill=penColor,only marks,mark=*] coordinates{(5,1/36)};  %% closed hole 
          	\addplot [draw=penColor, fill = fillp] plot coordinates {(5,0) (6,0) (6, 1/36) (5,1/36) (5, 0)};          
           \addplot [draw=penColor,very thick,smooth, domain=.5:6] {1/x^2};
        \end{axis}
\end{tikzpicture}
\end{image}

Note that we have a slight annoyance if we consider $\int_{0}^{\infty} \frac{1}{x^2} \d x$ since $\frac{1}{x^2}$ has a vertical asymptote at $x=0$.  This is easily avoided if we instead consider $\frac{1}{x^2}$ on the interval $[1,\infty)$.  This requires that we consider the rectangles on that interval too.  We update our picture.


\begin{image}
\begin{tikzpicture}
	\begin{axis}[
            domain=0:6,xmin=0,xmax=6,ymin=0,ymax=1.5,
            width=4in,
            height=2in,
            xtick={1,2,...,5},
            ytick={1,.5,.333,.25,.2},
            yticklabels={},%$a_1 = 10$,$a_2=30$,$a_3=90$,$a_4=270$,$a_5=810$},
            axis lines =middle, xlabel={}, ylabel={},
            every axis y label/.style={at=(current axis.above origin),anchor=south},
            every axis x label/.style={at=(current axis.right of origin),anchor=west},
            axis on top,
          ]
          
                    \addplot[color=penColor,fill=penColor,only marks,mark=*] coordinates{(0,1)};  %% closed hole
		\addplot [draw=penColor, fill = fillp] plot coordinates {(0,0) (1,0) (1, 1)(0,1) };   
		
		
          \addplot[color=penColor,fill=penColor,only marks,mark=*] coordinates{(1,1/4)};  %% closed hole
		\addplot [draw=penColor, fill = fillp] plot coordinates {(1,0) (2,0) (2, 1/4) (1,1/4) (1, 0)};          
          
	\addplot[color=penColor,fill=penColor,only marks,mark=*] coordinates{(2,1/9)};  %% closed hole
		\addplot [draw=penColor, fill = fillp] plot coordinates {(2,0) (3,0) (3, 1/9) (2,1/9) (2, 0)};          
          
          \addplot[color=penColor,fill=penColor,only marks,mark=*] coordinates{(3,1/16)};  %% closed hole
          	\addplot [draw=penColor, fill = fillp] plot coordinates {(3,0) (4,0) (4, 1/16) (3,1/16) (3, 0)};          

          
          \addplot[color=penColor,fill=penColor,only marks,mark=*] coordinates{(4,1/25)};  %% closed hole        
          	\addplot [draw=penColor, fill = fillp] plot coordinates {(4,0) (5,0) (5, 1/25) (4,1/25) (4, 0)};          

		  
          \addplot[color=penColor,fill=penColor,only marks,mark=*] coordinates{(5,1/36)};  %% closed hole 
          	\addplot [draw=penColor, fill = fillp] plot coordinates {(5,0) (6,0) (6, 1/36) (5,1/36) (5, 0)};          
           \addplot [draw=penColor,very thick,smooth, domain=.5:6] {1/x^2};
        \end{axis}
\end{tikzpicture}
\end{image}

Notice that the sum of the areas of the rectangles now is the series \wordChoice{\choice{$\sum_{k=0}^{\infty} \frac{1}{k^2}$}\choice{$\sum_{k=1}^{\infty} \frac{1}{k^2}$}\choice[correct]{$\sum_{k=2}^{\infty} \frac{1}{k^2}$}}.

This is not an issue, because we know

\begin{quote}
The value of the lower index of summation does not affect whether an infinite series converges or diverges.
\end{quote}

That is, if we can show $\sum_{k=2}^{\infty} \frac{1}{k^2}$ converges, then $\sum_{k=1}^{\infty} \frac{1}{k^2}$ must also converge.  Now to determine whether $\sum_{k=2}^{\infty} \frac{1}{k^2}$ converges, let $s_n = \sum_{k=2}^n \frac{1}{k^2}$.  We must determine if $\lim_{n \to \infty} s_n$ exists.  Since we expect $\int_1^{\infty} f(x) \d x$ to converge, we expect $\sum_{k=2}^{\infty} \frac{1}{k^2}$ to converge.  To show this, we should establish  \wordChoice{\choice[correct]{$\{s_n\}$ is bounded and monotonic, and hence $\lim_{n \to \infty} s_n$ exists}\choice{$\{s_n\}$ is unbounded so $\lim_{n \to \infty} s_n$ does not exist}} . 

Notice that since $\frac{1}{k^2} > 0$ for all $k$, and $s_n = \sum_{k=1}^n$, the sequence $\{s_n\}$ is \wordChoice{\choice[correct]{increasing}\choice{decreasing}}.  

To show that it is also bounded, note that for any $n \geq 2$, we can observe from the picture that 

\[
s_n = \sum_{k=1}^n\frac{1}{k^2} \leq
\int_1^n \frac{1}{x^2} \d x.
\]
In particular, since we shade more area if we increase $n$, we have that $\int_1^n \frac{1}{x^2} \d x \leq \int_1^\infty \frac{1}{x^2} \d x$.  By a routine calculation of the latter improper integral, we can show

\[
\int_1^\infty \frac{1}{x^2} \d x = 1,
\]

and we thus have 

\[
s_n = \sum_{k=1}^n\frac{1}{k^2} \leq \int_1^n \frac{1}{x^2} \d x \leq 1 \textrm{ for all } n \geq 2.
\]

Hence $\{s_n\}$ is both bounded and monotonic, so $\lim_{n \to \infty} s_n$ exists, and $\sum_{k=2}^{\infty} \frac{1}{k^2}$ converges.  Since $\sum_{k=2}^{\infty} \frac{1}{k^2}$ converges, $\sum_{k=1}^{\infty} \frac{1}{k^2}$ will also converge since the index where we start the summation does not affect whether the series converges.

Note that from the above, the value of the improper integral \wordChoice{\choice{is}\choice[correct]{is not}} the value of the series; indeed, by visualizing the series as the sum of the areas of the rectangles in the image, we see that the value of the series $\sum_{k=2}^{\infty} \frac{1}{k^2}$ should be \wordChoice{\choice[correct]{less than}\choice{equal to}\choice{greater than}} the value of the improper integral $\int_2^{\infty} \frac{1}{x^2}$.

\end{model}


Now, let's take a step back and see what we really needed in the this example.

\begin{itemize}
\item We needed to find a function for which the area under the curve over any particular interval $[n,n+1]$ was \emph{greater} than the area of the rectangle whose height is $a_k$ to establish a lower bound for each $s_n$.  Note that we can always do this if $f(x)$ is eventually positive and \wordChoice{\choice[correct]{decreasing}\choice{increasing}} since we may view each $a_k$ as the area of the rectangle that coincides with $f(x)$ at its \wordChoice{\choice{lefthand}\choice[correct]{righthand}} endpoint.
\item We needed to establish that the sequence of partial sums is eventually increasing.  This must happen if all of the $a_k$ are \wordChoice{\choice[correct]{positive}\choice{negative}}.
\item We needed the function to be ``eventually continuous'' so the improper integral $\int_{a}^{\infty} f(x) \d x$ can be computed as the limit of a single definite integral.
\end{itemize}

By ``eventually'' above, we really mean that $f(x)$ should be continuous, positive, and decreasing on some interval $[a,\infty)$; it doesn't need to happen right away, but it should hold for \emph{all} real large enough $x$-values.  This leads us to an interesting observation.
\begin{quote}
  Let $f(x)$ be an eventually continuous, positive, and decreasing function with
    $a_k = f(k)$.  If $\int_1^\infty f(x) \d x$ converges, so does
    $\sum_{k=1}^\infty a_k$.
\end{quote}

%%%%%%%%%%%%%%%%%%%%%%%%%%%%%%%%%%%%%%%%%%%%%%%%%%%%%%%%%%%%%%%%%%%%%%
\section{The Integral Test}
The observations from the previous examples give us a new convergence test called the \textit{integral test}:
\begin{theorem}[Integral Test]
  Suppose that $\{a_n\}_{n=n_0}$ is a sequence, and let $f(x)$ be a continuous, positive, and decreasing function
    with $a_k = f(k)$ on $[N,\infty)$ for some index $N \geq n_0$.  Then, 
    \[
    \sum_{k=n_0}^\infty a_k\text{ and }\int_N^\infty f(x) \d x
    \]
    \textbf{either both converge or both diverge}.
\end{theorem}
%%%%%%%%%%%%%%%%%%%%%%%%%%%%%%%%%%%%%%%%%%%%%%%%%%%%%%%%%%%%%%%%%%%%%%

\begin{example}
Determine whether $\sum_{k=1}^{\infty} \frac{k}{(2k^2+1)^2}$ converges.

\begin{explanation}
Note that $\lim_{n \to \infty}  \frac{n}{(2n+1)^2} =0$ so the divergence test is inconclusive.  Also, this is not a geometric series.  Let's try the integral test with $f(x) =  \frac{x}{(2x+1)^2}$.  Notice that the function $f(x) = \frac{x}{(2x^2+1)^2}$ is continuous and positive on $[1,\infty)$.  We can check where it is decreasing part by computing $f'(x) = \frac{-6x^2+1}{(2x^2+1)^3} < 0$.  So, $f'(x)$ is certainly negative for $x >1$ and hence $f(x)$ is also decreasing on $[1,\infty)$.

Notice that, after performing a substitution if necessary,  

\[
\int \frac{x}{(2x+1)^2} \d x = -\frac{1}{8x^2+4} +C
\]

so $\lim_{b \to \infty} \eval{ -\frac{1}{8x^2+4}}_1^b = \frac{1}{8}$ and hence the improper integral $\int_1^{\infty}  \frac{x}{(2x+1)^2} \d x $ \wordChoice{\choice[correct]{converges}\choice{diverges}}.  Thus,  $\sum_{k=1}^{\infty} \frac{k}{(2k^2+1)^2}$ \wordChoice{\choice[correct]{converges}\choice{diverges}}.

\end{explanation}

\end{example}

The next examples synthesizes some concepts we have seen thus far.
\begin{example}
Let $a_n = \frac{1}{n^2+4}$ for $n \geq 1$ and let $s_n = \sum_{k=1}^n a_k$.  Determine if $\{a_n\}_{n=1}$ and $\{s_n\}_{n=1}$ are bounded and/or monotonic.

\begin{explanation}
Note that $\{a_n\}_{n=1}$ is decreasing (and hence monotonic) because denominator increases as $n$ increases.  Hence, the fraction becomes smaller as $n$ grows larger.  

$\{a_n\}_{n=1}$ is also bounded since $0 \leq a_n \leq 1$ for all $n$ (i.e. no terms are larger than $1$ or smaller than $0$).

For $s_n$, note that 

\[
s_n = s_{n-1} +a_n = s_{n-1} + \frac{1}{n^2+4}. 
\]

Since $ \frac{1}{n^2+4} >0$, we have that $s_{n+1} > s_n$ for all $n$, so $\{s_n\}_{n=1}$ is increasing (and hence monotonic).  To determine if $\{s_n\}_{n=1}$ bounded, we can determine whether $\lim_{n \to \infty} s_n$ exists.  Indeed, since $\{s_n\}_{n=1}$ is increasing, we have that $s_n < \lim_{n \to \infty} s_n$, so if the limit exists, it serves as an upper bound for all of the terms in the sequence.

We can write out the limit in this case and find

\[
\lim_{n \to \infty} s_n = \lim_{n \to \infty} \left[ \sum_{k=1}^n \frac{1}{k^2+4} \right] =  \sum_{k=1}^{\infty} \frac{1}{k^2+4}
\]
\end{explanation}

Thus, all we have to do is determine if the infinite series above converges.  

To do this, we can apply the integral test.  Let $f(x)= \frac{1}{x^2+4}$.  Notice that $f(x)$ is positive, continuous, and decreasing on $[1,\infty)$, and 

\[
\lim_{b \to \infty} \left[ \int_0^b \frac{1}{x^2+4} \right]= \lim_{b \to \infty} \eval{\frac{1}{2} \arctan\left(\frac{x}{2}\right)}_1^b =\frac{\pi}{4}- \frac{1}{2}\arctan\left(\frac{1}{2}\right).
\]

Since the improper integral converges, the series $\sum_{k=1}^{\infty} \frac{1}{k^2+4}$ converges.  Hence, $\lim_{n \to \infty}s_n$ exists so $\{s_n\}_{n=1}$ is bounded.


\end{example}
\section{p-Series}

A very important type of series for future sections is $\sum_{k=1}^\infty \frac{1}{k^p}$, where $p>0$.  We call a series that can be brought into this form a $p$-\emph{series}.  We want to determine for which values of $p$ these series converge and diverge.

Notice that in our model examples, both series were $p$-series.

\begin{itemize}
\item The harmonic series $\sum_{k=1}^{\infty} \frac{1}{k}$ is a $p$-series with $p = \answer{1}$.  It \wordChoice{\choice{converges}\choice[correct]{diverges}}.
\item The series $\sum_{k=1}^{\infty} \frac{1}{k^2}$ is a $p$-series with $p = \answer{2}$. It \wordChoice{\choice[correct]{converges}\choice{diverges}}.
\end{itemize}
  
 \begin{explanation}
    First, recall that we have already dealt with the case that $p = 1$, since 
    this case is exactly the harmonic series.  We have used the integral test 
    to show that the harmonic series diverges.
    Now, assume that $p \ne 1$.  By the integral test,
    \[
    \sum_{k=1}^\infty \frac{1}{k^p}
    \]
    converges if and only if
    \[
    \int_1^\infty \answer[given]{\frac{1}{x^p}} \d x
    \]
    converges. Let's check this, write with me:
    \begin{align*}
      \int_1^\infty \frac{1}{x^p} \d x &= \lim_{b\to\infty} \int_1^b x^{-p} \d x\\
      &= \lim_{b\to\infty} \eval{\answer[given]{\frac{x^{1-p}}{1-p}}}_1^b\\
      &= \lim_{b\to\infty} \left[ \frac{b^{1-p}}{1-p} - \frac{1^{1-p}}{1-p} \right] \\
      &= \lim_{b\to\infty} \left[\frac{b^{1-p}}{1-p}\right] - \frac{1}{1-p}
    \end{align*}
Note that if $p >1$, the exponent $1-p$ is negative so $ \lim_{b\to\infty} \left[\frac{b^{1-p}}{1-p}\right] = \answer{0}$.   If $p <1$, the exponent $1-p$ is psoitive so $ \lim_{b\to\infty} \left[\frac{b^{1-p}}{1-p}\right] = \answer{\infty}$.  Thus, $  \int_1^\infty \frac{1}{x^p} \d x$  only converges when $p>1$ and thus  $\sum_{k=1}^\infty \frac{1}{k^p}$ converges if and only if $p>1$.
  \end{explanation}

This result is important enough to list as a theorem.

\begin{theorem}[$p$-series test]\index{p-series test@$p$-test}
 The series $\sum_{k=1}^\infty \frac{1}{k^p}$ converges if and only if $p > 1$.
 \end{theorem}
 
\begin{question}
Which of the following series converge? (Select all that apply)

\begin{selectAll}
\choice{$\sum_{k=2}^{\infty} \frac{1}{\sqrt{k}}$} 
\choice[correct]{$\sum_{k=3}^{\infty} \frac{1}{k^{3/2}}$} 
\choice[correct]{$\sum_{k=5}^{\infty} k^{-4}$} 
\choice{$\sum_{k=1}^{\infty} \frac{1}{k^{.1}}$} 
\end{selectAll}
\end{question}

\end{document}


