\documentclass{ximera}

%\usepackage{todonotes}
%\usepackage{mathtools} %% Required for wide table Curl and Greens
%\usepackage{cuted} %% Required for wide table Curl and Greens
\newcommand{\todo}{}

\usepackage{esint} % for \oiint
\ifxake%%https://math.meta.stackexchange.com/questions/9973/how-do-you-render-a-closed-surface-double-integral
\renewcommand{\oiint}{{\large\bigcirc}\kern-1.56em\iint}
\fi


\graphicspath{
  {./}
  {ximeraTutorial/}
  {basicPhilosophy/}
  {functionsOfSeveralVariables/}
  {normalVectors/}
  {lagrangeMultipliers/}
  {vectorFields/}
  {greensTheorem/}
  {shapeOfThingsToCome/}
  {dotProducts/}
  {partialDerivativesAndTheGradientVector/}
  {../productAndQuotientRules/exercises/}
  {../normalVectors/exercisesParametricPlots/}
  {../continuityOfFunctionsOfSeveralVariables/exercises/}
  {../partialDerivativesAndTheGradientVector/exercises/}
  {../directionalDerivativeAndChainRule/exercises/}
  {../commonCoordinates/exercisesCylindricalCoordinates/}
  {../commonCoordinates/exercisesSphericalCoordinates/}
  {../greensTheorem/exercisesCurlAndLineIntegrals/}
  {../greensTheorem/exercisesDivergenceAndLineIntegrals/}
  {../shapeOfThingsToCome/exercisesDivergenceTheorem/}
  {../greensTheorem/}
  {../shapeOfThingsToCome/}
  {../separableDifferentialEquations/exercises/}
  {vectorFields/}
}

\newcommand{\mooculus}{\textsf{\textbf{MOOC}\textnormal{\textsf{ULUS}}}}

\usepackage{tkz-euclide}\usepackage{tikz}
\usepackage{tikz-cd}
\usetikzlibrary{arrows}
\tikzset{>=stealth,commutative diagrams/.cd,
  arrow style=tikz,diagrams={>=stealth}} %% cool arrow head
\tikzset{shorten <>/.style={ shorten >=#1, shorten <=#1 } } %% allows shorter vectors

\usetikzlibrary{backgrounds} %% for boxes around graphs
\usetikzlibrary{shapes,positioning}  %% Clouds and stars
\usetikzlibrary{matrix} %% for matrix
\usepgfplotslibrary{polar} %% for polar plots
\usepgfplotslibrary{fillbetween} %% to shade area between curves in TikZ
\usetkzobj{all}
\usepackage[makeroom]{cancel} %% for strike outs
%\usepackage{mathtools} %% for pretty underbrace % Breaks Ximera
%\usepackage{multicol}
\usepackage{pgffor} %% required for integral for loops



%% http://tex.stackexchange.com/questions/66490/drawing-a-tikz-arc-specifying-the-center
%% Draws beach ball
\tikzset{pics/carc/.style args={#1:#2:#3}{code={\draw[pic actions] (#1:#3) arc(#1:#2:#3);}}}



\usepackage{array}
\setlength{\extrarowheight}{+.1cm}
\newdimen\digitwidth
\settowidth\digitwidth{9}
\def\divrule#1#2{
\noalign{\moveright#1\digitwidth
\vbox{\hrule width#2\digitwidth}}}





\newcommand{\RR}{\mathbb R}
\newcommand{\R}{\mathbb R}
\newcommand{\N}{\mathbb N}
\newcommand{\Z}{\mathbb Z}

\newcommand{\sagemath}{\textsf{SageMath}}


%\renewcommand{\d}{\,d\!}
\renewcommand{\d}{\mathop{}\!d}
\newcommand{\dd}[2][]{\frac{\d #1}{\d #2}}
\newcommand{\pp}[2][]{\frac{\partial #1}{\partial #2}}
\renewcommand{\l}{\ell}
\newcommand{\ddx}{\frac{d}{\d x}}

\newcommand{\zeroOverZero}{\ensuremath{\boldsymbol{\tfrac{0}{0}}}}
\newcommand{\inftyOverInfty}{\ensuremath{\boldsymbol{\tfrac{\infty}{\infty}}}}
\newcommand{\zeroOverInfty}{\ensuremath{\boldsymbol{\tfrac{0}{\infty}}}}
\newcommand{\zeroTimesInfty}{\ensuremath{\small\boldsymbol{0\cdot \infty}}}
\newcommand{\inftyMinusInfty}{\ensuremath{\small\boldsymbol{\infty - \infty}}}
\newcommand{\oneToInfty}{\ensuremath{\boldsymbol{1^\infty}}}
\newcommand{\zeroToZero}{\ensuremath{\boldsymbol{0^0}}}
\newcommand{\inftyToZero}{\ensuremath{\boldsymbol{\infty^0}}}



\newcommand{\numOverZero}{\ensuremath{\boldsymbol{\tfrac{\#}{0}}}}
\newcommand{\dfn}{\textbf}
%\newcommand{\unit}{\,\mathrm}
\newcommand{\unit}{\mathop{}\!\mathrm}
\newcommand{\eval}[1]{\bigg[ #1 \bigg]}
\newcommand{\seq}[1]{\left( #1 \right)}
\renewcommand{\epsilon}{\varepsilon}
\renewcommand{\phi}{\varphi}


\renewcommand{\iff}{\Leftrightarrow}

\DeclareMathOperator{\arccot}{arccot}
\DeclareMathOperator{\arcsec}{arcsec}
\DeclareMathOperator{\arccsc}{arccsc}
\DeclareMathOperator{\si}{Si}
\DeclareMathOperator{\scal}{scal}
\DeclareMathOperator{\sign}{sign}


%% \newcommand{\tightoverset}[2]{% for arrow vec
%%   \mathop{#2}\limits^{\vbox to -.5ex{\kern-0.75ex\hbox{$#1$}\vss}}}
\newcommand{\arrowvec}[1]{{\overset{\rightharpoonup}{#1}}}
%\renewcommand{\vec}[1]{\arrowvec{\mathbf{#1}}}
\renewcommand{\vec}[1]{{\overset{\boldsymbol{\rightharpoonup}}{\mathbf{#1}}}\hspace{0in}}

\newcommand{\point}[1]{\left(#1\right)} %this allows \vector{ to be changed to \vector{ with a quick find and replace
\newcommand{\pt}[1]{\mathbf{#1}} %this allows \vec{ to be changed to \vec{ with a quick find and replace
\newcommand{\Lim}[2]{\lim_{\point{#1} \to \point{#2}}} %Bart, I changed this to point since I want to use it.  It runs through both of the exercise and exerciseE files in limits section, which is why it was in each document to start with.

\DeclareMathOperator{\proj}{\mathbf{proj}}
\newcommand{\veci}{{\boldsymbol{\hat{\imath}}}}
\newcommand{\vecj}{{\boldsymbol{\hat{\jmath}}}}
\newcommand{\veck}{{\boldsymbol{\hat{k}}}}
\newcommand{\vecl}{\vec{\boldsymbol{\l}}}
\newcommand{\uvec}[1]{\mathbf{\hat{#1}}}
\newcommand{\utan}{\mathbf{\hat{t}}}
\newcommand{\unormal}{\mathbf{\hat{n}}}
\newcommand{\ubinormal}{\mathbf{\hat{b}}}

\newcommand{\dotp}{\bullet}
\newcommand{\cross}{\boldsymbol\times}
\newcommand{\grad}{\boldsymbol\nabla}
\newcommand{\divergence}{\grad\dotp}
\newcommand{\curl}{\grad\cross}
%\DeclareMathOperator{\divergence}{divergence}
%\DeclareMathOperator{\curl}[1]{\grad\cross #1}
\newcommand{\lto}{\mathop{\longrightarrow\,}\limits}

\renewcommand{\bar}{\overline}

\colorlet{textColor}{black}
\colorlet{background}{white}
\colorlet{penColor}{blue!50!black} % Color of a curve in a plot
\colorlet{penColor2}{red!50!black}% Color of a curve in a plot
\colorlet{penColor3}{red!50!blue} % Color of a curve in a plot
\colorlet{penColor4}{green!50!black} % Color of a curve in a plot
\colorlet{penColor5}{orange!80!black} % Color of a curve in a plot
\colorlet{penColor6}{yellow!70!black} % Color of a curve in a plot
\colorlet{fill1}{penColor!20} % Color of fill in a plot
\colorlet{fill2}{penColor2!20} % Color of fill in a plot
\colorlet{fillp}{fill1} % Color of positive area
\colorlet{filln}{penColor2!20} % Color of negative area
\colorlet{fill3}{penColor3!20} % Fill
\colorlet{fill4}{penColor4!20} % Fill
\colorlet{fill5}{penColor5!20} % Fill
\colorlet{gridColor}{gray!50} % Color of grid in a plot

\newcommand{\surfaceColor}{violet}
\newcommand{\surfaceColorTwo}{redyellow}
\newcommand{\sliceColor}{greenyellow}




\pgfmathdeclarefunction{gauss}{2}{% gives gaussian
  \pgfmathparse{1/(#2*sqrt(2*pi))*exp(-((x-#1)^2)/(2*#2^2))}%
}


%%%%%%%%%%%%%
%% Vectors
%%%%%%%%%%%%%

%% Simple horiz vectors
\renewcommand{\vector}[1]{\left\langle #1\right\rangle}


%% %% Complex Horiz Vectors with angle brackets
%% \makeatletter
%% \renewcommand{\vector}[2][ , ]{\left\langle%
%%   \def\nextitem{\def\nextitem{#1}}%
%%   \@for \el:=#2\do{\nextitem\el}\right\rangle%
%% }
%% \makeatother

%% %% Vertical Vectors
%% \def\vector#1{\begin{bmatrix}\vecListA#1,,\end{bmatrix}}
%% \def\vecListA#1,{\if,#1,\else #1\cr \expandafter \vecListA \fi}

%%%%%%%%%%%%%
%% End of vectors
%%%%%%%%%%%%%

%\newcommand{\fullwidth}{}
%\newcommand{\normalwidth}{}



%% makes a snazzy t-chart for evaluating functions
%\newenvironment{tchart}{\rowcolors{2}{}{background!90!textColor}\array}{\endarray}

%%This is to help with formatting on future title pages.
\newenvironment{sectionOutcomes}{}{}



%% Flowchart stuff
%\tikzstyle{startstop} = [rectangle, rounded corners, minimum width=3cm, minimum height=1cm,text centered, draw=black]
%\tikzstyle{question} = [rectangle, minimum width=3cm, minimum height=1cm, text centered, draw=black]
%\tikzstyle{decision} = [trapezium, trapezium left angle=70, trapezium right angle=110, minimum width=3cm, minimum height=1cm, text centered, draw=black]
%\tikzstyle{question} = [rectangle, rounded corners, minimum width=3cm, minimum height=1cm,text centered, draw=black]
%\tikzstyle{process} = [rectangle, minimum width=3cm, minimum height=1cm, text centered, draw=black]
%\tikzstyle{decision} = [trapezium, trapezium left angle=70, trapezium right angle=110, minimum width=3cm, minimum height=1cm, text centered, draw=black]

\pgfplotsset{compat=1.13}  %%%JIM
\author{Jim Talamo}
\license{Creative Commons 3.0 By-bC}


\outcome{}


\begin{document}
\begin{exercise}
Consider the power series $f(x) = \sum_{k=0}^{\infty} \frac{2^k}{5k-1}(x-1)^{k}$.

We want to determine the radius and interval of convergence for this power series. 

First, we use the Ratio Test to determine the radius of convergence. 

To do this, we'll think of the power series as a sum of functions of $x$ by writing: 

\[
\sum_{k=0}^{\infty} \frac{2^k}{5k-1}(x-1)^{k} = \sum_{k=0}^{\infty} b_k(x)
\]

We need to determine the limit $L(x) = \lim_{k \to \infty} \left| \frac{b_{k+1}(x)}{b_k(x)}\right|$, where we have explicitly indicated here that this limit likely depends on the $x$-value we choose. 

We calculate $b_{k+1}(x)=\answer{\frac{2^{k+1}}{5k+4}(x-1)^{k+1}}$ and $b_k(x)=\answer{\frac{2^k}{5k-1}(x-1)^k}$. 

\begin{exercise}

Simplifying the ratio $\left|\frac{b_{k+1}}{b_k}\right|$ gives us $\left|\frac{b_{k+1}}{b_k}\right|=\answer{\frac{2(5k-1)}{5k-4} |x-1|}$.


\begin{exercise}

Since the $|x-1|$ term does not depend on $k$, we can factor it out of the limit and then calculate:

\[
L(x) = |x-1| \lim_{k \to \infty}   \frac{2(5k-1)}{5k-4}  =\answer{2 |x-1|}
\]

\begin{exercise}

Recall the Ratio Test tells us that a series $\sum^{\infty}_{k=1} b_k$ converges if $L <1$ where $L=\lim_{k \to \infty}\left| \frac{b_{k+1}}{b_k}\right|$. 

Thus, in order to determine the set of $x$ for which our power series converges, we need to determine those $x$-values that satisfy the inequality $2|x-1| <1$. 

Rewriting this inequality we obtain $|x-1|< \answer{\frac{1}{2}}$. 

Thus, the  power series $\sum_{k=0}^{\infty} \frac{2^k}{5k-1}(x-1)^{k}$ has radius of convergence $\answer{\frac{1}{2}}$.

\begin{exercise}

Now, recall that we can interpret an inequality of the form:

\[
|x-c|<R
\]

as the set of all points that are at most $R$ units from $x=c$.  Here, $R=\answer{\frac{1}{2}}$ and $c=1$.

\begin{image}
\begin{tikzpicture}

\begin{axis}
	[
	domain=-6:3,
	axis x line = middle,
	axis line style=-,
	axis y line = none,
	xmin=-6.5,
	xmax=3.5,
	ymin=-1,
	ymax=1,
	xtick={-1.5},
	xticklabels={},
	]
	
	\draw[very thick,penColor] (-3,0) -- (0,0);
	\draw[->,very thick,penColor2] (-3,0) -- (-6.5,0);
	\draw[->,very thick,penColor2] (0,0) -- (3.5,0);
%Vertical LInes%%%
	\draw[->,thick,black,dashed] (-3,-.5) -- (-3,.3);
	\draw[->,thick,black,dashed] (0,-.5) -- (0,.3);
	
%arrows and ROC
	\draw[->,thick,penColor4] (-1.4,.15) -- (0,.15);
	\draw[->,thick,penColor4] (-1.6,.15) -- (-3,.15);
      	\node[penColor4] at (-.75,.22) {\small{$1/2$}};
        \node[penColor4] at (-2.3,.22) {\small{$1/2$}};
	
	\node at (-3,0) [scale=.5,shape=circle, fill=white, draw=penColor] {};
	%\node at (-1.5,0) [scale=.5,shape=circle, fill=penColor] {};
	\node at (-1.5,-.05) [below, penColor] {$1$};
	\node at (0,0) [scale=.5,shape=circle, fill=white, draw=penColor] {};
	\addplot[draw=none] coordinates {(-4.5,0)};
	\node at (-1.5,0) [scale=.5,shape=circle, fill=black, draw=penColor] {};
	\addplot[draw=none] coordinates {(-4.5,0)};


%	\node[excl] at (1.5,0) [scale=.5,shape=circle, fill=penColor2] {};

%%BRACES%%%%%%%
	\draw [penColor2,thick,decoration={brace,mirror,raise=2em},decorate] 
        (axis cs:-6.5,0) --
        (axis cs:-3,0);
    \draw [penColor,thick,decoration={brace,mirror,raise=2em},decorate] 
        (axis cs:-3,0) --
        (axis cs:0,0);
    \draw [penColor2,thick,decoration={brace,mirror,raise=2em},decorate] 
        (axis cs:0,0) --
        (axis cs:3.5,0);
        
 %%%text%%%
    \node[penColor2] at (-5,-.4) {diverges};
    \node[penColor] at (-1.5,-.41) {converges};
    \node[penColor2] at (2,-.4) {diverges};

    \node[black] at (-1.5,.5) {We need to check for convergence};
        \node[black] at (-1.5,.4) {separately at the endpoints.};
\end{axis}

\end{tikzpicture}
\end{image}

The lefthand endpoint of the interval of convergence is $x=\answer{\frac{1}{2}}$ and the righthand endpoint is $x= \answer{\frac{3}{2}}$.

\begin{hint}
The lefthand endpoint will be $c-R$ ($R$ units left of $x=c$) while the righthand one will be $c+R$ ($R$ units right of $x=c$) 
\end{hint}

\begin{exercise} 
Now we need to determine if our power series converges at the endpoints $x=\frac{1}{2}$ and $x=\frac{3}{2}$.  Note that these are precisely the points where $L(x)=1$, and the Ratio test is thus inconclusive, so we will have to appeal to another test to determine whether the resulting series will converge or diverge.

First we test convergence at $x=\frac{1}{2}$. 

We substitute $x=\frac{1}{2}$ into $f(x) = \sum_{k=0}^{\infty} \frac{2^k}{5k-1}(x-1)^{k}$ to obtain:

\[
f\left(\frac{1}{2}\right) = \sum^{\infty}_{k=0} \answer{\frac{(-1)^k}{5k-1}}
\]

Note that there is no geometric growth or decay for this series anymore since $(-1)^k$ is alternating.

This series: 

\begin{multipleChoice}
\choice[correct]{converges by the Alternating Series Test}
\choice{diverges}
\end{multipleChoice}

Now we plug $x=\frac{3}{2}$ into $\sum_{k=0}^{\infty} \frac{2^k}{5k-1}(x-1)^{k}$ to obtain:

\[
\sum^{\infty}_{k=0} \frac{1}{5k-1}
\]

Which method should we use to test for convergence?

\begin{multipleChoice}
\choice{Use the Alternating Series Test.}
\choice{Use the Ratio or Root Test.}
\choice{Use the Divergence Test.}
\choice[correct]{Use the Limit Comparison Test.}
\end{multipleChoice}

Using this test, the series: 
\begin{multipleChoice}
\choice{converges.}
\choice[correct]{diverges.}
\end{multipleChoice}

\begin{exercise}
The interval of convergence of the power series $\sum_{k=0}^{\infty} \frac{2^k}{5k-1}(x-1)^{k}$ is:
\begin{multipleChoice}
\choice{$\left(\frac{1}{2},\frac{3}{2}\right)$}
\choice{$\left(\frac{1}{2},\frac{3}{2}\right]$}
\choice[correct]{$\left[\frac{1}{2},\frac{3}{2}\right)$}
\choice{$\left[\frac{1}{2},\frac{3}{2}\right]$}
\end{multipleChoice}







\end{exercise}

\end{exercise}
\end{exercise}
\end{exercise}
\end{exercise}
\end{exercise}
\end{exercise}
\end{document}
