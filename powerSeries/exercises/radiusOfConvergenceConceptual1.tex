\documentclass{ximera}

%\usepackage{todonotes}
%\usepackage{mathtools} %% Required for wide table Curl and Greens
%\usepackage{cuted} %% Required for wide table Curl and Greens
\newcommand{\todo}{}

\usepackage{esint} % for \oiint
\ifxake%%https://math.meta.stackexchange.com/questions/9973/how-do-you-render-a-closed-surface-double-integral
\renewcommand{\oiint}{{\large\bigcirc}\kern-1.56em\iint}
\fi


\graphicspath{
  {./}
  {ximeraTutorial/}
  {basicPhilosophy/}
  {functionsOfSeveralVariables/}
  {normalVectors/}
  {lagrangeMultipliers/}
  {vectorFields/}
  {greensTheorem/}
  {shapeOfThingsToCome/}
  {dotProducts/}
  {partialDerivativesAndTheGradientVector/}
  {../productAndQuotientRules/exercises/}
  {../normalVectors/exercisesParametricPlots/}
  {../continuityOfFunctionsOfSeveralVariables/exercises/}
  {../partialDerivativesAndTheGradientVector/exercises/}
  {../directionalDerivativeAndChainRule/exercises/}
  {../commonCoordinates/exercisesCylindricalCoordinates/}
  {../commonCoordinates/exercisesSphericalCoordinates/}
  {../greensTheorem/exercisesCurlAndLineIntegrals/}
  {../greensTheorem/exercisesDivergenceAndLineIntegrals/}
  {../shapeOfThingsToCome/exercisesDivergenceTheorem/}
  {../greensTheorem/}
  {../shapeOfThingsToCome/}
  {../separableDifferentialEquations/exercises/}
  {vectorFields/}
}

\newcommand{\mooculus}{\textsf{\textbf{MOOC}\textnormal{\textsf{ULUS}}}}

\usepackage{tkz-euclide}\usepackage{tikz}
\usepackage{tikz-cd}
\usetikzlibrary{arrows}
\tikzset{>=stealth,commutative diagrams/.cd,
  arrow style=tikz,diagrams={>=stealth}} %% cool arrow head
\tikzset{shorten <>/.style={ shorten >=#1, shorten <=#1 } } %% allows shorter vectors

\usetikzlibrary{backgrounds} %% for boxes around graphs
\usetikzlibrary{shapes,positioning}  %% Clouds and stars
\usetikzlibrary{matrix} %% for matrix
\usepgfplotslibrary{polar} %% for polar plots
\usepgfplotslibrary{fillbetween} %% to shade area between curves in TikZ
\usetkzobj{all}
\usepackage[makeroom]{cancel} %% for strike outs
%\usepackage{mathtools} %% for pretty underbrace % Breaks Ximera
%\usepackage{multicol}
\usepackage{pgffor} %% required for integral for loops



%% http://tex.stackexchange.com/questions/66490/drawing-a-tikz-arc-specifying-the-center
%% Draws beach ball
\tikzset{pics/carc/.style args={#1:#2:#3}{code={\draw[pic actions] (#1:#3) arc(#1:#2:#3);}}}



\usepackage{array}
\setlength{\extrarowheight}{+.1cm}
\newdimen\digitwidth
\settowidth\digitwidth{9}
\def\divrule#1#2{
\noalign{\moveright#1\digitwidth
\vbox{\hrule width#2\digitwidth}}}





\newcommand{\RR}{\mathbb R}
\newcommand{\R}{\mathbb R}
\newcommand{\N}{\mathbb N}
\newcommand{\Z}{\mathbb Z}

\newcommand{\sagemath}{\textsf{SageMath}}


%\renewcommand{\d}{\,d\!}
\renewcommand{\d}{\mathop{}\!d}
\newcommand{\dd}[2][]{\frac{\d #1}{\d #2}}
\newcommand{\pp}[2][]{\frac{\partial #1}{\partial #2}}
\renewcommand{\l}{\ell}
\newcommand{\ddx}{\frac{d}{\d x}}

\newcommand{\zeroOverZero}{\ensuremath{\boldsymbol{\tfrac{0}{0}}}}
\newcommand{\inftyOverInfty}{\ensuremath{\boldsymbol{\tfrac{\infty}{\infty}}}}
\newcommand{\zeroOverInfty}{\ensuremath{\boldsymbol{\tfrac{0}{\infty}}}}
\newcommand{\zeroTimesInfty}{\ensuremath{\small\boldsymbol{0\cdot \infty}}}
\newcommand{\inftyMinusInfty}{\ensuremath{\small\boldsymbol{\infty - \infty}}}
\newcommand{\oneToInfty}{\ensuremath{\boldsymbol{1^\infty}}}
\newcommand{\zeroToZero}{\ensuremath{\boldsymbol{0^0}}}
\newcommand{\inftyToZero}{\ensuremath{\boldsymbol{\infty^0}}}



\newcommand{\numOverZero}{\ensuremath{\boldsymbol{\tfrac{\#}{0}}}}
\newcommand{\dfn}{\textbf}
%\newcommand{\unit}{\,\mathrm}
\newcommand{\unit}{\mathop{}\!\mathrm}
\newcommand{\eval}[1]{\bigg[ #1 \bigg]}
\newcommand{\seq}[1]{\left( #1 \right)}
\renewcommand{\epsilon}{\varepsilon}
\renewcommand{\phi}{\varphi}


\renewcommand{\iff}{\Leftrightarrow}

\DeclareMathOperator{\arccot}{arccot}
\DeclareMathOperator{\arcsec}{arcsec}
\DeclareMathOperator{\arccsc}{arccsc}
\DeclareMathOperator{\si}{Si}
\DeclareMathOperator{\scal}{scal}
\DeclareMathOperator{\sign}{sign}


%% \newcommand{\tightoverset}[2]{% for arrow vec
%%   \mathop{#2}\limits^{\vbox to -.5ex{\kern-0.75ex\hbox{$#1$}\vss}}}
\newcommand{\arrowvec}[1]{{\overset{\rightharpoonup}{#1}}}
%\renewcommand{\vec}[1]{\arrowvec{\mathbf{#1}}}
\renewcommand{\vec}[1]{{\overset{\boldsymbol{\rightharpoonup}}{\mathbf{#1}}}\hspace{0in}}

\newcommand{\point}[1]{\left(#1\right)} %this allows \vector{ to be changed to \vector{ with a quick find and replace
\newcommand{\pt}[1]{\mathbf{#1}} %this allows \vec{ to be changed to \vec{ with a quick find and replace
\newcommand{\Lim}[2]{\lim_{\point{#1} \to \point{#2}}} %Bart, I changed this to point since I want to use it.  It runs through both of the exercise and exerciseE files in limits section, which is why it was in each document to start with.

\DeclareMathOperator{\proj}{\mathbf{proj}}
\newcommand{\veci}{{\boldsymbol{\hat{\imath}}}}
\newcommand{\vecj}{{\boldsymbol{\hat{\jmath}}}}
\newcommand{\veck}{{\boldsymbol{\hat{k}}}}
\newcommand{\vecl}{\vec{\boldsymbol{\l}}}
\newcommand{\uvec}[1]{\mathbf{\hat{#1}}}
\newcommand{\utan}{\mathbf{\hat{t}}}
\newcommand{\unormal}{\mathbf{\hat{n}}}
\newcommand{\ubinormal}{\mathbf{\hat{b}}}

\newcommand{\dotp}{\bullet}
\newcommand{\cross}{\boldsymbol\times}
\newcommand{\grad}{\boldsymbol\nabla}
\newcommand{\divergence}{\grad\dotp}
\newcommand{\curl}{\grad\cross}
%\DeclareMathOperator{\divergence}{divergence}
%\DeclareMathOperator{\curl}[1]{\grad\cross #1}
\newcommand{\lto}{\mathop{\longrightarrow\,}\limits}

\renewcommand{\bar}{\overline}

\colorlet{textColor}{black}
\colorlet{background}{white}
\colorlet{penColor}{blue!50!black} % Color of a curve in a plot
\colorlet{penColor2}{red!50!black}% Color of a curve in a plot
\colorlet{penColor3}{red!50!blue} % Color of a curve in a plot
\colorlet{penColor4}{green!50!black} % Color of a curve in a plot
\colorlet{penColor5}{orange!80!black} % Color of a curve in a plot
\colorlet{penColor6}{yellow!70!black} % Color of a curve in a plot
\colorlet{fill1}{penColor!20} % Color of fill in a plot
\colorlet{fill2}{penColor2!20} % Color of fill in a plot
\colorlet{fillp}{fill1} % Color of positive area
\colorlet{filln}{penColor2!20} % Color of negative area
\colorlet{fill3}{penColor3!20} % Fill
\colorlet{fill4}{penColor4!20} % Fill
\colorlet{fill5}{penColor5!20} % Fill
\colorlet{gridColor}{gray!50} % Color of grid in a plot

\newcommand{\surfaceColor}{violet}
\newcommand{\surfaceColorTwo}{redyellow}
\newcommand{\sliceColor}{greenyellow}




\pgfmathdeclarefunction{gauss}{2}{% gives gaussian
  \pgfmathparse{1/(#2*sqrt(2*pi))*exp(-((x-#1)^2)/(2*#2^2))}%
}


%%%%%%%%%%%%%
%% Vectors
%%%%%%%%%%%%%

%% Simple horiz vectors
\renewcommand{\vector}[1]{\left\langle #1\right\rangle}


%% %% Complex Horiz Vectors with angle brackets
%% \makeatletter
%% \renewcommand{\vector}[2][ , ]{\left\langle%
%%   \def\nextitem{\def\nextitem{#1}}%
%%   \@for \el:=#2\do{\nextitem\el}\right\rangle%
%% }
%% \makeatother

%% %% Vertical Vectors
%% \def\vector#1{\begin{bmatrix}\vecListA#1,,\end{bmatrix}}
%% \def\vecListA#1,{\if,#1,\else #1\cr \expandafter \vecListA \fi}

%%%%%%%%%%%%%
%% End of vectors
%%%%%%%%%%%%%

%\newcommand{\fullwidth}{}
%\newcommand{\normalwidth}{}



%% makes a snazzy t-chart for evaluating functions
%\newenvironment{tchart}{\rowcolors{2}{}{background!90!textColor}\array}{\endarray}

%%This is to help with formatting on future title pages.
\newenvironment{sectionOutcomes}{}{}



%% Flowchart stuff
%\tikzstyle{startstop} = [rectangle, rounded corners, minimum width=3cm, minimum height=1cm,text centered, draw=black]
%\tikzstyle{question} = [rectangle, minimum width=3cm, minimum height=1cm, text centered, draw=black]
%\tikzstyle{decision} = [trapezium, trapezium left angle=70, trapezium right angle=110, minimum width=3cm, minimum height=1cm, text centered, draw=black]
%\tikzstyle{question} = [rectangle, rounded corners, minimum width=3cm, minimum height=1cm,text centered, draw=black]
%\tikzstyle{process} = [rectangle, minimum width=3cm, minimum height=1cm, text centered, draw=black]
%\tikzstyle{decision} = [trapezium, trapezium left angle=70, trapezium right angle=110, minimum width=3cm, minimum height=1cm, text centered, draw=black]


\author{Jim Talamo}
\license{Creative Commons 3.0 By-bC}

\outcome{Answer conceptual questions about the radius an interval of convergence}

\begin{document}
\begin{exercise}
Suppose that $f(x) = \sum_{k=0}^{\infty} a_k (x+1)^k$ and it is known that:

\begin{itemize}
\item $\sum_{k=0}^{\infty} 4^k a_k$ converges.
\item The series represented by $f(-6)$ diverges.
\end{itemize}


%%%%%%%%%%%%%%%%%%%%%%%%%%%%%%%

Answer the following questions:

\begin{exercise}
Select all of the series below that \emph{MUST} converge.
\begin{selectAll}
\choice[correct]{$\sum_{k=0}^{\infty} 2^ka_k$}
\choice[correct]{$\sum_{k=0}^{\infty} 4^k a_k$}
\choice[correct]{$\sum_{k=0}^{\infty} \left(\frac{1}{3}\right)^k a_k$}
\choice{The series represented by $f(-5.5)$}
\choice[correct]{The series represented by $f(-1)$}
\end{selectAll}
\end{exercise}

\begin{exercise}
Select all of the series below that \emph{MUST} diverge.
\begin{selectAll}
\choice[correct]{$\sum_{k=0}^{\infty} (-5)^ka_k$}
\choice[correct]{$\sum_{k=0}^{\infty} 10^k a_k$}
\choice{The series represented by $f(-5.5)$}
\choice{The series represented by $f(2)$}
\end{selectAll}
\end{exercise}

\begin{exercise}
The series represented by $f(5)$:

\begin{multipleChoice}
\choice{must converge.}
\choice[correct]{must diverge.}
\choice{could converge or diverge; more information is needed.}
\end{multipleChoice}
\end{exercise}


%%%%%%%%%%%%%%%%%%%%%%%%%%%%%
%%%%%%%%%%%%%HINT%%%%%%%%%%%%%
%%%%%%%%%%%%%%%%%%%%%%%%%%%%%
\begin{hint}

%%%%%%%%%%%%%%%%%%%%%%%%%%%%%
\begin{question}
Let's explore what the first condition tells us.  First, note that to relate the given information to the function $f(x)$, we need to find an $x$-value for which the function equals $\sum_{k=0}^{\infty} 4^k a_k$.  Indeed, we find $f(\answer{3}) = \sum_{k=0}^{\infty} a_k$.  

\begin{question}
The series represented by $f(x)$ when $x=3$ converges.  Since the center of this series is at $x=\answer{-1}$, $x=3$ is $\answer{4}$ units from the center of the series.  Thus, the \emph{minimum} possible radius of convergence is $\answer{4}$.
\end{question}
\end{question}

%%%%%%%%%%%%%%%%%%%%%%%%%%%%%

\begin{question}
Let's explore what the second condition tells us.  The series represented by $f(x)$ when $x=-6$ diverges.  Since the center of this series is at $x=\answer{-1}$ and $x=-6$ is $\answer{5}$ units from the center of the series, the \emph{maximum} possible radius of convergence is $\answer{5}$.
\end{question}

%%%%%%%%%%%%%%%%%%%%%%%%%%%%%
\begin{question}
Now, we can sketch the minimum and maximum possible interval of convergence, and use it to answer answer many questions.  We start by indicating on a number line where the series must converge, must diverge, and where more information is needed:

%%%%%%%%%BEGIN IMAGE%%%%%%%%%%%  
\begin{image}
\begin{tikzpicture}

\begin{axis}
	[
	domain=-4:10,
	axis x line = middle,
	axis line style=-,
	axis y line = none,
	xmin=-5,
	xmax=11,
	ymin=-1,
	ymax=1,
	xtick={-1.5},
	xticklabels={},
	]
%%%%number line%%%%	
	\draw[very thick,penColor] (1,0) -- (5,0);
	\draw[->,very thick,penColor2] (-1.5,0) -- (-5,0);
	\draw[->,very thick,penColor2] (7.5,0) -- (11,0);
	\draw[very thick,penColor4] (-1.5,0) -- (1,0);
	\draw[very thick,penColor4] (5,0) -- (7.5,0);

%Vertical LInes%%%
	\draw[thick,black,dashed] (-1.5,-.25) -- (-1.5,.25);
	\draw[thick,black,dashed] (1,-.25) -- (1,.25);
	\draw[thick,black,dashed] (5,-.25) -- (5,.25);
	\draw[thick,black,dashed] (7.5,-.25) -- (7.5,.25);

%%%%Center%%%%%	
	\node at (3,-.05) [below, penColor] {$-1$};
	\draw[very thick,penColor] (3,-.05) -- (3,.05);
	
	
%%%%CONVERGENCE: arrows and minimum IOC
	\draw[->,thick,black!75] (3.1,.1) -- (5,.1);
	\draw[->,thick,black!75] (2.9,.1) -- (1,.1);
      	\node[black] at (4.05,.17) {\small{$4$}};
        \node[black] at (2.05,.17) {\small{$4$}};
       	%%%%%%brace and text
        \draw [penColor,thick,decoration={brace,mirror,raise=2em},decorate] 
        (axis cs:1,0) --
        (axis cs:5,0);
        \node[penColor] at (3,-.41) {converges};
	%%%Endpoints
	\node at (1,0) [scale=.5,shape=circle, draw=penColor4, fill=white] {};
	\node at (.4,-.05) [below, black] {$-5$};
	\node at (5,0) [scale=.5,shape=circle, fill=penColor, draw=penColor] {};
	\node at (5.3,-.05) [below, black] {$3$};    
	
%%%%DIVERGENCE: arrows and maximum IOC
	\draw[->,thick,black!50] (3.1,.05) -- (7.5,.05);
	\draw[->,thick,black!50] (2.9,.05) -- (-1.5,.05);
      	\node[black] at (6.35,.12) {\small{$5$}};
        \node[black] at (-.25,.12) {\small{$5$}};
        %%%%%%brace and text
    	\draw [penColor2,thick,decoration={brace,mirror,raise=2em},decorate] 
        (axis cs:7.5,0) --
        (axis cs:11,0);
    	\node[penColor2] at (9.5,-.4) {diverges};
    	\draw [penColor2,thick,decoration={brace,raise=2em},decorate] 
        (axis cs:-1.5,0) --
        (axis cs:-5,0);
	\node[penColor2] at (-3.5,-.4) {diverges};
	%%%Endpoints
	\node at (-1.5,0) [scale=.5,shape=circle, draw=penColor2, fill=penColor2] {};
	\node at (-2.2,-.05) [below, black] {$-6$};
	\node at (7.5,0) [scale=.5,shape=circle, fill=white, draw=penColor4] {};
	\node at (8,-.05) [below, black] {$4$};
	%\node at (-1.5,0) [scale=.5,shape=circle, fill=penColor] {};        

%%%%no man's land

    \draw [penColor4,thick,decoration={brace,raise=2em},decorate] 
        (axis cs:-1.5,0) --
        (axis cs:1,0);

    \draw [penColor4,thick,decoration={brace,raise=2em},decorate] 
        (axis cs:5,0) --
        (axis cs:7.5,0);

    \node[penColor4] at (3,.4) {may or may not converge};
\end{axis}

\end{tikzpicture}
\end{image}

%%%%%%%%%END IMAGE%%%%%%%%%%%

Note that an open circle indicates that convergence cannot be determined without further information, while each closed circle has been colored coded with the choice of color for convergence and divergence.

Thus:

\begin{question}
The series the series represented by $f(-6)$, which is $\sum_{k=0}^{\infty} \left(\answer{-5}\right)^k a_k$:
\begin{multipleChoice}
\choice{converges}
\choice[correct]{diverges}
\choice{could converge or diverge; more information is needed.}
\end{multipleChoice}
\end{question}

 \begin{question}
The series the series represented by $f(-5)$, which is $\sum_{k=0}^{\infty} \left(\answer{-4}\right)^k a_k$:
\begin{multipleChoice}
\choice{converges}
\choice{diverges}
\choice[correct]{could converge or diverge; more information is needed.}
\end{multipleChoice}
\end{question}

\begin{question}
The series the series represented by $f(3)$, which is $\sum_{k=0}^{\infty} \left(\answer{4}\right)^k a_k$:
\begin{multipleChoice}
\choice[correct]{converges}
\choice{diverges}
\choice{could converge or diverge; more information is needed.}
\end{multipleChoice}
\end{question}

\begin{question}
The series the series represented by $f(4)$, which is $\sum_{k=0}^{\infty} \left(\answer{5}\right)^k a_k$:
\begin{multipleChoice}
\choice{converges}
\choice{diverges}
\choice[correct]{could converge or diverge; more information is needed.}
\end{multipleChoice}
\end{question}

We have previously seen examples of series that converge at one endpoint of the interval of convergence, but not the other; more information about the coefficients $a_k$ would be needed to determine what happens at $x=-5$ and $x=4$.
\end{question}
\end{hint}


\end{exercise}
\end{document}
