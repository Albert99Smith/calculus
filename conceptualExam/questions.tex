\documentclass{ximera}

\begin{document}
\begin{abstract}
\end{abstract}

\section{Understanding functions}

\begin{problem}
  Suppose $f(x) = x^2$.  How does $f(f(x))$ compare to $f(x)$?
  \begin{multipleChoice}
    \choice{Whenever $x$ is close to one, $f(f(x))$ is close to zero.}
    \choice{Whenever $x$ is close to one, $f(f(x))$ is larger than $x$.}
    \choice{Whenever $x$ is close to zero, $f(f(x))$ is larger than $x$.}
    \choice[correct]{Whenever $x$ is large, $f(f(x))$ is larger than $x$.}
  \end{multipleChoice}
\end{problem}

\section{Review of famous functions}

\begin{problem}
   Which of the following statements is true?
   \begin{multipleChoice}
     \choice{$\sin^{-1}(x)$ is the inverse function of $\sin(x)$}
     \choice[correct]{$\sin\left(\sin^{-1}\left(\frac{1}{2}\right)\right)
       = \frac{1}{2}$} 
     \choice{$\sin^{-1}\left(\sin\left(\frac{5\pi}{2}\right)\right) = \frac{5\pi}{2}$}
     
     \choice{$\sin^{-1}(x) = \frac{1}{\sin(x)}$}
   \end{multipleChoice}  
\end{problem}

\begin{problem}
  The expression $\log_b(x) = y$ is equivalent to:
  \begin{multipleChoice}
    \choice{$b^x = y$}
    \choice[correct]{$b^y = x$}
    \choice{$x^b = y$}
    \choice{$x^y = b$}
    \choice{$y^b = x$}
    \choice{$y^x = b$}
  \end{multipleChoice}  
\end{problem}

\begin{problem}
  Suppose $f(x) = x \sin^2 x$.  What is true about $f(x)$?
  \begin{multipleChoice}
    \choice[correct]{When $x$ is close to zero, $f(x)$ is close to zero.}
    \choice{When $x$ is very large, $f(x)$ is very large.}
    \choice{When $x$ is very negative, $f(x)$ is very negative.}
    \choice{When $x$ is close to one, $f(x)$ is close to one.}
  \end{multipleChoice}
\end{problem}



\section{What is a limit?}

\begin{problem}
  Suppose $x$ is a positive number close to zero, and $y$ is a very large number.  What can be said about $y/x$?
  \begin{multipleChoice}
    \choice[correct]{It is very large.}
    \choice{It is close to zero.}
    \choice{It is very negative.}
    \choice{It could be very positive or very negative.}
  \end{multipleChoice}
\end{problem}

\begin{problem}
  Suppose $x$ is a number close to zero, and $y$ is a very large number.  What can be said about $y/x$?
  \begin{multipleChoice}
    \choice{It is very large.}
    \choice{It is close to zero.}
    \choice{It is very negative.}
    \choice[correct]{It could be very positive or very negative.}
  \end{multipleChoice}
\end{problem}

\begin{problem}
  Suppose $x$ and $y$ are numbers close to zero.  What can be said about $x+y$?
  \begin{multipleChoice}
    \choice[correct]{It is close to zero.}
    \choice{It is negative.}
    \choice{It is positive.}
    \choice{It is larger than $x$.}
  \end{multipleChoice}
\end{problem}

\section{Limit laws}



\section{Indeterminate forms}
\section{Using limits to detect asymptotes}

\begin{problem}
  Suppose whenever $x$ is very large, $f(x)$ is close to 1.  What is true about $f$?
  \begin{multipleChoice}
    \choice{The graph of $f$ does not cross the line $y = 1$.}
    \choice{The graph of $f$ does not cross the line $x = 1$.}
    \choice{Whenever $x$ is close to zero, $f(1/x)$ is close to 1.}
    \choice[correct]{Whenever $x$ is very large, $f(x^2)$ is close to 1.}
  \end{multipleChoice}
\end{problem}

\section{Continuity and the Intermediate Value Theorem}

\begin{problem}
  Suppose $f$ is a continuous function so that whenever $0 \leq x \leq 1$ we have $0 \leq f(x) \leq 1$.  What can be about $f$?
  \begin{multipleChoice}
    \choice[correct]{There is an $x$ so that $f(x) = x$.}
    \choice{There is an $x$ so that $f(x) > 1$.}
    \choice{There is an $x$ so that $f(x) > x$.}
    \choice{There is an $x$ so that $f(x) < 0$.}
    \choice{There is an $x$ so that $f(x) < x$.}
  \end{multipleChoice}
\end{problem}

\section{An Application of limits}
\section{Definition of the derivative}

\begin{problem}
  Let $x = \log_b 1001 - \log_b 1000$ and $y = \log_b 101 - \log_b 100$.  How do $x$ and $y$ compare?
  \begin{multipleChoice}
    \choice[correct]{$x < y$}
    \choice{$x > y$}
    \choice{$x = y$}
    \choice{It depends on the base $b$.}
  \end{multipleChoice}
\end{problem}

\section{The derivative as a function}




\section{Higher order derivatives and graphs}
\section{Rules of differentiation}
\section{The product and quotient rules}
\section{The chain rule}
\section{Mean Value Theorem}

\begin{problem}
  Suppose $f$ is an increasing function, and $x$ and $h$ are positive numbers between $0$ and $1$.  How does $f(x + h)$ compare to $f(f + h^2)$?
  \begin{multipleChoice}
    \choice[correct]{$f(x+h) > f(f+h^2)$}
    \choice{$f(x+h) < f(f+h^2)$}
    \choice{It depends on the function $f$.}
  \end{multipleChoice}
\end{problem}

\begin{problem}
  Suppose $f$ and $g$ are increasing functions.  What can be said about the function $h(x) = f(g(x))$?
  \begin{multipleChoice}
    \choice[correct]{It is an increasing function.}
    \choice{It is a decreasing function.}
    \choice{It depends on exactly which functions $f$ and $g$ are being considered.}
  \end{multipleChoice}
\end{problem}


\begin{problem}
  Suppose $f$ and $g$ are decreasing functions.  What can be said about the function $h(x) = f(g(x))$?
  \begin{multipleChoice}
    \choice[correct]{It is an increasing function.}
    \choice{It is a decreasing function.}
    \choice{It depends on exactly which functions $f$ and $g$ are being considered.}
  \end{multipleChoice}
\end{problem}



\section{Linear approximation}

\begin{problem}
  Suppose $y = mx + b$ is the equation for a tangent line at the point $(100,\log 100)$ to the graph of $f(x) = \log x$.  How does $200m + b$ compare to $\log 200$?
  \begin{multipleChoice}
    \choice[correct]{$200m + b > \log 200$}
    \choice{$200m + b > \log 200$}
    \choice{$200m + b = \log 200$}
  \end{multipleChoice}
\end{problem}

\section{Maximums and minimums}

\begin{problem}
  Suppose a smooth function $f$ has infinitely many local minima.  What is true of $f$?
  \begin{multipleChoice}
    \choice[correct]{$f$ has infinitely many local maxima.}
    \choice{$f(x)$ can be made as large as desired by choosing appropriate $x$}
    \choice{$f(x)$ can be made as small as desired by choosing appropriate $x$}
  \end{multipleChoice}
\end{problem}

\section{Optimization}
\section{Applied optimization}
\section{Derivatives of trigonometric functions}
\section{Implicit differentiation}
\section{Derivatives of inverse functions}
\section{Logarithmic differentiation}

\begin{problem}
  Consider the continuous function $f$ and its logarithmic derivative $g(x) = \frac{f'(x)}{f(x)}$.  Suppose $g(x) = 0.05$ when $0 < x < 10$ and $g(x) = -0.05$ when $10 < x < 20$.  How do $f(0)$ and $f(20)$ compare?
  \begin{multipleChoice}
    \choice[correct]{$f(20) > f(0)$}
    \choice{$f(0) < f(20)$}
    \choice{$f(0) = f(20)$}
  \end{multipleChoice}
\end{problem}

\section{Advanced graphing of functions}
\section{More than one rate}
\section{Applied related rates}
\section{Differential equations}
\section{Antiderivatives}
\section{L'Hospital's rule}
\section{Approximating the area under a curve}
\section{Net area}
\section{The definite integral}
\section{First Fundamental Theorem of Calculus}
\section{Second Fundamental Theorem of Calculus}
\section{The idea of substitution}
\section{Working with substitution}
\section{Applications of integrals}






\end{document}

%%% Local Variables:
%%% mode: latex
%%% TeX-master: t
%%% End:
