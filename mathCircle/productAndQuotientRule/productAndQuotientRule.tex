\documentclass{ximera}

\usepackage{todonotes}

\newcommand{\RR}{\mathbb R}
\renewcommand{\d}{\,d}
\newcommand{\dd}[2][]{\frac{d #1}{d #2}}
\renewcommand{\l}{\ell}
\newcommand{\ddx}{\frac{d}{dx}}
\newcommand{\dfn}{\textbf}
\newcommand{\eval}[1]{\bigg[ #1 \bigg]}
\renewcommand{\epsilon}{\varepsilon}

\let\prelim\lim
\renewcommand{\lim}{\displaystyle\prelim}

\colorlet{textColor}{black} 
\colorlet{background}{white}
\colorlet{penColor}{blue!50!black} % Color of a curve in a plot
\colorlet{penColor2}{red!50!black}% Color of a curve in a plot
\colorlet{penColor3}{red!50!blue} % Color of a curve in a plot
\colorlet{penColor4}{green!50!black} % Color of a curve in a plot
\colorlet{penColor5}{orange!80!black} % Color of a curve in a plot
\colorlet{fill1}{blue!50!black!20} % Color of fill in a plot
\colorlet{fill2}{blue!10} % Color of fill in a plot
\colorlet{fillp}{fill1} % Color of positive area
\colorlet{filln}{red!50!black!20} % Color of negative area
\colorlet{gridColor}{gray!50} % Color of grid in a plot


\newcommand{\fullwidth}{}
\newcommand{\normalwidth}{}



%% makes a snazzy t-chart for evaluating functions
\newenvironment{tchart}{\rowcolors{2}{}{background!90!textColor}\array}{\endarray}


\begin{document}

\section{Setting}

\vfil

\[
\ddx\left(f(x) \cdot g(x)\right) = f'(x)\cdot g'(x).
\]

\vfil

\vfil

\newpage


\section{Questions}

\paragraph{Level 1}
\begin{enumerate}
\item Is this true for all differentiable functions $f$ and $g$?
\item Is this the product rule?
\item What is the product rule?
\item Find functions so that this equation does not hold for all $x$.
\item When $f$ is a constant, and
  \[
  \ddx\left(f(x)\cdot g(x)\right) = f'(x)\cdot g'(x)
  \]
  what must we conclude about $g$?
\item What if $f = g$?
\end{enumerate}

\paragraph{Level 2}
\begin{enumerate}
\item Find functions $f$ and $g$ so that this equation holds for all x.
\item Is there more than one pair of functions that will make this true?
\item Is there a choice of $f$, (with minimal restrictions on $g$)
  that makes this false?
\item Is there a choice of $f$, (regardless of g) that makes this true?
\item How about non-constant functions that make this true?
\item What happens when f and/or g are ``fundamental'' functions, meaning constants, polynomials, exponential functions, and trigonometric functions?
\item What happens when you add constants to the functions?
\item What does this mean geometrically about the functions $f$ and $g$?
\item What does this mean numerically about the functions $f$ and $g$?
\item Does there exists $f$ and $g$ such that
  \[
  \ddx \left(f(x)/g(x)\right) = f'(x)\cdot g'(x)?
  \]
\item Does there exists $f$ and $g$ such that
  \[
  \ddx \left(f(x)/g(x)\right) = f'(x)/g'(x)?
  \]
  \item Does there exists $f$ and $g$ such that
  \[
  \ddx \left(f(x)\cdot g(x)\right) = f'(x)/g'(x)?
  \]
  \item Does there exists $f$ and $g$ such that
  \[
  \ddx f(g(x)) = f'(g'(x))?
  \]
\end{enumerate}

\paragraph{Level 3}
\begin{enumerate}
\item Classify all functions that satisfy this ``false'' product rule.
\item What about higher derivatives?
\item Why is it important? In this applied?
\item What about the product of $3$ functions? $4$ functions? $5$ functions? 
\item What about the product of $n$ functions?
\item What about infinite product of functions?
\item Is there a linear functional where this true?

\item When $g(x)$ is a rational function and the formula is true, what
  must be true about $f$?
\item Can you come up with a real-world context for this problem?
\item Given a pair of solutions, $f$ and $g$, are they linearly dependent?
\item Can you put conditions on $f$ and $g$ so that this holds at $1$ point?
\item What is it about the functions that make this true or false, is
  there a larger conceptual way to think about this?
\item What is the false integration by parts?
\item Can you make ``false'' calculus?
\end{enumerate}


\section{Content}

Note we have the following counter example
\begin{align*}
  f(x) &= \sin(x)\\
  g(x) &= \cos(x)
\end{align*}

The equality is true if $f(x) = 0$ and $g(x)$, $g'(x)$ exist.

We have the example
\[
f(x) = e^{2x} \qquad g(x) = e^{2x}
\]

Note if $f$ and $g$ are solutions, then
\[
f'(a) = 0 \Leftrightarrow g'(a) = 0
\]
in otherwords, $f$ and $g$ have the same critical points.


If $f$ and $g$ are polynomials, then they are both constant. 


$f(x) = e^{a(x)}$ with $g(x) = e^{b(x)}$ provided that
\[
a'(x)+b'(x) = a'(x)\cdot b'(x).
\]




\end{document}
