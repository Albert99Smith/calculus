\documentclass{ximera}

%\usepackage{todonotes}
%\usepackage{mathtools} %% Required for wide table Curl and Greens
%\usepackage{cuted} %% Required for wide table Curl and Greens
\newcommand{\todo}{}

\usepackage{esint} % for \oiint
\ifxake%%https://math.meta.stackexchange.com/questions/9973/how-do-you-render-a-closed-surface-double-integral
\renewcommand{\oiint}{{\large\bigcirc}\kern-1.56em\iint}
\fi


\graphicspath{
  {./}
  {ximeraTutorial/}
  {basicPhilosophy/}
  {functionsOfSeveralVariables/}
  {normalVectors/}
  {lagrangeMultipliers/}
  {vectorFields/}
  {greensTheorem/}
  {shapeOfThingsToCome/}
  {dotProducts/}
  {partialDerivativesAndTheGradientVector/}
  {../productAndQuotientRules/exercises/}
  {../normalVectors/exercisesParametricPlots/}
  {../continuityOfFunctionsOfSeveralVariables/exercises/}
  {../partialDerivativesAndTheGradientVector/exercises/}
  {../directionalDerivativeAndChainRule/exercises/}
  {../commonCoordinates/exercisesCylindricalCoordinates/}
  {../commonCoordinates/exercisesSphericalCoordinates/}
  {../greensTheorem/exercisesCurlAndLineIntegrals/}
  {../greensTheorem/exercisesDivergenceAndLineIntegrals/}
  {../shapeOfThingsToCome/exercisesDivergenceTheorem/}
  {../greensTheorem/}
  {../shapeOfThingsToCome/}
  {../separableDifferentialEquations/exercises/}
  {vectorFields/}
}

\newcommand{\mooculus}{\textsf{\textbf{MOOC}\textnormal{\textsf{ULUS}}}}

\usepackage{tkz-euclide}\usepackage{tikz}
\usepackage{tikz-cd}
\usetikzlibrary{arrows}
\tikzset{>=stealth,commutative diagrams/.cd,
  arrow style=tikz,diagrams={>=stealth}} %% cool arrow head
\tikzset{shorten <>/.style={ shorten >=#1, shorten <=#1 } } %% allows shorter vectors

\usetikzlibrary{backgrounds} %% for boxes around graphs
\usetikzlibrary{shapes,positioning}  %% Clouds and stars
\usetikzlibrary{matrix} %% for matrix
\usepgfplotslibrary{polar} %% for polar plots
\usepgfplotslibrary{fillbetween} %% to shade area between curves in TikZ
\usetkzobj{all}
\usepackage[makeroom]{cancel} %% for strike outs
%\usepackage{mathtools} %% for pretty underbrace % Breaks Ximera
%\usepackage{multicol}
\usepackage{pgffor} %% required for integral for loops



%% http://tex.stackexchange.com/questions/66490/drawing-a-tikz-arc-specifying-the-center
%% Draws beach ball
\tikzset{pics/carc/.style args={#1:#2:#3}{code={\draw[pic actions] (#1:#3) arc(#1:#2:#3);}}}



\usepackage{array}
\setlength{\extrarowheight}{+.1cm}
\newdimen\digitwidth
\settowidth\digitwidth{9}
\def\divrule#1#2{
\noalign{\moveright#1\digitwidth
\vbox{\hrule width#2\digitwidth}}}





\newcommand{\RR}{\mathbb R}
\newcommand{\R}{\mathbb R}
\newcommand{\N}{\mathbb N}
\newcommand{\Z}{\mathbb Z}

\newcommand{\sagemath}{\textsf{SageMath}}


%\renewcommand{\d}{\,d\!}
\renewcommand{\d}{\mathop{}\!d}
\newcommand{\dd}[2][]{\frac{\d #1}{\d #2}}
\newcommand{\pp}[2][]{\frac{\partial #1}{\partial #2}}
\renewcommand{\l}{\ell}
\newcommand{\ddx}{\frac{d}{\d x}}

\newcommand{\zeroOverZero}{\ensuremath{\boldsymbol{\tfrac{0}{0}}}}
\newcommand{\inftyOverInfty}{\ensuremath{\boldsymbol{\tfrac{\infty}{\infty}}}}
\newcommand{\zeroOverInfty}{\ensuremath{\boldsymbol{\tfrac{0}{\infty}}}}
\newcommand{\zeroTimesInfty}{\ensuremath{\small\boldsymbol{0\cdot \infty}}}
\newcommand{\inftyMinusInfty}{\ensuremath{\small\boldsymbol{\infty - \infty}}}
\newcommand{\oneToInfty}{\ensuremath{\boldsymbol{1^\infty}}}
\newcommand{\zeroToZero}{\ensuremath{\boldsymbol{0^0}}}
\newcommand{\inftyToZero}{\ensuremath{\boldsymbol{\infty^0}}}



\newcommand{\numOverZero}{\ensuremath{\boldsymbol{\tfrac{\#}{0}}}}
\newcommand{\dfn}{\textbf}
%\newcommand{\unit}{\,\mathrm}
\newcommand{\unit}{\mathop{}\!\mathrm}
\newcommand{\eval}[1]{\bigg[ #1 \bigg]}
\newcommand{\seq}[1]{\left( #1 \right)}
\renewcommand{\epsilon}{\varepsilon}
\renewcommand{\phi}{\varphi}


\renewcommand{\iff}{\Leftrightarrow}

\DeclareMathOperator{\arccot}{arccot}
\DeclareMathOperator{\arcsec}{arcsec}
\DeclareMathOperator{\arccsc}{arccsc}
\DeclareMathOperator{\si}{Si}
\DeclareMathOperator{\scal}{scal}
\DeclareMathOperator{\sign}{sign}


%% \newcommand{\tightoverset}[2]{% for arrow vec
%%   \mathop{#2}\limits^{\vbox to -.5ex{\kern-0.75ex\hbox{$#1$}\vss}}}
\newcommand{\arrowvec}[1]{{\overset{\rightharpoonup}{#1}}}
%\renewcommand{\vec}[1]{\arrowvec{\mathbf{#1}}}
\renewcommand{\vec}[1]{{\overset{\boldsymbol{\rightharpoonup}}{\mathbf{#1}}}\hspace{0in}}

\newcommand{\point}[1]{\left(#1\right)} %this allows \vector{ to be changed to \vector{ with a quick find and replace
\newcommand{\pt}[1]{\mathbf{#1}} %this allows \vec{ to be changed to \vec{ with a quick find and replace
\newcommand{\Lim}[2]{\lim_{\point{#1} \to \point{#2}}} %Bart, I changed this to point since I want to use it.  It runs through both of the exercise and exerciseE files in limits section, which is why it was in each document to start with.

\DeclareMathOperator{\proj}{\mathbf{proj}}
\newcommand{\veci}{{\boldsymbol{\hat{\imath}}}}
\newcommand{\vecj}{{\boldsymbol{\hat{\jmath}}}}
\newcommand{\veck}{{\boldsymbol{\hat{k}}}}
\newcommand{\vecl}{\vec{\boldsymbol{\l}}}
\newcommand{\uvec}[1]{\mathbf{\hat{#1}}}
\newcommand{\utan}{\mathbf{\hat{t}}}
\newcommand{\unormal}{\mathbf{\hat{n}}}
\newcommand{\ubinormal}{\mathbf{\hat{b}}}

\newcommand{\dotp}{\bullet}
\newcommand{\cross}{\boldsymbol\times}
\newcommand{\grad}{\boldsymbol\nabla}
\newcommand{\divergence}{\grad\dotp}
\newcommand{\curl}{\grad\cross}
%\DeclareMathOperator{\divergence}{divergence}
%\DeclareMathOperator{\curl}[1]{\grad\cross #1}
\newcommand{\lto}{\mathop{\longrightarrow\,}\limits}

\renewcommand{\bar}{\overline}

\colorlet{textColor}{black}
\colorlet{background}{white}
\colorlet{penColor}{blue!50!black} % Color of a curve in a plot
\colorlet{penColor2}{red!50!black}% Color of a curve in a plot
\colorlet{penColor3}{red!50!blue} % Color of a curve in a plot
\colorlet{penColor4}{green!50!black} % Color of a curve in a plot
\colorlet{penColor5}{orange!80!black} % Color of a curve in a plot
\colorlet{penColor6}{yellow!70!black} % Color of a curve in a plot
\colorlet{fill1}{penColor!20} % Color of fill in a plot
\colorlet{fill2}{penColor2!20} % Color of fill in a plot
\colorlet{fillp}{fill1} % Color of positive area
\colorlet{filln}{penColor2!20} % Color of negative area
\colorlet{fill3}{penColor3!20} % Fill
\colorlet{fill4}{penColor4!20} % Fill
\colorlet{fill5}{penColor5!20} % Fill
\colorlet{gridColor}{gray!50} % Color of grid in a plot

\newcommand{\surfaceColor}{violet}
\newcommand{\surfaceColorTwo}{redyellow}
\newcommand{\sliceColor}{greenyellow}




\pgfmathdeclarefunction{gauss}{2}{% gives gaussian
  \pgfmathparse{1/(#2*sqrt(2*pi))*exp(-((x-#1)^2)/(2*#2^2))}%
}


%%%%%%%%%%%%%
%% Vectors
%%%%%%%%%%%%%

%% Simple horiz vectors
\renewcommand{\vector}[1]{\left\langle #1\right\rangle}


%% %% Complex Horiz Vectors with angle brackets
%% \makeatletter
%% \renewcommand{\vector}[2][ , ]{\left\langle%
%%   \def\nextitem{\def\nextitem{#1}}%
%%   \@for \el:=#2\do{\nextitem\el}\right\rangle%
%% }
%% \makeatother

%% %% Vertical Vectors
%% \def\vector#1{\begin{bmatrix}\vecListA#1,,\end{bmatrix}}
%% \def\vecListA#1,{\if,#1,\else #1\cr \expandafter \vecListA \fi}

%%%%%%%%%%%%%
%% End of vectors
%%%%%%%%%%%%%

%\newcommand{\fullwidth}{}
%\newcommand{\normalwidth}{}



%% makes a snazzy t-chart for evaluating functions
%\newenvironment{tchart}{\rowcolors{2}{}{background!90!textColor}\array}{\endarray}

%%This is to help with formatting on future title pages.
\newenvironment{sectionOutcomes}{}{}



%% Flowchart stuff
%\tikzstyle{startstop} = [rectangle, rounded corners, minimum width=3cm, minimum height=1cm,text centered, draw=black]
%\tikzstyle{question} = [rectangle, minimum width=3cm, minimum height=1cm, text centered, draw=black]
%\tikzstyle{decision} = [trapezium, trapezium left angle=70, trapezium right angle=110, minimum width=3cm, minimum height=1cm, text centered, draw=black]
%\tikzstyle{question} = [rectangle, rounded corners, minimum width=3cm, minimum height=1cm,text centered, draw=black]
%\tikzstyle{process} = [rectangle, minimum width=3cm, minimum height=1cm, text centered, draw=black]
%\tikzstyle{decision} = [trapezium, trapezium left angle=70, trapezium right angle=110, minimum width=3cm, minimum height=1cm, text centered, draw=black]


\title[Dig-In:]{The Mean Value Theorem}

\outcome{Understand the statement of the Mean Value Theorem.}
\outcome{Sketch pictures to illustrate why the Mean Value Theorem is true.}
\outcome{Determine whether Rolle's Theorem or the Mean Value Theorem can be applied.}
\outcome{Find the values guaranteed by Rolle's Theorem or the Mean Value Theorem.}
\outcome{Use the Mean Value Theorem to solve word problems.}
\outcome{Compare and contrast the Intermediate Value Theorem, Mean Value Theorem, and Rolle's Theorem.}
\outcome{Identify calculus ideas which are consequences of the Mean Value Theorem.}

\begin{document}
\begin{abstract}
  Here we see a key theorem of calculus.
\end{abstract}
\maketitle


Here are some interesting questions involving derivatives:

\begin{enumerate}
\item Suppose you toss a ball into the air and then catch it. Must the
  ball's vertical velocity have been zero at some point?
\item Suppose you drive a car from toll booth on a toll road to
  another toll booth $30$ miles away in half of an hour. Must you have
  been driving at $60$ miles per hour at some point?
\item Suppose two different functions have the same derivative. What
  can you say about the relationship between the two functions?
\end{enumerate}

While these problems sound very different, it turns out that the
problems are very closely related. We'll start simply:

\begin{theorem}[Rolle's Theorem]\index{Rolle's Theorem} 
Suppose that $f$ is differentiable on the interval $(a,b)$, is
continuous on the interval $[a,b]$, and $f(a)=f(b)$.
\begin{image}
\begin{tikzpicture}
	\begin{axis}[
            xmin=0, xmax=4.5,ymin=1,ymax=5,
            axis lines =left, xlabel=$x$, ylabel=$y$,
            every axis y label/.style={at=(current axis.above origin),anchor=south},
            every axis x label/.style={at=(current axis.right of origin),anchor=west},
            xtick={1,2,3}, xticklabels={$a$,$c$,$b$},
            ytickmin=1, ytickmax=0,
            axis on top,
          ]       
          \addplot [draw=none, fill=fill1,domain=(1:3)] {5} \closedcycle;       
	  \addplot [very thick,penColor, smooth] {-(x-2)^2+4};
          \addplot [very thick,penColor2, smooth] {4};
          \node at (axis cs:.4,2.5) [penColor] {$f$}; 
          \addplot [textColor,dashed] plot coordinates {(2,0) (2,4)};
          \addplot [textColor,dashed] plot coordinates {(1,3) (3,3)};
          \addplot[color=penColor3,fill=penColor3,only marks,mark=*] coordinates{(2,4)};  %% closed hole          
          \addplot[color=penColor,fill=penColor,only marks,mark=*] coordinates{(1,3)};  %% closed hole          
          \addplot[color=penColor,fill=penColor,only marks,mark=*] coordinates{(3,3)};  %% closed hole          
        \end{axis}
\end{tikzpicture}
%% \caption{A geometric interpretation of Rolle's Theorem.}
%% \label{figure:geoRolle}
%% \end{marginfigure}
\end{image}
Then
\[
f'(c)=0
\]
for some $a<c<b$.
%% \begin{explanation}
%% By the Extreme Value Theorem, Theorem, we know that $f$ has a
%% maximum and minimum value on $[a,b]$.

%% If maximum and minimum both occur at the endpoints, then
%% $f(x)=f(a)=f(b)$ at every point in $[a,b]$. Hence the function is a
%% horizontal line, and it has derivative zero everywhere on
%% $(a,b)$. We may choose any $c$ at all to get $f'(c)=0$.

%% If the maximum or minimum occurs at a point $c$ with $a<c<b$, then by
%% Fermat's Theorem, $f'(c)=0$.
%% \end{explanation}
\end{theorem}


We can now answer our first question above.

\begin{example}
Suppose you toss a ball into the air and then catch it. Must the
ball's vertical velocity have been zero at some point?
\begin{explanation}
  Let $p(t)$ be the position of the ball at time $t$. Our interval in question will be
  \[
  [t_\mathrm{start},t_\mathrm{finish}]
  \]
  we may assume that $p$ is continuous on $[a,b]$ and differentiable
  on $(a,b)$. We may now apply Rolle's Theorem to see at some time
  $c$, $p'(c)=\answer[given]{0}$. Hence the velocity must be zero at
  some point.
\end{explanation}
\end{example}

Rolle's Theorem is a special case of a more general theorem.

\begin{theorem}[Mean Value Theorem]\label{thm:mvt}\index{Mean Value Theorem}
Suppose that $f$ has a derivative on the interval $(a,b)$ and is
continuous on the interval $[a,b]$.
%\begin{marginfigure}[.5in]
\begin{image}
\begin{tikzpicture}
	\begin{axis}[
            xmin=.5, xmax=5.5,ymin=0,ymax=3.1,
            axis lines =center, xlabel=$x$, ylabel=$y$,
            every axis y label/.style={at=(current axis.above origin),anchor=south},
            every axis x label/.style={at=(current axis.right of origin),anchor=west},
            xtick={1,2.04,5}, xticklabels={$a$,$c$,$b$},
            ytickmin=1, ytickmax=0,
            axis on top,
          ] 
          \addplot [draw=none, fill=fill1,domain=(1:5)] {3.1} \closedcycle;       
          \addplot [penColor2!40!background,very thick,dashed] plot coordinates {(1,.84+1.5) (5,1.5-.96)};        
          \addplot [textColor,dashed] plot coordinates {(2.04,0) (2.04,1.5+.89)};        
	  \addplot [very thick,penColor, smooth,domain=(1:5)] {sin(deg(x))+1.5};
          \addplot [very thick,penColor2,domain=(.5:5.5)] {-.45*(x-2.04)+.89+1.5};
          %\node at (axis cs:.4,2.5) [penColor] {$f(x)$}; 
          \addplot[color=penColor,fill=penColor,only marks,mark=*] coordinates{(1,.84+1.5)};  %% closed hole          
          \addplot[color=penColor,fill=penColor,only marks,mark=*] coordinates{(5,-.96+1.5)};  %% closed hole          
          \addplot[color=penColor3,fill=penColor3,only marks,mark=*] coordinates{(2.04,.89+1.5)};  %% closed hole          
        \end{axis}
\end{tikzpicture}
%% \caption{A geometric interpretation of the Mean Value Theorem}
%% \label{figure:geoMVT}
%% \end{marginfigure}
\end{image}
Then
\[
f'(c)=\frac{f(b)-f(a)}{b-a}
\]
for some $a<c<b$. 
%% \begin{explanation}
%% Let 
%% \[
%% m=\frac{f(b)-f(a)}{b-a},
%% \] 
%% and consider a new function $g(x)=f(x) - m(x-a)-f(a)$.  We know that
%% $g(x)$ has a derivative on $[a,b]$, since $g'(x)=f'(x)-m$. We can
%% compute $g(a)=f(a)- m(a-a)-f(a) =0$ and
%% \begin{align*}
%% g(b)=f(b)-m(b-a)-f(a)&=f(b)-{f(b)-f(a)\over b-a}(b-a)-f(a) \\
%% &=f(b)-(f(b)-f(a))-f(a)\\
%% &=0. 
%% \end{align*}
%% So $g(a) = g(b) = 0$. Now by Rolle's Theorem, that at some $c$,
%% \[
%% g'(c)=0\qquad\text{for some $a<c<b$}.
%% \]
%% But we know that $g'(c)=f'(c)-m$, so
%% \[
%% 0=f'(c)-m=f'(c)-\frac{f(b)-f(a)}{b-a}.
%% \]
%% Hence
%% \[
%% f'(c)=\frac{f(b)-f(a)}{b-a}.
%% \]
%% \end{explanation}
\end{theorem}

We can now answer our second question above.  

\begin{example}
Suppose you drive a car from toll booth on a toll road to another toll
booth $30$ miles away in half of an hour. Must you have been driving
at $60$ miles per hour at some point?

\begin{explanation}
If $p(t)$ is the position of the car at time $t$, and $0$ hours is the
starting time with $1/2$ hours being the final time, then we may
assume that $p$ is continuous on $[0,1/2]$ and differentiable on
$(0,1/2)$. Now the Mean Value Theorem states there is a time $c$
\[
p'(c) = \frac{30-0}{1/2} = \answer[given]{60}\qquad \text{where $0<c<1/2$.}
\]
Since the derivative of position is velocity, this says that the car
must have been driving at $60$ miles per hour at some point.
\end{explanation}
\end{example}

Now we will address the unthinkable, could there be a continuous
function $f$ on $[a,b]$ whose derivative is zero on $(a,b)$ that is
not constant? As we will see, the answer is ``no.''

\begin{theorem} 
If $f'(x)=0$ for all $x$ in an interval $I$, then $f(x)$ is constant
on $I$.
\begin{explanation}
Let $a< b$ be two points in $I$. Since $f$ is continuous on $[a,b]$
and differentiable on $(a,b)$, by the Mean Value Theorem we know
\[
\frac{f(b)-f(a)}{b-a} = f'(c)
\]
for some $c$ in the interval $(a,b)$. Since $f'(c)=0$ we see that
$f(b)=f(a)$. %%BADBAD I would like to ask a question here.
Moreover, since $a$ and $b$ were arbitrarily chosen,
$f(x)$ must be the constant function.
\end{explanation}
\end{theorem}


Now let's answer our third question.

\begin{example}
Suppose two different functions have the same derivative. What can you
say about the relationship between the two functions?

\begin{explanation}
Set $h(x) = f(x) - g(x)$, so $h'(x) = f'(x) -g'(x)$. Now $h'(x) = 0$
on the interval $(a,b)$. This means that $h(x) = k$ where $k$ is some
constant. Hence
\[
g(x) = f(x) + k.
\]
\end{explanation}
\end{example}


\begin{example}
Describe all functions whose derivative is $\sin(x)$.
\begin{explanation}
One such function is $-\cos(x)$, so all such functions have the form
$-\cos(x)+k$,
%% \begin{marginfigure}[0in]
\begin{image}
\begin{tikzpicture}
	\begin{axis}[
            xmin=0, xmax=6.2,ymin=-4,ymax=4,domain=(0:6.2),
            axis lines =center, xlabel=$x$, ylabel=$y$,
            every axis y label/.style={at=(current axis.above origin),anchor=south},
            every axis x label/.style={at=(current axis.right of origin),anchor=west},
            axis on top,
          ] 
          \addplot [very thick,penColor, smooth] {-cos(deg(x))};
          \addplot [very thick,penColor2!30!background, smooth] {-cos(deg(x))+1};
          \addplot [very thick,penColor3!30!background, smooth] {-cos(deg(x))-1};
          \addplot [very thick,penColor4!30!background, smooth] {-cos(deg(x))+2};
          \addplot [very thick,penColor5!30!background, smooth] {-cos(deg(x))-2};         
        \end{axis}
\end{tikzpicture}
%% \caption{Functions of the form $-\cos(x)+k$, each of whose derivative is $\sin(x)$.}
%% \label{figure:cos+k}
%% \end{marginfigure}
\end{image}
\end{explanation}
\end{example}

Finally, let us investigate two young mathematicians who run to class.

\begin{example}
  Two students Devyn and Riley raced to class. Was there a point
  during the race that Devyn and Riley were running at exactly the
  same velocity?
  \begin{explanation}
    Let $P_\mathrm{Devyn}$ represent Devyn's position with respect to
    time, and let $P_\mathrm{Riley}$ represent Riley's position with
    respect to time. Let $t_\mathrm{start}$ be the starting time of
    the race, and $t_\mathrm{finish}$ be the end of the race. Set
    \[
    f(t) =P_\mathrm{Devyn}(t)-P_\mathrm{Riley}(t).
    \]
    Note, we may assume that $P_\mathrm{Devyn}$ and
    $P_\mathrm{Riley}(t)$ are continuous on
    $[t_\mathrm{start},t_\mathrm{finish}]$ and that they are
    differentiable on $(t_\mathrm{start},t_\mathrm{finish})$. Hence
    the same is true for $f$. Since both runners start and finish at
    the same place,
    \begin{align*}
    f(t_\mathrm{start}) = P_\mathrm{Devyn}(t_\mathrm{start})-P_\mathrm{Riley}(t_\mathrm{start}) &= \answer[given]{0}\qquad\text{and}\\
    f(t_\mathrm{finish})=P_\mathrm{Devyn}(t_\mathrm{finish})-P_\mathrm{Riley}(t_\mathrm{finish}) &= \answer[given]{0}.\\
    \end{align*}
    In fact, this shows us that the average rate of change of
    \[
    f(t) = P_\mathrm{Devyn}(t)-P_\mathrm{Riley}(t)
    \qquad\text{on}\qquad [t_\mathrm{start},t_\mathrm{finish}]
    \]
    is $\answer[given]{0}$. Hence by the mean value theorem, there is a point $c$
    \[
    t_\mathrm{start}\le c \le t_{\mathrm{finish}}
    \]
    with $f'(c) = \answer[given]{0}$. However,
    \[
    \answer[given]{0} = f'(x) = P_\mathrm{Devyn}'(c)-P_\mathrm{Riley}'(c).
    \]
    Hence at $c$,
    \[
    P_\mathrm{Devyn}'(c)=P_\mathrm{Riley}'(c),
    \]
    this means that there was a time when they were running at exactly the same velocity. 
  \end{explanation}
\end{example}



\end{document}
