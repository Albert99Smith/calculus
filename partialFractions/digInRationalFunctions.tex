\documentclass{ximera}

%\usepackage{todonotes}
%\usepackage{mathtools} %% Required for wide table Curl and Greens
%\usepackage{cuted} %% Required for wide table Curl and Greens
\newcommand{\todo}{}

\usepackage{esint} % for \oiint
\ifxake%%https://math.meta.stackexchange.com/questions/9973/how-do-you-render-a-closed-surface-double-integral
\renewcommand{\oiint}{{\large\bigcirc}\kern-1.56em\iint}
\fi


\graphicspath{
  {./}
  {ximeraTutorial/}
  {basicPhilosophy/}
  {functionsOfSeveralVariables/}
  {normalVectors/}
  {lagrangeMultipliers/}
  {vectorFields/}
  {greensTheorem/}
  {shapeOfThingsToCome/}
  {dotProducts/}
  {partialDerivativesAndTheGradientVector/}
  {../productAndQuotientRules/exercises/}
  {../normalVectors/exercisesParametricPlots/}
  {../continuityOfFunctionsOfSeveralVariables/exercises/}
  {../partialDerivativesAndTheGradientVector/exercises/}
  {../directionalDerivativeAndChainRule/exercises/}
  {../commonCoordinates/exercisesCylindricalCoordinates/}
  {../commonCoordinates/exercisesSphericalCoordinates/}
  {../greensTheorem/exercisesCurlAndLineIntegrals/}
  {../greensTheorem/exercisesDivergenceAndLineIntegrals/}
  {../shapeOfThingsToCome/exercisesDivergenceTheorem/}
  {../greensTheorem/}
  {../shapeOfThingsToCome/}
  {../separableDifferentialEquations/exercises/}
  {vectorFields/}
}

\newcommand{\mooculus}{\textsf{\textbf{MOOC}\textnormal{\textsf{ULUS}}}}

\usepackage{tkz-euclide}\usepackage{tikz}
\usepackage{tikz-cd}
\usetikzlibrary{arrows}
\tikzset{>=stealth,commutative diagrams/.cd,
  arrow style=tikz,diagrams={>=stealth}} %% cool arrow head
\tikzset{shorten <>/.style={ shorten >=#1, shorten <=#1 } } %% allows shorter vectors

\usetikzlibrary{backgrounds} %% for boxes around graphs
\usetikzlibrary{shapes,positioning}  %% Clouds and stars
\usetikzlibrary{matrix} %% for matrix
\usepgfplotslibrary{polar} %% for polar plots
\usepgfplotslibrary{fillbetween} %% to shade area between curves in TikZ
\usetkzobj{all}
\usepackage[makeroom]{cancel} %% for strike outs
%\usepackage{mathtools} %% for pretty underbrace % Breaks Ximera
%\usepackage{multicol}
\usepackage{pgffor} %% required for integral for loops



%% http://tex.stackexchange.com/questions/66490/drawing-a-tikz-arc-specifying-the-center
%% Draws beach ball
\tikzset{pics/carc/.style args={#1:#2:#3}{code={\draw[pic actions] (#1:#3) arc(#1:#2:#3);}}}



\usepackage{array}
\setlength{\extrarowheight}{+.1cm}
\newdimen\digitwidth
\settowidth\digitwidth{9}
\def\divrule#1#2{
\noalign{\moveright#1\digitwidth
\vbox{\hrule width#2\digitwidth}}}





\newcommand{\RR}{\mathbb R}
\newcommand{\R}{\mathbb R}
\newcommand{\N}{\mathbb N}
\newcommand{\Z}{\mathbb Z}

\newcommand{\sagemath}{\textsf{SageMath}}


%\renewcommand{\d}{\,d\!}
\renewcommand{\d}{\mathop{}\!d}
\newcommand{\dd}[2][]{\frac{\d #1}{\d #2}}
\newcommand{\pp}[2][]{\frac{\partial #1}{\partial #2}}
\renewcommand{\l}{\ell}
\newcommand{\ddx}{\frac{d}{\d x}}

\newcommand{\zeroOverZero}{\ensuremath{\boldsymbol{\tfrac{0}{0}}}}
\newcommand{\inftyOverInfty}{\ensuremath{\boldsymbol{\tfrac{\infty}{\infty}}}}
\newcommand{\zeroOverInfty}{\ensuremath{\boldsymbol{\tfrac{0}{\infty}}}}
\newcommand{\zeroTimesInfty}{\ensuremath{\small\boldsymbol{0\cdot \infty}}}
\newcommand{\inftyMinusInfty}{\ensuremath{\small\boldsymbol{\infty - \infty}}}
\newcommand{\oneToInfty}{\ensuremath{\boldsymbol{1^\infty}}}
\newcommand{\zeroToZero}{\ensuremath{\boldsymbol{0^0}}}
\newcommand{\inftyToZero}{\ensuremath{\boldsymbol{\infty^0}}}



\newcommand{\numOverZero}{\ensuremath{\boldsymbol{\tfrac{\#}{0}}}}
\newcommand{\dfn}{\textbf}
%\newcommand{\unit}{\,\mathrm}
\newcommand{\unit}{\mathop{}\!\mathrm}
\newcommand{\eval}[1]{\bigg[ #1 \bigg]}
\newcommand{\seq}[1]{\left( #1 \right)}
\renewcommand{\epsilon}{\varepsilon}
\renewcommand{\phi}{\varphi}


\renewcommand{\iff}{\Leftrightarrow}

\DeclareMathOperator{\arccot}{arccot}
\DeclareMathOperator{\arcsec}{arcsec}
\DeclareMathOperator{\arccsc}{arccsc}
\DeclareMathOperator{\si}{Si}
\DeclareMathOperator{\scal}{scal}
\DeclareMathOperator{\sign}{sign}


%% \newcommand{\tightoverset}[2]{% for arrow vec
%%   \mathop{#2}\limits^{\vbox to -.5ex{\kern-0.75ex\hbox{$#1$}\vss}}}
\newcommand{\arrowvec}[1]{{\overset{\rightharpoonup}{#1}}}
%\renewcommand{\vec}[1]{\arrowvec{\mathbf{#1}}}
\renewcommand{\vec}[1]{{\overset{\boldsymbol{\rightharpoonup}}{\mathbf{#1}}}\hspace{0in}}

\newcommand{\point}[1]{\left(#1\right)} %this allows \vector{ to be changed to \vector{ with a quick find and replace
\newcommand{\pt}[1]{\mathbf{#1}} %this allows \vec{ to be changed to \vec{ with a quick find and replace
\newcommand{\Lim}[2]{\lim_{\point{#1} \to \point{#2}}} %Bart, I changed this to point since I want to use it.  It runs through both of the exercise and exerciseE files in limits section, which is why it was in each document to start with.

\DeclareMathOperator{\proj}{\mathbf{proj}}
\newcommand{\veci}{{\boldsymbol{\hat{\imath}}}}
\newcommand{\vecj}{{\boldsymbol{\hat{\jmath}}}}
\newcommand{\veck}{{\boldsymbol{\hat{k}}}}
\newcommand{\vecl}{\vec{\boldsymbol{\l}}}
\newcommand{\uvec}[1]{\mathbf{\hat{#1}}}
\newcommand{\utan}{\mathbf{\hat{t}}}
\newcommand{\unormal}{\mathbf{\hat{n}}}
\newcommand{\ubinormal}{\mathbf{\hat{b}}}

\newcommand{\dotp}{\bullet}
\newcommand{\cross}{\boldsymbol\times}
\newcommand{\grad}{\boldsymbol\nabla}
\newcommand{\divergence}{\grad\dotp}
\newcommand{\curl}{\grad\cross}
%\DeclareMathOperator{\divergence}{divergence}
%\DeclareMathOperator{\curl}[1]{\grad\cross #1}
\newcommand{\lto}{\mathop{\longrightarrow\,}\limits}

\renewcommand{\bar}{\overline}

\colorlet{textColor}{black}
\colorlet{background}{white}
\colorlet{penColor}{blue!50!black} % Color of a curve in a plot
\colorlet{penColor2}{red!50!black}% Color of a curve in a plot
\colorlet{penColor3}{red!50!blue} % Color of a curve in a plot
\colorlet{penColor4}{green!50!black} % Color of a curve in a plot
\colorlet{penColor5}{orange!80!black} % Color of a curve in a plot
\colorlet{penColor6}{yellow!70!black} % Color of a curve in a plot
\colorlet{fill1}{penColor!20} % Color of fill in a plot
\colorlet{fill2}{penColor2!20} % Color of fill in a plot
\colorlet{fillp}{fill1} % Color of positive area
\colorlet{filln}{penColor2!20} % Color of negative area
\colorlet{fill3}{penColor3!20} % Fill
\colorlet{fill4}{penColor4!20} % Fill
\colorlet{fill5}{penColor5!20} % Fill
\colorlet{gridColor}{gray!50} % Color of grid in a plot

\newcommand{\surfaceColor}{violet}
\newcommand{\surfaceColorTwo}{redyellow}
\newcommand{\sliceColor}{greenyellow}




\pgfmathdeclarefunction{gauss}{2}{% gives gaussian
  \pgfmathparse{1/(#2*sqrt(2*pi))*exp(-((x-#1)^2)/(2*#2^2))}%
}


%%%%%%%%%%%%%
%% Vectors
%%%%%%%%%%%%%

%% Simple horiz vectors
\renewcommand{\vector}[1]{\left\langle #1\right\rangle}


%% %% Complex Horiz Vectors with angle brackets
%% \makeatletter
%% \renewcommand{\vector}[2][ , ]{\left\langle%
%%   \def\nextitem{\def\nextitem{#1}}%
%%   \@for \el:=#2\do{\nextitem\el}\right\rangle%
%% }
%% \makeatother

%% %% Vertical Vectors
%% \def\vector#1{\begin{bmatrix}\vecListA#1,,\end{bmatrix}}
%% \def\vecListA#1,{\if,#1,\else #1\cr \expandafter \vecListA \fi}

%%%%%%%%%%%%%
%% End of vectors
%%%%%%%%%%%%%

%\newcommand{\fullwidth}{}
%\newcommand{\normalwidth}{}



%% makes a snazzy t-chart for evaluating functions
%\newenvironment{tchart}{\rowcolors{2}{}{background!90!textColor}\array}{\endarray}

%%This is to help with formatting on future title pages.
\newenvironment{sectionOutcomes}{}{}



%% Flowchart stuff
%\tikzstyle{startstop} = [rectangle, rounded corners, minimum width=3cm, minimum height=1cm,text centered, draw=black]
%\tikzstyle{question} = [rectangle, minimum width=3cm, minimum height=1cm, text centered, draw=black]
%\tikzstyle{decision} = [trapezium, trapezium left angle=70, trapezium right angle=110, minimum width=3cm, minimum height=1cm, text centered, draw=black]
%\tikzstyle{question} = [rectangle, rounded corners, minimum width=3cm, minimum height=1cm,text centered, draw=black]
%\tikzstyle{process} = [rectangle, minimum width=3cm, minimum height=1cm, text centered, draw=black]
%\tikzstyle{decision} = [trapezium, trapezium left angle=70, trapezium right angle=110, minimum width=3cm, minimum height=1cm, text centered, draw=black]


\outcome{Understand how to break up fractions.}
\outcome{Recognize integrals that are good candidates for the method of partial fractions.}
\outcome{Use long-division to simplify rational expressions.}
\outcome{Find the coefficients of partial fraction decomposition.}
\outcome{Use partial fractions to integrate functions.}

\title[Dig-In:]{Rational functions}

\begin{document}
\begin{abstract}
We can now integrate a large class of functions.  
\end{abstract}
\maketitle

\section{Basics of polynomial and rational functions}

In this course, we are attempting to learn to work with as many
functions as possible. A basic class of functions are \textit{polynomial functions}:

\begin{definition}
  A \dfn{polynomial function} in the variable $x$ is a function
  which can be written in the form
  \[
  f(x) = a_nx^n + a_{n-1}x^{n-1} + \dots + a_1 x + a_0
  \]
  where the $a_k$'s are all constants (called the \dfn{coefficients})
  and $n$ is a whole number (called the \dfn{degree} when $n\ne
  0$). The domain of a polynomial function is $(-\infty,\infty)$.
\end{definition}

\begin{question}
  Which of the following are polynomial functions?
  \begin{selectAll}
    \choice[correct]{$f(x) = 0$}
    \choice[correct]{$f(x) = -9$}
    \choice[correct]{$f(x) = 3x+1$}
    \choice{$f(x) = x^{1/2}-x +8$}
    \choice{$f(x) = -4x^{-3}+5x^{-1}+7-18x^2$}
    \choice[correct]{$f(x) = (x+1)(x-1)+e^x - e^x $}
    \choice{$f(x) = \frac{x^2 - 3x + 2}{x-2}$}
    \choice[correct]{$f(x) = x^7-32x^6-\pi x^3+45/84$}
  \end{selectAll}
\end{question}

As we will see, there is a fact about polynomials that is of critical
importance for this section:
\begin{quote}
  \textbf{Polynomials are equal as functions if and only if their
    coefficients are equal.}
\end{quote}

\begin{question}
  Given two polynomials equal as functions:
  \[
  6x^5+a_4 x^4 -x^2 + a_0 = a_5 x^5 - 24 x^4 + a_3 x^3 + a_2 x^2 - 5
  \]
  What are $a_0$, $a_1$, $a_2$, $a_3$, $a_4$, $a_5$?
  \begin{prompt}
    \begin{itemize}
    \item $a_0 = \answer{-5}$
    \item $a_1 = \answer{0}$
    \item $a_2 = \answer{-1}$
    \item $a_3 = \answer{0}$
    \item $a_4 = \answer{-24}$
    \item $a_5 = \answer{6}$
    \end{itemize}
  \end{prompt}
\end{question}

In the world of mathematics, polynomials are a generalization of
``integers,'' and rational numbers are fractions of integers. This
brings us to our next definition:

\begin{definition}
  A \dfn{rational function} in the variable $x$ is a function the form
  \[
  f(x) = \frac{p(x)}{q(x)}
  \]
  where $p$ and $q$ are polynomial functions. The domain of a rational
  function is all real numbers except for where the denominator is
  equal to zero.
\end{definition}

\begin{question}
  Which of the following are rational functions?
  \begin{selectAll}
    \choice[correct]{$f(x) = 0$}
    \choice[correct]{$f(x) = \frac{3x+1}{x^2-4x+5}$}
    \choice{$f(x)=e^x$}
    \choice{$f(x)=\frac{\sin(x)}{\cos(x)}$}
    \choice[correct]{$f(x) = -4x^{-3}+5x^{-1}+7-18x^2$}
    \choice{$f(x) = x^{1/2}-x +8$}
    \choice{$f(x)=\frac{\sqrt{x}}{x^3-x}$}
  \end{selectAll}
  \begin{feedback}
    All polynomials can be thought of as rational functions.
  \end{feedback}
\end{question}


\section{Denominators with distinct linear factors}


We are already skilled at working with polynomials, we can
differentiate and integrate any polynomial function. Being able to
integrate \textit{any} rational function is the next logical step in
our (rather ambitious) quest to integrate \textit{all}
functions. Let's dig right in with an example.

\begin{example}
  Compute:
  \[
  \int \frac{1}{x^2-1} \d x
  \]
  \begin{explanation}
    We will suppose that there are numbers $A$ and $B$ such that
    \[
    \frac{A}{x-1} + \frac{B}{x+1} = \frac{1}{x^2-1}.
    \]
    Clearing denominators, we find
    \[
    A\answer[given]{(x+1)} + B\answer[given]{(x-1)} = 1.
    \]
    Expanding the left-hand side, we find a polynomial equal (as a
    function!) to the constant polynomial $1$
    \[
    (A+ B)x + (A-B) = 1
    \]
    Since \textbf{polynomials are equal as functions if and only if
      their coefficients are equal}, we may rewrite this as
    \textbf{two} equations:
    \begin{align*}
      \answer[given]{A+B} &= 0 &\text{(the coefficients for $x$)}\\
      \answer[given]{A-B} &= 1 &\text{(the coefficients for the constant)}
    \end{align*}
    Solving these three equations for $A$ and $B$ we find
    \begin{itemize}
    \item $A = \answer[given]{1/2}$,
    \item $B = \answer[given]{-1/2}$.
    \end{itemize} 
    From this we can now rewrite our integral as
    \begin{align*}
      \int \frac{1}{x^2-1} \d x &=  \int \frac{1/2}{x-1} + \frac{-1/2}{x+1} \d x\\
      &= (1/2)\ln|x-1| + (-1/2)\ln|x+1|+ K.
    \end{align*}
  \end{explanation}
\end{example}

What we have seen is part of a general technique of integration called
``partial fractions''\index{partial fractions} that, in principle,
allows us to integrate any rational function.

\subsection{The general technique for distinct linear factors}

Suppose you wish to compute
\[
\int \frac{p(x)}{q(x)} \d x
\]
where $p$ and $q$ are both polynomial functions, the degree of $p$ is
less than the degree of $q$, and $q$ factors into $n$
\textbf{distinct} linear factors:
\[
q(x) = (x-r_1) (x-r_2) \cdots (x-r_n),
\]
then we can \textbf{always} write
\[
\frac{p(x)}{q(x)}  = \frac{A_1}{(x-r_1)} + \frac{A_2}{(x-r_2)} + \cdots + \frac{A_n}{(x-r_n)}. 
\]
The right-hand side of the equation above is easy to antidifferentiate,
as we can integrate it term-by-term and
\[
\int \frac{A_k}{(x-r_k)} \d x = A \ln(x-r_k) + K,
\]
hence
\[
\int \frac{p(x)}{q(x)} \d x = \sum_{k=1}^n A_k \ln|x-r_k| +K.
\]








\section{Denominators with repeated linear factors}

Here we work as we did before, except we add an extra variable for
each of the repeated factors. Let's do an example.

\begin{example}
  Compute:
  \[
  \int \frac{1}{(x-1)(x+2)^2} \d x
  \]
  \begin{explanation}
    We will suppose that there are numbers $A$, $B$, and $C$ such that
    \[
    \frac{A}{x-1} + \frac{B}{x+2} + \frac{C}{(x+2)^2} = \frac{1}{(x-1)(x+2)^2}
    \]
    Clearing denominators, we find
    \[
    A\answer[given]{(x+2)^2} + B\answer[given]{(x-1)(x+2)} + C\answer[given]{(x-1)} = 1.
    \]
    Expanding the left-hand side, we find a polynomial equal (as a
    function!) to the constant polynomial $1$
    \[
    (A+B)x^2 + (4A+B+C)x + (4A-2B-C) = 1
    \]
    Since \textbf{polynomials are equal as functions if and only if
      their coefficients are equal}, we may rewrite this as
    \textbf{three} equations:
    \begin{align*}
      \answer[given]{A+B} &= 0 &\text{(the coefficients for $x^2$)}\\
      \answer[given]{4A+B+C} &= 0 &\text{(the coefficients for $x$)}\\
      \answer[given]{4A-2B-C} &= 1 &\text{(the coefficients for the constant)}
    \end{align*}
    Solving these three equations for $A$, $B$ and $C$ we find
    \begin{itemize}
    \item $A = \answer[given]{1/9}$,
    \item $B = \answer[given]{-1/9}$,
    \item $C = \answer[given]{-1/3}$.
    \end{itemize}
    From this we can now rewrite our integral as
    \begin{align*}
      \int&\frac{1}{(x-1)(x+2)^2}\d x= \int \frac{1/9}{x-1}+ \frac{-1/9}{x+2} + \frac{-1/3}{(x+2)^2}\d x\\
      &= \int \frac{1/9}{x-1}\d x + \int \frac{-1/9}{x+2}\d x + \int \frac{-1/3}{(x+2)^2}\d x,\\
      &= \frac{1}{9}\ln|x-1| -\frac{1}{9}\ln|x+2| +\frac{1}{3(x+2)}+K.
    \end{align*}
  \end{explanation}
\end{example}



\subsection{The general technique for repeated linear factors}

Suppose you wish to compute
\[
\int \frac{p(x)}{q(x)} \d x
\]
where $p$ and $q$ are both polynomial functions, the degree of $p$ is
less than the degree of $q$, and $q$ factors into $n$
\textbf{repeated} linear factors:
\[
q(x) = (x-r)^n,
\]
then we can \textbf{always} write
\[
\frac{p(x)}{q(x)}  = \frac{A_1}{(x-r)} + \frac{A_2}{(x-r)^2} + \cdots + \frac{A_n}{(x-r)^n}. 
\]
The right-hand side of the equation above is easy to
antidifferentiate, as we can integrate it term-by-term and
\begin{align*}
  \int \frac{A_1}{(x-r)} \d x &= A_1 \ln|x-r| + K,\\
  \int \frac{A_k}{(x-r)^k} \d x &= \frac{A_k}{(1-k)(x-r)^{k-1}}+ K, &\text{(if $k>1$)}
\end{align*}
hence
\[
\int \frac{p(x)}{q(x)} \d x = A_1 \ln|x-r|  + \sum_{k=2}^n \frac{A_k}{(1-k)(x-r)^{k-1}} + K.
\]






\section{Denominators with distinct quadratic factors}

Here is a fact about polynomials:

\begin{theorem}[The Fundamental Theorem of Algebra]\index{Fundamental Theorem of Algebra}
  Every polynomial of the form
  \[
  a_n x^n + a_{n-1} x^{n-1} + \dots + a_1 x + a_0
  \]
  where the $a_i$'s are real (or even complex!) numbers and $a_n \ne 0$ has exactly
  $n$ (possibly repeated) complex roots.
\end{theorem}

Remember, a \dfn{root} is where a polynomial is zero. The theorem
above is a deep fact of mathematics. The great mathematician Gauss
%(spelled Gau\ss\ for fancy people)
proved the theorem in 1799 for his doctoral thesis. This fact can be
used to show the following:
\begin{quote}
  \textbf{Every polynomial function will factor as a product of linear
    terms and quadratic terms over the real numbers.}
\end{quote}

So now let's work an example where the denominator of our rational
function has distinct quadratic factors.

\begin{example}
  Compute:
  \[
  \int\frac{7x^2+31x+54}{(x+1)(x^2+6x+11)}\d x
  \]
  \begin{explanation}
    We will suppose that there are numbers $A$, $B$, and $C$ such that
    \[
    \frac{A}{x+1} + \frac{Bx+C}{x^2+6x+11} = \frac{7x^2+31x+54}{(x+1)(x^2+6x+11)}.
    \]
    Clearing denominators, we find
    \[
    A\answer[given]{(x^2+6x+11)} + (Bx+C)\answer[given]{(x+1)} = 7x^2+31x+54.
    \]
    Expanding the left-hand side, we find a polynomial equal (as a
    function!) to the polynomial $7x^2+31x+54$
    \[
    (A+B)x^2 + (6A+B+C)x + (11A+C) = 7x^2+31x+54.
    \]
    Since \textbf{polynomials are equal as functions if and only if
      their coefficients are equal}, we may rewrite this as
    \textbf{three} equations:
    \begin{align*}
      \answer[given]{A+B} &= 7 &\text{(the coefficients for $x^2$)}\\
      \answer[given]{6A+B+C} &= 31 &\text{(the coefficients for $x$)}\\
      \answer[given]{11A+C} &= 54 &\text{(the coefficients for the constant)}
    \end{align*}
    Solving these three equations for $A$, $B$ and $C$ we find
    \begin{itemize}
    \item $A = \answer[given]{5}$,
    \item $B = \answer[given]{2}$,
    \item $C = \answer[given]{-1}$.
    \end{itemize}
    From this we can now rewrite our integral as
    \[
    \int\frac{7x^2+31x+54}{(x+1)(x^2+6x+11)}\d x = \int\frac{5}{x+1} + \frac{2x-1}{x^2+6x+11} \d x\\
    \]
    The first term of this new integrand is easy to evaluate. We find
    \[
    \int \frac{5}{x+1} \d x = 5\ln|x+1|+K.
    \]
    The second term is not hard, but takes several steps and uses
    substitution techniques.

    The integrand $\frac{2x-1}{x^2+6x+11}$ has a quadratic in the
    denominator and a linear term in the numerator. This leads us to
    try substitution. Let
    \begin{align*}
      g &= x^2+6x+11,\\
      \d g  &= (2x+6)\d x.
    \end{align*}
    However, the numerator is $2x-1$, not $2x+6$! We can bypass this
    difficulty by adding ``$0$'' in the form of ``$7-7$.''
\begin{align*}
  \frac{2x-1}{x^2+6x+11} &= \frac{2x-1+7-7}{x^2+6x+11} \\
  &= \frac{2x+6}{x^2+6x+11} - \frac{7}{x^2+6x+11}.
\end{align*}
We can now integrate the first term with substitution, leading to
\[
\int \frac{2x+6}{x^2+6x+11} \d x = \ln|x^2+6x+11| + K.
\]
The final term can be integrated using arctangent. First, complete the
square in the denominator:
\[
\frac{7}{x^2+6x+11} = \frac{7}{(x+3)^2+2}.
\]
then use a substitution of $g = x+3$ to find
\[
\int \frac{7}{x^2+6x+11}\d x = \frac{7}{\sqrt{2}}\arctan\left(\frac{x+3}{\sqrt{2}}\right)+K.
\]
Let's start at the beginning and put all of the steps together.
\begin{align*}
  \int&\frac{7x^2+31x+54}{(x+1)(x^2+6x+11)}\d x \\
  &= \int\left(\frac{5}{x+1} + \frac{2x-1}{x^2+6x+11}\right)\d x
\end{align*}
breaking this integral up we find
\begin{align*}
  = \int\frac{5}{x+1}\d x  &+ \int\frac{2x+6}{x^2+6x+11}\d x\\
  &-\int\frac{7}{x^2+6x+11}\d x
\end{align*}
and antidifferentiating we find
\[
  = 5\ln|x+1|+ \ln|x^2+6x+11|-\frac{7}{\sqrt{2}}\arctan\left(\frac{x+3}{\sqrt{2}}\right)+K.
\]
  \end{explanation}
\end{example}


\subsection{The general technique for distinct quadratic factors}

Suppose you wish to compute
\[
\int \frac{p(x)}{q(x)} \d x
\]
where $p$ and $q$ are both polynomial functions, the degree of $p$ is
less than the degree of $q$, and $q$ factors into $n$
\textbf{distinct} irreducible quadratic factors:
\[
q(x) = (a_1x^2 + b_1 x + c_1) (a_2x^2 + b_2 x + c_2)\cdots  (a_nx^2 + b_n x + c_n) 
\]
then we can \textbf{always} write
\[
\frac{p(x)}{q(x)}  = \frac{A_1x+B_1}{a_1x^2 + b_1 x + c_1} + \cdots + \frac{A_nx+B_n}{a_nx^2 + b_n x + c_n}. 
\]
The right-hand side of the equation can be antidifferentiated, though it is not always ``easy.''







\section{Denominators with repeated quadratic factors}

For completeness sake, we will work a problem with repeated quadratic factors.

\begin{example}
  Compute
  \[
  \int \frac{1}{x^5 + 2x^3  + x}\d x
  \]
  \begin{explanation}
    Start by factoring the denominator
    \[
    x^5 + 2x^3  + x = x(x^2+1)^2.
    \]
    We will suppose that there are numbers $A$, $B$, $C$, $D$, and $E$
    such that
    \[
    \frac{A}{x} + \frac{Bx+C}{x^2+1} + \frac{Dx+E}{(x^2+1)^2} = \frac{1}{x(x^2+1)^2}.
    \]
    Clearing denominators, we find
    \[
    A\answer[given]{(x^2+1)^2} + (Bx+C)\answer[given]{(x^3+x)}+ (Dx+E)\answer[given]{(x)} = 1
    \]
    Expanding the left-hand side, we find a polynomial equal (as a
    function!) to the polynomial $7x^2+31x+54$
    \[
    (A+B)^4 +  Cx^3 + (2A+B+D)x^2 + (C+E)x + A = 1.
    \]
    Since \textbf{polynomials are equal as functions if and only if
      their coefficients are equal}, we may rewrite this as
    \textbf{five} equations:
    \begin{align*}
      \answer[given]{A+B} &= 0 &\text{(the coefficients for $x^4$)}\\
      \answer[given]{C} &= 0 &\text{(the coefficients for $x^3$)}\\
      \answer[given]{2A+B+D} &= 0 &\text{(the coefficients for $x^2$)}\\
      \answer[given]{C+E} &= 0 &\text{(the coefficients for $x$)}\\
      \answer[given]{A} &= 1 &\text{(the coefficients for the constant)}
    \end{align*}
    Solving these three equations for $A$, $B$, $C$, $D$, and $E$ we find
    \begin{itemize}
    \item $A = \answer[given]{1}$,
    \item $B = \answer[given]{-1}$,
    \item $C = \answer[given]{0}$,
    \item $D = \answer[given]{-1}$,
    \item $E = \answer[given]{0}$.
    \end{itemize}
    From this we can now rewrite our integral as
    \[
    \int \frac{1}{x^5 + 2x^3  + x}\d x =\int \frac{1}{x} + \frac{-x}{x^2+1} + \frac{-x}{(x^2+1)^2}\d x
    \]
    Each term of this new integrand is easy to evaluate, write
    \begin{align*}
      \int \frac{1}{x} \d x &= \ln|x| + K,\\
      \int \frac{-x}{x^2+1} \d x &= \frac{-\ln|x^2+1|}{2} +K,\\
      \int \frac{-x}{(x^2+1)^2} \d x &= \frac{1}{2(x^2+1)}+K.
    \end{align*}
    So
    \begin{align*}
    \int &\frac{1}{x^5 + 2x^3  + x}\d x \\
    &=\ln|x| + \frac{-\ln|x^2+1|}{2} + \frac{1}{2(x^2+1)}+K.
    \end{align*}
  \end{explanation}
\end{example}



\subsection{The general technique for repeated quadratic factors}
Suppose you wish to compute
\[
\int \frac{p(x)}{q(x)} \d x
\]
where $p$ and $q$ are both polynomial functions, the degree of $p$ is
less than the degree of $q$, and $q$ factors into $n$
\textbf{distinct} irreducible quadratic factors:
\[
q(x) = (ax^2 + b x + c)^n 
\]
then we can \textbf{always} write
\[
\frac{p(x)}{q(x)}  = \frac{A_1x+B_1}{ax^2 + bx + c} + \frac{A_2x+B_2}{(ax^2 + bx + c)^2} + \cdots + \frac{A_nx+B_n}{(ax^2 + b x + c)^n}. 
\]
The right-hand side of the equation can be antidifferentiated, though it is not always ``easy.''


\section{Reducing rational functions}

When computing
\[
\int \frac{p(x)}{q(x)} \d x
\]
all of the techniques above rely on the fact that the degree of $p$ is
less than the degree of $q$. What if this is not the case? Use
\index{long-division}long-division.

\begin{example}
  Compute:
  \[
  \int \frac{x^3+1}{x^2-1} \d x
  \]
  \begin{explanation}
    We start by using long-division to reduce the numerator:
    \begin{image}[2in]
      \begin{tikzpicture}[]
        \node at (0,0) {
          $x^2-0x-1\,\begin{array}[b]{@{}r@{}r} 
          x &\\ 
          \cline{1-1}
          \Bigg)\begin{array}[t]{@{}l@{}} x^3+0x^2+0x+1\\ 
            x^3+0x^2-x \\ 
            \divrule{0}{8}  ~~~~~~~~~x+1
          \end{array}
          \end{array}
          $
        };
      \end{tikzpicture}
    \end{image}
    Now our integral becomes
    \[
    \int \frac{x^3+1}{x^2-1}\d x = \int x+ \frac{x+1}{x^2-1}\d x
    \]
    From here, we can use our techniques from before to complete the
    computation.
  \end{explanation}
\end{example}
    


As with many other problems in calculus, it is important to remember
that one is not expected to ``see'' the final answer immediately after
seeing the problem. Rather, given the initial problem, we break it
down into smaller problems that are easier to solve. The final answer
is a combination of the answers of the smaller problems.
\end{document}
