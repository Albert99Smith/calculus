\documentclass{ximera}

\author{Jim Talamo}

%\usepackage{todonotes}
%\usepackage{mathtools} %% Required for wide table Curl and Greens
%\usepackage{cuted} %% Required for wide table Curl and Greens
\newcommand{\todo}{}

\usepackage{esint} % for \oiint
\ifxake%%https://math.meta.stackexchange.com/questions/9973/how-do-you-render-a-closed-surface-double-integral
\renewcommand{\oiint}{{\large\bigcirc}\kern-1.56em\iint}
\fi


\graphicspath{
  {./}
  {ximeraTutorial/}
  {basicPhilosophy/}
  {functionsOfSeveralVariables/}
  {normalVectors/}
  {lagrangeMultipliers/}
  {vectorFields/}
  {greensTheorem/}
  {shapeOfThingsToCome/}
  {dotProducts/}
  {partialDerivativesAndTheGradientVector/}
  {../productAndQuotientRules/exercises/}
  {../normalVectors/exercisesParametricPlots/}
  {../continuityOfFunctionsOfSeveralVariables/exercises/}
  {../partialDerivativesAndTheGradientVector/exercises/}
  {../directionalDerivativeAndChainRule/exercises/}
  {../commonCoordinates/exercisesCylindricalCoordinates/}
  {../commonCoordinates/exercisesSphericalCoordinates/}
  {../greensTheorem/exercisesCurlAndLineIntegrals/}
  {../greensTheorem/exercisesDivergenceAndLineIntegrals/}
  {../shapeOfThingsToCome/exercisesDivergenceTheorem/}
  {../greensTheorem/}
  {../shapeOfThingsToCome/}
  {../separableDifferentialEquations/exercises/}
  {vectorFields/}
}

\newcommand{\mooculus}{\textsf{\textbf{MOOC}\textnormal{\textsf{ULUS}}}}

\usepackage{tkz-euclide}\usepackage{tikz}
\usepackage{tikz-cd}
\usetikzlibrary{arrows}
\tikzset{>=stealth,commutative diagrams/.cd,
  arrow style=tikz,diagrams={>=stealth}} %% cool arrow head
\tikzset{shorten <>/.style={ shorten >=#1, shorten <=#1 } } %% allows shorter vectors

\usetikzlibrary{backgrounds} %% for boxes around graphs
\usetikzlibrary{shapes,positioning}  %% Clouds and stars
\usetikzlibrary{matrix} %% for matrix
\usepgfplotslibrary{polar} %% for polar plots
\usepgfplotslibrary{fillbetween} %% to shade area between curves in TikZ
\usetkzobj{all}
\usepackage[makeroom]{cancel} %% for strike outs
%\usepackage{mathtools} %% for pretty underbrace % Breaks Ximera
%\usepackage{multicol}
\usepackage{pgffor} %% required for integral for loops



%% http://tex.stackexchange.com/questions/66490/drawing-a-tikz-arc-specifying-the-center
%% Draws beach ball
\tikzset{pics/carc/.style args={#1:#2:#3}{code={\draw[pic actions] (#1:#3) arc(#1:#2:#3);}}}



\usepackage{array}
\setlength{\extrarowheight}{+.1cm}
\newdimen\digitwidth
\settowidth\digitwidth{9}
\def\divrule#1#2{
\noalign{\moveright#1\digitwidth
\vbox{\hrule width#2\digitwidth}}}





\newcommand{\RR}{\mathbb R}
\newcommand{\R}{\mathbb R}
\newcommand{\N}{\mathbb N}
\newcommand{\Z}{\mathbb Z}

\newcommand{\sagemath}{\textsf{SageMath}}


%\renewcommand{\d}{\,d\!}
\renewcommand{\d}{\mathop{}\!d}
\newcommand{\dd}[2][]{\frac{\d #1}{\d #2}}
\newcommand{\pp}[2][]{\frac{\partial #1}{\partial #2}}
\renewcommand{\l}{\ell}
\newcommand{\ddx}{\frac{d}{\d x}}

\newcommand{\zeroOverZero}{\ensuremath{\boldsymbol{\tfrac{0}{0}}}}
\newcommand{\inftyOverInfty}{\ensuremath{\boldsymbol{\tfrac{\infty}{\infty}}}}
\newcommand{\zeroOverInfty}{\ensuremath{\boldsymbol{\tfrac{0}{\infty}}}}
\newcommand{\zeroTimesInfty}{\ensuremath{\small\boldsymbol{0\cdot \infty}}}
\newcommand{\inftyMinusInfty}{\ensuremath{\small\boldsymbol{\infty - \infty}}}
\newcommand{\oneToInfty}{\ensuremath{\boldsymbol{1^\infty}}}
\newcommand{\zeroToZero}{\ensuremath{\boldsymbol{0^0}}}
\newcommand{\inftyToZero}{\ensuremath{\boldsymbol{\infty^0}}}



\newcommand{\numOverZero}{\ensuremath{\boldsymbol{\tfrac{\#}{0}}}}
\newcommand{\dfn}{\textbf}
%\newcommand{\unit}{\,\mathrm}
\newcommand{\unit}{\mathop{}\!\mathrm}
\newcommand{\eval}[1]{\bigg[ #1 \bigg]}
\newcommand{\seq}[1]{\left( #1 \right)}
\renewcommand{\epsilon}{\varepsilon}
\renewcommand{\phi}{\varphi}


\renewcommand{\iff}{\Leftrightarrow}

\DeclareMathOperator{\arccot}{arccot}
\DeclareMathOperator{\arcsec}{arcsec}
\DeclareMathOperator{\arccsc}{arccsc}
\DeclareMathOperator{\si}{Si}
\DeclareMathOperator{\scal}{scal}
\DeclareMathOperator{\sign}{sign}


%% \newcommand{\tightoverset}[2]{% for arrow vec
%%   \mathop{#2}\limits^{\vbox to -.5ex{\kern-0.75ex\hbox{$#1$}\vss}}}
\newcommand{\arrowvec}[1]{{\overset{\rightharpoonup}{#1}}}
%\renewcommand{\vec}[1]{\arrowvec{\mathbf{#1}}}
\renewcommand{\vec}[1]{{\overset{\boldsymbol{\rightharpoonup}}{\mathbf{#1}}}\hspace{0in}}

\newcommand{\point}[1]{\left(#1\right)} %this allows \vector{ to be changed to \vector{ with a quick find and replace
\newcommand{\pt}[1]{\mathbf{#1}} %this allows \vec{ to be changed to \vec{ with a quick find and replace
\newcommand{\Lim}[2]{\lim_{\point{#1} \to \point{#2}}} %Bart, I changed this to point since I want to use it.  It runs through both of the exercise and exerciseE files in limits section, which is why it was in each document to start with.

\DeclareMathOperator{\proj}{\mathbf{proj}}
\newcommand{\veci}{{\boldsymbol{\hat{\imath}}}}
\newcommand{\vecj}{{\boldsymbol{\hat{\jmath}}}}
\newcommand{\veck}{{\boldsymbol{\hat{k}}}}
\newcommand{\vecl}{\vec{\boldsymbol{\l}}}
\newcommand{\uvec}[1]{\mathbf{\hat{#1}}}
\newcommand{\utan}{\mathbf{\hat{t}}}
\newcommand{\unormal}{\mathbf{\hat{n}}}
\newcommand{\ubinormal}{\mathbf{\hat{b}}}

\newcommand{\dotp}{\bullet}
\newcommand{\cross}{\boldsymbol\times}
\newcommand{\grad}{\boldsymbol\nabla}
\newcommand{\divergence}{\grad\dotp}
\newcommand{\curl}{\grad\cross}
%\DeclareMathOperator{\divergence}{divergence}
%\DeclareMathOperator{\curl}[1]{\grad\cross #1}
\newcommand{\lto}{\mathop{\longrightarrow\,}\limits}

\renewcommand{\bar}{\overline}

\colorlet{textColor}{black}
\colorlet{background}{white}
\colorlet{penColor}{blue!50!black} % Color of a curve in a plot
\colorlet{penColor2}{red!50!black}% Color of a curve in a plot
\colorlet{penColor3}{red!50!blue} % Color of a curve in a plot
\colorlet{penColor4}{green!50!black} % Color of a curve in a plot
\colorlet{penColor5}{orange!80!black} % Color of a curve in a plot
\colorlet{penColor6}{yellow!70!black} % Color of a curve in a plot
\colorlet{fill1}{penColor!20} % Color of fill in a plot
\colorlet{fill2}{penColor2!20} % Color of fill in a plot
\colorlet{fillp}{fill1} % Color of positive area
\colorlet{filln}{penColor2!20} % Color of negative area
\colorlet{fill3}{penColor3!20} % Fill
\colorlet{fill4}{penColor4!20} % Fill
\colorlet{fill5}{penColor5!20} % Fill
\colorlet{gridColor}{gray!50} % Color of grid in a plot

\newcommand{\surfaceColor}{violet}
\newcommand{\surfaceColorTwo}{redyellow}
\newcommand{\sliceColor}{greenyellow}




\pgfmathdeclarefunction{gauss}{2}{% gives gaussian
  \pgfmathparse{1/(#2*sqrt(2*pi))*exp(-((x-#1)^2)/(2*#2^2))}%
}


%%%%%%%%%%%%%
%% Vectors
%%%%%%%%%%%%%

%% Simple horiz vectors
\renewcommand{\vector}[1]{\left\langle #1\right\rangle}


%% %% Complex Horiz Vectors with angle brackets
%% \makeatletter
%% \renewcommand{\vector}[2][ , ]{\left\langle%
%%   \def\nextitem{\def\nextitem{#1}}%
%%   \@for \el:=#2\do{\nextitem\el}\right\rangle%
%% }
%% \makeatother

%% %% Vertical Vectors
%% \def\vector#1{\begin{bmatrix}\vecListA#1,,\end{bmatrix}}
%% \def\vecListA#1,{\if,#1,\else #1\cr \expandafter \vecListA \fi}

%%%%%%%%%%%%%
%% End of vectors
%%%%%%%%%%%%%

%\newcommand{\fullwidth}{}
%\newcommand{\normalwidth}{}



%% makes a snazzy t-chart for evaluating functions
%\newenvironment{tchart}{\rowcolors{2}{}{background!90!textColor}\array}{\endarray}

%%This is to help with formatting on future title pages.
\newenvironment{sectionOutcomes}{}{}



%% Flowchart stuff
%\tikzstyle{startstop} = [rectangle, rounded corners, minimum width=3cm, minimum height=1cm,text centered, draw=black]
%\tikzstyle{question} = [rectangle, minimum width=3cm, minimum height=1cm, text centered, draw=black]
%\tikzstyle{decision} = [trapezium, trapezium left angle=70, trapezium right angle=110, minimum width=3cm, minimum height=1cm, text centered, draw=black]
%\tikzstyle{question} = [rectangle, rounded corners, minimum width=3cm, minimum height=1cm,text centered, draw=black]
%\tikzstyle{process} = [rectangle, minimum width=3cm, minimum height=1cm, text centered, draw=black]
%\tikzstyle{decision} = [trapezium, trapezium left angle=70, trapezium right angle=110, minimum width=3cm, minimum height=1cm, text centered, draw=black]


\outcome{Evaluate limits of functions of several variables.}
\newcommand{\point}[1]{\left(#1\right)} %this allows \vector{ to be changed to \point{ with a quick find and replace
\newcommand{\pt}[1]{\mathbf{#1}} %this allows \vec{ to be changed to \pt{ with a quick find and replace
\newcommand{\Lim}[2]{\lim_{#1 \to #2}}

\begin{document}
\begin{exercise}
 
Consider the function $f(x,y) = \frac{2x^3y-4x^2y^2}{2x^4+y^4}$.

The domain of this function is \wordChoice{\choice{$\R$}\choice{$\left\{ z \in \R \big| z \neq 0 \right\}$}\choice{$\R^2$}\choice[correct]{$\left\{ (x,y) \in \R^2 \big| (x,y) \neq (0,0) \right\}$}}.

We now examine whether $\Lim{\vector{x,y}}{\vector{0,0}} f(x,y)$ exists.  By noting that the degree of each term in the numerator and denominator are the same, we can try to choose a type of path along which we can induce algebraic cancellation.

Which type of path would likely be a good choice?

\begin{multipleChoice}
\choice[correct]{lines}
\choice{planes}
\choice{parabolas}
\choice{circles}
\end{multipleChoice}

\begin{exercise}
Let's try to analyze along two specific lines.

\begin{itemize}
\item Along $x=0$, we find that 

\[
f(x,y) = f(0,y) =\frac{2\left(\answer{0}\right)^3y-4\left(\answer{0}\right)^2y^2}{2\left(\answer{0}\right)^4+y^4} = \answer{0}, \textrm{ for } y \neq 0.
\]

Thus, $f(x,y) \to \answer{0}$ as $\vector{x,y} \to \vector{0,0}$ along the path $x=0$.

\item Along $y=0$, we find that 

\[
f(x,y) = f\left(\answer{x},\answer{0}\right) = \answer{0} , \textrm{ for } x \neq 0.
\]

Thus, $f(x,y) \to \answer{0}$ as $\vector{x,y} \to \vector{0,0}$ along the path $y=0$.
\end{itemize}

Is this enough to conclude that $\Lim{\vector{x,y}}{\vector{0,0}}$ exists?

\begin{multipleChoice}
\choice{Yes}
\choice[correct]{No}
\end{multipleChoice}

\begin{exercise}
Note that \emph{if} $\Lim{\vector{x,y}}{\vector{0,0}} f(x,y)$ exists, the above shows that it \emph{must} be $0$; that is, along \emph{every} path, the function must tend to $0$.

Let's pick another explicit path, say $y=2x$.  

\[
f(x,y) = f\left(x,\answer{2x}\right) =\frac{2x^3\left(\answer{2x}\right)-4x^2\left(\answer{2x}\right)^2}{2x^4+\left(\answer{2x}\right)^4} = \frac{\answer{-12}\cdot x^4}{\answer{18}\cdot x^4} = -\frac{2}{3}, x \neq 0.
\]

Thus, $f(x,y) \to \answer{-\frac{2}{3}}$ as $\vector{x,y} \to \vector{0,0}$ along the path $y=2x$.

Is this enough to conclude that $\Lim{\vector{x,y}}{\vector{0,0}}$ exists or does not exist?

\begin{multipleChoice}
\choice{Yes; $\Lim{\vector{x,y}}{\vector{0,0}}$ exists.}
\choice[correct]{Yes; $\Lim{\vector{x,y}}{\vector{0,0}}$ does not exist.}
\choice{No}
\end{multipleChoice}

\end{exercise}
\end{exercise}
%%%%%%%%%%%%%%%%%%%%%%%%%%%%%
\begin{exercise}
Since we think that analyzing the outputs of the function along lines in its domain, we can require that $y=mx$ and see what happens.

Along $y=mx$, we have the following.  

\[
f(x,y) = f\left(x,\answer{mx}\right) =\frac{2x^3\left(\answer{mx}\right)-4x^2\left(\answer{mx}\right)^2}{2x^4+\left(\answer{mx}\right)^4} = \frac{\answer{2m-4m^2}\cdot x^4}{\answer{2+m^4}\cdot x^4} = \frac{\answer{2m-4m^2}}{2+m^4}, x \neq 0.
\]

Thus, $f(x,y) \to \frac{\answer{2m-4m^2}}{2+m^4}$ as $\vector{x,y} \to \vector{0,0}$ along the path $y=mx$.

Is this enough to conclude that $\Lim{\vector{x,y}}{\vector{0,0}}$ exists or does not exist?

\begin{multipleChoice}
\choice{Yes; $\Lim{\vector{x,y}}{\vector{0,0}}$ exists because $f(x,y)$ tends to a number along each path.}
\choice[correct]{Yes; $\Lim{\vector{x,y}}{\vector{0,0}}$ does not exist because $f(x,y)$ approaches a different value for different choices of $m$.}
\choice{No}
\end{multipleChoice}

\begin{exercise}
The results of both approaches are related.

\begin{itemize}
\item The path $y=0$ is obtained from $y=mx$ by setting $m = \answer{0}$.  
\item The path $x=0$ is is a vertical line, so we take $m \to \infty$.
\item The path $y=2x$ is obtained from $y=mx$ by setting $m = \answer{2}$.

Do the results of both agree?

One nice consequence of analyzing the function along paths of the form $y=mx$ is that this gives explicit information about how the outputs of the function depend on the choice of path.
\end{itemize}
\end{exercise}
\end{exercise}
 
\end{exercise}
\end{document}
