\documentclass{ximera}

%\usepackage{todonotes}
%\usepackage{mathtools} %% Required for wide table Curl and Greens
%\usepackage{cuted} %% Required for wide table Curl and Greens
\newcommand{\todo}{}

\usepackage{esint} % for \oiint
\ifxake%%https://math.meta.stackexchange.com/questions/9973/how-do-you-render-a-closed-surface-double-integral
\renewcommand{\oiint}{{\large\bigcirc}\kern-1.56em\iint}
\fi


\graphicspath{
  {./}
  {ximeraTutorial/}
  {basicPhilosophy/}
  {functionsOfSeveralVariables/}
  {normalVectors/}
  {lagrangeMultipliers/}
  {vectorFields/}
  {greensTheorem/}
  {shapeOfThingsToCome/}
  {dotProducts/}
  {partialDerivativesAndTheGradientVector/}
  {../productAndQuotientRules/exercises/}
  {../normalVectors/exercisesParametricPlots/}
  {../continuityOfFunctionsOfSeveralVariables/exercises/}
  {../partialDerivativesAndTheGradientVector/exercises/}
  {../directionalDerivativeAndChainRule/exercises/}
  {../commonCoordinates/exercisesCylindricalCoordinates/}
  {../commonCoordinates/exercisesSphericalCoordinates/}
  {../greensTheorem/exercisesCurlAndLineIntegrals/}
  {../greensTheorem/exercisesDivergenceAndLineIntegrals/}
  {../shapeOfThingsToCome/exercisesDivergenceTheorem/}
  {../greensTheorem/}
  {../shapeOfThingsToCome/}
  {../separableDifferentialEquations/exercises/}
  {vectorFields/}
}

\newcommand{\mooculus}{\textsf{\textbf{MOOC}\textnormal{\textsf{ULUS}}}}

\usepackage{tkz-euclide}\usepackage{tikz}
\usepackage{tikz-cd}
\usetikzlibrary{arrows}
\tikzset{>=stealth,commutative diagrams/.cd,
  arrow style=tikz,diagrams={>=stealth}} %% cool arrow head
\tikzset{shorten <>/.style={ shorten >=#1, shorten <=#1 } } %% allows shorter vectors

\usetikzlibrary{backgrounds} %% for boxes around graphs
\usetikzlibrary{shapes,positioning}  %% Clouds and stars
\usetikzlibrary{matrix} %% for matrix
\usepgfplotslibrary{polar} %% for polar plots
\usepgfplotslibrary{fillbetween} %% to shade area between curves in TikZ
\usetkzobj{all}
\usepackage[makeroom]{cancel} %% for strike outs
%\usepackage{mathtools} %% for pretty underbrace % Breaks Ximera
%\usepackage{multicol}
\usepackage{pgffor} %% required for integral for loops



%% http://tex.stackexchange.com/questions/66490/drawing-a-tikz-arc-specifying-the-center
%% Draws beach ball
\tikzset{pics/carc/.style args={#1:#2:#3}{code={\draw[pic actions] (#1:#3) arc(#1:#2:#3);}}}



\usepackage{array}
\setlength{\extrarowheight}{+.1cm}
\newdimen\digitwidth
\settowidth\digitwidth{9}
\def\divrule#1#2{
\noalign{\moveright#1\digitwidth
\vbox{\hrule width#2\digitwidth}}}





\newcommand{\RR}{\mathbb R}
\newcommand{\R}{\mathbb R}
\newcommand{\N}{\mathbb N}
\newcommand{\Z}{\mathbb Z}

\newcommand{\sagemath}{\textsf{SageMath}}


%\renewcommand{\d}{\,d\!}
\renewcommand{\d}{\mathop{}\!d}
\newcommand{\dd}[2][]{\frac{\d #1}{\d #2}}
\newcommand{\pp}[2][]{\frac{\partial #1}{\partial #2}}
\renewcommand{\l}{\ell}
\newcommand{\ddx}{\frac{d}{\d x}}

\newcommand{\zeroOverZero}{\ensuremath{\boldsymbol{\tfrac{0}{0}}}}
\newcommand{\inftyOverInfty}{\ensuremath{\boldsymbol{\tfrac{\infty}{\infty}}}}
\newcommand{\zeroOverInfty}{\ensuremath{\boldsymbol{\tfrac{0}{\infty}}}}
\newcommand{\zeroTimesInfty}{\ensuremath{\small\boldsymbol{0\cdot \infty}}}
\newcommand{\inftyMinusInfty}{\ensuremath{\small\boldsymbol{\infty - \infty}}}
\newcommand{\oneToInfty}{\ensuremath{\boldsymbol{1^\infty}}}
\newcommand{\zeroToZero}{\ensuremath{\boldsymbol{0^0}}}
\newcommand{\inftyToZero}{\ensuremath{\boldsymbol{\infty^0}}}



\newcommand{\numOverZero}{\ensuremath{\boldsymbol{\tfrac{\#}{0}}}}
\newcommand{\dfn}{\textbf}
%\newcommand{\unit}{\,\mathrm}
\newcommand{\unit}{\mathop{}\!\mathrm}
\newcommand{\eval}[1]{\bigg[ #1 \bigg]}
\newcommand{\seq}[1]{\left( #1 \right)}
\renewcommand{\epsilon}{\varepsilon}
\renewcommand{\phi}{\varphi}


\renewcommand{\iff}{\Leftrightarrow}

\DeclareMathOperator{\arccot}{arccot}
\DeclareMathOperator{\arcsec}{arcsec}
\DeclareMathOperator{\arccsc}{arccsc}
\DeclareMathOperator{\si}{Si}
\DeclareMathOperator{\scal}{scal}
\DeclareMathOperator{\sign}{sign}


%% \newcommand{\tightoverset}[2]{% for arrow vec
%%   \mathop{#2}\limits^{\vbox to -.5ex{\kern-0.75ex\hbox{$#1$}\vss}}}
\newcommand{\arrowvec}[1]{{\overset{\rightharpoonup}{#1}}}
%\renewcommand{\vec}[1]{\arrowvec{\mathbf{#1}}}
\renewcommand{\vec}[1]{{\overset{\boldsymbol{\rightharpoonup}}{\mathbf{#1}}}\hspace{0in}}

\newcommand{\point}[1]{\left(#1\right)} %this allows \vector{ to be changed to \vector{ with a quick find and replace
\newcommand{\pt}[1]{\mathbf{#1}} %this allows \vec{ to be changed to \vec{ with a quick find and replace
\newcommand{\Lim}[2]{\lim_{\point{#1} \to \point{#2}}} %Bart, I changed this to point since I want to use it.  It runs through both of the exercise and exerciseE files in limits section, which is why it was in each document to start with.

\DeclareMathOperator{\proj}{\mathbf{proj}}
\newcommand{\veci}{{\boldsymbol{\hat{\imath}}}}
\newcommand{\vecj}{{\boldsymbol{\hat{\jmath}}}}
\newcommand{\veck}{{\boldsymbol{\hat{k}}}}
\newcommand{\vecl}{\vec{\boldsymbol{\l}}}
\newcommand{\uvec}[1]{\mathbf{\hat{#1}}}
\newcommand{\utan}{\mathbf{\hat{t}}}
\newcommand{\unormal}{\mathbf{\hat{n}}}
\newcommand{\ubinormal}{\mathbf{\hat{b}}}

\newcommand{\dotp}{\bullet}
\newcommand{\cross}{\boldsymbol\times}
\newcommand{\grad}{\boldsymbol\nabla}
\newcommand{\divergence}{\grad\dotp}
\newcommand{\curl}{\grad\cross}
%\DeclareMathOperator{\divergence}{divergence}
%\DeclareMathOperator{\curl}[1]{\grad\cross #1}
\newcommand{\lto}{\mathop{\longrightarrow\,}\limits}

\renewcommand{\bar}{\overline}

\colorlet{textColor}{black}
\colorlet{background}{white}
\colorlet{penColor}{blue!50!black} % Color of a curve in a plot
\colorlet{penColor2}{red!50!black}% Color of a curve in a plot
\colorlet{penColor3}{red!50!blue} % Color of a curve in a plot
\colorlet{penColor4}{green!50!black} % Color of a curve in a plot
\colorlet{penColor5}{orange!80!black} % Color of a curve in a plot
\colorlet{penColor6}{yellow!70!black} % Color of a curve in a plot
\colorlet{fill1}{penColor!20} % Color of fill in a plot
\colorlet{fill2}{penColor2!20} % Color of fill in a plot
\colorlet{fillp}{fill1} % Color of positive area
\colorlet{filln}{penColor2!20} % Color of negative area
\colorlet{fill3}{penColor3!20} % Fill
\colorlet{fill4}{penColor4!20} % Fill
\colorlet{fill5}{penColor5!20} % Fill
\colorlet{gridColor}{gray!50} % Color of grid in a plot

\newcommand{\surfaceColor}{violet}
\newcommand{\surfaceColorTwo}{redyellow}
\newcommand{\sliceColor}{greenyellow}




\pgfmathdeclarefunction{gauss}{2}{% gives gaussian
  \pgfmathparse{1/(#2*sqrt(2*pi))*exp(-((x-#1)^2)/(2*#2^2))}%
}


%%%%%%%%%%%%%
%% Vectors
%%%%%%%%%%%%%

%% Simple horiz vectors
\renewcommand{\vector}[1]{\left\langle #1\right\rangle}


%% %% Complex Horiz Vectors with angle brackets
%% \makeatletter
%% \renewcommand{\vector}[2][ , ]{\left\langle%
%%   \def\nextitem{\def\nextitem{#1}}%
%%   \@for \el:=#2\do{\nextitem\el}\right\rangle%
%% }
%% \makeatother

%% %% Vertical Vectors
%% \def\vector#1{\begin{bmatrix}\vecListA#1,,\end{bmatrix}}
%% \def\vecListA#1,{\if,#1,\else #1\cr \expandafter \vecListA \fi}

%%%%%%%%%%%%%
%% End of vectors
%%%%%%%%%%%%%

%\newcommand{\fullwidth}{}
%\newcommand{\normalwidth}{}



%% makes a snazzy t-chart for evaluating functions
%\newenvironment{tchart}{\rowcolors{2}{}{background!90!textColor}\array}{\endarray}

%%This is to help with formatting on future title pages.
\newenvironment{sectionOutcomes}{}{}



%% Flowchart stuff
%\tikzstyle{startstop} = [rectangle, rounded corners, minimum width=3cm, minimum height=1cm,text centered, draw=black]
%\tikzstyle{question} = [rectangle, minimum width=3cm, minimum height=1cm, text centered, draw=black]
%\tikzstyle{decision} = [trapezium, trapezium left angle=70, trapezium right angle=110, minimum width=3cm, minimum height=1cm, text centered, draw=black]
%\tikzstyle{question} = [rectangle, rounded corners, minimum width=3cm, minimum height=1cm,text centered, draw=black]
%\tikzstyle{process} = [rectangle, minimum width=3cm, minimum height=1cm, text centered, draw=black]
%\tikzstyle{decision} = [trapezium, trapezium left angle=70, trapezium right angle=110, minimum width=3cm, minimum height=1cm, text centered, draw=black]


\outcome{Evaluate limits of functions of several variables.}
\outcome{Determine continuity of functions of several variables.}

\title[Dig-In:]{Continuity}

\begin{document}
\begin{abstract}
We investigate what continuity means for functions of several variables.
\end{abstract}
\maketitle


This section investigates what it means for functions
\[
F:\R^n\to \R
\]
to be ``continuous.'' We begin with a series of definitions. We are
used to ``open intervals'' such as $(1,3)$, which represents the set
of all $x$ such that $1<x<3$, and ``closed intervals'' such as
$[1,3]$, which represents the set of all $x$ such that $1\leq x\leq
3$. We need analogous definitions for open and closed sets in $\R^n$

\begin{definition}
  We give these definitions in general, for when one is working in
  $\R^n$:
  \begin{itemize}
  \item An \dfn{open disk} $B$ in $\R^n$ centered at $\vec{a} =
    \vector{a_1,a_2,\dots,a_n}$ with radius $r$ is the set of all
    vectors $\vec{x}=\vector{x_1,x_2,\dots,x_n}$ such that
    $|\vec{x}-\vec{a}| < r$.
  \item A point $P$ in $S$ is an \dfn{interior point} of $S$ if there
    is an open disk $B$ centered at $P$ that contains only points in
    $S$. We can write this in symbols as
    \[
    P\in B\subseteq S
    \]
  \item Let $S$ be a set of points in $\R^n$. A point $P$ in $\R^2$ is
    a \dfn{boundary point} of $S$ if all open disks centered at $P$
    contain both points in $S$ and points not in $S$.
  \item A set $S$ is \dfn{open} if every point in $S$ is an interior
    point.
  \item A set $S$ is \dfn{closed} if it contains all of its boundary
    points.
  \item A set $S$ is \dfn{bounded} if there is an open disk $B$
    centered at the origin of radius $M$ such that
    \[
    S\subseteq B.
    \]
    A set that is not bounded is \dfn{unbounded}.
  \end{itemize}
\end{definition}

%% Figure \ref{fig:multilimit_intro} shows several sets in the $x$-$y$ plane. In each set, point $P_1$ lies on the boundary of the set as all open disks centered there contain both points in, and not in, the set. In contrast, point $P_2$ is an interior point for there is an open disk centered there that lies entirely within the set.
%% \mtable{.6}{Illustrating open and closed sets in the $x$-$y$ plane.}{fig:multilimit_intro}{%
%% \begin{tabular}{c}
%% \myincludegraphics{figures/figmultilimit_introa}\\
%% (a)\\[10pt]
%% \myincludegraphics{figures/figmultilimit_introb}\\
%% (b)\\[10pt]
%% \myincludegraphics{figures/figmultilimit_introc}\\
%% (c)\\[10pt]
%% \end{tabular}
%% }

%% The set depicted in Figure \ref{fig:multilimit_intro}(a) is a closed set as it contains all of its boundary points. The set in (b) is open, for all of its points are interior points (or, equivalently, it does not contain any of its boundary points). The set in (c) is neither open nor closed as it contains just some of its boundary points.\\
%% \clearpage

\begin{example}
  Determine if the domain of the function
  $F(x,y)=\sqrt{1-\frac{x^2}9-\frac{y^2}4}$ is open, closed, or
  neither, and if it is bounded.
  \begin{explanation}
    We've already found the domain of this function to be
    \[
    D = \{(x,y): \answer[given]{x^2/9+y^2/4}\leq 1\}.
    \]
    This is the region \textit{bounded} by the ellipse
    $\frac{x^2}9+\frac{y^2}4=1$. Since the region includes the
    boundary (indicated by the use of ``$\leq$''), the set
    \wordChoice{\choice[correct]{contains}\choice{does not contain}}
    all of its boundary points and hence is closed. The region is
    \wordChoice{\choice[correct]{bounded}\choice{unbounded}} as a disk
    of radius $4$, centered at the origin, contains $D$.
  \end{explanation}
\end{example}

\begin{example}
  Determine if the domain of $F(x,y) = \frac{1}{x-y}$ is open, closed,
  or neither, and if it is bounded.
  \begin{explanation}
    As we cannot divide by $0$, we find the domain to be
    \[
    D = \{(x,y):x-y\neq \answer[given]{0}\}.
    \]
    In other words, the domain is the set of all points $(x,y)$
    \textit{not} on the line $y=x$. For your viewing pleasure, we have
    included a graph:
    \begin{image}
      \begin{tikzpicture}
        \begin{axis}[
            tick label style={font=\scriptsize},axis y line=middle,axis x line=middle,name=myplot,axis on top,%
            xtick=\empty,
            ytick=\empty,
            ymin=-1,ymax=1,%
            xmin=-1,xmax=1%
          ]
          \filldraw [fill1,fill=fill1] (axis cs:-1,-1) rectangle (axis cs: 1,1);          
          \addplot [ultra thick,white]coordinates {(-1.,-1.)(1,1)};
        \end{axis}
        \node [right] at (myplot.right of origin) {\scriptsize $x$};
        \node [above] at (myplot.above origin) {\scriptsize $y$};
      \end{tikzpicture}
    \end{image}
    Note how we can draw an open disk around any point in the domain
    that lies entirely inside the domain, and also note how the only
    boundary points of the domain are the points on the line $y=x$. We
    conclude the domain is \wordChoice{\choice[correct]{an open
        set}\choice{a closed set}\choice{neither open nor closed
        set}}. Moreover, the set is \wordChoice{\choice{bounded}\choice[correct]{unbounded}}.
  \end{explanation}
\end{example}

\section{Limits}

On to the definition of a limit!

\begin{definition}
 Suppose that $F:\R^n\to\R$, intuitively,
  \begin{center}
    the \dfn{limit} of $F$ as $\vec{x}$ approaches $\vec{a}$ is $L$,
  \end{center}
  written
  \[
  \lim_{\vec{x}\to \vec{a}} F(\vec{x}) = L,
  \]
  if the value of $F(\vec{x})$ can be made as close as one wishes to $L$ for
  all $\vec{x}$ sufficiently close, but not equal to, $\vec{a}$.
\end{definition}

\begin{question}
  Suppose that $F:\R^2\to\R$, $\vec{x} = \vector{x,y}$, and $\vec{a} =
  \vector{a,b}$. What do we write in this case for $\lim_{\vec{x}\to
    \vec{a}} F(\vec{x}) = L$?
  \begin{prompt}
    \[
    \lim_{\vector{\answer{x},\answer{y}}\to \vector{\answer{a},\answer{b}}} F(\answer{x},\answer{y}) = L
    \]
  \end{prompt}
  \begin{question}
    Suppose that $F:\R^3\to\R$, $\vec{x} = \vector{x,y,z}$, and $\vec{a} =
    \vector{a,b,c}$. What do we write in this case for $\lim_{\vec{x}\to
      \vec{a}} F(\vec{x}) = L$?
    \begin{prompt}
      \[
      \lim_{\vector{\answer{x},\answer{y},\answer{z}}\to \vector{\answer{a},\answer{b},\answer{c}}} F(\answer{x},\answer{y},\answer{z}) = L
      \]
    \end{prompt}
  \end{question}
\end{question}

\section{Continuity}

Now we will use our definition of a limit to define continuity.

\begin{definition}
  Let $F:\R^n\to \R$ be defined on an open disk $B$ centered at
  $\vec{a}$.
  \begin{itemize}
  \item $F$ is \dfn{continuous} at $\vec{a}$, if
    \[
    \lim_{\vec{x}\to\vec{a}} F(\vec{x}) = F(\vec{a}).
    \]
  \item $F$ is \dfn{continuous on an open disk} $B$ if $F$ is
    continuous at all points in $B$.
  \end{itemize}
\end{definition}

To really use this definition, we need \textit{limit laws} which in
some sense are really \textit{continuity laws}.

\begin{theorem}[Limit Laws]
  Let $F:\R^n\to \R$ and $G:\R^n\to \R$ be functions of several
  variables, $b$, $L$ and $M$ are real numbers,
  \begin{align*}
    \vec{x} &= \vector{x_1,x_2,\dots,x_n}\\ \vec{a} &=
    \vector{a_1,a_2,\dots,a_n},
  \end{align*}
  where
  \[
  \lim_{\vec{x}\to\vec{a}}F(\vec{x}) = L \quad \text{and}\quad \lim_{\vec{x}\to\vec{a}} G(\vec{x}) = M.
  \]
\begin{description}
\item[Constant Law] $\lim_{\vec{x}\to \vec{a}} b = b$.
\item[Identity Law] $\lim_{\vec{x}\to \vec{a}} x_i = a_i$.
\item[Sum/Difference Law] $\lim_{\vec{x}\to \vec{a}}\big(F(\vec{x})\pm G(\vec{x})\big) = L\pm M$.
\item[Scalar Multiple Law] $\lim_{\vec{x}\to \vec{a}} b\cdot F(\vec{x}) = bL$.
\item[Product Law] $\lim_{\vec{x}\to \vec{a}} \left(F(\vec{x})\cdot G(\vec{x})\right) = LM$.
\item[Quotient Law] $\lim_{\vec{x}\to \vec{a}} \frac{F(\vec{x})}{G(\vec{x})} = \frac{L}{M}$, if $M\neq 0$.
\end{description}
\end{theorem}

\begin{question}
  True or false: If $F:\R^2\to\R$ and $G:\R^2\to\R$ are continuous
  functions on an open disk $B$, then $F\pm G$ is continuous on $B$.
  \begin{prompt}
    \begin{multipleChoice}
      \choice[correct]{True}
      \choice{False}
  \end{multipleChoice}
  \end{prompt}
\end{question}

\begin{question}
  True or false: If $F:\R^2\to\R$ and $G:\R^2\to\R$ are continuous
  functions on an open disk $B$, then $F/G$ is continuous on $B$.
  \begin{prompt}
    \begin{multipleChoice}
      \choice{True}
      \choice[correct]{False}
    \end{multipleChoice}
    \begin{feedback}
      The function $F/G$ may or may not be continuous, it depends on
      whether $G(x,y)=0$. If $G(x,y)=0$, then $F/G$ not continuous at that point.
    \end{feedback}
  \end{prompt}
\end{question}


\begin{theorem}[Composition Limit Law]
  Let $f:\R\to\R$ be a continuous function on an interval $J$. Let
  $G:\R^n\to \R$ be a function whose range is contained in (or equal
  to) $J$, Then
  \[
  \lim_{\vec{x}\to\vec{a}} f( G(\vec{x})) = f(\lim_{\vec{x}\to\vec{a}}G(\vec{x}))
  \]
\end{theorem}


\begin{corollary}[Composition of Composite Functions]
  Let $G:\R^n\to \R$ be continuous on an open disk $B$, where the
  range of $G$ on $B$ is $J$, and let $g$ be a single variable
  function that is continuous on $J$. Then
  \[
  g\circ F(\vec{x}) =g(F(\vec{x})),
  \]
  is continuous on $B$.
\end{corollary}



\begin{example}
  Show that the function
  \[
  F(x,y) = \sin(x^2\cos(y))
  \]
  is continuous for all points in $\R^2$.
  \begin{explanation}
    Let
    \[
    F_1(x,y) = x^2.
    \]
    Since $y$ is not actually used in the function, and polynomials
    \wordChoice{\choice[correct]{are continuous},\choice{are not
        continuous}}, we conclude $F_1$ is continuous everywhere. A
    similar statement can be made about
    \[
    F_2(x,y) = \cos(y).
    \]
    Setting
    \[
    F_3=F_1\cdot F_2
    \]
    we obtain a continuous function from $\R^2\to \R$. Since sine \wordChoice{\choice[correct]{is
    continuous}\choice{is not continuous}} for all real values, the composition of sine with $F_3$
    is continuous. Hence, $\sin (F_3(x,y)) = \sin(x^2\cos y)$ is
    continuous everywhere.
    \begin{onlineOnly}
      We finish by presenting you with a plot of $F$:
\begin{sageCell}
f(x,y) = sin(x^2*cos(y))
plot3d(f(x,y),(x,-5,5),(y,-5,5))
\end{sageCell}
    \end{onlineOnly}
  \end{explanation}
\end{example}


\begin{example}
  Let
  \[
  F(x,y) = \begin{cases}
    \frac{\cos(y)\sin(x)}{x} & x\neq 0 \\
    \cos(y) & x=0
  \end{cases}
  \]
  Is $F$ continuous at $(0,0)$? Is $F$ continuous everywhere?
  \begin{explanation}
    To determine if $F$ is continuous at $(0,0)$, we need to compare
    \[
    \lim_{\vector{x,y}\to\vector{0,0}} F(x,y)\quad\text{to}\quad F(0,0).
    \]
    Applying the definition of $F$, we see that
    \[
    F(0,0) = \answer[given]{1}
    \]
    We now consider the limit
    \[
    \lim_{\vector{x,y}\to\vector{0,0}}F(x,y).
    \]
    Substituting $0$ for $x$ and $y$ in $(\cos(y)\sin(x))/x$ returns the
    indeterminate form \zeroOverZero, so we need to do more work to
    evaluate this limit.
    
    Consider two related limits:
    \begin{align*}
      \lim_{\vector{x,y}\to\vector{0,0}} \cos(y)\\
      \lim_{\vector{x,y}\to\vector{0,0}} \frac{\sin(x)}{x}.
    \end{align*}
    The first limit does not contain $x$, and since $\cos(y)$ is
    continuous,
    \begin{align*}
    \lim_{\vector{x,y}\to\vector{0,0}} \cos(y) &=\lim_{y\to 0} \cos(y) \\
    &=\answer[given]{1}
    \end{align*}
    The second limit does not contain $y$. But we know
    \begin{align*}
      \lim_{\vector{x,y}\to\vector{0,0}} \frac{\sin(x)}{x} &= \lim_{x\to 0} \frac{\sin(x)}{x} \\
      &= \answer[given]{1}.
    \end{align*}
    Finally, we know that we can combine these two limits so that 
    $\lim_{\vector{x,y}\to\vector{0,0}} \frac{\cos(y)\sin(x)}{x}$
    \begin{align*}
      &= \lim_{\vector{x,y}\to\vector{0,0}} (\cos(y))\left(\frac{\sin(x)}{x}\right) \\ 
      &=\left(\lim_{\vector{x,y}\to\vector{0,0}} \cos(y)\right)\left(\lim_{\vector{x,y}\to\vector{0,0}} \frac{\sin(x)}{x}\right) \\
            &=\answer[given]{1}\cdot \answer[given]{1}.
    \end{align*}
    We have found that $\lim_{\vector{x,y}\to\vector{0,0}} \frac{\cos(y)\sin(x)}{x} =
    F(0,0)$, so $F$ is continuous at $(0,0)$.

    A similar analysis shows that $F$ is continuous at all points in
    $\mathbb{R}^2$. As long as $x\neq0$, we can evaluate the limit
    directly; when $x=0$, a similar analysis shows that the limit is $\cos
    y$. Thus we can say that $F$ is continuous everywhere.
    \begin{onlineOnly}
      We finish by presenting you with a plot of $F$:
\begin{sageCell}
f(x,y) = cos(y)*sin(x)/x
plot3d(f(x,y),(x,-5,5),(y,-5,5))
\end{sageCell}
    \end{onlineOnly}
  \end{explanation}
\end{example}


\subsection{When limits don't exist}

When dealing with functions of a single variable we often considered
one-sided limits and stated
\[
\lim_{x\to a}f(x) = L
\]
if and only if
\[
\lim_{x\to a^+}f(x) =L \quad\textbf{and}\quad \lim_{x\to a^-}f(x) =L.
\]
That is, the limit is $L$ if and only if $F$ approaches $L$ when
$x$ approaches $a$ from \textbf{either} direction.

In $\R^n$ when $n>2$ there are \textbf{infinite paths} from which
$\vec{x}$ might approach $\vec{a}$. Now we have a fact, 
\[
\lim_{\vec{x}\to \vec{a}}F(\vec{x}) = L
\]
if and only if
\[
F(\vec{x})\to  L \quad{as}\quad \vec{x}\to \vec{a}
\]
along every path.
\begin{quote}
  \textbf{If it is possible to arrive at different limiting values by
    approaching along different paths, the limit does not exist.}
\end{quote}
This is analogous to the left and right hand limits of single variable
functions not being equal, implying that the limit does not exist.
Our theorems tell us that we can evaluate most limits quite simply,
without worrying about paths. When indeterminate forms arise, the
limit may or may not exist. If it does exist, it can be difficult to
prove this as we need to show the same limiting value is obtained
regardless of the path chosen.  The case where the limit does not
exist is often easier to deal with, for we can often pick two paths
along which the limit is different.


\begin{example}
  Show
  \[
  \lim_{\vector{x,y}\to\vector{0,0}} \frac{3xy}{x^2+y^2}
  \]
  does not exist by finding the limits along the lines $y=mx$.
  \begin{explanation}
    Evaluating $\lim_{\vector{x,y}\to\vector{0,0}} \frac{3xy}{x^2+y^2}$ along
    the lines $y=mx$ means replace all $y$'s with $mx$ and evaluating
    the resulting limit:
    \begin{align*}
      \lim_{\vector{x,mx}\to\vector{0,0}} \frac{3x(\answer[given]{m x})}{x^2+(\answer[given]{m x})^2} &=\lim_{x\to0} \frac{3mx^2}{x^2(m^2+1)}\\
      &= \lim_{x\to 0}\frac{3m}{m^2+1}\\
      &= \answer[given]{\frac{3m}{m^2+1}}.
    \end{align*}
    While the limit exists for each choice of $m$, we get a
    \textit{different} limit for each choice of $m$. That is, along
    different lines we get differing limiting values, meaning
    \textit{the} limit does not exist.
    \begin{onlineOnly}
      We finish by presenting you with a plot of $F$:
\begin{sageCell}
f(x,y) = 3*x*y/(x^2+y^2)
plot3d(f(x,y),(x,-5,5),(y,-5,5))
\end{sageCell}
    \end{onlineOnly}
  \end{explanation}
\end{example}

\begin{example}
  Show
  \[
  \lim_{\vector{x,y}\to\vector{0,0}} \frac{\sin(xy)}{x+y}
  \]
  does not exist by finding the limit along the path $y=-\sin(x)$.
  \begin{explanation}
    Let
    \[
    F(x,y) = \frac{\sin(xy)}{x+y}.
    \]
    We will show that
    \[
    \lim_{\vector{x,y}\to\vector{0,0}} F(x,y)
    \]
    does not exist by finding the limit along the path
    $y=-\sin(x)$. First, however, let's try the same trick that worked
    before and see what happens. Consider the limits found along the
    lines $y=mx$ as done above.
    \begin{align*}
      \lim_{\vector{x,mx}\to\vector{0,0}} \frac{\sin\big(x(\answer[given]{mx})\big)}{x+\answer[given]{mx}} &= \lim_{x\to 0} \frac{\sin (mx^2)}{x(m+1)} \\
      &= \lim_{x\to 0} \frac{\sin(mx^2)}{x}\cdot\answer[given]{\frac{1}{m+1}}.
    \end{align*}
    By applying L'H\^opital's Rule, we can show this limit is $0$
    \emph{except} when $m=-1$, that is, along the line $y=-x$. This
    line is not in the domain of $F$, so we have found the following
    fact: along every line $y=mx$ in the domain of $F$,
    \[
    \lim_{\vector{x,y}\to\vector{0,0}} F(x,y)=0.
    \]
    Now consider the limit along the path $y=-\sin(x)$:
    \begin{align*}
      \lim_{\vector{x,-\sin(x)}\to\vector{0,0}} \frac{\sin\big(-x\sin(x)\big)}{x-\sin(x)} &= \lim_{x\to0} \frac{\sin\big(-x\sin(x)\big)}{x-\sin(x)}
    \end{align*}
    Now apply L'H\^opital's Rule twice to find a limit is of the form
    \numOverZero.  Hence the limit does not exist.  Step back and
    consider what we have just discovered.
    \begin{itemize}
    \item Along any line $y=mx$ in the domain of the $F(x,y)$, the
      limit is $0$.
    \item However, along the path $y=-\sin(x)$, which lies in the
      domain of the $F(x,y)$ for all $x\neq 0$, the limit does not
      exist.
    \end{itemize}
    Since the limit is not the same along every path to $(0,0)$, we say
    $\lim_{\vector{x,y}\to\vector{0,0}}\frac{\sin(xy)}{x+y}$ does not exist.
        \begin{onlineOnly}
          We finish by presenting you with a plot of $F$:
\begin{sageCell}
f(x,y) = sin(x*y)/(x+y)
plot3d(f(x,y),(x,-5,5),(y,-5,5))
\end{sageCell}
    \end{onlineOnly}
  \end{explanation}
\end{example}



\end{document}
