\documentclass{ximera}

%\usepackage{todonotes}
%\usepackage{mathtools} %% Required for wide table Curl and Greens
%\usepackage{cuted} %% Required for wide table Curl and Greens
\newcommand{\todo}{}

\usepackage{esint} % for \oiint
\ifxake%%https://math.meta.stackexchange.com/questions/9973/how-do-you-render-a-closed-surface-double-integral
\renewcommand{\oiint}{{\large\bigcirc}\kern-1.56em\iint}
\fi


\graphicspath{
  {./}
  {ximeraTutorial/}
  {basicPhilosophy/}
  {functionsOfSeveralVariables/}
  {normalVectors/}
  {lagrangeMultipliers/}
  {vectorFields/}
  {greensTheorem/}
  {shapeOfThingsToCome/}
  {dotProducts/}
  {partialDerivativesAndTheGradientVector/}
  {../productAndQuotientRules/exercises/}
  {../normalVectors/exercisesParametricPlots/}
  {../continuityOfFunctionsOfSeveralVariables/exercises/}
  {../partialDerivativesAndTheGradientVector/exercises/}
  {../directionalDerivativeAndChainRule/exercises/}
  {../commonCoordinates/exercisesCylindricalCoordinates/}
  {../commonCoordinates/exercisesSphericalCoordinates/}
  {../greensTheorem/exercisesCurlAndLineIntegrals/}
  {../greensTheorem/exercisesDivergenceAndLineIntegrals/}
  {../shapeOfThingsToCome/exercisesDivergenceTheorem/}
  {../greensTheorem/}
  {../shapeOfThingsToCome/}
  {../separableDifferentialEquations/exercises/}
  {vectorFields/}
}

\newcommand{\mooculus}{\textsf{\textbf{MOOC}\textnormal{\textsf{ULUS}}}}

\usepackage{tkz-euclide}\usepackage{tikz}
\usepackage{tikz-cd}
\usetikzlibrary{arrows}
\tikzset{>=stealth,commutative diagrams/.cd,
  arrow style=tikz,diagrams={>=stealth}} %% cool arrow head
\tikzset{shorten <>/.style={ shorten >=#1, shorten <=#1 } } %% allows shorter vectors

\usetikzlibrary{backgrounds} %% for boxes around graphs
\usetikzlibrary{shapes,positioning}  %% Clouds and stars
\usetikzlibrary{matrix} %% for matrix
\usepgfplotslibrary{polar} %% for polar plots
\usepgfplotslibrary{fillbetween} %% to shade area between curves in TikZ
\usetkzobj{all}
\usepackage[makeroom]{cancel} %% for strike outs
%\usepackage{mathtools} %% for pretty underbrace % Breaks Ximera
%\usepackage{multicol}
\usepackage{pgffor} %% required for integral for loops



%% http://tex.stackexchange.com/questions/66490/drawing-a-tikz-arc-specifying-the-center
%% Draws beach ball
\tikzset{pics/carc/.style args={#1:#2:#3}{code={\draw[pic actions] (#1:#3) arc(#1:#2:#3);}}}



\usepackage{array}
\setlength{\extrarowheight}{+.1cm}
\newdimen\digitwidth
\settowidth\digitwidth{9}
\def\divrule#1#2{
\noalign{\moveright#1\digitwidth
\vbox{\hrule width#2\digitwidth}}}





\newcommand{\RR}{\mathbb R}
\newcommand{\R}{\mathbb R}
\newcommand{\N}{\mathbb N}
\newcommand{\Z}{\mathbb Z}

\newcommand{\sagemath}{\textsf{SageMath}}


%\renewcommand{\d}{\,d\!}
\renewcommand{\d}{\mathop{}\!d}
\newcommand{\dd}[2][]{\frac{\d #1}{\d #2}}
\newcommand{\pp}[2][]{\frac{\partial #1}{\partial #2}}
\renewcommand{\l}{\ell}
\newcommand{\ddx}{\frac{d}{\d x}}

\newcommand{\zeroOverZero}{\ensuremath{\boldsymbol{\tfrac{0}{0}}}}
\newcommand{\inftyOverInfty}{\ensuremath{\boldsymbol{\tfrac{\infty}{\infty}}}}
\newcommand{\zeroOverInfty}{\ensuremath{\boldsymbol{\tfrac{0}{\infty}}}}
\newcommand{\zeroTimesInfty}{\ensuremath{\small\boldsymbol{0\cdot \infty}}}
\newcommand{\inftyMinusInfty}{\ensuremath{\small\boldsymbol{\infty - \infty}}}
\newcommand{\oneToInfty}{\ensuremath{\boldsymbol{1^\infty}}}
\newcommand{\zeroToZero}{\ensuremath{\boldsymbol{0^0}}}
\newcommand{\inftyToZero}{\ensuremath{\boldsymbol{\infty^0}}}



\newcommand{\numOverZero}{\ensuremath{\boldsymbol{\tfrac{\#}{0}}}}
\newcommand{\dfn}{\textbf}
%\newcommand{\unit}{\,\mathrm}
\newcommand{\unit}{\mathop{}\!\mathrm}
\newcommand{\eval}[1]{\bigg[ #1 \bigg]}
\newcommand{\seq}[1]{\left( #1 \right)}
\renewcommand{\epsilon}{\varepsilon}
\renewcommand{\phi}{\varphi}


\renewcommand{\iff}{\Leftrightarrow}

\DeclareMathOperator{\arccot}{arccot}
\DeclareMathOperator{\arcsec}{arcsec}
\DeclareMathOperator{\arccsc}{arccsc}
\DeclareMathOperator{\si}{Si}
\DeclareMathOperator{\scal}{scal}
\DeclareMathOperator{\sign}{sign}


%% \newcommand{\tightoverset}[2]{% for arrow vec
%%   \mathop{#2}\limits^{\vbox to -.5ex{\kern-0.75ex\hbox{$#1$}\vss}}}
\newcommand{\arrowvec}[1]{{\overset{\rightharpoonup}{#1}}}
%\renewcommand{\vec}[1]{\arrowvec{\mathbf{#1}}}
\renewcommand{\vec}[1]{{\overset{\boldsymbol{\rightharpoonup}}{\mathbf{#1}}}\hspace{0in}}

\newcommand{\point}[1]{\left(#1\right)} %this allows \vector{ to be changed to \vector{ with a quick find and replace
\newcommand{\pt}[1]{\mathbf{#1}} %this allows \vec{ to be changed to \vec{ with a quick find and replace
\newcommand{\Lim}[2]{\lim_{\point{#1} \to \point{#2}}} %Bart, I changed this to point since I want to use it.  It runs through both of the exercise and exerciseE files in limits section, which is why it was in each document to start with.

\DeclareMathOperator{\proj}{\mathbf{proj}}
\newcommand{\veci}{{\boldsymbol{\hat{\imath}}}}
\newcommand{\vecj}{{\boldsymbol{\hat{\jmath}}}}
\newcommand{\veck}{{\boldsymbol{\hat{k}}}}
\newcommand{\vecl}{\vec{\boldsymbol{\l}}}
\newcommand{\uvec}[1]{\mathbf{\hat{#1}}}
\newcommand{\utan}{\mathbf{\hat{t}}}
\newcommand{\unormal}{\mathbf{\hat{n}}}
\newcommand{\ubinormal}{\mathbf{\hat{b}}}

\newcommand{\dotp}{\bullet}
\newcommand{\cross}{\boldsymbol\times}
\newcommand{\grad}{\boldsymbol\nabla}
\newcommand{\divergence}{\grad\dotp}
\newcommand{\curl}{\grad\cross}
%\DeclareMathOperator{\divergence}{divergence}
%\DeclareMathOperator{\curl}[1]{\grad\cross #1}
\newcommand{\lto}{\mathop{\longrightarrow\,}\limits}

\renewcommand{\bar}{\overline}

\colorlet{textColor}{black}
\colorlet{background}{white}
\colorlet{penColor}{blue!50!black} % Color of a curve in a plot
\colorlet{penColor2}{red!50!black}% Color of a curve in a plot
\colorlet{penColor3}{red!50!blue} % Color of a curve in a plot
\colorlet{penColor4}{green!50!black} % Color of a curve in a plot
\colorlet{penColor5}{orange!80!black} % Color of a curve in a plot
\colorlet{penColor6}{yellow!70!black} % Color of a curve in a plot
\colorlet{fill1}{penColor!20} % Color of fill in a plot
\colorlet{fill2}{penColor2!20} % Color of fill in a plot
\colorlet{fillp}{fill1} % Color of positive area
\colorlet{filln}{penColor2!20} % Color of negative area
\colorlet{fill3}{penColor3!20} % Fill
\colorlet{fill4}{penColor4!20} % Fill
\colorlet{fill5}{penColor5!20} % Fill
\colorlet{gridColor}{gray!50} % Color of grid in a plot

\newcommand{\surfaceColor}{violet}
\newcommand{\surfaceColorTwo}{redyellow}
\newcommand{\sliceColor}{greenyellow}




\pgfmathdeclarefunction{gauss}{2}{% gives gaussian
  \pgfmathparse{1/(#2*sqrt(2*pi))*exp(-((x-#1)^2)/(2*#2^2))}%
}


%%%%%%%%%%%%%
%% Vectors
%%%%%%%%%%%%%

%% Simple horiz vectors
\renewcommand{\vector}[1]{\left\langle #1\right\rangle}


%% %% Complex Horiz Vectors with angle brackets
%% \makeatletter
%% \renewcommand{\vector}[2][ , ]{\left\langle%
%%   \def\nextitem{\def\nextitem{#1}}%
%%   \@for \el:=#2\do{\nextitem\el}\right\rangle%
%% }
%% \makeatother

%% %% Vertical Vectors
%% \def\vector#1{\begin{bmatrix}\vecListA#1,,\end{bmatrix}}
%% \def\vecListA#1,{\if,#1,\else #1\cr \expandafter \vecListA \fi}

%%%%%%%%%%%%%
%% End of vectors
%%%%%%%%%%%%%

%\newcommand{\fullwidth}{}
%\newcommand{\normalwidth}{}



%% makes a snazzy t-chart for evaluating functions
%\newenvironment{tchart}{\rowcolors{2}{}{background!90!textColor}\array}{\endarray}

%%This is to help with formatting on future title pages.
\newenvironment{sectionOutcomes}{}{}



%% Flowchart stuff
%\tikzstyle{startstop} = [rectangle, rounded corners, minimum width=3cm, minimum height=1cm,text centered, draw=black]
%\tikzstyle{question} = [rectangle, minimum width=3cm, minimum height=1cm, text centered, draw=black]
%\tikzstyle{decision} = [trapezium, trapezium left angle=70, trapezium right angle=110, minimum width=3cm, minimum height=1cm, text centered, draw=black]
%\tikzstyle{question} = [rectangle, rounded corners, minimum width=3cm, minimum height=1cm,text centered, draw=black]
%\tikzstyle{process} = [rectangle, minimum width=3cm, minimum height=1cm, text centered, draw=black]
%\tikzstyle{decision} = [trapezium, trapezium left angle=70, trapezium right angle=110, minimum width=3cm, minimum height=1cm, text centered, draw=black]


\outcome{Write equations representing lines in space.}
\outcome{Answer questions about lines and curves in space.}

\title[Dig-In:]{Lines and curves in space}

\begin{document}
\begin{abstract}
  Vector-valued functions are parameterized curves.
\end{abstract}
\maketitle

\section{Vector-valued functions}

A function $\vec{f}: \R \to \R^3$ can be thought of as associating to
each time $t$ a vector $\vector{x(t),y(t),z(t)}$.
\begin{definition}\index{vector!valued function}
  A \dfn{vector-valued function} maps real numbers to vectors in $\R^n$.
\end{definition}
Vector-valued functions simply map numbers to lists of numbers, that we interpret as vectors:
\begin{image}[2in]
  \begin{tikzpicture}
    \node at (0,0) {$\vec{f}(t) = \underbrace{\vector{x(t),y(t),z(t)}}_{\text{vector}}$};
    \draw[ultra thick,->,gray] (.5,1) -- (0,.5);
    \draw[ultra thick,->,gray] (.5,1) -- (.5,.5);
    \draw[ultra thick,->,gray] (.5,1) -- (1,.5);
    \node at (.5,1.2) {\scriptsize numbers};
\end{tikzpicture}
\end{image}
Placing the tail of the vector at the origin, its head will sweep out
a curve parameterized by $t$. Below we see a plot of the vector-valued function:
\[
\vec{f}(t) = \vector{t,sin(t),cos(t)}
\]
\begin{image}
    \begin{tikzpicture}
      \begin{axis}%
        [tick label style={font=\scriptsize},axis on top,
	  axis lines=center,
	  view={155}{10},no markers,
	  ymin=-1.1,ymax=1.1,
	  xmin=-7,xmax=7,
	  zmin=-1.1, zmax=1.1,
	  every axis x label/.style={at={(axis cs:\pgfkeysvalueof{/pgfplots/xmax},0,0)},xshift=-3pt,yshift=-3pt},
	  xlabel={\scriptsize $x$},
	  every axis y label/.style={at={(axis cs:0,\pgfkeysvalueof{/pgfplots/ymax},0)},xshift=5pt,yshift=-2pt},
	  ylabel={\scriptsize $y$},
	  every axis z label/.style={at={(axis cs:0,0,\pgfkeysvalueof{/pgfplots/zmax})},xshift=0pt,yshift=4pt},
	  zlabel={\scriptsize $z$}
	]
        \draw[thick,->,penColor!50!white] (axis cs: 0,0,0)--(axis cs: {-4},{sin(deg(-4))},{cos(deg(-4))});
        \draw[thick,->,penColor!50!white] (axis cs: 0,0,0)--(axis cs: {-3},{sin(deg(-3))},{cos(deg(-3))});
        \draw[thick,->,penColor!50!white] (axis cs: 0,0,0)--(axis cs: {-2},{sin(deg(-2))},{cos(deg(-2))});
        \draw[thick,->,penColor!50!white] (axis cs: 0,0,0)--(axis cs: {-1},{sin(deg(-1))},{cos(deg(-1))});
        \draw[thick,->,penColor!50!white] (axis cs: 0,0,0)--(axis cs: {0},{sin(deg(0))},{cos(deg(0))});
        \draw[thick,->,penColor!50!white] (axis cs: 0,0,0)--(axis cs: {1},{sin(deg(1))},{cos(deg(1))});
        \draw[thick,->,penColor!50!white] (axis cs: 0,0,0)--(axis cs: {2},{sin(deg(2))},{cos(deg(2))});
        \draw[thick,->,penColor!50!white] (axis cs: 0,0,0)--(axis cs: {3},{sin(deg(3))},{cos(deg(3))});
        \draw[thick,->,penColor!50!white] (axis cs: 0,0,0)--(axis cs: {4},{sin(deg(4))},{cos(deg(4))});
        \addplot3+[very thick, penColor, smooth,samples=200,samples y=0,domain=-12:12] ({x},{sin(deg(x))},{cos(deg(x)});
      \end{axis}
    \end{tikzpicture}
\end{image}
\begin{onlineOnly}
  Use the slider to see how the vector-valued function is ``drawn'' by the tip of the vector
  \begin{center}
    \geogebra{FEHgHN2t}{800}{600}%% https://www.geogebra.org/m/FEHgHN2t
  \end{center}
\end{onlineOnly}
\begin{example}
  Consider the function $\vec{f}(t) = \vector{\cos(t),\sin(t),t}$.  The
  projection of the point $\vec{f}(t)$ into the $(x,y)$-plane moves around the
  unit circle in the positive direction.  The projection onto the $z$
  axis moves at a constant rate in the positive direction.  So we
  expect that $\vec{f}$ parameterizes
  \begin{multipleChoice}
    \choice{a straight line}
    \choice{a circle around the $x$-axis}
    \choice{a circle around the $y$-axis}
    \choice{a circle around the $z$-axis}
    \choice{a spiral around the $x$-axis}
    \choice{a spiral around the $y$-axis}
    \choice[correct]{a spiral around the $z$-axis}
  \end{multipleChoice}
  \begin{feedback}
    Here is the graph of $\vec{f}$:
    \begin{image}
    \begin{tikzpicture}
      \begin{axis}%
        [tick label style={font=\scriptsize},axis on top,
	  axis lines=center,
	  view={155}{10},no markers,
	  ymin=-1.1,ymax=1.1,
	  xmin=-1.1,xmax=1.1,
	  zmin=-10, zmax=10,
	  every axis x label/.style={at={(axis cs:\pgfkeysvalueof{/pgfplots/xmax},0,0)},xshift=-3pt,yshift=-3pt},
	  xlabel={\scriptsize $x$},
	  every axis y label/.style={at={(axis cs:0,\pgfkeysvalueof{/pgfplots/ymax},0)},xshift=5pt,yshift=-2pt},
	  ylabel={\scriptsize $y$},
	  every axis z label/.style={at={(axis cs:0,0,\pgfkeysvalueof{/pgfplots/zmax})},xshift=0pt,yshift=4pt},
	  zlabel={\scriptsize $z$}
	]
        \addplot3+[very thick, penColor, smooth,samples=200,samples y=0,domain=-12:12] ({cos(deg(x))},{sin(deg(x))},{x});
      \end{axis}
    \end{tikzpicture}
    \end{image}
  \end{feedback}
\end{example}







\subsection{How are vector-valued functions useful?}

To get your imagination going, here are a few examples of what a
function $f: \R \to \R^3$ could represent:

\begin{itemize}
\item The $3$-dimensional position of a rocket in space as a function
  of time.
\item The population of $3$ different species of bacteria found in a
  swimming pool as a function of the amount of chlorine in the
  water.
\item The performance of $3$ different stocks as a function of time.
\item The trunk width, height, and canopy radius of a tree as a function of time.
\item The average temperature, humidity, and air pressure at a given latitude as a function of that latitude.
\item The RGB color of a single pixel of a LCD screen varying over time. 
\end{itemize}

Of the examples above, perhaps ``position in space'' is the best
mental model to use to help you understand vector-valued functions.

\section{Lines in space}

It is easy to create a vector-valued function that passes through two
points $\vec{p}$ and $\vec{q}$:
\[
\vecl(t) = \vec{p} + t(\vec{q}-\vec{p}).
\]
\begin{question}
  What is the value of $\vecl(0)$?
  \begin{prompt}
  \begin{multipleChoice}
    \choice{$\vecl(0)$ is unknowable}
    \choice[correct]{$\vecl(0)=\vec{p}$}
    \choice{$\vecl(0)=\vec{q}$}
    \choice{$\vecl(0)=\vec{q}-\vec{p}$}
  \end{multipleChoice}
  \end{prompt}
  \begin{question}
    What is the value of $\vecl(1)$?
    \begin{prompt}
    \begin{multipleChoice}
      \choice{$\vecl(1)$ is unknowable}
      \choice{$\vecl(1)=\vec{p}$}
      \choice[correct]{$\vecl(1)=\vec{q}$}
      \choice{$\vecl(1)=\vec{q}-\vec{p}$}
    \end{multipleChoice}
    \end{prompt}
    \begin{question}
      What value of $t$ gives the midpoint of the tips of vectors $\vec{p}$ and $\vec{q}$?
      \begin{prompt}
        \[
        t = \answer{1/2}
        \]
      \end{prompt}
    \end{question}
  \end{question}
\end{question}

\begin{onlineOnly}
  Here we see vectors $\vec{p}$ and $\vec{q}$. In blue below we see
  the vector $t(\vec{p}-\vec{q})$ starting at point $\vec{p}$. Convince yourself that
  \[
  \vecl(t) = \vec{p} + t(\vec{p}-\vec{q}) 
  \]
  draws a line.
  \begin{center}
    \geogebra{DMzaWJ7Q}{800}{600}%% https://www.geogebra.org/m/DMzaWJ7Q
  \end{center}
\end{onlineOnly}
If we know that a line passes through two points (that we'll notate
with vectors) $\vec{p}$ and $\vec{q}$, then we know that it points in
the direction $\vec{v} = \vec{q} - \vec{p}$, and passes through the
tip of $\vec{p}$. When we will know that the line passes through the
tip of $\vec{p}$, and points in a direction $\vec{v}$.  Then we write
\[
\vecl(t) = \vec{p}+t\vec{v}.
\]
\begin{onlineOnly}
  Play around with the interactive below to see if you get the idea:
  \begin{center}
    \geogebra{ZwZU2tHj}{800}{600}
  \end{center}
\end{onlineOnly}

\begin{question}
  Using the ideas above, find an expression in terms of $t$
  parameterizing the line passing through $\vec{p} = \vector{0,2,4}$
  when $t=0$, and $\vec{q} = \vector{1,1,1}$ when $t=1$.
  \begin{prompt}
  \[
  \vecl(t) = \vector{\answer{t},\answer{2-t},\answer{4-3t}}
  \]
  \end{prompt}
  \begin{hint}
    The line passes through $\vec{p}$ and points in the direction
    \[
    \vec{q} - \vec{p} = \vector{1,1,1} - \vector{0,2,4}
    \]
  \end{hint}
\end{question}


\begin{question}
  Let $\vecl$ be a line that passes through the points $\vecl(0) =
  \vector{1,2,3}$ and $\vecl(1) = \vector{2,2,2}$. What are the
  components of $\vecl(t)$?
  \begin{prompt}
  \begin{align*}
    x(t) &= \answer{1+t}\\
    y(t) &= \answer{2}\\
    z(t) &= \answer{3-t}
  \end{align*}
  \end{prompt}
  \begin{hint}
    Let $\vecl(t) = \vec{p}+t\vec{v}$, then $\vecl(0) = \vec{p}$. 
  \end{hint}
  \begin{hint}
    \[
    \vec{v}= \vector{2,2,2}-\vector{1,2,3}
    \]
  \end{hint}
\end{question}

There are an infinite number of ways to parameterize the same
line. Try your hand at the following puzzlers:

\begin{question}
  Compare and contrast the curves $\vec{f}(t) =
  \vector{-3+t,5+2t,1+3t}$ and $\vec{g}(t)=\vector{-3+2t,5+4t,1+6t}$.
  \begin{prompt}
     \begin{multipleChoice}
       \choice{They parameterize different lines.}
       \choice{They parameterize the same line, but $\vec{f}(t)$ moves ``twice as fast'' as $\vec{g}(t)$.}
       \choice[correct]{They parameterize the same line, but $\vec{g}(t)$ moves ``twice as fast'' as $\vec{f}(t)$.}
       \choice{These are the same function!}
     \end{multipleChoice}
  \end{prompt}
  \begin{hint}
    Note, both lines start at the same point when $t=0$.
  \end{hint}
  \begin{hint}
    We can rewrite $\vec{f}$ and $\vec{g}$ as:
    \begin{align*}
      \vec{f}(t) &= \vector{-3,5,1}+t\vector{1,2,3}\\
      \vec{g}(t) &= \vector{-3,5,1}+t\vector{2,4,6}
    \end{align*}
  \end{hint}
  \begin{hint}
    We can further rewrite $\vec{g}$ as:
    \[
    \vec{g}(t) = \vector{-3,5,1}+2t\vector{1,2,3}
    \]
  \end{hint}
\end{question}

\begin{question}
  Compare and contrast the curves $\vec{f}(t) =
  \vector{-3+t,5+2t,1+3t}$ and $\vec{g}(t)=\vector{-3-t,5-2t,1-3t}$.
  \begin{prompt}
    \begin{multipleChoice}
      \choice{They parameterize different lines.}
      \choice[correct]{They parameterize the same line, but $\vec{f}(t)$ moves in the opposite direction compared with $\vec{g}(t)$.}
      \choice{They parameterize the same line, but $\vec{g}(t)$ moves ``twice as fast'' as $\vec{f}(t)$.}
      \choice{These are the same function!}
    \end{multipleChoice}
  \end{prompt}
  \begin{hint}
    Note both lines start at the same point when $t=0$. 
  \end{hint}
  \begin{hint}
    We can rewrite $\vec{f}$ and $\vec{g}$ as:
    \begin{align*}
      \vec{f}(t) &= \vector{-3,5,1}+t\vector{1,2,3}\\
      \vec{g}(t) &= \vector{-3,5,1}+t\vector{-1,-2,-3}
    \end{align*}
  \end{hint}
  \begin{hint}
    We can further rewrite $\vec{g}$ as:
    \[
    \vec{g}(t) = \vector{-3,5,1}-t\vector{1,2,3}
    \]
  \end{hint}
\end{question}

We can use these ideas to parameterize any line in space. However, our
parameterizations will not be unique as there are infinitely many
different ways to parameterize the same line.  Some parameterizations
may ``move faster'' than others, or in the opposite direction, or even
at uneven rates!


\subsection{Distance between a point and a line}

Given a point $\vec{p}$, notated as the tip of a vector with its tail
at the origin, and a line
\[
\vecl(t) = \vec{q} + t\vec{v}
\]
we often want to know the distance between $\vec{p}$ and $\vecl$.
\begin{image}
  \begin{tikzpicture}
    \begin{axis}[
        xmin=0,xmax=6,ymin=0,ymax=3,
        axis lines=none,
        unit vector ratio*=1 1 1,
      ]
      \addplot[dashed] plot coordinates {(4,1) (3.59,2.65)};
      \addplot[] plot coordinates {(3.39,2.6) (3.44,2.4) (3.64,2.45)};
      \addplot[very thick,penColor!50!white] {x/4+7/4};
      \addplot[very thick,penColor,->] plot coordinates {(1,2) (3,2.5)};
      \addplot[color=penColor,fill=penColor,only marks,mark=*] coordinates{(1,2)};  %% closed hole
      \addplot[color=penColor2,fill=penColor2,only marks,mark=*] coordinates{(4,1)};  %% closed hole

      \node[above] at (axis cs:2,2.25) [penColor] {$\vec{v}$};
      \node[above] at (axis cs:1,2) [penColor] {$\vec{q}$};
      \node[below] at (axis cs:4,1) [penColor2] {$\vec{p}$};
      \node[above] at (axis cs:.4,1.84) [penColor] {$\vecl$};
    \end{axis}
  \end{tikzpicture}
\end{image}
This distance is the length of the shortest path from $\vec{p}$ to the
line $\vecl$. How do we find this distance? Well:
\begin{enumerate}
  \item Recalling that the magnitude of a vector $|\vec{w}|
    =\sqrt{\vec{w}\dotp\vec{w}}$ we could attempt to minimize the
    function
    \[
    \mathrm{distance}(t)^2= \vecl(t) \dotp \vec{p}
    \]
    using the derivative. The square-root of the minimum value will be
    the distance.
  \item We could compute the distance between
    $\proj_{\vec{v}}(\vec{p}-\vec{q})$ and $\vec{p}-\vec{q}$. This is:
    \[
    |\proj_{\vec{v}}(\vec{p}-\vec{q})- (\vec{p}-\vec{q})|
    \]
    Checkout the diagram below:
\end{enumerate}
\begin{image}
  \begin{tikzpicture}
    \begin{axis}[
        xmin=0,xmax=6,ymin=0,ymax=3,domain=0:6,
        axis lines=none,
        unit vector ratio*=1 1 1,clip=false,
      ]
      \addplot[dashed] plot coordinates {(4,1) (3.59,2.65)};
      \addplot[decoration={brace,mirror,raise=.1cm},decorate,thin] plot coordinates {(4,1) (3.59,2.65)};
      \addplot[] plot coordinates {(3.39,2.6) (3.44,2.4) (3.64,2.45)};
      \draw[] (axis cs: 1.3,1.9) arc (-18.5:14:.4cm);
              
      
      \addplot[very thick,penColor!50!white] {x/4+7/4};\
      \addplot[very thick,penColor,->] plot coordinates {(1,2) (3.59,2.65)};
      \addplot[very thick,penColor2,->] plot coordinates {(1,2) (4,1)};

      \node[above] at (axis cs:2,2.4) [penColor] {$\proj_{\vec{v}}(\vec{p}-\vec{q})$};
      \node[below left] at (axis cs:2.5,1.5) [penColor2] {$\vec{p}-\vec{q}$};
      \node[right] at (axis cs:4,1.83) {$|\proj_{\vec{v}}(\vec{p}-\vec{q})- (\vec{p}-\vec{q})|$};
    \end{axis}
  \end{tikzpicture}
\end{image}
However, both of these methods are somewhat involved. Perhaps the
\textit{quickest} method for determining the distance between a point
and a line is by using the cross product. Since the cross product is
only defined in $\R^3$, we need $3$-dimensional vectors. If we
consider the vector $\vec{p}-\vec{q}$, we see by the definition of
sine
\begin{image}
  \begin{tikzpicture}
    \begin{axis}[
        xmin=0,xmax=6,ymin=0,ymax=3,
        axis lines=none,
        unit vector ratio*=1 1 1,
      ]
      \addplot[dashed] plot coordinates {(4,1) (3.59,2.65)};
      \addplot[decoration={brace,mirror,raise=.1cm},decorate,thin] plot coordinates {(4,1) (3.59,2.65)};
      \addplot[] plot coordinates {(3.39,2.6) (3.44,2.4) (3.64,2.45)};
      \draw[] (axis cs: 1.3,1.9) arc (-18.5:14:.4cm);
              
      
      \addplot[very thick,penColor!50!white] {x/4+7/4};
      \addplot[very thick,penColor,->] plot coordinates {(1,2) (3,2.5)};
      \addplot[very thick,penColor2,->] plot coordinates {(1,2) (4,1)};

      \node[above] at (axis cs:2,2.25) [penColor] {$\vec{v}$};
      \node[below left] at (axis cs:2.5,1.5) [penColor2] {$\vec{p}-\vec{q}$};
      \node[right] at (axis cs:1.4,2) {$\theta$};
      \node[right] at (axis cs:4,1.83) {$|\vec{p}-\vec{q}|\sin(\theta)$};
    \end{axis}
  \end{tikzpicture}
\end{image}
that the distance we are looking for is given by
\[
|\vec{p}-\vec{q}|\sin(\theta).
\]
However,
\[
|\vec{p}-\vec{q}||\vec{v}|\sin(\theta) = |(\vec{p}-\vec{q})\cross \vec{v}|
\]
so we see that
\[
\mathrm{distance} = |\vec{p}-\vec{q}|\sin(\theta) = \frac{|(\vec{p}-\vec{q})\cross \vec{v}|}{|\vec{v}|}.
\]
Try your hand at it by answering the following questions:

\begin{question}
  What is the distance between the point $(1,2,3)$ and the line that
  passes through the origin and $(1,-2,2)$?
  \begin{prompt}
    \[
    \mathrm{distance} = \answer{\sqrt{13}}
    \]
  \end{prompt}
\end{question}

Try to use a similar technique for points and lines in $\R^2$:

\begin{question}
  What is the distance between the point $(3,1)$ and the line that
  passes through the points $(1,1)$ and $(3,2)$?
   \begin{hint}
    To use the cross product, make these points $3$-dimensional by
    adding a $z$-component of $0$ to each point.
  \end{hint}
  \begin{hint}
    In this case $\vecl(t) = \vector{1,1,0} + t
    (\vector{3,2,0}-\vector{1,1,0})$.
  \end{hint}
  \begin{prompt}
    \[
    \mathrm{distance} = \answer{2/\sqrt{5}}
    \]
  \end{prompt}
\end{question}

However, depending on the question, you might want to think before
blindly applying formulas. Try your hand at this last question:

\begin{question}
  Consider the line:
  \[
  \vecl(t) = \vector{-3,5,2} +t \vector{-1,-1,1}
  \]
  What point on $\vecl$ is nearest to the point $(1,-1,-3)$?
  \begin{hint}
    Here, we are not asking for the distance, we are asking for the nearest point.
  \end{hint}
  \begin{hint}
    It will be easiest to use an orthogonal projection to answer this question.
  \end{hint}
  \begin{prompt}
    The point on $\vecl$ closest to $(1,-1,-3)$ is:
    \[
    \left(\answer{-2},\answer{6},\answer{1}\right)
    \]
  \end{prompt}
\end{question}




\section{Circles and ellipses}


Given two orthogonal unit vectors, $\uvec{u}$ and $\uvec{v}$, and any
other vector $\vec{p}$, the vector-valued function
\[
\vec{f}(t) = \vec{p}+r\cdot \cos(t)\cdot \uvec{u} + r \cdot \sin(t)\cdot\uvec{v}
\]
gives a circle of radius $r$, centered at the tip of $\vec{p}$, lying
in the plane containing $\uvec{u}$ and $\uvec{v}$. Moreover, to produce
an ellipse, we write:
\[
\vec{g}(t) = \vec{p}+a\cdot \cos(t)\cdot \uvec{u} + b \cdot \sin(t)\cdot\uvec{v}
\]
\begin{definition}
  Given an ellipse, the \dfn{major axis} of an ellipse is its longest
  diameter, and the \dfn{minor axis} is its smallest diameter. The
  \dfn{semi-major axis} is half of the major axis, and the
  \dfn{semi-minor axis} is half of the minor axis.
  Given an ellipse of the form
  \[
  \vec{g}(t) = \vec{p}+a\cdot \cos(t)\cdot \uvec{u} + b \cdot \sin(t)\cdot\uvec{v}
  \]
  where $a>b$, $a$ is the semi-major axis and $b$ is the semi-minor axis.
\end{definition}

Let's see an example.

\begin{example}
  Give a vector-valued formula for an ellipse that is drawn in the
  $(y,z)$-plane centered at the point $(0,2,3)$ whose semi-major axis
  is $5$ on a line parallel to the $y$-axis, and whose semi-minor axis
  is $1$ on a line parallel to the $z$-axis.
  \begin{explanation}
    There are actually infinitely many solutions to this problem,
    though we'll just give one. Write:
    \[
    \vec{g}(t) = \vector{\answer[given]{0},\answer[given]{2},\answer[given]{3}} + \answer[given]{5}\cdot \cos(t) \vecj + \answer[given]{1} \cdot \sin(t) \veck
    \]
  \end{explanation}
\end{example}


\begin{question}
  Can you find a vector-valued formula for a circle of radius $3$ in
  the plane $y=2$ centered at $(3,2,0)$?
  \begin{prompt}
    A circle we seek is:
    \[
    \vec{f}(t) = \vector{\answer{3},\answer{2},\answer{0}} + \answer{3}\cos(t) \veci + \answer{3}\sin(t) \veck
    \]
  \end{prompt}
\end{question}



\section{Lines and curves embedded in surfaces}

Curves can lie on surfaces. Typically, the surface is defined
implicitly, and the curve is a vector-valued function. To check if the
curve lies on the surface, break the curve into components and
substitute:
\begin{itemize}
  \item The $x$-component of the curve for $x$ in the equation of the
    surface.
  \item The $y$-component of the curve for $y$ in the equation of
    the surface.
  \item The $z$-component of the curve for $z$ in the equation of the
    surface.
\end{itemize}
If the equation defining the surfaces holds after the substitution, the
curve lies on the surface. Try your hand at these puzzles:

\begin{question}
  Consider the plane:
  \[
  x+2y+3z = 6
  \]
  Which of the following lines are on this plane?
  \begin{hint}
    Separate each line into its component functions: $x$, $y$, and
    $z$, and see if the equation defining the surface is valid for all
    $t$.
  \end{hint}
  \begin{selectAll}
    \choice[correct]{$\vector{3,-3,3}+t\vector{-7,5,-1}$}
    \choice{$\vector{-1,-1,3}+t\vector{3,-2,3}$}
    \choice[correct]{$\vector{-4,2,2}+t\vector{5,-1,-1}$}
    \choice[correct]{$\vector{1,1,1}+t\vector{2,-4,2}$}
    \choice{$\vector{1,-1,-1}+t\vector{-7,5,1}$}
    \choice{$\vector{-2,-2,4}+t\vector{4,-2,2}$}
  \end{selectAll}
\end{question}

\begin{question}
  Consider the planes:
  \begin{align*}
    -2x+3y+4z &=6\\
    4x+y+6z &=12
  \end{align*}
  \begin{hint}
    Separate each line into its component functions: $x$, $y$, and
    $z$, and see if the equation defining the surface is valid for all
    $t$.
  \end{hint}
 \begin{selectAll}
    \choice{$\vector{22/7,38/7,-1}+t\vector{1,2,-1}$}
    \choice[correct]{$\vector{22/7,38/7,-1}+ t\vector{-2,-4,2}$}
    \choice{$\vector{8/7,10/7,1}+t\vector{2,4,-2}$}
    \choice{$\vector{15/7, 24/7, 0}+t\vector{7,-14,-7}$}
    \choice[correct]{$\vector{15/7, 24/7, 0}+t\vector{7,14,-7}$}
    \choice[correct]{$\vector{8/7,10/7,1}+t\vector{1,2,-1}$}
  \end{selectAll}
\end{question}


Sometimes lines lie on surprising surfaces:

\begin{question}
  Consider the surface determined by all $x$, $y$ and $z$ such that:
  \[
  x^2+y^2=z^2+1
  \]
  This surface looks something like:
  \begin{image}
    \begin{tikzpicture}
      \begin{axis}[
          xmin=-4, xmax=4, ymin =-4, ymax = 4,
          clip=false,
          width=4in,
          height=2in,
          axis lines =none,
        ]
        \addplot [penColor,very thick,fill=fillp,domain=0 :360,smooth] ({2.9*cos(x)},{.5*sin(x)+2.8});
        \addplot [penColor,very thick,domain=180 :360,smooth] ({cos(x)},{.3*sin(x)});
        \addplot [penColor!50!white,very thick,domain=0 :180,smooth] ({cos(x)},{.3*sin(x)});
        \addplot [penColor,very thick,domain=180 :360,smooth] ({2.9*cos(x)},{.5*sin(x)-2.7});
        \addplot [penColor!50!white,very thick,domain=0 :180,smooth] ({2.9*cos(x)},{.5*sin(x)-2.7});
         
        \addplot [penColor,very thick,domain=-2.9:-1,smooth,samples=100] {sqrt(x^2-1)};%(sqrt(1+u^2) cos(v), sqrt(1+u^2) sin(v), u)
        \addplot [penColor,very thick,domain=1:2.9,smooth,samples=100] {sqrt(x^2-1)};    % v in 0 2pi % u is whatevs
        \addplot [penColor,very thick,domain=1:2.9,smooth,samples=100] {-sqrt(x^2-1)};
        \addplot [penColor,very thick,domain=-2.9:-1,smooth,samples=100] {-sqrt(x^2-1)};

        \draw[->] (axis cs: -3,0)--(axis cs: 3,0);
        \draw[->] (axis cs: 0,-3.5)--(axis cs: 0,3.5);
        \draw[->] (axis cs: 1.2,2.1)--(axis cs: -1.2,-2.1);
        \node[right] at (axis cs: 3,0) {$y$};
        \node[above] at (axis cs: 0,3.5) {$z$};
        \node[below] at (axis cs: -1.2,-2.1) {$x$};
      \end{axis}
    \end{tikzpicture}
  \end{image}
  Which of the following lines lie on the surface $x^2+y^2=z^2+1$?
  \begin{hint}
    Separate each line into its component functions: $x$, $y$, and
    $z$, and see if the equation defining the surface is valid for all
    $t$.
  \end{hint}
  \begin{selectAll}% l(t) = (cos v,sin v, 0) + t(sin v,-cos v,1)% m(t) = (cos v,sin v, 0) + t(-sin v,cos v,1)
    \choice{$\vector{1,0,0} + t\vector{-5,0,5}$}
    \choice[correct]{$\vector{1,0,0} + t\vector{0,-3,3}$}
    \choice[correct]{$\vector{0,1,0} + t\vector{-7,0,-7}$}
    \choice{$\vector{0,1,0} + t\vector{-4,4,0}$}
    \choice[correct]{$\vector{1/2,\sqrt{3}/2,0} + t\vector{-\sqrt{3},1,2}$}
    \choice{$\vector{1/2,\sqrt{3}/2,0} + t\vector{-1,\sqrt{3},2}$}
  \end{selectAll}
\end{question}

Though their formulation may be more complex, a vector-valued function
that produces a curve is no different from that which produces a line
(a line is a special type of curve!).

\begin{question}
  Consider the unit sphere:
  \[
  x^2+y^2+z^2 = 1
  \]
  Which of the following curves lie on this sphere?
  \begin{hint}
    Separate each curve into its component functions: $x$, $y$, and
    $z$, and see if the equation defining the surface is valid for all
    $t$.
  \end{hint}
  \begin{selectAll}
    \choice[correct]{$\vector{\cos(t),\sin(t),0}$}
    \choice[correct]{$\vector{\sin(t),0,\cos(t)}$}
    \choice[correct]{$\vector{\cos(t),\sin(t),1}/\sqrt{2}$}
    \choice[correct]{$\vector{\sin(t),\sin(t),\cos(t)}/\sqrt{2}$}
  \end{selectAll}
\end{question}




\begin{question}
  Consider the surface determined by all $x$, $y$ and $z$ such that:
  \[
  z^2=x^2+y^2
  \]
  Which of the following curves lie on the surface $z^2=x^2+y^2$?
  \begin{hint}
    Separate each curve into its component functions: $x$, $y$, and
    $z$, and see if the equation defining the surface is valid for all
    $t$.
  \end{hint}
  \begin{selectAll}
    \choice{$\vector{t\cos(t),\sin(t),t}$}
    \choice{$\vector{6\cos(t),6\sin(t),1}$}
    \choice[correct]{$\vector{t\cos(t),t\sin(t),t}$}
    \choice{$\vector{\cos(t),-2\sin(t),-2}$}
    \choice[correct]{$\vector{6\cos(t),6\sin(t),6}$}
    \choice[correct]{$\vector{-2\cos(t),-2\sin(t),-2}$}
  \end{selectAll}
\end{question}

\end{document}
