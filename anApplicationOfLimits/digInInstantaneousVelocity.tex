\documentclass{ximera}

%\usepackage{todonotes}
%\usepackage{mathtools} %% Required for wide table Curl and Greens
%\usepackage{cuted} %% Required for wide table Curl and Greens
\newcommand{\todo}{}

\usepackage{esint} % for \oiint
\ifxake%%https://math.meta.stackexchange.com/questions/9973/how-do-you-render-a-closed-surface-double-integral
\renewcommand{\oiint}{{\large\bigcirc}\kern-1.56em\iint}
\fi


\graphicspath{
  {./}
  {ximeraTutorial/}
  {basicPhilosophy/}
  {functionsOfSeveralVariables/}
  {normalVectors/}
  {lagrangeMultipliers/}
  {vectorFields/}
  {greensTheorem/}
  {shapeOfThingsToCome/}
  {dotProducts/}
  {partialDerivativesAndTheGradientVector/}
  {../productAndQuotientRules/exercises/}
  {../normalVectors/exercisesParametricPlots/}
  {../continuityOfFunctionsOfSeveralVariables/exercises/}
  {../partialDerivativesAndTheGradientVector/exercises/}
  {../directionalDerivativeAndChainRule/exercises/}
  {../commonCoordinates/exercisesCylindricalCoordinates/}
  {../commonCoordinates/exercisesSphericalCoordinates/}
  {../greensTheorem/exercisesCurlAndLineIntegrals/}
  {../greensTheorem/exercisesDivergenceAndLineIntegrals/}
  {../shapeOfThingsToCome/exercisesDivergenceTheorem/}
  {../greensTheorem/}
  {../shapeOfThingsToCome/}
  {../separableDifferentialEquations/exercises/}
  {vectorFields/}
}

\newcommand{\mooculus}{\textsf{\textbf{MOOC}\textnormal{\textsf{ULUS}}}}

\usepackage{tkz-euclide}\usepackage{tikz}
\usepackage{tikz-cd}
\usetikzlibrary{arrows}
\tikzset{>=stealth,commutative diagrams/.cd,
  arrow style=tikz,diagrams={>=stealth}} %% cool arrow head
\tikzset{shorten <>/.style={ shorten >=#1, shorten <=#1 } } %% allows shorter vectors

\usetikzlibrary{backgrounds} %% for boxes around graphs
\usetikzlibrary{shapes,positioning}  %% Clouds and stars
\usetikzlibrary{matrix} %% for matrix
\usepgfplotslibrary{polar} %% for polar plots
\usepgfplotslibrary{fillbetween} %% to shade area between curves in TikZ
\usetkzobj{all}
\usepackage[makeroom]{cancel} %% for strike outs
%\usepackage{mathtools} %% for pretty underbrace % Breaks Ximera
%\usepackage{multicol}
\usepackage{pgffor} %% required for integral for loops



%% http://tex.stackexchange.com/questions/66490/drawing-a-tikz-arc-specifying-the-center
%% Draws beach ball
\tikzset{pics/carc/.style args={#1:#2:#3}{code={\draw[pic actions] (#1:#3) arc(#1:#2:#3);}}}



\usepackage{array}
\setlength{\extrarowheight}{+.1cm}
\newdimen\digitwidth
\settowidth\digitwidth{9}
\def\divrule#1#2{
\noalign{\moveright#1\digitwidth
\vbox{\hrule width#2\digitwidth}}}





\newcommand{\RR}{\mathbb R}
\newcommand{\R}{\mathbb R}
\newcommand{\N}{\mathbb N}
\newcommand{\Z}{\mathbb Z}

\newcommand{\sagemath}{\textsf{SageMath}}


%\renewcommand{\d}{\,d\!}
\renewcommand{\d}{\mathop{}\!d}
\newcommand{\dd}[2][]{\frac{\d #1}{\d #2}}
\newcommand{\pp}[2][]{\frac{\partial #1}{\partial #2}}
\renewcommand{\l}{\ell}
\newcommand{\ddx}{\frac{d}{\d x}}

\newcommand{\zeroOverZero}{\ensuremath{\boldsymbol{\tfrac{0}{0}}}}
\newcommand{\inftyOverInfty}{\ensuremath{\boldsymbol{\tfrac{\infty}{\infty}}}}
\newcommand{\zeroOverInfty}{\ensuremath{\boldsymbol{\tfrac{0}{\infty}}}}
\newcommand{\zeroTimesInfty}{\ensuremath{\small\boldsymbol{0\cdot \infty}}}
\newcommand{\inftyMinusInfty}{\ensuremath{\small\boldsymbol{\infty - \infty}}}
\newcommand{\oneToInfty}{\ensuremath{\boldsymbol{1^\infty}}}
\newcommand{\zeroToZero}{\ensuremath{\boldsymbol{0^0}}}
\newcommand{\inftyToZero}{\ensuremath{\boldsymbol{\infty^0}}}



\newcommand{\numOverZero}{\ensuremath{\boldsymbol{\tfrac{\#}{0}}}}
\newcommand{\dfn}{\textbf}
%\newcommand{\unit}{\,\mathrm}
\newcommand{\unit}{\mathop{}\!\mathrm}
\newcommand{\eval}[1]{\bigg[ #1 \bigg]}
\newcommand{\seq}[1]{\left( #1 \right)}
\renewcommand{\epsilon}{\varepsilon}
\renewcommand{\phi}{\varphi}


\renewcommand{\iff}{\Leftrightarrow}

\DeclareMathOperator{\arccot}{arccot}
\DeclareMathOperator{\arcsec}{arcsec}
\DeclareMathOperator{\arccsc}{arccsc}
\DeclareMathOperator{\si}{Si}
\DeclareMathOperator{\scal}{scal}
\DeclareMathOperator{\sign}{sign}


%% \newcommand{\tightoverset}[2]{% for arrow vec
%%   \mathop{#2}\limits^{\vbox to -.5ex{\kern-0.75ex\hbox{$#1$}\vss}}}
\newcommand{\arrowvec}[1]{{\overset{\rightharpoonup}{#1}}}
%\renewcommand{\vec}[1]{\arrowvec{\mathbf{#1}}}
\renewcommand{\vec}[1]{{\overset{\boldsymbol{\rightharpoonup}}{\mathbf{#1}}}\hspace{0in}}

\newcommand{\point}[1]{\left(#1\right)} %this allows \vector{ to be changed to \vector{ with a quick find and replace
\newcommand{\pt}[1]{\mathbf{#1}} %this allows \vec{ to be changed to \vec{ with a quick find and replace
\newcommand{\Lim}[2]{\lim_{\point{#1} \to \point{#2}}} %Bart, I changed this to point since I want to use it.  It runs through both of the exercise and exerciseE files in limits section, which is why it was in each document to start with.

\DeclareMathOperator{\proj}{\mathbf{proj}}
\newcommand{\veci}{{\boldsymbol{\hat{\imath}}}}
\newcommand{\vecj}{{\boldsymbol{\hat{\jmath}}}}
\newcommand{\veck}{{\boldsymbol{\hat{k}}}}
\newcommand{\vecl}{\vec{\boldsymbol{\l}}}
\newcommand{\uvec}[1]{\mathbf{\hat{#1}}}
\newcommand{\utan}{\mathbf{\hat{t}}}
\newcommand{\unormal}{\mathbf{\hat{n}}}
\newcommand{\ubinormal}{\mathbf{\hat{b}}}

\newcommand{\dotp}{\bullet}
\newcommand{\cross}{\boldsymbol\times}
\newcommand{\grad}{\boldsymbol\nabla}
\newcommand{\divergence}{\grad\dotp}
\newcommand{\curl}{\grad\cross}
%\DeclareMathOperator{\divergence}{divergence}
%\DeclareMathOperator{\curl}[1]{\grad\cross #1}
\newcommand{\lto}{\mathop{\longrightarrow\,}\limits}

\renewcommand{\bar}{\overline}

\colorlet{textColor}{black}
\colorlet{background}{white}
\colorlet{penColor}{blue!50!black} % Color of a curve in a plot
\colorlet{penColor2}{red!50!black}% Color of a curve in a plot
\colorlet{penColor3}{red!50!blue} % Color of a curve in a plot
\colorlet{penColor4}{green!50!black} % Color of a curve in a plot
\colorlet{penColor5}{orange!80!black} % Color of a curve in a plot
\colorlet{penColor6}{yellow!70!black} % Color of a curve in a plot
\colorlet{fill1}{penColor!20} % Color of fill in a plot
\colorlet{fill2}{penColor2!20} % Color of fill in a plot
\colorlet{fillp}{fill1} % Color of positive area
\colorlet{filln}{penColor2!20} % Color of negative area
\colorlet{fill3}{penColor3!20} % Fill
\colorlet{fill4}{penColor4!20} % Fill
\colorlet{fill5}{penColor5!20} % Fill
\colorlet{gridColor}{gray!50} % Color of grid in a plot

\newcommand{\surfaceColor}{violet}
\newcommand{\surfaceColorTwo}{redyellow}
\newcommand{\sliceColor}{greenyellow}




\pgfmathdeclarefunction{gauss}{2}{% gives gaussian
  \pgfmathparse{1/(#2*sqrt(2*pi))*exp(-((x-#1)^2)/(2*#2^2))}%
}


%%%%%%%%%%%%%
%% Vectors
%%%%%%%%%%%%%

%% Simple horiz vectors
\renewcommand{\vector}[1]{\left\langle #1\right\rangle}


%% %% Complex Horiz Vectors with angle brackets
%% \makeatletter
%% \renewcommand{\vector}[2][ , ]{\left\langle%
%%   \def\nextitem{\def\nextitem{#1}}%
%%   \@for \el:=#2\do{\nextitem\el}\right\rangle%
%% }
%% \makeatother

%% %% Vertical Vectors
%% \def\vector#1{\begin{bmatrix}\vecListA#1,,\end{bmatrix}}
%% \def\vecListA#1,{\if,#1,\else #1\cr \expandafter \vecListA \fi}

%%%%%%%%%%%%%
%% End of vectors
%%%%%%%%%%%%%

%\newcommand{\fullwidth}{}
%\newcommand{\normalwidth}{}



%% makes a snazzy t-chart for evaluating functions
%\newenvironment{tchart}{\rowcolors{2}{}{background!90!textColor}\array}{\endarray}

%%This is to help with formatting on future title pages.
\newenvironment{sectionOutcomes}{}{}



%% Flowchart stuff
%\tikzstyle{startstop} = [rectangle, rounded corners, minimum width=3cm, minimum height=1cm,text centered, draw=black]
%\tikzstyle{question} = [rectangle, minimum width=3cm, minimum height=1cm, text centered, draw=black]
%\tikzstyle{decision} = [trapezium, trapezium left angle=70, trapezium right angle=110, minimum width=3cm, minimum height=1cm, text centered, draw=black]
%\tikzstyle{question} = [rectangle, rounded corners, minimum width=3cm, minimum height=1cm,text centered, draw=black]
%\tikzstyle{process} = [rectangle, minimum width=3cm, minimum height=1cm, text centered, draw=black]
%\tikzstyle{decision} = [trapezium, trapezium left angle=70, trapezium right angle=110, minimum width=3cm, minimum height=1cm, text centered, draw=black]


\outcome{Compute limits of families of functions.} 
\outcome{Compute average velocity.}
\outcome{Approximate instantaneous velocity.}
\outcome{Compare average and instantaneous velocity.}
\outcome{Plot difference quotients for varying approximations of the instantaneous rate of change.}

\title[Dig-In:]{Instantaneous velocity}

\begin{document}
\begin{abstract}
We use limits to compute instantaneous velocity.
\end{abstract}
\maketitle

When we compute average velocity, we look at 
\[
\frac{\text{change in position}}{\text{change in time}}.
\]
To obtain the (instantaneous) velocity, we want the change in time to
``go to'' zero. By this point we should know that ``go to'' is a
buzz-word for a \emph{limit}. The change in time is often given as
the length of an interval, and this length goes to zero.

The average velocity on the (time) interval $[a,b]$ is given by
\[
v_{\text{av}} = 
\frac{\text{change in position}}{\text{change in time}} =
\frac{s(b)-s(a)}{b-a}.
\]
Here $s(t)$ denotes the position, at the time $t$, of an object moving along a line.

Let's put all of this together by working an example.

\begin{example}
A young mathematician throws a ball straight into the air with 
a velocity of 40ft/sec. Its height (in feet) after $t$ seconds 
is given by
\[
s(t) = 40t-16t^2 \qquad\text{(feet above the ground)} .
\]

When will the ball hit the ground?

\begin{explanation}
To determine when the ball hits the ground we need to solve the
equation
\[
s(t)=0
\]
for t.  That is,
\begin{align*}
40t-16t^2 &= 0\\
t(40-16t) &= 0.
\end{align*}
This has solutions $t=0$
seconds and $t=\answer[given]{2.5}$ seconds.  Since the ball hits
the ground \wordChoice{\choice{before}\choice[correct]{after}} it's
thrown, we know that the ball hits the ground at $t=\answer[given]{2.5}$
seconds.
\end{explanation}

What is the height of the ball after $2$ seconds?

\begin{explanation}
To find the height of the ball after $2$ seconds we simply need 
to plug $2$ into the equation for $s(t)$ to find
\[
s(2) = 40\cdot\answer[given]{2} - 16\cdot\answer[given]{2}^2 = 
\answer[given]{16}\unit{ft.}
\]

\end{explanation}

Consider the following points lying along the $s$ axis.
\begin{image}
\begin{tikzpicture}
    \begin{axis}[
        ymin=-10.3,ymax=25.3,xmin=-20,xmax=20,
        clip=false,
        unit vector ratio*=1 1 1,
        axis lines=center,
        ytick={-10,-5,...,25},
        hide x axis,
        ylabel=$s$,
        every axis y label/.style={
          at={(ticklabel* cs:1)},
          anchor=south},
      ]
        \addplot[only marks,very thick,penColor,mark=*]
	        coordinates{(0,-6)};
	    \node at (axis cs:0,-6) [right] {$A$};
	    
        \addplot[only marks,very thick,penColor,mark=*]
	        coordinates{(0,0)};
	    \node at (axis cs:0,0) [right] {$B$};
	    
	    \addplot[only marks,very thick,penColor,mark=*]
	        coordinates{(0,40-16)};
	    \node at (axis cs:0,40-16) [right] {$D$};
	    
        \addplot[only marks,very thick,penColor,mark=*]
	        coordinates{(0,40*2-16*2^2)};
	        \node at (axis cs:0,40*2-16*2^2) [right] {$C$};
        \addplot[only marks,very thick,white,mark=*]
	coordinates{(19,0)};
        \addplot[only marks,very thick,white,mark=*]
	        coordinates{(-19,0)};
    \end{axis}`
\end{tikzpicture}
\end{image}

Which points correspond to the height of the ball at times $0$, $1$ and $2$? 
\begin{explanation}
The point that corresponds to $s(0)$, the position (height) of the 
ball at $t=0$, is \wordChoice{\choice{A}\choice[correct]{B}\choice{C}\choice{D}}.

The point that corresponds to $s(1)$, the position (height) of the 
ball at $t=1$, is \wordChoice{\choice{A}\choice{B}\choice{C}\choice[correct]{D}}.

The point that corresponds to $s(2)$, the position (height) of the 
ball at $t=2$, is \wordChoice{\choice{A}\choice{B}\choice[correct]{C}\choice{D}}.

\end{explanation}

Next let's consider the average velocity of the ball. What is the average velocity of the ball on the interval $[1,2]$?

\begin{explanation}

In order to find the average velocity of the ball on the 
interval $[1,2]$ we recall that the average velocity on 
the interval $[a,b]$ is given by
\[
v_{\text{av}} = 
\frac{s(b) - s\left(\answer[given]{a}\right)}
{\answer[given]{b} - \answer[given]{a}}.
\]
Plugging in $a=1$ and $b=2$ we find that 
\[
v_{\text{av}} = \frac{s(2)-s(1)}{2-1} = 
\frac{\answer[given]{16} - \answer[given]{24}}{1} =
\answer[given]{-8}\text{ft/s.}
\] 

\end{explanation}

What is the average velocity of the ball on the interval
$[t,2]$ for $0<t<2$?.  

\begin{explanation}

We use the same formula we used to find the average velocity on
the interval $[1,2]$ to find the average velocity on the interval
$[t,2]$ for $0<t<2$.
\begin{align*}
v_{\text{av}} &= 
\frac{s(2)-s\left(\answer[given]{t}\right)}
{\answer[given]{2}-\answer[given]{t}}\\
&= \frac{16 - (40t-16t^2)}{2-t}\\
&= \frac{8(2t^2-5t+2)}{2-t}\\
&= \frac{8(2t-1)(t-2)}{-(t-2)}\\
&= -8(2t-1)\\
&= (8-16t)\text{ft/s}
\end{align*}
for $0<t<2$.

\end{explanation}

What is the average velocity of the ball on the interval $[2,t]$ for
$2<t<2.5$?

\begin{explanation}

To calculate the average velocity on the interval $[2,t]$ for 
$2<t<2.5$ we will use our average velocity formula one more time
to find
\[
v_{\text{av}} = \frac{s(t)-s(2)}{t-2} = \frac{s(2)-s(t)}{2-t}. 
\]
However, this is exactly the same expression we got when calculating
the average velocity on the interval $[t,2]$ for $0<t<2$.  So the 
average velocity on the interval $[2,t]$ for $2<t<2.5$ is given by
\[
v_{\text{av}} =\answer[given]{(8-16t)}\text{ft/s.}
\]
\end{explanation}
\end{example}

In our previous example, we computed \textit{average velocity} on
several different intervals. If we let the size of the interval go to
zero, we get \dfn{instantaneous velocity}. Limits will allow us to
compute instantaneous velocity.  Let's use the same setting as before.

\begin{example}
The height of a ball above the ground between 0 and 2.5 seconds
is given by
\[
s(t) = 40t - 16t^2.
\] 
Find the instantaneous velocity of the ball 2 seconds after it
is thrown.
\begin{explanation}
From the previous example, we know that the average velocity of the
ball on the interval $[t,2]$ for $0<t<2$ and the average velocity
on the interval $[2,t]$ for $2<t<2.5$ are both given by
\[
v_{\text{av}} =  \frac{s(t)-s(2)}{t-2}= \answer[given]{(8-16t)}\text{ft/s.}
\]
All we need to do to find the instantaneous velocity at $t=2$ is 
take the limit as $t$ goes to $\answer[given]{2}$ of the expression
above.  Doing so we find
\[
v = \lim_{t\to2}v_{\text{av}}
=\lim_{t\to2}  \frac{s(t)-s(2)}{t-2}
= \lim_{t\to2}(8-16t)
= \answer[given]{-24}\text{ft/s.}
\]
\end{explanation}
\end{example}
\end{document}
