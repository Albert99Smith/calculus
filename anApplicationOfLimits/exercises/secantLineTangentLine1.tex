\documentclass{ximera}

%\usepackage{todonotes}
%\usepackage{mathtools} %% Required for wide table Curl and Greens
%\usepackage{cuted} %% Required for wide table Curl and Greens
\newcommand{\todo}{}

\usepackage{esint} % for \oiint
\ifxake%%https://math.meta.stackexchange.com/questions/9973/how-do-you-render-a-closed-surface-double-integral
\renewcommand{\oiint}{{\large\bigcirc}\kern-1.56em\iint}
\fi


\graphicspath{
  {./}
  {ximeraTutorial/}
  {basicPhilosophy/}
  {functionsOfSeveralVariables/}
  {normalVectors/}
  {lagrangeMultipliers/}
  {vectorFields/}
  {greensTheorem/}
  {shapeOfThingsToCome/}
  {dotProducts/}
  {partialDerivativesAndTheGradientVector/}
  {../productAndQuotientRules/exercises/}
  {../normalVectors/exercisesParametricPlots/}
  {../continuityOfFunctionsOfSeveralVariables/exercises/}
  {../partialDerivativesAndTheGradientVector/exercises/}
  {../directionalDerivativeAndChainRule/exercises/}
  {../commonCoordinates/exercisesCylindricalCoordinates/}
  {../commonCoordinates/exercisesSphericalCoordinates/}
  {../greensTheorem/exercisesCurlAndLineIntegrals/}
  {../greensTheorem/exercisesDivergenceAndLineIntegrals/}
  {../shapeOfThingsToCome/exercisesDivergenceTheorem/}
  {../greensTheorem/}
  {../shapeOfThingsToCome/}
  {../separableDifferentialEquations/exercises/}
  {vectorFields/}
}

\newcommand{\mooculus}{\textsf{\textbf{MOOC}\textnormal{\textsf{ULUS}}}}

\usepackage{tkz-euclide}\usepackage{tikz}
\usepackage{tikz-cd}
\usetikzlibrary{arrows}
\tikzset{>=stealth,commutative diagrams/.cd,
  arrow style=tikz,diagrams={>=stealth}} %% cool arrow head
\tikzset{shorten <>/.style={ shorten >=#1, shorten <=#1 } } %% allows shorter vectors

\usetikzlibrary{backgrounds} %% for boxes around graphs
\usetikzlibrary{shapes,positioning}  %% Clouds and stars
\usetikzlibrary{matrix} %% for matrix
\usepgfplotslibrary{polar} %% for polar plots
\usepgfplotslibrary{fillbetween} %% to shade area between curves in TikZ
\usetkzobj{all}
\usepackage[makeroom]{cancel} %% for strike outs
%\usepackage{mathtools} %% for pretty underbrace % Breaks Ximera
%\usepackage{multicol}
\usepackage{pgffor} %% required for integral for loops



%% http://tex.stackexchange.com/questions/66490/drawing-a-tikz-arc-specifying-the-center
%% Draws beach ball
\tikzset{pics/carc/.style args={#1:#2:#3}{code={\draw[pic actions] (#1:#3) arc(#1:#2:#3);}}}



\usepackage{array}
\setlength{\extrarowheight}{+.1cm}
\newdimen\digitwidth
\settowidth\digitwidth{9}
\def\divrule#1#2{
\noalign{\moveright#1\digitwidth
\vbox{\hrule width#2\digitwidth}}}





\newcommand{\RR}{\mathbb R}
\newcommand{\R}{\mathbb R}
\newcommand{\N}{\mathbb N}
\newcommand{\Z}{\mathbb Z}

\newcommand{\sagemath}{\textsf{SageMath}}


%\renewcommand{\d}{\,d\!}
\renewcommand{\d}{\mathop{}\!d}
\newcommand{\dd}[2][]{\frac{\d #1}{\d #2}}
\newcommand{\pp}[2][]{\frac{\partial #1}{\partial #2}}
\renewcommand{\l}{\ell}
\newcommand{\ddx}{\frac{d}{\d x}}

\newcommand{\zeroOverZero}{\ensuremath{\boldsymbol{\tfrac{0}{0}}}}
\newcommand{\inftyOverInfty}{\ensuremath{\boldsymbol{\tfrac{\infty}{\infty}}}}
\newcommand{\zeroOverInfty}{\ensuremath{\boldsymbol{\tfrac{0}{\infty}}}}
\newcommand{\zeroTimesInfty}{\ensuremath{\small\boldsymbol{0\cdot \infty}}}
\newcommand{\inftyMinusInfty}{\ensuremath{\small\boldsymbol{\infty - \infty}}}
\newcommand{\oneToInfty}{\ensuremath{\boldsymbol{1^\infty}}}
\newcommand{\zeroToZero}{\ensuremath{\boldsymbol{0^0}}}
\newcommand{\inftyToZero}{\ensuremath{\boldsymbol{\infty^0}}}



\newcommand{\numOverZero}{\ensuremath{\boldsymbol{\tfrac{\#}{0}}}}
\newcommand{\dfn}{\textbf}
%\newcommand{\unit}{\,\mathrm}
\newcommand{\unit}{\mathop{}\!\mathrm}
\newcommand{\eval}[1]{\bigg[ #1 \bigg]}
\newcommand{\seq}[1]{\left( #1 \right)}
\renewcommand{\epsilon}{\varepsilon}
\renewcommand{\phi}{\varphi}


\renewcommand{\iff}{\Leftrightarrow}

\DeclareMathOperator{\arccot}{arccot}
\DeclareMathOperator{\arcsec}{arcsec}
\DeclareMathOperator{\arccsc}{arccsc}
\DeclareMathOperator{\si}{Si}
\DeclareMathOperator{\scal}{scal}
\DeclareMathOperator{\sign}{sign}


%% \newcommand{\tightoverset}[2]{% for arrow vec
%%   \mathop{#2}\limits^{\vbox to -.5ex{\kern-0.75ex\hbox{$#1$}\vss}}}
\newcommand{\arrowvec}[1]{{\overset{\rightharpoonup}{#1}}}
%\renewcommand{\vec}[1]{\arrowvec{\mathbf{#1}}}
\renewcommand{\vec}[1]{{\overset{\boldsymbol{\rightharpoonup}}{\mathbf{#1}}}\hspace{0in}}

\newcommand{\point}[1]{\left(#1\right)} %this allows \vector{ to be changed to \vector{ with a quick find and replace
\newcommand{\pt}[1]{\mathbf{#1}} %this allows \vec{ to be changed to \vec{ with a quick find and replace
\newcommand{\Lim}[2]{\lim_{\point{#1} \to \point{#2}}} %Bart, I changed this to point since I want to use it.  It runs through both of the exercise and exerciseE files in limits section, which is why it was in each document to start with.

\DeclareMathOperator{\proj}{\mathbf{proj}}
\newcommand{\veci}{{\boldsymbol{\hat{\imath}}}}
\newcommand{\vecj}{{\boldsymbol{\hat{\jmath}}}}
\newcommand{\veck}{{\boldsymbol{\hat{k}}}}
\newcommand{\vecl}{\vec{\boldsymbol{\l}}}
\newcommand{\uvec}[1]{\mathbf{\hat{#1}}}
\newcommand{\utan}{\mathbf{\hat{t}}}
\newcommand{\unormal}{\mathbf{\hat{n}}}
\newcommand{\ubinormal}{\mathbf{\hat{b}}}

\newcommand{\dotp}{\bullet}
\newcommand{\cross}{\boldsymbol\times}
\newcommand{\grad}{\boldsymbol\nabla}
\newcommand{\divergence}{\grad\dotp}
\newcommand{\curl}{\grad\cross}
%\DeclareMathOperator{\divergence}{divergence}
%\DeclareMathOperator{\curl}[1]{\grad\cross #1}
\newcommand{\lto}{\mathop{\longrightarrow\,}\limits}

\renewcommand{\bar}{\overline}

\colorlet{textColor}{black}
\colorlet{background}{white}
\colorlet{penColor}{blue!50!black} % Color of a curve in a plot
\colorlet{penColor2}{red!50!black}% Color of a curve in a plot
\colorlet{penColor3}{red!50!blue} % Color of a curve in a plot
\colorlet{penColor4}{green!50!black} % Color of a curve in a plot
\colorlet{penColor5}{orange!80!black} % Color of a curve in a plot
\colorlet{penColor6}{yellow!70!black} % Color of a curve in a plot
\colorlet{fill1}{penColor!20} % Color of fill in a plot
\colorlet{fill2}{penColor2!20} % Color of fill in a plot
\colorlet{fillp}{fill1} % Color of positive area
\colorlet{filln}{penColor2!20} % Color of negative area
\colorlet{fill3}{penColor3!20} % Fill
\colorlet{fill4}{penColor4!20} % Fill
\colorlet{fill5}{penColor5!20} % Fill
\colorlet{gridColor}{gray!50} % Color of grid in a plot

\newcommand{\surfaceColor}{violet}
\newcommand{\surfaceColorTwo}{redyellow}
\newcommand{\sliceColor}{greenyellow}




\pgfmathdeclarefunction{gauss}{2}{% gives gaussian
  \pgfmathparse{1/(#2*sqrt(2*pi))*exp(-((x-#1)^2)/(2*#2^2))}%
}


%%%%%%%%%%%%%
%% Vectors
%%%%%%%%%%%%%

%% Simple horiz vectors
\renewcommand{\vector}[1]{\left\langle #1\right\rangle}


%% %% Complex Horiz Vectors with angle brackets
%% \makeatletter
%% \renewcommand{\vector}[2][ , ]{\left\langle%
%%   \def\nextitem{\def\nextitem{#1}}%
%%   \@for \el:=#2\do{\nextitem\el}\right\rangle%
%% }
%% \makeatother

%% %% Vertical Vectors
%% \def\vector#1{\begin{bmatrix}\vecListA#1,,\end{bmatrix}}
%% \def\vecListA#1,{\if,#1,\else #1\cr \expandafter \vecListA \fi}

%%%%%%%%%%%%%
%% End of vectors
%%%%%%%%%%%%%

%\newcommand{\fullwidth}{}
%\newcommand{\normalwidth}{}



%% makes a snazzy t-chart for evaluating functions
%\newenvironment{tchart}{\rowcolors{2}{}{background!90!textColor}\array}{\endarray}

%%This is to help with formatting on future title pages.
\newenvironment{sectionOutcomes}{}{}



%% Flowchart stuff
%\tikzstyle{startstop} = [rectangle, rounded corners, minimum width=3cm, minimum height=1cm,text centered, draw=black]
%\tikzstyle{question} = [rectangle, minimum width=3cm, minimum height=1cm, text centered, draw=black]
%\tikzstyle{decision} = [trapezium, trapezium left angle=70, trapezium right angle=110, minimum width=3cm, minimum height=1cm, text centered, draw=black]
%\tikzstyle{question} = [rectangle, rounded corners, minimum width=3cm, minimum height=1cm,text centered, draw=black]
%\tikzstyle{process} = [rectangle, minimum width=3cm, minimum height=1cm, text centered, draw=black]
%\tikzstyle{decision} = [trapezium, trapezium left angle=70, trapezium right angle=110, minimum width=3cm, minimum height=1cm, text centered, draw=black]


\outcome{Compute average velocity.}
\outcome{Compare average and instantaneous velocity.}

\author{Nela Lakos \and Kyle Parsons}

\begin{document}
\begin{exercise}

An object is moving along a horizontal line.  Its position in feet is given by
\[
s(t) = t^2 - 2
\]
where $0\leq t\leq 5$ is in seconds.

Consider the points on the line below.

\begin{tikzpicture}
    \begin{axis}[
        xmin=-4.3,xmax=4.3,ymin=-.1,ymax=.1,
        clip=false,
        unit vector ratio*=1 1 1,
        axis lines=center,
        hide obscured x ticks=false,
        grid = major,
        xtick={-4,-3,...,4},
        hide y axis,
        xlabel=$s$,
        every axis x label/.style={anchor=south}
        ]
        \addplot[only marks,very thick,penColor,mark=*]
        	coordinates{(-2,0) (-1,0) (0,0) (2,0) (4,0)};
	
		\node at (axis cs:-2,0) [penColor,above] {$A$};
		\node at (axis cs:-1,0) [penColor,above] {$B$};
		\node at (axis cs:0,0) [penColor,above] {$C$};
		\node at (axis cs:2,0) [penColor,above] {$D$};
		\node at (axis cs:4,0) [penColor,above] {$E$};
    \end{axis}`
\end{tikzpicture}

The point that corresponds to the position of the particle at $t=1$ is $\answer{B}$.

\begin{exercise}

The average velocity of the object on the interval $[1,3]$ is
\[
v_{\text{av}} = \answer{4}\text{ft/s.}
\]

The average velocity of the object on the interval $[1,t]$ for $t>1$ is
\[
v_{\text{av}} = \answer{t+1}\text{ft/s.}
\]

The average velocity of the object on the interval $[t,1]$ for $0<t<1$ is
\[
v_{\text{av}} = \answer{t+1}\text{ft/s.}
\]

\begin{exercise}

The instantaneous velocity of the object at $t=1$ is
\[
v_{\text{inst}} = \answer{2}\text{ft/s.}
\]

\begin{exercise}

The graph of $s(t)$ is given below.

\begin{tikzpicture}
    \begin{axis}[
        xmin=-0.3,xmax=3.3,ymin=-2.3,ymax=7.3,
        clip=true,
        unit vector ratio*=1 1 1,
        axis lines=center,
        hide obscured x ticks=true,
        grid = major,
        xtick={0,1,...,3},
        ytick={-2,-1,...,7},
        xlabel=$s$,
        ylabel=$t$,
        every axis y label/.style={at=(current axis.above origin),anchor=south},
        every axis x label/.style={at=(current axis.right of origin),anchor=west},
        ]
        \addplot[very thick,penColor,domain=0:3.3,samples=50] plot{x^2-2};
	
		\node at (axis cs:1.2,5.1) [penColor,above] {$s=s(t)$};
    \end{axis}`
\end{tikzpicture}

Assume that $P$ and $A$ are points on the graph of $s(t)$.  Then
\[
P = \left(1,\answer{-1}\right) \text{ and } A = \left(t,\answer{t^2-2}\right).
\]

\begin{exercise}

The secant line through $P$ and $A$ is shown in the figure below.

\begin{tikzpicture}
    \begin{axis}[
        xmin=-0.3,xmax=3.3,ymin=-2.3,ymax=7.3,
        clip=true,
        unit vector ratio*=1 1 1,
        axis lines=center,
        hide obscured x ticks=true,
        grid = major,
        xtick={0,1,...,3},
        ytick={-2,-1,...,7},
        xlabel=$s$,
        ylabel=$t$,
        every axis y label/.style={at=(current axis.above origin),anchor=south},
        every axis x label/.style={at=(current axis.right of origin),anchor=west},
        ]
        \addplot[very thick,penColor,domain=0:3.3,samples=50] plot{x^2-2};
		
		\addplot[very thick,black,only marks,mark=*] coordinates{(0.3,0.3^2-2)};
		\node at (axis cs:0.3,0.3^2-2) [below right] {$A$};
		
		\addplot[very thick,black,only marks,mark=*] coordinates{(1,1^2-2)};
		\node at (axis cs:1,1^2-2) [below right] {$P$};
		
		\addplot[very thick,red,domain=-0.3:3.3] plot{(-1-(.3^2-2))/(1-.3)*(x-1)-1};
		
		\node at (axis cs:1.2,5.1) [penColor,above] {$s=s(t)$};
    \end{axis}`
\end{tikzpicture}

The slope of this secant line for $0<t<1$ is
\[
m_{\text{sec}} = \answer{\frac{1-t^2}{1-t}}.
\]

What is the connection between the slope of this secant line and the average velocity of the object over the interval $[t,1]$?

\begin{multipleChoice}
\choice{$m_{\text{sec}}$ is greater than $v_{\text{av}}$.}
\choice[correct]{$m_{\text{sec}}$ equals $v_{\text{av}}$.}
\choice{$m_{\text{sec}}$ is less than $v_{\text{av}}$.}
\choice{There is no connection between $m_{\text{sec}}$ and $v_{\text{av}}$.}
\end{multipleChoice}

\begin{exercise}

The point $P$ and the tangent line to $s(t)$ at $P$ are given in the figure below.

\begin{tikzpicture}
    \begin{axis}[
        xmin=-0.3,xmax=3.3,ymin=-2.3,ymax=7.3,
        clip=true,
        unit vector ratio*=1 1 1,
        axis lines=center,
        hide obscured x ticks=true,
        grid = major,
        xtick={0,1,...,3},
        ytick={-2,-1,...,7},
        xlabel=$s$,
        ylabel=$t$,
        every axis y label/.style={at=(current axis.above origin),anchor=south},
        every axis x label/.style={at=(current axis.right of origin),anchor=west},
        ]
        \addplot[very thick,penColor,domain=0:3.3,samples=50] plot{x^2-2};
		
		\addplot[very thick,black,only marks,mark=*] coordinates{(1,1^2-2)};
		\node at (axis cs:1,1^2-2) [below right] {$P$};
		
		\addplot[very thick,red,domain=-0.3:3.3] plot{2*(x-1)-1};
		
		\node at (axis cs:1.2,5.1) [penColor,above] {$s=s(t)$};
    \end{axis}`
\end{tikzpicture}

The slope of the tangent line above is
\[
m_{\text{tan}} = \answer{2}.
\]


\end{exercise}
\end{exercise}
\end{exercise}
\end{exercise}
\end{exercise}
\end{exercise}
\end{document}