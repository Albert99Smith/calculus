\documentclass{ximera}

%\usepackage{todonotes}
%\usepackage{mathtools} %% Required for wide table Curl and Greens
%\usepackage{cuted} %% Required for wide table Curl and Greens
\newcommand{\todo}{}

\usepackage{esint} % for \oiint
\ifxake%%https://math.meta.stackexchange.com/questions/9973/how-do-you-render-a-closed-surface-double-integral
\renewcommand{\oiint}{{\large\bigcirc}\kern-1.56em\iint}
\fi


\graphicspath{
  {./}
  {ximeraTutorial/}
  {basicPhilosophy/}
  {functionsOfSeveralVariables/}
  {normalVectors/}
  {lagrangeMultipliers/}
  {vectorFields/}
  {greensTheorem/}
  {shapeOfThingsToCome/}
  {dotProducts/}
  {partialDerivativesAndTheGradientVector/}
  {../productAndQuotientRules/exercises/}
  {../normalVectors/exercisesParametricPlots/}
  {../continuityOfFunctionsOfSeveralVariables/exercises/}
  {../partialDerivativesAndTheGradientVector/exercises/}
  {../directionalDerivativeAndChainRule/exercises/}
  {../commonCoordinates/exercisesCylindricalCoordinates/}
  {../commonCoordinates/exercisesSphericalCoordinates/}
  {../greensTheorem/exercisesCurlAndLineIntegrals/}
  {../greensTheorem/exercisesDivergenceAndLineIntegrals/}
  {../shapeOfThingsToCome/exercisesDivergenceTheorem/}
  {../greensTheorem/}
  {../shapeOfThingsToCome/}
  {../separableDifferentialEquations/exercises/}
  {vectorFields/}
}

\newcommand{\mooculus}{\textsf{\textbf{MOOC}\textnormal{\textsf{ULUS}}}}

\usepackage{tkz-euclide}\usepackage{tikz}
\usepackage{tikz-cd}
\usetikzlibrary{arrows}
\tikzset{>=stealth,commutative diagrams/.cd,
  arrow style=tikz,diagrams={>=stealth}} %% cool arrow head
\tikzset{shorten <>/.style={ shorten >=#1, shorten <=#1 } } %% allows shorter vectors

\usetikzlibrary{backgrounds} %% for boxes around graphs
\usetikzlibrary{shapes,positioning}  %% Clouds and stars
\usetikzlibrary{matrix} %% for matrix
\usepgfplotslibrary{polar} %% for polar plots
\usepgfplotslibrary{fillbetween} %% to shade area between curves in TikZ
\usetkzobj{all}
\usepackage[makeroom]{cancel} %% for strike outs
%\usepackage{mathtools} %% for pretty underbrace % Breaks Ximera
%\usepackage{multicol}
\usepackage{pgffor} %% required for integral for loops



%% http://tex.stackexchange.com/questions/66490/drawing-a-tikz-arc-specifying-the-center
%% Draws beach ball
\tikzset{pics/carc/.style args={#1:#2:#3}{code={\draw[pic actions] (#1:#3) arc(#1:#2:#3);}}}



\usepackage{array}
\setlength{\extrarowheight}{+.1cm}
\newdimen\digitwidth
\settowidth\digitwidth{9}
\def\divrule#1#2{
\noalign{\moveright#1\digitwidth
\vbox{\hrule width#2\digitwidth}}}





\newcommand{\RR}{\mathbb R}
\newcommand{\R}{\mathbb R}
\newcommand{\N}{\mathbb N}
\newcommand{\Z}{\mathbb Z}

\newcommand{\sagemath}{\textsf{SageMath}}


%\renewcommand{\d}{\,d\!}
\renewcommand{\d}{\mathop{}\!d}
\newcommand{\dd}[2][]{\frac{\d #1}{\d #2}}
\newcommand{\pp}[2][]{\frac{\partial #1}{\partial #2}}
\renewcommand{\l}{\ell}
\newcommand{\ddx}{\frac{d}{\d x}}

\newcommand{\zeroOverZero}{\ensuremath{\boldsymbol{\tfrac{0}{0}}}}
\newcommand{\inftyOverInfty}{\ensuremath{\boldsymbol{\tfrac{\infty}{\infty}}}}
\newcommand{\zeroOverInfty}{\ensuremath{\boldsymbol{\tfrac{0}{\infty}}}}
\newcommand{\zeroTimesInfty}{\ensuremath{\small\boldsymbol{0\cdot \infty}}}
\newcommand{\inftyMinusInfty}{\ensuremath{\small\boldsymbol{\infty - \infty}}}
\newcommand{\oneToInfty}{\ensuremath{\boldsymbol{1^\infty}}}
\newcommand{\zeroToZero}{\ensuremath{\boldsymbol{0^0}}}
\newcommand{\inftyToZero}{\ensuremath{\boldsymbol{\infty^0}}}



\newcommand{\numOverZero}{\ensuremath{\boldsymbol{\tfrac{\#}{0}}}}
\newcommand{\dfn}{\textbf}
%\newcommand{\unit}{\,\mathrm}
\newcommand{\unit}{\mathop{}\!\mathrm}
\newcommand{\eval}[1]{\bigg[ #1 \bigg]}
\newcommand{\seq}[1]{\left( #1 \right)}
\renewcommand{\epsilon}{\varepsilon}
\renewcommand{\phi}{\varphi}


\renewcommand{\iff}{\Leftrightarrow}

\DeclareMathOperator{\arccot}{arccot}
\DeclareMathOperator{\arcsec}{arcsec}
\DeclareMathOperator{\arccsc}{arccsc}
\DeclareMathOperator{\si}{Si}
\DeclareMathOperator{\scal}{scal}
\DeclareMathOperator{\sign}{sign}


%% \newcommand{\tightoverset}[2]{% for arrow vec
%%   \mathop{#2}\limits^{\vbox to -.5ex{\kern-0.75ex\hbox{$#1$}\vss}}}
\newcommand{\arrowvec}[1]{{\overset{\rightharpoonup}{#1}}}
%\renewcommand{\vec}[1]{\arrowvec{\mathbf{#1}}}
\renewcommand{\vec}[1]{{\overset{\boldsymbol{\rightharpoonup}}{\mathbf{#1}}}\hspace{0in}}

\newcommand{\point}[1]{\left(#1\right)} %this allows \vector{ to be changed to \vector{ with a quick find and replace
\newcommand{\pt}[1]{\mathbf{#1}} %this allows \vec{ to be changed to \vec{ with a quick find and replace
\newcommand{\Lim}[2]{\lim_{\point{#1} \to \point{#2}}} %Bart, I changed this to point since I want to use it.  It runs through both of the exercise and exerciseE files in limits section, which is why it was in each document to start with.

\DeclareMathOperator{\proj}{\mathbf{proj}}
\newcommand{\veci}{{\boldsymbol{\hat{\imath}}}}
\newcommand{\vecj}{{\boldsymbol{\hat{\jmath}}}}
\newcommand{\veck}{{\boldsymbol{\hat{k}}}}
\newcommand{\vecl}{\vec{\boldsymbol{\l}}}
\newcommand{\uvec}[1]{\mathbf{\hat{#1}}}
\newcommand{\utan}{\mathbf{\hat{t}}}
\newcommand{\unormal}{\mathbf{\hat{n}}}
\newcommand{\ubinormal}{\mathbf{\hat{b}}}

\newcommand{\dotp}{\bullet}
\newcommand{\cross}{\boldsymbol\times}
\newcommand{\grad}{\boldsymbol\nabla}
\newcommand{\divergence}{\grad\dotp}
\newcommand{\curl}{\grad\cross}
%\DeclareMathOperator{\divergence}{divergence}
%\DeclareMathOperator{\curl}[1]{\grad\cross #1}
\newcommand{\lto}{\mathop{\longrightarrow\,}\limits}

\renewcommand{\bar}{\overline}

\colorlet{textColor}{black}
\colorlet{background}{white}
\colorlet{penColor}{blue!50!black} % Color of a curve in a plot
\colorlet{penColor2}{red!50!black}% Color of a curve in a plot
\colorlet{penColor3}{red!50!blue} % Color of a curve in a plot
\colorlet{penColor4}{green!50!black} % Color of a curve in a plot
\colorlet{penColor5}{orange!80!black} % Color of a curve in a plot
\colorlet{penColor6}{yellow!70!black} % Color of a curve in a plot
\colorlet{fill1}{penColor!20} % Color of fill in a plot
\colorlet{fill2}{penColor2!20} % Color of fill in a plot
\colorlet{fillp}{fill1} % Color of positive area
\colorlet{filln}{penColor2!20} % Color of negative area
\colorlet{fill3}{penColor3!20} % Fill
\colorlet{fill4}{penColor4!20} % Fill
\colorlet{fill5}{penColor5!20} % Fill
\colorlet{gridColor}{gray!50} % Color of grid in a plot

\newcommand{\surfaceColor}{violet}
\newcommand{\surfaceColorTwo}{redyellow}
\newcommand{\sliceColor}{greenyellow}




\pgfmathdeclarefunction{gauss}{2}{% gives gaussian
  \pgfmathparse{1/(#2*sqrt(2*pi))*exp(-((x-#1)^2)/(2*#2^2))}%
}


%%%%%%%%%%%%%
%% Vectors
%%%%%%%%%%%%%

%% Simple horiz vectors
\renewcommand{\vector}[1]{\left\langle #1\right\rangle}


%% %% Complex Horiz Vectors with angle brackets
%% \makeatletter
%% \renewcommand{\vector}[2][ , ]{\left\langle%
%%   \def\nextitem{\def\nextitem{#1}}%
%%   \@for \el:=#2\do{\nextitem\el}\right\rangle%
%% }
%% \makeatother

%% %% Vertical Vectors
%% \def\vector#1{\begin{bmatrix}\vecListA#1,,\end{bmatrix}}
%% \def\vecListA#1,{\if,#1,\else #1\cr \expandafter \vecListA \fi}

%%%%%%%%%%%%%
%% End of vectors
%%%%%%%%%%%%%

%\newcommand{\fullwidth}{}
%\newcommand{\normalwidth}{}



%% makes a snazzy t-chart for evaluating functions
%\newenvironment{tchart}{\rowcolors{2}{}{background!90!textColor}\array}{\endarray}

%%This is to help with formatting on future title pages.
\newenvironment{sectionOutcomes}{}{}



%% Flowchart stuff
%\tikzstyle{startstop} = [rectangle, rounded corners, minimum width=3cm, minimum height=1cm,text centered, draw=black]
%\tikzstyle{question} = [rectangle, minimum width=3cm, minimum height=1cm, text centered, draw=black]
%\tikzstyle{decision} = [trapezium, trapezium left angle=70, trapezium right angle=110, minimum width=3cm, minimum height=1cm, text centered, draw=black]
%\tikzstyle{question} = [rectangle, rounded corners, minimum width=3cm, minimum height=1cm,text centered, draw=black]
%\tikzstyle{process} = [rectangle, minimum width=3cm, minimum height=1cm, text centered, draw=black]
%\tikzstyle{decision} = [trapezium, trapezium left angle=70, trapezium right angle=110, minimum width=3cm, minimum height=1cm, text centered, draw=black]


\outcome{Use the ratio test to determine if a series diverges or converges.}

\title[Dig-In:]{The ratio test}

\begin{document}
\begin{abstract}
Some infinite series can be compared to geometric series.
\end{abstract}
\maketitle

As mathematicians, we are explorers. We explore the implications of
seemingly simple quantitative facts. 

\begin{exploration}
Consider the infinite series
\[
\sum_{k=0}^\infty \frac{k}{2^k}
\]
Let $a_k = \answer[given]{\frac{k}{2^k}}$ be the sequence of terms of
this series.  When $k$ is large, $a_{k+1}$ is pretty close to
\wordChoice{\choice[correct]{half}\choice{double}} of $a_{k}$.  The
effect of the numerator increasing by $1$ is dwarfed by the effect of
the denominator being doubled.  We can formalize this by looking at
the ratio of consecutive terms:
\[
\lim_{k \to \infty} \frac{a_{k+1}}{a_k} = \answer[given]{\frac{1}{2}}
\]
When we choose a very large whole number $N$,
\[
a_{k+1} \approx a_k\cdot \frac{1}{2}
\]
for $k>N$, and so we get the following approximations:
\begin{align*}
  a_{N+1} &\approx a_N\cdot \frac{1}{2} \\
  a_{N+2} &\approx a_N\cdot \answer[given]{\frac{1}{2^2}} \\
  a_{N+3} &\approx a_N\cdot \answer[given]{\frac{1}{2^3}}\\
\vdots
\end{align*}
In other words, the tail of the sequence $(a_k)$ beginning with $k=N$
is ``approximately'' a geometric series with ratio $\frac{1}{2}$.

\begin{question}
Does a geometric series with ratio $\frac{1}{2}$ converge or diverge?
\begin{prompt}
  \begin{multipleChoice}
    \choice[correct]{converge}
    \choice{diverge}
  \end{multipleChoice}
\end{prompt}
\begin{question}
  Given your answer above, do you suspect that the original sum
  $\sum_{k=0}^\infty \frac{k}{2^k}$ converges or diverges?
  \begin{prompt}
    \begin{multipleChoice}
      \choice[correct]{converge}
      \choice{diverge}
    \end{multipleChoice}
  \end{prompt}
\end{question}
\end{question}
\end{exploration}

The above exploration motivates the following theorem.  The proof of
this theorem is slightly beyond the scope of the course.

\begin{theorem}[The Ratio Test]\index{ratio test}
  Let $\sum_{k=0}^\infty a_k$ be an infinite series with positive terms.  If $r =
  \lim_{k \to \infty} \frac{a_{k+1}}{a_k}$ exists, then we can conclude the following.
  \begin{itemize}
  \item If $0 \leq r < 1$, then the series converges.
  \item If $r>1$, then the series diverges.
  \item If $r = 1$, then we learn nothing:  the series could diverge or converge.
  \end{itemize}
\end{theorem}

Note that this is easy to remember if you just use the following heuristic.
\begin{quote}
  If the ratio test gives a limit of $r$, then the series is like a
  geometric series of ratio $r$.
\end{quote}
The case of $r=1$ is an ``edge'' case, and can go either way.  Now
that you have the basic idea, we give examples showing:
\begin{itemize}
\item The ratio test indicating convergence.
\item The ratio test indicating divergence.
\item The ratio test being inconclusive, but the series actually converges.
\item The ratio test being inconclusive, but the series actually diverges.
\end{itemize}
It is important that examples illustrating the final two behaviors
exist, because it shows that the ratio test really is inconclusive in
the case $r=1$.

\begin{example}
  Consider:
  \[
  \sum_{k=4}^\infty \frac{2^k}{k!}
  \]
  Discuss the convergence of this series.
  \begin{explanation}
    We will attempt to use the ratio test. Setting $a_k = \frac{2^k}{k!}$. Write with me.
    \[
    \lim_{k \to \infty} \frac{a_{k+1}}{a_k} = \answer[given]{0}
    \]
    So, the ratio test
	 \wordChoice{
	   \choice[correct]{says the series is convergent}
	   \choice{says the series is divergent}
	   \choice{gives no information in this case, but we know the series is convergent through some other method}
	   \choice{gives no information in this case, but we know the series is divergent through some other method}}.	
	 \begin{hint}
	   \begin{align*}
	     \lim_{k \to \infty} \frac{a_{k+1}}{a_k} &= \lim_{k \to \infty} \frac{2^{k+1}}{(k+1)!} \frac{k!}{2^k}\\
	     &=\lim_{k \to \infty} \frac{2}{k+1}\\
	     &=0
	   \end{align*}
	   So the series is convergent by the ratio test.  Note that
           this shows that $k!$ grows \textbf{much faster} than the
           exponential function $2^k$.
	 \end{hint}
  \end{explanation}
\end{example}



\begin{example}
  Consider: 
  \[
  \sum_{k=1}^\infty \frac{1}{k}
  \]
  Discuss the convergence of this series.
  \begin{explanation}
    We will attempt to use the ratio test. Setting $a_k =
    \frac{1}{k}$. Write with me.
    \[
    \lim_{k \to \infty} \frac{a_{k+1}}{a_k} = \answer[given]{1}
    \]
    So, the ratio test
	 \wordChoice{
	   \choice{says the series is convergent}
	   \choice{says the series is divergent}
	   \choice{gives no information in this case, but we know the series is convergent using the integral test}
	   \choice[correct]{gives no information in this case, but we know the series is divergent using the integral test}}.
	 \begin{hint}
           \begin{align*}
	     \lim_{k \to \infty} \frac{a_{k+1}}{a_k} &= \lim_{k \to \infty} \frac{k+1}{k}\\
	     &=1
	   \end{align*}
	   So the ratio test gives no information.
	   However, we know that the harmonic series is divergent (we proved this using the integral test).
	 \end{hint}
  \end{explanation}
\end{example}

\begin{example}
  Consider:
  \[
  \sum_{k=1}^\infty \frac{5^{1+k}}{k 2^{2k+1}}
  \]
  Discuss the convergence of this series.
  \begin{explanation}
    We will attempt to use the ratio test. Setting $a_k =
    \frac{5^{1+k}}{k 2^{2k+1}}$.  Write with me.
    \[
    \lim_{k \to \infty} \frac{a_{k+1}}{a_k} = \answer[given]{\frac{5}{4}}
    \]
    So, the ratio test
    \wordChoice{
      \choice{says the series is convergent}
      \choice[correct]{says the series is divergent}
      \choice{gives no information in this case, but we know the series is convergent through some other method}
      \choice{gives no information in this case, but we know the series is divergent through some other method}}. 
    \begin{hint}
      \begin{align*}
	\lim_{k \to \infty} \frac{a_{k+1}}{a_k} &= \lim_{k \to \infty} \frac{5^{k+2}}{(k+1)2^{2(k+1)+1}} \frac{k 2^{2k+1}}{5^{1+k}}\\
	&=\lim_{k \to \infty} \frac{5^{k+2}}{5^{1+k}} \frac{2^{2k+1}}{2^{2k+3}} \frac{k}{k+1}\\
	&=\lim_{k \to \infty} \frac{5}{4} \frac{k}{k+1}\\
	&=\frac{5}{4}
      \end{align*}	
      So the series is divergent by the ratio test.
    \end{hint}
  \end{explanation}
\end{example}


\begin{example}
  Consider:	
  \[
  \sum_{k=1}^\infty \frac{1}{k^2}
  \]
  Discuss the convergence of this series.
  \begin{explanation}
    We will attempt to use the ratio test. Setting $a_k =
    \frac{1}{k^2}$. Write with me.
    \[ 
    \lim_{k \to \infty} \frac{a_{k+1}}{a_k} = \answer[given]{1}	
    \]
    So, the ratio test
    \wordChoice{
      \choice{says the series is convergent}
      \choice{says the series is divergent}
      \choice[correct]{gives no information in this case, but we know the series is convergent through some other method}
      \choice{gives no information in this case, but we know the series is divergent through some other method}}.		
    \begin{hint}
      \begin{align*}
	\lim_{k \to \infty} \frac{a_{k+1}}{a_k} &= \lim_{k \to \infty} \frac{(k+1)^2}{k^2}\\
	&= \lim_{k \to \infty} \left( \frac{k+1}{k}\right)^2\\
	&= \lim_{k \to \infty} \left( 1+\frac{1}{k}\right)^2\\
	&= 1
      \end{align*}
      So the ratio test gives no information.  However, we know that
      this series is convergent by the $p$ series test with $p=2$
      (which ultimately derives from the integral test).
    \end{hint}
  \end{explanation}
\end{example}
\end{document}


