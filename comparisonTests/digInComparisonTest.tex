\documentclass{ximera}

%\usepackage{todonotes}
%\usepackage{mathtools} %% Required for wide table Curl and Greens
%\usepackage{cuted} %% Required for wide table Curl and Greens
\newcommand{\todo}{}

\usepackage{esint} % for \oiint
\ifxake%%https://math.meta.stackexchange.com/questions/9973/how-do-you-render-a-closed-surface-double-integral
\renewcommand{\oiint}{{\large\bigcirc}\kern-1.56em\iint}
\fi


\graphicspath{
  {./}
  {ximeraTutorial/}
  {basicPhilosophy/}
  {functionsOfSeveralVariables/}
  {normalVectors/}
  {lagrangeMultipliers/}
  {vectorFields/}
  {greensTheorem/}
  {shapeOfThingsToCome/}
  {dotProducts/}
  {partialDerivativesAndTheGradientVector/}
  {../productAndQuotientRules/exercises/}
  {../normalVectors/exercisesParametricPlots/}
  {../continuityOfFunctionsOfSeveralVariables/exercises/}
  {../partialDerivativesAndTheGradientVector/exercises/}
  {../directionalDerivativeAndChainRule/exercises/}
  {../commonCoordinates/exercisesCylindricalCoordinates/}
  {../commonCoordinates/exercisesSphericalCoordinates/}
  {../greensTheorem/exercisesCurlAndLineIntegrals/}
  {../greensTheorem/exercisesDivergenceAndLineIntegrals/}
  {../shapeOfThingsToCome/exercisesDivergenceTheorem/}
  {../greensTheorem/}
  {../shapeOfThingsToCome/}
  {../separableDifferentialEquations/exercises/}
  {vectorFields/}
}

\newcommand{\mooculus}{\textsf{\textbf{MOOC}\textnormal{\textsf{ULUS}}}}

\usepackage{tkz-euclide}\usepackage{tikz}
\usepackage{tikz-cd}
\usetikzlibrary{arrows}
\tikzset{>=stealth,commutative diagrams/.cd,
  arrow style=tikz,diagrams={>=stealth}} %% cool arrow head
\tikzset{shorten <>/.style={ shorten >=#1, shorten <=#1 } } %% allows shorter vectors

\usetikzlibrary{backgrounds} %% for boxes around graphs
\usetikzlibrary{shapes,positioning}  %% Clouds and stars
\usetikzlibrary{matrix} %% for matrix
\usepgfplotslibrary{polar} %% for polar plots
\usepgfplotslibrary{fillbetween} %% to shade area between curves in TikZ
\usetkzobj{all}
\usepackage[makeroom]{cancel} %% for strike outs
%\usepackage{mathtools} %% for pretty underbrace % Breaks Ximera
%\usepackage{multicol}
\usepackage{pgffor} %% required for integral for loops



%% http://tex.stackexchange.com/questions/66490/drawing-a-tikz-arc-specifying-the-center
%% Draws beach ball
\tikzset{pics/carc/.style args={#1:#2:#3}{code={\draw[pic actions] (#1:#3) arc(#1:#2:#3);}}}



\usepackage{array}
\setlength{\extrarowheight}{+.1cm}
\newdimen\digitwidth
\settowidth\digitwidth{9}
\def\divrule#1#2{
\noalign{\moveright#1\digitwidth
\vbox{\hrule width#2\digitwidth}}}





\newcommand{\RR}{\mathbb R}
\newcommand{\R}{\mathbb R}
\newcommand{\N}{\mathbb N}
\newcommand{\Z}{\mathbb Z}

\newcommand{\sagemath}{\textsf{SageMath}}


%\renewcommand{\d}{\,d\!}
\renewcommand{\d}{\mathop{}\!d}
\newcommand{\dd}[2][]{\frac{\d #1}{\d #2}}
\newcommand{\pp}[2][]{\frac{\partial #1}{\partial #2}}
\renewcommand{\l}{\ell}
\newcommand{\ddx}{\frac{d}{\d x}}

\newcommand{\zeroOverZero}{\ensuremath{\boldsymbol{\tfrac{0}{0}}}}
\newcommand{\inftyOverInfty}{\ensuremath{\boldsymbol{\tfrac{\infty}{\infty}}}}
\newcommand{\zeroOverInfty}{\ensuremath{\boldsymbol{\tfrac{0}{\infty}}}}
\newcommand{\zeroTimesInfty}{\ensuremath{\small\boldsymbol{0\cdot \infty}}}
\newcommand{\inftyMinusInfty}{\ensuremath{\small\boldsymbol{\infty - \infty}}}
\newcommand{\oneToInfty}{\ensuremath{\boldsymbol{1^\infty}}}
\newcommand{\zeroToZero}{\ensuremath{\boldsymbol{0^0}}}
\newcommand{\inftyToZero}{\ensuremath{\boldsymbol{\infty^0}}}



\newcommand{\numOverZero}{\ensuremath{\boldsymbol{\tfrac{\#}{0}}}}
\newcommand{\dfn}{\textbf}
%\newcommand{\unit}{\,\mathrm}
\newcommand{\unit}{\mathop{}\!\mathrm}
\newcommand{\eval}[1]{\bigg[ #1 \bigg]}
\newcommand{\seq}[1]{\left( #1 \right)}
\renewcommand{\epsilon}{\varepsilon}
\renewcommand{\phi}{\varphi}


\renewcommand{\iff}{\Leftrightarrow}

\DeclareMathOperator{\arccot}{arccot}
\DeclareMathOperator{\arcsec}{arcsec}
\DeclareMathOperator{\arccsc}{arccsc}
\DeclareMathOperator{\si}{Si}
\DeclareMathOperator{\scal}{scal}
\DeclareMathOperator{\sign}{sign}


%% \newcommand{\tightoverset}[2]{% for arrow vec
%%   \mathop{#2}\limits^{\vbox to -.5ex{\kern-0.75ex\hbox{$#1$}\vss}}}
\newcommand{\arrowvec}[1]{{\overset{\rightharpoonup}{#1}}}
%\renewcommand{\vec}[1]{\arrowvec{\mathbf{#1}}}
\renewcommand{\vec}[1]{{\overset{\boldsymbol{\rightharpoonup}}{\mathbf{#1}}}\hspace{0in}}

\newcommand{\point}[1]{\left(#1\right)} %this allows \vector{ to be changed to \vector{ with a quick find and replace
\newcommand{\pt}[1]{\mathbf{#1}} %this allows \vec{ to be changed to \vec{ with a quick find and replace
\newcommand{\Lim}[2]{\lim_{\point{#1} \to \point{#2}}} %Bart, I changed this to point since I want to use it.  It runs through both of the exercise and exerciseE files in limits section, which is why it was in each document to start with.

\DeclareMathOperator{\proj}{\mathbf{proj}}
\newcommand{\veci}{{\boldsymbol{\hat{\imath}}}}
\newcommand{\vecj}{{\boldsymbol{\hat{\jmath}}}}
\newcommand{\veck}{{\boldsymbol{\hat{k}}}}
\newcommand{\vecl}{\vec{\boldsymbol{\l}}}
\newcommand{\uvec}[1]{\mathbf{\hat{#1}}}
\newcommand{\utan}{\mathbf{\hat{t}}}
\newcommand{\unormal}{\mathbf{\hat{n}}}
\newcommand{\ubinormal}{\mathbf{\hat{b}}}

\newcommand{\dotp}{\bullet}
\newcommand{\cross}{\boldsymbol\times}
\newcommand{\grad}{\boldsymbol\nabla}
\newcommand{\divergence}{\grad\dotp}
\newcommand{\curl}{\grad\cross}
%\DeclareMathOperator{\divergence}{divergence}
%\DeclareMathOperator{\curl}[1]{\grad\cross #1}
\newcommand{\lto}{\mathop{\longrightarrow\,}\limits}

\renewcommand{\bar}{\overline}

\colorlet{textColor}{black}
\colorlet{background}{white}
\colorlet{penColor}{blue!50!black} % Color of a curve in a plot
\colorlet{penColor2}{red!50!black}% Color of a curve in a plot
\colorlet{penColor3}{red!50!blue} % Color of a curve in a plot
\colorlet{penColor4}{green!50!black} % Color of a curve in a plot
\colorlet{penColor5}{orange!80!black} % Color of a curve in a plot
\colorlet{penColor6}{yellow!70!black} % Color of a curve in a plot
\colorlet{fill1}{penColor!20} % Color of fill in a plot
\colorlet{fill2}{penColor2!20} % Color of fill in a plot
\colorlet{fillp}{fill1} % Color of positive area
\colorlet{filln}{penColor2!20} % Color of negative area
\colorlet{fill3}{penColor3!20} % Fill
\colorlet{fill4}{penColor4!20} % Fill
\colorlet{fill5}{penColor5!20} % Fill
\colorlet{gridColor}{gray!50} % Color of grid in a plot

\newcommand{\surfaceColor}{violet}
\newcommand{\surfaceColorTwo}{redyellow}
\newcommand{\sliceColor}{greenyellow}




\pgfmathdeclarefunction{gauss}{2}{% gives gaussian
  \pgfmathparse{1/(#2*sqrt(2*pi))*exp(-((x-#1)^2)/(2*#2^2))}%
}


%%%%%%%%%%%%%
%% Vectors
%%%%%%%%%%%%%

%% Simple horiz vectors
\renewcommand{\vector}[1]{\left\langle #1\right\rangle}


%% %% Complex Horiz Vectors with angle brackets
%% \makeatletter
%% \renewcommand{\vector}[2][ , ]{\left\langle%
%%   \def\nextitem{\def\nextitem{#1}}%
%%   \@for \el:=#2\do{\nextitem\el}\right\rangle%
%% }
%% \makeatother

%% %% Vertical Vectors
%% \def\vector#1{\begin{bmatrix}\vecListA#1,,\end{bmatrix}}
%% \def\vecListA#1,{\if,#1,\else #1\cr \expandafter \vecListA \fi}

%%%%%%%%%%%%%
%% End of vectors
%%%%%%%%%%%%%

%\newcommand{\fullwidth}{}
%\newcommand{\normalwidth}{}



%% makes a snazzy t-chart for evaluating functions
%\newenvironment{tchart}{\rowcolors{2}{}{background!90!textColor}\array}{\endarray}

%%This is to help with formatting on future title pages.
\newenvironment{sectionOutcomes}{}{}



%% Flowchart stuff
%\tikzstyle{startstop} = [rectangle, rounded corners, minimum width=3cm, minimum height=1cm,text centered, draw=black]
%\tikzstyle{question} = [rectangle, minimum width=3cm, minimum height=1cm, text centered, draw=black]
%\tikzstyle{decision} = [trapezium, trapezium left angle=70, trapezium right angle=110, minimum width=3cm, minimum height=1cm, text centered, draw=black]
%\tikzstyle{question} = [rectangle, rounded corners, minimum width=3cm, minimum height=1cm,text centered, draw=black]
%\tikzstyle{process} = [rectangle, minimum width=3cm, minimum height=1cm, text centered, draw=black]
%\tikzstyle{decision} = [trapezium, trapezium left angle=70, trapezium right angle=110, minimum width=3cm, minimum height=1cm, text centered, draw=black]


\outcome{Use the comparison test to determine if a series diverges or converges.}

\title[Dig-In:]{The comparison test}

\begin{document}
\begin{abstract}
We compare infinite series to each other using inequalities.
\end{abstract}
\maketitle

\begin{question}
  If $0 \leq a_k \leq b_k$ for all $k$, which of the following do you
  think are true?
  \begin{selectAll}
    \choice[correct]{If $\sum^\infty b_k$ converges then $\sum^\infty a_k$ converges}
    \choice{If $\sum^\infty a_k$ converges then $\sum^\infty b_k$ converges}
    \choice{If $\sum^\infty b_k$ diverges then $\sum^\infty a_k$ diverges}
    \choice[correct]{If $\sum^\infty a_k$ diverges then $\sum^\infty b_k$ diverges}
  \end{selectAll}
\end{question}

If you got this question right, you \textbf{already} understand the
\textit{comparison test}!

\begin{theorem}[The Comparison Test]\index{comparison test}
  Let $\sum_{k=0}^\infty a_k$ and $\sum_{k=0}^\infty b_k$ be series with positive
  terms:
  \begin{itemize}
  \item If $a_k \leq b_k$ and $\sum a_k$ is divergent, then
    $\sum b_k$ is divergent.
  \item If $a_k \leq b_k$ and $\sum b_k$ is convergent, then
    $\sum a_k$ is convergent.
  \end{itemize}
\end{theorem}

Notice that this test, just like the root and ratio tests, require us to have 
positive terms in our series.  Of course, what we really mean by 
positive terms is that a series should eventually have only positive terms. 
As always, a finite number of negative terms doesn't affect the overall 
convergence or divergence of a series.

The proof of this theorem is beyond this course, but it should make
intuitive sense.
\begin{itemize}
  \item Making the terms of a convergent series smaller should result
    in another convergent series.
  \item Making the terms of divergent series larger should result in
      another divergent series.
\end{itemize}

While this theorem is intuitive, its use involves considerable
creativity.  You have to:
\begin{enumerate}
\item Find a simpler series which ``behaves like'' your series.
\item Use this simpler series to predict whether the original series
  converges or diverges.
\item If you predict your series is convergent, you have to hunt for a
  simpler series which is also convergent, but all of whose terms are
  larger.
\item If you predict your series is divergent, you have to hunt for a
  simpler series which is also divergent, but all of whose terms are
  smaller.
\end{enumerate}

All of these steps (aside from the second) require real insight and
creativity.  This is not a ``mechanical'' test.

\begin{example}
Is $\sum_{k=1}^\infty \frac{k}{7+k^3}$ convergent or divergent?
Justify your answer using the comparison test.
\begin{explanation}
  Intuitively, we should feel that this should be similar to the
  series
  \[
  \sum_{k=1}^\infty \frac{k}{7+ k^3} \approx \sum_{k=1}^\infty \frac{k}{0+k^3}
  \]
  since the additional term of $\answer[given]{7}$ in the denominator
  should not matter when $k$ is large. Since
  \[
  \sum_{k=1}^\infty \frac{k}{k^3} =\sum_{k=1}^\infty \frac{1}{k^2}
  \]
  we see that our original sum
  \[
  \sum_{k=1}^\infty \frac{k}{7+ k^3} \approx \sum_{k=1}^\infty \frac{1}{k^2}.
  \]
  We know $\sum_{k=1}^\infty \frac{1}{k^2}$ is convergent by the $p$-series
  test (which ultimately comes from the integral test), so we guess that our 
  series in question is also convergent.  Since we want
  to prove convergence, we need find a series whose terms are always
  \textbf{larger} than $\frac{k}{7+ k^3}$. Note
  \[
  \frac{k}{7+k^3} < \frac{k}{0+k^3}
  \]
  since we have made the denominator
  \wordChoice{\choice{larger}\choice[correct]{smaller}}, which makes
  the fraction \wordChoice{\choice[correct]{larger}\choice{smaller}}.
  Thus $\frac{k}{7+k^3} < \frac{1}{k^2}$.  Since $\sum_{k=1}^\infty
  \frac{1}{k^2}$ converges, then our original series also converges by
  the comparison test.
\end{explanation}
\end{example}





\begin{example}
Is $\sum_{k=1}^\infty \frac{k^2}{5+k^3}$ convergent or divergent?
Justify your answer using the comparison Test.
\begin{explanation}
  Intuitively, we should feel that this should be similar to the
  series
  \[
  \sum_{k=1}^\infty \frac{k^2}{5+k^3} \approx \sum_{k=1}^\infty \frac{k^2}{0+k^3}
  \]
  since the additional term of $\answer[given]{5}$ in the denominator should not
  matter when $k$ is large. Since
  \[
  \sum_{k=1}^\infty \frac{k^2}{k^3} =\sum_{k=1}^\infty \frac{1}{k}
  \]
  we see that our original sum
  \[
  \sum_{k=1}^\infty \frac{k^2}{5+ k^3} \approx \sum_{k=1}^\infty \frac{1}{k}.
  \]
  We know $\sum^\infty \frac{1}{k}$ is divergent by the $p$-series
  test (which ultimately comes from the integral test), so we guess that our series 
  in question is divergent.  Since we want
  to prove divergence, we need to find a series whose terms are
  always \textbf{smaller} than $\frac{k^2}{5+ k^3}$.  We can't use $\frac{1}{k}$, because 
  those terms are larger.  Hence, \textbf{this
    requires some creativity.} Write with me.
  \[
  \frac{k^2}{5+k^3} \geq \frac{k^2}{5k^3+k^3} 
  \]
  since we have made the denominator
  \wordChoice{\choice[correct]{larger}\choice{smaller}}, which makes
  the fraction \wordChoice{\choice{larger}\choice[correct]{smaller}},
  note that $k^3 \geq 1$ whenever $k \geq 1$.  Thus $\frac{k}{1+k^3}
  \geq \frac{1}{6k}$.  Since
  \[
  \sum_{k=1}^\infty\frac{1}{6k} = \frac{1}{6} \sum_{k=1}^\infty
  \frac{1}{k}
  \]
  diverges, then our original series also diverges by the comparison
  test.
\end{explanation}
\end{example}


\end{document}


