\documentclass{ximera}

%\usepackage{todonotes}
%\usepackage{mathtools} %% Required for wide table Curl and Greens
%\usepackage{cuted} %% Required for wide table Curl and Greens
\newcommand{\todo}{}

\usepackage{esint} % for \oiint
\ifxake%%https://math.meta.stackexchange.com/questions/9973/how-do-you-render-a-closed-surface-double-integral
\renewcommand{\oiint}{{\large\bigcirc}\kern-1.56em\iint}
\fi


\graphicspath{
  {./}
  {ximeraTutorial/}
  {basicPhilosophy/}
  {functionsOfSeveralVariables/}
  {normalVectors/}
  {lagrangeMultipliers/}
  {vectorFields/}
  {greensTheorem/}
  {shapeOfThingsToCome/}
  {dotProducts/}
  {partialDerivativesAndTheGradientVector/}
  {../productAndQuotientRules/exercises/}
  {../normalVectors/exercisesParametricPlots/}
  {../continuityOfFunctionsOfSeveralVariables/exercises/}
  {../partialDerivativesAndTheGradientVector/exercises/}
  {../directionalDerivativeAndChainRule/exercises/}
  {../commonCoordinates/exercisesCylindricalCoordinates/}
  {../commonCoordinates/exercisesSphericalCoordinates/}
  {../greensTheorem/exercisesCurlAndLineIntegrals/}
  {../greensTheorem/exercisesDivergenceAndLineIntegrals/}
  {../shapeOfThingsToCome/exercisesDivergenceTheorem/}
  {../greensTheorem/}
  {../shapeOfThingsToCome/}
  {../separableDifferentialEquations/exercises/}
  {vectorFields/}
}

\newcommand{\mooculus}{\textsf{\textbf{MOOC}\textnormal{\textsf{ULUS}}}}

\usepackage{tkz-euclide}\usepackage{tikz}
\usepackage{tikz-cd}
\usetikzlibrary{arrows}
\tikzset{>=stealth,commutative diagrams/.cd,
  arrow style=tikz,diagrams={>=stealth}} %% cool arrow head
\tikzset{shorten <>/.style={ shorten >=#1, shorten <=#1 } } %% allows shorter vectors

\usetikzlibrary{backgrounds} %% for boxes around graphs
\usetikzlibrary{shapes,positioning}  %% Clouds and stars
\usetikzlibrary{matrix} %% for matrix
\usepgfplotslibrary{polar} %% for polar plots
\usepgfplotslibrary{fillbetween} %% to shade area between curves in TikZ
\usetkzobj{all}
\usepackage[makeroom]{cancel} %% for strike outs
%\usepackage{mathtools} %% for pretty underbrace % Breaks Ximera
%\usepackage{multicol}
\usepackage{pgffor} %% required for integral for loops



%% http://tex.stackexchange.com/questions/66490/drawing-a-tikz-arc-specifying-the-center
%% Draws beach ball
\tikzset{pics/carc/.style args={#1:#2:#3}{code={\draw[pic actions] (#1:#3) arc(#1:#2:#3);}}}



\usepackage{array}
\setlength{\extrarowheight}{+.1cm}
\newdimen\digitwidth
\settowidth\digitwidth{9}
\def\divrule#1#2{
\noalign{\moveright#1\digitwidth
\vbox{\hrule width#2\digitwidth}}}





\newcommand{\RR}{\mathbb R}
\newcommand{\R}{\mathbb R}
\newcommand{\N}{\mathbb N}
\newcommand{\Z}{\mathbb Z}

\newcommand{\sagemath}{\textsf{SageMath}}


%\renewcommand{\d}{\,d\!}
\renewcommand{\d}{\mathop{}\!d}
\newcommand{\dd}[2][]{\frac{\d #1}{\d #2}}
\newcommand{\pp}[2][]{\frac{\partial #1}{\partial #2}}
\renewcommand{\l}{\ell}
\newcommand{\ddx}{\frac{d}{\d x}}

\newcommand{\zeroOverZero}{\ensuremath{\boldsymbol{\tfrac{0}{0}}}}
\newcommand{\inftyOverInfty}{\ensuremath{\boldsymbol{\tfrac{\infty}{\infty}}}}
\newcommand{\zeroOverInfty}{\ensuremath{\boldsymbol{\tfrac{0}{\infty}}}}
\newcommand{\zeroTimesInfty}{\ensuremath{\small\boldsymbol{0\cdot \infty}}}
\newcommand{\inftyMinusInfty}{\ensuremath{\small\boldsymbol{\infty - \infty}}}
\newcommand{\oneToInfty}{\ensuremath{\boldsymbol{1^\infty}}}
\newcommand{\zeroToZero}{\ensuremath{\boldsymbol{0^0}}}
\newcommand{\inftyToZero}{\ensuremath{\boldsymbol{\infty^0}}}



\newcommand{\numOverZero}{\ensuremath{\boldsymbol{\tfrac{\#}{0}}}}
\newcommand{\dfn}{\textbf}
%\newcommand{\unit}{\,\mathrm}
\newcommand{\unit}{\mathop{}\!\mathrm}
\newcommand{\eval}[1]{\bigg[ #1 \bigg]}
\newcommand{\seq}[1]{\left( #1 \right)}
\renewcommand{\epsilon}{\varepsilon}
\renewcommand{\phi}{\varphi}


\renewcommand{\iff}{\Leftrightarrow}

\DeclareMathOperator{\arccot}{arccot}
\DeclareMathOperator{\arcsec}{arcsec}
\DeclareMathOperator{\arccsc}{arccsc}
\DeclareMathOperator{\si}{Si}
\DeclareMathOperator{\scal}{scal}
\DeclareMathOperator{\sign}{sign}


%% \newcommand{\tightoverset}[2]{% for arrow vec
%%   \mathop{#2}\limits^{\vbox to -.5ex{\kern-0.75ex\hbox{$#1$}\vss}}}
\newcommand{\arrowvec}[1]{{\overset{\rightharpoonup}{#1}}}
%\renewcommand{\vec}[1]{\arrowvec{\mathbf{#1}}}
\renewcommand{\vec}[1]{{\overset{\boldsymbol{\rightharpoonup}}{\mathbf{#1}}}\hspace{0in}}

\newcommand{\point}[1]{\left(#1\right)} %this allows \vector{ to be changed to \vector{ with a quick find and replace
\newcommand{\pt}[1]{\mathbf{#1}} %this allows \vec{ to be changed to \vec{ with a quick find and replace
\newcommand{\Lim}[2]{\lim_{\point{#1} \to \point{#2}}} %Bart, I changed this to point since I want to use it.  It runs through both of the exercise and exerciseE files in limits section, which is why it was in each document to start with.

\DeclareMathOperator{\proj}{\mathbf{proj}}
\newcommand{\veci}{{\boldsymbol{\hat{\imath}}}}
\newcommand{\vecj}{{\boldsymbol{\hat{\jmath}}}}
\newcommand{\veck}{{\boldsymbol{\hat{k}}}}
\newcommand{\vecl}{\vec{\boldsymbol{\l}}}
\newcommand{\uvec}[1]{\mathbf{\hat{#1}}}
\newcommand{\utan}{\mathbf{\hat{t}}}
\newcommand{\unormal}{\mathbf{\hat{n}}}
\newcommand{\ubinormal}{\mathbf{\hat{b}}}

\newcommand{\dotp}{\bullet}
\newcommand{\cross}{\boldsymbol\times}
\newcommand{\grad}{\boldsymbol\nabla}
\newcommand{\divergence}{\grad\dotp}
\newcommand{\curl}{\grad\cross}
%\DeclareMathOperator{\divergence}{divergence}
%\DeclareMathOperator{\curl}[1]{\grad\cross #1}
\newcommand{\lto}{\mathop{\longrightarrow\,}\limits}

\renewcommand{\bar}{\overline}

\colorlet{textColor}{black}
\colorlet{background}{white}
\colorlet{penColor}{blue!50!black} % Color of a curve in a plot
\colorlet{penColor2}{red!50!black}% Color of a curve in a plot
\colorlet{penColor3}{red!50!blue} % Color of a curve in a plot
\colorlet{penColor4}{green!50!black} % Color of a curve in a plot
\colorlet{penColor5}{orange!80!black} % Color of a curve in a plot
\colorlet{penColor6}{yellow!70!black} % Color of a curve in a plot
\colorlet{fill1}{penColor!20} % Color of fill in a plot
\colorlet{fill2}{penColor2!20} % Color of fill in a plot
\colorlet{fillp}{fill1} % Color of positive area
\colorlet{filln}{penColor2!20} % Color of negative area
\colorlet{fill3}{penColor3!20} % Fill
\colorlet{fill4}{penColor4!20} % Fill
\colorlet{fill5}{penColor5!20} % Fill
\colorlet{gridColor}{gray!50} % Color of grid in a plot

\newcommand{\surfaceColor}{violet}
\newcommand{\surfaceColorTwo}{redyellow}
\newcommand{\sliceColor}{greenyellow}




\pgfmathdeclarefunction{gauss}{2}{% gives gaussian
  \pgfmathparse{1/(#2*sqrt(2*pi))*exp(-((x-#1)^2)/(2*#2^2))}%
}


%%%%%%%%%%%%%
%% Vectors
%%%%%%%%%%%%%

%% Simple horiz vectors
\renewcommand{\vector}[1]{\left\langle #1\right\rangle}


%% %% Complex Horiz Vectors with angle brackets
%% \makeatletter
%% \renewcommand{\vector}[2][ , ]{\left\langle%
%%   \def\nextitem{\def\nextitem{#1}}%
%%   \@for \el:=#2\do{\nextitem\el}\right\rangle%
%% }
%% \makeatother

%% %% Vertical Vectors
%% \def\vector#1{\begin{bmatrix}\vecListA#1,,\end{bmatrix}}
%% \def\vecListA#1,{\if,#1,\else #1\cr \expandafter \vecListA \fi}

%%%%%%%%%%%%%
%% End of vectors
%%%%%%%%%%%%%

%\newcommand{\fullwidth}{}
%\newcommand{\normalwidth}{}



%% makes a snazzy t-chart for evaluating functions
%\newenvironment{tchart}{\rowcolors{2}{}{background!90!textColor}\array}{\endarray}

%%This is to help with formatting on future title pages.
\newenvironment{sectionOutcomes}{}{}



%% Flowchart stuff
%\tikzstyle{startstop} = [rectangle, rounded corners, minimum width=3cm, minimum height=1cm,text centered, draw=black]
%\tikzstyle{question} = [rectangle, minimum width=3cm, minimum height=1cm, text centered, draw=black]
%\tikzstyle{decision} = [trapezium, trapezium left angle=70, trapezium right angle=110, minimum width=3cm, minimum height=1cm, text centered, draw=black]
%\tikzstyle{question} = [rectangle, rounded corners, minimum width=3cm, minimum height=1cm,text centered, draw=black]
%\tikzstyle{process} = [rectangle, minimum width=3cm, minimum height=1cm, text centered, draw=black]
%\tikzstyle{decision} = [trapezium, trapezium left angle=70, trapezium right angle=110, minimum width=3cm, minimum height=1cm, text centered, draw=black]


\title[Dig-In:]{Divergence theorem}

\outcome{State and use the divergence theorem.}

\begin{document}
\begin{abstract}
  We introduce the divergence theorem.
\end{abstract}
\maketitle




\section{The divergence theorem}

The divergence theorem states that certain volume integrals are equal
to certain surface integrals. Let's see the statement.

\begin{theorem}[Divergence Theorem]\index{divergence theorem}
  Suppose that the components of $\vec{F}:\R^3\to\R^3$ have continuous
  partial derivatives. If $R$ is a solid bounded by a surface
  $\partial R$ oriented with the normal vectors pointing outside, then:
  \[
  \iiint_R \divergence \vec{F}  \d V =   \oiint_{\partial R} \vec{F}\dotp\uvec{n}\d S
  \]
\end{theorem}

Integrals of the type above arise any time we wish to understand
``fluid flow'' through a surface.  The ``fluid'' in question could be
a real fluid like air or water, or it could be an electromagnetic
field, or something else entirely. Unfortunately, many of the ``real''
applications of the divergence theorem require a deeper understanding
of the specific context where the integral arises. For our part, we
will focus on using the divergence theorem as a tool for transforming
one integral into another (hopefully easier!) integral.


\begin{example}
  Let
  \[
  R = \{(x,y,z):-2\le x\le1, 1\le y\le 3, -5\le z\le 1\}
  \]
  and
  \[
  \vec{F} = \vector{\sin(z),y^2,e^x},
  \]
  compute:
  \[
  \oiint_{\partial R} \vec{F}\dotp \uvec{n} \d S
  \]
  \begin{explanation}
    To compute this integral, we'll use the divergence theorem. Since
    our region is a box, the limits of the triple integral will be
    easy to work with. Moreover, computing the divergence of $\vec{F}$ we see
    \[
    \divergence \vec{F}(x,y,z) = \answer[given]{2y}.
    \]
    So by the divergence theorem, we have
    \begin{align*}
      \oiint_{\partial R} \vec{F}\dotp \uvec{n} \d S &= \iiint_R \divergence \vec{F}  \d V \\
      &= \int_{-2}^{\answer[given]{1}} \int_{-5}^{\answer[given]{1}} \int_{1}^{\answer[given]{3}}
      \answer[given]{2y} \d \answer[given]{y}\d \answer[given]{z}\d \answer[given]{x}\\
      &= \answer[given]{144}.
    \end{align*}
  \end{explanation}
\end{example}

Now let's see another example:



\begin{example}
  Let
  \begin{align*}
  R = \{(x,y,z):&0\le x\le1, \\
  &0\le y\le 2-2x, \\
  &0\le z\le 3-3x-3y/2\}
  \end{align*}
  and
  \[
  \vec{F} = \vector{x,y,z},
  \]
  compute:
  \[
  \oiint_{\partial R} \vec{F}\dotp \uvec{n} \d S
  \]
  \begin{explanation}
    To compute this integral, we'll use the divergence theorem. This
    time our region is a tetrahedron, we'll work in cartesian
    coordinates. Moreover, computing the divergence of $\vec{F}$ we
    see
    \[
    \divergence \vec{F}(x,y,z) = \answer[given]{3}.
    \]
    So by the divergence theorem, we have
    \begin{align*}
      \oiint_{\partial R} \vec{F}\dotp \uvec{n} \d S &= \iiint_R \divergence \vec{F}  \d V \\
      &= \int_{\answer[given]{0}}^{\answer[given]{1}} \int_{\answer[given]{0}}^{\answer[given]{2-2x}} \int_{\answer[given]{0}}^{\answer[given]{3-3x-3y/2}}
      \answer[given]{3} \d z \d y \d x\\
      &= \answer[given]{3}.
    \end{align*}
  \end{explanation}
\end{example}

With our next two examples, we cannot help but flex our mathematical muscles a bit:


\begin{example}
  Let
  \begin{align*}
    R = \{(x,y,z):&-1\le x\le1, \\
    &1\le y\le 1,\\
    &0\le z\le \sqrt{1-x^2-y^2}\}
  \end{align*}
  and
  \[
  \vec{F} = \vector{x,2y,3z},
  \]
  compute:
  \[
  \oiint_{\partial R} \vec{F}\dotp \uvec{n} \d S
  \]
  \begin{explanation}
    To compute this integral, we'll use the divergence theorem.
    Computing the divergence of $\vec{F}$ we see
    \[
    \divergence \vec{F}(x,y,z) = \answer[given]{6}.
    \]
    So by the divergence theorem, we have
    \begin{align*}
      \oiint_{\partial R} \vec{F}\dotp \uvec{n} \d S &= \iiint_R \divergence \vec{F}  \d V \\
      &= \iiint_R \answer[given]{6}  \d V \\
      &= \answer[given]{6} \iiint_R \d V
    \end{align*}
    But $\iiint_R \d V$ is just the volume of the hemisphere of radius
    $1$, which we know is $\answer[given]{2\pi/3}$. Hence:
    \[
    \oiint_{\partial R} \vec{F}\dotp \uvec{n} \d S = \answer[given]{4\pi}
    \]
  \end{explanation}
\end{example}


Above, we used the fact that we \textit{know} that the volume of a
sphere is $4\pi r^3/3$. When you know the volume that's great! If not
you have to compute the integral.

\begin{example}
  Let $R$ be the cone whose base is a disk of radius $2$ in the plane
  $z=1$ and whose vertex is a the origin. Compute the flux of
  \[
  \vec{F} = \vector{x-\sin(y), 2y +e^x,  \cos(xy)-3z}
  \]
  across the boundary of $R$.
  \begin{explanation}
    Again, we will use the divergence theorem. Computing the
    divergence of $\vec{F}$ we see:
    \[
    \divergence\vec{F} = \answer[given]{-1}
    \]
    Now write with me
    \begin{align*}
      \oiint_{\partial R} \vec{F}\dotp \uvec{n} \d S &= \iiint_R \divergence \vec{F}  \d V \\
      &= \iiint_R \answer[given]{-1} \d V \\
      &= \answer[given]{-1} \iiint_R \d V
    \end{align*}
    If you know the volume of the cone described above, you can be
    done! If not, do not despair, we'll simply use cylindrical
    coordinates to compute it. Write with me:
    \begin{align*}
    \iiint_R \d V &= \int_{\answer[given]{0}}^{\answer[given]{2\pi}}
    \int_{\answer[given]{0}}^{\answer[given]{2}}
    \int_{\answer[given]{r/2}}^{\answer[given]{1}} r \d z \d r \d\theta\\
    &=\int_{\answer[given]{0}}^{\answer[given]{2\pi}}
    \int_{\answer[given]{0}}^{\answer[given]{2}} \answer[given]{r - r^2/2}\d r \d\theta\\
    &=\int_{\answer[given]{0}}^{\answer[given]{2\pi}} \answer[given]{2/3}  \d\theta\\
    &=\answer[given]{4\pi/3}
    \end{align*}
    Hence:
    \[
    \oiint_{\partial R} \vec{F}\dotp \uvec{n} \d S = \answer[given]{-4\pi/3}
    \]
  \end{explanation}
\end{example}












\section{A new fundamental theorem of calculus}

How is the divergence theorem a fundamental theorem of calculus? Well
consider this:
\begin{image}
  \begin{tikzpicture}
    \draw[ultra thick, gray!50!black] plot [smooth cycle] coordinates {(-1.5,2) (.5,1) (1.5,2) (.5,3) (-1.5,3)};
    \shade[ball color=gray!50!white] plot [smooth cycle] coordinates {(-1.5,2) (.5,1) (1.5,2) (.5,3) (-1.5,3)};
    \node[inner sep=0pt] at (0,0) {$\iiint_R \divergence\vec{F}\d V\quad =\quad \oiint_{\partial R} \vec{F}\dotp\uvec{n}\d S$};

    \node at (-1.7,-.7) {$\underbrace{\hspace{7em}}$};

    \node at (1.7,-.7) {$\underbrace{\hspace{6.5em}}$};

    \node[below,inner sep=0pt,text width=4cm,scale=.5] at (-1.7,-1)
         {To compute the triple integral of $\divergence \vec{F}$ over a
           solid $V\subseteq\R^3$, };

    \node[below,inner sep=0pt,text width=4cm,scale=.5] at (1.8,-1) {we can compute the accumulation of $\vec{F}$ across a boundary surface $\partial R$.};
  \end{tikzpicture}
\end{image}


Are there more fundamental theorems of calculus? Absolutely, and we're
ready for the last one of this course. Read on young mathematician!



\end{document}
