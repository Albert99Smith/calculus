\documentclass{ximera}

%\usepackage{todonotes}
%\usepackage{mathtools} %% Required for wide table Curl and Greens
%\usepackage{cuted} %% Required for wide table Curl and Greens
\newcommand{\todo}{}

\usepackage{esint} % for \oiint
\ifxake%%https://math.meta.stackexchange.com/questions/9973/how-do-you-render-a-closed-surface-double-integral
\renewcommand{\oiint}{{\large\bigcirc}\kern-1.56em\iint}
\fi


\graphicspath{
  {./}
  {ximeraTutorial/}
  {basicPhilosophy/}
  {functionsOfSeveralVariables/}
  {normalVectors/}
  {lagrangeMultipliers/}
  {vectorFields/}
  {greensTheorem/}
  {shapeOfThingsToCome/}
  {dotProducts/}
  {partialDerivativesAndTheGradientVector/}
  {../productAndQuotientRules/exercises/}
  {../normalVectors/exercisesParametricPlots/}
  {../continuityOfFunctionsOfSeveralVariables/exercises/}
  {../partialDerivativesAndTheGradientVector/exercises/}
  {../directionalDerivativeAndChainRule/exercises/}
  {../commonCoordinates/exercisesCylindricalCoordinates/}
  {../commonCoordinates/exercisesSphericalCoordinates/}
  {../greensTheorem/exercisesCurlAndLineIntegrals/}
  {../greensTheorem/exercisesDivergenceAndLineIntegrals/}
  {../shapeOfThingsToCome/exercisesDivergenceTheorem/}
  {../greensTheorem/}
  {../shapeOfThingsToCome/}
  {../separableDifferentialEquations/exercises/}
  {vectorFields/}
}

\newcommand{\mooculus}{\textsf{\textbf{MOOC}\textnormal{\textsf{ULUS}}}}

\usepackage{tkz-euclide}\usepackage{tikz}
\usepackage{tikz-cd}
\usetikzlibrary{arrows}
\tikzset{>=stealth,commutative diagrams/.cd,
  arrow style=tikz,diagrams={>=stealth}} %% cool arrow head
\tikzset{shorten <>/.style={ shorten >=#1, shorten <=#1 } } %% allows shorter vectors

\usetikzlibrary{backgrounds} %% for boxes around graphs
\usetikzlibrary{shapes,positioning}  %% Clouds and stars
\usetikzlibrary{matrix} %% for matrix
\usepgfplotslibrary{polar} %% for polar plots
\usepgfplotslibrary{fillbetween} %% to shade area between curves in TikZ
\usetkzobj{all}
\usepackage[makeroom]{cancel} %% for strike outs
%\usepackage{mathtools} %% for pretty underbrace % Breaks Ximera
%\usepackage{multicol}
\usepackage{pgffor} %% required for integral for loops



%% http://tex.stackexchange.com/questions/66490/drawing-a-tikz-arc-specifying-the-center
%% Draws beach ball
\tikzset{pics/carc/.style args={#1:#2:#3}{code={\draw[pic actions] (#1:#3) arc(#1:#2:#3);}}}



\usepackage{array}
\setlength{\extrarowheight}{+.1cm}
\newdimen\digitwidth
\settowidth\digitwidth{9}
\def\divrule#1#2{
\noalign{\moveright#1\digitwidth
\vbox{\hrule width#2\digitwidth}}}





\newcommand{\RR}{\mathbb R}
\newcommand{\R}{\mathbb R}
\newcommand{\N}{\mathbb N}
\newcommand{\Z}{\mathbb Z}

\newcommand{\sagemath}{\textsf{SageMath}}


%\renewcommand{\d}{\,d\!}
\renewcommand{\d}{\mathop{}\!d}
\newcommand{\dd}[2][]{\frac{\d #1}{\d #2}}
\newcommand{\pp}[2][]{\frac{\partial #1}{\partial #2}}
\renewcommand{\l}{\ell}
\newcommand{\ddx}{\frac{d}{\d x}}

\newcommand{\zeroOverZero}{\ensuremath{\boldsymbol{\tfrac{0}{0}}}}
\newcommand{\inftyOverInfty}{\ensuremath{\boldsymbol{\tfrac{\infty}{\infty}}}}
\newcommand{\zeroOverInfty}{\ensuremath{\boldsymbol{\tfrac{0}{\infty}}}}
\newcommand{\zeroTimesInfty}{\ensuremath{\small\boldsymbol{0\cdot \infty}}}
\newcommand{\inftyMinusInfty}{\ensuremath{\small\boldsymbol{\infty - \infty}}}
\newcommand{\oneToInfty}{\ensuremath{\boldsymbol{1^\infty}}}
\newcommand{\zeroToZero}{\ensuremath{\boldsymbol{0^0}}}
\newcommand{\inftyToZero}{\ensuremath{\boldsymbol{\infty^0}}}



\newcommand{\numOverZero}{\ensuremath{\boldsymbol{\tfrac{\#}{0}}}}
\newcommand{\dfn}{\textbf}
%\newcommand{\unit}{\,\mathrm}
\newcommand{\unit}{\mathop{}\!\mathrm}
\newcommand{\eval}[1]{\bigg[ #1 \bigg]}
\newcommand{\seq}[1]{\left( #1 \right)}
\renewcommand{\epsilon}{\varepsilon}
\renewcommand{\phi}{\varphi}


\renewcommand{\iff}{\Leftrightarrow}

\DeclareMathOperator{\arccot}{arccot}
\DeclareMathOperator{\arcsec}{arcsec}
\DeclareMathOperator{\arccsc}{arccsc}
\DeclareMathOperator{\si}{Si}
\DeclareMathOperator{\scal}{scal}
\DeclareMathOperator{\sign}{sign}


%% \newcommand{\tightoverset}[2]{% for arrow vec
%%   \mathop{#2}\limits^{\vbox to -.5ex{\kern-0.75ex\hbox{$#1$}\vss}}}
\newcommand{\arrowvec}[1]{{\overset{\rightharpoonup}{#1}}}
%\renewcommand{\vec}[1]{\arrowvec{\mathbf{#1}}}
\renewcommand{\vec}[1]{{\overset{\boldsymbol{\rightharpoonup}}{\mathbf{#1}}}\hspace{0in}}

\newcommand{\point}[1]{\left(#1\right)} %this allows \vector{ to be changed to \vector{ with a quick find and replace
\newcommand{\pt}[1]{\mathbf{#1}} %this allows \vec{ to be changed to \vec{ with a quick find and replace
\newcommand{\Lim}[2]{\lim_{\point{#1} \to \point{#2}}} %Bart, I changed this to point since I want to use it.  It runs through both of the exercise and exerciseE files in limits section, which is why it was in each document to start with.

\DeclareMathOperator{\proj}{\mathbf{proj}}
\newcommand{\veci}{{\boldsymbol{\hat{\imath}}}}
\newcommand{\vecj}{{\boldsymbol{\hat{\jmath}}}}
\newcommand{\veck}{{\boldsymbol{\hat{k}}}}
\newcommand{\vecl}{\vec{\boldsymbol{\l}}}
\newcommand{\uvec}[1]{\mathbf{\hat{#1}}}
\newcommand{\utan}{\mathbf{\hat{t}}}
\newcommand{\unormal}{\mathbf{\hat{n}}}
\newcommand{\ubinormal}{\mathbf{\hat{b}}}

\newcommand{\dotp}{\bullet}
\newcommand{\cross}{\boldsymbol\times}
\newcommand{\grad}{\boldsymbol\nabla}
\newcommand{\divergence}{\grad\dotp}
\newcommand{\curl}{\grad\cross}
%\DeclareMathOperator{\divergence}{divergence}
%\DeclareMathOperator{\curl}[1]{\grad\cross #1}
\newcommand{\lto}{\mathop{\longrightarrow\,}\limits}

\renewcommand{\bar}{\overline}

\colorlet{textColor}{black}
\colorlet{background}{white}
\colorlet{penColor}{blue!50!black} % Color of a curve in a plot
\colorlet{penColor2}{red!50!black}% Color of a curve in a plot
\colorlet{penColor3}{red!50!blue} % Color of a curve in a plot
\colorlet{penColor4}{green!50!black} % Color of a curve in a plot
\colorlet{penColor5}{orange!80!black} % Color of a curve in a plot
\colorlet{penColor6}{yellow!70!black} % Color of a curve in a plot
\colorlet{fill1}{penColor!20} % Color of fill in a plot
\colorlet{fill2}{penColor2!20} % Color of fill in a plot
\colorlet{fillp}{fill1} % Color of positive area
\colorlet{filln}{penColor2!20} % Color of negative area
\colorlet{fill3}{penColor3!20} % Fill
\colorlet{fill4}{penColor4!20} % Fill
\colorlet{fill5}{penColor5!20} % Fill
\colorlet{gridColor}{gray!50} % Color of grid in a plot

\newcommand{\surfaceColor}{violet}
\newcommand{\surfaceColorTwo}{redyellow}
\newcommand{\sliceColor}{greenyellow}




\pgfmathdeclarefunction{gauss}{2}{% gives gaussian
  \pgfmathparse{1/(#2*sqrt(2*pi))*exp(-((x-#1)^2)/(2*#2^2))}%
}


%%%%%%%%%%%%%
%% Vectors
%%%%%%%%%%%%%

%% Simple horiz vectors
\renewcommand{\vector}[1]{\left\langle #1\right\rangle}


%% %% Complex Horiz Vectors with angle brackets
%% \makeatletter
%% \renewcommand{\vector}[2][ , ]{\left\langle%
%%   \def\nextitem{\def\nextitem{#1}}%
%%   \@for \el:=#2\do{\nextitem\el}\right\rangle%
%% }
%% \makeatother

%% %% Vertical Vectors
%% \def\vector#1{\begin{bmatrix}\vecListA#1,,\end{bmatrix}}
%% \def\vecListA#1,{\if,#1,\else #1\cr \expandafter \vecListA \fi}

%%%%%%%%%%%%%
%% End of vectors
%%%%%%%%%%%%%

%\newcommand{\fullwidth}{}
%\newcommand{\normalwidth}{}



%% makes a snazzy t-chart for evaluating functions
%\newenvironment{tchart}{\rowcolors{2}{}{background!90!textColor}\array}{\endarray}

%%This is to help with formatting on future title pages.
\newenvironment{sectionOutcomes}{}{}



%% Flowchart stuff
%\tikzstyle{startstop} = [rectangle, rounded corners, minimum width=3cm, minimum height=1cm,text centered, draw=black]
%\tikzstyle{question} = [rectangle, minimum width=3cm, minimum height=1cm, text centered, draw=black]
%\tikzstyle{decision} = [trapezium, trapezium left angle=70, trapezium right angle=110, minimum width=3cm, minimum height=1cm, text centered, draw=black]
%\tikzstyle{question} = [rectangle, rounded corners, minimum width=3cm, minimum height=1cm,text centered, draw=black]
%\tikzstyle{process} = [rectangle, minimum width=3cm, minimum height=1cm, text centered, draw=black]
%\tikzstyle{decision} = [trapezium, trapezium left angle=70, trapezium right angle=110, minimum width=3cm, minimum height=1cm, text centered, draw=black]


\author{Bart Snapp}

\outcome{Undestand the relationship between grad, curl, and div.}

\title[Dig-In:]{Grad, Curl, Div}

\begin{document}
\begin{abstract}
  We explore the relationship between the gradient, the curl, and the
  divergence of a vector field.
\end{abstract}
\maketitle

At this point in our study, we have many fundamental theorems. Let's
try to use them together, and see what we can discover.

\section{A two-dimensional dream}

So far we have two fundamental theorems of calculus for functions of
two variables.
\begin{itemize}
\item The fundamental theorem of line integrals:
  \begin{image}
    \begin{tikzpicture}
      \draw [ultra thick, gray] plot [smooth] coordinates {(-2,2) (0,3) (-.5,2) (2,3)};
      \draw [ultra thick, gray,->,opacity=1] plot [smooth] coordinates {(0,2.9) (-.35,2.4)};
      \draw[black,fill=black] (-2,2) circle (.5ex);
      \draw[black,fill=black] (2,3) circle (.5ex);
      \node[black,left] at (-2,2) {$\vec{a}$};
      \node[black,right] at (2,3) {$\vec{b}$};
    \end{tikzpicture}
\end{image}
  \[
  \int_C \grad F \dotp\d\vec{p} = F(\vec{b}) -F(\vec{a})
  \]
\item Green's theorem:
  \begin{image}
    \begin{tikzpicture}
      \draw[ultra thick, black,fill=gray] plot [smooth cycle] coordinates {(-1.5,.5) (.5,1) (1,2.5) (-1,2.5) (-2,1.5)};
      \draw[ultra thick,black, ->] plot [smooth] coordinates {(.85,1.51) (1.02,1.95)};
      \node[black] at (-.5,1.5) {$R$};
      \node[black,right] at (.5,1) {$\partial R$};
    \end{tikzpicture}
  \end{image}
  \[
  \iint_R \curl \vec{F} \d A = \oint_{\partial R} \vec{F} \dotp \d \vec{p}
  \]
\end{itemize}
Wouldn't it be cool if we could \textit{combine} these theorems to
make something like:
\[
\iint_R \curl \grad F \d A = \oint_{\partial R} \grad F \dotp \d \vec{p} = F(\vec{b}) - F(\vec{a})
\]
A double integral, equaling a single integral, equaling a difference!
It's an amazing idea, and while it is true, we shouldn't get too
excited. In fact each expression above is equal to zero. Let's see
why.

First note that both sides of the equation:
\[
\oint_{\partial R} \grad F \dotp \d \vec{p} = F(\vec{b}) - F(\vec{a})
\]
are zero. This is because, with a closed curve, $\vec{a}=\vec{b}$, and
so $F(\vec{a}) = F(\vec{b})$. But this means that
\[
\iint_R \curl \grad F \d A = 0
\]
Note, this is totally independent of the region of the region $R$ and
the surface $F$. The upshot: $\curl\grad F = 0$ for all functions
$\vec{F}:\R^2\to\R^2$ with continuous second derivatives!

\begin{question}
  Let $F:\R^2\to \R$ be a function with continuous second
  derivatives. Which of the following make sense?
  \begin{selectAll}
    \choice[correct]{$\grad F$}
    \choice{$\curl F$}
    \choice{$\divergence F$}
    \choice[correct]{$\curl\grad F$}
    \choice{$\curl \divergence F$}
    \choice[correct]{$\divergence \grad F$}
    \choice{$\divergence \curl F$}
  \end{selectAll}
  \begin{question}
    Of the choices above that make sense, which must be equal to $0$?
    \begin{prompt}
    \begin{selectAll}
      \choice{$\grad F$}
      \choice[correct]{$\curl\grad F$}
      \choice{$\divergence \grad F$}
    \end{selectAll}
    \end{prompt}
    \end{question}
\end{question}

Of course, you could poo-poo the work above and say ``I already knew
that!  This is just just the Clairaut gradient test!'' Well, sure, but
what we just presented is \textit{another} reason that $\curl \grad F
= 0$. Moreover, this line of reasoning will lead us to new ideas
too. Read-on young mathematician, there is not much further to go in
this course.




\section{A three-dimensional dream}

Working in a similar way to how we worked above, let us recall the fundamental theorems of calculus for functions of three variables.

\begin{itemize}
\item The fundamental theorem of line integrals:
  \begin{image}
    \begin{tikzpicture}
      \draw [ultra thick, gray] plot [smooth] coordinates {(-2,2) (0,3) (-.5,2) (2,3)};
      \draw [ultra thick, gray,->,opacity=1] plot [smooth] coordinates {(0,2.9) (-.35,2.4)};
      \draw[black,fill=black] (-2,2) circle (.5ex);
      \draw[black,fill=black] (2,3) circle (.5ex);
      \node[black,left] at (-2,2) {$\vec{a}$};
      \node[black,right] at (2,3) {$\vec{b}$};
    \end{tikzpicture}
\end{image}
  \[
  \int_C \grad F \dotp\d\vec{p} = F(\vec{b}) - F(\vec{a}) 
  \]
\item Stokes' theorem:
  \begin{image}
    \begin{tikzpicture}
      \draw[gray,ultra thick] plot[smooth] coordinates {(-1.5,1.5) (-1,3) (0,3.3)  (1.5,1.5)};
      \shade[ball color=white!50!gray] plot[smooth] coordinates {(-1.5,1.5) (-1,3) (0,3.3)  (1.5,1.5)};
      \shade[top color=black!50!gray, middle color=black]  (0,1.5) ellipse (1.5 and .3);
      \draw[black,ultra thick] (0,1.5) ellipse (1.5 and .3);
    \end{tikzpicture}
  \end{image}
  \[
  \iint_R \curl\vec{F}\dotp \uvec{n} \d S = \oint_{\partial R} \vec{F} \dotp \d \vec{p}
  \]
\item The divergence theorem:
  \begin{image}
  \begin{tikzpicture}
    \draw[ultra thick, gray!50!black] plot [smooth cycle] coordinates {(-1.5,2) (.5,1) (1.5,2) (.5,3) (-1.5,3)};
    \shade[ball color=gray!50!white] plot [smooth cycle] coordinates {(-1.5,2) (.5,1) (1.5,2) (.5,3) (-1.5,3)};
  \end{tikzpicture}
\end{image}

  \[
  \iiint_R \divergence\vec{F} \d V = \oiint_{\partial R} \vec{F}\dotp\uvec{n} \d S
  \]
\end{itemize}
Putting Stokes' theorem and the divergence theorem together, we find the beautiful expression:
\[
\iiint_R \divergence\left(\curl\vec{F}\right)\d V = \oiint_{\partial R} \left(\curl \vec{F}\right)\dotp \uvec{n} \d S = \oint_{\partial\partial R} \vec{F}\dotp \d \vec{p}
\]
Again, what a fantastic idea, a triple integral, equaling a double
integral, equaling a single integral! However, there is again a rub,
the boundary of a closed curve is \textit{empty}. A closed curve has
\textbf{no boundary}! Hence:
\[
\oint_{\partial\partial R} \vec{F}\dotp \d \vec{p} = 0
\]
This fact, in turn means that all of the integrals above, including:
\[
\iiint_R \divergence\left(\curl\vec{F}\right)\d V = \oiint_{\partial R} \left(\curl \vec{F}\right)\dotp \uvec{n} \d S
\]
Are all equal to to zero!  We thus conclude:
\begin{itemize}
\item $\oiint_{\partial R} \left(\curl \vec{F}\right)\dotp \uvec{n} \d
  S=0$ tells us that the circulation along \textbf{any} closed surface
  is zero, and
\item $\iiint_R \divergence\left(\curl\vec{F}\right)\d V = 0$ tells us
  that for any vector field $\vec{F}:\R^3\to\R^3$,
  $\divergence\left(\curl\vec{F}\right)=0$.
\end{itemize}



\section{Grad, Curl, Div}

\begin{image}
  \begin{tikzpicture}
    \node{
    \begin{tikzcd}[ampersand replacement=\&,row sep=.1em]
     C^{\infty}(\R^2,\R) \arrow[r,"\grad"] \& C^{\infty}(\R^2,\R^2) \arrow[r,"\curl"] \& C^{\infty}(\R^2,\R)\\
     F \arrow[r,mapsto] \& \grad F  \arrow[r,mapsto] \& 0
    \end{tikzcd}};
  \end{tikzpicture}
\end{image}


\begin{image}
  \begin{tikzpicture}
    \node{
  \begin{tikzcd}[ampersand replacement=\&, row sep=.1em]
    C^{\infty}(\R^3,\R) \arrow[r,"\grad"] \& C^{\infty}(\R^3,\R^3) \arrow[r,"\curl"] \& C^{\infty}(\R^3,\R^3) \arrow[r,"\divergence"] \& C^{\infty}(\R^3,\R)\\
    F \arrow[r,mapsto] \& \grad F  \arrow[r,mapsto] \& \vec{0}   \&\\
                       \& \vec{F}  \arrow[r,mapsto] \& \curl\vec{F} \arrow[r,mapsto]   \& 0
    \end{tikzcd}};
  \end{tikzpicture}
\end{image}
  


There is so much more to know. 


\end{document}
