\documentclass{ximera}

%\usepackage{todonotes}
%\usepackage{mathtools} %% Required for wide table Curl and Greens
%\usepackage{cuted} %% Required for wide table Curl and Greens
\newcommand{\todo}{}

\usepackage{esint} % for \oiint
\ifxake%%https://math.meta.stackexchange.com/questions/9973/how-do-you-render-a-closed-surface-double-integral
\renewcommand{\oiint}{{\large\bigcirc}\kern-1.56em\iint}
\fi


\graphicspath{
  {./}
  {ximeraTutorial/}
  {basicPhilosophy/}
  {functionsOfSeveralVariables/}
  {normalVectors/}
  {lagrangeMultipliers/}
  {vectorFields/}
  {greensTheorem/}
  {shapeOfThingsToCome/}
  {dotProducts/}
  {partialDerivativesAndTheGradientVector/}
  {../productAndQuotientRules/exercises/}
  {../normalVectors/exercisesParametricPlots/}
  {../continuityOfFunctionsOfSeveralVariables/exercises/}
  {../partialDerivativesAndTheGradientVector/exercises/}
  {../directionalDerivativeAndChainRule/exercises/}
  {../commonCoordinates/exercisesCylindricalCoordinates/}
  {../commonCoordinates/exercisesSphericalCoordinates/}
  {../greensTheorem/exercisesCurlAndLineIntegrals/}
  {../greensTheorem/exercisesDivergenceAndLineIntegrals/}
  {../shapeOfThingsToCome/exercisesDivergenceTheorem/}
  {../greensTheorem/}
  {../shapeOfThingsToCome/}
  {../separableDifferentialEquations/exercises/}
  {vectorFields/}
}

\newcommand{\mooculus}{\textsf{\textbf{MOOC}\textnormal{\textsf{ULUS}}}}

\usepackage{tkz-euclide}\usepackage{tikz}
\usepackage{tikz-cd}
\usetikzlibrary{arrows}
\tikzset{>=stealth,commutative diagrams/.cd,
  arrow style=tikz,diagrams={>=stealth}} %% cool arrow head
\tikzset{shorten <>/.style={ shorten >=#1, shorten <=#1 } } %% allows shorter vectors

\usetikzlibrary{backgrounds} %% for boxes around graphs
\usetikzlibrary{shapes,positioning}  %% Clouds and stars
\usetikzlibrary{matrix} %% for matrix
\usepgfplotslibrary{polar} %% for polar plots
\usepgfplotslibrary{fillbetween} %% to shade area between curves in TikZ
\usetkzobj{all}
\usepackage[makeroom]{cancel} %% for strike outs
%\usepackage{mathtools} %% for pretty underbrace % Breaks Ximera
%\usepackage{multicol}
\usepackage{pgffor} %% required for integral for loops



%% http://tex.stackexchange.com/questions/66490/drawing-a-tikz-arc-specifying-the-center
%% Draws beach ball
\tikzset{pics/carc/.style args={#1:#2:#3}{code={\draw[pic actions] (#1:#3) arc(#1:#2:#3);}}}



\usepackage{array}
\setlength{\extrarowheight}{+.1cm}
\newdimen\digitwidth
\settowidth\digitwidth{9}
\def\divrule#1#2{
\noalign{\moveright#1\digitwidth
\vbox{\hrule width#2\digitwidth}}}





\newcommand{\RR}{\mathbb R}
\newcommand{\R}{\mathbb R}
\newcommand{\N}{\mathbb N}
\newcommand{\Z}{\mathbb Z}

\newcommand{\sagemath}{\textsf{SageMath}}


%\renewcommand{\d}{\,d\!}
\renewcommand{\d}{\mathop{}\!d}
\newcommand{\dd}[2][]{\frac{\d #1}{\d #2}}
\newcommand{\pp}[2][]{\frac{\partial #1}{\partial #2}}
\renewcommand{\l}{\ell}
\newcommand{\ddx}{\frac{d}{\d x}}

\newcommand{\zeroOverZero}{\ensuremath{\boldsymbol{\tfrac{0}{0}}}}
\newcommand{\inftyOverInfty}{\ensuremath{\boldsymbol{\tfrac{\infty}{\infty}}}}
\newcommand{\zeroOverInfty}{\ensuremath{\boldsymbol{\tfrac{0}{\infty}}}}
\newcommand{\zeroTimesInfty}{\ensuremath{\small\boldsymbol{0\cdot \infty}}}
\newcommand{\inftyMinusInfty}{\ensuremath{\small\boldsymbol{\infty - \infty}}}
\newcommand{\oneToInfty}{\ensuremath{\boldsymbol{1^\infty}}}
\newcommand{\zeroToZero}{\ensuremath{\boldsymbol{0^0}}}
\newcommand{\inftyToZero}{\ensuremath{\boldsymbol{\infty^0}}}



\newcommand{\numOverZero}{\ensuremath{\boldsymbol{\tfrac{\#}{0}}}}
\newcommand{\dfn}{\textbf}
%\newcommand{\unit}{\,\mathrm}
\newcommand{\unit}{\mathop{}\!\mathrm}
\newcommand{\eval}[1]{\bigg[ #1 \bigg]}
\newcommand{\seq}[1]{\left( #1 \right)}
\renewcommand{\epsilon}{\varepsilon}
\renewcommand{\phi}{\varphi}


\renewcommand{\iff}{\Leftrightarrow}

\DeclareMathOperator{\arccot}{arccot}
\DeclareMathOperator{\arcsec}{arcsec}
\DeclareMathOperator{\arccsc}{arccsc}
\DeclareMathOperator{\si}{Si}
\DeclareMathOperator{\scal}{scal}
\DeclareMathOperator{\sign}{sign}


%% \newcommand{\tightoverset}[2]{% for arrow vec
%%   \mathop{#2}\limits^{\vbox to -.5ex{\kern-0.75ex\hbox{$#1$}\vss}}}
\newcommand{\arrowvec}[1]{{\overset{\rightharpoonup}{#1}}}
%\renewcommand{\vec}[1]{\arrowvec{\mathbf{#1}}}
\renewcommand{\vec}[1]{{\overset{\boldsymbol{\rightharpoonup}}{\mathbf{#1}}}\hspace{0in}}

\newcommand{\point}[1]{\left(#1\right)} %this allows \vector{ to be changed to \vector{ with a quick find and replace
\newcommand{\pt}[1]{\mathbf{#1}} %this allows \vec{ to be changed to \vec{ with a quick find and replace
\newcommand{\Lim}[2]{\lim_{\point{#1} \to \point{#2}}} %Bart, I changed this to point since I want to use it.  It runs through both of the exercise and exerciseE files in limits section, which is why it was in each document to start with.

\DeclareMathOperator{\proj}{\mathbf{proj}}
\newcommand{\veci}{{\boldsymbol{\hat{\imath}}}}
\newcommand{\vecj}{{\boldsymbol{\hat{\jmath}}}}
\newcommand{\veck}{{\boldsymbol{\hat{k}}}}
\newcommand{\vecl}{\vec{\boldsymbol{\l}}}
\newcommand{\uvec}[1]{\mathbf{\hat{#1}}}
\newcommand{\utan}{\mathbf{\hat{t}}}
\newcommand{\unormal}{\mathbf{\hat{n}}}
\newcommand{\ubinormal}{\mathbf{\hat{b}}}

\newcommand{\dotp}{\bullet}
\newcommand{\cross}{\boldsymbol\times}
\newcommand{\grad}{\boldsymbol\nabla}
\newcommand{\divergence}{\grad\dotp}
\newcommand{\curl}{\grad\cross}
%\DeclareMathOperator{\divergence}{divergence}
%\DeclareMathOperator{\curl}[1]{\grad\cross #1}
\newcommand{\lto}{\mathop{\longrightarrow\,}\limits}

\renewcommand{\bar}{\overline}

\colorlet{textColor}{black}
\colorlet{background}{white}
\colorlet{penColor}{blue!50!black} % Color of a curve in a plot
\colorlet{penColor2}{red!50!black}% Color of a curve in a plot
\colorlet{penColor3}{red!50!blue} % Color of a curve in a plot
\colorlet{penColor4}{green!50!black} % Color of a curve in a plot
\colorlet{penColor5}{orange!80!black} % Color of a curve in a plot
\colorlet{penColor6}{yellow!70!black} % Color of a curve in a plot
\colorlet{fill1}{penColor!20} % Color of fill in a plot
\colorlet{fill2}{penColor2!20} % Color of fill in a plot
\colorlet{fillp}{fill1} % Color of positive area
\colorlet{filln}{penColor2!20} % Color of negative area
\colorlet{fill3}{penColor3!20} % Fill
\colorlet{fill4}{penColor4!20} % Fill
\colorlet{fill5}{penColor5!20} % Fill
\colorlet{gridColor}{gray!50} % Color of grid in a plot

\newcommand{\surfaceColor}{violet}
\newcommand{\surfaceColorTwo}{redyellow}
\newcommand{\sliceColor}{greenyellow}




\pgfmathdeclarefunction{gauss}{2}{% gives gaussian
  \pgfmathparse{1/(#2*sqrt(2*pi))*exp(-((x-#1)^2)/(2*#2^2))}%
}


%%%%%%%%%%%%%
%% Vectors
%%%%%%%%%%%%%

%% Simple horiz vectors
\renewcommand{\vector}[1]{\left\langle #1\right\rangle}


%% %% Complex Horiz Vectors with angle brackets
%% \makeatletter
%% \renewcommand{\vector}[2][ , ]{\left\langle%
%%   \def\nextitem{\def\nextitem{#1}}%
%%   \@for \el:=#2\do{\nextitem\el}\right\rangle%
%% }
%% \makeatother

%% %% Vertical Vectors
%% \def\vector#1{\begin{bmatrix}\vecListA#1,,\end{bmatrix}}
%% \def\vecListA#1,{\if,#1,\else #1\cr \expandafter \vecListA \fi}

%%%%%%%%%%%%%
%% End of vectors
%%%%%%%%%%%%%

%\newcommand{\fullwidth}{}
%\newcommand{\normalwidth}{}



%% makes a snazzy t-chart for evaluating functions
%\newenvironment{tchart}{\rowcolors{2}{}{background!90!textColor}\array}{\endarray}

%%This is to help with formatting on future title pages.
\newenvironment{sectionOutcomes}{}{}



%% Flowchart stuff
%\tikzstyle{startstop} = [rectangle, rounded corners, minimum width=3cm, minimum height=1cm,text centered, draw=black]
%\tikzstyle{question} = [rectangle, minimum width=3cm, minimum height=1cm, text centered, draw=black]
%\tikzstyle{decision} = [trapezium, trapezium left angle=70, trapezium right angle=110, minimum width=3cm, minimum height=1cm, text centered, draw=black]
%\tikzstyle{question} = [rectangle, rounded corners, minimum width=3cm, minimum height=1cm,text centered, draw=black]
%\tikzstyle{process} = [rectangle, minimum width=3cm, minimum height=1cm, text centered, draw=black]
%\tikzstyle{decision} = [trapezium, trapezium left angle=70, trapezium right angle=110, minimum width=3cm, minimum height=1cm, text centered, draw=black]


\title[Dig-In:]{Concepts of graphing functions}

\begin{document}
\begin{abstract}
  We use the language of calculus to describe graphs of functions.
\end{abstract}
\maketitle

In this section, we review the graphical implications of limits, and
the sign of the first and second derivative.  You already know all
this stuff: it is just important enough to hit it more than once, and
put it all together.

\begin{example}
  Sketch the graph of a function $f$ which has the following properties:
  \begin{itemize}
  \item $f(0)=0$
  \item $\lim_{x \to 10^+} f(x) = +\infty$
  \item $\lim_{x \to 10^-} f(x) = -\infty$
  \item $f'(x)<0$ on $(-\infty,0) \cup (6,10) \cup (10,14)$
  \item $f'(x)>0$ on $(0,6) \cup (14,\infty)$
  \item $f''(x)<0$ on $(4,10)$
  \item $f''(x)>0$ on $(-\infty,4) \cup (10,\infty)$
  \end{itemize}
  \begin{explanation}
    Try this on your own first, then either check with a friend or
    check the online version.
    
    \begin{hint}
      The first thing we will do is to plot the point $(0,0)$ and
      indicate the appropriate vertical asymptote due to the limit
      conditions.  We also mark all of the places where $f'$ or $f''$
      change sign.
      
      \begin{image}
        \begin{tikzpicture}
	\begin{axis}[
            domain=-5:20,
            ymax=12,
            ymin=-12,
	    ticks = none,
            axis lines =middle, xlabel=$x$, ylabel=$y$,
            every axis y label/.style={at=(current axis.above origin),anchor=south},
            every axis x label/.style={at=(current axis.right of origin),anchor=west}
          ]
	  \addplot[black, smooth]{0};
	  \addplot [dashed, textColor, smooth] plot coordinates {(10,-12) (10,12)};
	  \addplot [textColor, smooth, draw opacity = 0.1] plot coordinates {(6.3,-12) (6.3,12)};  
	  \addplot [textColor, smooth, draw opacity = 0.1] plot coordinates {(14,-12) (14,12)};  
	  \addplot [textColor, smooth, draw opacity = 0.1] plot coordinates {(4.15,-12) (4.15,12)};  
          \addplot [very thick, penColor2, smooth, domain = 8.5:9.9, samples=100] {x^2*(x-8)/(x-10)/4};
          \addplot [very thick, penColor2, smooth, domain = 10.1:12] {(x/2+1)^2*(x-6)/(x-10)/10-10};
          \addplot[color=penColor2,fill=penColor2,only marks,mark=*] coordinates{(0,0)};  %% closed hole
	 \node at (axis cs:10,-0.7) [anchor=west] {\color{textColor}$10$}; 
	 \node at (axis cs:4.15,-0.7) [anchor=west] {\color{textColor}$4$}; 
	 \node at (axis cs:6.3,-0.7) [anchor=west] {\color{textColor}$6$};  
	 \node at (axis cs:14,-0.7) [anchor=west] {\color{textColor}$14$};      
        \end{axis}
\end{tikzpicture}
      \end{image}
      
    \end{hint}

    \begin{hint}
      Now we classify the behavior on each of the intervals:
      
	\begin{itemize}
	\item On $(-\infty, 0)$, $f$ is \wordChoice{\choice{increasing} \choice[correct]{decreasing}} and concave \wordChoice{\choice[correct]{up}\choice{down}} 
	\item On $(0, 4)$, $f$ is \wordChoice{\choice[correct]{increasing} \choice{decreasing}} and concave \wordChoice{\choice[correct]{up}\choice{down}}
	\item On $(4, 6)$, $f$ is \wordChoice{\choice[correct]{increasing} \choice{decreasing}} and concave \wordChoice{\choice{up}\choice[correct]{down}}
	\item On $(6, 10)$, $f$ is \wordChoice{\choice{increasing} \choice[correct]{decreasing}} and concave \wordChoice{\choice{up}\choice[correct]{down}}
	\item On $(10, 14)$, $f$ is \wordChoice{\choice{increasing} \choice[correct]{decreasing}} and concave \wordChoice{\choice[correct]{up}\choice{down}}
	\item On $(14, \infty)$, $f$ is \wordChoice{\choice[correct]{increasing} \choice{decreasing}} and concave \wordChoice{\choice[correct]{up}\choice{down}}
	\end{itemize}
    \end{hint}

    \begin{hint}
      Utilizing all of this information, we are forced to sketch something like the following:
      
	\begin{image}
          \begin{tikzpicture}
	    \begin{axis}[
                domain=-5:20,
                ymax=12,
            ymin=-12,
	    ticks = none,
            axis lines =middle, xlabel=$x$, ylabel=$y$,
            every axis y label/.style={at=(current axis.above origin),anchor=south},
            every axis x label/.style={at=(current axis.right of origin),anchor=west}
              ]
	      \addplot[black, smooth]{0};
	      \addplot [dashed, textColor, smooth] plot coordinates {(10,-12) (10,12)};
	      \addplot [textColor, smooth, draw opacity = 0.1] plot coordinates {(6.3,-12) (6.3,12)};  
	      \addplot [textColor, smooth, draw opacity = 0.1] plot coordinates {(14,-12) (14,12)};  
	      \addplot [textColor, smooth, draw opacity = 0.1] plot coordinates {(4.15,-12) (4.15,12)};  
              \addplot [very thick, penColor2, smooth, domain = -5:9.9, samples=100] {x^2*(x-8)/(x-10)/4};
              \addplot [very thick, penColor2, smooth, domain = 10.1:20] {(x/2+1)^2*(x-6)/(x-10)/10-10};
              \addplot[color=penColor2,fill=penColor2,only marks,mark=*] coordinates{(0,0)};  %% closed hole
	      \node at (axis cs:10,-0.7) [anchor=west] {\color{textColor}$10$}; 
	      \node at (axis cs:4.15,-0.7) [anchor=west] {\color{textColor}$4$}; 
	      \node at (axis cs:6.3,-0.7) [anchor=west] {\color{textColor}$6$};  
	      \node at (axis cs:14,-0.7) [anchor=west] {\color{textColor}$14$};      
            \end{axis}
          \end{tikzpicture}
        \end{image}
    \end{hint}    
  \end{explanation}
\end{example}

\begin{example}
  Sketch the graph of a function $f$ which has the following properties:
  \begin{itemize}
  \item $f(0)=1$
  \item $f(6)=2$
  \item $\lim_{x \to 6^+} f(x) = 3$
  \item $\lim_{x \to 6^-} f(x) = 1$
  \item $f'(x)<0$ on $(-\infty,1)$
  \item $f'(x)>0$ on $(1,6)$
  \item $f'(x) = -2$ on $(6, \infty)$
  \item $f''(x)<0$ on $(2.5,5)$
  \item $f''(x)>0$ on $(-\infty,2.5) \cup (5,6)$
  \end{itemize}
  
  \begin{explanation}
    Try this on your own first, then either check with a friend or
    check the online version.
    
    \begin{hint}
      The first thing we will do is to plot the points $(0,1)$ and $(6,2)$, and the ``holes" at $(6,3)$ and $(6,1)$ due to the limit conditions.  We can immediately draw in what $f$ looks like on $(6,\infty)$ since it is linear with slope $2$, and must connect to the hole at $(6,2)$.  We also mark all of the places where $f'$ or $f''$ change sign.
      
	\begin{image}
          \begin{tikzpicture}
	    \begin{axis}[
                domain=-1:8,
                ymax=4,
                ymin=-3,
	        ticks = none,
                axis lines =middle, xlabel=$x$, ylabel=$y$,
                every axis y label/.style={at=(current axis.above origin),anchor=south},
                every axis x label/.style={at=(current axis.right of origin),anchor=west}
              ]
	      \addplot[black, smooth]{0};
	      \addplot [textColor, smooth, draw opacity = 0.1] plot coordinates {(7.5,1.2) (6,1.2)};  
	      \addplot [textColor, smooth, draw opacity = 0.1] plot coordinates {(7.5,3) (6,3)};  
	      \addplot [textColor, smooth, draw opacity = 0.1] plot coordinates {(7.5,2) (6,2)};  
	      \addplot [textColor, smooth, draw opacity = 0.1] plot coordinates {(1.2,-3) (1.2,3)};  
	      \addplot [textColor, smooth, draw opacity = 0.1] plot coordinates {(2.5,-3) (2.5,3)};  
	      \addplot [textColor, smooth, draw opacity = 0.1] plot coordinates {(5,-3) (5,3)};  
	      \addplot [textColor, smooth, draw opacity = 0.1] plot coordinates {(6,-3) (6,3)};  
	      
              % \addplot [very thick, penColor2, smooth, domain = -1:6, samples=100] {5*x*(x/5-1)^3+1};
              \addplot [very thick, penColor2, smooth, domain = 6:8] {-2*(x-6)+3};
              \addplot[color=penColor2,fill=penColor2,only marks,mark=*] coordinates{(6,2)};  
              \addplot[color=penColor2,fill=background, only marks,mark=*] coordinates{(6,3)}; 
              \addplot[color=penColor2,fill=background, only marks,mark=*] coordinates{(6,1.2)}; 
	      \node at (axis cs:1.2,-0.7) [anchor=west] {\color{textColor}$1$}; 
	      \node at (axis cs:2.5,-0.7) [anchor=west] {\color{textColor}$2.5$}; 
	      \node at (axis cs:5,-0.7) [anchor=west] {\color{textColor}$5$};  
	      \node at (axis cs:6,-0.7) [anchor=west] {\color{textColor}$6$};  
              
	      \node at (axis cs:7.5,3) [anchor=west] {\color{textColor}$3$}; 
	      \node at (axis cs:7.5,2) [anchor=west] {\color{textColor}$2$}; 
	      \node at (axis cs:7.5,1.2) [anchor=west] {\color{textColor}$1$};      
            \end{axis}
          \end{tikzpicture}
        \end{image}
        
    \end{hint}

\begin{hint}
  Now we classify the behavior on each of the intervals:
	\begin{itemize}
	\item On $(-\infty, 1)$, $f$ is \wordChoice{\choice{increasing} \choice[correct]{decreasing}} and concave \wordChoice{\choice[correct]{up}\choice{down}} 
	\item On $(1, 2.5)$, $f$ is \wordChoice{\choice[correct]{increasing} \choice{decreasing}} and concave \wordChoice{\choice[correct]{up}\choice{down}}
	\item On $(2.5, 5)$, $f$ is \wordChoice{\choice[correct]{increasing} \choice{decreasing}} and concave \wordChoice{\choice{up}\choice[correct]{down}}
	\item On $(5, 6)$, $f$ is \wordChoice{\choice[correct]{increasing} \choice{decreasing}} and concave \wordChoice{\choice[correct]{up}\choice{down}}
	\end{itemize}
\end{hint}

\begin{hint}
  Utilizing all of this information, we are forced to draw something like the following:
  
  \begin{image}
    \begin{tikzpicture}
      \begin{axis}[
          domain=-1:8,
          ymax=4,
          ymin=-3,
	  ticks = none,
          axis lines =middle, xlabel=$x$, ylabel=$y$,
          every axis y label/.style={at=(current axis.above origin),anchor=south},
          every axis x label/.style={at=(current axis.right of origin),anchor=west}
        ]
	\addplot[black, smooth]{0};
	\addplot [textColor, smooth, draw opacity = 0.1] plot coordinates {(7.5,1.2) (6,1.2)};  
	\addplot [textColor, smooth, draw opacity = 0.1] plot coordinates {(7.5,3) (6,3)};  
	\addplot [textColor, smooth, draw opacity = 0.1] plot coordinates {(7.5,2) (6,2)};  
	\addplot [textColor, smooth, draw opacity = 0.1] plot coordinates {(1.2,-3) (1.2,3)};  
	\addplot [textColor, smooth, draw opacity = 0.1] plot coordinates {(2.5,-3) (2.5,3)};  
	\addplot [textColor, smooth, draw opacity = 0.1] plot coordinates {(5,-3) (5,3)};  
	\addplot [textColor, smooth, draw opacity = 0.1] plot coordinates {(6,-3) (6,3)};  
	
        \addplot [very thick, penColor2, smooth, domain = -1:6, samples=100] {5*x*(x/5-1)^3+1};
        \addplot [very thick, penColor2, smooth, domain = 6:8] {-2*(x-6)+3};
        \addplot[color=penColor2,fill=penColor2,only marks,mark=*] coordinates{(6,2)};  
        \addplot[color=penColor2,fill=background, only marks,mark=*] coordinates{(6,3)}; 
        \addplot[color=penColor2,fill=background, only marks,mark=*] coordinates{(6,1.2)}; 
	\node at (axis cs:1.2,-0.7) [anchor=west] {\color{textColor}$1$}; 
	\node at (axis cs:2.5,-0.7) [anchor=west] {\color{textColor}$2.5$}; 
	\node at (axis cs:5,-0.7) [anchor=west] {\color{textColor}$5$};  
	\node at (axis cs:6,-0.7) [anchor=west] {\color{textColor}$6$};  
        
	\node at (axis cs:7.5,3) [anchor=west] {\color{textColor}$3$}; 
	\node at (axis cs:7.5,2) [anchor=west] {\color{textColor}$2$}; 
	\node at (axis cs:7.5,1.2) [anchor=west] {\color{textColor}$1$};      
      \end{axis}
    \end{tikzpicture}
  \end{image}
\end{hint}
\end{explanation}
\end{example}

\begin{example}
  The graph of $f'$ (the derivative of $f$ ) is shown below. 
  
  Assume $f$ is continuous for all real numbers.
  
  \begin{image}
    \begin{tikzpicture}
	\begin{axis}[
            domain=-5:5,
            ymax=5,
            ymin=-5,
	    xtick = {-4,...,4},
            axis lines =middle, xlabel=$x$, ylabel=$y$,
            every axis y label/.style={at=(current axis.above origin),anchor=south},
            every axis x label/.style={at=(current axis.right of origin),anchor=west}
          ]
          
	
          \addplot [very thick, penColor2, smooth, domain = -5:3, samples=100] {(abs(x-1)-1)*(1+x^2)/(1+0.4*x^2)};
          \addplot [very thick, penColor2, smooth, domain = 3:5] {-10/x};
          \addplot[color=penColor2,fill=background, only marks,mark=*] coordinates{(3,2.173)}; 
          \addplot[color=penColor2,fill=background, only marks,mark=*] coordinates{(3,-10/3)}; 
        \end{axis}
    \end{tikzpicture}
  \end{image}
  
  
  \begin{question}
    On which of the following intervals is $f$ increasing? 
    \begin{selectAll}
      \choice[correct]{$(-\infty,0)$}
      \choice{$(0,1)$}
      \choice{$(1,2)$}
      \choice[correct]{$(2,3)$}
      \choice{$(3,\infty)$}
    \end{selectAll}
    \begin{hint}
      $f$ is increasing where $f'(x)>0$, i.e. on the intervals
      $(-\infty, 0)$ and $(2,3)$.
    \end{hint}
  \end{question}
  
  \begin{question}
    Which of the following are critical points of $f$?
    \begin{selectAll}
      \choice[correct]{$x=0$}
      \choice{$x=1$}
      \choice[correct]{$x=2$}
      \choice[correct]{$x=3$}
    \end{selectAll}
    
    \begin{hint}
      $f$ has a critical point at the zeros of $f'$, and the places where $f'$ does not exist.  In this case, $x=0$, $x=2$, and $x=3$.
    \end{hint}
  \end{question}
  
  \begin{question}
    Where do the local maxima occur?
    \begin{selectAll}
      \choice[correct]{$x=0$}
      \choice{$x=1$}
      \choice{$x=2$}
      \choice[correct]{$x=3$}
    \end{selectAll}
    \begin{hint}
      A local maximum occurs at a critical point where the function
      transitions from increasing to decreasing, i.e. the derivative
      passes from positive to negative.  In this case, we see that the
      local maxima occur at $x=0$ and $x=3$.
    \end{hint}
  \end{question}
  
  \begin{question}
    Where does a point of inflection occur?
    \begin{selectAll}
      \choice{$x=0$}
      \choice[correct]{$x=1$}
      \choice{$x=2$}
      \choice{$x=3$}
\end{selectAll}
    
    \begin{hint}
      A point of inflection occurs when the concavity of $f$ changes.  This is reflected in the sign of $f''$ changing.  This only occurs at one point in this graph, namely $x=1$.
    \end{hint}
    
  \end{question}

  \begin{question}
    On which of the following intervals is $f$ concave down?
    \begin{selectAll}
      \choice[correct]{$(-\infty,0)$}
      \choice[correct]{$(0,1)$}
      \choice{$(1,2)$}
      \choice{$(2,3)$}
      \choice{$(3,\infty)$}
    \end{selectAll}
    
    \begin{hint}
      $f$ is concave down when $f''(x)<0$.  This occurs for $x<1$ on this graph.  So the correct answer is to select both $(-\infty, 0)$ and $(0,1)$. 
    \end{hint}
  \end{question}
  
  
\end{example}

\end{document}
