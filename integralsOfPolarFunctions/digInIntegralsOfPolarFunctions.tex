\documentclass{ximera}

%\usepackage{todonotes}
%\usepackage{mathtools} %% Required for wide table Curl and Greens
%\usepackage{cuted} %% Required for wide table Curl and Greens
\newcommand{\todo}{}

\usepackage{esint} % for \oiint
\ifxake%%https://math.meta.stackexchange.com/questions/9973/how-do-you-render-a-closed-surface-double-integral
\renewcommand{\oiint}{{\large\bigcirc}\kern-1.56em\iint}
\fi


\graphicspath{
  {./}
  {ximeraTutorial/}
  {basicPhilosophy/}
  {functionsOfSeveralVariables/}
  {normalVectors/}
  {lagrangeMultipliers/}
  {vectorFields/}
  {greensTheorem/}
  {shapeOfThingsToCome/}
  {dotProducts/}
  {partialDerivativesAndTheGradientVector/}
  {../productAndQuotientRules/exercises/}
  {../normalVectors/exercisesParametricPlots/}
  {../continuityOfFunctionsOfSeveralVariables/exercises/}
  {../partialDerivativesAndTheGradientVector/exercises/}
  {../directionalDerivativeAndChainRule/exercises/}
  {../commonCoordinates/exercisesCylindricalCoordinates/}
  {../commonCoordinates/exercisesSphericalCoordinates/}
  {../greensTheorem/exercisesCurlAndLineIntegrals/}
  {../greensTheorem/exercisesDivergenceAndLineIntegrals/}
  {../shapeOfThingsToCome/exercisesDivergenceTheorem/}
  {../greensTheorem/}
  {../shapeOfThingsToCome/}
  {../separableDifferentialEquations/exercises/}
  {vectorFields/}
}

\newcommand{\mooculus}{\textsf{\textbf{MOOC}\textnormal{\textsf{ULUS}}}}

\usepackage{tkz-euclide}\usepackage{tikz}
\usepackage{tikz-cd}
\usetikzlibrary{arrows}
\tikzset{>=stealth,commutative diagrams/.cd,
  arrow style=tikz,diagrams={>=stealth}} %% cool arrow head
\tikzset{shorten <>/.style={ shorten >=#1, shorten <=#1 } } %% allows shorter vectors

\usetikzlibrary{backgrounds} %% for boxes around graphs
\usetikzlibrary{shapes,positioning}  %% Clouds and stars
\usetikzlibrary{matrix} %% for matrix
\usepgfplotslibrary{polar} %% for polar plots
\usepgfplotslibrary{fillbetween} %% to shade area between curves in TikZ
\usetkzobj{all}
\usepackage[makeroom]{cancel} %% for strike outs
%\usepackage{mathtools} %% for pretty underbrace % Breaks Ximera
%\usepackage{multicol}
\usepackage{pgffor} %% required for integral for loops



%% http://tex.stackexchange.com/questions/66490/drawing-a-tikz-arc-specifying-the-center
%% Draws beach ball
\tikzset{pics/carc/.style args={#1:#2:#3}{code={\draw[pic actions] (#1:#3) arc(#1:#2:#3);}}}



\usepackage{array}
\setlength{\extrarowheight}{+.1cm}
\newdimen\digitwidth
\settowidth\digitwidth{9}
\def\divrule#1#2{
\noalign{\moveright#1\digitwidth
\vbox{\hrule width#2\digitwidth}}}





\newcommand{\RR}{\mathbb R}
\newcommand{\R}{\mathbb R}
\newcommand{\N}{\mathbb N}
\newcommand{\Z}{\mathbb Z}

\newcommand{\sagemath}{\textsf{SageMath}}


%\renewcommand{\d}{\,d\!}
\renewcommand{\d}{\mathop{}\!d}
\newcommand{\dd}[2][]{\frac{\d #1}{\d #2}}
\newcommand{\pp}[2][]{\frac{\partial #1}{\partial #2}}
\renewcommand{\l}{\ell}
\newcommand{\ddx}{\frac{d}{\d x}}

\newcommand{\zeroOverZero}{\ensuremath{\boldsymbol{\tfrac{0}{0}}}}
\newcommand{\inftyOverInfty}{\ensuremath{\boldsymbol{\tfrac{\infty}{\infty}}}}
\newcommand{\zeroOverInfty}{\ensuremath{\boldsymbol{\tfrac{0}{\infty}}}}
\newcommand{\zeroTimesInfty}{\ensuremath{\small\boldsymbol{0\cdot \infty}}}
\newcommand{\inftyMinusInfty}{\ensuremath{\small\boldsymbol{\infty - \infty}}}
\newcommand{\oneToInfty}{\ensuremath{\boldsymbol{1^\infty}}}
\newcommand{\zeroToZero}{\ensuremath{\boldsymbol{0^0}}}
\newcommand{\inftyToZero}{\ensuremath{\boldsymbol{\infty^0}}}



\newcommand{\numOverZero}{\ensuremath{\boldsymbol{\tfrac{\#}{0}}}}
\newcommand{\dfn}{\textbf}
%\newcommand{\unit}{\,\mathrm}
\newcommand{\unit}{\mathop{}\!\mathrm}
\newcommand{\eval}[1]{\bigg[ #1 \bigg]}
\newcommand{\seq}[1]{\left( #1 \right)}
\renewcommand{\epsilon}{\varepsilon}
\renewcommand{\phi}{\varphi}


\renewcommand{\iff}{\Leftrightarrow}

\DeclareMathOperator{\arccot}{arccot}
\DeclareMathOperator{\arcsec}{arcsec}
\DeclareMathOperator{\arccsc}{arccsc}
\DeclareMathOperator{\si}{Si}
\DeclareMathOperator{\scal}{scal}
\DeclareMathOperator{\sign}{sign}


%% \newcommand{\tightoverset}[2]{% for arrow vec
%%   \mathop{#2}\limits^{\vbox to -.5ex{\kern-0.75ex\hbox{$#1$}\vss}}}
\newcommand{\arrowvec}[1]{{\overset{\rightharpoonup}{#1}}}
%\renewcommand{\vec}[1]{\arrowvec{\mathbf{#1}}}
\renewcommand{\vec}[1]{{\overset{\boldsymbol{\rightharpoonup}}{\mathbf{#1}}}\hspace{0in}}

\newcommand{\point}[1]{\left(#1\right)} %this allows \vector{ to be changed to \vector{ with a quick find and replace
\newcommand{\pt}[1]{\mathbf{#1}} %this allows \vec{ to be changed to \vec{ with a quick find and replace
\newcommand{\Lim}[2]{\lim_{\point{#1} \to \point{#2}}} %Bart, I changed this to point since I want to use it.  It runs through both of the exercise and exerciseE files in limits section, which is why it was in each document to start with.

\DeclareMathOperator{\proj}{\mathbf{proj}}
\newcommand{\veci}{{\boldsymbol{\hat{\imath}}}}
\newcommand{\vecj}{{\boldsymbol{\hat{\jmath}}}}
\newcommand{\veck}{{\boldsymbol{\hat{k}}}}
\newcommand{\vecl}{\vec{\boldsymbol{\l}}}
\newcommand{\uvec}[1]{\mathbf{\hat{#1}}}
\newcommand{\utan}{\mathbf{\hat{t}}}
\newcommand{\unormal}{\mathbf{\hat{n}}}
\newcommand{\ubinormal}{\mathbf{\hat{b}}}

\newcommand{\dotp}{\bullet}
\newcommand{\cross}{\boldsymbol\times}
\newcommand{\grad}{\boldsymbol\nabla}
\newcommand{\divergence}{\grad\dotp}
\newcommand{\curl}{\grad\cross}
%\DeclareMathOperator{\divergence}{divergence}
%\DeclareMathOperator{\curl}[1]{\grad\cross #1}
\newcommand{\lto}{\mathop{\longrightarrow\,}\limits}

\renewcommand{\bar}{\overline}

\colorlet{textColor}{black}
\colorlet{background}{white}
\colorlet{penColor}{blue!50!black} % Color of a curve in a plot
\colorlet{penColor2}{red!50!black}% Color of a curve in a plot
\colorlet{penColor3}{red!50!blue} % Color of a curve in a plot
\colorlet{penColor4}{green!50!black} % Color of a curve in a plot
\colorlet{penColor5}{orange!80!black} % Color of a curve in a plot
\colorlet{penColor6}{yellow!70!black} % Color of a curve in a plot
\colorlet{fill1}{penColor!20} % Color of fill in a plot
\colorlet{fill2}{penColor2!20} % Color of fill in a plot
\colorlet{fillp}{fill1} % Color of positive area
\colorlet{filln}{penColor2!20} % Color of negative area
\colorlet{fill3}{penColor3!20} % Fill
\colorlet{fill4}{penColor4!20} % Fill
\colorlet{fill5}{penColor5!20} % Fill
\colorlet{gridColor}{gray!50} % Color of grid in a plot

\newcommand{\surfaceColor}{violet}
\newcommand{\surfaceColorTwo}{redyellow}
\newcommand{\sliceColor}{greenyellow}




\pgfmathdeclarefunction{gauss}{2}{% gives gaussian
  \pgfmathparse{1/(#2*sqrt(2*pi))*exp(-((x-#1)^2)/(2*#2^2))}%
}


%%%%%%%%%%%%%
%% Vectors
%%%%%%%%%%%%%

%% Simple horiz vectors
\renewcommand{\vector}[1]{\left\langle #1\right\rangle}


%% %% Complex Horiz Vectors with angle brackets
%% \makeatletter
%% \renewcommand{\vector}[2][ , ]{\left\langle%
%%   \def\nextitem{\def\nextitem{#1}}%
%%   \@for \el:=#2\do{\nextitem\el}\right\rangle%
%% }
%% \makeatother

%% %% Vertical Vectors
%% \def\vector#1{\begin{bmatrix}\vecListA#1,,\end{bmatrix}}
%% \def\vecListA#1,{\if,#1,\else #1\cr \expandafter \vecListA \fi}

%%%%%%%%%%%%%
%% End of vectors
%%%%%%%%%%%%%

%\newcommand{\fullwidth}{}
%\newcommand{\normalwidth}{}



%% makes a snazzy t-chart for evaluating functions
%\newenvironment{tchart}{\rowcolors{2}{}{background!90!textColor}\array}{\endarray}

%%This is to help with formatting on future title pages.
\newenvironment{sectionOutcomes}{}{}



%% Flowchart stuff
%\tikzstyle{startstop} = [rectangle, rounded corners, minimum width=3cm, minimum height=1cm,text centered, draw=black]
%\tikzstyle{question} = [rectangle, minimum width=3cm, minimum height=1cm, text centered, draw=black]
%\tikzstyle{decision} = [trapezium, trapezium left angle=70, trapezium right angle=110, minimum width=3cm, minimum height=1cm, text centered, draw=black]
%\tikzstyle{question} = [rectangle, rounded corners, minimum width=3cm, minimum height=1cm,text centered, draw=black]
%\tikzstyle{process} = [rectangle, minimum width=3cm, minimum height=1cm, text centered, draw=black]
%\tikzstyle{decision} = [trapezium, trapezium left angle=70, trapezium right angle=110, minimum width=3cm, minimum height=1cm, text centered, draw=black]


\outcome{Find the area of a region bound by a polar curve.}
\outcome{Find the intersection points of two polar curves.}
\outcome{Find the area of a region bound by two polar curves.} 

\title[Dig-In:]{Integrals of polar functions}

\begin{document}
\begin{abstract}
  We integrate polar functions.
\end{abstract}
\maketitle


When using rectangular coordinates, the equations $x=h$ and $y=k$
defined vertical and horizontal lines, respectively, and combinations
of these lines create rectangles (hence the name ``rectangular
coordinates''). It is then somewhat natural to use rectangles to
approximate area as we did when learning about the definite
integral.

When using polar coordinates, the equations $\theta=\alpha$ and $r=c$
form lines through the origin and circles centered at the origin,
respectively, and combinations of these curves form sectors of
circles. It is then somewhat natural to calculate the area of regions
defined by polar functions by first approximating with sectors of
circles. Recall that the area of a sector of a circle with radius $r$
subtended by an angle $\theta$
\begin{image}
  \begin{tikzpicture}
	\begin{axis}[
            xmin=-.1,xmax=1.1,ymin=-.1,ymax=1.1,
            axis lines=center,
            ticks=none,
            unit vector ratio*=1 1 1,
            xlabel=$x$, ylabel=$y$,
            every axis y label/.style={at=(current axis.above origin),anchor=south},
            every axis x label/.style={at=(current axis.right of origin),anchor=west},
          ]        
          \addplot [draw=none,fill=fill1] plot coordinates {(0,0) (.766,.643)}\closedcycle; %% sector
          \addplot [draw=none, fill=fill1, samples=100, domain=(0:40)] ({cos(x)},{sin(x)})\closedcycle; %% sector 
          \addplot [very thick, penColor2, smooth, domain=(-.2:.2+pi/2)] ({cos(deg(x))},{sin(deg(x))});
          \addplot [textColor,smooth, domain=(0:40)] ({.15*cos(x)},{.15*sin(x)});
          \addplot [very thick,penColor] plot coordinates {(0,0) (.766,.643)}; %% sector
          \addplot [very thick,penColor] plot coordinates {(0,0) (1,0)}; %% sector
          \addplot [very thick, penColor, smooth, domain=(0:40)] ({cos(x)},{sin(x)}); %% sector
          \node at (axis cs:.15,.07) [anchor=west] {$\theta$};
          \node at (axis cs:.5,0) [anchor=north] {$r$};
        \end{axis}
\end{tikzpicture} 
\end{image}
is $A = \frac{1}{2}\theta r^2$. So given a polar plot, partition the
interval $[\alpha,\beta]$ into $n$ equally spaced subintervals as
$\alpha = \theta_1 < \theta_2 <\cdots <\theta_{n+1}=\beta$:
\begin{image}
  \begin{tikzpicture}
\begin{axis}[
axis y line=middle,axis x line=middle,name=myplot,%
			%x=.37\marginparwidth,
			%y=.37\marginparwidth,
			%xtick={-1,1},
			%minor x tick num=1,% 
%			extra x ticks={.33},
%			extra x tick labels={$1/3$},
			%ytick={-1,1},
			%minor y tick num=1,%extra y ticks={-5,-3,...,7},%
			ymin=-.1,ymax=1.1,%
			xmin=-.1,xmax=1.1%
]

\addplot [fill1,fill=fill1,area style, smooth,domain=18:72,samples=30] ({cos(x)*(1+.05*cos(9*x))},{sin(x)*(1+.05*cos(9*x))}) -- (axis cs:0,0) -- cycle;

\addplot [penColor,thick, smooth,domain=0:90,samples=30] ({cos(x)*(1+.05*cos(9*x))},{sin(x)*(1+.05*cos(9*x))});

%\addplot [{\colortwo},thick, smooth,domain=28.65:45.84,samples=30] ({1.045*cos(x)},{1.045*sin(x)}) ;-- (axis cs:0,0) -- cycle;

\draw [thick,penColor,] (axis cs:0,0) -- (axis cs: 0.905831, 0.294322) node [pos=.7,below,rotate=18,black] { $\theta=\alpha$};

\draw [thick,penColor,] (axis cs:0,0) -- (axis cs:0.313792, 0.965751) node [pos=.7,above,rotate=72,black] { $\theta=\beta$};

\draw [thick,penColor,dashed] (axis cs:0,0) -- (axis cs: 0.862592, 0.528597)
(axis cs:0,0) -- (axis cs: 0.732107, 0.732107)
(axis cs:0,0) -- (axis cs: 0.497095, 0.811186);

\draw (axis cs:.8,.85) node { $r(\theta)$};


%\addplot [{\colortwo},thick, smooth,domain=-1:1,samples=2] ({x},{.577*x});
%\addplot [{\colortwo},thick, smooth,domain=-1:1,samples=2] ({x},{-.577*x});


%\draw (axis cs:.5,0) circle (25pt);
\end{axis}

\node [right] at (myplot.right of origin) { $0$};
\node [above] at (myplot.above origin) { $\pi/2$};
\end{tikzpicture}
\end{image}
The length of each subinterval is $\Delta\theta = (\beta-\alpha)/n$,
representing a small change in angle. The area of the region defined
by the $i$th subinterval $[\theta_i,\theta_{i+1}]$ can be approximated
with a sector of a circle with radius $r(\theta^*_i)$, for some
$\theta^*_i$ in $[\theta_i,\theta_{i+1}]$. The area of this sector is
$\frac12r(\theta^*_i)^2\Delta\theta$. This is shown here
\begin{image}
  \begin{tikzpicture}
\begin{axis}[
axis y line=middle,axis x line=middle,name=myplot,%
			%x=.37\marginparwidth,
			%y=.37\marginparwidth,
			%xtick={-1,1},
			%minor x tick num=1,% 
%			extra x ticks={.33},
%			extra x tick labels={$1/3$},
			%ytick={-1,1},
			%minor y tick num=1,%extra y ticks={-5,-3,...,7},%
			ymin=-.1,ymax=1.1,%
			xmin=-.1,xmax=1.1%
]

\addplot [fill1,fill=fill1,area style, smooth,domain=18:72,samples=30] ({cos(x)*(1+.05*cos(9*x))},{sin(x)*(1+.05*cos(9*x))}) -- (axis cs:0,0) -- cycle;

\draw [thick,penColor,] (axis cs:0,0) -- (axis cs:0.313792, 0.965751) node [pos=.7,above,rotate=72,black] { $\theta=\beta$};

\addplot [penColor2,fill=fill2,thick, smooth,domain=18:31.5,samples=30] ({.96*cos(x)},{.96*sin(x)}) -- (axis cs:0,0) -- cycle;
\addplot [penColor2,fill=fill2,thick, smooth,domain=31.5:45,samples=30] ({1.05*cos(x)},{1.05*sin(x)}) -- (axis cs:0,0) -- cycle;
\addplot [penColor2,fill=fill2,thick, smooth,domain=45:58.5,samples=30] ({1.0*cos(x)},{1*sin(x)}) -- (axis cs:0,0) -- cycle;
\addplot [penColor2,fill=fill2,thick, smooth,domain=58.5:72,samples=30] ({.96*cos(x)},{.96*sin(x)}) -- (axis cs:0,0) -- cycle;

\draw (axis cs:.8,.85) node { $r(\theta)$};


\draw [thick,penColor2,] (axis cs:0,0) -- (axis cs: 0.905831, 0.294322) node [pos=.7,below,rotate=18,black] { $\theta=\alpha$};



\addplot [penColor,thick, smooth,domain=0:90,samples=30] ({cos(x)*(1+.05*cos(9*x))},{sin(x)*(1+.05*cos(9*x))});
%\draw [thick,penColor2] (axis cs:0,0) -- (axis cs: 0.862592, 0.528597)
%(axis cs:0,0) -- (axis cs: 0.732107, 0.732107)
%(axis cs:0,0) -- (axis cs: 0.497095, 0.811186);

\end{axis}

\node [right] at (myplot.right of origin) {$0$};
\node [above] at (myplot.above origin) {$\pi/2$};
\end{tikzpicture}
\end{image}
where $[\alpha,\beta]$ has been divided into $4$ subintervals. We
approximate the area of the whole region by summing the areas of all
sectors:
\[
\text{Area} \approx \sum_{i=1}^n \frac12r(\theta^*_i)^2\Delta\theta.
\]
This is a Riemann sum! By taking the limit of the sum as $n\to\infty$,
we find the exact area of the region in the form of a definite
integral.  

\begin{theorem}[Area of a Polar Region]
Let $r$ be continuous and non-negative on $[\alpha,\beta]$, where
$0\leq \beta-\alpha\leq 2\pi$. The area $A$ of the region bounded by
the curve $r(\theta)$ and the lines $\theta=\alpha$ and
$\theta=\beta$ is
\[
A  =  \frac12\int_\alpha^\beta r(\theta)^2 \d \theta  
\]
\end{theorem}

The theorem states that $0\leq \beta-\alpha\leq 2\pi$. This ensures
that region does not overlap itself, giving a result that does not
correspond directly to the area.


\begin{example}
  Compute the area of one petal of the polar curve $r(\theta) =
  \cos(3\theta)$:
  \begin{image}
    \begin{tikzpicture}
          \begin{polaraxis}[
              xtick={0,30,...,360},
              xticklabels={$0$,$\frac{\pi}{6}$,$\frac{\pi}{3}$,$\frac{\pi}{2}$,$\frac{2\pi}{3}$,$\frac{5\pi}{6}$,$\pi$,$\frac{7\pi}{6}$,$\frac{4\pi}{3}$,$\frac{3\pi}{2}$,$\frac{5\pi}{3}$,$\frac{11\pi}{6}$,$2\pi$},
              ytick={.5,1,...,2},
              yticklabels={},axis on top
            ]
            \addplot+[very thick, mark=none,fill=fill1,domain=-30:30,samples=100,smooth] {cos(3*x)} \closedcycle;
            \addplot+[very thick, mark=none,penColor,domain=0:180,samples=100,smooth] {cos(3*x)};
          \end{polaraxis}
    \end{tikzpicture}
  \end{image}
  \begin{explanation}
    From the picture it looks like integrating from
    $\theta=-\pi/6$ to $\pi/6$ will give us the area
      of our desired region. We can convince ourselves that this is
      correct by inspecting
      \[
      r(\theta) = \cos(3\theta)
      \]
      and noting that our curve starts at
      $(\answer[given]{1},\answer[given]{0})$ in the $(x,y)$-plane
      when $\theta=0$, and then moves to the origin. Since the first
      positive value of $\theta$ that make $r(\theta)= 0$ is
      $\answer[given]{\pi/6}$, we see that our eyes did not lie to
      us. Write with me
      \begin{align*}
      \frac{1}{2}\int_{-\pi/6}^{\pi/6} r(\theta)^2 \d \theta &= \frac{1}{2}\int_{-\pi/6}^{\pi/6} \cos^2(3\theta) \d \theta\\
      &= \frac{1}{2}\int_{-\pi/6}^{\pi/6} \frac{1-\cos(6\theta)}{2} \d \theta\\
      &= \frac{1}{4}\int_{-\pi/6}^{\pi/6} 1-\cos(6\theta) \d \theta\\
      &= \frac{1}{4}\eval{\theta-\frac{\sin(6\theta)}{6}}_{-\pi/6}^{\pi/6}\\
      &= \answer[given]{\frac{\pi}{12}}.
      \end{align*}
  \end{explanation}
\end{example}

\begin{question}
Now I present you with a mystery: You know that the area of one petal
of $r(\theta) = \cos(3\theta)$ has area $\pi/12$. Hence, all three
petals must have an area of $\pi/4$. Compare this to
\begin{align*}
\frac{1}{2}\int_0^{2\pi}  \cos^2(3\theta) \d \theta &= \frac{1}{2}\int_{0}^{2\pi} \frac{1-\cos(6\theta)}{2} \d \theta\\
&= \frac{1}{4}\int_{0}^{2\pi} 1-\cos(6\theta) \d \theta\\
&= \frac{1}{4}\int_{0}^{2\pi} \theta-\frac{\sin(6\theta)}{6} \d \theta\\
&= \frac{1}{4}\eval{\theta-\frac{\sin(6\theta)}{6}}_{0}^{2\pi}\\
&= \frac{\pi}{2}
\end{align*}
How is it possible that the total area of the three petals is $\pi/4$,
but the integral above is $\pi/2$?
\begin{prompt}
  \begin{multipleChoice}
    \choice{one (or more!) of our computations has a mistake}
    \choice{$\pi/4 = \pi/2$}
    \choice[correct]{a complete curve is drawn when $\theta$ runs from $0$ to $\pi$}
  \end{multipleChoice}
  \begin{feedback}
    A good way to understand the polar graph is to graph $y=r(x)$
    along side of $r(\theta)$:
    \begin{image}
      \begin{tikzpicture}
	\begin{axis}[
            domain=0:pi,
            xmin=-.3,xmax=3.14,ymin=-1.3,ymax=1.3,
            axis lines =middle, xlabel=$x$, ylabel=$y$,
            every axis y label/.style={at=(current axis.above origin),anchor=south},
            every axis x label/.style={at=(current axis.right of origin),anchor=west},
            xtick={0,.785,...,6.28},
            xticklabels={$0$,$\frac{\pi}{6}$,$\frac{\pi}{3}$,$\frac{\pi}{2}$,$\frac{2\pi}{3}$,$\frac{5\pi}{6}$,$\pi$,$\frac{7\pi}{6}$,$\frac{4\pi}{3}$,$\frac{3\pi}{2}$,$\frac{5\pi}{3}$,$\frac{11\pi}{6}$,$2\pi$},
            ytick={-1,.5,0,.5,1},
          ]
          \addplot [very thick, penColor, smooth,domain=0:3.14] {cos(3*deg(x)};
        \end{axis}
      \end{tikzpicture}
      \qquad
      \begin{tikzpicture}
          \begin{polaraxis}[
              xtick={0,30,...,360},
              xticklabels={$0$,$\frac{\pi}{6}$,$\frac{\pi}{3}$,$\frac{\pi}{2}$,$\frac{2\pi}{3}$,$\frac{5\pi}{6}$,$\pi$,$\frac{7\pi}{6}$,$\frac{4\pi}{3}$,$\frac{3\pi}{2}$,$\frac{5\pi}{3}$,$\frac{11\pi}{6}$,$2\pi$},
              ytick={.5,1,...,2},
              yticklabels={},axis on top
            ]
            \addplot+[very thick, mark=none,penColor,domain=0:180,samples=100,smooth] {cos(3*x)};
          \end{polaraxis}
      \end{tikzpicture}
    \end{image}
  \end{feedback}
\end{prompt}
\end{question}

\section{Areas between polar curves}

Consider the shaded region:
  \begin{image}
    \begin{tikzpicture}
          \begin{polaraxis}[
              xtick={0,30,...,360},
              xticklabels={$0$,$\frac{\pi}{6}$,$\frac{\pi}{3}$,$\frac{\pi}{2}$,$\frac{2\pi}{3}$,$\frac{5\pi}{6}$,$\pi$,$\frac{7\pi}{6}$,$\frac{4\pi}{3}$,$\frac{3\pi}{2}$,$\frac{5\pi}{3}$,$\frac{11\pi}{6}$,$2\pi$},
              ytick={.5,1,...,2},
              yticklabels={},axis on top
            ]
            \addplot+[draw=none,very thick, mark=none,fill=fill1,domain=0:180,smooth] {1} \closedcycle;
            \addplot+[draw=none,very thick, mark=none,fill=white,domain=0:180,samples=100,smooth] {1-sin(x)}\closedcycle;

            \addplot+[draw=penColor,very thick, mark=none,domain=0:360,smooth] {1};
            \addplot+[draw=penColor2,very thick, mark=none,domain=0:360,samples=100,smooth] {1-sin(x)};
          \end{polaraxis}
    \end{tikzpicture}
  \end{image}
  If we let $r(\theta)$ represent the circle, and $s(\theta)$
  represent the cardioid, we can find the area of this region by
  computing the area bounded by $r(\theta)$ and subtracting the area
  bounded by $s(\theta)$ on $[\alpha,\beta]$. Thus
\begin{align*}
  \text{Area} &= \frac12\int_\alpha^\beta r(\theta)^2\d \theta -\frac12\int_\alpha^\beta s(\theta)^2\d\theta \\
  &= \frac12\int_\alpha^\beta
\big(r(\theta)^2-s(\theta)^2\big)\d\theta.
\end{align*}

\begin{theorem}[Area Between Polar Curves]
The area $A$ of the region bounded by $r(\theta)$ and $s(\theta)$,
$\theta=\alpha$ and $\theta=\beta$, where
\[
r(\theta)\ge s(\theta)
\]
on $[\alpha,\beta]$, is
\[
A = \frac{1}{2}\int_\alpha^\beta \big(r(\theta)^2-s(\theta)^2\big)\d \theta.
\]
\end{theorem}


\begin{example}
  Find the area of the region that lies within
  \[
  r(\theta) = 1
  \]
  but not within
  \[
  s(\theta) = 1-\sin(\theta)
  \]
  \begin{explanation}
    This corresponds to the following region:
    \begin{image}
    \begin{tikzpicture}
          \begin{polaraxis}[
              xtick={0,30,...,360},
              xticklabels={$0$,$\frac{\pi}{6}$,$\frac{\pi}{3}$,$\frac{\pi}{2}$,$\frac{2\pi}{3}$,$\frac{5\pi}{6}$,$\pi$,$\frac{7\pi}{6}$,$\frac{4\pi}{3}$,$\frac{3\pi}{2}$,$\frac{5\pi}{3}$,$\frac{11\pi}{6}$,$2\pi$},
              ytick={.5,1,...,2},
              yticklabels={},axis on top
            ]
            \addplot+[draw=none,very thick, mark=none,fill=fill1,domain=0:180,smooth] {1} \closedcycle;
            \addplot+[draw=none,very thick, mark=none,fill=white,domain=0:180,samples=100,smooth] {1-sin(x)}\closedcycle;

            \addplot+[draw=penColor,very thick, mark=none,domain=0:360,smooth] {1};
            \addplot+[draw=penColor2,very thick, mark=none,domain=0:360,samples=100,smooth] {1-sin(x)};
          \end{polaraxis}
    \end{tikzpicture}
    \end{image}
    Hence to find the are we compute
    \begin{align*}
      \frac{1}{2}\int_0^{\pi} r(\theta)^2 - s(\theta)^2 \d \theta &=  \frac{1}{2}\int_0^{\pi} 1 - \left(1-\sin(\theta)\right)^2 \d \theta \\
      &=  \frac{1}{2}\int_0^{\pi} 1 - \left(1 - 2\sin(\theta) + \sin^2(\theta)\right) \d \theta \\
      &=  \frac{1}{2}\int_0^{\pi} 1 -1 + 2\sin(\theta) - \sin^2(\theta) \d \theta \\
      &=  \frac{1}{2}\int_0^{\pi} 2\sin(\theta) - \sin^2(\theta) \d \theta \\
      &=  \frac{1}{2}\int_0^{\pi} 2\sin(\theta) - \frac{1-\cos(2\theta)}{2} \d \theta \\
      &=  \frac{1}{4}\int_0^{\pi} 4\sin(\theta) - 1+\cos(2\theta) \d \theta \\
      &=  \frac{1}{4}\eval{-4\cos(\theta) - \theta+\frac{\sin(2\theta)}{2}}_0^\pi \\
      &=  \answer[given]{2-\frac{\pi}{4}}.
    \end{align*}
  \end{explanation}
\end{example}


When computing regions between polar curves, one must be
careful. Sometimes, one must compute the points of intersection of the
two curves, and this can be difficult.
\begin{example}
  Find the points of intersection between the curves
  \begin{align*}
    r(\theta) &= 1+ \cos(\theta)\\
    s(\theta) &= 1+ \sin(\theta).
  \end{align*}
  \begin{explanation}
    To find the points of intersection of two polar curves, the best
    thing to do is to look at a graph.
    \begin{image}
      \begin{tikzpicture}
        \begin{polaraxis}[
            xtick={0,30,...,360},
            xticklabels={$0$,$\frac{\pi}{6}$,$\frac{\pi}{3}$,$\frac{\pi}{2}$,$\frac{2\pi}{3}$,$\frac{5\pi}{6}$,$\pi$,$\frac{7\pi}{6}$,$\frac{4\pi}{3}$,$\frac{3\pi}{2}$,$\frac{5\pi}{3}$,$\frac{11\pi}{6}$,$2\pi$},
            ytick={.5,1,...,2},
            yticklabels={},axis on top
          ]
          \addplot+[draw=penColor,very thick, mark=none,domain=0:360,samples=100,smooth] {1+cos(x)};
          \addplot+[draw=penColor2,very thick, mark=none,domain=0:360,samples=100,smooth] {1+sin(x)};
        \end{polaraxis}
      \end{tikzpicture}
    \end{image}
    From the graph above, we see that there are $3$ points of
    intersection.  Let's try to use algebra to find them. Write with me
    \begin{align*}
      1+ \cos(\theta) &= 1+ \sin(\theta)\\
      \cos(\theta) &=\sin(\theta)
    \end{align*}
    Ah, and this is true when:
    \[
    \theta = \answer[given]{\pi/4} \qquad\text{and}\qquad\theta= \answer[given]{5\pi/4}
    \]
    However, from the graph we see there is one more point of
    intersection. Namely
    \[
    r(\pi) = s\left(\answer[given]{3\pi/2}\right) = 0.
    \]
    When you are unable to look at a graph, you can
    still use algebra to try and find the points of intersection, but
    some may be hidden.
  \end{explanation}
\end{example}


\begin{example}
  Find the area of the region that lies within both polar curves
  \begin{align*}
    r(\theta) &= 1+ \cos(\theta)\\
    s(\theta) &= 1+ \sin(\theta).
  \end{align*}
  \begin{explanation}
    To compute this area, we'll use the previous example. First we'll compute the area of this region:
    \begin{image}
      \begin{tikzpicture}
        \begin{polaraxis}[
            xtick={0,45,...,360},
            xticklabels={$0$,$\frac{\pi}{4}$,$\frac{\pi}{2}$,$\frac{3\pi}{4}$,$\pi$,$\frac{5\pi}{4}$,$\frac{3\pi}{2}$,$\frac{7\pi}{4}$,$2\pi$},
            ytick={.5,1,...,2},
            yticklabels={},axis on top
          ]
          %\addplot[fill=fill1,draw=none,very thick, mark=none,domain=45:135,samples=100,smooth] {1+cos(x)}\closedcycle;
          
          \addplot[draw=penColor,very thick, mark=none,domain=0:360,samples=100,smooth] {1+cos(x)};
          
          \addplot[draw=penColor2,very thick, mark=none,domain=0:360,samples=100,smooth] {1+sin(x)};
        \end{polaraxis}
      \end{tikzpicture}
    \end{image}
    So to compute the region we can either compute:
    \[
    \int_{\pi/4}^{5\pi/4} (1+\cos(\theta))^2 \d \theta \qquad\text{or}\qquad\int_{-3\pi/4}^{\pi/4} (1+\sin(\theta))^2 \d \theta
    \]
    Evaluating either integral tells us the area is
    $\answer[given]{3\pi-4\sqrt{2}}$.
  \end{explanation}
\end{example}


\end{document}
















