\documentclass{ximera}
\author{Jim Talamo}
%\usepackage{todonotes}
%\usepackage{mathtools} %% Required for wide table Curl and Greens
%\usepackage{cuted} %% Required for wide table Curl and Greens
\newcommand{\todo}{}

\usepackage{esint} % for \oiint
\ifxake%%https://math.meta.stackexchange.com/questions/9973/how-do-you-render-a-closed-surface-double-integral
\renewcommand{\oiint}{{\large\bigcirc}\kern-1.56em\iint}
\fi


\graphicspath{
  {./}
  {ximeraTutorial/}
  {basicPhilosophy/}
  {functionsOfSeveralVariables/}
  {normalVectors/}
  {lagrangeMultipliers/}
  {vectorFields/}
  {greensTheorem/}
  {shapeOfThingsToCome/}
  {dotProducts/}
  {partialDerivativesAndTheGradientVector/}
  {../productAndQuotientRules/exercises/}
  {../normalVectors/exercisesParametricPlots/}
  {../continuityOfFunctionsOfSeveralVariables/exercises/}
  {../partialDerivativesAndTheGradientVector/exercises/}
  {../directionalDerivativeAndChainRule/exercises/}
  {../commonCoordinates/exercisesCylindricalCoordinates/}
  {../commonCoordinates/exercisesSphericalCoordinates/}
  {../greensTheorem/exercisesCurlAndLineIntegrals/}
  {../greensTheorem/exercisesDivergenceAndLineIntegrals/}
  {../shapeOfThingsToCome/exercisesDivergenceTheorem/}
  {../greensTheorem/}
  {../shapeOfThingsToCome/}
  {../separableDifferentialEquations/exercises/}
  {vectorFields/}
}

\newcommand{\mooculus}{\textsf{\textbf{MOOC}\textnormal{\textsf{ULUS}}}}

\usepackage{tkz-euclide}\usepackage{tikz}
\usepackage{tikz-cd}
\usetikzlibrary{arrows}
\tikzset{>=stealth,commutative diagrams/.cd,
  arrow style=tikz,diagrams={>=stealth}} %% cool arrow head
\tikzset{shorten <>/.style={ shorten >=#1, shorten <=#1 } } %% allows shorter vectors

\usetikzlibrary{backgrounds} %% for boxes around graphs
\usetikzlibrary{shapes,positioning}  %% Clouds and stars
\usetikzlibrary{matrix} %% for matrix
\usepgfplotslibrary{polar} %% for polar plots
\usepgfplotslibrary{fillbetween} %% to shade area between curves in TikZ
\usetkzobj{all}
\usepackage[makeroom]{cancel} %% for strike outs
%\usepackage{mathtools} %% for pretty underbrace % Breaks Ximera
%\usepackage{multicol}
\usepackage{pgffor} %% required for integral for loops



%% http://tex.stackexchange.com/questions/66490/drawing-a-tikz-arc-specifying-the-center
%% Draws beach ball
\tikzset{pics/carc/.style args={#1:#2:#3}{code={\draw[pic actions] (#1:#3) arc(#1:#2:#3);}}}



\usepackage{array}
\setlength{\extrarowheight}{+.1cm}
\newdimen\digitwidth
\settowidth\digitwidth{9}
\def\divrule#1#2{
\noalign{\moveright#1\digitwidth
\vbox{\hrule width#2\digitwidth}}}





\newcommand{\RR}{\mathbb R}
\newcommand{\R}{\mathbb R}
\newcommand{\N}{\mathbb N}
\newcommand{\Z}{\mathbb Z}

\newcommand{\sagemath}{\textsf{SageMath}}


%\renewcommand{\d}{\,d\!}
\renewcommand{\d}{\mathop{}\!d}
\newcommand{\dd}[2][]{\frac{\d #1}{\d #2}}
\newcommand{\pp}[2][]{\frac{\partial #1}{\partial #2}}
\renewcommand{\l}{\ell}
\newcommand{\ddx}{\frac{d}{\d x}}

\newcommand{\zeroOverZero}{\ensuremath{\boldsymbol{\tfrac{0}{0}}}}
\newcommand{\inftyOverInfty}{\ensuremath{\boldsymbol{\tfrac{\infty}{\infty}}}}
\newcommand{\zeroOverInfty}{\ensuremath{\boldsymbol{\tfrac{0}{\infty}}}}
\newcommand{\zeroTimesInfty}{\ensuremath{\small\boldsymbol{0\cdot \infty}}}
\newcommand{\inftyMinusInfty}{\ensuremath{\small\boldsymbol{\infty - \infty}}}
\newcommand{\oneToInfty}{\ensuremath{\boldsymbol{1^\infty}}}
\newcommand{\zeroToZero}{\ensuremath{\boldsymbol{0^0}}}
\newcommand{\inftyToZero}{\ensuremath{\boldsymbol{\infty^0}}}



\newcommand{\numOverZero}{\ensuremath{\boldsymbol{\tfrac{\#}{0}}}}
\newcommand{\dfn}{\textbf}
%\newcommand{\unit}{\,\mathrm}
\newcommand{\unit}{\mathop{}\!\mathrm}
\newcommand{\eval}[1]{\bigg[ #1 \bigg]}
\newcommand{\seq}[1]{\left( #1 \right)}
\renewcommand{\epsilon}{\varepsilon}
\renewcommand{\phi}{\varphi}


\renewcommand{\iff}{\Leftrightarrow}

\DeclareMathOperator{\arccot}{arccot}
\DeclareMathOperator{\arcsec}{arcsec}
\DeclareMathOperator{\arccsc}{arccsc}
\DeclareMathOperator{\si}{Si}
\DeclareMathOperator{\scal}{scal}
\DeclareMathOperator{\sign}{sign}


%% \newcommand{\tightoverset}[2]{% for arrow vec
%%   \mathop{#2}\limits^{\vbox to -.5ex{\kern-0.75ex\hbox{$#1$}\vss}}}
\newcommand{\arrowvec}[1]{{\overset{\rightharpoonup}{#1}}}
%\renewcommand{\vec}[1]{\arrowvec{\mathbf{#1}}}
\renewcommand{\vec}[1]{{\overset{\boldsymbol{\rightharpoonup}}{\mathbf{#1}}}\hspace{0in}}

\newcommand{\point}[1]{\left(#1\right)} %this allows \vector{ to be changed to \vector{ with a quick find and replace
\newcommand{\pt}[1]{\mathbf{#1}} %this allows \vec{ to be changed to \vec{ with a quick find and replace
\newcommand{\Lim}[2]{\lim_{\point{#1} \to \point{#2}}} %Bart, I changed this to point since I want to use it.  It runs through both of the exercise and exerciseE files in limits section, which is why it was in each document to start with.

\DeclareMathOperator{\proj}{\mathbf{proj}}
\newcommand{\veci}{{\boldsymbol{\hat{\imath}}}}
\newcommand{\vecj}{{\boldsymbol{\hat{\jmath}}}}
\newcommand{\veck}{{\boldsymbol{\hat{k}}}}
\newcommand{\vecl}{\vec{\boldsymbol{\l}}}
\newcommand{\uvec}[1]{\mathbf{\hat{#1}}}
\newcommand{\utan}{\mathbf{\hat{t}}}
\newcommand{\unormal}{\mathbf{\hat{n}}}
\newcommand{\ubinormal}{\mathbf{\hat{b}}}

\newcommand{\dotp}{\bullet}
\newcommand{\cross}{\boldsymbol\times}
\newcommand{\grad}{\boldsymbol\nabla}
\newcommand{\divergence}{\grad\dotp}
\newcommand{\curl}{\grad\cross}
%\DeclareMathOperator{\divergence}{divergence}
%\DeclareMathOperator{\curl}[1]{\grad\cross #1}
\newcommand{\lto}{\mathop{\longrightarrow\,}\limits}

\renewcommand{\bar}{\overline}

\colorlet{textColor}{black}
\colorlet{background}{white}
\colorlet{penColor}{blue!50!black} % Color of a curve in a plot
\colorlet{penColor2}{red!50!black}% Color of a curve in a plot
\colorlet{penColor3}{red!50!blue} % Color of a curve in a plot
\colorlet{penColor4}{green!50!black} % Color of a curve in a plot
\colorlet{penColor5}{orange!80!black} % Color of a curve in a plot
\colorlet{penColor6}{yellow!70!black} % Color of a curve in a plot
\colorlet{fill1}{penColor!20} % Color of fill in a plot
\colorlet{fill2}{penColor2!20} % Color of fill in a plot
\colorlet{fillp}{fill1} % Color of positive area
\colorlet{filln}{penColor2!20} % Color of negative area
\colorlet{fill3}{penColor3!20} % Fill
\colorlet{fill4}{penColor4!20} % Fill
\colorlet{fill5}{penColor5!20} % Fill
\colorlet{gridColor}{gray!50} % Color of grid in a plot

\newcommand{\surfaceColor}{violet}
\newcommand{\surfaceColorTwo}{redyellow}
\newcommand{\sliceColor}{greenyellow}




\pgfmathdeclarefunction{gauss}{2}{% gives gaussian
  \pgfmathparse{1/(#2*sqrt(2*pi))*exp(-((x-#1)^2)/(2*#2^2))}%
}


%%%%%%%%%%%%%
%% Vectors
%%%%%%%%%%%%%

%% Simple horiz vectors
\renewcommand{\vector}[1]{\left\langle #1\right\rangle}


%% %% Complex Horiz Vectors with angle brackets
%% \makeatletter
%% \renewcommand{\vector}[2][ , ]{\left\langle%
%%   \def\nextitem{\def\nextitem{#1}}%
%%   \@for \el:=#2\do{\nextitem\el}\right\rangle%
%% }
%% \makeatother

%% %% Vertical Vectors
%% \def\vector#1{\begin{bmatrix}\vecListA#1,,\end{bmatrix}}
%% \def\vecListA#1,{\if,#1,\else #1\cr \expandafter \vecListA \fi}

%%%%%%%%%%%%%
%% End of vectors
%%%%%%%%%%%%%

%\newcommand{\fullwidth}{}
%\newcommand{\normalwidth}{}



%% makes a snazzy t-chart for evaluating functions
%\newenvironment{tchart}{\rowcolors{2}{}{background!90!textColor}\array}{\endarray}

%%This is to help with formatting on future title pages.
\newenvironment{sectionOutcomes}{}{}



%% Flowchart stuff
%\tikzstyle{startstop} = [rectangle, rounded corners, minimum width=3cm, minimum height=1cm,text centered, draw=black]
%\tikzstyle{question} = [rectangle, minimum width=3cm, minimum height=1cm, text centered, draw=black]
%\tikzstyle{decision} = [trapezium, trapezium left angle=70, trapezium right angle=110, minimum width=3cm, minimum height=1cm, text centered, draw=black]
%\tikzstyle{question} = [rectangle, rounded corners, minimum width=3cm, minimum height=1cm,text centered, draw=black]
%\tikzstyle{process} = [rectangle, minimum width=3cm, minimum height=1cm, text centered, draw=black]
%\tikzstyle{decision} = [trapezium, trapezium left angle=70, trapezium right angle=110, minimum width=3cm, minimum height=1cm, text centered, draw=black]


\title[Dig-In:]{Accumulated cross-sections}

\outcome{Apply the procedure of ``Slice, Approximate, Integrate" to compute volumes of solids with given cross-sections.}

\begin{document}
\begin{abstract}
  We can also use the procedure of ``Slice, Approximate, Integrate" to set up integrals to compute volumes.
\end{abstract}
\maketitle

\section{Accumulation of cross-sections}

We have seen how to compute certain areas by using integration. The same technique used to find those areas can be applied to find volumes as well! In this section, we consider volumes whose cross-sections taken through their bases are common shapes from geometry.  In fact, we can think of these cross-sections as being ``slabs'' that we are layer either next to each other or over each other to obtain the solid in question.\index{slabs}\index{cross-sections}  We begin with a motivating example.


%%%%%%%%

\paragraph{Motivating Example:} The base of a solid is the region in the $xy$-plane bounded by $x=4-y^2$ and the $y$-axis.  Slices through the solid that are perpendicular to the $x$-axis are squares.  

\begin{image}
\begin{tikzpicture}
  \begin{axis}[
      xmin=-.3, xmax=4.8,domain=-1:1,
      ymin=-.4,ymax=1.7,
      clip=false,
      ytick={0,.8},yticklabels={$0$,$1$},
      axis lines =center, xlabel=$x$, ylabel=$z$,
      every axis y label/.style={at=(current axis.above origin),anchor=south},
      every axis x label/.style={at=(current axis.right of origin),anchor=west},
      axis on top,
    ] 
    \addplot [penColor2,very thick,domain=-.65:4,samples=150] {-.2*sqrt(4-x)};
     \addplot [penColor2,very thick,domain=.65:3.9,samples=150,dashed] {.05+.2*sqrt(4-x)};
    

  
	
    	\addplot [draw=penColor, fill=none,very thick] plot coordinates { (.63,1.57)(-.63,.73)(-.63,-.42)};
	\addplot [draw=penColor, fill=none,very thick,dashed] plot coordinates {(-.63,-.42)  (.63,.42) (.63,1.57) };

    \node at (axis cs:3,-0.4) [penColor2] {$x=4-y^2$};

    \addplot [black,very thick,domain=.63:4,samples=100] {sqrt((4-x)/1.3672)};
     \addplot [black,very thick,domain=-.63:4,samples=100] {sqrt((4-x)/8.688)};

    \addplot [->] plot coordinates {(-1.2,-.8) (1.2,0.8)};
    \node[anchor=south west] at (axis cs:1.2,0.6) {$y$};
    
               
\addplot [name path=A,domain=-.65:4,samples=150] {-.2*sqrt(4-x)};
\addplot [name path=B,domain=-.63:4,samples=100] {sqrt((4-x)/8.688)};
\addplot [fillp] fill between[of=A and B];
               
\addplot [name path=C,domain=-.63:4,samples=100] {sqrt((4-x)/8.688)};
\addplot [name path=D,domain=.63:4,samples=100] {sqrt((4-x)/1.3672)};
\addplot [fillp!60] fill between[of=C and D];


  \end{axis}
\end{tikzpicture}
\end{image}

How do we find the volume of this solid?

\begin{example}
We can apply the procedure of 	``Slice, Approximate, Integrate" to set up an integral that computes the volume of the solid.

\paragraph{Step 1: Slice} Since the cross-sections are perpendicular to the $x$-axis, our slices should be perpendicular to the $x$-axis on the base of the region.  This means:

\begin{multipleChoice}
\choice{the slices are horizontal.}
\choice[correct]{the slices are vertical.}
\end{multipleChoice}  

The slices on the base are one side of the square. These slices are shown below:

\begin{image}
\begin{tikzpicture}
\begin{axis}[
            domain=-3:4, ymax=3.5,xmax=4.5, ymin=-3, xmin=-1,
            axis lines =center, xlabel=$x$, ylabel=$y$,
            every axis y label/.style={at=(current axis.above origin),anchor=south},
            every axis x label/.style={at=(current axis.right of origin),anchor=west},
            axis on top,
          ]
          
\addplot[draw=penColor2,very thick,smooth] (4-x^2,x);

           
\addplot [name path=A,domain=0:4,samples=50,draw=none] {sqrt(4-x)}; 
\addplot [name path=B,domain=0:4,samples=50,draw=none] {-sqrt(4-x)};  
\addplot [fillp] fill between[of=A and B];

%Dark Shade
\addplot [name path=C,domain=1:2,samples=50,draw=none] {sqrt(4-x)}; 
\addplot [name path=D,domain=1:2,samples=50,draw=none] {-sqrt(4-x)};  
\addplot [gray!50] fill between[of=C and D];


\addplot [draw=penColor,very thick] coordinates {(1,-1.713)(1,1.713)};
\addplot [draw=penColor,very thick] coordinates {(2,-1.4)(2,1.4)};
\addplot [draw=penColor,very thick] coordinates {(3,-1)(3,1)};


\node at (axis cs:3,-1.75) [penColor2] {$x=4-y^2$};

\end{axis}
\end{tikzpicture}

\end{image}

Recall that these slices on the base are the side of a square cross-section.  The other side extends above the $xy$-plane.  Thus, the actual slices on the solid look like:

\begin{image}
\begin{tikzpicture}
  \begin{axis}[
      xmin=-.3, xmax=4.8,domain=-1:1,
      ymin=-.4,ymax=1.7,
      clip=false,
      ytick={0,.8},yticklabels={$0$,$1$},
      axis lines =center, xlabel=$x$, ylabel=$z$,
      every axis y label/.style={at=(current axis.above origin),anchor=south},
      every axis x label/.style={at=(current axis.right of origin),anchor=west},
      axis on top,
    ] 
    \addplot [penColor2,very thick,domain=-.65:4,samples=150] {-.2*sqrt(4-x)};
    \addplot [penColor2,very thick,domain=.65:4,samples=150,dashed] {.2*sqrt(4-x)};
    

  %Adds slices
  	\addplot [draw=penColor, fill=none,very thick] plot coordinates {(3.25,.74)(2.65,.394)(2.65,-.233) };
     	\addplot [draw=penColor, fill=none,very thick] plot coordinates { (2.37,1.09) (1.5,.536)(1.5,-.32)};
	\addplot [draw=penColor, fill=none,very thick] plot coordinates {(1.47,1.36) (.43,.64)(.43,-.38)};
	\addplot [draw=penColor, fill=none,very thick] plot coordinates { (.63,1.57)(-.63,.73)(-.63,-.42)};

%adds dashed
	
	\addplot [draw=penColor, fill=none,very thick,dashed] plot coordinates {(2.65,-.233) (3.25,.167) (3.25,.74) };
     	\addplot [draw=penColor, fill=none,very thick,dashed] plot coordinates {(1.5,-.32) (2.37,.246) (2.37,1.09) };
     	\addplot [draw=penColor, fill=none,very thick,dashed] plot coordinates {(.43,-.38) (1.47,.316) (1.47,1.36) };
	\addplot [draw=penColor, fill=none,very thick,dashed] plot coordinates {(-.63,-.42)  (.63,.42) (.63,1.57) };

	
	%%%Adds label
	
    \node at (axis cs:3,-0.4) [penColor2] {$x=4-y^2$};

    \addplot [black,very thick,domain=.63:4,samples=100] {sqrt((4-x)/1.3672)};
     \addplot [black,very thick,domain=-.63:4,samples=100] {sqrt((4-x)/8.688)};

    \addplot [->] plot coordinates {(-1.2,-.8) (1.2,0.8)};
    \node[anchor=south west] at (axis cs:1.2,0.6) {$y$};
    
    \addplot [name path=A,domain=-.65:4,samples=150] {-.2*sqrt(4-x)};
\addplot [name path=B,domain=-.63:4,samples=100] {sqrt((4-x)/8.688)};
\addplot [fillp] fill between[of=A and B];
               
\addplot [name path=C,domain=-.63:4,samples=100] {sqrt((4-x)/8.688)};
\addplot [name path=D,domain=.63:4,samples=100] {sqrt((4-x)/1.3672)};
\addplot [fillp!60] fill between[of=C and D];

%Dark Shade

   \addplot [name path=E,domain=.43:1.47,samples=150] {-.2*sqrt(4-x)};
\addplot [name path=F,domain=.43:1.47,samples=100] {sqrt((4-x)/8.688)};
\addplot [gray!50] fill between[of=E and F];
               
\addplot [name path=G,domain=.43:1.47,samples=100] {sqrt((4-x)/8.688)};
\addplot [name path=H,domain=1.5:2.37,samples=100] {sqrt((4-x)/1.3672)};
\addplot [gray!50] fill between[of=G and H];


  \end{axis}
\end{tikzpicture}
\end{image}

\paragraph{Step 2: Approximate}

As usual, we approximate the slices on the base as rectangles.  For the sake of example, we have used righthand endpoints and rectangles of uniform width $\Delta x$ to draw the picture:

%%%%BASE%%%%%%%%%%%%%
\begin{image}
\begin{tikzpicture}
\begin{axis}[
            domain=-3:4, ymax=3.5,xmax=4.5, ymin=-3, xmin=-1,
            axis lines =center, xlabel=$x$, ylabel=$y$,
            every axis y label/.style={at=(current axis.above origin),anchor=south},
            every axis x label/.style={at=(current axis.right of origin),anchor=west},
            axis on top,
          ]
          
\addplot[draw=penColor2,very thick,smooth] (4-x^2,x);

\addplot [draw=penColor,very thick,fill=fillp] coordinates {(0,-1.713)(1,-1.713)(1,1.713)(0,1.713)(0,-1.713)};
\addplot [draw=penColor,very thick,fill=gray!50] coordinates {(1,-1.4)(2,-1.4)(2,1.4)(1,1.4)(1,-1.4)};
\addplot [draw=penColor,very thick,fill=fillp] coordinates {(2,-1)(3,-1)(3,1)(2,1)(2,-1)};

  %Adds Delta x
    \draw[decoration={brace,mirror, raise=.35cm},decorate,thin] (axis cs:1,-1.4)--(axis cs:2,-1.4);  
    \node[anchor=north] at (axis cs:1.5,-1.85) {$\Delta x$};
%Adds s

    \draw[decoration={brace, raise=.1cm},decorate,thin] (axis cs:1,-1.4)--(axis cs:1,1.4);  
    \node[anchor=north] at (axis cs:.75,.5) {$s$};
    
    
\node at (axis cs:3,-1.75) [penColor2] {$x=4-y^2$};

\end{axis}
\end{tikzpicture}

\end{image}

The corresponding approximate slices on the solid are shown below:
%%%%%%%%%%%%%SOLID%%%%%%%%%

\begin{image}
\begin{tikzpicture}
  \begin{axis}[
      xmin=-.3, xmax=4.8,domain=-1:1,
      ymin=-.4,ymax=1.7,
      clip=false,
      ytick={0,.8},yticklabels={$0$,$1$},
      axis lines =center, xlabel=$x$, ylabel=$z$,
      every axis y label/.style={at=(current axis.above origin),anchor=south},
      every axis x label/.style={at=(current axis.right of origin),anchor=west},
      axis on top,
    ] 
 
    
	\addplot [penColor2,very thick,domain=-.65:4,samples=150,] {-.2*sqrt(4-x)};
 	\addplot [penColor2,very thick,domain=.65:4,samples=150,dashed] {.2*sqrt(4-x)};
  
	     	     	
    	\addplot [draw=penColor, fill=none,very thick] plot coordinates { (.63,.78) (.63,1.57)(-.63,.73)(-.63,-.42)};

%Draws the approximation 1
\addplot [draw=penColor, fill=fillp,very thick] plot coordinates {(.43,-.38) (1.47,.316) (1.47,1.36) (.43,.64)(.43,-.38)};
\addplot [draw=penColor, fill=fillp,very thick] plot coordinates {(.43,-.38) (1.43-2,-.38) (1.43-2,.64) (.43,.64)(.43,-.38)};
\addplot [draw=penColor, fill=fillp,very thick] plot coordinates {(.43,.64) (1.43-2,.64) (2.47-2,1.36)  (1.47,1.36)(.43,.64)};

%Draws the approximation 2
\addplot [draw=penColor, fill=gray!50,very thick] plot coordinates {(1.5,-.32) (2.37,.246) (2.37,1.09) (1.5,.536)(1.5,-.32)};
\addplot [draw=penColor, fill=gray!50,very thick] plot coordinates {(2.5-2,-.32) (3.37-2,.246) (3.37-2,1.09) (2.5-2,.536)(2.5-2,-.32)};
\addplot [draw=penColor, fill=gray!50,very thick] plot coordinates {(1.5,-.32) (2.5-2,-.32) (2.5-2,.536)(1.5,.536) (1.5,-.32)};
\addplot [draw=penColor, fill=gray!50,very thick] plot coordinates {(1.5,.536) (2.5-2,.536) (3.37-2,1.09)  (2.37,1.09)(1.5,.536) };

%Draws the approximation 3
\addplot [draw=penColor, fill=fillp,very thick] plot coordinates {(2.65,-.233) (3.25,.167) (3.25,.74) (2.65,.394)(2.65,-.233)};
\addplot [draw=penColor, fill=fillp,very thick] plot coordinates {(3.65-2,-.233) (4.25-2,.167) (4.25-2,.74) (3.65-2,.394)(3.65-2,-.233)};
\addplot [draw=penColor, fill=fillp,very thick] plot coordinates {(2.65,-.233)(3.65-2,-.233) (3.65-2,.394) (2.65,.394)(2.65,-.233)};
\addplot [draw=penColor, fill=fillp,very thick] plot coordinates {(3.25,.74) (4.25-2,.74) (3.65-2,.394) (2.65,.394)(3.25,.74)};

    \node at (axis cs:3,-0.4) [penColor2] {$x=4-y^2$};

    	\addplot [black,very thick,domain=.63:4,samples=100] {sqrt((4-x)/1.3672)};
    	\addplot [black,very thick,domain=-.63:4,samples=100] {sqrt((4-x)/8.688)};

%Adds Delta x
   \draw[decoration={brace,mirror, raise=.35cm},decorate,thin] (axis cs:.5,-.32)--(axis cs:1.47,-.32);  
    \node[anchor=north] at (axis cs:1,-.52) {$\Delta x$};

    \addplot [->] plot coordinates {(-1.2,-.8) (1.2,0.8)};
    \node[anchor=south west] at (axis cs:1.2,0.6) {$y$};
  \end{axis}
\end{tikzpicture}
\end{image}

The dark slab is a rectangular prism, so its volume is:

\begin{align*}
\textrm{Volume} &= \textrm{Area} \times \textrm{thickness} \\
\Delta V &= s^2 \Delta x
\end{align*}

Since we are slicing with respect to $x$, we must express the side length $s$ in terms of $x$.  We should thus express the curves in the image as functions of $y$.  

Note that the parabola $x=4-y^2$ must be described using $\answer[given]{2}$ functions of $x$!  In fact, solving for $y$, we obtain: $y=+\answer[given]{\sqrt{4-x}}$ and $y=-\answer[given]{\sqrt{4-x}}$.

We draw a more helpful image:

\begin{image}
\begin{tikzpicture}
\begin{axis}[
            domain=-3:4, ymax=3.5,xmax=4.5, ymin=-3, xmin=-1,
            axis lines =center, xlabel=$x$, ylabel=$y$,
            every axis y label/.style={at=(current axis.above origin),anchor=south},
            every axis x label/.style={at=(current axis.right of origin),anchor=west},
            axis on top,
          ]
          
\addplot[draw=penColor2,very thick,smooth,domain=-1:4,samples=150] {sqrt(4-x)};
\addplot[draw=penColor,very thick,smooth,domain=-1:4,samples=150] {-sqrt(4-x)};
           
\addplot [name path=A,domain=0:4,samples=50,draw=none] {sqrt(4-x)}; 
\addplot [name path=B,domain=0:4,samples=50,draw=none] {-sqrt(4-x)};  
\addplot [fillp] fill between[of=A and B];





\addplot [draw=black,very thick,fill=gray!50] coordinates {(1,-1.4)(2,-1.4)(2,1.4)(1,1.4)(1,-1.4)};


  %Adds Delta x
    \draw[decoration={brace,mirror, raise=.35cm},decorate,thin] (axis cs:1,-1.4)--(axis cs:2,-1.4);  
    \node[anchor=north] at (axis cs:1.5,-1.85) {$\Delta x$};
%Adds s

    \draw[decoration={brace, mirror, raise=.1cm},decorate,thin] (axis cs:2,-1.4)--(axis cs:2,1.4);  
    \node[anchor=north] at (axis cs:2.25,.5) {$s$};


\node at (axis cs:3,1.75) [penColor2] {$y=\sqrt{4-x}$};
\node at (axis cs:3,-1.75) [penColor] {$y=-\sqrt{4-x}$};

\end{axis}
\end{tikzpicture}
\end{image}

From the image, we see that $s$ is a:

\begin{multipleChoice}
\choice[correct]{vertical distance.}
\choice{horizontal distance.}
\end{multipleChoice}

Thus, we may find $s$ using:

\begin{multipleChoice}
\choice{$x_{right}-x_{left}$}
\choice[correct]{$y_{top}-y_{bot}$}
\end{multipleChoice}

From our picture, we find $y_{top} = \sqrt{\answer[given]{4-x}}$ and $y_{bot} =-\sqrt{ \answer[given]{4-x}}$.  Thus:

\[
s = \answer[given]{2\sqrt{4-x}}
\]

and using $\Delta V = s^2 \Delta x$, we have that the volume of a \emph{single} slice at a given $x$-value is:

\[
\Delta V = (\answer[given]{16-4x}) \Delta x
\]

\paragraph{Step 3: Integrate}
As before, the approximate volume is obtained by adding all of the volumes of all of the approximate slices together.  The exact volume requires simultaneously that the width of the slices become arbitrarily thin and that the number of slices to be added becomes arbitrarily large!

As before, the definite integral performs both of these operations simultaneously!  In fact, since $\Delta V = (16-4x) \Delta x$, we can write:

\[
V = \int_{x=0}^{x=4} 16-4x \d x
\]

Evaluating this, we find the volume is $\answer[given]{32}$ cubic units.

\end{example}


\textbf{Remark:} Since we slice with respect to $x$, we must express the curves in the image as functions of $x$; that is, we must write $y_{top}$ and $y_{bot}$ in terms of $x$.  Once we choose a variable of integration, every quantity (limits of integration, functions in the integrand) must be written in terms of that variable.  This is an important point that arises when we use a Riemann integral to compute any quantity of interest!


%%Add some image that delineates how to think of integral; draw for Hans




\section{Volumes of solids with known cross-sections}

We can summarize the previous procedure with a simple formula that respects the geometrical reasoning used to generate the volume of a solid with a known type of cross-section:

\begin{formula}
The volume of a solid with a known type of cross-sectional area is given by either: 

\[
V=\int_{x=a}^{x=b} A \d x \qquad \textrm{or} \qquad  V =\int_{y=c}^{y=d} A \d y
\]

where $A$ is the cross-sectional area of a slice of the solid.

Note that both the area $A$ and the limits of integration must be expressed in terms of the variable of integration.

\end{formula}

So how do we determine which formula to use?  The problem will indicate an orientation for the slice. Draw the base of the solid in the $xy$-plane, and indicate a prototypical slice on your picture. The orientation of the slice will give you the variable of integration!

\begin{question}
Suppose that slices are taken \emph{parallel} to the $y$-axis.  Then, the slices are
\begin{multipleChoice}
\choice[correct]{the slices are vertical.  We should integrate with respect to $x$.}
\choice{the slices are vertical.  We should integrate with respect to $y$.}
\choice{the slices are horizontal.  We should integrate with respect to $x$.}
\choice{the slices are horizontal.  We should integrate with respect to $y$.}
\end{multipleChoice}

Suppose that slices are taken \emph{perpendicular} to the $y$-axis.  Then, the slices are
\begin{multipleChoice}
\choice{the slices are vertical.  We should integrate with respect to $x$.}
\choice{the slices are vertical.  We should integrate with respect to $y$.}
\choice{the slices are horizontal.  We should integrate with respect to $x$.}
\choice[correct]{the slices are horizontal.  We should integrate with respect to $y$.}
\end{multipleChoice}
\end{question}

\begin{remark}
In order to work these problems, you need to remember some basic formulas from geometry!  Some important ones are:

\[
\begin{array}{ll}
\textrm{Shape} & \textrm{Area Formula} \\
\textrm{Square:} & A=s^2, \textrm{where } s \textrm{ is the length of a side of the square} \\ 
\textrm{Equilateral Triangle:} & A=\frac{3}{4} s^2, \textrm{where } s \textrm{ is the length of a side of the triangle} \\ 
\textrm{Right Isosceles Triangle:} & A=\frac{1}{2} s^2, \textrm{where } s \textrm{ is the length of one of the two equal length sides} \\ 
\textrm{Semicircle:} & A=\frac{1}{2} \pi r^2, \textrm{where } r \textrm{ is the radius of the circle} \\
\end{array}
\]

\end{remark}


%%%%%%%%%%%%%%%%%%%%%%%%%%%%%%


Let's see some examples:

\begin{example}
The base of a solid is the region in the $xy$-plane bounded by $y=9-3x$, $y=-3$, $x=-2$, and $x=0$.  cross-sections through the solid taken parallel to the $x$-axis are semicircles.  Find the volume of the solid.

\begin{explanation}
Since the slices are taken \emph{parallel} to the $x$-axis.  Then, the slices are
\begin{multipleChoice}
\choice{the slices are vertical.  We should integrate with respect to $x$.}
\choice{the slices are vertical.  We should integrate with respect to $y$.}
\choice{the slices are horizontal.  We should integrate with respect to $x$.}
\choice[correct]{the slices are horizontal.  We should integrate with respect to $y$.}
\end{multipleChoice}


Since we have to integrate with respect to $y$, we should describe the line $y=9-3x$ as a function of $y$.  Solving for $x$, we obtain: $x=\answer[given]{3- \frac{1}{3}y}$.

We draw a picture of the base and indicate a typical slice: 

\begin{image}
\begin{tikzpicture}
\begin{axis}[
            domain=-5:5, ymax=18.9,xmax=4.9, ymin=-5.9, xmin=-3.9,
            axis lines =center, xlabel=$x$, ylabel=$y$,
            every axis y label/.style={at=(current axis.above origin),anchor=south},
            every axis x label/.style={at=(current axis.right of origin),anchor=west},
            axis on top,
          ]
          
\addplot[draw=penColor2,very thick,smooth] {-3};
\addplot[draw=penColor,very thick,smooth] {9-3*x};

           
\addplot [name path=A,domain=-2:4,samples=50,draw=none] {9-3*x}; 
\addplot [name path=B,domain=-2:4,samples=50,draw=none] {-3};  
\addplot [fillp] fill between[of=A and B];
\addplot [draw=black!50!green,very thick,] coordinates {(-2,-6)(-2,20)};
\addplot [draw=penColor,very thick,fill=gray!50] coordinates {(-2,2.5)(2,2.5)(2,3)(-2,3)(-2,2.5)};


\node at (axis cs:3,5) [penColor] {$x=3-\frac{1}{3}y$};
\node at (axis cs:2,-4.5) [penColor2] {$y=-1$};
\node at (axis cs:-3,6) [black!50!green] {$x=-2$};

%Draws brace and d

    \draw[decoration={brace, mirror, raise=.1cm},decorate,thin] (axis cs:-2,2.3)--(axis cs:2,2.3);  
    \node[anchor=north] at (axis cs:0.3,2) {$d$};
    
    
\end{axis}
\end{tikzpicture}
\end{image}

Note that the cross-sections are semicircles whose base is in the $xy$-plane.  Thus, the quantity labelled $d$ in the image is the:
\begin{multipleChoice}
\choice{radius of the semicircle.}
\choice[correct]{diameter of the semicircle.}
\end{multipleChoice}

From the image, we see that $d$ is a:

\begin{multipleChoice}
\choice{vertical distance.}
\choice[correct]{horizontal distance.}
\end{multipleChoice}

Thus, we may find $s$ using:

\begin{multipleChoice}
\choice[correct]{$x_{right}-x_{left}$}
\choice{$y_{top}-y_{bot}$}
\end{multipleChoice}

The right curve is $x_{right} = \answer[given]{3-\frac{1}{3}y}$ and the left curve is $x_{left} = \answer[given]{-2}$.  Thus:

\[
d=x_{right}-x_{left} = \answer[given]{\left(3-\frac{1}{3} y \right)-(-2)}
\]
Notice that is is completely irrelevant of the quadrant in which the left and right curves appear; we can always find a horizontal quantity of interest (in this case $d$), by taking $x_{right}-x_{left}$ and using the expressions that describe the relevant curves in terms of $y$!
 
The radius $r$ of the semicircle is (after a little algebra):
\[
r = \frac{1}{2}d = \answer[given]{ \left(\frac{5}{2}-\frac{1}{6} y\right)}
\]
and the area of the semicircle is found using:

\[
A = \frac{1}{2} \pi r^2 = \answer[given]{\frac{1}{2} \pi\left(\frac{5}{2} - \frac{1}{6}y\right)^2}
\]

Thus, an integral that gives the volume of the solid is:

\[
V = \int_{y=\answer[given]{-3}}^{y=\answer[given]{15}} \answer[given]{\frac{1}{2} \pi\left(\frac{5}{2} - \frac{1}{6}y\right)^2} \d y
\]

Evaluating this integral (which you should verify by working it out on your own!), we find that the volume of the solid is $\answer[given]{27} \pi$ cubic units.
\end{explanation}

\end{example}

Sometimes, more than one integral is needed to set up a volume of a solid with known cross-sections!  If you draw a picture, it should be clear when this will be necessary!

\begin{example}
The base of a solid is the region in the first quadrant of the $xy$-plane bounded by $y=e^x$ and $x=1$. cross-sections taken perpendicular to the $y$-axis are equilateral triangles.  Set up, but do not evaluate, an integral or sum of integrals that would give the volume of the solid.

\begin{explanation}

Since the slices are taken \emph{perpendicular} to the $y$-axis.  Then, the slices are
\begin{multipleChoice}
\choice{the slices are vertical.  We should integrate with respect to $x$.}
\choice{the slices are vertical.  We should integrate with respect to $y$.}
\choice{the slices are horizontal.  We should integrate with respect to $x$.}
\choice[correct]{the slices are horizontal.  We should integrate with respect to $y$.}
\end{multipleChoice}

Since we have to integrate with respect to $y$, we should describe the line $y=e^x$ as a function of $y$.  Solving for $x$, we obtain: $x=\answer[given]{\ln(y)}$.

We draw a picture of the base and indicate a typical slice:
 
 \begin{image}
            \begin{tikzpicture}
            	\begin{axis}[
            domain=-10:1.8, ymax=3.4,xmax=1.8, ymin=-.4, xmin=-.4,
            axis lines =center, xlabel=$x$, ylabel=$y$,
            every axis y label/.style={at=(current axis.above origin),anchor=south},
            every axis x label/.style={at=(current axis.right of origin),anchor=west},
            axis on top,
          ]                      
        %curves
            	\addplot [draw=penColor,very thick,smooth] {e^x};
	       	\addplot [draw=penColor2,very thick,] coordinates {(1,-6)(1,20)};
		\addplot [draw=black!50!green,very thick,dashed] coordinates {(-6,1)(6,1)};
	                            
            	\addplot [name path=A,domain=0:1,draw=none] {e^x};   
            	\addplot [name path=B,domain=0:1,draw=none] {1};
	        \addplot [name path=C,domain=0:1,draw=none] {0};
            	\addplot [penColor!30] fill between[of=A and C];
	        \addplot [penColor2!30] fill between[of=B and C];
                      
                      
            	\node at (axis cs:.43,2.3) [penColor] {$x=\ln(y)$};
            	\node at (axis cs:1.2,2) [penColor2] {$x=1$};
		\node at (axis cs:1.5,.7) [black!50!green] {$y=1$};
	\node at (axis cs:.5,.7) [black!75] {I};
	\node at (axis cs:.6,1.3) [black!75] {II};
	
	%%%Draws the rectangles
	\addplot [draw=penColor, fill = gray!50] plot coordinates {(0,.4) (1,.4) (1, .5) (0,.5) (0,.4)};        
          \addplot [draw=penColor, fill = gray!50] plot coordinates {(ln(1.7),1.6) (1,1.6) (1, 1.7) (ln(1.7),1.7) (ln(1.7),1.6)};    
	
	%%%Draws dashed line
	\addplot [draw=black!75,thick,dashed] coordinates {(2,0)(2,7)};
	    
	    %%%% draws brace and d

    \draw[decoration={brace, mirror, raise=.1cm},decorate,thin] (axis cs:0,.4)--(axis cs:1,.4);  
    \node[anchor=north] at (axis cs:0.5,.3) {$s_1$};

    \draw[decoration={brace, mirror, raise=.2cm},decorate,thin] (axis cs:.53,1.7)--(axis cs:1,1.7);  
    \node[anchor=north] at (axis cs:.75,1.5) {$s_2$};
    
	      \end{axis}
            \end{tikzpicture}
            \end{image}
            
            
As you can see, the curve used to determine the lefthand coordinate of rectangle changes depending on where the slice is drawn! We thus need $\answer[given]{2}$ integrals!

The left curve will change at the $y$-value where the curve $x=ln(y)$ intersects the $y$-axis, which occurs when $y=\answer[given]{1}$.  

For the second region, we find the upper $y$-value by requiring that the curves $x=1$ and $x=\ln(y)$ intersect. This occurs when $y=\answer[given]{e}$.

Using the picture above, we may write down the important information about each region: 
\[
\begin{array}{ll}
\text{In Region I:}  & \text{In Region II:}  \\
\answer[given]{0} \le y \le \answer[given]{1}  \qquad \qquad \qquad & \answer[given]{1} \le y \le \answer[given]{e}\\
x_{right} = \answer[given]{1}    &  x_{right} = \answer[given]{1} \\
x_{left} = \answer[given]{0}  & x_{left} = \answer[given]{\ln(y)} \\
s_1= \answer[given]{1} & s_2= \answer[given]{1-\ln(y)} \\
A_1 = \frac{\sqrt{3}}{4} &  A_2  = \frac{\sqrt{3}}{4}(1-\ln(y))^2 \\
\end{array}
\]

Note that since the cross-sections are equilateral triangles, $A= \frac{\sqrt{3}}{4}s^2$.

Putting this together, we can write down a sum of integrals that gives the volume:

\[V= \int_{y= 0}^{y=1}\frac{\sqrt{3}}{4} \d y + \int_{y= 1}^{y=e} \frac{\sqrt{3}}{4}(1-\ln(y))^2 \d y \]

            
            
\end{explanation}            
\end{example}

\section{Final thoughts}
To summarize some recurring ideas we have seen we have seen (and will see again!), always draw and label a picture.  Interpret the quantities in your picture and write down the relevant geometric quantities in terms of the variable of integration.


\begin{fact}
To find vertical distances, we always take $y_{top} - y_{bot}$.

To find horizontal distances, we always take $x_{right}-x_{left}$.
\end{fact}

\begin{fact}
When we integrate with respect to $x$, we use vertical slices and when we use vertical slices, we integrate with respect to $x$.

When we integrate with respect to $y$, we use horizontal slices and when we use horizontal slices, we integrate with respect to $y$.
\end{fact}

\begin{fact}
Once we choose a variable of integration, everything in the integrand must be expressed in terms of that variable!  This includes both the limits of integration and any functions that arise in the integrand.
\end{fact}

Remember, it takes practice to learn math.  Don't just read through examples; work them out yourself as you read along. Calculus is a hard subject! Don't get discouraged!

\begin{quote}
``The only way to learn mathematics is to do mathematics.'' --- Paul Halmos
\end{quote}


\end{document}
