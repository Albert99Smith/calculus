\documentclass[nooutcomes]{ximera}
\author{Jim Talamo and Bart Snapp}
%\usepackage{todonotes}
%\usepackage{mathtools} %% Required for wide table Curl and Greens
%\usepackage{cuted} %% Required for wide table Curl and Greens
\newcommand{\todo}{}

\usepackage{esint} % for \oiint
\ifxake%%https://math.meta.stackexchange.com/questions/9973/how-do-you-render-a-closed-surface-double-integral
\renewcommand{\oiint}{{\large\bigcirc}\kern-1.56em\iint}
\fi


\graphicspath{
  {./}
  {ximeraTutorial/}
  {basicPhilosophy/}
  {functionsOfSeveralVariables/}
  {normalVectors/}
  {lagrangeMultipliers/}
  {vectorFields/}
  {greensTheorem/}
  {shapeOfThingsToCome/}
  {dotProducts/}
  {partialDerivativesAndTheGradientVector/}
  {../productAndQuotientRules/exercises/}
  {../normalVectors/exercisesParametricPlots/}
  {../continuityOfFunctionsOfSeveralVariables/exercises/}
  {../partialDerivativesAndTheGradientVector/exercises/}
  {../directionalDerivativeAndChainRule/exercises/}
  {../commonCoordinates/exercisesCylindricalCoordinates/}
  {../commonCoordinates/exercisesSphericalCoordinates/}
  {../greensTheorem/exercisesCurlAndLineIntegrals/}
  {../greensTheorem/exercisesDivergenceAndLineIntegrals/}
  {../shapeOfThingsToCome/exercisesDivergenceTheorem/}
  {../greensTheorem/}
  {../shapeOfThingsToCome/}
  {../separableDifferentialEquations/exercises/}
  {vectorFields/}
}

\newcommand{\mooculus}{\textsf{\textbf{MOOC}\textnormal{\textsf{ULUS}}}}

\usepackage{tkz-euclide}\usepackage{tikz}
\usepackage{tikz-cd}
\usetikzlibrary{arrows}
\tikzset{>=stealth,commutative diagrams/.cd,
  arrow style=tikz,diagrams={>=stealth}} %% cool arrow head
\tikzset{shorten <>/.style={ shorten >=#1, shorten <=#1 } } %% allows shorter vectors

\usetikzlibrary{backgrounds} %% for boxes around graphs
\usetikzlibrary{shapes,positioning}  %% Clouds and stars
\usetikzlibrary{matrix} %% for matrix
\usepgfplotslibrary{polar} %% for polar plots
\usepgfplotslibrary{fillbetween} %% to shade area between curves in TikZ
\usetkzobj{all}
\usepackage[makeroom]{cancel} %% for strike outs
%\usepackage{mathtools} %% for pretty underbrace % Breaks Ximera
%\usepackage{multicol}
\usepackage{pgffor} %% required for integral for loops



%% http://tex.stackexchange.com/questions/66490/drawing-a-tikz-arc-specifying-the-center
%% Draws beach ball
\tikzset{pics/carc/.style args={#1:#2:#3}{code={\draw[pic actions] (#1:#3) arc(#1:#2:#3);}}}



\usepackage{array}
\setlength{\extrarowheight}{+.1cm}
\newdimen\digitwidth
\settowidth\digitwidth{9}
\def\divrule#1#2{
\noalign{\moveright#1\digitwidth
\vbox{\hrule width#2\digitwidth}}}





\newcommand{\RR}{\mathbb R}
\newcommand{\R}{\mathbb R}
\newcommand{\N}{\mathbb N}
\newcommand{\Z}{\mathbb Z}

\newcommand{\sagemath}{\textsf{SageMath}}


%\renewcommand{\d}{\,d\!}
\renewcommand{\d}{\mathop{}\!d}
\newcommand{\dd}[2][]{\frac{\d #1}{\d #2}}
\newcommand{\pp}[2][]{\frac{\partial #1}{\partial #2}}
\renewcommand{\l}{\ell}
\newcommand{\ddx}{\frac{d}{\d x}}

\newcommand{\zeroOverZero}{\ensuremath{\boldsymbol{\tfrac{0}{0}}}}
\newcommand{\inftyOverInfty}{\ensuremath{\boldsymbol{\tfrac{\infty}{\infty}}}}
\newcommand{\zeroOverInfty}{\ensuremath{\boldsymbol{\tfrac{0}{\infty}}}}
\newcommand{\zeroTimesInfty}{\ensuremath{\small\boldsymbol{0\cdot \infty}}}
\newcommand{\inftyMinusInfty}{\ensuremath{\small\boldsymbol{\infty - \infty}}}
\newcommand{\oneToInfty}{\ensuremath{\boldsymbol{1^\infty}}}
\newcommand{\zeroToZero}{\ensuremath{\boldsymbol{0^0}}}
\newcommand{\inftyToZero}{\ensuremath{\boldsymbol{\infty^0}}}



\newcommand{\numOverZero}{\ensuremath{\boldsymbol{\tfrac{\#}{0}}}}
\newcommand{\dfn}{\textbf}
%\newcommand{\unit}{\,\mathrm}
\newcommand{\unit}{\mathop{}\!\mathrm}
\newcommand{\eval}[1]{\bigg[ #1 \bigg]}
\newcommand{\seq}[1]{\left( #1 \right)}
\renewcommand{\epsilon}{\varepsilon}
\renewcommand{\phi}{\varphi}


\renewcommand{\iff}{\Leftrightarrow}

\DeclareMathOperator{\arccot}{arccot}
\DeclareMathOperator{\arcsec}{arcsec}
\DeclareMathOperator{\arccsc}{arccsc}
\DeclareMathOperator{\si}{Si}
\DeclareMathOperator{\scal}{scal}
\DeclareMathOperator{\sign}{sign}


%% \newcommand{\tightoverset}[2]{% for arrow vec
%%   \mathop{#2}\limits^{\vbox to -.5ex{\kern-0.75ex\hbox{$#1$}\vss}}}
\newcommand{\arrowvec}[1]{{\overset{\rightharpoonup}{#1}}}
%\renewcommand{\vec}[1]{\arrowvec{\mathbf{#1}}}
\renewcommand{\vec}[1]{{\overset{\boldsymbol{\rightharpoonup}}{\mathbf{#1}}}\hspace{0in}}

\newcommand{\point}[1]{\left(#1\right)} %this allows \vector{ to be changed to \vector{ with a quick find and replace
\newcommand{\pt}[1]{\mathbf{#1}} %this allows \vec{ to be changed to \vec{ with a quick find and replace
\newcommand{\Lim}[2]{\lim_{\point{#1} \to \point{#2}}} %Bart, I changed this to point since I want to use it.  It runs through both of the exercise and exerciseE files in limits section, which is why it was in each document to start with.

\DeclareMathOperator{\proj}{\mathbf{proj}}
\newcommand{\veci}{{\boldsymbol{\hat{\imath}}}}
\newcommand{\vecj}{{\boldsymbol{\hat{\jmath}}}}
\newcommand{\veck}{{\boldsymbol{\hat{k}}}}
\newcommand{\vecl}{\vec{\boldsymbol{\l}}}
\newcommand{\uvec}[1]{\mathbf{\hat{#1}}}
\newcommand{\utan}{\mathbf{\hat{t}}}
\newcommand{\unormal}{\mathbf{\hat{n}}}
\newcommand{\ubinormal}{\mathbf{\hat{b}}}

\newcommand{\dotp}{\bullet}
\newcommand{\cross}{\boldsymbol\times}
\newcommand{\grad}{\boldsymbol\nabla}
\newcommand{\divergence}{\grad\dotp}
\newcommand{\curl}{\grad\cross}
%\DeclareMathOperator{\divergence}{divergence}
%\DeclareMathOperator{\curl}[1]{\grad\cross #1}
\newcommand{\lto}{\mathop{\longrightarrow\,}\limits}

\renewcommand{\bar}{\overline}

\colorlet{textColor}{black}
\colorlet{background}{white}
\colorlet{penColor}{blue!50!black} % Color of a curve in a plot
\colorlet{penColor2}{red!50!black}% Color of a curve in a plot
\colorlet{penColor3}{red!50!blue} % Color of a curve in a plot
\colorlet{penColor4}{green!50!black} % Color of a curve in a plot
\colorlet{penColor5}{orange!80!black} % Color of a curve in a plot
\colorlet{penColor6}{yellow!70!black} % Color of a curve in a plot
\colorlet{fill1}{penColor!20} % Color of fill in a plot
\colorlet{fill2}{penColor2!20} % Color of fill in a plot
\colorlet{fillp}{fill1} % Color of positive area
\colorlet{filln}{penColor2!20} % Color of negative area
\colorlet{fill3}{penColor3!20} % Fill
\colorlet{fill4}{penColor4!20} % Fill
\colorlet{fill5}{penColor5!20} % Fill
\colorlet{gridColor}{gray!50} % Color of grid in a plot

\newcommand{\surfaceColor}{violet}
\newcommand{\surfaceColorTwo}{redyellow}
\newcommand{\sliceColor}{greenyellow}




\pgfmathdeclarefunction{gauss}{2}{% gives gaussian
  \pgfmathparse{1/(#2*sqrt(2*pi))*exp(-((x-#1)^2)/(2*#2^2))}%
}


%%%%%%%%%%%%%
%% Vectors
%%%%%%%%%%%%%

%% Simple horiz vectors
\renewcommand{\vector}[1]{\left\langle #1\right\rangle}


%% %% Complex Horiz Vectors with angle brackets
%% \makeatletter
%% \renewcommand{\vector}[2][ , ]{\left\langle%
%%   \def\nextitem{\def\nextitem{#1}}%
%%   \@for \el:=#2\do{\nextitem\el}\right\rangle%
%% }
%% \makeatother

%% %% Vertical Vectors
%% \def\vector#1{\begin{bmatrix}\vecListA#1,,\end{bmatrix}}
%% \def\vecListA#1,{\if,#1,\else #1\cr \expandafter \vecListA \fi}

%%%%%%%%%%%%%
%% End of vectors
%%%%%%%%%%%%%

%\newcommand{\fullwidth}{}
%\newcommand{\normalwidth}{}



%% makes a snazzy t-chart for evaluating functions
%\newenvironment{tchart}{\rowcolors{2}{}{background!90!textColor}\array}{\endarray}

%%This is to help with formatting on future title pages.
\newenvironment{sectionOutcomes}{}{}



%% Flowchart stuff
%\tikzstyle{startstop} = [rectangle, rounded corners, minimum width=3cm, minimum height=1cm,text centered, draw=black]
%\tikzstyle{question} = [rectangle, minimum width=3cm, minimum height=1cm, text centered, draw=black]
%\tikzstyle{decision} = [trapezium, trapezium left angle=70, trapezium right angle=110, minimum width=3cm, minimum height=1cm, text centered, draw=black]
%\tikzstyle{question} = [rectangle, rounded corners, minimum width=3cm, minimum height=1cm,text centered, draw=black]
%\tikzstyle{process} = [rectangle, minimum width=3cm, minimum height=1cm, text centered, draw=black]
%\tikzstyle{decision} = [trapezium, trapezium left angle=70, trapezium right angle=110, minimum width=3cm, minimum height=1cm, text centered, draw=black]



\outcome{Review the basic rules of differentiation.}
\outcome{Review the relationship between the derivative and tangent lines.}

%I want a separate section that reviews the FTC

\title[Dig-In:]{A review of integration}

\begin{document}
\begin{abstract}
  We review differentiation and integration.
\end{abstract}
\maketitle


\section{Review of derivative rules}

One of the fundamental objects of differential calculus is the \emph{derivative}.  As a reminder, here are some important results:

\index{derivative rules} 
  Let $n\ne 0$, $k$ be a constant and $a>0$.


\paragraph{Powers of $x$:}
\begin{itemize}
\item $\ddx k =0$
\item $\ddx x^n  = \answer{n x^{n-1}}$
\end{itemize}

\paragraph{Exponentials:}
\begin{itemize}
\item $\ddx e^x = \answer{e^x}$
\item $\ddx a^x = a^x\ln(a)$
\end{itemize}

\paragraph{Logarithms:}
\begin{itemize}
\item $\ddx \ln(x) = \frac{1}{x}$
\item $\ddx \log_a (x) = \frac{1}{\ln a} \frac{1}{x}$
\end{itemize}

\paragraph{Trigonometric Functions:}
\begin{itemize}
\item $\ddx \sin(x) = \answer{\cos(x)}$
\item $\ddx \cos(x) = \answer{-\sin(x)}$  
\item $\ddx \tan(x) = \sec^2(x)$  
\item $\ddx \sec(x) = \answer{\sec(x)\tan(x)}$ 
\item $\ddx \csc(x) = -\csc(x)\cot(x)$
\item $\ddx \cot(x) = \answer{-\csc^2(x)}$
\end{itemize}

\paragraph{Inverse Trigonometric Functions:}
\begin{itemize}
\item $\ddx \arcsin(x) = \frac{1}{\sqrt{1-x^2}}$
\item $\ddx \arccos(x) = \frac{-1}{\sqrt{1-x^2}}$
\item $\ddx \arctan(x) =\frac{1}{1+x^2}$
%\item $\ddx \arcsec(x) = \frac{1}{|x|\sqrt{x^2-1}}$ for $|x|>1$
%\item $\ddx \arccsc(x) = \frac{-1}{|x|\sqrt{x^2-1}}$ for $|x|>1$
%\item $\ddx \arccot(x) = \frac{-1}{1+x^2}$
\end{itemize}



\begin{question} 
  Using the above rules only, what is the derivative of $\frac{1}{\sqrt[4]{x^3}}$ with respect to $x$?
  \begin{prompt} 
    \[
    \ddx \left[\frac{1}{\sqrt[4]{x^3}}\right] = \answer[given]{- \frac{3}{4} x^{-7/4}}
    \]
  \end{prompt}
\end{question}


Once the derivatives of basic functions have been established, we can use them to build the derivatives of more complicated functions:


\begin{theorem}[Derivatives of combinations of basic functions]
  Let $f(x)$ and $g(x)$ be differentiable functions.
\begin{itemize}
\item Addition: $\ddx \left( f(x) + g(x) \right) = f'(x) + g'(x)$
\item Scalar Multiplication: $\ddx \left( af(x) \right) = a f'(x)$
\item Product Rule: $\ddx \left( f(x) \cdot g(x) \right) = f(x)g'(x) + f'(x)g(x)$
\item Quotient Rule: $\ddx \frac{f(x)}{g(x)} = \frac{f'(x)g(x) - f(x)g'(x)}{(g(x))^2}$
\item Chain Rule: $\ddx f(g(x)) = f'(g(x)) \cdot g'(x)$
\end{itemize}
%\end{multicols}
\end{theorem}

%%%%%%%%%%%
\begin{question} 
  What is the derivative of $4x^2 +\frac{2}{3\sqrt{x}} $ with respect to $x$?
  \begin{prompt} 
    \[
    \ddx \left(4x^2 +\frac{2}{3\sqrt{x}} \right) = \answer[given]{8x -\frac{1}{3} x^{-3/2}}
    \]
  \end{prompt}
\end{question}

 %%%%%%%%%%%
 
%%%%%%%%%%%
\begin{question} 
  What is the derivative of $x^2 \sin(x) $ with respect to $x$?
  \begin{prompt} 
    \[
    \ddx x^2 \sin(x) = \answer[given]{2x \sin(x) +x^2 \cos(x)}
    \]
  \end{prompt}
\end{question}

% %%%%%%%%%%%
%\begin{question} 
%  What is the derivative of $\frac{e^x}{x^2+1}$ with respect to $x$?
%  \begin{prompt} 
%    \[
%    \ddx \frac{e^x}{x^2+1} = \answer[given]{\frac{( x^2 - 2x +1)e^x}{(x^2+1)^2}}
%    \]
%  \end{prompt}
% %%%%%%%%%%% 
%\end{question}\begin{question} 
%  What is the derivative of $\arcsin\left(\frac{x}{a}\right)$ with respect to $x$?
%  \begin{prompt} 
%    \[
%    \ddx \arcsin\left(\frac{x}{a}\right) = \answer[given]{\frac{1}{a \sqrt{1-(x/a)^2}}}
%    \]
%  \end{prompt}
%\end{question}

\section{Derivatives and Tangent Lines}

While there are many useful applications of derivatives, on of the most important ones is the relationship between derivatives and tangent lines.

Suppose that $f(x)$ is a differentiable function on an interval $I$.  If we pick a value $a$ in the interval, and ``zoom in" on the graph around $x=a$, notice what happens:

\begin{image}
\begin{tikzpicture}
  \begin{axis}[
            domain=0:6, range=0:7,
            ymin=-.2,ymax=7,
            width=6in,
            height=2.5in, %% Hard coded height! Moreover this effects the aspect ratio of the zoom--sort of BAD
            axis lines=none,
          ]   
          \addplot [draw=none, fill=textColor!10!background] plot coordinates {(.8,1.6) (2.834,5)} \closedcycle; %% zoom fill
          \addplot [draw=none, fill=textColor!10!background] plot coordinates {(2.834,5) (4.166,5)} \closedcycle; %% zoom fill
          \addplot [draw=none, fill=background] plot coordinates {(1.2,1.6) (4.166,5)} \closedcycle; %% zoom fill
          \addplot [draw=none, fill=background] plot coordinates {(.8,1.6) (1.2,1.6)} \closedcycle; %% zoom fill

          \addplot [draw=none, fill=textColor!10!background] plot coordinates {(3.3,3.6) (5.334,5)} \closedcycle; %% zoom fill
          \addplot [draw=none, fill=textColor!10!background] plot coordinates {(5.334,5) (6.666,5)} \closedcycle; %% zoom fill
          \addplot [draw=none, fill=background] plot coordinates {(3.7,3.6) (6.666,5)} \closedcycle; %% zoom fill
          \addplot [draw=none, fill=background] plot coordinates {(3.3,3.6) (3.7,3.6)} \closedcycle; %% zoom fill
          
          \addplot [draw=none, fill=textColor!10!background] plot coordinates {(3.7,2.4) (6.666,1)} \closedcycle; %% zoom fill
          \addplot [draw=none, fill=textColor!10!background] plot coordinates {(3.3,2.4) (3.7,2.4)} \closedcycle; %% zoom fill
          \addplot [draw=none, fill=background] plot coordinates {(3.3,2.4) (5.334,1)} \closedcycle; %% zoom fill          
          \addplot [draw=none, fill=background] plot coordinates {(5.334,1) (6.666,1)} \closedcycle; %% zoom fill
          

          \addplot [draw=none, fill=textColor!10!background] plot coordinates {(.8,.4) (2.834,1)} \closedcycle; %% zoom fill
          \addplot [draw=none, fill=textColor!10!background] plot coordinates {(2.834,1) (4.166,1)} \closedcycle; %% zoom fill
          \addplot [draw=none, fill=background] plot coordinates {(1.2,.4) (4.166,1)} \closedcycle; %% zoom fill
          \addplot [draw=none, fill=background] plot coordinates {(.8,.4) (1.2,.4)} \closedcycle; %% zoom fill

          \addplot[very thick,penColor, smooth,domain=(0:1.833)] {-1/(x-2)};
          \addplot[very thick,penColor, smooth,domain=(2.834:4.166)] {3.333/(2.050-.3*x)-0.333}; %% 2.5 to 4.333
          %\addplot[very thick,penColor, smooth,domain=(5.334:6.666)] {11.11/(1.540-.09*x)-8.109}; %% 5 to 6.833
          \addplot[very thick,penColor, smooth,domain=(5.334:6.666)] {x-3}; %% 5 to 6.833
          
          \addplot[color=penColor,fill=penColor,only marks,mark=*] coordinates{(1,1)};  %% point to be zoomed
          \addplot[color=penColor,fill=penColor,only marks,mark=*] coordinates{(3.5,3)};  %% zoomed pt 1
          \addplot[color=penColor,fill=penColor,only marks,mark=*] coordinates{(6,3)};  %% zoomed pt 2

          \addplot [->,textColor] plot coordinates {(0,0) (0,6)}; %% axis
          \addplot [->,textColor] plot coordinates {(0,0) (2,0)}; %% axis
          
          \addplot [textColor!50!background] plot coordinates {(.8,.4) (.8,1.6)}; %% box around pt
          \addplot [textColor!50!background] plot coordinates {(1.2,.4) (1.2,1.6)}; %% box around pt
          \addplot [textColor!50!background] plot coordinates {(.8,1.6) (1.2,1.6)}; %% box around pt
          \addplot [textColor!50!background] plot coordinates {(.8,.4) (1.2,.4)}; %% box around pt
          
          \addplot [textColor!50!background] plot coordinates {(2.834,1) (2.834,5)}; %% zoomed box 1
          \addplot [textColor!50!background] plot coordinates {(4.166,1) (4.166,5)}; %% zoomed box 1
          \addplot [textColor!50!background] plot coordinates {(2.834,1) (4.166,1)}; %% zoomed box 1
          \addplot [textColor!50!background] plot coordinates {(2.834,5) (4.166,5)}; %% zoomed box 1

          \addplot [textColor] plot coordinates {(3.3,2.4) (3.3,3.6)}; %% box around zoomed pt
          \addplot [textColor] plot coordinates {(3.7,2.4) (3.7,3.6)}; %% box around zoomed pt
          \addplot [textColor] plot coordinates {(3.3,3.6) (3.7,3.6)}; %% box around zoomed pt
          \addplot [textColor] plot coordinates {(3.3,2.4) (3.7,2.4)}; %% box around zoomed pt

          \addplot [textColor] plot coordinates {(5.334,1) (5.334,5)}; %% zoomed box 2
          \addplot [textColor] plot coordinates {(6.666,1) (6.666,5)}; %% zoomed box 2
          \addplot [textColor] plot coordinates {(5.334,1) (6.666,1)}; %% zoomed box 2
          \addplot [textColor] plot coordinates {(5.334,5) (6.666,5)}; %% zoomed box 2

          \node at (axis cs:2.2,0) [anchor=east] {$x$};
          \node at (axis cs:0,6.6) [anchor=north] {$y$};
        \end{axis}
\end{tikzpicture}

\end{image}

As evidenced by the image, when the function $f(x)$ is differentiable at a given $x$-value, the graph of $f(x)$ becomes closer to a line as we ``zoom in", and we call this line the \emph{tangent line} at $x=a$.

To find the equation of this line, we need a point of the line and the slope of the line.  The slope of the line is $m_{tan} =$ \wordChoice{\choice[correct]{$f'(a)$}\choice{$f(a)$}} and the point on the line is \wordChoice{\choice{$a$}\choice{$f(a)$}\choice[correct]{$(a,f(a))$}\choice{$(a,f'(a))$}}.


This result is also a direct consequence of the limit definition of the derivative and some algebra.  If $f'(a)$ exists, then:

\[
f'(a) = \lim_{x \to a} \frac{f(x)-f(a)}{x-a}
\]
In order to write everything as a single limit, note we can write $f'(a)$ as:

\[
f'(a) = \lim_{x \to a} f'(a) =  \lim_{x \to a} \left(f'(a) \cdot \frac{x-a}{x-a}\right) = \lim_{x \to a} \frac{f'(a)(x-a)}{x-a}
\]

We now return to the limit definition of $f'(a)$ and write: 
\begin{align*}
f'(a) &= \lim_{x \to a} \frac{f(x)-f(a)}{x-a}\\
\lim_{x \to a} \frac{f'(a)(x-a)}{x-a} &= \lim_{x \to a} \frac{f(x)-f(a)}{x-a}\\
0 &= \lim_{x \to a} \frac{f(x)-f(a)}{x-a} - \lim_{x \to a} \frac{f'(a)(x-a)}{x-a}\\
0 &= \lim_{x \to a} \left(\frac{f(x)-f(a)}{x-a} - \frac{f'(a)(x-a)}{x-a} \right)\\
0 &= \lim_{x \to a} \frac{f(x)-f'(a)(x-a) - f(a)}{x-a}\\
\end{align*} 
Denoting $ f'(a)(x-a) - f(a)$ by $l_a(x)$, we note that $l_a(x)= f'(a)(x-a) - f(a)$ is the equation of a line, and we can write the last equation above as:

\[
\lim_{x \to a} \frac{f(x)-l_a(x)}{x-a} = 0
\]
Since this limit is $0$, we can interpret it geometrically by saying that as the distance between $x$ and $a$ becomes smaller, the distance between the function $f(x)$ and the line $l_a(x)$ becomes even smaller.  That is, as we ``zoom in'' on the graph of a function at a point where it is differentiable, the graph becomes closer and closer to a line and we call this line the \emph{tangent line}.

\begin{warning}
Many students mistakenly think that the tangent line can only touch the graph of the function at the point of tangency, but this is not true.  For instance, if $f(x) = 2x+1$, then $f'(x) = \answer{2}$ and $f(1) = \answer[given]{3}$, so the equation of the tangent line at $x=1$ is:

\begin{align*}
y-y(1) &= m_{tan}(x-x(1)) \\
y - \answer[given]{3} &= \answer[given]{2}\left(x-\answer{1}\right) \\
y &= \answer{2x+1}
\end{align*}

This is exactly the original function.  Thus, in any interval around $x=1$, the graph of the tangent line and the function intersect \wordChoice{\choice{once}\choice{only finitely many times}\choice[correct]{at every point}}.

As such, it is important to realize that a good intuitive way to think about tangent lines is that a function has a tangent line at a point $x=a$ if the graph of the function becomes less distinguishable from its tangent line as we continue to ``zoom in" on the graph of the function at $x=a$.
\end{warning}

\end{document}
