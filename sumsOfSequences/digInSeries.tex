\documentclass{ximera}

%\usepackage{todonotes}
%\usepackage{mathtools} %% Required for wide table Curl and Greens
%\usepackage{cuted} %% Required for wide table Curl and Greens
\newcommand{\todo}{}

\usepackage{esint} % for \oiint
\ifxake%%https://math.meta.stackexchange.com/questions/9973/how-do-you-render-a-closed-surface-double-integral
\renewcommand{\oiint}{{\large\bigcirc}\kern-1.56em\iint}
\fi


\graphicspath{
  {./}
  {ximeraTutorial/}
  {basicPhilosophy/}
  {functionsOfSeveralVariables/}
  {normalVectors/}
  {lagrangeMultipliers/}
  {vectorFields/}
  {greensTheorem/}
  {shapeOfThingsToCome/}
  {dotProducts/}
  {partialDerivativesAndTheGradientVector/}
  {../productAndQuotientRules/exercises/}
  {../normalVectors/exercisesParametricPlots/}
  {../continuityOfFunctionsOfSeveralVariables/exercises/}
  {../partialDerivativesAndTheGradientVector/exercises/}
  {../directionalDerivativeAndChainRule/exercises/}
  {../commonCoordinates/exercisesCylindricalCoordinates/}
  {../commonCoordinates/exercisesSphericalCoordinates/}
  {../greensTheorem/exercisesCurlAndLineIntegrals/}
  {../greensTheorem/exercisesDivergenceAndLineIntegrals/}
  {../shapeOfThingsToCome/exercisesDivergenceTheorem/}
  {../greensTheorem/}
  {../shapeOfThingsToCome/}
  {../separableDifferentialEquations/exercises/}
  {vectorFields/}
}

\newcommand{\mooculus}{\textsf{\textbf{MOOC}\textnormal{\textsf{ULUS}}}}

\usepackage{tkz-euclide}\usepackage{tikz}
\usepackage{tikz-cd}
\usetikzlibrary{arrows}
\tikzset{>=stealth,commutative diagrams/.cd,
  arrow style=tikz,diagrams={>=stealth}} %% cool arrow head
\tikzset{shorten <>/.style={ shorten >=#1, shorten <=#1 } } %% allows shorter vectors

\usetikzlibrary{backgrounds} %% for boxes around graphs
\usetikzlibrary{shapes,positioning}  %% Clouds and stars
\usetikzlibrary{matrix} %% for matrix
\usepgfplotslibrary{polar} %% for polar plots
\usepgfplotslibrary{fillbetween} %% to shade area between curves in TikZ
\usetkzobj{all}
\usepackage[makeroom]{cancel} %% for strike outs
%\usepackage{mathtools} %% for pretty underbrace % Breaks Ximera
%\usepackage{multicol}
\usepackage{pgffor} %% required for integral for loops



%% http://tex.stackexchange.com/questions/66490/drawing-a-tikz-arc-specifying-the-center
%% Draws beach ball
\tikzset{pics/carc/.style args={#1:#2:#3}{code={\draw[pic actions] (#1:#3) arc(#1:#2:#3);}}}



\usepackage{array}
\setlength{\extrarowheight}{+.1cm}
\newdimen\digitwidth
\settowidth\digitwidth{9}
\def\divrule#1#2{
\noalign{\moveright#1\digitwidth
\vbox{\hrule width#2\digitwidth}}}





\newcommand{\RR}{\mathbb R}
\newcommand{\R}{\mathbb R}
\newcommand{\N}{\mathbb N}
\newcommand{\Z}{\mathbb Z}

\newcommand{\sagemath}{\textsf{SageMath}}


%\renewcommand{\d}{\,d\!}
\renewcommand{\d}{\mathop{}\!d}
\newcommand{\dd}[2][]{\frac{\d #1}{\d #2}}
\newcommand{\pp}[2][]{\frac{\partial #1}{\partial #2}}
\renewcommand{\l}{\ell}
\newcommand{\ddx}{\frac{d}{\d x}}

\newcommand{\zeroOverZero}{\ensuremath{\boldsymbol{\tfrac{0}{0}}}}
\newcommand{\inftyOverInfty}{\ensuremath{\boldsymbol{\tfrac{\infty}{\infty}}}}
\newcommand{\zeroOverInfty}{\ensuremath{\boldsymbol{\tfrac{0}{\infty}}}}
\newcommand{\zeroTimesInfty}{\ensuremath{\small\boldsymbol{0\cdot \infty}}}
\newcommand{\inftyMinusInfty}{\ensuremath{\small\boldsymbol{\infty - \infty}}}
\newcommand{\oneToInfty}{\ensuremath{\boldsymbol{1^\infty}}}
\newcommand{\zeroToZero}{\ensuremath{\boldsymbol{0^0}}}
\newcommand{\inftyToZero}{\ensuremath{\boldsymbol{\infty^0}}}



\newcommand{\numOverZero}{\ensuremath{\boldsymbol{\tfrac{\#}{0}}}}
\newcommand{\dfn}{\textbf}
%\newcommand{\unit}{\,\mathrm}
\newcommand{\unit}{\mathop{}\!\mathrm}
\newcommand{\eval}[1]{\bigg[ #1 \bigg]}
\newcommand{\seq}[1]{\left( #1 \right)}
\renewcommand{\epsilon}{\varepsilon}
\renewcommand{\phi}{\varphi}


\renewcommand{\iff}{\Leftrightarrow}

\DeclareMathOperator{\arccot}{arccot}
\DeclareMathOperator{\arcsec}{arcsec}
\DeclareMathOperator{\arccsc}{arccsc}
\DeclareMathOperator{\si}{Si}
\DeclareMathOperator{\scal}{scal}
\DeclareMathOperator{\sign}{sign}


%% \newcommand{\tightoverset}[2]{% for arrow vec
%%   \mathop{#2}\limits^{\vbox to -.5ex{\kern-0.75ex\hbox{$#1$}\vss}}}
\newcommand{\arrowvec}[1]{{\overset{\rightharpoonup}{#1}}}
%\renewcommand{\vec}[1]{\arrowvec{\mathbf{#1}}}
\renewcommand{\vec}[1]{{\overset{\boldsymbol{\rightharpoonup}}{\mathbf{#1}}}\hspace{0in}}

\newcommand{\point}[1]{\left(#1\right)} %this allows \vector{ to be changed to \vector{ with a quick find and replace
\newcommand{\pt}[1]{\mathbf{#1}} %this allows \vec{ to be changed to \vec{ with a quick find and replace
\newcommand{\Lim}[2]{\lim_{\point{#1} \to \point{#2}}} %Bart, I changed this to point since I want to use it.  It runs through both of the exercise and exerciseE files in limits section, which is why it was in each document to start with.

\DeclareMathOperator{\proj}{\mathbf{proj}}
\newcommand{\veci}{{\boldsymbol{\hat{\imath}}}}
\newcommand{\vecj}{{\boldsymbol{\hat{\jmath}}}}
\newcommand{\veck}{{\boldsymbol{\hat{k}}}}
\newcommand{\vecl}{\vec{\boldsymbol{\l}}}
\newcommand{\uvec}[1]{\mathbf{\hat{#1}}}
\newcommand{\utan}{\mathbf{\hat{t}}}
\newcommand{\unormal}{\mathbf{\hat{n}}}
\newcommand{\ubinormal}{\mathbf{\hat{b}}}

\newcommand{\dotp}{\bullet}
\newcommand{\cross}{\boldsymbol\times}
\newcommand{\grad}{\boldsymbol\nabla}
\newcommand{\divergence}{\grad\dotp}
\newcommand{\curl}{\grad\cross}
%\DeclareMathOperator{\divergence}{divergence}
%\DeclareMathOperator{\curl}[1]{\grad\cross #1}
\newcommand{\lto}{\mathop{\longrightarrow\,}\limits}

\renewcommand{\bar}{\overline}

\colorlet{textColor}{black}
\colorlet{background}{white}
\colorlet{penColor}{blue!50!black} % Color of a curve in a plot
\colorlet{penColor2}{red!50!black}% Color of a curve in a plot
\colorlet{penColor3}{red!50!blue} % Color of a curve in a plot
\colorlet{penColor4}{green!50!black} % Color of a curve in a plot
\colorlet{penColor5}{orange!80!black} % Color of a curve in a plot
\colorlet{penColor6}{yellow!70!black} % Color of a curve in a plot
\colorlet{fill1}{penColor!20} % Color of fill in a plot
\colorlet{fill2}{penColor2!20} % Color of fill in a plot
\colorlet{fillp}{fill1} % Color of positive area
\colorlet{filln}{penColor2!20} % Color of negative area
\colorlet{fill3}{penColor3!20} % Fill
\colorlet{fill4}{penColor4!20} % Fill
\colorlet{fill5}{penColor5!20} % Fill
\colorlet{gridColor}{gray!50} % Color of grid in a plot

\newcommand{\surfaceColor}{violet}
\newcommand{\surfaceColorTwo}{redyellow}
\newcommand{\sliceColor}{greenyellow}




\pgfmathdeclarefunction{gauss}{2}{% gives gaussian
  \pgfmathparse{1/(#2*sqrt(2*pi))*exp(-((x-#1)^2)/(2*#2^2))}%
}


%%%%%%%%%%%%%
%% Vectors
%%%%%%%%%%%%%

%% Simple horiz vectors
\renewcommand{\vector}[1]{\left\langle #1\right\rangle}


%% %% Complex Horiz Vectors with angle brackets
%% \makeatletter
%% \renewcommand{\vector}[2][ , ]{\left\langle%
%%   \def\nextitem{\def\nextitem{#1}}%
%%   \@for \el:=#2\do{\nextitem\el}\right\rangle%
%% }
%% \makeatother

%% %% Vertical Vectors
%% \def\vector#1{\begin{bmatrix}\vecListA#1,,\end{bmatrix}}
%% \def\vecListA#1,{\if,#1,\else #1\cr \expandafter \vecListA \fi}

%%%%%%%%%%%%%
%% End of vectors
%%%%%%%%%%%%%

%\newcommand{\fullwidth}{}
%\newcommand{\normalwidth}{}



%% makes a snazzy t-chart for evaluating functions
%\newenvironment{tchart}{\rowcolors{2}{}{background!90!textColor}\array}{\endarray}

%%This is to help with formatting on future title pages.
\newenvironment{sectionOutcomes}{}{}



%% Flowchart stuff
%\tikzstyle{startstop} = [rectangle, rounded corners, minimum width=3cm, minimum height=1cm,text centered, draw=black]
%\tikzstyle{question} = [rectangle, minimum width=3cm, minimum height=1cm, text centered, draw=black]
%\tikzstyle{decision} = [trapezium, trapezium left angle=70, trapezium right angle=110, minimum width=3cm, minimum height=1cm, text centered, draw=black]
%\tikzstyle{question} = [rectangle, rounded corners, minimum width=3cm, minimum height=1cm,text centered, draw=black]
%\tikzstyle{process} = [rectangle, minimum width=3cm, minimum height=1cm, text centered, draw=black]
%\tikzstyle{decision} = [trapezium, trapezium left angle=70, trapezium right angle=110, minimum width=3cm, minimum height=1cm, text centered, draw=black]

\author{Jim Talamo}

\outcome{Define a series.}
\outcome{Recognize a geometric series.}
\outcome{Recognize a telescoping series.}
\outcome{Compute the sum of a geometric series.}
\outcome{Compute the sum of a telescoping series.}

\title[Dig-In:]{Series}

\begin{document}
\begin{abstract}
A series is an infinite summation of the terms of sequence.
\end{abstract}
\maketitle

In the previous sections, we've seen several examples of sequence.  If we have a sequence $\{a_n\}_{n=1}$, and represent it as an ordered list below: 

\[
a_1, a_2, a_3 , \ldots
\]

we can ask two important questions about it.

\begin{itemize}
\item[1.] Do the numbers in the list approach a finite value?
\item[2.] Can I sum all of the numbers in the list and obtain a finite result?
\end{itemize}

The first question is really whether the \emph{limit} $\lim_{n \to \infty} a_n$ exists and we studied several ways to determine this previously.  Before diving into the second question, let's study an example.

\example{Shading A Square}
Suppose that we want to study the infinite sum below.

\[
\frac{1}{2} + \left(\frac{1}{2}\right)^2+ \left(\frac{1}{2}\right)^3+ \ldots
\]

A student, feeling quite clever, decides to illustrate the sum by drawing a square with side length one unit and shading it in a special way.  In order to understand the concepts here more fully, it is recommended that you draw and shade as you read.

\paragraph{Step 1:} Shade the left half of the square.  

\begin{image}[1in]
  \begin{tikzpicture}[scale=3,rounded corners=.5pt]      
    \tkzDefPoint(0,0){A1} 
    \tkzDefPoint(1,0){A2}
    \tkzDefPoint(1,1){A3}
    \tkzDefPoint(0,1){A4}
    \draw[penColor,very thick] (A1)--(A2)--(A3)--(A4)--cycle;

    \tkzDefPoint(0,0){B1} 
    \tkzDefPoint(.5,0){B2}
    \tkzDefPoint(.5,1){B3}
    \tkzDefPoint(0,1){B4}
    \draw[penColor,fill=fill2,very thick] (B1)--(B2)--(B3)--(B4)--cycle;
  \end{tikzpicture}

\end{image}
There are two quantities of which we can keep track now.
\begin{itemize}
\item Call the area shaded this step $A_1$.  Then, we have $A_1=\answer[given]{1/2}$.
\item Call and the total shaded area  of the square $S_1$.  Then, we have $S_1=\answer[given]{1/2}$.
\end{itemize}

\paragraph{Step 2:} Shade the bottom half of the unshaded region.  

\begin{image}[1in]
  \begin{tikzpicture}[scale=3,rounded corners=.5pt]      
    \tkzDefPoint(0,0){A1} 
    \tkzDefPoint(1,0){A2}
    \tkzDefPoint(1,1){A3}
    \tkzDefPoint(0,1){A4}
    \draw[penColor,very thick] (A1)--(A2)--(A3)--(A4)--cycle;

    \tkzDefPoint(0,0){B1} 
    \tkzDefPoint(.5,0){B2}
    \tkzDefPoint(.5,1){B3}
    \tkzDefPoint(0,1){B4}
    \draw[penColor,fill=fill1,very thick] (B1)--(B2)--(B3)--(B4)--cycle;
    
    \tkzDefPoint(.5,0){C1} 
    \tkzDefPoint(1,0){C2}
    \tkzDefPoint(1,.5){C3}
    \tkzDefPoint(.5,.5){C4}
    \draw[penColor,fill=fill2,very thick] (C1)--(C2)--(C3)--(C4)--cycle;
    
  \end{tikzpicture}
\end{image}

\begin{itemize}
\item Call the area shaded this step $A_2$.  Then, we have $A_2=\answer[given]{1/4}$.
\item Call and the total shaded area  of the square $S_2$.  Then, we have $S_2=\answer[given]{3/4}$.
\end{itemize}

Visually, notice that we can find $S_2$ by noting
$$(area ~ shaded ~ after ~ Step ~ 2) = (area ~ from ~ Step ~ 1) + (area ~ from ~ Step ~ 2)$$
and writing down the total shaded are of the square.

Analytically, we can write: $$S_2=A_1+A_2.$$

\paragraph{Step 3:} Shade the left half of the unshaded region.  \begin{image}[1in]
  \begin{tikzpicture}[scale=3,rounded corners=.5pt]      
    \tkzDefPoint(0,0){A1} 
    \tkzDefPoint(1,0){A2}
    \tkzDefPoint(1,1){A3}
    \tkzDefPoint(0,1){A4}
    \draw[penColor,very thick] (A1)--(A2)--(A3)--(A4)--cycle;

    \tkzDefPoint(0,0){B1} 
    \tkzDefPoint(.5,0){B2}
    \tkzDefPoint(.5,1){B3}
    \tkzDefPoint(0,1){B4}
    \draw[penColor,fill=fill1,very thick] (B1)--(B2)--(B3)--(B4)--cycle;
    
    \tkzDefPoint(.5,0){C1} 
    \tkzDefPoint(1,0){C2}
    \tkzDefPoint(1,.5){C3}
    \tkzDefPoint(.5,.5){C4}
    \draw[penColor,fill=fill1,very thick] (C1)--(C2)--(C3)--(C4)--cycle;
    
    \tkzDefPoint(.5,.5){D1} 
    \tkzDefPoint(.75,.5){D2}
    \tkzDefPoint(.75,1){D3}
    \tkzDefPoint(.5,1){D4}
    \draw[penColor,fill=fill2,very thick] (D1)--(D2)--(D3)--(D4)--cycle;
    
  \end{tikzpicture}
\end{image}

\begin{itemize}
\item Call the area shaded this step $A_3$.  Then, we have $A_3=[given]{1/8}$.
\item Call and the total shaded area  of the square $S_3$.  Then, we have $S_3=[given]{7/8}$.
\end{itemize}

We can think of $S_3$ visually or analytically. 

%\begin{image}[1in]
%  \begin{tikzpicture}[scale=3,rounded corners=.5pt]      
%    \tkzDefPoint(0,1){A1} 
%    \tkzDefPoint(-.58,0){A2}
%    \tkzDefPoint(.58,0){A3}
%    \draw[penColor,very thick] (A1)--(A2)--(A3)--cycle;
%  \end{tikzpicture}
%\end{image}


%\begin{figure}[!htb]
%\minipage{0.32\textwidth}
%  \includegraphics[width=\linewidth]{SquareA1}
% \phantom{n}  \hspace{17mm} Step 1 
% 
%  \hspace{19mm}  $S_1 = A_1$
%  \endminipage\hfill
%\minipage{0.32\textwidth}
%  \includegraphics[width=\linewidth]{SquareA2}
%   \phantom{n}  \hspace{17mm} Step 2
% 
% \hspace{15mm}  $S_2=A_1+A_2$
%\endminipage\hfill
%\minipage{0.32\textwidth}%
%  \includegraphics[width=\linewidth]{SquareA4}
%   \phantom{n}  \hspace{17mm} Step 4
%   
%  \hspace{5mm}    $S_4=A_1+A_2+A_3+A_4$
%\endminipage
%
%\caption{\small The area $A_n$ shaded in the $n$-th step is shown in blue.  The total shaded area $s_n$ is the sum of the all of the areas shaded in each previous step as well as the current one.}
%\end{figure}

\vspace{3mm}

Hopefully, the pattern used to shade the square is clear.  We can define the area  shaded during the $n$-th step to be $A_n$.  From the procedure, we can write $A_n(1/2)^n$.

We can also let $s_n$ denote the total shaded area after the $n$-th step.  Analytically, we have $s_n = A_1+A_2 + \ldots A_n$, or by using summation notation, we can write $\displaystyle s_n = \sum_{k=1}^n A_k$.

Looking at the pictures drawn so far, notice that the only unshaded area after the $n$-th step is a rectangle of area $A_n$, so we can write a formula for $s_n$.   

\[
s_n = \left<(\textrm{total area of the square})\right>-A_n = 1-\left(\frac{1}{2^n}\right)
\]

We can now evaluate the limit and find that $\lim_{n \to \infty} s_n =\answer[given]{1}$.

We also have another method of thinking about this limit; after we continue shading the square indefinitely, there will be no portion of it that has not been shaded.  Thus, the total shaded area should be $1$.

We would thus like to conclude:

\[
\frac{1}{2} + \left(\frac{1}{2}\right)^2+ \left(\frac{1}{2}\right)^3+ \ldots = \sum_{k=1}^{\infty} \left(\frac{1}{2}\right)^n =1.
\]

\section{Infinite series}
The above question can be thought of in the context of sequences.  Note that we can write down an ordered list where each successive term denotes the area shaded in the subsequent step.

\[
A_1, A_2,A_3, \ldots = \frac{1}{2},\left(\frac{1}{2}\right)^2,\left(\frac{1}{2}\right)^3,\ldots
\]
and we can interpret the series $$\frac{1}{2} + \left(\frac{1}{2}\right)^2+ \left(\frac{1}{2}\right)^3+ \ldots$$ as the symbolic attempt to add up all of the terms in the above sequence.  In fact, during the last example, we found a new sequence that was created from the original sequence sequence that track of a finite portion of the addition.

\[
S_1, S_2,S_3, \ldots = \frac{1}{2},\frac{3}{4},\frac{7}{8},\ldots
\]
We saw that since we could find the limit $\lim_{n \to \infty} s_n$, we could \emph{add} up all of the terms in the sequence $\{A_n\}$.  

While it's not always easy (or possible!) to generalize the procedure from the last example, we can generalize the ideas that led us to study whether it was possible to find the desired sum in the first place.

\begin{definition}
Let $\{a_n\}_{n=n_0}$ be a sequence.  Let $s_n = \sum_{k=k_0}^n a_k$; the sequence $\{s_n\}_{n=n_0}$ is the called the
  \dfn{sequence of partial sums} of $\{a_n\}$.  
\end{definition}

Note that both the original sequence and the sequence of partial sums have the same lower index!

Just as in our example, we can use this \emph{new} sequence $\{s_n\}$ to study whether we can sum all of the terms in the sequence $\{a_n\}$.



\begin{definition}
Let $\{a_n\}_{n=n_0}$ be a sequence and $\{s_n\}_{n=n_0}$ be its sequence of partial sums.

\begin{enumerate}
\item If $\lim_{n\to\infty} s_n$ exists, we say the series
  $\sum_{k=k_0}^\infty a_k$ \dfn{converges}.  Furthermore, if $\lim_{n\to\infty} s_n =L$, we say the series
  $\sum_{k=k_0}^\infty a_k$ converges to $L$. 
\item If $\lim_{n\to\infty} s_n = \infty, \lim_{n\to\infty} s_n = -\infty$ or $\lim_{n\to\infty} s_n $ otherwise does not exist, we say the series
  $\sum_{k=k_0}^\infty a_k$ \dfn{diverges}.  
\end{enumerate}
\end{definition}

In most cases of interest, we will usually have $k_0=0$ or $1$.  A few remarks are in order before we continue.

\begin{quote}
The symbols ``$\lim_{n \to \infty} s_n$'' and ``$\sum_{k=k_0}^{\infty} a_k$'' should be thought of as analogous; if the first is defined, so is the second and vice-versa.  If the first is found to be a constant $L$, then the second is $L$ and vice-versa.
\end{quote}

\begin{remark}
This definition makes the content of the previous example more precise.  The major idea here is that we have techniques that we can use to determine whether limits exist and can even find what those limits are sometimes.  Since we are now able to recast the new question ``Can I sum all of the terms in a sequence?'' into the old question ``Does a sequence have a limit?", we can now utilize all of our previous techniques.  
\end{remark}

\begin{warning}
Note that the sequences $\{a_n\}_{n=1}$ and $\{s_n\}_{n=1}$ are very different sequences.  Each term in the sequence $\{s_n\}$ is the result of adding terms in the sequence $\{a_n\}$.  The limit $\lim_{n \to \infty} s_n$ represents the attempt to perform the addition of terms in the original sequence.
\end{warning}

%Look back at the definition again. Previously, we've been using 
%notation like $(a_n)$ to define our sequences.  Suddenly, we've switched 
%the index to a $k$ instead of an $n$ in some places. This notation is to 
%help you keep in mind the differences between the two at work, here: $(a_k)$ and $\{s_n\}$.  The first 
%sequence gives us the terms of our series; we are not usually concerned 
%about its limit.  We do want to consider the limit of the sequence of 
%partial sums $\{s_n\}$, since this will give us the sum of the series.

\begin{question}
  Using our new terminology, what is the behavior of the series we 
  considered above?
  \begin{prompt}
    The series $\sum_{n=1}^\infty \left(\frac{1}{2}\right)^n$
    \wordChoice{\choice[correct]{converges}\choice{diverges}}, and
      $\sum_{n=1}^\infty \left(\frac{1}{2}\right)^n = \answer[given]{1}$.
  \end{prompt}
\end{question}
%\begin{definition}
%  A \dfn{series} is a sum of an infinite sequence.
%\end{definition}
%
%
%Let's start our investigation on this topic with a question (a little unfair, I know!).
%
%\begin{question}
%  Can the sum of an infinite number of terms be a finite value?
%  \begin{prompt}
%    \begin{multipleChoice}
%      \choice{no}
%      \choice[correct]{sometimes}
%    \end{multipleChoice}
%  \end{prompt}
%\end{question}
%As we will see, the answer is ``sometimes.''  Believe it or not, you
%have been working with infinite sums of numbers (also called
%\textit{series}) for a long time. Consider the number
%\[
%\frac{1}{3} = 0.3333333333\dots .
%\]
%This is the infinite sum of the geometric sequence
%$(a_n)_{n=1}^\infty$ where $a_n = \frac{3}{10^{n}}$, as
%\begin{align*}
%  \sum_{n=1}^\infty 3\cdot \frac{1}{10^{n}} &= 0.3 + 0.03+0.003+ 0.0003+ 0.00003+ \cdots\\
%  &= \frac{3}{10} + \frac{3}{10^2} + \frac{3}{10^3} + \frac{3}{10^4} + \frac{3}{10^5} + \cdots\\
%  &=\frac{1}{3}.
%\end{align*}

%%%%%%%%%%%%%%%%%%%%%%

%MOVE TO EXERCISES

%%%%%%%%%%%%%%%%%%%%%%

%We can sum other geometric series to finite values as well. Consider
%\[
%\sum_{n=1}^\infty \left(\frac{1}{4}\right)^n =
%\frac{1}{4} + \left(\frac{1}{4}\right)^2 + \left(\frac{1}{4}\right)^3 + \left(\frac{1}{4}\right)^4 + \cdots 
%\]
%A very clever method of summing this sequence is as follows. Consider
%an equilateral triangle with area $1$:
%\begin{image}[1in]
%  \begin{tikzpicture}[scale=3,rounded corners=.5pt]      
%    \tkzDefPoint(0,1){A1} 
%    \tkzDefPoint(-.58,0){A2}
%    \tkzDefPoint(.58,0){A3}
%    \draw[penColor,very thick] (A1)--(A2)--(A3)--cycle;
%  \end{tikzpicture}
%\end{image}
%We can break this triangle into $4$ congruent triangles, each of area
%$1/4$:
%\begin{image}[1in]
%  \begin{tikzpicture}[scale=3,rounded corners=.5pt]      
%    \tkzDefPoint(0,1){A1} 
%    \tkzDefPoint(-.58,0){A2}
%    \tkzDefPoint(.58,0){A3}
%    \draw[penColor,very thick] (A1)--(A2)--(A3)--cycle;
%
%    \tkzDefPoint(0,0){B1} 
%    \tkzDefPoint(-.29,.5){B2}
%    \tkzDefPoint(.29,.5){B3}
%    \draw[penColor,fill=fill1,very thick] (B1)--(B2)--(B3)--cycle;
%  \end{tikzpicture}
%\end{image}
%We can break the upper triangle into $4$ more congruent triangles, each
%with area $(1/4)^2$:
%\begin{image}[1in]
%  \begin{tikzpicture}[scale=3,rounded corners=.5pt]      
%    \tkzDefPoint(0,1){A1} 
%    \tkzDefPoint(-.58,0){A2}
%    \tkzDefPoint(.58,0){A3}
%    \draw[penColor,very thick] (A1)--(A2)--(A3)--cycle;
%
%    \tkzDefPoint(0,0){B1} 
%    \tkzDefPoint(-.29,.5){B2}
%    \tkzDefPoint(.29,.5){B3}
%    \draw[penColor,fill=fill1,very thick] (B1)--(B2)--(B3)--cycle;
%
%    \tkzDefPoint(0,.5){C1} 
%    \tkzDefPoint(-.14,.75){C2}
%    \tkzDefPoint(.14,.75){C3}
%    \draw[penColor,fill=fill1,very thick] (C1)--(C2)--(C3)--cycle;
%  \end{tikzpicture}
%\end{image}
%Repeating this process, we find:
%\begin{image}[1in]
%  \begin{tikzpicture}[scale=3,rounded corners=.5pt]      
%    \tkzDefPoint(0,1){A1} 
%    \tkzDefPoint(-.58,0){A2}
%    \tkzDefPoint(.58,0){A3}
%    \draw[penColor,very thick] (A1)--(A2)--(A3)--cycle;
%
%    \tkzDefPoint(0,0){B1} 
%    \tkzDefPoint(-.29,.5){B2}
%    \tkzDefPoint(.29,.5){B3}
%    \draw[penColor,fill=fill1,very thick] (B1)--(B2)--(B3)--cycle;
%
%    \tkzDefPoint(0,.5){C1} 
%    \tkzDefPoint(-.14,.75){C2}
%    \tkzDefPoint(.14,.75){C3}
%    \draw[penColor,fill=fill1,very thick] (C1)--(C2)--(C3)--cycle;
%
%    \tkzDefPoint(0,.75){D1} 
%    \tkzDefPoint(-.07,.875){D2}
%    \tkzDefPoint(.07,.875){D3}
%    \draw[penColor,fill=fill1,very thick] (D1)--(D2)--(D3)--cycle;
%
%    \tkzDefPoint(0,.875){E1} 
%    \tkzDefPoint(-.04,.94){E2}
%    \tkzDefPoint(.04,.94){E3}
%    \draw[penColor,fill=fill1,very thick] (E1)--(E2)--(E3)--cycle;
%
%    \tkzDefPoint(0,.94 ){F1} 
%    \tkzDefPoint(-.02,.97){F2}
%    \tkzDefPoint(.02,.97){F3}
%    \draw[penColor,fill=fill1,very thick] (F1)--(F2)--(F3)--cycle;
%  \end{tikzpicture}
%\end{image}
%where the area of the shaded triangles is our geometric series:
%\[
%\frac{1}{4} + \left(\frac{1}{4}\right)^2 + \left(\frac{1}{4}\right)^3 + \left(\frac{1}{4}\right)^4 + \cdots 
%\]
%The area is clearly finite (it is between $0$ and $1$!). What is the
%shaded area? Well, if you look at any ``row'' of the triangle, we've
%shaded in exactly one third of the row. Hence we've shaded in one third of
%the entire area, so we see
%\[
%\frac{1}{3}=\frac{1}{4} + \left(\frac{1}{4}\right)^2 + \left(\frac{1}{4}\right)^3 + \left(\frac{1}{4}\right)^4 + \cdots 
%\]
%While this is a very cool argument, it doesn't generalize well. Let's consider
% an argument that will apply to more settings.
%
%\begin{example}
%  Explain why
%  \[
%  \frac{1}{3}=\frac{1}{4} + \left(\frac{1}{4}\right)^2 + \left(\frac{1}{4}\right)^3 + \left(\frac{1}{4}\right)^4 + \cdots 
%  \]
%  \begin{explanation}
%    Here is the idea: first, ``name'' your sum $S$.
%    \[
%    S = \frac{1}{4} + \left(\frac{1}{4}\right)^2 + \left(\frac{1}{4}\right)^3 + \left(\frac{1}{4}\right)^4 + \cdots 
%    \]
%    Now, multiply $S$ by $\frac{1}{4}$ and write this suggestively under $S$.
%    \begin{align*}
%      S &= \frac{1}{4} + \left(\frac{1}{4}\right)^2 + \left(\frac{1}{4}\right)^3 + \left(\frac{1}{4}\right)^4 + \cdots\\
%     \left(\frac{1}{4}\right)S &=   \left(\frac{1}{4}\right)^2 + \left(\frac{1}{4}\right)^3 + \left(\frac{1}{4}\right)^4 + \left(\frac{1}{4}\right)^5+ \cdots
%    \end{align*}
%    subtracting the lower line from the upper line we find
%    \begin{align*}
%      S - \left(\frac{1}{4}\right)S &=  \frac{1}{4}\\
%      S(1-\frac{1}{4}) &= \frac{1}{4}\\
%      S &= \frac{\frac{1}{4}}{1-\frac{1}{4}}\\
%      S &= \frac{1}{3}.
%    \end{align*}
%  \end{explanation}
%\end{example}
%
%This is a good method for understanding infinite sums of geometric
%sequences (assuming you know the sequence sums to a finite value).
%
%To make this precise, we need some definitions. 

\section{Two special types of series}
The definitions above give us a way to determine whether a given series converges.  In fact, to determine whether $\sum_{k=k_0} a_k$ converges, we can do the following.

\begin{itemize}
\item[1.] Consider the associated sequence $\{s_n\}$ of partial sums.
\item[2.] Try to find an explicit formula for the term $s_n$.  If you can find such a formula, analyze $\lim_{n \to \infty s_n}$.  
\begin{itemize}
\item If the limit exists, $\sum_{k=k_0}^{\infty} a_k$ converges, and if we can determine that $\lim_{n \to \infty} s_n =L$, then $\sum_{k=k_0} a_k=L$.  \item If  $\lim_{n \to \infty} s_n$ does not exist, then $\sum_{k=k_0} a_k$ diverges.
\end{itemize}
\item[3.] If an explicit formula for $s_n$ cannot be found, further analysis is needed.  We'll expound on this in later sections.
\end{itemize}

For now, we will consider two special types of series for which an explicit formula for $s_n$ can be found.

\section{Geometric series}
Recall that a \emph{geometric sequence} is a sequence for which the ratio of successive terms is constant.  If $\{a_n\}$ is such a sequence, then there are constants $a \ne 0$ and $r$ for which $a_n = a\cdot r^n$.  We can now try to determine when adding together the terms in such a series is possible.

\begin{definition}
  A \dfn{geometric series} is a series of the form $\sum_{k=0}^\infty ar^k$
  for some real numbers $a \ne 0$ and $r$.
\end{definition}

Thus, we can think of a geometric series as the infinite sum of the terms of a geometric sequence.  

\begin{example}
The series $\sum_{k=4}^\infty \frac{2^{2k+1}}{3^k}$ \wordChoice{\choice[correct]{is}\choice{is not}} geometric since $a_k =\frac{2^{2k+1}}{3^k}$ \wordChoice{\choice[correct]{can}\choice{cannot}} be brought into the form $a \cdot r^k$.  

Using the laws of exponents shows us:

\[
\frac{2^{2k+1}}{3^k} = \frac{2^{2k} \cdot 2^1}{3^k}= 2 \cdot \frac{\left(2^{k}\right)^2}{3^k} = 2 \cdot \frac{4^k}{3^k} = 2 \cdot \left(\frac{4}{3}\right)^k.
\]
Indeed, $a= \answer[given]{2}$ and $r = \answer[given]{\frac{4}{3}}$.
\end{example}

\begin{example}
The series $\sum_{k=0}^\infty k^2 \left(\frac{1}{2}\right)^k$ \wordChoice{\choice{is}\choice[correct]{is not}} geometric since $a_k = \answer[given]{k^2 \cdot \frac{1}{2}^k}$ \wordChoice{\choice{can}\choice[correct]{cannot}} be brought into the form $a \cdot r^k$.  Indeed, the coefficient, $k^2$, is not the same for each term in the series.
\end{example}

As it turns out, we can find an explicit formula for the $n$-th term in the sequence of partial sums.

%We started this section with two different geometric series that sum
%to the same value. One reason geometric series are important is that
%they have nice convergence properties.

\begin{example}
 Let $r \neq 1$ and consider the geometric series $\sum_{k=0}^\infty a r^k$, and let $s_n = \sum_{k=0}^{\infty} a r^k $.  We find an explicit formula for the term $s_n$.
  
  \begin{explanation}
First, note that the sum above represents the attempt to add all of the terms in the sequence $\{a_n\}_{n=0}$, where $a_n =  r^n$.  Let's start by writing out the first several terms in the sequence $\{a_n\}$.  

\[
a_0 = 1 , \qquad a_1 = r, \qquad a_2 = r^2 , \qquad \ldots ,
\]
    \begin{align*}
      s_0 &= a_0 = 1 \\
      s_1 &= 1 + 1r\\
      s_2 &= 1 + r + r^2\\
      &\vdots\\
      s_n &= 1 + r + r^2 + \dots + r^n
    \end{align*}
The general difficulty for finding a closed formula for $s_n$ arises because, without specifying what $n$ is, we cannot actually perform the indicated addition.  However, there's a nice trick we can exploit here.

We start by multiplying $s_n$ by the common ratio $r$.
    \begin{align*}
      s_n   &= 1 + r + r^2 + \dots + r^n\\
      r s_n &= ~ \phantom{ 1 + } r + r^2 + \dots + r^n + r^{n+1}
    \end{align*}
Now, we can subtract away the middle terms.

 \[     \begin{array}{rl}
      s_n   &= 1 + \cancel{ r + r^2 + \dots + r^n}\\
 -\left(  \phantom{ r^{n+1}} r s_n \right.&=~ \left. \phantom{  1 +  } \cancel{r + r^2 + \dots + r^n} + r^{n+1}\right) \\
 \hline 
     s_n - r s_n &= 1 \phantom{  +  r + r^2 + \dots + r^n } ~ - r^{n+1}\\
    \end{array}
 \]   
 
 We can now solve for $s_n$.
 
     \begin{align*}
      s_n - r s_n &= 1 - r^{n+1}\\
      s_n(1-r)    &= 1 - r^{n+1}\\
      s_n &= \frac{1 - r^{n+1}}{1-r}.
    \end{align*}
    Since $s_n$ is \textbf{always} a finite sum, and $r \ne 1$, there is no issue
    with manipulating it the way we did.
  \end{explanation}
\end{example}

From our work above, we see that the $n$-th partial sum of the
geometric series $a_n = r^n$ is
\[
s_n = \sum_{k=0}^{n} r^k= \frac{1 - r^{n+1}}{1-r}.
\]
We now have an \emph{explicit} formula so we can determine for which values of $r$ the limit $\lim_{n \to \infty} s_n$ exists.  First, note that the limit in question, $\lim_{n \to \infty} r^{n+1}$ is the limit of a \emph{geometric} sequence.  In fact, 

\begin{itemize}
\item if $-1<r<1$, then $\lim_{n \to \infty} r^{n+1}$ \wordChoice{\choice[correct]{exists}\choice{does not exist}}.
\item if $r>1$ or $r\le -1$, then $\lim_{n \to \infty} r^{n+1}$ \wordChoice{\choice{exists}\choice[correct]{does not exist}}.
\end{itemize}

In fact, if $r>1$, the $r^{n+1}$ is not bounded above.  If $r=-1$, the $r^{n+1}$ is bounded, but the terms oscillate between $-1$ and $1$.  If $r>1$, the terms $r^{n+1}$ both oscillate in sign and become arbitrarily large in magnitude.

The above formula covers every case except when $r= 1$, but notice that  $$\sum_{k=0}^n 1 = n+1,$$ so if $r=1$, $s_n = \answer[given]{n+1}$ and $\lim_{n \to \infty} s_n = \infty$, so $\sum_{k=0}^{\infty} 1$ diverges. 

When $-1<r<1$, note $\lim_{n \to \infty} r^{n+1}=0$, so in this case,     \[
    \lim_{n\to\infty}\frac{1 - r^{n+1}}{1-r} = \frac{1-\answer[given]{0}}{1-r}.
    \]

By noting that $\sum_{k=0}^n ar^k = a \sum_{k=0}^n r^k$, we can combine this observation with the above argument and write the result in a theorem.

\begin{theorem}
  The geometric series $\sum_{k= 0}^\infty a \cdot r^k$ 
  
  \begin{itemize} 
  \item converges to $\frac{a}{1-r}$ when $|r| < 1$.
  \item diverges if $|r| \geq 1$.  
  \end{itemize}
  \end{theorem}

\begin{example}
Let's take another look at the series that started off the section, $\sum_{k=1}^{\infty} \left(\frac{1}{2}\right)^k$.  Here, $a=0$ and $r=1/2$, but $\frac{a}{1-r} = \frac{1}{1-1/2} =2$.  What is happening?

The starting index for a sequence and series is very important.  When $|r|<1$, the above result only gives us a formula to add together the terms of a sequence whose lower index is $0$.  Let's write out a few terms and see what's happening.

\begin{image}
  \begin{tikzpicture}
        \node at (0,0) {
          $\underbrace{\sum_{k=0}^{\infty} \left(\frac{1}{2}\right)^k}= \left(\frac{1}{2}\right)^0+ \underbrace{\left(\frac{1}{2}\right)^1+ \left(\frac{1}{2}\right)^2 + \ldots}$};
        \node at (1.6,-.8) {\small{This is the sum from}};
        \node at (1.6,-1.1) {\small{the original example.}};
        
        \node at (-2.5,-.9) {\small{By the formula,}};
        \node at (-2.5,-1.2) {\small{this is $2$.}};        
      \end{tikzpicture}
  \end{image}
  
Thus, we can write 

\[2 = \left(\frac{1}{2}\right)^0 +\sum_{k=1}^{\infty} \left(\frac{1}{2}\right)^k.\]

and confirm that $\sum_{k=1}^{\infty} \left(\frac{1}{2}\right)^k =1$ after some quick algebra.
\end{example}

The above example illustrate an important fact.

\begin{quote}
The lower index in a series does not affect whether the series converges or diverges, but if the series converges, it can affect the value to which the series converges.
\end{quote}

There is a useful trick that allows us to find the sum of a convergent geometric series when the lower index does not start at $0$.  

\begin{example}
The series $\sum_{k=3}^{\infty} \left(\frac{2}{3}\right)^k$ is a geometric series with $r=\frac{2}{3}<1$, so it converges.  To find the value to which it converges, notice the following.

\begin{align*}
\sum_{k=3}^{\infty} \left(\frac{2}{3}\right)^k &=  \left(\frac{2}{3}\right)^3+ \left(\frac{2}{3}\right)^4+ \left(\frac{2}{3}\right)^5+\ldots \\
&= \left(\frac{2}{3}\right)^3 \cdot \left(1+ \frac{2}{3}+ \left(\frac{2}{3}\right)^2+\ldots\right) \\
&= \frac{8}{27}  \cdot  \sum_{k=0}^{\infty}\left(\frac{2}{3}\right)^k \\
&= \sum_{k=0}^{\infty} \frac{8}{27}  \cdot \left(\frac{2}{3}\right)^k
\end{align*}
This is now a geometric series whose lower index is $0$, so we can use the formula to find its value. Noting that $a=\answer[given]{ \frac{8}{27} }$ and $r= \frac{2}{3}$ gives:

\[
\sum_{k=3}^{\infty} \left(\frac{2}{3}\right)^k = \frac{8/27}{1-2/3} = \frac{8}{9}.
\]
\end{example}

We can easily generalize this example and doing so allows us to write down a more comprehensive theorem about geometric series.

\begin{theorem}
\index{series!geometric}\index{geometric series}\index{geometric series!convergence}\index{geometric series!divergence}
  The geometric series $\sum_{k= k_0}^\infty a \cdot r^k$ 
  
  \begin{itemize} 
  \item converges to $\frac{ar^{k_0}}{1-r}$ when $|r| < 1$.
  \item diverges if $|r| \geq 1$.  
  \end{itemize}
  \end{theorem}
  
  
%According to the theorem above the series
%\[
%\sum_{k=0}^\infty \left(\frac{1}{4}\right)^k = 1 + \frac{1}{4} + \left(\frac{1}{4}\right)^2 + \left(\frac{1}{4}\right)^3 + \cdots
%\]
%converges, and
%\[
%\sum_{k=0}^\infty  \left(\frac{1}{4}\right)^k = \frac{1}{1-1/4} = 4/3.
%\]
%This concurs with our introductory example; while there we got a sum
%of $1/3$, we skipped the first term of $1$.
%
%\begin{warning}
%  You must pay close attention to how the series is indexed, since
%  \[
%  \sum_{k=0}^n r^k \ne \sum_{k=1}^n r^k.
%  \]
%\end{warning}

Now, try some questions to check your understanding of the above material.

\begin{question}
  Which of the following series converge?
  \begin{selectAll}
    \choice{$\sum_{k=0}^\infty \left(\frac{3}{2}\right)^k$}
    \choice[correct]{$\sum_{k=0}^\infty \left(\frac{-2}{3}\right)^k$}
    \choice[correct]{$\sum_{k=9}^\infty \left(\frac{1}{7}\right)^k$}
    \choice{$\sum_{k=1}^\infty (-1)^k$}
    \choice[correct]{$\sum_{k=-9}^\infty \left(\frac{1}{2}\right)^k$}    
  \end{selectAll}
  \begin{hint}
    The initial index doesn't matter as far as convergence is
    concerned, it is the ``tail'' of the sequence that determines
    convergence.
  \end{hint}
\end{question}

\begin{question}
Determine if the series $\sum_{k=2}^{\infty} 2^{3-2k}$ converges or diverges.  If it converges, give the value to which it converges.

\begin{explanation}
First, note that the series \wordChoice{\choice[correct]{is}\choice{is not}} geometric since the laws of exponents allow us to write the following.

\[
2^{3-2k} = \frac{2^3}{2^{2k}} = \frac{8}{\left(2^2\right)^k} =  8 \cdot \frac{1}{4^k} =  8 \cdot \frac{1^k}{4^k} =  8 \cdot \left(\frac{1}{4}\right)^k
\]

The series is geometric with $r = \answer[given]{1/4}$, and using the result $\sum_{k=k_0} ar^k = \frac{ar^{k_0}}{1-r}$ gives:

\[
\sum_{k=2}^{\infty} 2^{3-2k} =  8 \cdot \left(\frac{1}{4}\right)^k =  \frac{8 \cdot (1/4)^2}{1-1/4}  =  \answer[given]{\frac{2}{3}}.
\]
\end{explanation}

\end{question}

%Let's use our new tools to find the sums of some geometric series.
%
%\begin{example}
%If the series 
%\[
%\sum_{n=-2}^\infty \left(\frac{3}{4}\right)^n
%\]
%converges, find its sum.
%\begin{explanation}
%Since the common ratio between the terms of this series is
%$\answer[given]{3/4}$, we see that this series
%\wordChoice{\choice[correct]{converges}\choice{diverges}}. Write with me.
%\begin{align*}
%  S &= \left(\frac{3}{4}\right)^{-2} + \left(\frac{3}{4}\right)^{-1} + \left(\frac{3}{4}\right)^{0} + \cdots\\
%  \left(\frac{3}{4}\right) S &= \left(\frac{3}{4}\right)^{-1} + \left(\frac{3}{4}\right)^{0} + \left(\frac{3}{4}\right)^{1} + \cdots
%\end{align*}
%Subtracting these two lines we find
%\begin{align*}
%  S -  \left(\frac{3}{4}\right) S  &= \left(\frac{3}{4}\right)^{-2}\\
%  S\left(1-\frac{3}{4}\right) &= \left(\frac{3}{4}\right)^{-2}\\
%  S &= \frac{(3/4)^{-2}}{1-3/4}.
%\end{align*}
%\end{explanation}
%\end{example}
%
%
%
%
%\begin{example}
%If the series 
%\[
%\sum_{n=8}^\infty \left(\frac{-1}{2}\right)^n
%\]
%converges, find its sum.
%\begin{explanation}
%  Since the common ratio between the terms of this series is
%$\answer[given]{-1/2}$, we see that this series
%\wordChoice{\choice[correct]{converges}\choice{diverges}}. Write with me.
%\begin{align*}
%  S &= \left(\frac{-1}{2}\right)^{8} + \left(\frac{-1}{2}\right)^{9} + \left(\frac{-1}{2}\right)^{10} + \cdots\\
%  \left(\frac{-1}{2}\right) S &= \left(\frac{-1}{2}\right)^{9} + \left(\frac{-1}{2}\right)^{10} + \left(\frac{-1}{2}\right)^{11} + \cdots
%\end{align*}
%Subtracting these two lines we find
%\begin{align*}
%  S -  \left(\frac{-1}{2}\right) S  &= \left(\frac{-1}{2}\right)^{8}\\
%  S\left(1-\frac{-1}{2}\right) &= \left(\frac{-1}{2}\right)^{8}\\
%  S &= \frac{(-1/2)^{8}}{1+1/2}.
%\end{align*}
%\end{explanation}
%\end{example}

%%%%%%%MOVE BELOW TO EXERCISES%%%%%%%%%%%%%%%%%
%\subsection{Connections to decimals}
%
%Remember how we pointed out that 
%\[
%\frac{1}{3} = 0.3333333333\dots
%\]
%is a geometric series? We can use our techniques for summing geometric
%series to find fractions equal to given decimals.
%
%\begin{example}
%  Find a fraction equal to
%  \[
%  0.47474747474747\dots
%  \]
%  \begin{explanation}
%    Do this exactly the same way as the examples we've done
%    before. Write
%    \begin{align*}
%    N &=      0.47474747474747\dots\\
%    100 N &= 47.47474747474747\dots
%    \end{align*}
%    Now subtract the top line from the bottom line, to find
%    \begin{align*}
%      100N - N &= 47\\
%      N(100-1) &= 47\\
%      N &= \frac{47}{99}
%    \end{align*}
%  \end{explanation}
%\end{example}
%
%
%\begin{example}
%  Find a fraction equal to
%  \[
%  9.42764864864864864864\dots
%  \]
%  \begin{explanation}
%    Do this exactly the same way as the examples we've done
%    before. Write
%    \begin{align*}
%    N &=         9.42764864864864864864\dots\\
%    1000 N &= 9427.64864864864864864\dots
%    \end{align*}
%    Now subtract the top line from the bottom line, to find
%    \begin{align*}
%      1000N - N &= \\
%      N(1000-1) &= 9418.221\\
%      N &= \frac{9418.221}{999}.
%    \end{align*}
%    Multiplying the numerator and denominator of this fraction by $1000$, 
%     our fraction will be
%    \[
%    \frac{9418221}{999000}.
%    \]
%  \end{explanation}
%\end{example}
    

\section{Telescoping series}
\index{series!telescoping}\index{telescoping series}

A second type of series for which we can find an explicit formula for $s_n$ are ``telescoping series''.  Rather than try to give a formal definition, we think of telescoping series are infinite sums for which the required addition required to find a formula for $s_n$ can be done so many of the intermediate terms naturally cancel.  An example will make this point more clear.

\begin{example}
  Evaluate the sum
  \[
  \sum_{k=1}^\infty\left(\frac{1}{k}-\frac{1}{k+1}\right).
  \]
  \begin{explanation}
It will help to write down the first few partial sums for this series.
\begin{image}
\begin{tikzpicture}
    \node at (0,0) {
      $\begin{aligned}
        s_1 &=	\frac11-\frac12 & & = 1-\frac12\\
        s_2 &=	\left(\frac11-\frac12\right) + \left(\frac12-\frac13\right) & & = 1-\frac13\\
        s_3 &=	\left(\frac11-\frac12\right) + \left(\frac12-\frac13\right)+\left(\frac13-\frac14\right) & &= 1-\frac14\\
        s_4 &=	\left(\frac11-\frac12\right) + \left(\frac12-\frac13\right)+\left(\frac13-\frac14\right) +\left(\frac14-\frac15\right)& &= 1-\frac15
      \end{aligned}$};
\end{tikzpicture}
\end{image}
For $s_2$ and beyond, note how most of the intermediate terms in each partial sum cancel out! In
general, we can notice from pattern recognition (specifically by looking at the denominator in each expression and comparing it to the index) that $s_n =$ \wordChoice{\choice{$1-\frac{1}{n}$},\choice[correct]{$1-\frac{1}{n+1}$}}. The sequence $\{s_n\}_{n=1}$ thus \wordChoice{\choice[correct]{converges}\choice{diverges}} since $\lim_{n \to \infty} s_n$ \wordChoice{\choice[correct]{exists}\choice{does not exist}}. Furthermore, since $\lim_{n\to\infty}s_n = \lim_{n\to\infty}\left(1-\frac1{n+1}\right) = \answer[given]1$, we conclude
that $\sum_{n=1}^\infty \left(\frac1n-\frac1{n+1}\right) = 1$.
  \end{explanation}
\end{example}

\begin{remark}
Finding the above formula required us to use pattern recognition.  Validating that the pattern must hold for \emph{all} terms in the sequence can be done formally by using an idea called \emph{mathematical induction}.  We leave it to the curious reader to explore this idea further if desired.
\end{remark}

We've just seen an example of a \dfn{telescoping series}. Informally,
a telescoping series is one in which the partial sums reduce to just a
finite sum of terms. In the last example, the partial sum $s_n$ only was the sum of two nonzero terms: 
\[
s_n = 1 - \frac{1}{n-1}.
\]

\begin{example}
Determine if the series $\sum_{n=1}^\infty \frac{2}{n^2+2n}$ converges or diverges.  If it converges, find the value to which it converges.

\begin{explanation}
All of the terms in the above sum are positive, so there is no convenient cancellation that will occur if we try to find a formula for $s_n$ yet.  However, we can use partial fractions to write
  \[
  \frac2{n^2+2n} = \frac1n-\frac1{n+2}.
  \]  
  Expressing the terms of $\{s_n\}$ now produces a pattern.  
  \begin{image}
    \begin{tikzpicture}
      \node at (0,0) {
        $\begin{aligned}
          s_1 &= 1-\frac13 &&= 1-\frac13\\
          s_2 &= \left(1-\frac13\right) + \left(\frac12-\frac14\right) &&= 1+\frac12-\frac13-\frac14\\
          s_3 &= \left(1-\frac13\right) + \left(\frac12-\frac14\right)+\left(\frac13-\frac15\right) &&= 1+\frac12-\frac14-\frac15\\
          s_4 &= \left(1-\frac13\right) + \left(\frac12-\frac14\right)+\left(\frac13-\frac15\right)+\left(\frac14-\frac16\right) &&= 1+\frac12-\frac15-\frac16\\
          s_5 &= \left(1-\frac13\right) + \left(\frac12-\frac14\right)+\left(\frac13-\frac15\right)+\left(\frac14-\frac16\right)+\left(\frac15-\frac17\right) &&= 1+\frac12-\frac16-\frac17\\
        \end{aligned}$};
    \end{tikzpicture}
  \end{image}
\textbf{I WANT TO WRITE SOMETHING ABOUT THE DENOMINATORS BEING 2 APART, WHICH IS WHYIT TAKES WRITING OUT 3 TERMS UNTIL TO EXHIBIT THE EVENTUAL PATTERN.  CAN ANYONE HELP ME WORD THIS?}

We again have a telescoping series. In each partial sum, most of the intermediate 
  terms cancel and we obtain the formula
  \[
  s_n =1+\frac12-\frac1{n+1}-\frac1{n+2}.
  \]
  Taking limits allows us to determine the convergence of the series. Since
  \[
  \lim_{n\to\infty}s_n = \lim_{n\to\infty} \left(1+\frac12-\frac1{n+1}-\frac1{n+2}\right) = \frac32,
  \]
we conclude that
  \[
  \sum_{n=1}^\infty \frac1{n^2+2n} = \frac32.
  \]
\end{explanation}
\end{example}



\begin{example}
Determine if the series $\sum_{k=1}^\infty \ln\left(\frac{k+1}{k}\right)$ converges or diverges. 
 
\begin{explanation}
We begin by writing the first few partial sums of the series:
\begin{align*}
s_1 &= \ln\left(2\right) \\
s_2 &= \ln\left(2\right)+\ln\left(\frac32\right) \\
s_3 &= \ln\left(2\right)+\ln\left(\frac32\right)+\ln\left(\frac43\right) \\
s_4 &= \ln\left(2\right)+\ln\left(\frac32\right)+\ln\left(\frac43\right)+\ln\left(\frac54\right) 
\end{align*}
At first, it doesn't look like we will have much luck writing this as a telescoping series, but noting that $ \ln\left(\frac{n+1}{n}\right) = \ln(n+1)-\ln(n)$ allows us to write out terms of $s_n$ in a more convenient way.

  \begin{image}
    \begin{tikzpicture}
      \node at (0,0) {
        $\begin{aligned}
          s_1 &= \ln(2)-\ln(1) &&= \ln(2)\\
          s_2 &= \left( \ln(3)-\ln(2)\right) + \left( \ln(2)-\ln(1)\right) &&= \ln(3)\\
          s_3 &= \left( \ln(4)-\ln(3)\right) + \left( \ln(3)-\ln(2)\right) + \left( \ln(2)-\ln(1)\right) &&= \ln(4)\\
%          s_4 &=  \left( \ln(5)-\ln(4)\right) +\left( \ln(4)-\ln(3)\right) + \left( \ln(3)-\ln(2)\right) + \left( \ln(2)-\ln(1)\right) &&= \ln(5)\\
        \end{aligned}$};
    \end{tikzpicture}
  \end{image}
  
  

%At first, this does not seem helpful, but recall the logarithmic rule
%$\ln(x)+\ln(y) = \ln (x\cdot y)$. Applying this rule to $S_4$ gives:
%\begin{align*}
%S_4 &= \ln\left(2\right)+\ln\left(\frac32\right)+\ln\left(\frac43\right)+\ln\left(\frac54\right) \\
%&= \ln\left(\frac21\cdot\frac32\cdot\frac43\cdot\frac54\right)\\
%&= \ln\left(5\right).

We can conclude that $s_n =\answer[given]{\ln (n+1)}$ and analyze $\lim_{n \to \infty} s_n$.  

Since $\lim_{n\to\infty}s_n=\answer[given]{\infty}$, $\sum_{k=1}^\infty \ln\left(\frac{k+1}{k}\right)$ diverges.
\end{explanation}
\end{example}


\section{Properties of sums}
\textbf{WHAT ROLE DO WE WANT THESE TO PLAY? I'LL LEAVE THEM FOR NOW.}


We are learning about a new mathematical object, the series. As done
before, we apply ``old'' mathematics to this new topic.

\begin{theorem}[Properties of Infinite Series]
  Let
  \[
  \sum_{n=1}^\infty a_n = L,\quad \sum_{n=1}^\infty b_n =K, 
  \]
  and let $c$ be a constant.
\begin{enumerate}
\item Constant Multiple Rule: $\sum_{n=1}^\infty c\cdot a_n =
  c\cdot\sum_{n=1}^\infty a_n = c\cdot L.$\index{series!Constant Multiple
    Rule}
\item Sum/Difference Rule: $\sum_{n=1}^\infty \big(a_n\pm b_n\big) =
  \sum_{n=1}^\infty a_n \pm \sum_{n=1}^\infty b_n = L \pm K.$
  \index{series!properties}\index{series!Sum/Difference Rule}
\end{enumerate} 
\end{theorem}

Notice, of course, that we're working with convergent series in this 
theorem.  The results don't necessarily hold if $\sum_{n=1}^\infty a_n$ 
or $\sum_{n=1}^\infty b_n$ are divergent!

\section{Understanding the relationship between sequences and series.}

For a given series $\sum_{k=k_0}^{\infty} a_k$, we've previously defined the associated sequence of partial sums and worked several examples that illustrate how this sequence can be used to determine whether the original series converges or diverges.  The sequence of partial sums will play a pivotal role in the coming sections, and it will become increasingly important that you are mindful of notation and have a clear picture of what each object we define is and the role it plays in determining convergence.  

\textbf{Not sure I captured what I wanted to in this last sentence.  Trying to encourage they pay attention and start classifying what things are and the roles they play.  Let's discuss at the meeting}

The following examples are designed to increase your understanding of the terminology and concepts presented so far.  They also require that you read what information is given very carefully and understand how it relates to what you are trying to determine.

\begin{example}
Suppose that $\{a_k\}_{k=1}$ is a sequence and let $s_n = \sum_{k=1}^{\infty} a_k$.  Suppose that it is known that $s_n = \left(\frac{2}{3}\right)^n.$

\begin{question}
Determine if $\sum_{k=1}^{\infty} a_k$ converges or diverges. If it converges, give the value to which it converges.

\begin{explanation}
We are given information about $\{s_n\}$, not the original sequence $\{a_n\}$.  In order to determine whether $\sum_{k=1}^{\infty} a_k$ converges or diverges, we should study $\lim_{n \to \infty} s_n$.  By noting that

\[
\lim_{n \to \infty} s_n = \lim_{n \to \infty} \left(\frac{2}{3}\right)^n = \answer[given]{0},
\]
we know that $\sum_{k=1}^{\infty} a_k$  converges since $\lim_{n \to \infty} s_n$ exists.  

Furthermore, since $\lim_{n \to \infty} s_n = \answer[given]{0}$, $\sum_{k=1}^{\infty} a_k$ converges to $\answer[given]{0}$.
\end{explanation}
\end{question}

\begin{question}
Determine if $\sum_{k=1}^{\infty} s_k$ converges or diverges. If it converges, give the value to which it converges.

\begin{explanation}
Note that we are now being asked to sum the terms of a sequence whose $n$-th term is given in the problem statement.  That is, we are asked to determine whether

\[
\sum_{k=1}^{\infty} s_k = \sum_{k=1}^{\infty} \left(\frac{2}{3}\right)^k
\]
converges or diverges.  This is simply a geometric \emph{series} with $r = 2/3 <1$, so it converges, and using the result that for $|r|<1$, $\sum_{k=k_0}^{\infty} ar^k = \frac{ar^{k_0}}{1-r}$, we find:

\[
\sum_{k=1}^{\infty} s_k = \sum_{k=1}^{\infty} \left(\frac{2}{3}\right)^k = \frac{2/3}{1-2/3} = 2.
\]

\begin{remark}
Thinking about summing the terms in a sequence of partial sums may seem daunting at first, but notice here that we actually know what all of the terms are since we are given an explicit formula for $s_n$.  Even though the terms in $s_n$ represent something, the sequence $\{s_n\}_{n=1}$ is a sequence in its own right and we the question as to whether we can sum up all of the terms in this list. 

At the risk of being repetitive, we expound further.  The sequence $\{s_n\}_{n=1}$ is represented by the list below:

\[
\left(\frac{2}{3}\right)^1 , \left(\frac{2}{3}\right)^2 ,  \left(\frac{2}{3}\right)^3 , \ldots  ~, 
\]
and the actual sum is simply:

\[
\frac{2}{3} + \left(\frac{2}{3}\right)^2 +  \left(\frac{2}{3}\right)^3 + \ldots    = \sum_{k=1}^{\infty} \left(\frac{2}{3}\right)^k
\]
\end{remark}

\end{explanation}

\end{question}
\end{example}

\begin{remark}
The above example is one of many that we will see of the principle that began this section.

\begin{quote}
Suppose that we have a sequence.  Perhaps the terms in the sequence carry some information about other sequences, and perhaps the sequence was constructed from other sequences.  Regardless, we can ask two fundamental questions of our sequence.  

\begin{itemize}
\item[1.] Do the numbers in the list approach a finite value?
\item[2.] Can I sum all of the numbers in the list and obtain a finite result?
\end{itemize}
\end{quote}
\end{remark}




\begin{example}
Suppose that $\{a_n\}_{n=1}$ is a sequence and define its sequence of partial sums $\{s_n\}_{n=1}$ by the usual rule $s_n = \sum_{k=1}^n a_k$.  Suppose it is known that:

\[
s_n = \frac{8n^2}{n^4-9}
\]

\begin{question}
What is $a_1+a_2+a_3$?  \wordChoice{\choice{$-1$}\choice{$\frac{18}{7}$}\choice{It is not defined.}}

\begin{explanation}
Note that we are given the formula for the terms $s_n$ in the sequence $\{s_n\}$.  By definition, we have $s_3 = a_1+a_2+a_3$, so all we have to do to find $a_1+a_2+a_3$ is to evaluate the formula for $s_n$ when $n=3$.

\begin{align*}
a_1+a_2+a_3 &= s_3 \\
& = \frac{8(3)^2}{(3)^4-9} \\
&= 1
\end{align*}
\end{explanation}
\end{question}

\begin{question}
What is $a_2+a_3$?

\begin{explanation}
Note that $s_3 = a_1+a_2+a_3$ and $s_1 = a_1$.  Thus, 

\begin{tabular}{rl}
$s_3$ &= $\cancel{a_1}+a_2+a_3$\\
$-(~ s_1$ &= $\cancel{a_1} ~)$\\
\hline
$s_3-s_1$ &= $a_2+a_3$\\
\end{tabular}

Using the formula for $s_n$ gives $s_3 = \answer[given]{1}$ and $s_1=\answer[given]{-1}$, so $a_1+a_2 =  \answer[given]{2}$.
\end{explanation}
\end{question}

\begin{question}
Determine whether $\sum_{k=1}^{\infty} a_k$ converges or diverges.  If it converges, can you find its value?

\begin{explanation}
We can determine whether $\sum_{k=1}^{\infty} a_k$ converges or diverges by analyzing $\lim_{n \to \infty} s_n = \lim_{n \to \infty} \frac{8n^2}{n^4-9}$.  Since this limit is zero, we know that $\sum_{k=1}^{\infty} a_k$ converges to $0$.
\end{explanation}
\end{question}

\begin{question}
Determine whether $\sum_{k=4}^{\infty} a_k$ converges or diverges.  If it converges, can you find its value?

\begin{explanation}
The lower index $k=4$ \wordChoice{\choice{will}\choice[correct]{will not}}  affect whether the series converges.  Since we found that $\sum_{k=1}^{\infty} a_k$ converges, so too will $\sum_{k=4}^{\infty} a_k$.  The fact that $k=4$ is our lower index here \wordChoice{\choice[correct]{will}\choice{will not}} potentially affect the value of the series.

Let's write out the series and make a few observations.

\begin{image}
  \begin{tikzpicture}
        \node at (0,0) {
          $\underbrace{\sum_{k=1}^{\infty} a_k}=\overbrace{a_1+a_2+a_3}+ \underbrace{a_4+a_5 + \ldots}$};
        \node at (1.8,-.7) {\small{This is the series}};
        \node at (1.8,-1) {\small{we want to find.}};
        
        \node at (-2.5,-.9) {\small{This is $0$ from }};
        \node at (-2.5,-1.2) {\small{the last part.}};    
        
        \node at (-.6,.9) {\small{This is $s_3$. }};
        
      \end{tikzpicture}
  \end{image}
  
 Putting this together, we have:
 
 \[
 0 = s_3 + \sum_{k=4}^{\infty} a_k,
 \] 
 and we find that $ \sum_{k=4}^{\infty} a_k = -1$. 

We can determine whether $\sum_{k=1}^{\infty} a_k$ converges or diverges by analyzing $\lim_{n \to \infty} s_n = \lim_{n \to \infty} \frac{8n^2}{n^4-9}$.  Since this limit is zero, we know that $\sum_{k=1}^{\infty} a_k$ converges to $0$.
\end{explanation}
\end{question}

\end{example}





\end{document}
