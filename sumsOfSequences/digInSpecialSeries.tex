\documentclass{ximera}

%\usepackage{todonotes}
%\usepackage{mathtools} %% Required for wide table Curl and Greens
%\usepackage{cuted} %% Required for wide table Curl and Greens
\newcommand{\todo}{}

\usepackage{esint} % for \oiint
\ifxake%%https://math.meta.stackexchange.com/questions/9973/how-do-you-render-a-closed-surface-double-integral
\renewcommand{\oiint}{{\large\bigcirc}\kern-1.56em\iint}
\fi


\graphicspath{
  {./}
  {ximeraTutorial/}
  {basicPhilosophy/}
  {functionsOfSeveralVariables/}
  {normalVectors/}
  {lagrangeMultipliers/}
  {vectorFields/}
  {greensTheorem/}
  {shapeOfThingsToCome/}
  {dotProducts/}
  {partialDerivativesAndTheGradientVector/}
  {../productAndQuotientRules/exercises/}
  {../normalVectors/exercisesParametricPlots/}
  {../continuityOfFunctionsOfSeveralVariables/exercises/}
  {../partialDerivativesAndTheGradientVector/exercises/}
  {../directionalDerivativeAndChainRule/exercises/}
  {../commonCoordinates/exercisesCylindricalCoordinates/}
  {../commonCoordinates/exercisesSphericalCoordinates/}
  {../greensTheorem/exercisesCurlAndLineIntegrals/}
  {../greensTheorem/exercisesDivergenceAndLineIntegrals/}
  {../shapeOfThingsToCome/exercisesDivergenceTheorem/}
  {../greensTheorem/}
  {../shapeOfThingsToCome/}
  {../separableDifferentialEquations/exercises/}
  {vectorFields/}
}

\newcommand{\mooculus}{\textsf{\textbf{MOOC}\textnormal{\textsf{ULUS}}}}

\usepackage{tkz-euclide}\usepackage{tikz}
\usepackage{tikz-cd}
\usetikzlibrary{arrows}
\tikzset{>=stealth,commutative diagrams/.cd,
  arrow style=tikz,diagrams={>=stealth}} %% cool arrow head
\tikzset{shorten <>/.style={ shorten >=#1, shorten <=#1 } } %% allows shorter vectors

\usetikzlibrary{backgrounds} %% for boxes around graphs
\usetikzlibrary{shapes,positioning}  %% Clouds and stars
\usetikzlibrary{matrix} %% for matrix
\usepgfplotslibrary{polar} %% for polar plots
\usepgfplotslibrary{fillbetween} %% to shade area between curves in TikZ
\usetkzobj{all}
\usepackage[makeroom]{cancel} %% for strike outs
%\usepackage{mathtools} %% for pretty underbrace % Breaks Ximera
%\usepackage{multicol}
\usepackage{pgffor} %% required for integral for loops



%% http://tex.stackexchange.com/questions/66490/drawing-a-tikz-arc-specifying-the-center
%% Draws beach ball
\tikzset{pics/carc/.style args={#1:#2:#3}{code={\draw[pic actions] (#1:#3) arc(#1:#2:#3);}}}



\usepackage{array}
\setlength{\extrarowheight}{+.1cm}
\newdimen\digitwidth
\settowidth\digitwidth{9}
\def\divrule#1#2{
\noalign{\moveright#1\digitwidth
\vbox{\hrule width#2\digitwidth}}}





\newcommand{\RR}{\mathbb R}
\newcommand{\R}{\mathbb R}
\newcommand{\N}{\mathbb N}
\newcommand{\Z}{\mathbb Z}

\newcommand{\sagemath}{\textsf{SageMath}}


%\renewcommand{\d}{\,d\!}
\renewcommand{\d}{\mathop{}\!d}
\newcommand{\dd}[2][]{\frac{\d #1}{\d #2}}
\newcommand{\pp}[2][]{\frac{\partial #1}{\partial #2}}
\renewcommand{\l}{\ell}
\newcommand{\ddx}{\frac{d}{\d x}}

\newcommand{\zeroOverZero}{\ensuremath{\boldsymbol{\tfrac{0}{0}}}}
\newcommand{\inftyOverInfty}{\ensuremath{\boldsymbol{\tfrac{\infty}{\infty}}}}
\newcommand{\zeroOverInfty}{\ensuremath{\boldsymbol{\tfrac{0}{\infty}}}}
\newcommand{\zeroTimesInfty}{\ensuremath{\small\boldsymbol{0\cdot \infty}}}
\newcommand{\inftyMinusInfty}{\ensuremath{\small\boldsymbol{\infty - \infty}}}
\newcommand{\oneToInfty}{\ensuremath{\boldsymbol{1^\infty}}}
\newcommand{\zeroToZero}{\ensuremath{\boldsymbol{0^0}}}
\newcommand{\inftyToZero}{\ensuremath{\boldsymbol{\infty^0}}}



\newcommand{\numOverZero}{\ensuremath{\boldsymbol{\tfrac{\#}{0}}}}
\newcommand{\dfn}{\textbf}
%\newcommand{\unit}{\,\mathrm}
\newcommand{\unit}{\mathop{}\!\mathrm}
\newcommand{\eval}[1]{\bigg[ #1 \bigg]}
\newcommand{\seq}[1]{\left( #1 \right)}
\renewcommand{\epsilon}{\varepsilon}
\renewcommand{\phi}{\varphi}


\renewcommand{\iff}{\Leftrightarrow}

\DeclareMathOperator{\arccot}{arccot}
\DeclareMathOperator{\arcsec}{arcsec}
\DeclareMathOperator{\arccsc}{arccsc}
\DeclareMathOperator{\si}{Si}
\DeclareMathOperator{\scal}{scal}
\DeclareMathOperator{\sign}{sign}


%% \newcommand{\tightoverset}[2]{% for arrow vec
%%   \mathop{#2}\limits^{\vbox to -.5ex{\kern-0.75ex\hbox{$#1$}\vss}}}
\newcommand{\arrowvec}[1]{{\overset{\rightharpoonup}{#1}}}
%\renewcommand{\vec}[1]{\arrowvec{\mathbf{#1}}}
\renewcommand{\vec}[1]{{\overset{\boldsymbol{\rightharpoonup}}{\mathbf{#1}}}\hspace{0in}}

\newcommand{\point}[1]{\left(#1\right)} %this allows \vector{ to be changed to \vector{ with a quick find and replace
\newcommand{\pt}[1]{\mathbf{#1}} %this allows \vec{ to be changed to \vec{ with a quick find and replace
\newcommand{\Lim}[2]{\lim_{\point{#1} \to \point{#2}}} %Bart, I changed this to point since I want to use it.  It runs through both of the exercise and exerciseE files in limits section, which is why it was in each document to start with.

\DeclareMathOperator{\proj}{\mathbf{proj}}
\newcommand{\veci}{{\boldsymbol{\hat{\imath}}}}
\newcommand{\vecj}{{\boldsymbol{\hat{\jmath}}}}
\newcommand{\veck}{{\boldsymbol{\hat{k}}}}
\newcommand{\vecl}{\vec{\boldsymbol{\l}}}
\newcommand{\uvec}[1]{\mathbf{\hat{#1}}}
\newcommand{\utan}{\mathbf{\hat{t}}}
\newcommand{\unormal}{\mathbf{\hat{n}}}
\newcommand{\ubinormal}{\mathbf{\hat{b}}}

\newcommand{\dotp}{\bullet}
\newcommand{\cross}{\boldsymbol\times}
\newcommand{\grad}{\boldsymbol\nabla}
\newcommand{\divergence}{\grad\dotp}
\newcommand{\curl}{\grad\cross}
%\DeclareMathOperator{\divergence}{divergence}
%\DeclareMathOperator{\curl}[1]{\grad\cross #1}
\newcommand{\lto}{\mathop{\longrightarrow\,}\limits}

\renewcommand{\bar}{\overline}

\colorlet{textColor}{black}
\colorlet{background}{white}
\colorlet{penColor}{blue!50!black} % Color of a curve in a plot
\colorlet{penColor2}{red!50!black}% Color of a curve in a plot
\colorlet{penColor3}{red!50!blue} % Color of a curve in a plot
\colorlet{penColor4}{green!50!black} % Color of a curve in a plot
\colorlet{penColor5}{orange!80!black} % Color of a curve in a plot
\colorlet{penColor6}{yellow!70!black} % Color of a curve in a plot
\colorlet{fill1}{penColor!20} % Color of fill in a plot
\colorlet{fill2}{penColor2!20} % Color of fill in a plot
\colorlet{fillp}{fill1} % Color of positive area
\colorlet{filln}{penColor2!20} % Color of negative area
\colorlet{fill3}{penColor3!20} % Fill
\colorlet{fill4}{penColor4!20} % Fill
\colorlet{fill5}{penColor5!20} % Fill
\colorlet{gridColor}{gray!50} % Color of grid in a plot

\newcommand{\surfaceColor}{violet}
\newcommand{\surfaceColorTwo}{redyellow}
\newcommand{\sliceColor}{greenyellow}




\pgfmathdeclarefunction{gauss}{2}{% gives gaussian
  \pgfmathparse{1/(#2*sqrt(2*pi))*exp(-((x-#1)^2)/(2*#2^2))}%
}


%%%%%%%%%%%%%
%% Vectors
%%%%%%%%%%%%%

%% Simple horiz vectors
\renewcommand{\vector}[1]{\left\langle #1\right\rangle}


%% %% Complex Horiz Vectors with angle brackets
%% \makeatletter
%% \renewcommand{\vector}[2][ , ]{\left\langle%
%%   \def\nextitem{\def\nextitem{#1}}%
%%   \@for \el:=#2\do{\nextitem\el}\right\rangle%
%% }
%% \makeatother

%% %% Vertical Vectors
%% \def\vector#1{\begin{bmatrix}\vecListA#1,,\end{bmatrix}}
%% \def\vecListA#1,{\if,#1,\else #1\cr \expandafter \vecListA \fi}

%%%%%%%%%%%%%
%% End of vectors
%%%%%%%%%%%%%

%\newcommand{\fullwidth}{}
%\newcommand{\normalwidth}{}



%% makes a snazzy t-chart for evaluating functions
%\newenvironment{tchart}{\rowcolors{2}{}{background!90!textColor}\array}{\endarray}

%%This is to help with formatting on future title pages.
\newenvironment{sectionOutcomes}{}{}



%% Flowchart stuff
%\tikzstyle{startstop} = [rectangle, rounded corners, minimum width=3cm, minimum height=1cm,text centered, draw=black]
%\tikzstyle{question} = [rectangle, minimum width=3cm, minimum height=1cm, text centered, draw=black]
%\tikzstyle{decision} = [trapezium, trapezium left angle=70, trapezium right angle=110, minimum width=3cm, minimum height=1cm, text centered, draw=black]
%\tikzstyle{question} = [rectangle, rounded corners, minimum width=3cm, minimum height=1cm,text centered, draw=black]
%\tikzstyle{process} = [rectangle, minimum width=3cm, minimum height=1cm, text centered, draw=black]
%\tikzstyle{decision} = [trapezium, trapezium left angle=70, trapezium right angle=110, minimum width=3cm, minimum height=1cm, text centered, draw=black]

\author{Jim Talamo}

\outcome{Define a series.}
\outcome{Recognize a geometric series.}
\outcome{Recognize a telescoping series.}
\outcome{Compute the sum of a geometric series.}
\outcome{Compute the sum of a telescoping series.}

\title[Dig-In:]{Special Series}

\begin{document}
\begin{abstract}
We discuss convergence results for geometric series and telescoping series.
\end{abstract}
\maketitle

Suppose that we have a \emph{series} $\sum_{k=k_0}^{\infty} a_k$ and have to determine whether it converges or diverges.  To answer this question, we define a new \emph{sequence} $\{s_n\}_{n=k_0}$ where $s_n = \sum_{k=k_0}^{n}$ for all $n \geq k_0$.  We saw previously that
 
 \begin{itemize}
\item the series $\sum_{k=k_0}^\infty a_k$ \dfn{converges} if and only if $\lim_{n\to\infty} s_n$ exists.  
\item the series $\sum_{k=k_0}^\infty a_k$ \dfn{diverges} if and only if $\lim_{n\to\infty} s_n$ does not exist.  
\end{itemize}

The definitions above give us a way to determine whether a given series converges.  In fact, to determine whether $\sum_{k=k_0}^{\infty} a_k$ converges, we can do the following.

\begin{itemize}
\item[1.] Consider the associated sequence $\{s_n\}$ of partial sums.
\item[2.] Try to find an explicit formula for the term $s_n$.  If you can find such a formula, analyze $\lim_{n \to \infty} s_n$.  
\begin{itemize}
\item If the limit exists, $\sum_{k=k_0}^{\infty} a_k$ converges, and if we can determine that $\lim_{n \to \infty} s_n =L$, then $\sum_{k=k_0} a_k=L$.  \item If  $\lim_{n \to \infty} s_n$ does not exist, then $\sum_{k=k_0} a_k$ diverges.
\end{itemize}
\item[3.] If an explicit formula for $s_n$ cannot be found, further analysis is needed.  We'll expound on this in later sections.
\end{itemize}


\section{A recursive formula for $s_n$}
The most straightforward way to determine whether $\lim_{n \to \infty} s_n$ exists is to have an explicit formula for the $n$-th term $s_n$.  Note that this is not an easy task; for example, can you find a formula for $s_n$ for the series $\sum_{k=1}^{\infty} \frac{1}{k}$? It's not too hard to write out the first several terms in the sequence $\{s_n\}_{n=1}$, but try to find an explicit formula for $s_n$!

As it turns out, there is always a recursive formula for $s_n$, and this will play an important role in later sections.  Suppose that we want to consider $\sum_{k=1}^{\infty} a_k$.  Let's write out the formula for $s_n$.

\[
s_n = a_1+a_2+a_3+\ldots+a_{n-1}+a_n
\]

We can make an observation by considering $s_{n-1}$ in a similar way.

\[
s_{n-1} = a_1+a_2+a_3+\ldots+a_{n-1}
\]

Now returning to our expression for $s_n$, we can make an observation. 
\begin{image}
  \begin{tikzpicture}
        \node at (0,0) {
          $s_n = \underbrace{a_1+a_2+a_3 + \ldots + a_{n-1}}+ a_n$};
        \node at (.1,-.65) {\small{This is $s_{n-1}$}};
      
      \end{tikzpicture}
  \end{image}

We thus have the formula 

\[
s_n = s_{n-1}+a_n.
\]

If we apply this to the series $\sum_{k=1}^{\infty} \frac{1}{k}$, we have $s_n = \sum_{k=1}^n \frac{1}{k}$ and $a_n = \frac{1}{n}$.  The recursive formula reads 

\begin{align*}
s_n &= s_{n-1} +a_n\\
s_n &= s_{n-1} +  \frac{1}{n}.
\end{align*}

This does not help us analyze whether $\lim_{n \to \infty} s_n$ actually exists.  Sometimes, however, we can find an \emph{explicit} formula for $s_n$, and we study two special types of series for which this is possible.

\section{Two special types of series}



\section{Geometric series}
Recall that a \emph{geometric sequence} is a sequence for which the ratio of successive terms is constant.  If $\{a_n\}_{n=n_0}$ is such a sequence, then there are constants $a \ne 0$ and $r$ for which $a_n = a\cdot r^n$.  

We thus represent this sequence by the ordered list

\[
ar^{n_0} , ar^{n_0+1}, ar^{n_0+2}, \ldots
\]

and we have a result that characterizes the behavior of this type of sequence, which we recall now.


\begin{theorem}
  Given a geometric sequence $\{a_n\}_{n=n_0}$ where $a_n = a \cdot r^{n}$,
  \[
  \lim_{n\to\infty} a_n =
  \begin{cases}
    0 &\text{if $|r|<1$,}\\
    1 &\text{if $r=1$,}\\
    \text{DNE} &\text{if $|r|>1$ or $r=-1$.}
  \end{cases}
  \]
\end{theorem}

We can now ask when we are able to sum the terms of a geometric sequence.

\begin{definition}
  A \dfn{geometric series} is a series of the form $\sum_{k=k_0}^\infty ar^k$
  for some real numbers $a \ne 0$ and $r$.
\end{definition}

Before exploring when such a series converges, note that sometimes, some preliminary algebra is necessary to recognize a series as geometric.

\begin{example}
The series $\sum_{k=4}^\infty \frac{2^{2k+1}}{3^k}$ \wordChoice{\choice[correct]{is}\choice{is not}} geometric since $a_k =\frac{2^{2k+1}}{3^k}$ \wordChoice{\choice[correct]{can}\choice{cannot}} be brought into the form $a \cdot r^k$.  

Using the laws of exponents shows us:

\[
\frac{2^{2k+1}}{3^k} = \frac{2^{2k} \cdot 2^1}{3^k}= 2 \cdot \frac{\left(2^{k}\right)^2}{3^k} = 2 \cdot \frac{4^k}{3^k} = 2 \cdot \left(\frac{4}{3}\right)^k.
\]
Indeed, $a= \answer[given]{2}$ and $r = \answer[given]{\frac{4}{3}}$.
\end{example}

\begin{example}
The series $\sum_{k=0}^\infty k^2 \left(\frac{1}{2}\right)^k$ \wordChoice{\choice{is}\choice[correct]{is not}} geometric since $a_k = \answer[given]{k^2 \cdot \frac{1}{2}^k}$ \wordChoice{\choice{can}\choice[correct]{cannot}} be brought into the form $a \cdot r^k$.  Indeed, the coefficient, $k^2$, is not the same for each term in the series.
\end{example}

We can now try to determine when adding together the terms in such a series is possible; that is, we can explore for which values of $a$ and $r$ the \emph{series} $\sum_{k=n_0}^{\infty} a_k$ converges.  

\begin{model}
 Let $r \neq 1$ and consider the geometric series $\sum_{k=0}^\infty a r^k$, and let $s_n = \sum_{k=0}^{\infty} a r^k $.  We find an explicit formula for the term $s_n$.
  
  \begin{explanation}
First, note that the sum above represents the attempt to add all of the terms in the sequence $\{a_n\}_{n=0}$, where $a_n =  r^n$.  Let's start by writing out $s_n$.  

    \begin{align*}
      s_n   &= 1 + r + r^2 + \dots + r^n\\
    \end{align*}

The issue with finding a formula for $s_n$ arises from the fact that we cannot perform the above for an unspecified value $n$.  However, to go from one term in the sum to the next, we multiply by $r$, so let's multiply both sides of the above equation by $r$.
    \begin{align*}
      s_n   &= 1 + r + r^2 + \dots + r^n\\
      r s_n &= ~ \phantom{ 1 + } r + r^2 + \dots + r^n + r^{n+1}
    \end{align*}
Now, we can subtract away the middle terms, which we show (with some slight abuse of notation) below.
 \[     \begin{array}{rl}
      s_n   &= 1 + \cancel{ r + r^2 + \dots + r^n}\\
 -\left(  \phantom{ r^{n+1}} r s_n \right.&=~ \left. \phantom{  1 +  } \cancel{r + r^2 + \dots + r^n} + r^{n+1}\right) \\
 \hline 
     s_n - r s_n &= 1 \phantom{  +  r + r^2 + \dots + r^n } ~ - r^{n+1}\\
    \end{array}
 \]   
 We can now solve for $s_n$.
     \begin{align*}
      s_n - r s_n &= 1 - r^{n+1}\\
      s_n(1-r)    &= 1 - r^{n+1}\\
      s_n &= \frac{1 - r^{n+1}}{1-r}.
    \end{align*}
    Since $r \ne 1$, there is no issue dividing by $1-r$; we will treat the case $r=1$ a bit later.
  \end{explanation}
\end{model}

From our work above, we see that the $n$-th partial sum of the
geometric series $a_n = r^n$ is
\[
s_n = \sum_{k=0}^{n} r^k= \frac{1 - r^{n+1}}{1-r}.
\]
We now have an \emph{explicit} formula so we can determine for which values of $r$ the limit $\lim_{n \to \infty} s_n$ exists.  First, note that by using the limit laws, 

\[
\lim_{n \to \infty} s_n = \lim_{n \to \infty}  \frac{1 - r^{n+1}}{1-r} =  \frac{1}{1-r}  - \frac{1}{1-r}  \lim_{n \to \infty} r^{n+1}.
\]

The existence of $\lim_{n \to \infty} s_n $is thus entirely determined by whether $\lim_{n \to \infty} r^{n+1}$ exists, and this limit is the limit of a geometric \emph{sequence}!  In fact, 

\begin{itemize}
\item if $-1<r<1$, then $\lim_{n \to \infty} r^{n+1}$ \wordChoice{\choice[correct]{exists}\choice{does not exist}}.
\item if $r>1$ or $r\le -1$, then $\lim_{n \to \infty} r^{n+1}$ \wordChoice{\choice{exists}\choice[correct]{does not exist}}.
\end{itemize}

The above formula covers every case except when $r= 1$, but notice that  $$\sum_{k=0}^n 1 = n+1,$$ so if $r=1$, $s_n = \answer[given]{n+1}$ and $\lim_{n \to \infty} s_n = \infty$, so $\sum_{k=0}^{\infty} 1$ diverges. 

When $-1<r<1$, note $\lim_{n \to \infty} r^{n+1}=0$, so in this case,     \[
    \lim_{n\to\infty}\frac{1 - r^{n+1}}{1-r} = \frac{1-\answer[given]{0}}{1-r}.
    \]

By noting that $\sum_{k=0}^n ar^k = a \sum_{k=0}^n r^k$, we can combine this observation with the above argument and write the result in a theorem.

\begin{theorem}
  The geometric series $\sum_{k= 0}^\infty a \cdot r^k$ 
  
  \begin{itemize} 
  \item converges to $\frac{a}{1-r}$ when $|r| < 1$.
  \item diverges if $|r| \geq 1$.  
  \end{itemize}
  \end{theorem}

There is a useful trick that allows us to find the sum of a convergent geometric series when the lower index does not start at $0$.  

\begin{example}
The series $\sum_{k=3}^{\infty} \left(\frac{2}{3}\right)^k$ is a geometric series with $r=\frac{2}{3}<1$, so it converges.  To find the value to which it converges, notice the following.

\begin{align*}
\sum_{k=3}^{\infty} \left(\frac{2}{3}\right)^k &=  \left(\frac{2}{3}\right)^3+ \left(\frac{2}{3}\right)^4+ \left(\frac{2}{3}\right)^5+\ldots \\
&= \left(\frac{2}{3}\right)^3 \cdot \left(1+ \frac{2}{3}+ \left(\frac{2}{3}\right)^2+\ldots\right) \\
&= \frac{8}{27}  \cdot  \sum_{k=0}^{\infty}\left(\frac{2}{3}\right)^k \\
&= \sum_{k=0}^{\infty} \frac{8}{27}  \cdot \left(\frac{2}{3}\right)^k
\end{align*}
This is now a geometric series whose lower index is $0$, so we can use the formula to find its value. Noting that $a=\answer[given]{ \frac{8}{27} }$ and $r= \frac{2}{3}$ gives:

\[
\sum_{k=3}^{\infty} \left(\frac{2}{3}\right)^k = \frac{8/27}{1-2/3} = \frac{8}{9}.
\]
\end{example}

We can easily generalize this example and doing so allows us to write down a more comprehensive theorem about geometric series.

\begin{theorem}
\index{series!geometric}\index{geometric series}\index{geometric series!convergence}\index{geometric series!divergence}
  The geometric series $\sum_{k= k_0}^\infty a \cdot r^k$ 
  
  \begin{itemize} 
  \item converges to $\frac{ar^{k_0}}{1-r}$ when $|r| < 1$.
  \item diverges if $|r| \geq 1$.  
  \end{itemize}
  \end{theorem}
 
\begin{example}
Let's take another look at the series that started off the section, $\sum_{k=1}^{\infty} \left(\frac{1}{2}\right)^k$.  Here, $a=\answer[given]{1}$, $r=\answer[given]{1/2}$ and $k_0 = \answer[given]{1}$.  Since $|r|<a$, the series \wordChoice{\choice[correct]{converges}\choice{diverges}}, and using the formula above, we have that

\[ \sum_{k=1}^{\infty} \left(\frac{1}{2}\right)^k = \frac{ar^{k_0}}{1-r} = \frac{1\cdot \frac{1}{2}}{1-1/2} =1.\]

This matches the earlier result!  
\end{example}

\begin{remark} 
Although we have mentioned it before, we mention it again here:

\begin{quote}
The lower index in a series does not affect whether the series converges or diverges, but if the series converges, it can affect the value to which the series converges.
\end{quote}
The formula listed above very explicitly shows exactly how the lower index $k_0$ affects the value to which a convergent geometric series converges.

\end{remark}

Now, try some questions to check your understanding of the above material.

\begin{question}
  Which of the following series converge?
  \begin{selectAll}
    \choice{$\sum_{k=0}^\infty \left(\frac{3}{2}\right)^k$}
    \choice[correct]{$\sum_{k=0}^\infty \left(\frac{-2}{3}\right)^k$}
    \choice[correct]{$\sum_{k=9}^\infty \left(\frac{1}{7}\right)^k$}
    \choice{$\sum_{k=1}^\infty (-1)^k$}
    \choice[correct]{$\sum_{k=-9}^\infty \left(\frac{1}{2}\right)^k$}    
  \end{selectAll}
  \begin{hint}
    The initial index doesn't matter as far as convergence is
    concerned, it is the ``tail'' of the sequence that determines
    convergence.
  \end{hint}
\end{question}

\begin{question}
Determine if the series $\sum_{k=2}^{\infty} 2^{3-2k}$ converges or diverges.  If it converges, give the value to which it converges.

\begin{explanation}
First, note that the series \wordChoice{\choice[correct]{is}\choice{is not}} geometric since the laws of exponents allow us to write the following.

\[
2^{3-2k} = \frac{2^3}{2^{2k}} = \frac{8}{\left(2^2\right)^k} =  8 \cdot \frac{1}{4^k} =  8 \cdot \frac{1^k}{4^k} =  8 \cdot \left(\frac{1}{4}\right)^k
\]

The series is geometric with $r = \answer[given]{1/4}$, and using the result $\sum_{k=k_0} ar^k = \frac{ar^{k_0}}{1-r}$ gives:

\[
\sum_{k=2}^{\infty} 2^{3-2k} =  8 \cdot \left(\frac{1}{4}\right)^k =  \frac{8 \cdot (1/4)^2}{1-1/4}  =  \answer[given]{\frac{2}{3}}.
\]
\end{explanation}

\end{question}    

\section{Telescoping series}
\index{series!telescoping}\index{telescoping series}

A second type of series for which we can find an explicit formula for $s_n$ are ``telescoping series''.  Rather than try to give a formal definition, we think of telescoping series are infinite sums for which the required addition required to find a formula for $s_n$ can be done so many of the intermediate terms naturally cancel.  An example will make this point more clear.

\begin{example}
  Evaluate the sum
  \[
  \sum_{k=1}^\infty\left(\frac{1}{k}-\frac{1}{k+1}\right).
  \]
  \begin{explanation}
It will help to write down the first few partial sums for this series.
\begin{image}
\begin{tikzpicture}
    \node at (0,0) {
      $\begin{aligned}
        s_1 &=	\frac11-\frac12 & & = 1-\frac12\\
        s_2 &=	\left(\frac11-\frac12\right) + \left(\frac12-\frac13\right) & & = 1-\frac13\\
        s_3 &=	\left(\frac11-\frac12\right) + \left(\frac12-\frac13\right)+\left(\frac13-\frac14\right) & &= 1-\frac14\\
        s_4 &=	\left(\frac11-\frac12\right) + \left(\frac12-\frac13\right)+\left(\frac13-\frac14\right) +\left(\frac14-\frac15\right)& &= 1-\frac15
      \end{aligned}$};
\end{tikzpicture}
\end{image}
For $s_2$ and beyond, note how most of the intermediate terms in each partial sum cancel out! In
general, we can notice from pattern recognition (specifically by looking at the denominator in each expression and comparing it to the index) that $s_n =$ \wordChoice{\choice{$1-\frac{1}{n}$},\choice[correct]{$1-\frac{1}{n+1}$}}. The sequence $\{s_n\}_{n=1}$ thus \wordChoice{\choice[correct]{converges}\choice{diverges}} since $\lim_{n \to \infty} s_n$ \wordChoice{\choice[correct]{exists}\choice{does not exist}}. Furthermore, since $\lim_{n\to\infty}s_n = \lim_{n\to\infty}\left(1-\frac1{n+1}\right) = \answer[given]1$, we conclude
that $\sum_{n=1}^\infty \left(\frac1n-\frac1{n+1}\right) = 1$.
  \end{explanation}
\end{example}

\begin{remark}
Finding the above formula required us to use pattern recognition.  Validating that the pattern must hold for \emph{all} terms in the sequence can be done formally by using an idea called \emph{mathematical induction}.  We leave it to the curious reader to explore this idea further if desired.
\end{remark}

We've just seen an example of a \dfn{telescoping series}. Informally,
a telescoping series is one in which the partial sums reduce to just a
finite sum of terms. In the last example, the partial sum $s_n$ only was the sum of two nonzero terms: 
\[
s_n = 1 - \frac{1}{n-1}.
\]

%\begin{example}
%Determine if the series $\sum_{n=1}^\infty \frac{2}{n^2+2n}$ converges or diverges.  If it converges, find the value to which it converges.
%
%\begin{explanation}
%All of the terms in the above sum are positive, so there is no convenient cancellation that will occur if we try to find a formula for $s_n$ yet.  However, we can use partial fractions to write
%  \[
%  \frac{2}{n^2+2n} = \frac{1}{n}-\frac{1}{n+2}.
%  \]  
%  Expressing the terms of $\{s_n\}$ now produces a pattern.  
%  \begin{image}
%    \begin{tikzpicture}
%      \node at (0,0) {
%        $\begin{aligned}
%          s_1 &= 1-\frac13 &&= 1-\frac13\\
%          s_2 &= \left(1-\frac13\right) + \left(\frac12-\frac14\right) &&= 1+\frac12-\frac13-\frac14\\
%          s_3 &= \left(1-\frac13\right) + \left(\frac12-\frac14\right)+\left(\frac13-\frac15\right) &&= 1+\frac12-\frac14-\frac15\\
%          s_4 &= \left(1-\frac13\right) + \left(\frac12-\frac14\right)+\left(\frac13-\frac15\right)+\left(\frac14-\frac16\right) &&= 1+\frac12-\frac15-\frac16\\
%          s_5 &= \left(1-\frac13\right) + \left(\frac12-\frac14\right)+\left(\frac13-\frac15\right)+\left(\frac14-\frac16\right)+\left(\frac15-\frac17\right) &&= 1+\frac12-\frac16-\frac17\\
%        \end{aligned}$};
%    \end{tikzpicture}
%  \end{image}
%\textbf{I WANT TO WRITE SOMETHING ABOUT THE DENOMINATORS BEING 2 APART, WHICH IS WHYIT TAKES WRITING OUT 3 TERMS UNTIL TO EXHIBIT THE EVENTUAL PATTERN.  CAN ANYONE HELP ME WORD THIS?}
%
%We again have a telescoping series. In each partial sum, most of the intermediate 
%  terms cancel and we obtain the formula
%  \[
%  s_n =1+\frac12-\frac1{n+1}-\frac1{n+2}.
%  \]
%  Taking limits allows us to determine the convergence of the series. Since
%  \[
%  \lim_{n\to\infty}s_n = \lim_{n\to\infty} \left(1+\frac12-\frac1{n+1}-\frac1{n+2}\right) = \frac32,
%  \]
%we conclude that
%  \[
%  \sum_{n=1}^\infty \frac1{n^2+2n} = \frac32.
%  \]
%\end{explanation}
%\end{example}



\begin{example}
Determine if the series $\sum_{k=1}^\infty \ln\left(\frac{k+1}{k}\right)$ converges or diverges. 
 
\begin{explanation}
We begin by writing the first few partial sums of the series:
\begin{align*}
s_1 &= \ln\left(2\right) \\
s_2 &= \ln\left(2\right)+\ln\left(\frac32\right) \\
s_3 &= \ln\left(2\right)+\ln\left(\frac32\right)+\ln\left(\frac43\right) \\
s_4 &= \ln\left(2\right)+\ln\left(\frac32\right)+\ln\left(\frac43\right)+\ln\left(\frac54\right) 
\end{align*}
At first, it doesn't look like we will have much luck writing this as a telescoping series, but noting that $ \ln\left(\frac{n+1}{n}\right) = \ln(n+1)-\ln(n)$ allows us to write out terms of $s_n$ in a more convenient way.

  \begin{image}
    \begin{tikzpicture}
      \node at (0,0) {
        $\begin{aligned}
          s_1 &= \ln(2)-\ln(1) &&= \ln(2)\\
          s_2 &= \left( \ln(3)-\ln(2)\right) + \left( \ln(2)-\ln(1)\right) &&= \ln(3)\\
          s_3 &= \left( \ln(4)-\ln(3)\right) + \left( \ln(3)-\ln(2)\right) + \left( \ln(2)-\ln(1)\right) &&= \ln(4)\\
%          s_4 &=  \left( \ln(5)-\ln(4)\right) +\left( \ln(4)-\ln(3)\right) + \left( \ln(3)-\ln(2)\right) + \left( \ln(2)-\ln(1)\right) &&= \ln(5)\\
        \end{aligned}$};
    \end{tikzpicture}
  \end{image}
  
  

%At first, this does not seem helpful, but recall the logarithmic rule
%$\ln(x)+\ln(y) = \ln (x\cdot y)$. Applying this rule to $S_4$ gives:
%\begin{align*}
%S_4 &= \ln\left(2\right)+\ln\left(\frac32\right)+\ln\left(\frac43\right)+\ln\left(\frac54\right) \\
%&= \ln\left(\frac21\cdot\frac32\cdot\frac43\cdot\frac54\right)\\
%&= \ln\left(5\right).

We can conclude that $s_n =\answer[given]{\ln (n+1)}$ and analyze $\lim_{n \to \infty} s_n$.  

Since $\lim_{n\to\infty}s_n=\answer[given]{\infty}$, $\sum_{k=1}^\infty \ln\left(\frac{k+1}{k}\right)$ diverges.
\end{explanation}
\end{example}


%%%%%%%%%%%%%%%%%%%%%%%%%%%%%%
\section{Summary}
Now that we have seen two special types of series for which we can find an explicit formula for the $n$-th term in the sequence of partial sums, it helps to summarize the logic that we employed.

\begin{itemize}
\item[1.] Consider the associated sequence $\{s_n\}$ of partial sums.
\item[2.] Try to find an explicit formula for the term $s_n$.  If you can find such a formula, analyze $\lim_{n \to \infty s_n}$.  
\begin{itemize}
\item If the limit exists, $\sum_{k=k_0}^{\infty} a_k$ converges, and if we can determine that $\lim_{n \to \infty} s_n =L$, then $\sum_{k=k_0} a_k=L$.  \item If  $\lim_{n \to \infty} s_n$ does not exist, then $\sum_{k=k_0} a_k$ diverges.
\end{itemize}
\end{itemize}


\end{document}
