\documentclass{ximera}
\author{Jim Fowler}
\title{Start Here}
\begin{document}

\begin{abstract}
  This is your practice exam.
\end{abstract}

\maketitle

Here are some problems which may give you an idea of the sorts of
things that could be discussed on the upcoming exam.  \textbf{Any
  topic from the course is fair-game on an in-class examination} but
the following problems hopefully give you an idea of the sorts of
tasks you may experience during the exam.  During the exam, you
will also be asked to do other sorts of tasks on the exam as well
(such as problems involving the partial derivatives and the gradient,
which aren't represented among these practice problems).

The best way to study is to not only answer the questions, but also to
\textit{question the answers} (a lovely turn of phrase attributed to
Glenn Stevens).  When you ``solve'' a problem, try to invent a similar
problem of your own.  Can you solve your problem?  Challenge your
friends with the problems you invent.

What's a ``hard'' problem that you can imagine on an exam?
\begin{freeResponse}
\end{freeResponse}

What's an ``easy'' problem you can imagine?
\begin{freeResponse}
\end{freeResponse}

By cooking up your own problems, you are exploring the ``space of
possible exam problems'' which prepares you to think about what is
likely (and unlikely!) to appear on a exam.

Good luck in your studies---and beyond!

\end{document}
