\documentclass{ximera}
%\usepackage{todonotes}
%\usepackage{mathtools} %% Required for wide table Curl and Greens
%\usepackage{cuted} %% Required for wide table Curl and Greens
\newcommand{\todo}{}

\usepackage{esint} % for \oiint
\ifxake%%https://math.meta.stackexchange.com/questions/9973/how-do-you-render-a-closed-surface-double-integral
\renewcommand{\oiint}{{\large\bigcirc}\kern-1.56em\iint}
\fi


\graphicspath{
  {./}
  {ximeraTutorial/}
  {basicPhilosophy/}
  {functionsOfSeveralVariables/}
  {normalVectors/}
  {lagrangeMultipliers/}
  {vectorFields/}
  {greensTheorem/}
  {shapeOfThingsToCome/}
  {dotProducts/}
  {partialDerivativesAndTheGradientVector/}
  {../productAndQuotientRules/exercises/}
  {../normalVectors/exercisesParametricPlots/}
  {../continuityOfFunctionsOfSeveralVariables/exercises/}
  {../partialDerivativesAndTheGradientVector/exercises/}
  {../directionalDerivativeAndChainRule/exercises/}
  {../commonCoordinates/exercisesCylindricalCoordinates/}
  {../commonCoordinates/exercisesSphericalCoordinates/}
  {../greensTheorem/exercisesCurlAndLineIntegrals/}
  {../greensTheorem/exercisesDivergenceAndLineIntegrals/}
  {../shapeOfThingsToCome/exercisesDivergenceTheorem/}
  {../greensTheorem/}
  {../shapeOfThingsToCome/}
  {../separableDifferentialEquations/exercises/}
  {vectorFields/}
}

\newcommand{\mooculus}{\textsf{\textbf{MOOC}\textnormal{\textsf{ULUS}}}}

\usepackage{tkz-euclide}\usepackage{tikz}
\usepackage{tikz-cd}
\usetikzlibrary{arrows}
\tikzset{>=stealth,commutative diagrams/.cd,
  arrow style=tikz,diagrams={>=stealth}} %% cool arrow head
\tikzset{shorten <>/.style={ shorten >=#1, shorten <=#1 } } %% allows shorter vectors

\usetikzlibrary{backgrounds} %% for boxes around graphs
\usetikzlibrary{shapes,positioning}  %% Clouds and stars
\usetikzlibrary{matrix} %% for matrix
\usepgfplotslibrary{polar} %% for polar plots
\usepgfplotslibrary{fillbetween} %% to shade area between curves in TikZ
\usetkzobj{all}
\usepackage[makeroom]{cancel} %% for strike outs
%\usepackage{mathtools} %% for pretty underbrace % Breaks Ximera
%\usepackage{multicol}
\usepackage{pgffor} %% required for integral for loops



%% http://tex.stackexchange.com/questions/66490/drawing-a-tikz-arc-specifying-the-center
%% Draws beach ball
\tikzset{pics/carc/.style args={#1:#2:#3}{code={\draw[pic actions] (#1:#3) arc(#1:#2:#3);}}}



\usepackage{array}
\setlength{\extrarowheight}{+.1cm}
\newdimen\digitwidth
\settowidth\digitwidth{9}
\def\divrule#1#2{
\noalign{\moveright#1\digitwidth
\vbox{\hrule width#2\digitwidth}}}





\newcommand{\RR}{\mathbb R}
\newcommand{\R}{\mathbb R}
\newcommand{\N}{\mathbb N}
\newcommand{\Z}{\mathbb Z}

\newcommand{\sagemath}{\textsf{SageMath}}


%\renewcommand{\d}{\,d\!}
\renewcommand{\d}{\mathop{}\!d}
\newcommand{\dd}[2][]{\frac{\d #1}{\d #2}}
\newcommand{\pp}[2][]{\frac{\partial #1}{\partial #2}}
\renewcommand{\l}{\ell}
\newcommand{\ddx}{\frac{d}{\d x}}

\newcommand{\zeroOverZero}{\ensuremath{\boldsymbol{\tfrac{0}{0}}}}
\newcommand{\inftyOverInfty}{\ensuremath{\boldsymbol{\tfrac{\infty}{\infty}}}}
\newcommand{\zeroOverInfty}{\ensuremath{\boldsymbol{\tfrac{0}{\infty}}}}
\newcommand{\zeroTimesInfty}{\ensuremath{\small\boldsymbol{0\cdot \infty}}}
\newcommand{\inftyMinusInfty}{\ensuremath{\small\boldsymbol{\infty - \infty}}}
\newcommand{\oneToInfty}{\ensuremath{\boldsymbol{1^\infty}}}
\newcommand{\zeroToZero}{\ensuremath{\boldsymbol{0^0}}}
\newcommand{\inftyToZero}{\ensuremath{\boldsymbol{\infty^0}}}



\newcommand{\numOverZero}{\ensuremath{\boldsymbol{\tfrac{\#}{0}}}}
\newcommand{\dfn}{\textbf}
%\newcommand{\unit}{\,\mathrm}
\newcommand{\unit}{\mathop{}\!\mathrm}
\newcommand{\eval}[1]{\bigg[ #1 \bigg]}
\newcommand{\seq}[1]{\left( #1 \right)}
\renewcommand{\epsilon}{\varepsilon}
\renewcommand{\phi}{\varphi}


\renewcommand{\iff}{\Leftrightarrow}

\DeclareMathOperator{\arccot}{arccot}
\DeclareMathOperator{\arcsec}{arcsec}
\DeclareMathOperator{\arccsc}{arccsc}
\DeclareMathOperator{\si}{Si}
\DeclareMathOperator{\scal}{scal}
\DeclareMathOperator{\sign}{sign}


%% \newcommand{\tightoverset}[2]{% for arrow vec
%%   \mathop{#2}\limits^{\vbox to -.5ex{\kern-0.75ex\hbox{$#1$}\vss}}}
\newcommand{\arrowvec}[1]{{\overset{\rightharpoonup}{#1}}}
%\renewcommand{\vec}[1]{\arrowvec{\mathbf{#1}}}
\renewcommand{\vec}[1]{{\overset{\boldsymbol{\rightharpoonup}}{\mathbf{#1}}}\hspace{0in}}

\newcommand{\point}[1]{\left(#1\right)} %this allows \vector{ to be changed to \vector{ with a quick find and replace
\newcommand{\pt}[1]{\mathbf{#1}} %this allows \vec{ to be changed to \vec{ with a quick find and replace
\newcommand{\Lim}[2]{\lim_{\point{#1} \to \point{#2}}} %Bart, I changed this to point since I want to use it.  It runs through both of the exercise and exerciseE files in limits section, which is why it was in each document to start with.

\DeclareMathOperator{\proj}{\mathbf{proj}}
\newcommand{\veci}{{\boldsymbol{\hat{\imath}}}}
\newcommand{\vecj}{{\boldsymbol{\hat{\jmath}}}}
\newcommand{\veck}{{\boldsymbol{\hat{k}}}}
\newcommand{\vecl}{\vec{\boldsymbol{\l}}}
\newcommand{\uvec}[1]{\mathbf{\hat{#1}}}
\newcommand{\utan}{\mathbf{\hat{t}}}
\newcommand{\unormal}{\mathbf{\hat{n}}}
\newcommand{\ubinormal}{\mathbf{\hat{b}}}

\newcommand{\dotp}{\bullet}
\newcommand{\cross}{\boldsymbol\times}
\newcommand{\grad}{\boldsymbol\nabla}
\newcommand{\divergence}{\grad\dotp}
\newcommand{\curl}{\grad\cross}
%\DeclareMathOperator{\divergence}{divergence}
%\DeclareMathOperator{\curl}[1]{\grad\cross #1}
\newcommand{\lto}{\mathop{\longrightarrow\,}\limits}

\renewcommand{\bar}{\overline}

\colorlet{textColor}{black}
\colorlet{background}{white}
\colorlet{penColor}{blue!50!black} % Color of a curve in a plot
\colorlet{penColor2}{red!50!black}% Color of a curve in a plot
\colorlet{penColor3}{red!50!blue} % Color of a curve in a plot
\colorlet{penColor4}{green!50!black} % Color of a curve in a plot
\colorlet{penColor5}{orange!80!black} % Color of a curve in a plot
\colorlet{penColor6}{yellow!70!black} % Color of a curve in a plot
\colorlet{fill1}{penColor!20} % Color of fill in a plot
\colorlet{fill2}{penColor2!20} % Color of fill in a plot
\colorlet{fillp}{fill1} % Color of positive area
\colorlet{filln}{penColor2!20} % Color of negative area
\colorlet{fill3}{penColor3!20} % Fill
\colorlet{fill4}{penColor4!20} % Fill
\colorlet{fill5}{penColor5!20} % Fill
\colorlet{gridColor}{gray!50} % Color of grid in a plot

\newcommand{\surfaceColor}{violet}
\newcommand{\surfaceColorTwo}{redyellow}
\newcommand{\sliceColor}{greenyellow}




\pgfmathdeclarefunction{gauss}{2}{% gives gaussian
  \pgfmathparse{1/(#2*sqrt(2*pi))*exp(-((x-#1)^2)/(2*#2^2))}%
}


%%%%%%%%%%%%%
%% Vectors
%%%%%%%%%%%%%

%% Simple horiz vectors
\renewcommand{\vector}[1]{\left\langle #1\right\rangle}


%% %% Complex Horiz Vectors with angle brackets
%% \makeatletter
%% \renewcommand{\vector}[2][ , ]{\left\langle%
%%   \def\nextitem{\def\nextitem{#1}}%
%%   \@for \el:=#2\do{\nextitem\el}\right\rangle%
%% }
%% \makeatother

%% %% Vertical Vectors
%% \def\vector#1{\begin{bmatrix}\vecListA#1,,\end{bmatrix}}
%% \def\vecListA#1,{\if,#1,\else #1\cr \expandafter \vecListA \fi}

%%%%%%%%%%%%%
%% End of vectors
%%%%%%%%%%%%%

%\newcommand{\fullwidth}{}
%\newcommand{\normalwidth}{}



%% makes a snazzy t-chart for evaluating functions
%\newenvironment{tchart}{\rowcolors{2}{}{background!90!textColor}\array}{\endarray}

%%This is to help with formatting on future title pages.
\newenvironment{sectionOutcomes}{}{}



%% Flowchart stuff
%\tikzstyle{startstop} = [rectangle, rounded corners, minimum width=3cm, minimum height=1cm,text centered, draw=black]
%\tikzstyle{question} = [rectangle, minimum width=3cm, minimum height=1cm, text centered, draw=black]
%\tikzstyle{decision} = [trapezium, trapezium left angle=70, trapezium right angle=110, minimum width=3cm, minimum height=1cm, text centered, draw=black]
%\tikzstyle{question} = [rectangle, rounded corners, minimum width=3cm, minimum height=1cm,text centered, draw=black]
%\tikzstyle{process} = [rectangle, minimum width=3cm, minimum height=1cm, text centered, draw=black]
%\tikzstyle{decision} = [trapezium, trapezium left angle=70, trapezium right angle=110, minimum width=3cm, minimum height=1cm, text centered, draw=black]

\author{Tom Dinitz and Nela Lakos}
\license{Creative Commons 3.0 By-NC}
\title{Exam One Review}

\begin{document}
\begin{abstract}
Review questions for exam 1.
\end{abstract}
\maketitle
%Exercise 0

\begin{exercise}
Find the following values or, if the value is not defined, say `DNE':
\[
\begin{array}{l l l}
(a) \sin\left(\frac{17\pi}{6}\right)=\answer{\frac{1}{2}} & (b) \sin^{-1}\left(\frac{17\pi}{6}\right)=\answer[format=string]{DNE} & (c) \sin^{-1}\left(\frac{-\sqrt{3}}{2}\right)=\answer{-\frac{\pi}{3}} \\
(d) e^{-2\ln(3)}=\answer{3^{-2}} & (e) \ln\left(-e^2\right)=\answer[format=string]{DNE} & (f) \ln\left(e^{21}\right)=\answer{21} \\
(g) 10^{3\log_{10}(4)}=\answer{4^3} & (h) \ln\left(1\right)=\answer{0} & (i) \tan^{-1}\left(-\sqrt{3}\right) =\answer{-\frac{\pi}{3}} \\
(j) \cot\left(\frac{\pi}{3}\right)=\answer{\frac{1}{\sqrt{3}}} & (k) \cot^{-1}\left(-\sqrt{3}\right)=\answer{-\frac{\pi}{6}} & 
\end{array}
\]
\end{exercise}

%Exercise 1
\begin{exercise}
Consider the functions: $g(x)=-5|x-1|$, and $h(x)=(x-1)^2$.\\

(i) Find the limit. (Possible answers include $+\infty, -\infty$ or `DNE')

(a) $\lim_{x\to 3} \frac{g(x)-g(3)}{x-3}=\begin{prompt}{\answer{-5}}\end{prompt}$

(b) $\lim_{x\to 1^-} \frac{g(x)}{h(x)}=\begin{prompt} \answer{-\infty}\end{prompt}$

(ii) Let $f$ be a function such that $g(x)\leq f(x)\leq h(x), 0<x<2$.

(a) Find the limit or say `DNE': $\lim_{x\to 1} f(x)= \begin{prompt}\answer{0}\end{prompt}$
\end{exercise}

%Exercise 2
\begin{exercise}
  Evaluate the following limits. Note: You may not use a table of values, a graph,
  or L'Hospital's Rule to justify your answer.
\[
\begin{array}{l l}
  (a) \lim_{x\to 3} \frac{\sqrt{2x-2}-2}{x-3}=\begin{prompt}\answer{.5}\end{prompt}
	&
  (b) \lim_{x\to -\infty} \frac{x^6-2x^5-2}{x^5-3}=\begin{prompt}\answer{-\infty}\end{prompt} \\
  (c) \lim_{x\to 0^+} x\cos(\frac{\pi}{x})=\begin{prompt}\answer{0}\end{prompt} &
  (d) \lim_{x\to 0} \frac{(x^2+3)^2-9}{x^2}=\begin{prompt}\answer{6}\end{prompt}\\
  (e) \lim_{x\to -\infty} \frac{\sqrt{9x^2+1}}{6x-5}=\begin{prompt}\answer{-.5}\end{prompt} &
  (f) \lim_{x\to 7^-} \frac{|x-7|}{x-7}=\begin{prompt}\answer{-1}\end{prompt}\\
  (g) \lim_{x\to \infty} \frac{6e^x+e^{-x}}{3e^x+4e^{-x}}=\begin{prompt}\answer{2}\end{prompt}&
  (h) \lim_{x\to 4^-} \frac{\sin(\frac{\pi}{3}x)}{\ln(5-x)}=\begin{prompt}\answer{-\infty}\end{prompt}\\
  (i) \lim_{x\to 0} \frac{\ln(\cos(\sqrt{3x^4+x^2}))}{\sin(x)+\cos(x)}=\begin{prompt}\answer{0}\end{prompt}
&
  (j) \lim_{x\to 0^-} \frac{e^{-x}}{x^2+4x}=\begin{prompt}\answer{-\infty}\end{prompt}
\end{array}
\]
\end{exercise}

%Exercise 3
\begin{exercise}
  Let
  \[
  f(x) =
  \begin{cases}
    \frac{x^2-x-12}{x+3} &\text{if $x<4$ and $x\ne -3$}\\
    5 &\text{if $x=-3$}\\
    \frac{x}{x-4} &\text{if $x>4$}
  \end{cases}
  \]
  (i) Determine if the following limits exist. If they do not, say `DNE'. Note: You may not use a table of values, a graph, or L'Hospital's Rule to justify your answer.

  (a) $\lim_{x\to -3} f(x)=\begin{prompt}\answer{-7}\end{prompt}$

  (b) $\lim_{x\to 4^-} f(x)=\begin{prompt}\answer{0}\end{prompt}$

  (c) $\lim_{x\to 4^+} f(x)=\begin{prompt}\answer{\infty}\end{prompt}$

  (d) $\lim_{x\to 4} f(x)=\begin{prompt}\answer[format=string]{DNE}\end{prompt}$

  (ii) Find all vertical asymptotes of f: $x=\begin{prompt}\answer{4}\end{prompt}$

  (iii) Find all horizontal asymptotes of f: $y=\begin{prompt}\answer{1}\end{prompt}$

  (iv) List the (largest) intervals of continuity of f: $\begin{prompt}\left(\answer{-\infty},\answer{-3}\right),\left(\answer{-3},\answer{4}\right),\left(\answer{4},\answer{\infty}\right)\end{prompt}$
\end{exercise}

%Exercise 4
\begin{exercise}
Fill in the blanks to explain how the Intermediate Value Theorem can be used to show that the equation $x^3+4x+2=0$ has a solution on the interval $[-1,0]$.

Let $f(x)=x^3+4x+2$. Since $f$ is a \wordChoice{\choice{monotone}\choice[correct]{continuous}}
 function for all $x$ in the interval \wordChoice{\choice[correct]{$[-1,0]$}\choice{$(-1,0)$}},
and \wordChoice{\choice[correct]{0}\choice{1}} is between
 \wordChoice{\choice{-1}\choice[correct]{f(-1)=-3}} and
 \wordChoice{\choice{0}\choice[correct]{f(0)=2}}, then the IVT guarantees the existence of at least one number
 \wordChoice{\choice[correct]{u}\choice{f(u)}} in $[-1, 0]$ such that
 \wordChoice{\choice[correct]{f(u) = 0}\choice{f(0) = u}}.
\end{exercise}

%Exercise 5

\begin{exercise}

  (i) Evaluate the following expressions:

  (a) $\sin^{-1}(\sin(\frac{4\pi}{9}))=\begin{prompt}\answer{\frac{4\pi}{9}}\end{prompt}$

  (b) $\sin^{-1}(\sin(\frac{4\pi}{5}))=\begin{prompt}\answer{\frac{\pi}{5}}\end{prompt}$

  (ii) Determine if the following statements are true or false:

  (a) Given a one-to-one function $f$ and its inverse $f^{-1}$, $f^{-1}(f(x))=x$, where x is in the domain of $f$.
\begin{prompt}
\begin{multipleChoice}
\choice[correct]{True}
\choice{False}
\end{multipleChoice}
\end{prompt}

  (b) $\sin^{-1}(\sin(\frac{2\pi}{3}))=\frac{2\pi}{3}$.
\begin{prompt}
\begin{multipleChoice}
\choice{True}
\choice[correct]{False}
\end{multipleChoice}
\end{prompt}

  (c) Given $f(x)=\frac{1}{x}, f^{-1}(x)=\frac{1}{x}$.
\begin{prompt}
\begin{multipleChoice}
\choice[correct]{True}
\choice{False}
\end{multipleChoice}
\end{prompt}

  (iii) Find the inverse of $f(x)=\sqrt[3]{x-2}+4$. $f^{-1}(x)=\begin{prompt}\answer{x^3-12x^2+48x-62}\end{prompt}$

  (iv) Use a right triangle to simplify $\tan(\cos^{-1}(x))$. $\begin{prompt}\answer{\frac{\sqrt{1-x^2}}{x}}\end{prompt}$
\end{exercise}

%Exercise 6
\begin{exercise}
A table of values for $f(x)$, along with a graph of a function $g(x)$ is shown below.
\[
\begin{array}{c|c}
x & f(x)\\ \hline
1 & 2 \\ \hline
2 & 3 \\ \hline
3 & 4 \\ \hline
\end{array}
\]
\begin{image}
\begin{tikzpicture}
    \begin{axis}[
            domain=0:6,
            ymax=5,
	    ymin=-1,
            samples=100,
            axis lines =middle, xlabel=$x$, ylabel=$y$,
	    ytick={1,2,3,4},
            every axis y label/.style={at=(current axis.above origin),anchor=south},
            every axis x label/.style={at=(current axis.right of origin),anchor=west}
          ]
          \addplot [very thick, penColor, smooth, domain=(0:1)] {3*x};
          \addplot [very thick, penColor, smooth, domain=(1:4)] {4-x};
	  \addplot [very thick, penColor, smooth, domain=(4:6)] {(x-4)^2};
        \end{axis}
\end{tikzpicture}
\end{image}

\begin{enumerate}
\item Find expressions for $g(x)$ on the following intervals:
\begin{enumerate}
\item For $0<x<1$, $g(x)=\begin{prompt}\answer{3x}\end{prompt}$
\item For $1<x<4$, $g(x)=\begin{prompt}\answer{4-x}\end{prompt}$
\end{enumerate}
\item Find the values or write DNE.
\[
\begin{array}{ l l l}
\text{ at } x=1,  f(x)g(x)= \begin{prompt}\answer{6}\end{prompt} &
\text{ at } x=3, f(g(x))= \begin{prompt}\answer{2}\end{prompt} \\
\text{ at } x=2, g(f(x))= \begin{prompt}\answer{1}\end{prompt} &
\text{ at } x=1.5, g(x)= \begin{prompt}\answer{2.5}\end{prompt} \\
\lim_{x\to 1} g(x)=\begin{prompt}\answer{3}\end{prompt} &
\lim_{x\to 1^-} \frac{g(x)-g(1)}{x-1}=\begin{prompt}\answer{3}\end{prompt}\\
\lim_{x\to 1^+} \frac{g(x)-g(1)}{x-1}=\begin{prompt}\answer{-1}\end{prompt} &
\lim_{x\to 1} \frac{g(x)-g(1)}{x-1}=\begin{prompt}\answer[format=string]{DNE}\end{prompt}
\end{array}
\]
\end{enumerate}
\end{exercise}

%Exercise 7

\begin{exercise}

The (entire) graph of a function $f$ is given in the figure below.

\begin{image}

\begin{tikzpicture}
    \begin{axis}[
            xmin=-10, xmax=10, ymin=-6,ymax=6,
            unit vector ratio*=1 1 1,
            axis lines =middle, xlabel=$x$, ylabel=$y$,
            every axis y label/.style={at=(current axis.above origin),anchor=south},
            every axis x label/.style={at=(current axis.right of origin),anchor=west},
            xtick={-10,...,10}, ytick={-6,...,6},
            grid=major,width=4in,
            grid style={dashed, gridColor},
          ]
          \addplot[color=penColor,very thick] plot coordinates
                  {(-8,0) (0,-4)};
	  \addplot[color=penColor,very thick] plot coordinates
		  {(3,2) (8,0)};
	  \addplot[very thick, color=penColor, smooth, domain=(0:3)] {6-(2*x/3)^2};
          \addplot[color=penColor,fill=penColor,only marks,mark=*] coordinates{(-8,2)};  %% closed hole
          \addplot[color=penColor,fill=background,only marks,mark=*] coordinates{(-8,0)};  %% open hole
          \addplot[color=penColor,fill=background,only marks,mark=*] coordinates{(0,-4)};  %% open hole
  	  \addplot[color=penColor,fill=background,only marks,mark=*] coordinates{(0,6)};
	  \addplot[color=penColor,fill=background,only marks,mark=*] coordinates{(3,2)};
	  \addplot[color=penColor,fill=penColor,only marks,mark=*] coordinates{(3,0)};
	  \addplot[color=penColor,fill=background,only marks,mark=*] coordinates{(8,0)};
        \end{axis}
\end{tikzpicture}
\end{image}


(i) Find the domain and range of $f$. Write your answer in interval notation.

Domain of $f$: \begin{prompt}$\Big[\answer{-8},\answer{0}\Big)\cup \Big(\answer{0},\answer{8}\Big)$\end{prompt}

Range of $f$: \begin{prompt}$\Big(\answer{-4},\answer{6}\Big)$\end{prompt}

(ii) List the largest intervals of continuity for $f$: \begin{prompt}$\Big(\answer{-8},\answer{0}\Big)$ and $\Big(\answer{0},\answer{3}\Big)$ and $\Big(\answer{3},\answer{8}\Big)$\end{prompt}
\end{exercise}

\begin{exercise}
(iii) Determine if there are any points $a$ in the interval $[-8,8]$ for which the following statements are true. If there are any such points, find all of them.

(a) $\lim_{x\to a} f(x)$ exists, but the function $f$ is NOT continuous at $a$.
\begin{prompt}
\begin{multipleChoice}
\choice{There are no points}
\choice[correct]{There is at least one point}
\end{multipleChoice}

\begin{exercise}
$a=\answer{3}$
\end{exercise}
\end{prompt}
\end{exercise}

\begin{exercise}
(b) Both limits $\lim_{x\to a^+} f(x)$ and $\lim_{x\to a^-}$ exist, but the limit $\lim\limits_{x\to a} f(x)$ does not exist.

\begin{prompt}
\begin{multipleChoice}
\choice{There are no points}
\choice[correct]{There is at least one point}
\end{multipleChoice}

\begin{exercise}
$a=\answer{0}$
\end{exercise}
\end{prompt}
\end{exercise}

\begin{exercise}
(c) $\lim_{x\to a} f(x)=-2$.
\begin{prompt}
\begin{multipleChoice}
\choice{There are no points}
\choice[correct]{There is at least one point}
\end{multipleChoice}

\begin{exercise}
$a=\answer{-4}$
\end{exercise}
\end{prompt}
\end{exercise}

\begin{exercise}
(d) $\lim_{x\to a^+} f(x)=0$.
\begin{prompt}
\begin{multipleChoice}
\choice{There are no points}
\choice[correct]{There is at least one point}
\end{multipleChoice}

\begin{exercise}
$a=\answer{-8}$
\end{exercise}
\end{prompt}
\end{exercise}

\begin{exercise}
(e) $\lim_{x\to a^-} f(x)=0$.
\begin{prompt}
\begin{multipleChoice}
\choice{There are no points}
\choice[correct]{There is at least one point}
\end{multipleChoice}

\begin{exercise}
$a=\answer{8}$
\end{exercise}
\end{prompt}
\end{exercise}

\begin{exercise}

(iv) Find the following value or say `DNE'
\begin{enumerate}
\item $f(0)=\begin{prompt}\answer[format=string]{DNE}\end{prompt}$

\item $f(3)=\begin{prompt}\answer{0}\end{prompt}$

\item  $f^{-1}(0)=\begin{prompt}\answer{3}\end{prompt}$

\item $f^{-1}(2)=\begin{prompt}\answer{-8}\end{prompt}$

\item $f^{-1}(-4)=\begin{prompt}\answer[format=string]{DNE}\end{prompt}$

\item $f^{-1}(-2)=\begin{prompt}\answer{-4}\end{prompt}$
\end{enumerate}
\end{exercise}

%Exercise 8

\begin{exercise}

The (entire) graph of a function $f$ is given in the figure below.

\begin{image}
\begin{tikzpicture}
    \begin{axis}[
            xmin=-6, xmax=10, ymin=-6,ymax=7,
            unit vector ratio*=1 1 1,
            axis lines =middle, xlabel=$x$, ylabel=$y$,
            every axis y label/.style={at=(current axis.above origin),anchor=south},
            every axis x label/.style={at=(current axis.right of origin),anchor=west},
            xtick={-6,...,10}, ytick={-6,...,7},
            grid=major,width=4in,
            grid style={dashed, gridColor},
          ]
          \addplot[color=penColor,very thick] plot coordinates
                  {(-4,6) (0,2)};
	  \addplot[very thick, color=penColor, smooth, domain=(0:3)] {2-(x*(2)^(.5)/3)^2};
          \addplot[very thick, color=penColor, smooth, domain=(3:7)] {(x-7)^2/8-4};
          \addplot[color=penColor,fill=penColor,only marks,mark=*] coordinates{(-4,6)};  %% closed hole
          \addplot[color=penColor,fill=background,only marks,mark=*] coordinates{(0,2)};  %% open hole
          \addplot[color=penColor,fill=background,only marks,mark=*] coordinates{(3,0)};  %% open hole
	  \addplot[color=penColor,fill=penColor,only marks,mark=*] coordinates{(3,-2)};
	  \addplot[color=penColor,fill=penColor,only marks,mark=*] coordinates{(7,-4)};
        \end{axis}
\end{tikzpicture}
\end{image}

(a) Find the domain and range of $f$

Domain: $\begin{prompt}
\Big[\answer{-4},\answer{0}\Big) \cup \Big(\answer{0},\answer{7}\Big]
\end{prompt}$

Range: $\begin{prompt}
\Big[\answer{-4},\answer{-2}\Big] \cup \Big(\answer{0},\answer{2}\Big) \cup \Big(\answer{2},\answer{6}\Big]
\end{prompt}$

(b) Which of the following represents the graph of  $f=\frac{f(x-2)}{2}+4$
\begin{multipleChoice}
\choice{\begin{image}
\begin{tikzpicture}
    \begin{axis}[
            xmin=-6, xmax=10, ymin=-6,ymax=7,
            unit vector ratio*=1 1 1,
            axis lines =middle, xlabel=$x$, ylabel=$y$,
            every axis y label/.style={at=(current axis.above origin),anchor=south},
            every axis x label/.style={at=(current axis.right of origin),anchor=west},
            xtick={-6,...,10}, ytick={-6,...,7},
            grid=major,width=4in,
            grid style={dashed, gridColor},
          ]
          \addplot[color=penColor,very thick] plot coordinates
                  {(-6,7) (-2,5)};
	  \addplot[very thick, color=penColor, smooth, domain=(-2:1)] {5-(x+2)^2/9};
          \addplot[very thick, color=penColor, smooth, domain=(1:5)] {(x-5)^2/16+2};
          \addplot[color=penColor,fill=penColor,only marks,mark=*] coordinates{(-6,7)};  %% closed hole
          \addplot[color=penColor,fill=background,only marks,mark=*] coordinates{(-2,5)};  %% open hole
          \addplot[color=penColor,fill=background,only marks,mark=*] coordinates{(1,4)};  %% open hole
	  \addplot[color=penColor,fill=penColor,only marks,mark=*] coordinates{(1,3)};
	  \addplot[color=penColor,fill=penColor,only marks,mark=*] coordinates{(5,2)};
        \end{axis}
\end{tikzpicture}
\end{image}
}

\choice[correct]{\begin{image}
\begin{tikzpicture}
    \begin{axis}[
            xmin=-6, xmax=10, ymin=-6,ymax=7,
            unit vector ratio*=1 1 1,
            axis lines =middle, xlabel=$x$, ylabel=$y$,
            every axis y label/.style={at=(current axis.above origin),anchor=south},
            every axis x label/.style={at=(current axis.right of origin),anchor=west},
            xtick={-6,...,10}, ytick={-6,...,7},
            grid=major,width=4in,
            grid style={dashed, gridColor},
          ]
          \addplot[color=penColor,very thick] plot coordinates
                  {(-2,7) (2,5)};
	  \addplot[very thick, color=penColor, smooth, domain=(2:5)] {5-(x-2)^2/9};
          \addplot[very thick, color=penColor, smooth, domain=(5:9)] {(x-9)^2/16+2};
          \addplot[color=penColor,fill=penColor,only marks,mark=*] coordinates{(-2,7)};  %% closed hole
          \addplot[color=penColor,fill=background,only marks,mark=*] coordinates{(2,5)};  %% open hole
          \addplot[color=penColor,fill=background,only marks,mark=*] coordinates{(5,4)};  %% open hole
	  \addplot[color=penColor,fill=penColor,only marks,mark=*] coordinates{(5,3)};
	  \addplot[color=penColor,fill=penColor,only marks,mark=*] coordinates{(9,2)};
        \end{axis}
\end{tikzpicture}
\end{image}
}

\choice{\begin{image}
\begin{tikzpicture}
    \begin{axis}[
            xmin=-6, xmax=10, ymin=-6,ymax=7,
            unit vector ratio*=1 1 1,
            axis lines =middle, xlabel=$x$, ylabel=$y$,
            every axis y label/.style={at=(current axis.above origin),anchor=south},
            every axis x label/.style={at=(current axis.right of origin),anchor=west},
            xtick={-6,...,10}, ytick={-6,...,7},
            grid=major,width=4in,
            grid style={dashed, gridColor},
          ]
          \addplot[color=penColor,very thick] plot coordinates
                  {(-2,-1) (2,-3)};
	  \addplot[very thick, color=penColor, smooth, domain=(2:5)] {-3-(x-2)^2/9};
          \addplot[very thick, color=penColor, smooth, domain=(5:9)] {(x-9)^2/16-6};
          \addplot[color=penColor,fill=penColor,only marks,mark=*] coordinates{(-2,-1)};  %% closed hole
          \addplot[color=penColor,fill=background,only marks,mark=*] coordinates{(2,-3)};  %% open hole
          \addplot[color=penColor,fill=background,only marks,mark=*] coordinates{(5,-4)};  %% open hole
	  \addplot[color=penColor,fill=penColor,only marks,mark=*] coordinates{(5,-5)};
	  \addplot[color=penColor,fill=penColor,only marks,mark=*] coordinates{(9,-6)};
        \end{axis}
\end{tikzpicture}
\end{image}
}

\choice{\begin{image}
\begin{tikzpicture}
    \begin{axis}[
            xmin=-6, xmax=10, ymin=-6,ymax=7,
            unit vector ratio*=1 1 1,
            axis lines =middle, xlabel=$x$, ylabel=$y$,
            every axis y label/.style={at=(current axis.above origin),anchor=south},
            every axis x label/.style={at=(current axis.right of origin),anchor=west},
            xtick={-6,...,10}, ytick={-6,...,7},
            grid=major,width=4in,
            grid style={dashed, gridColor},
          ]
          \addplot[color=penColor,very thick] plot coordinates
                  {(-6,-1) (-2,-3)};
	  \addplot[very thick, color=penColor, smooth, domain=(-2:1)] {-3-(x+2)^2/9};
          \addplot[very thick, color=penColor, smooth, domain=(1:5)] {(x-5)^2/16-6};
          \addplot[color=penColor,fill=penColor,only marks,mark=*] coordinates{(-6,-1)};  %% closed hole
          \addplot[color=penColor,fill=background,only marks,mark=*] coordinates{(-2,-3)};  %% open hole
          \addplot[color=penColor,fill=background,only marks,mark=*] coordinates{(1,-4)};  %% open hole
	  \addplot[color=penColor,fill=penColor,only marks,mark=*] coordinates{(1,-5)};
	  \addplot[color=penColor,fill=penColor,only marks,mark=*] coordinates{(5,-6)};
        \end{axis}
\end{tikzpicture}
\end{image}
}

\end{multipleChoice}

(c) Find the following limits or say that a limit does not exist (DNE).

(i) $\lim_{x\to 0} f(x)=\begin{prompt}\answer{2}\end{prompt}$

(ii) $\lim_{x\to 3^-} f(x)=\begin{prompt}\answer{0}\end{prompt}$

(iii) $\lim_{x\to 3^+} f(x)=\begin{prompt}\answer{-2}\end{prompt}$

(iv) $\lim_{x\to 3} f(x)=\begin{prompt}\answer{DNE}\end{prompt}$

(d) List all the intervals of continuity:
$\begin{prompt}
\Big[\answer{-4},\answer{0}\Big)\text{ and } \Big(\answer{0},\answer{3}\Big)\text{ and } \Big[\answer{3},\answer{7}\Big]
\end{prompt}$

(e) Find the following values or expressions, or say `DNE'.

(i) $f(0)=\begin{prompt}\answer{DNE}\end{prompt}$

(ii) For $-4<x<0$, $f(x)=\begin{prompt}\answer{2-x}\end{prompt}$

(f) Find the domain of $f^{-1}(x)$:
$\begin{prompt}
\Big[\answer{-4},\answer{-2}\Big] \cup \Big(\answer{0},\answer{2}\Big)\text{ and } \Big(\answer{2},\answer{6}\Big]
\end{prompt}$

(g) Find the expression for $f^{-1}(x)$, for $2<x<6$. $f^{-1}(x)=\begin{prompt}\answer{2-x}\end{prompt}$
\end{exercise}

%Exercise 9
\begin{exercise}
  A function $f$ is an even function, defined on $[-6,6]$.  Part of the graph of $f$ is shown below
  \begin{image}
  \begin{tikzpicture}
      \begin{axis}[
              xmin=-6, xmax=8, ymin=-6,ymax=6,
              unit vector ratio*=1 1 1,
              axis lines =middle, xlabel=$x$, ylabel=$y$,
              every axis y label/.style={at=(current axis.above origin),anchor=south},
              every axis x label/.style={at=(current axis.right of origin),anchor=west},
              xtick={-6,...,8}, ytick={-6,...,8},
              grid=major,width=4in,
              grid style={dashed, gridColor},
            ]
  	        \addplot[very thick, color=penColor, smooth, domain=(0:3)] {2-(x*(2)^(.5)/3)^2};
            \addplot[very thick, color=penColor, smooth, domain=(3:6)] {(x-6)^2/3-5};
            \addplot[color=penColor,fill=penColor,only marks,mark=*] coordinates{(0,2)};  %% closed hole
            \addplot[color=penColor,fill=background,only marks,mark=*] coordinates{(3,0)};  %% open hole
            \addplot[color=penColor,fill=penColor,only marks,mark=*] coordinates{(3,-2)};
            \addplot[color=penColor,fill=penColor,only marks,mark=*] coordinates{(6,-5)};
          \end{axis}
  \end{tikzpicture}
  \end{image}

  Choose the correct (complete) graph of $f$.
  \begin{multipleChoice}
    \choice{\begin{image}
      \begin{tikzpicture}
          \begin{axis}[
                  xmin=-7, xmax=7, ymin=-6,ymax=6,
                  unit vector ratio*=1 1 1,
                  axis lines =middle, xlabel=$x$, ylabel=$y$,
                  every axis y label/.style={at=(current axis.above origin),anchor=south},
                  every axis x label/.style={at=(current axis.right of origin),anchor=west},
                  xtick={-7,...,7}, ytick={-6,...,8},
                  grid=major,width=4in,
                  grid style={dashed, gridColor},
                ]
      	        \addplot[very thick, color=penColor, smooth, domain=(0:3)] {2-(x*(2)^(.5)/3)^2};
                \addplot[very thick, color=penColor, smooth, domain=(3:6)] {(x-6)^2/3-5};
                \addplot[very thick, color=penColor, smooth, domain=(-3:0)] {-(2-(-x*(2)^(.5)/3)^2)};
                \addplot[very thick, color=penColor, smooth, domain=(-6:-3)] {-((-x-6)^2/3-5)};
                \addplot[color=penColor,fill=background,only marks,mark=*] coordinates{(3,0)};  %% open hole
                \addplot[color=penColor,fill=penColor,only marks,mark=*] coordinates{(3,-2)};
                \addplot[color=penColor,fill=penColor,only marks,mark=*] coordinates{(6,-5)};
                \addplot[color=penColor,fill=penColor,only marks,mark=*] coordinates{(0,2)};
                \addplot[color=penColor,fill=background,only marks,mark=*] coordinates{(0,-2)};
                \addplot[color=penColor,fill=background,only marks,mark=*] coordinates{(-3,0)};  %% open hole
                \addplot[color=penColor,fill=penColor,only marks,mark=*] coordinates{(-3,2)};
                \addplot[color=penColor,fill=penColor,only marks,mark=*] coordinates{(-6,5)};
              \end{axis}
      \end{tikzpicture}
      \end{image}
      }
    \choice[correct]{\begin{image}
      \begin{tikzpicture}
          \begin{axis}[
                  xmin=-7, xmax=7, ymin=-6,ymax=6,
                  unit vector ratio*=1 1 1,
                  axis lines =middle, xlabel=$x$, ylabel=$y$,
                  every axis y label/.style={at=(current axis.above origin),anchor=south},
                  every axis x label/.style={at=(current axis.right of origin),anchor=west},
                  xtick={-7,...,7}, ytick={-6,...,8},
                  grid=major,width=4in,
                  grid style={dashed, gridColor},
                ]
      	        \addplot[very thick, color=penColor, smooth, domain=(-3:3)] {2-(x*(2)^(.5)/3)^2};
                \addplot[very thick, color=penColor, smooth, domain=(3:6)] {(x-6)^2/3-5};
                \addplot[very thick, color=penColor, smooth, domain=(-6:-3)] {(-x-6)^2/3-5};
                \addplot[color=penColor,fill=background,only marks,mark=*] coordinates{(3,0)};  %% open hole
                \addplot[color=penColor,fill=penColor,only marks,mark=*] coordinates{(3,-2)};
                \addplot[color=penColor,fill=penColor,only marks,mark=*] coordinates{(6,-5)};
                \addplot[color=penColor,fill=background,only marks,mark=*] coordinates{(-3,0)};  %% open hole
                \addplot[color=penColor,fill=penColor,only marks,mark=*] coordinates{(-3,-2)};
                \addplot[color=penColor,fill=penColor,only marks,mark=*] coordinates{(-6,-5)};
              \end{axis}
      \end{tikzpicture}
      \end{image}
      }
  \end{multipleChoice}
\end{exercise}

%Exercise 10

\begin{exercise}

(a) A function $f$ is defined on the interval $(0,7)$. $f(1)=3$, and the following inequality holds:

$$\ln(x)\leq f(x) \leq x-1, x\neq 1, 0<x<7$$

Select the correct limit, and justification:

$\lim_{x\to 1} f(x)=$
\begin{prompt}
\begin{multipleChoice}
\choice{1, using the Intermediate Value Theorem}
\choice[correct]{0, using the Squeeze Theorem}
\choice{1, using the Squeeze Theorem}
\choice{Not enough information to determine the limit}
\end{multipleChoice}
\end{prompt}

(b) A function $f$ is defined on the interval $(0,2)$, and the following inequality holds:

$$\cos(\frac{\pi}{2}x)\leq f(x) \leq 1+(x-1)^2, x\neq 1, 0<x<2$$

Select the correct limit and justification:

$\lim_{x\to 1} f(x)=$
\begin{prompt}
\begin{multipleChoice}
\choice{1, using the Intermediate Value Theorem}
\choice{1, using the Squeeze Theorem}
\choice{0, using the Mean Value Theorem}
\choice[correct]{Not enough information to determine the limit}
\end{multipleChoice}
\end{prompt}

\end{exercise}

%Exercise 11

\begin{exercise}

The function $f$ is defined by $f(x)=\frac{x}{\sqrt{x^2-9}}$.

(a) Is the function $f$ defined on $[-3,3]$:
\begin{prompt}
\begin{multipleChoice}
\choice{Yes}
\choice[correct]{No}
\end{multipleChoice}
\end{prompt}

(b) Find the domain of $f$: $\begin{prompt}\Big(\answer{-\infty},\answer{-3}\Big) \cup \Big(\answer{3},\answer{\infty}\Big)\end{prompt}$

(c) Is the function $f$ odd, even, or neither:
\begin{prompt}
\begin{multipleChoice}
\choice[correct]{Odd}
\choice{Even}
\choice{Neither}
\end{multipleChoice}
\end{prompt}

(d) Find all horizontal asymptotes. \begin{prompt} $y=\answer{-1},\answer{1}$\end{prompt}

(e) Find all vertical asymptotes. \begin{prompt} $x=\answer{-3},\answer{3}$\end{prompt}
\end{exercise}

%Exercise 12

\begin{exercise}

Let
\[
f(x)=
\begin{cases}
\frac{e^x}{x+1} & \text{ if } x\leq 0, x\neq -1\\
\ln(x) & \text{ if } x>0
\end{cases}
\]

(i) Determine if the following limits exist. If they do, compute them analytically using the limit laws and techniques discussed in class. If they don't, say `DNE'. [You may not use a table of values, a graph, or L'Hospitals rule to justify your answer.]

(a) $\lim_{x\to -1^+} f(x)=\begin{prompt}\answer{\infty}\end{prompt}$

(b) $\lim_{x\to 0^+} f(x)=\begin{prompt}\answer{-\infty}\end{prompt}$

(c) $\lim_{x\to \infty} f(x)=\begin{prompt}\answer{\infty}\end{prompt}$

(d) $\lim_{x\to -\infty} f(x)=\begin{prompt}\answer{0}\end{prompt}$

(ii) Find all vertical asymptotes of $f$ or say `none':
$x=\begin{prompt}\answer{-1},\answer{0}\end{prompt}$

(iii) Find all horizontal asymptotes of $f$ or say `none':
$y=\begin{prompt} \answer{0}\end{prompt}$

(iv) Find the (largest) intervals of continuity of $f$:
$\begin{prompt}\Big(\answer{-\infty},\answer{-1}\Big) \text{ and}  \Big(\answer{-1},\answer{0}\Big)\text{ and } \Big(\answer{0},\answer{\infty}\Big)\end{prompt}$
\end{exercise}

%Exercise 13

\begin{exercise}

Let
\[
f(x)=
\begin{cases}
\frac{e^x}{x-1} & \text{ if } x\leq 0\\
\frac{6x+j}{2x+5} & \text{ if } x>0
\end{cases}
\]

(a) Find the value $j$ so that the function $f$ is continuous at $x=0$: $j=\begin{prompt}\answer{-5}\end{prompt}$

(ii) Find all vertical asymptotes of $f$ or say `none': $x=\begin{prompt}\answer[format=string]{none}\end{prompt}$

(iii) Find all horizontal asymptotes of $f$ or say `none': $y=\begin{prompt}\answer{0},\answer{3}\end{prompt}$

(iv) Find the (largest) interval(s) of continuity of $f$ (assuming $j$ is equal to the value you found in part (a)):
\begin{prompt}
$\Big(\answer{-\infty},\answer{\infty}\Big)$
\end{prompt}
\end{exercise}
\end{document}
