\documentclass{ximera}
%\usepackage{todonotes}
%\usepackage{mathtools} %% Required for wide table Curl and Greens
%\usepackage{cuted} %% Required for wide table Curl and Greens
\newcommand{\todo}{}

\usepackage{esint} % for \oiint
\ifxake%%https://math.meta.stackexchange.com/questions/9973/how-do-you-render-a-closed-surface-double-integral
\renewcommand{\oiint}{{\large\bigcirc}\kern-1.56em\iint}
\fi


\graphicspath{
  {./}
  {ximeraTutorial/}
  {basicPhilosophy/}
  {functionsOfSeveralVariables/}
  {normalVectors/}
  {lagrangeMultipliers/}
  {vectorFields/}
  {greensTheorem/}
  {shapeOfThingsToCome/}
  {dotProducts/}
  {partialDerivativesAndTheGradientVector/}
  {../productAndQuotientRules/exercises/}
  {../normalVectors/exercisesParametricPlots/}
  {../continuityOfFunctionsOfSeveralVariables/exercises/}
  {../partialDerivativesAndTheGradientVector/exercises/}
  {../directionalDerivativeAndChainRule/exercises/}
  {../commonCoordinates/exercisesCylindricalCoordinates/}
  {../commonCoordinates/exercisesSphericalCoordinates/}
  {../greensTheorem/exercisesCurlAndLineIntegrals/}
  {../greensTheorem/exercisesDivergenceAndLineIntegrals/}
  {../shapeOfThingsToCome/exercisesDivergenceTheorem/}
  {../greensTheorem/}
  {../shapeOfThingsToCome/}
  {../separableDifferentialEquations/exercises/}
  {vectorFields/}
}

\newcommand{\mooculus}{\textsf{\textbf{MOOC}\textnormal{\textsf{ULUS}}}}

\usepackage{tkz-euclide}\usepackage{tikz}
\usepackage{tikz-cd}
\usetikzlibrary{arrows}
\tikzset{>=stealth,commutative diagrams/.cd,
  arrow style=tikz,diagrams={>=stealth}} %% cool arrow head
\tikzset{shorten <>/.style={ shorten >=#1, shorten <=#1 } } %% allows shorter vectors

\usetikzlibrary{backgrounds} %% for boxes around graphs
\usetikzlibrary{shapes,positioning}  %% Clouds and stars
\usetikzlibrary{matrix} %% for matrix
\usepgfplotslibrary{polar} %% for polar plots
\usepgfplotslibrary{fillbetween} %% to shade area between curves in TikZ
\usetkzobj{all}
\usepackage[makeroom]{cancel} %% for strike outs
%\usepackage{mathtools} %% for pretty underbrace % Breaks Ximera
%\usepackage{multicol}
\usepackage{pgffor} %% required for integral for loops



%% http://tex.stackexchange.com/questions/66490/drawing-a-tikz-arc-specifying-the-center
%% Draws beach ball
\tikzset{pics/carc/.style args={#1:#2:#3}{code={\draw[pic actions] (#1:#3) arc(#1:#2:#3);}}}



\usepackage{array}
\setlength{\extrarowheight}{+.1cm}
\newdimen\digitwidth
\settowidth\digitwidth{9}
\def\divrule#1#2{
\noalign{\moveright#1\digitwidth
\vbox{\hrule width#2\digitwidth}}}





\newcommand{\RR}{\mathbb R}
\newcommand{\R}{\mathbb R}
\newcommand{\N}{\mathbb N}
\newcommand{\Z}{\mathbb Z}

\newcommand{\sagemath}{\textsf{SageMath}}


%\renewcommand{\d}{\,d\!}
\renewcommand{\d}{\mathop{}\!d}
\newcommand{\dd}[2][]{\frac{\d #1}{\d #2}}
\newcommand{\pp}[2][]{\frac{\partial #1}{\partial #2}}
\renewcommand{\l}{\ell}
\newcommand{\ddx}{\frac{d}{\d x}}

\newcommand{\zeroOverZero}{\ensuremath{\boldsymbol{\tfrac{0}{0}}}}
\newcommand{\inftyOverInfty}{\ensuremath{\boldsymbol{\tfrac{\infty}{\infty}}}}
\newcommand{\zeroOverInfty}{\ensuremath{\boldsymbol{\tfrac{0}{\infty}}}}
\newcommand{\zeroTimesInfty}{\ensuremath{\small\boldsymbol{0\cdot \infty}}}
\newcommand{\inftyMinusInfty}{\ensuremath{\small\boldsymbol{\infty - \infty}}}
\newcommand{\oneToInfty}{\ensuremath{\boldsymbol{1^\infty}}}
\newcommand{\zeroToZero}{\ensuremath{\boldsymbol{0^0}}}
\newcommand{\inftyToZero}{\ensuremath{\boldsymbol{\infty^0}}}



\newcommand{\numOverZero}{\ensuremath{\boldsymbol{\tfrac{\#}{0}}}}
\newcommand{\dfn}{\textbf}
%\newcommand{\unit}{\,\mathrm}
\newcommand{\unit}{\mathop{}\!\mathrm}
\newcommand{\eval}[1]{\bigg[ #1 \bigg]}
\newcommand{\seq}[1]{\left( #1 \right)}
\renewcommand{\epsilon}{\varepsilon}
\renewcommand{\phi}{\varphi}


\renewcommand{\iff}{\Leftrightarrow}

\DeclareMathOperator{\arccot}{arccot}
\DeclareMathOperator{\arcsec}{arcsec}
\DeclareMathOperator{\arccsc}{arccsc}
\DeclareMathOperator{\si}{Si}
\DeclareMathOperator{\scal}{scal}
\DeclareMathOperator{\sign}{sign}


%% \newcommand{\tightoverset}[2]{% for arrow vec
%%   \mathop{#2}\limits^{\vbox to -.5ex{\kern-0.75ex\hbox{$#1$}\vss}}}
\newcommand{\arrowvec}[1]{{\overset{\rightharpoonup}{#1}}}
%\renewcommand{\vec}[1]{\arrowvec{\mathbf{#1}}}
\renewcommand{\vec}[1]{{\overset{\boldsymbol{\rightharpoonup}}{\mathbf{#1}}}\hspace{0in}}

\newcommand{\point}[1]{\left(#1\right)} %this allows \vector{ to be changed to \vector{ with a quick find and replace
\newcommand{\pt}[1]{\mathbf{#1}} %this allows \vec{ to be changed to \vec{ with a quick find and replace
\newcommand{\Lim}[2]{\lim_{\point{#1} \to \point{#2}}} %Bart, I changed this to point since I want to use it.  It runs through both of the exercise and exerciseE files in limits section, which is why it was in each document to start with.

\DeclareMathOperator{\proj}{\mathbf{proj}}
\newcommand{\veci}{{\boldsymbol{\hat{\imath}}}}
\newcommand{\vecj}{{\boldsymbol{\hat{\jmath}}}}
\newcommand{\veck}{{\boldsymbol{\hat{k}}}}
\newcommand{\vecl}{\vec{\boldsymbol{\l}}}
\newcommand{\uvec}[1]{\mathbf{\hat{#1}}}
\newcommand{\utan}{\mathbf{\hat{t}}}
\newcommand{\unormal}{\mathbf{\hat{n}}}
\newcommand{\ubinormal}{\mathbf{\hat{b}}}

\newcommand{\dotp}{\bullet}
\newcommand{\cross}{\boldsymbol\times}
\newcommand{\grad}{\boldsymbol\nabla}
\newcommand{\divergence}{\grad\dotp}
\newcommand{\curl}{\grad\cross}
%\DeclareMathOperator{\divergence}{divergence}
%\DeclareMathOperator{\curl}[1]{\grad\cross #1}
\newcommand{\lto}{\mathop{\longrightarrow\,}\limits}

\renewcommand{\bar}{\overline}

\colorlet{textColor}{black}
\colorlet{background}{white}
\colorlet{penColor}{blue!50!black} % Color of a curve in a plot
\colorlet{penColor2}{red!50!black}% Color of a curve in a plot
\colorlet{penColor3}{red!50!blue} % Color of a curve in a plot
\colorlet{penColor4}{green!50!black} % Color of a curve in a plot
\colorlet{penColor5}{orange!80!black} % Color of a curve in a plot
\colorlet{penColor6}{yellow!70!black} % Color of a curve in a plot
\colorlet{fill1}{penColor!20} % Color of fill in a plot
\colorlet{fill2}{penColor2!20} % Color of fill in a plot
\colorlet{fillp}{fill1} % Color of positive area
\colorlet{filln}{penColor2!20} % Color of negative area
\colorlet{fill3}{penColor3!20} % Fill
\colorlet{fill4}{penColor4!20} % Fill
\colorlet{fill5}{penColor5!20} % Fill
\colorlet{gridColor}{gray!50} % Color of grid in a plot

\newcommand{\surfaceColor}{violet}
\newcommand{\surfaceColorTwo}{redyellow}
\newcommand{\sliceColor}{greenyellow}




\pgfmathdeclarefunction{gauss}{2}{% gives gaussian
  \pgfmathparse{1/(#2*sqrt(2*pi))*exp(-((x-#1)^2)/(2*#2^2))}%
}


%%%%%%%%%%%%%
%% Vectors
%%%%%%%%%%%%%

%% Simple horiz vectors
\renewcommand{\vector}[1]{\left\langle #1\right\rangle}


%% %% Complex Horiz Vectors with angle brackets
%% \makeatletter
%% \renewcommand{\vector}[2][ , ]{\left\langle%
%%   \def\nextitem{\def\nextitem{#1}}%
%%   \@for \el:=#2\do{\nextitem\el}\right\rangle%
%% }
%% \makeatother

%% %% Vertical Vectors
%% \def\vector#1{\begin{bmatrix}\vecListA#1,,\end{bmatrix}}
%% \def\vecListA#1,{\if,#1,\else #1\cr \expandafter \vecListA \fi}

%%%%%%%%%%%%%
%% End of vectors
%%%%%%%%%%%%%

%\newcommand{\fullwidth}{}
%\newcommand{\normalwidth}{}



%% makes a snazzy t-chart for evaluating functions
%\newenvironment{tchart}{\rowcolors{2}{}{background!90!textColor}\array}{\endarray}

%%This is to help with formatting on future title pages.
\newenvironment{sectionOutcomes}{}{}



%% Flowchart stuff
%\tikzstyle{startstop} = [rectangle, rounded corners, minimum width=3cm, minimum height=1cm,text centered, draw=black]
%\tikzstyle{question} = [rectangle, minimum width=3cm, minimum height=1cm, text centered, draw=black]
%\tikzstyle{decision} = [trapezium, trapezium left angle=70, trapezium right angle=110, minimum width=3cm, minimum height=1cm, text centered, draw=black]
%\tikzstyle{question} = [rectangle, rounded corners, minimum width=3cm, minimum height=1cm,text centered, draw=black]
%\tikzstyle{process} = [rectangle, minimum width=3cm, minimum height=1cm, text centered, draw=black]
%\tikzstyle{decision} = [trapezium, trapezium left angle=70, trapezium right angle=110, minimum width=3cm, minimum height=1cm, text centered, draw=black]

\author{Tom Dinitz and Nela Lakos}
\license{Creative Commons 3.0 By-NC}
\title{Midterm 3 Review}

\begin{document}
\begin{abstract}
Review questions for MIDTERM 3.
\end{abstract}
\maketitle
%Exercise 1

\begin{exercise}
The graph of $f'$ (the derivative of $f$) on the interval $(-6,7)$ is shown in the figure.

\begin{image}
\begin{tikzpicture}
    \begin{axis}[
            xmin=-6, xmax=7, ymin=-4,ymax=3,
            unit vector ratio*=1 1 1,
            axis lines =middle, xlabel=$x$, ylabel=$y$,
            every axis y label/.style={at=(current axis.above origin),anchor=south},
            every axis x label/.style={at=(current axis.right of origin),anchor=west},
            xtick={-6,...,7}, ytick={-6,...,6},
          ]
        \addplot[very thick, color=penColor, smooth, domain=(-6:0)] {x+2};
        \addplot[very thick, color=penColor, smooth, domain=(0:4)] {(2-x^2/8};
	\addplot[very thick, color=penColor, smooth, domain=(4:7)] {1-(x-5)^2};
    \end{axis}
\end{tikzpicture}
\end{image}



Use the given graph of $f'$ to answer the following questions about f:

(a) On what interval(s) is $f$ decreasing?\\

$(\answer{-6},\answer{-2})$ and $(\answer{6},\answer{7})$\\

(b) List the x- coordinates of all critical points of $f$ (in ascending order). \\

$x=\answer{-2},\answer{4},\answer{6}$\\

(c) List the x-coordinates of all critical points of $f$ that correspond to local maxima?\\

$x=\answer{6}$\\

(d)  List the x-coordinates of all critical points of $f$ that correspond to neither local maxima nor local minima? \\

$x=\answer{4}$\\

(e) On what intervals is $f$ concave up?\\

$(\answer{-6},0)$ and $(\answer{4},\answer{5})$\\

(f) List the x-coordinates of  all inflection points of $f$ (in ascending order). \\

$x=\answer{0},\answer{4},\answer{5}$\\
\end{exercise}
%Exercise 2
\begin{exercise}
Assume that a function $f$ is continuous on its domain, $(-1,6)$. The graph of $f'$, the derivative of $f$, is shown in the figure below.

\begin{image}
\begin{tikzpicture}
    \begin{axis}[
            xmin=-1, xmax=6, ymin=-2.5,ymax=1.5,
            unit vector ratio*=1 1 1,
            axis lines =middle, xlabel=$x$, ylabel=$y$,
            every axis y label/.style={at=(current axis.above origin),anchor=south},
            every axis x label/.style={at=(current axis.right of origin),anchor=west},
            xtick={-1,...,6}, ytick={-2,...,1},
          ]
        \addplot[ultra thick, color=penColor, smooth, domain=(-1:1)] {-2*x^2};
        \addplot[ultra thick, color=penColor, smooth, samples=500, domain=(1:3)] {-2*2^(-1/3)*(3-x)^(1/3)};
	\addplot[ultra thick, color=penColor, smooth, samples=100, domain=(3:5)] {2^(-1/3)*(x-3)^(1/3)};
        \addplot[ultra thick, color=penColor, smooth, domain=(5:6)] {x-6};
	\addplot[ultra thick, color=penColor, smooth]plot coordinates{(3,-.1) (3,0)};
	\addplot [color=penColor,fill=background,only marks,mark=*] coordinates{(-1,-2)};
	\addplot [color=penColor,fill=background,only marks,mark=*] coordinates{(5,1)};
	\addplot [color=penColor,fill=background, only marks, mark=*] coordinates{(5,-1)};
        \addplot[color=penColor,fill=background,only marks, mark=*] coordinates{(6,0)};
    \end{axis}
\end{tikzpicture}
\end{image}

(a) Write the x-coordinates of all critical points of $f$ (or write NONE), in ascending order.\\

$x=\answer{0},\answer{3},\answer{5}$\\

(b) Write the x-coordinates of all local maxima of $f$ (or write NONE).\\

$x=\answer{5}$\\

(c) Write the x-coordinates of all local minima of $f$ (or write NONE).\\

$x=\answer{3}$\\

(d)Find the interval(s) on which $f$ is increasing.\\

$(\answer{3},\answer{5})$\\

(e) Find the interval(s) on which $f$ is concave down. \\

$(\answer{0},\answer{1})$\\

(f) Write the x-coordinates of all inflection points (or write NONE), in ascending order.\\

$x=\answer{0},\answer{1}$\\
\end{exercise}

%Exercise 3
\begin{exercise}
Draw a possible graph of $f$ given that it satisfies all of the following conditions.

(a) Domain of $f=(-\infty,-2)\cup (-2,\infty)$,

(b) f is continuous on its domain and differentiable all all points in the domain except at $x=6$

(c) $f(2)=0, f(6)=4$,

(d) $\lim_{x\to -2} f(x)=-8$, $\lim_{x\to-\infty}f(x)=1$, $\lim_{x\to\infty}f(x)=1$,

(e) $f'(x)<0$ on $(-\infty,-2)$, and on $(6,\infty)$,

(f) $f'(x)>0$ on $(-2,6)$,

(g) $f''(x)<0$ on $(-\infty,-2)$ and on $(-2,2)$,

(h) $f''(x)>0$ on $(2,6)$ and on $(6,\infty)$

Once you've finished, select `Done', and compare your answer with the one shown

\begin{multipleChoice}
\choice[correct]{Done}
\end{multipleChoice}
\begin{exercise}
\begin{image}
\begin{tikzpicture}
    \begin{axis}[
            xmin=-8,xmax=10,ymin=-10,ymax=6,
            samples=100,
            axis lines =middle, xlabel=$x$, ylabel=$y$,
	    xtick={-8,-6,-4,-2,0,2,4,6,8,10},ytick={-8,-6,-4,-2,0,2,4,6},
            every axis y label/.style={at=(current axis.above origin),anchor=south},
            every axis x label/.style={at=(current axis.right of origin),anchor=west}
          ]
          \addplot [very thick, penColor, smooth, domain=(-8:-2)] {1+9*1/(4*x+7)};
	  \addplot [very thick, penColor, smooth, domain=(-2:2)] {-1/2*(x-2)^2};
	  \addplot [very thick, penColor, smooth, domain=(2:6)] {1/4*(x-2)^2};
	  \addplot [very thick, penColor, smooth, domain=(6:10)] {1+6/(2*(x-5))};
	  \addplot [color=penColor,fill=penColor, only marks, mark=*] coordinates{(2,0)};
 	  \addplot [color=penColor,fill=background,only marks,mark=*] coordinates{(-2,-8)};
	  \addplot [textColor, dashed] plot coordinates {(-8,1) (10,1)};
        \end{axis}
\end{tikzpicture}
\end{image}
\end{exercise}
\end{exercise}

%Exercise 4
\begin{exercise}
A function $f'$ (the derivative of $f$) is given by $f'(x)=(x-5)^3$. Answer the following, or write  `DNE' (does not exist).

(a) List all interval(s) on which $f$ is increasing.\\
$(\answer{5},\answer{\infty})$\\

(b) List x-coordinates of all points where $f$ has a local maximum. \\
$x=\answer[format=string]{DNE}$\\

(c)  List x-coordinates of all points where $f$ has a local minimum.\\

$x=\answer{5}$\\

(d) Find  $f''(x)$.\\

$f''(x)=\answer{3(x-5)^2}$\\

(e) List all interval(s) on which $f$ is concave up.\\

$(\answer{-\infty},\answer{\infty})$\\

(f) List x-coordinates of all inflection points of $f$.\\

$x=\answer[format=string]{DNE}$\\

\end{exercise}

%Exercise 5
\begin{exercise}
Evaluate the following limits.  You may use L`Hospital's Rule

(a) $\lim_{x\to 0^+} (e^x-1)^\frac{1}{x}=\answer{1}$

(b) $\lim_{x\to 0^+} \tan(x)^{x^2}=\answer{1}$

(c) $\lim_{x\to \infty} \frac{\ln(x^10)}{\sqrt{x}}=\answer{0}$

(d) $\lim_{x\to 0} \frac{e^x-1-x}{x^2}=\answer{\frac{1}{2}}$

(e) $\lim_{x\to (\frac{\pi}{2})^-} \frac{\cos(x)\sin(2x)}{(x-\frac{\pi}{2})^2}=\answer{2}$

(f) $\lim_{x\to\infty} (x-\sqrt{x^2+4x})=\answer{-2}$
\end{exercise}

%Exercise 6
\begin{exercise}
The figure shows a right triangle in the first quadrant.  One side of the triangle is on the x-axis; its hypotenuse runs from the origin to a point on the parabola $y=4-x^2$. Find the coordinates $x$ and $y$ that maximize the area of the triangle.

\begin{image}
\begin{tikzpicture}
    \begin{axis}[
            xmin=-1, xmax=3, ymin=-1,ymax=5,
            unit vector ratio*=1 1 1,
            axis lines =middle, xlabel=$x$, ylabel=$y$,
            every axis y label/.style={at=(current axis.above origin),anchor=south},
            every axis x label/.style={at=(current axis.right of origin),anchor=west},
            xtick={0,3}, ytick={-1,1,3,5},
            grid=major,width=4in,
            grid style={dashed, gridColor},
          ]
      \addplot[very thick, samples=200, color=penColor, smooth, domain=(0:3)] {4-x^2} node [pos=.15, above right] {$y=4-x^2$};
      \addplot[color=red,very thick] plot coordinates {(1.5,0) (1.5,1.75)} node [pos=1,above right] {(x,y)};
      \addplot[color=red,very thick] plot coordinates {(0,0) (1.5,0)};
      \addplot[color=red,very thick] plot coordinates {(0,0) (1.5,1.75)};
\end{axis}
\end{tikzpicture}
\end{image}

$x=\answer{\frac{4}{3}}, y=\answer{4-\frac{4}{3}^2}$
\end{exercise}

%Exercise 7
\begin{exercise}
(i) The point on the curve $y=\sqrt{x}$ that is closest to the point $(4,0)$ occurs at $x=\answer{\frac{7}{2}}$

(ii) A cruise line offers a trip for \$1000 per passenger. If at least 100 passengers sign up, the price is reduced for all passengers by \$5 for every additional passenger (beyond 100) who goes on the trip.  The boat can accomodate 250 passengers.

The number of passengers which maximizes the cruise line's total revenue is $\answer{150}$

What price does each passenger pay if that number of passengers goes on the cruise? $\answer{750}$

(iii) Find the dimensions of the right circular cylinder of maximum volume that can be places inside a sphere of radius R: $r=\answer{R\sqrt{\frac{2}{3}}}, h=\answer{2R\sqrt{3}}$

(iv) A certain tank consists of a right circular cylinder with hemispherical ends. For a given surface area S, find the dimensions (radius and length) of the tank with maximum volume (your answer should include S): $r=\answer{\sqrt{\frac{S}{4\pi}}}, l=\answer{0}$

(v) A square piece of tin 24 in on each side is to be made into an open-top box by cutting a small square from each corner and bending up the flaps to form the sides.  What is the side length of the square that should be cut from each corner to make the volume of the box as large as possible? $s=\answer{4}$
\end{exercise}

%Exercise 8
\begin{exercise}
A toy roller-coaster has been designed so that the rail has the shape of the curve given in the figure below, where $f(x)=x-\sin(\frac{\pi}{5}x)$ (x in inches, $f(x)$ gives the altitude)
\begin{image}
 \begin{tikzpicture}
    \begin{axis}[
        xmin=-0.3,xmax=10.3,ymin=-0.3,ymax=10.3,
        clip=true,
        unit vector ratio*=1 1 1,
        axis lines=center,
        grid = major,
        ytick={-20,-18,...,20},
        xtick={-20,-18,...,20},
        xlabel=$x$, ylabel=$y$,
        y tick label style={anchor=west},
        every axis y label/.style={at=(current axis.above origin),anchor=south},
        every axis x label/.style={at=(current axis.right of origin),anchor=west},
      ]
      \addplot[very thick,penColor,domain=0:10,samples=50] plot{x - sin(36*x)};

      \addplot[only marks,mark=*,penColor] coordinates{(0,0) (10,10)};

      \node at (axis cs:3,9) {$y=f(x)$};
      \end{axis}`
  \end{tikzpicture}
\end{image}

The average rate of change of the altitude of the roller coaster on the interval $[0,10]$ is $\answer{1}$.

Select the best interpretation of $f'(a)$ for $0<a<10$.
\begin{multipleChoice}
\choice{$f'(a)$ is the altitude at $a$.}
\choice{$f'(a)$ is the average rate of change of the altitude on the interval $[0,a]$.}
\choice{$f'(a)$ is the velocity of an object at $a$.}
\choice[correct]{$f'(a)$ is the instantaneous rate of change of the altitude at $a$.}
\choice{$f'(a)$ is the slope of the secant line passing through $(0,0)$ and $(10,10)$.}
\end{multipleChoice}

Because $f$ is \wordChoice{\choice[correct]{continuous}\choice{differentiable}} on the interval $[0,10]$ and $f$ is \wordChoice{\choice{continuous}\choice[correct]{differentiable}} on the interval $(0,10)$, $f$ satisfies the conditions of the Mean Value Theorem.

By the Mean Value Theorem, there exists $c$ in $(0,10)$ such that $f'(c) = 1$.  In face this happens twice, when $c_1=\answer{\frac{5}{2}}$ and when $c_2=\answer{\frac{15}{2}}$ (assume $c_1<c_2$).

The steepest point on the roller coaster is $\left(\answer{5},\answer{5}\right)$. (Hint: maximize $f'(x)$ on $(0,10)$.)

The linear approximation, $L$, to $f$ at $a=5$ is
\[
L(x) = \answer{\left(1+\frac{\pi}{5}\right)(x-5) + 5}.
\]

Using this linear approximation we estimate that $f(7)$ is approximately
\[
f(7) \approx \answer{\left(1+\frac{\pi}{5}\right)2+5}.
\]
This estimate is an \wordChoice{\choice[correct]{overestimate}\choice{underestimate}} because $f$ is \wordChoice{\choice{concave up}\choice[correct]{concave down}} between 5 and 7.

When $x$ changes from $x=5$ to $x+\Delta x=7$ the change in $f$, $\Delta y$b is
\[
\Delta y = \answer{2-\sin\left(\frac{7\pi}{5}\right)}.
\]
We can approximate this change with the differential $\mathrm{d}y$ which is
\[
\mathrm{d}y = \answer{\left(1+\frac{\pi}{5}\right)2}.
\]
\end{exercise}

%Exercise 10
\begin{exercise}
Select all correct answers for each question below.

(i) At what point(s) $c$ does the conclusion of the Mean Value Theorem hold for $f(x)=x^2$ on the interval $[0,2]$?

\begin{selectAll}
\choice{0}
\choice[correct]{1}
\choice{2}
\choice{$\frac{1}{2}$}
\choice{Such a point does not exit}
\choice{None of the above}
\end{selectAll}

(ii) The equation of the line that represents the linear approximation to the function $f(x)=\ln(x)$ at $a=1$ is

\begin{selectAll}
\choice{$y=x$}
\choice{$y=-x$}
\choice{$y=ln(x-1)$}
\choice[correct]{$y=x-1$}
\choice{Such a line does not exit}
\choice{None of the above}
\end{selectAll}

(iii) Determine the following indefinite integral $\int(\sec^2(x)+5)dx$

\begin{selectAll}
\choice{$\frac{\sec^3(x)}{3}+5x+C$}
\choice{$2\sec(x)\tan(x)+5x+C$}
\choice{$2\sec^2(x)\tan(x)+C$}
\choice{$\tan(x)+C$}
\choice[correct]{$\tan(x)+5x+C$}
\choice{None of the above}
\end{selectAll}

(iv) Evaluate the expression $\sum_{k=1}^3(4k-1)$

\begin{selectAll}
\choice{$11$}
\choice[correct]{$21$}
\choice{$4k-1$}
\choice{$3$}
\choice{None of the above}
\end{selectAll}
\end{exercise}

%Exercise 11
\begin{exercise}
(i) The population of a culture of cells grows according to the function $P(t)$, where $t\geq 0$ is measured in weeks.

\[
\begin{array}{|c|c|}\hline
t & P(t) \\ \hline
0 & 0 \\ \hline
1 & 40 \\ \hline
3 & 65 \\ \hline
4 & 80 \\ \hline
\end{array}
\]

(a) What is the average rate of change in the population during the time interval $[0,4]$? $\answer{20}$ $\frac{\text{cells}}{\text{week}}$

(b) Assume that $P(t)$ is continuous on $[0,+\infty)$ and differentiable on $(0,+\infty)$. Is there a moment in which the instantaneous rate of change of $P$ is equal to the average rate of change computed in part (a)? \begin{multipleChoice}\choice[correct]{Yes}\choice{No}\end{multipleChoice}

If yes, select the theorem which guarantees the existence of such a point. If no, select `No Theorem'. \begin{multipleChoice}\choice{Intermediate Value Theorem}\choice[correct]{Mean Value Theorem}\choice{L`Hospital's Rule}\choice{No Theorem}\end{multipleChoice}

(ii) Let $f(x)=\frac{4}{x}$. The graph of $f$ is given in the figure below.

\begin{image}
\begin{tikzpicture}
    \begin{axis}[
            xmin=0, xmax=4, ymin=0,ymax=6,
            unit vector ratio*=1 1 1,
            axis lines =middle, xlabel=$x$, ylabel=$y$,
            every axis y label/.style={at=(current axis.above origin),anchor=south},
            every axis x label/.style={at=(current axis.right of origin),anchor=west},
            xtick={0,...,4}, ytick={0,...,6},
          ]
        \addplot[ultra thick, color=penColor, smooth, domain=(.5:4)] {4/x} node [pos=0.5, above right] {$y=f(x)$};
    \end{axis}
\end{tikzpicture}
\end{image}

(a) The linear approximation L to the function $f$ at $a=2$ is $L(x)=\answer{4-x}$

(b) Select the figure which includes the graph of $L$:

\begin{multipleChoice}
\choice{
\begin{image}
\begin{tikzpicture}
    \begin{axis}[
            xmin=0, xmax=4, ymin=0,ymax=6,
            unit vector ratio*=1 1 1,
            axis lines =middle, xlabel=$x$, ylabel=$y$,
            every axis y label/.style={at=(current axis.above origin),anchor=south},
            every axis x label/.style={at=(current axis.right of origin),anchor=west},
            xtick={0,...,4}, ytick={0,...,6},
          ]
        \addplot[ultra thick, color=penColor, smooth, domain=(.5:4)] {4/x} node [pos=0.5, above
 right] {$y=f(x)$};
	\addplot[ultra thick, color=red, smooth] plot coordinates {(0,5) (4,1)};
    \end{axis}
\end{tikzpicture}
\end{image}
}

\choice[correct]{\begin{image}
\begin{tikzpicture}
    \begin{axis}[
            xmin=0, xmax=4, ymin=0,ymax=6,
            unit vector ratio*=1 1 1,
            axis lines =middle, xlabel=$x$, ylabel=$y$,
            every axis y label/.style={at=(current axis.above origin),anchor=south},
            every axis x label/.style={at=(current axis.right of origin),anchor=west},
            xtick={0,...,4}, ytick={0,...,6},
          ]
        \addplot[ultra thick, color=penColor, smooth, domain=(.5:4)] {4/x} node [pos=0.5, above
 right] {$y=f(x)$};
	\addplot[ultra thick, color=red, smooth, domain=(.5:4)] {4-x};
    \end{axis}
\end{tikzpicture}
\end{image}
}

\choice{\begin{image}
\begin{tikzpicture}
    \begin{axis}[
            xmin=0, xmax=4, ymin=0,ymax=6,
            unit vector ratio*=1 1 1,
            axis lines =middle, xlabel=$x$, ylabel=$y$,
            every axis y label/.style={at=(current axis.above origin),anchor=south},
            every axis x label/.style={at=(current axis.right of origin),anchor=west},
            xtick={0,...,4}, ytick={0,...,6},
          ]
        \addplot[ultra thick, color=penColor, smooth, domain=(.5:4)] {4/x} node [pos=0.5, above
 right] {$y=f(x)$};
	\addplot[ultra thick, color=red, smooth, domain=(.5:4)] {-4/9*(x-3)+4/3};
    \end{axis}
\end{tikzpicture}
\end{image}
}
\end{multipleChoice}

(c) Use the linear apprxoimation $L$ to estimate the value of $\frac{4}{2.8}$. $\answer{1.2}$

(d) The approximation in part (c) is an \wordChoice{\choice{overestimate} \choice[correct]{underestimate}} because $f$ is \wordChoice{\choice[correct]{concave up}\choice{concave down}}.
\end{exercise}

%Exercise 12
\begin{exercise}
Given $f(x)=8x-x^2$ on $[0,8]$, and $n=4$, complete the following steps:

(a) Calculate $\Delta x=\answer{2}$ and the grid points (in ascending order) $x_1=\answer{0}, x_2=\answer{2}, x_3=\answer{4}, x_4=\answer{6}, x_5=\answer{8}$

(b) Select the figure with the right Riemann sum drawn in

\begin{multipleChoice}

\choice{
\begin{image}
\begin{tikzpicture}
    \begin{axis}[
            xmin=0, xmax=8.5, ymin=0,ymax=20,
            unit vector ratio*=1 1 1,
            axis lines =middle, xlabel=$x$, ylabel=$y$,
            every axis y label/.style={at=(current axis.above origin),anchor=south},
            every axis x label/.style={at=(current axis.right of origin),anchor=west},
            xtick={0,...,8}, ytick={0,4,8,12,16,20},
          ]
        \addplot[ultra thick, color=penColor, smooth, domain=(0:8)] {8*x-x^2};
	\draw[red] (axis cs:0,0) rectangle (axis cs:2,0);
        \draw[red] (axis cs:2,0) rectangle (axis cs:4,12);
        \draw[red] (axis cs:4,0) rectangle (axis cs:6,16);
        \draw[red] (axis cs:6,0) rectangle (axis cs:8,12);
    \end{axis}
\end{tikzpicture}
\end{image}
}

\choice{
\begin{image}
\begin{tikzpicture}
    \begin{axis}[
            xmin=0, xmax=8.5, ymin=0,ymax=20,
            unit vector ratio*=1 1 1,
            axis lines =middle, xlabel=$x$, ylabel=$y$,
            every axis y label/.style={at=(current axis.above origin),anchor=south},
            every axis x label/.style={at=(current axis.right of origin),anchor=west},
            xtick={0,...,8}, ytick={0,4,8,12,16,20},
          ]
        \addplot[ultra thick, color=penColor, smooth, domain=(0:8)] {8*x-x^2};
        \draw[red] (axis cs:0,0) rectangle (axis cs:2,7);
        \draw[red] (axis cs:2,0) rectangle (axis cs:4,15);
        \draw[red] (axis cs:4,0) rectangle (axis cs:6,15);
        \draw[red] (axis cs:6,0) rectangle (axis cs:8,7);
    \end{axis}
\end{tikzpicture}
\end{image}
}

\choice[correct]{
\begin{image}
\begin{tikzpicture}
    \begin{axis}[
            xmin=0, xmax=8.5, ymin=0,ymax=20,
            unit vector ratio*=1 1 1,
            axis lines =middle, xlabel=$x$, ylabel=$y$,
            every axis y label/.style={at=(current axis.above origin),anchor=south},
            every axis x label/.style={at=(current axis.right of origin),anchor=west},
            xtick={0,...,8}, ytick={0,4,8,12,16,20},
          ]
        \addplot[ultra thick, color=penColor, smooth, domain=(0:8)] {8*x-x^2};
        \draw[red] (axis cs:0,0) rectangle (axis cs:2,12);
        \draw[red] (axis cs:2,0) rectangle (axis cs:4,16);
        \draw[red] (axis cs:4,0) rectangle (axis cs:6,12);
        \draw[red] (axis cs:6,0) rectangle (axis cs:8,0);
    \end{axis}
\end{tikzpicture}
\end{image}
}


\end{multipleChoice}

(c) Calculate the right Riemann sum: $\answer{80}$

\end{exercise}

%Exercise 13
\begin{exercise}
The figure illustrates the Riemann sum $\sum_{k=1}{n}f(x_k^*)\Delta x$ for $f$ on the interval $[4,8]$.  Use the figure to answer the following questions.

\begin{image}
\begin{tikzpicture}
    \begin{axis}[
            xmin=3, xmax=8, ymin=0,ymax=4,
            unit vector ratio*=1 1 1,samples=500,
            axis lines =middle, xlabel=$x$, ylabel=$y$,
            every axis y label/.style={at=(current axis.above origin),anchor=south},
            every axis x label/.style={at=(current axis.right of origin),anchor=west},
            xtick={3,...,8}, ytick={0},
          ]
        \addplot[ultra thick, color=penColor, smooth, domain=(4:8)] {sqrt(x-4)};
        \draw[draw=red,fill=lightgray] (axis cs:4,0) rectangle (axis cs:5,0);
        \draw[draw=red,fill=lightgray] (axis cs:5,0) rectangle (axis cs:6,1);
        \draw[draw=red,fill=lightgray] (axis cs:6,0) rectangle (axis cs:7,1.41421);
        \filldraw[draw=red,fill=lightgray] (axis cs:7,0) rectangle (axis cs:8,1.7321);
    \end{axis}
\end{tikzpicture}
\end{image}


(i) Given the interval $[4,8]$, what is $n$.
Select the best interpretation of $f'(a)$ for $0<a<10$.
\begin{multipleChoice}
\choice{n=3}
\choice{n=8}
\choice{n=5}
\choice[correct]{n=4}
\choice{None of the above}
\end{multipleChoice}

(ii) What is $\Delta x$?
\begin{multipleChoice}
\choice{4}
\choice{0}
\choice{1}
\choice[correct]{2}
\choice{$\frac{4}{k}$}
\choice{None of the above}
\end{multipleChoice}

(iii) Given that $x_0=4$, select the expression for the grid points $x_k$, for $k=1,2,...n$?
\begin{multipleChoice}
\choice[correct]{$4+(k-1)\Delta x$}
\choice{$4+k\Delta x$}
\choice{$k$}
\choice{$k\Delta x$}
\choice{None of the above}
\end{multipleChoice}

(iv) What is $x_k^*$ for $k=1,2,..,n$.
\begin{multipleChoice}
\choice{$\frac{x_{k-1}+x_k}{2}$}
\choice[correct]{$x_k$}
\choice{$x_{k-1}$}
\choice{$x_0$}
\choice{None of the above}
\end{multipleChoice}

(v) Evaluate this Riemann sum
\begin{multipleChoice}
\choice{$f(5)+f(6)+f(7)+f(8)$}
\choice{$f(6)+f(7)+f(8)$}
\choice[correct]{$f(4)+f(5)+f(6)+f(7)$}
\choice{None of the above}
\end{multipleChoice}

(vi) This Riemann sum is
\begin{multipleChoice}
\choice{midpoint Riemann sun}
\choice{right Riemann sum}
\choice[correct]{left Riemann sum}
\choice{None of the above}
\end{multipleChoice}
\end{exercise}

%Exercise 14
\begin{exercise}
Determine the following indefinite integrals.  Check your work by differentiation.

(a) $\int\frac{4+x^2}{1+x^2}dx=\answer{x_3\arctan(x)+C}$

(b) $\int\frac{\cos^3(\theta)+1}{\cos^2(\theta)}d\theta=\answer{\sin(x)+\tan(x)+C}$

(c) $\int\frac{x^2-3x}{x}dx=\answer{\frac{1}{2}x-3x+C}$

(d) $\int 5x(x^2-3x)dx=\answer{\frac{5}{4}x^4-5x^3+C}$

(e) $\int\frac{x^2-3\sqrt{x}}{\sqrt[3]{x}}dx = \answer{\frac{3}{8}x^{\frac{8}{3}}-\frac{18}{7}x^frac{7}{6}+C}$

(f) $\int(\sec^2(\theta)+\sec(\theta)\tan(\theta))d\theta=\answer{\tan(\theta)+\sec(\theta)+C}$
\end{exercise}

%Exercise 15
\begin{exercise}
The figure shows the graph of a function $f$. At what point(s) $c$ does the conclusion of the Mean Value Theorem hold for $f(x)$ on the interval $[4,8]$? (c is an integer)

\begin{image}
\begin{tikzpicture}
    \begin{axis}[
            xmin=3, xmax=8, ymin=0,ymax=4,
            unit vector ratio*=1 1 1, samples=500,
            axis lines =middle, xlabel=$x$, ylabel=$y$,
            every axis y label/.style={at=(current axis.above origin),anchor=south},
            every axis x label/.style={at=(current axis.right of origin),anchor=west},
            xtick={3,...,8}, ytick={0},
          ]
        \addplot[ultra thick, color=penColor, smooth, domain=(4:8)] {sqrt(x-4)};
    \end{axis}
\end{tikzpicture}
\end{image}


\begin{selectAll}
\choice{$4$}
\choice[correct]{$5$}
\choice{$6$}
\choice{$7$}
\end{selectAll}
\end{exercise}

%Exercise 16
\begin{exercise}
Consider the following limit of Riemann sums for a function $g$ on $[0, \pi]$. 

$$\lim_{n\to\infty}\sum_{k=1}^n x_k^* \cos(x_k^*)\Delta x_k$$

Express the limit as a definite integral.

\begin{multipleChoice}
\choice{$\int_0^n(-\frac{\sin(x^2)}{2})dx$}
\choice{$\int_0^\pi(-\frac{\sin(x^2)}{2})dx$}
\choice[correct]{$\int_0^\pi x\cos(x)dx$}
\choice{$\int_0^n x\cos(x)dx$}
\choice{None of the above}
\end{multipleChoice}
\end{exercise}

%Exercise 17
\begin{exercise}
Consider an object moving along a straight line with the velocity $v(t)=12-3t$ on $[0,6]$.  Find the distance traveled over the given interval.
\begin{multipleChoice}
\choice{$\int_0^6(12-3t)dt$}
\choice{$\int_0^6(3t-12)dt$}
\choice[correct]{$\int_0^4(12-3t)dt+\int_4^6(3t-12)dt$}
\choice{$\int_0^4(12-3t)dt-\int_4^6(3t-12)dt$}
\choice{None of the above}
\end{multipleChoice}
\end{exercise}

%Exercise 18
\begin{exercise}
Suppose that $f(x)\geq 0$ on $[1,3]$, $f(x)\leq 0$ on $[3,5]$, $\int_1^3f(x)dx=4$, and $\int_3^5f(x)dx=-9$.

(i) Evaluate the following integrals

(a) $\int_1^5 3f(x)dx=\answer{-15}$

(b) $\int_1^5 |f(x)|dx=\answer{13}$

(c) $\int_1^3 (f(x)-x+2)dx=\answer{4}$

(ii) Assume that $f$ is odd. Evaluate $\int_{-5}^{-3}f(x)dx=\answer{9}$

(ii) Assume that $f$ is even. Evaluate $\int_{-5}^{-3}f(x)dx=\answer{-9}$

\end{exercise}

%Exercise 19
\begin{exercise}
Given that $\int_0^3f(x)dx=4$, and $\int_3^6f(x)dx=-6$, evaluate the following integrals:

(a) $\int_3^0 5f(x)dx=\answer{-20}$

(b) $\int_0^6 f(x)dx=\answer{-2}$

(c) $\int_3^6 (f(x)+2)dx=\answer{0}$
\end{exercise}

%Exericse 20
\begin{exercise}
The acceleration function (in m/$s^2$) is for a particle moving along a line is given by $a(t)=2t-\sin(t)$, $0\leq t\leq 8$, $v(0)=3, s(0)=4$.

(a) Find the velocity at time t: $v(t)=\answer{t^2+\cos(t)+2} \frac{m}{s}$

(b) Find the distance traveled during the given time interval: $\answer{\frac{704}{3}+\sin(8)} m$

(c) Find the position at time t: $s(t)=\answer{\frac{1}{3}t^3+\sin(t)+2t+4}$
\end{exercise}

%Exericse 21
\begin{exercise}
The acceleration function (in m/$s^2$) is for a particle moving along a line is given by $a(t)=2t-4$, $0\leq t$, $v(0)=3, s(0)=0$.

(a) Find the velocity at time t: $v(t)=\answer{t^2-4t+3} \frac{m}{s}$

(b) Find the position at time t: $s(t)=\answer{\frac{1}{3}t^3-2t^2+3t}$
\end{exercise}

%Exercise 22
\begin{exercise}
Find the particular solution of $f'(x)=3x^2-1$ which satisfies the intial condition $f(2)=1$. $f(x)=\answer{x^3-x-5}$
\end{exercise}

%Exercise 23
\begin{exercise}
(i) The graph of a function $f$ is shown in the figure.

\begin{image}
\begin{tikzpicture}
    \begin{axis}[
            xmin=0, xmax=10, ymin=0,ymax=7,
            unit vector ratio*=1 1 1,
            axis lines =middle, xlabel=$x$, ylabel=$y$,
            every axis y label/.style={at=(current axis.above origin),anchor=south},
            every axis x label/.style={at=(current axis.right of origin),anchor=west},
            xtick={0,...,10}, ytick={0,...,6},
          ]
        \addplot[very thick, color=penColor, smooth, domain=(1:5)] {x+1};
        \addplot[very thick, color=penColor, smooth, domain=(5:9)] {11-x};
	\addplot [color=penColor,fill=penColor,only marks,mark=*] coordinates{(1,2)} node [below] {(1,2)};
	\addplot [color=penColor,fill=penColor,only marks,mark=*] coordinates{(5,6)} node [above] {(5,6)};
	\addplot [color=penColor,fill=penColor,only marks,mark=*] coordinates{(9,2)} node [below] {(9,2)};
    \end{axis}
\end{tikzpicture}
\end{image}

(a) Use geometry to evlaluate $\int_1^9 f(x)dx=\answer{32}$

(b) Select the graph of a rectangle whose net area is equal to $\int_1^9 f(x)dx=32$
\begin{multipleChoice}
\choice[correct]{
\begin{image}
\begin{tikzpicture}
    \begin{axis}[
            xmin=0, xmax=10, ymin=0,ymax=7,
            unit vector ratio*=1 1 1,
            axis lines =middle, xlabel=$x$, ylabel=$y$,
            every axis y label/.style={at=(current axis.above origin),anchor=south},
            every axis x label/.style={at=(current axis.right of origin),anchor=west},
            xtick={0,...,10}, ytick={0,...,6},
          ]
        \draw[thick, red] (axis cs:1,0) rectangle (axis cs:9,4);
    \end{axis}
\end{tikzpicture}
\end{image}
}

\choice{
\begin{image}
\begin{tikzpicture}
    \begin{axis}[
            xmin=0, xmax=10, ymin=0,ymax=7,
            unit vector ratio*=1 1 1,
            axis lines =middle, xlabel=$x$, ylabel=$y$,
            every axis y label/.style={at=(current axis.above origin),anchor=south},
            every axis x label/.style={at=(current axis.right of origin),anchor=west},
            xtick={0,...,10}, ytick={0,...,6},
          ]
        \draw[thick, red] (axis cs:1,0) rectangle (axis cs:9,2);
    \end{axis}
\end{tikzpicture}
\end{image}
}

\choice{
\begin{image}
\begin{tikzpicture}
    \begin{axis}[
            xmin=0, xmax=10, ymin=0,ymax=7,
            unit vector ratio*=1 1 1,
            axis lines =middle, xlabel=$x$, ylabel=$y$,
            every axis y label/.style={at=(current axis.above origin),anchor=south},
            every axis x label/.style={at=(current axis.right of origin),anchor=west},
            xtick={0,...,10}, ytick={0,...,6},
          ]
        \draw[thick, red] (axis cs:1,0) rectangle (axis cs:9,6);
    \end{axis}
\end{tikzpicture}
\end{image}
}


\end{multipleChoice}

(ii) The graph of $f$ on the interval $[0,6]$ is given in the figure

\begin{image}
\begin{tikzpicture}
    \begin{axis}[
            xmin=0, xmax=6, ymin=-2,ymax=2.3,
            unit vector ratio*=1 1 1,
            axis lines =middle, xlabel=$x$, ylabel=$y$,
            every axis y label/.style={at=(current axis.above origin),anchor=south},
            every axis x label/.style={at=(current axis.right of origin),anchor=west},
            xtick={0,...,6}, ytick={-2,...,2},
          ]
        \addplot[very thick, color=penColor, smooth, domain=(0:2)] {-sqrt(4-x^2)} node [pos=0.75, below right] {1/4 of circle of radius 2};
        \addplot[very thick, color=penColor, smooth, domain=(2:4)] {x-2};
        \addplot[very thick, color=penColor, smooth, domain=(4:6)] {2};
    \end{axis}
\end{tikzpicture}
\end{image}

(a) Use geometry to evaluate $\int_0^6 f(t)dt=\answer{6-\pi}$

(b) Compute $\int_0^6 |f(t)|dt=\answer{\pi+6}$
\end{exercise}

%Exercise 24
\begin{exercise}
An object is moving along a straight line. The graphs of its position function $s$, its velocity $v$, and its acceleration $a$, on the time interval $[1,3]$ are given below.

\begin{image}
  \begin{tikzpicture}
      \begin{axis}[
              xmin=0, xmax=3, ymin=-2,ymax=6,
              unit vector ratio*=1 1 1,
              axis lines =middle, xlabel=$x$, ylabel=$y$,
              every axis y label/.style={at=(current axis.above origin),anchor=south},
              every axis x label/.style={at=(current axis.right of origin),anchor=west},
              xtick={0,...,3}, ytick={-2,...,6},
            ]
          \addplot[ultra thick, color=penColor, smooth, domain=(0:3)] {cos(deg(2*x))+2} node [pos=.95, below] {A};
          \addplot[thick, color=penColor, smooth, domain=(0:3)] {-2*sin(deg(2*x))} node [pos=.35, below right] {B};
          \addplot[very thick, color=penColor, smooth, domain=(0:3)] {1/2*sin(deg(x))+2*x} node [pos=.55, right] {C};
      \end{axis}
  \end{tikzpicture}
  \end{image}

  Which curve is which: $s=\answer[format=string]{c}$; $v=\answer[format=string]{a}$; $a=\answer[format=string]{b}$
  \end{exercise}

%Exercise 25
\begin{exercise}
Consider an object moving along a line.  The graph of the velocity, $v$, of the object is shown in the figure below.

\begin{image}
\begin{tikzpicture}
    \begin{axis}[
            xmin=0, xmax=4, ymin=-1,ymax=1,
            unit vector ratio*=1 1 1,
            axis lines =middle, xlabel=$x$, ylabel=$y$,
            every axis y label/.style={at=(current axis.above origin),anchor=south},
            every axis x label/.style={at=(current axis.right of origin),anchor=west},
            xtick={0,...,4}, ytick={-1,0,1},
          ]
        \addplot[ultra thick, color=penColor, smooth, domain=(0:2)] {(x-1)^3};
        \addplot[ultra thick, color=penColor, smooth, domain=(2:4)] {e^(2*(-x+2))};
    \end{axis}
\end{tikzpicture}
\end{image}

Select the graph of $s(t)$ given that $s(0)=0$

\begin{multipleChoice}

\choice{
\begin{image}
\begin{tikzpicture}
    \begin{axis}[
            xmin=0, xmax=4, ymin=-1,ymax=1,
            unit vector ratio*=1 1 1,
            axis lines =middle, xlabel=$x$, ylabel=$y$,
            every axis y label/.style={at=(current axis.above origin),anchor=south},
            every axis x label/.style={at=(current axis.right of origin),anchor=west},
            xtick={0,...,4}, ytick={-1,0,1},
          ]
        \addplot[ultra thick, color=penColor, smooth, domain=(0:2)] {-1/4*(x-1)^4+1/4};
        \addplot[ultra thick, color=penColor, smooth, domain=(2:4)] {-1/2*e^(2*(-x+2))+1/2};
    \end{axis}
\end{tikzpicture}
\end{image}
}

\choice[correct]{
\begin{image}
\begin{tikzpicture}
    \begin{axis}[
            xmin=0, xmax=4, ymin=-1,ymax=1,
            unit vector ratio*=1 1 1,
            axis lines =middle, xlabel=$x$, ylabel=$y$,
            every axis y label/.style={at=(current axis.above origin),anchor=south},
            every axis x label/.style={at=(current axis.right of origin),anchor=west},
            xtick={0,...,4}, ytick={-1,0,1},
          ]
        \addplot[ultra thick, color=penColor, smooth, domain=(0:2)] {1/4*(x-1)^4-1/4};
        \addplot[ultra thick, color=penColor, smooth, domain=(2:4)] {-1/2*e^(2*(-x+2))+1/2};
    \end{axis}
\end{tikzpicture}
\end{image}
}

\choice{
\begin{image}
\begin{tikzpicture}
    \begin{axis}[
            xmin=0, xmax=4, ymin=-1,ymax=1,
            unit vector ratio*=1 1 1,
            axis lines =middle, xlabel=$x$, ylabel=$y$,
            every axis y label/.style={at=(current axis.above origin),anchor=south},
            every axis x label/.style={at=(current axis.right of origin),anchor=west},
            xtick={0,...,4}, ytick={-1,0,1},
          ]
        \addplot[ultra thick, color=penColor, smooth, domain=(0:2)] {x/2};
        \addplot[ultra thick, color=penColor, smooth, domain=(2:4)] {2-x/2};
    \end{axis}
\end{tikzpicture}
\end{image}
}

\end{multipleChoice}

\end{exercise}
\end{document}
