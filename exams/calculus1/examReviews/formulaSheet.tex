\documentclass{ximera}
%\usepackage{todonotes}
%\usepackage{mathtools} %% Required for wide table Curl and Greens
%\usepackage{cuted} %% Required for wide table Curl and Greens
\newcommand{\todo}{}

\usepackage{esint} % for \oiint
\ifxake%%https://math.meta.stackexchange.com/questions/9973/how-do-you-render-a-closed-surface-double-integral
\renewcommand{\oiint}{{\large\bigcirc}\kern-1.56em\iint}
\fi


\graphicspath{
  {./}
  {ximeraTutorial/}
  {basicPhilosophy/}
  {functionsOfSeveralVariables/}
  {normalVectors/}
  {lagrangeMultipliers/}
  {vectorFields/}
  {greensTheorem/}
  {shapeOfThingsToCome/}
  {dotProducts/}
  {partialDerivativesAndTheGradientVector/}
  {../productAndQuotientRules/exercises/}
  {../normalVectors/exercisesParametricPlots/}
  {../continuityOfFunctionsOfSeveralVariables/exercises/}
  {../partialDerivativesAndTheGradientVector/exercises/}
  {../directionalDerivativeAndChainRule/exercises/}
  {../commonCoordinates/exercisesCylindricalCoordinates/}
  {../commonCoordinates/exercisesSphericalCoordinates/}
  {../greensTheorem/exercisesCurlAndLineIntegrals/}
  {../greensTheorem/exercisesDivergenceAndLineIntegrals/}
  {../shapeOfThingsToCome/exercisesDivergenceTheorem/}
  {../greensTheorem/}
  {../shapeOfThingsToCome/}
  {../separableDifferentialEquations/exercises/}
  {vectorFields/}
}

\newcommand{\mooculus}{\textsf{\textbf{MOOC}\textnormal{\textsf{ULUS}}}}

\usepackage{tkz-euclide}\usepackage{tikz}
\usepackage{tikz-cd}
\usetikzlibrary{arrows}
\tikzset{>=stealth,commutative diagrams/.cd,
  arrow style=tikz,diagrams={>=stealth}} %% cool arrow head
\tikzset{shorten <>/.style={ shorten >=#1, shorten <=#1 } } %% allows shorter vectors

\usetikzlibrary{backgrounds} %% for boxes around graphs
\usetikzlibrary{shapes,positioning}  %% Clouds and stars
\usetikzlibrary{matrix} %% for matrix
\usepgfplotslibrary{polar} %% for polar plots
\usepgfplotslibrary{fillbetween} %% to shade area between curves in TikZ
\usetkzobj{all}
\usepackage[makeroom]{cancel} %% for strike outs
%\usepackage{mathtools} %% for pretty underbrace % Breaks Ximera
%\usepackage{multicol}
\usepackage{pgffor} %% required for integral for loops



%% http://tex.stackexchange.com/questions/66490/drawing-a-tikz-arc-specifying-the-center
%% Draws beach ball
\tikzset{pics/carc/.style args={#1:#2:#3}{code={\draw[pic actions] (#1:#3) arc(#1:#2:#3);}}}



\usepackage{array}
\setlength{\extrarowheight}{+.1cm}
\newdimen\digitwidth
\settowidth\digitwidth{9}
\def\divrule#1#2{
\noalign{\moveright#1\digitwidth
\vbox{\hrule width#2\digitwidth}}}





\newcommand{\RR}{\mathbb R}
\newcommand{\R}{\mathbb R}
\newcommand{\N}{\mathbb N}
\newcommand{\Z}{\mathbb Z}

\newcommand{\sagemath}{\textsf{SageMath}}


%\renewcommand{\d}{\,d\!}
\renewcommand{\d}{\mathop{}\!d}
\newcommand{\dd}[2][]{\frac{\d #1}{\d #2}}
\newcommand{\pp}[2][]{\frac{\partial #1}{\partial #2}}
\renewcommand{\l}{\ell}
\newcommand{\ddx}{\frac{d}{\d x}}

\newcommand{\zeroOverZero}{\ensuremath{\boldsymbol{\tfrac{0}{0}}}}
\newcommand{\inftyOverInfty}{\ensuremath{\boldsymbol{\tfrac{\infty}{\infty}}}}
\newcommand{\zeroOverInfty}{\ensuremath{\boldsymbol{\tfrac{0}{\infty}}}}
\newcommand{\zeroTimesInfty}{\ensuremath{\small\boldsymbol{0\cdot \infty}}}
\newcommand{\inftyMinusInfty}{\ensuremath{\small\boldsymbol{\infty - \infty}}}
\newcommand{\oneToInfty}{\ensuremath{\boldsymbol{1^\infty}}}
\newcommand{\zeroToZero}{\ensuremath{\boldsymbol{0^0}}}
\newcommand{\inftyToZero}{\ensuremath{\boldsymbol{\infty^0}}}



\newcommand{\numOverZero}{\ensuremath{\boldsymbol{\tfrac{\#}{0}}}}
\newcommand{\dfn}{\textbf}
%\newcommand{\unit}{\,\mathrm}
\newcommand{\unit}{\mathop{}\!\mathrm}
\newcommand{\eval}[1]{\bigg[ #1 \bigg]}
\newcommand{\seq}[1]{\left( #1 \right)}
\renewcommand{\epsilon}{\varepsilon}
\renewcommand{\phi}{\varphi}


\renewcommand{\iff}{\Leftrightarrow}

\DeclareMathOperator{\arccot}{arccot}
\DeclareMathOperator{\arcsec}{arcsec}
\DeclareMathOperator{\arccsc}{arccsc}
\DeclareMathOperator{\si}{Si}
\DeclareMathOperator{\scal}{scal}
\DeclareMathOperator{\sign}{sign}


%% \newcommand{\tightoverset}[2]{% for arrow vec
%%   \mathop{#2}\limits^{\vbox to -.5ex{\kern-0.75ex\hbox{$#1$}\vss}}}
\newcommand{\arrowvec}[1]{{\overset{\rightharpoonup}{#1}}}
%\renewcommand{\vec}[1]{\arrowvec{\mathbf{#1}}}
\renewcommand{\vec}[1]{{\overset{\boldsymbol{\rightharpoonup}}{\mathbf{#1}}}\hspace{0in}}

\newcommand{\point}[1]{\left(#1\right)} %this allows \vector{ to be changed to \vector{ with a quick find and replace
\newcommand{\pt}[1]{\mathbf{#1}} %this allows \vec{ to be changed to \vec{ with a quick find and replace
\newcommand{\Lim}[2]{\lim_{\point{#1} \to \point{#2}}} %Bart, I changed this to point since I want to use it.  It runs through both of the exercise and exerciseE files in limits section, which is why it was in each document to start with.

\DeclareMathOperator{\proj}{\mathbf{proj}}
\newcommand{\veci}{{\boldsymbol{\hat{\imath}}}}
\newcommand{\vecj}{{\boldsymbol{\hat{\jmath}}}}
\newcommand{\veck}{{\boldsymbol{\hat{k}}}}
\newcommand{\vecl}{\vec{\boldsymbol{\l}}}
\newcommand{\uvec}[1]{\mathbf{\hat{#1}}}
\newcommand{\utan}{\mathbf{\hat{t}}}
\newcommand{\unormal}{\mathbf{\hat{n}}}
\newcommand{\ubinormal}{\mathbf{\hat{b}}}

\newcommand{\dotp}{\bullet}
\newcommand{\cross}{\boldsymbol\times}
\newcommand{\grad}{\boldsymbol\nabla}
\newcommand{\divergence}{\grad\dotp}
\newcommand{\curl}{\grad\cross}
%\DeclareMathOperator{\divergence}{divergence}
%\DeclareMathOperator{\curl}[1]{\grad\cross #1}
\newcommand{\lto}{\mathop{\longrightarrow\,}\limits}

\renewcommand{\bar}{\overline}

\colorlet{textColor}{black}
\colorlet{background}{white}
\colorlet{penColor}{blue!50!black} % Color of a curve in a plot
\colorlet{penColor2}{red!50!black}% Color of a curve in a plot
\colorlet{penColor3}{red!50!blue} % Color of a curve in a plot
\colorlet{penColor4}{green!50!black} % Color of a curve in a plot
\colorlet{penColor5}{orange!80!black} % Color of a curve in a plot
\colorlet{penColor6}{yellow!70!black} % Color of a curve in a plot
\colorlet{fill1}{penColor!20} % Color of fill in a plot
\colorlet{fill2}{penColor2!20} % Color of fill in a plot
\colorlet{fillp}{fill1} % Color of positive area
\colorlet{filln}{penColor2!20} % Color of negative area
\colorlet{fill3}{penColor3!20} % Fill
\colorlet{fill4}{penColor4!20} % Fill
\colorlet{fill5}{penColor5!20} % Fill
\colorlet{gridColor}{gray!50} % Color of grid in a plot

\newcommand{\surfaceColor}{violet}
\newcommand{\surfaceColorTwo}{redyellow}
\newcommand{\sliceColor}{greenyellow}




\pgfmathdeclarefunction{gauss}{2}{% gives gaussian
  \pgfmathparse{1/(#2*sqrt(2*pi))*exp(-((x-#1)^2)/(2*#2^2))}%
}


%%%%%%%%%%%%%
%% Vectors
%%%%%%%%%%%%%

%% Simple horiz vectors
\renewcommand{\vector}[1]{\left\langle #1\right\rangle}


%% %% Complex Horiz Vectors with angle brackets
%% \makeatletter
%% \renewcommand{\vector}[2][ , ]{\left\langle%
%%   \def\nextitem{\def\nextitem{#1}}%
%%   \@for \el:=#2\do{\nextitem\el}\right\rangle%
%% }
%% \makeatother

%% %% Vertical Vectors
%% \def\vector#1{\begin{bmatrix}\vecListA#1,,\end{bmatrix}}
%% \def\vecListA#1,{\if,#1,\else #1\cr \expandafter \vecListA \fi}

%%%%%%%%%%%%%
%% End of vectors
%%%%%%%%%%%%%

%\newcommand{\fullwidth}{}
%\newcommand{\normalwidth}{}



%% makes a snazzy t-chart for evaluating functions
%\newenvironment{tchart}{\rowcolors{2}{}{background!90!textColor}\array}{\endarray}

%%This is to help with formatting on future title pages.
\newenvironment{sectionOutcomes}{}{}



%% Flowchart stuff
%\tikzstyle{startstop} = [rectangle, rounded corners, minimum width=3cm, minimum height=1cm,text centered, draw=black]
%\tikzstyle{question} = [rectangle, minimum width=3cm, minimum height=1cm, text centered, draw=black]
%\tikzstyle{decision} = [trapezium, trapezium left angle=70, trapezium right angle=110, minimum width=3cm, minimum height=1cm, text centered, draw=black]
%\tikzstyle{question} = [rectangle, rounded corners, minimum width=3cm, minimum height=1cm,text centered, draw=black]
%\tikzstyle{process} = [rectangle, minimum width=3cm, minimum height=1cm, text centered, draw=black]
%\tikzstyle{decision} = [trapezium, trapezium left angle=70, trapezium right angle=110, minimum width=3cm, minimum height=1cm, text centered, draw=black]

%\author{Tom Dinitz and Nela Lakos}
\license{Creative Commons 3.0 By-NC}
\title{Formula Sheet}

\begin{document}
\begin{abstract}
\end{abstract}
\maketitle
\begin{image}
\begin{tikzpicture}
%Pyth Thm
  \node at (1.5,4) [align=center] {\Large Pythagorean Theorem}; 
\draw (0,0) -- node[below,pos=.5]{$a$}(3,0) -- node[right,pos=.5]{$b$}(3,3)-- node[above left,pos=.5]{$c$}(0,0);
\draw (3,0) rectangle (2.75,.25);
\node at (1.5,-1) [align=center] {\large $c^2=a^2+b^2$};

\node at (7.25,4) [align=center] {\Large Distance Between\\ \Large Two Points}; 
\draw (5.5,.5) -- (9,.5);
\draw (6,0) -- (6,3);
 \fill[fill=black] (7,1) circle (1.5pt) node[below]{$(x,y)$};
 \fill[fill=black] (8.5,2) circle (1.5pt) node[below]{$(a,b)$};
\node at (7.25,-1) [align=center] {\large D=$\sqrt{(a-x)^2+(b-y)^2}$};
\end{tikzpicture}
\end{image}
PERIMETERS AND AREAS
\begin{image}
  \begin{tikzpicture}
   %CIRCLE
   \node at (1.5,11.5) [align=center] {\Large CIRCLE};
    \draw [ultra thick,domain=0:360] plot ({1.5*cos(\x)+1.5}, {1.5*sin(\x)+9.5});
    \draw [ultra thick](1.5,9.5) -- (3,9.5) node[pos=0.5, above]{\LARGE r};
    \node at (1.5,7) [align=center] {\Large $P=2\pi r$\\\\ \Large $A=\pi r^2$};

    %\draw [penColor,ultra thick,domain=270:360] plot ({2*cos(\x)+8}, {2*sin(\x)+11});
    %\draw [penColor,ultra thick,domain=0:90] plot ({2*cos(\x)+2}, {2*sin(\x)+3});
    %\draw [penColor,ultra thick,domain=180:90] plot ({2*cos(\x)+10}, {2*sin(\x)+3});
 
    %TRIANGLE    
	\node at (7.5,11.5) [align=center] {\Large TRIANGLE};
	\draw [ultra thick] (6,8.5) -- (9,8.5) node[pos=0.5, below]{\LARGE b};
 	\draw [ultra thick] (6,8.5) -- (7,11) node[pos=0.5, left]{\LARGE a};
 	\draw [ultra thick]  (7,11) -- (9,8.5) node[pos=0.5, right]{\LARGE c};
	\draw [ultra thick, dashed]  (7,8.5) -- (7,11) node[pos=0.5, right]{\LARGE h};
 	\node at (7.5,7) [align=center] {\Large $P=a+b+c$\\\\ \Large $A=\frac{1}{2}b
	\cdot h$};
	

    %RECTANGLE
        \node at (1.5,4.75) [align=center] {\Large RECTANGLE};
	\draw [ultra thick] (0,2) -- (3,2) node[pos=0.5, below]{\LARGE a};
 	\draw [ultra thick] (3,2) -- (3,4) node[pos=0.5, right]{\LARGE b};
 	\draw[ultra thick]  (3,4) -- (0,4);
	\draw [ultra thick] (0,4) -- (0,2);
	
	\node at (1.5,1) [align=center] {\Large $P=2a+2b$\\ \\\Large $A=a\cdot b$};
	
	%TRAPEZOID
	\node at (7.5,4.75) [align=center] {\Large TRAPEZOID};
	\draw [ultra thick] (6,2) -- (9,2) node[pos=0.5, below]{\LARGE a};
 	\draw [ultra thick] (9,2) -- (8.4,4) node[pos=0.5, right]{\LARGE b};
 	\draw[ultra thick]  (8.4,4) -- (7,4)node[pos=0.5, above]{\LARGE c};
	\draw [ultra thick] (7,4) -- (6,2)node[pos=0.5, left]{\LARGE d};
	\draw[dashed,ultra thick]  (7,2) -- (7,4) node[pos=0.5,right] {\LARGE h};
	\node at (7.5,0.5) [align=center] {\Large $P=a+b+c+d$\\ \\\Large $A=\frac{a+c}{2}\cdot h$};
	
 \end{tikzpicture}
 \end{image}
 \pagebreak
 VOLUMES AND SURFACE AREAS
  \begin{image}
\begin{tikzpicture}
%SPHERE
  \node at (1.5,2.5) [align=center] { \small{SPHERE}};
  \shade[ball color = gray!40, opacity = 0.4] (1.5,0) circle (1.5);
  %\node at (0,0) [align=center]{a};
  \draw (1.5,0) circle (1.5cm);
  \draw (0,0) arc (180:360:1.5 and 0.6);
  \draw[dashed] (3,0) arc (0:180:1.5 and 0.6);
  \fill[fill=black] (1.5,0) circle (1pt);
  \draw[dashed] (1.5,0 ) -- node[above]{$r$} (3,0);
  
  \node at (1.5,-2.25) [align=center] {\large $V=\frac{4}{3}\pi r^3$\\ \\\large $SA=4\pi r^2$};
  
  %CYLINDER
  \node at (7,2.5) [align=center] { \small{CYLINDER}};
\draw [fill=gray,opacity=.5,thick](7,1.5) ellipse (1.25 and 0.5);
\draw (5.75,1.5) -- (5.75,-1);
\draw (5.75,-1) arc (180:360:1.25 and 0.5);
\draw [dashed] (5.75,-1) arc (180:360:1.25 and -0.5);
\draw (8.25,1.5) -- (8.25,-1) node[right,pos=.5]{$h$};  
\fill [gray,opacity=0.5] (5.75,1.5) -- (5.75,-1) arc (180:360:1.25 and 0.5) -- (8.25,1.5) arc (0:180:1.25 and -0.5);
\draw[dashed](5.75+1.25,-1) -- (8.25,-1) node[above,pos=.5] {$r$};
\node at (7,-2.25) [align=center] {\large $V=\pi r^2h$\\ \\\large $SA=2\pi r^2+2\pi r\cdot h$};
\end{tikzpicture}
\end{image}

\begin{image}
\begin{tikzpicture}

%BOX
  \node at (2.5,.5) [align=center] { \small{RECTANGULAR BOX}};
\pgfmathsetmacro{\xr}{3.5}
\pgfmathsetmacro{\xl}{1.5}
\pgfmathsetmacro{\yt}{0}
\pgfmathsetmacro{\yb}{-1.5}
\pgfmathsetmacro{\zi}{0}
\pgfmathsetmacro{\zo}{1}
  \draw[thick](\xr,\yt,0)--(\xl,\yt,\zi)--(\xl,\yt,\zo)--(\xr,\yt,\zo)--(\xr,\yt,\zi)--(\xr,\yb,\zi)--(\xr,\yb,\zo)--(\xl,\yb,\zo)--(\xl,\yt,\zo);
  \draw[thick](\xr,\yt,\zo)--(\xr,\yb,\zo);
  \draw[dashed](\xr,\yb,\zi)--(\xl,\yb,\zi)--(\xl,\yt,\zi);
  \draw[dashed](\xl,\yb,\zi)--(\xl,\yb,\zo);
  \draw(\xr+.25,\yt*.5+\yb*.5,\zi) node{$h$};
  \draw(\xr+.25,\yb,\zi*.5+\zo*.5) node{$w$};
  \draw(\xl*.5+\xr*.5,\yb-.25,\zo) node{$l$};
\node at (2,-3) [align=center] {\large $V=lwh$\\ \\\large $SA=2lw+2wh+2lh$};  
  
  %CUBE
    \node at (8,.5) [align=center] { \small{CUBE}};
  \pgfmathsetmacro{\cxr}{9}
\pgfmathsetmacro{\cxl}{7.5}
\pgfmathsetmacro{\cyt}{0}
\pgfmathsetmacro{\cyb}{-1.5}
\pgfmathsetmacro{\czi}{0}
\pgfmathsetmacro{\czo}{1.5}
  \draw[thick](\cxr,\cyt,0)--(\cxl,\cyt,\czi)--(\cxl,\cyt,\czo)--(\cxr,\cyt,\czo)--(\cxr,\cyt,\czi)--(\cxr,\cyb,\czi)--(\cxr,\cyb,\czo)--(\cxl,\cyb,\czo)--(\cxl,\cyt,\czo);
  \draw[thick](\cxr,\cyt,\czo)--(\cxr,\cyb,\czo);
  \draw[dashed](\cxr,\cyb,\czi)--(\cxl,\cyb,\czi)--(\cxl,\cyt,\czi);
  \draw[dashed](\cxl,\cyb,\czi)--(\cxl,\cyb,\czo);
  \draw(\cxr+.25,\cyt*.5+\cyb*.5,\czi) node{$a$};
  \draw(\cxr+.25,\cyb,\czi*.5+\czo*.5) node{$a$};
  \draw(\cxl*.5+\cxr*.5,\cyb-.25,\czo) node{$a$};
\node at (8,-3) [align=center] {\large $V=a^3$\\\\ \large $SA=6a^2$};  
 \end{tikzpicture}
  \end{image}

\begin{image}
\begin{tikzpicture}
%PYRAMID
  \node at (1.5,3.25) [align=center] { \small{PYRAMID}};
    \draw (2,3,1) -- (1,0,0) -- (3,0,0) -- (3,0,2) -- (2,3,1) 
      -- (3,0,0);
    \draw (2,3,1) -- (1,0,2) 
      -- (1,0,0);
    \draw(1,0,2) -- (3,0,2);
    \draw (5.6,-.2,2);
    \draw[dashed](2,3,1) -- node[left,pos=.7] {$h$} (2,0,1);

 \node at (1.5,-1.75) [align=center] {\large B= area of base\\\\ \large $V=\frac{1}{3}Bh$}; 
 
%CONE 
  \node at (6.5,3.25) [align=center] { \small{CONE}}; 
\pgfmathsetmacro{\ys}{-.25}
\draw 
  (5,0+\ys) arc (180:360:1.5 and .5) -- (6.5,3+\ys) -- cycle;
\draw[dashed]
  (5,0+\ys) arc (180:0:1.5 and .5);
\draw[dashed]
  (8,0+\ys) -- node[below] {$r$} (6.5,0+\ys) -- node[left] {$h$} (6.5,3+\ys) ;
  \node at (6.5,-1.5+\ys) [align=center] {\large $B= \pi r^2$\\ \\\large $V=\frac{1}{3}\pi r^2h$}; 
\end{tikzpicture}
\end{image}
\pagebreak
\begin{center}
\Large {Trig Functions of Some Special Angles}
\end{center}
\[
{\LARGE{\begin{tabular}{|| c || c || c || c || c || c ||}\hline\hline
\color{blue}{ANGLE}& \color{blue}{0} & \color{blue}{$\pi / 6$} & \color{blue}{$\pi / 4$} & \color{blue}{$\pi / 3$} & \color{blue}{$\pi / 2$} \\ \hline\hline
 $\sin(\theta)$ & 0 & $1 / 2$ & $1/\sqrt{2}$& $\sqrt{3}/2$ &1\\

\hline
$\cos(\theta)$ & 1 & $\sqrt{3}/2$ & $1/\sqrt{2}$& $1/2$ &0\\

\hline 
$\tan(\theta)$ & 0 & $1/\sqrt{3}$ & 1& $\sqrt{3}$ & DNE \\

\hline 
$\cot(\theta)$ & DNE & $\sqrt{3}$ & 1& $1/\sqrt{3}$ & 0 \\\hline
$\sec(\theta)$ & 1 & $2/\sqrt{3}$ & $\sqrt{2}$& $2$ &DNE \\ \hline
$\csc(\theta)$ & DNE & $2$ & $\sqrt{2}$& $2/\sqrt{3}$ &$1$\\\hline 
\end{tabular}}}
\]
\pagebreak
\begin{center}
{\Large Famous Trig Identities}
\end{center}


$\tan(\theta)=\frac{\sin(\theta)}{\cos(\theta)}$, \text{          }$\cot(\theta)=\frac{\cos(\theta)}{\sin(\theta)}$\\

$\sec(\theta)=\frac{1}{\cos(\theta)}$, \text{           }$\csc(\theta)=\frac{1}{\sin(\theta)}$\\

$\cos^2(\theta)+\sin^2(\theta)=1$\\

$1+\tan^2(\theta)=\sec^2(\theta)$\\\\



$\sin(\alpha+\beta)=\sin(\alpha)\cos(\beta)+\cos(\alpha)\sin(\beta)$\\

$\sin(\alpha-\beta)=\sin(\alpha)\cos(\beta)-\cos(\alpha)\sin(\beta)$\\

$\cos(\alpha+\beta)=\cos(\alpha)\cos(\beta)-\sin(\alpha)\sin(\beta)$\\

$\cos(\alpha-\beta)=\cos(\alpha)\cos(\beta)+\sin(\alpha)\sin(\beta)$\\\\




$\tan(\alpha+\beta)=\frac{\tan(\alpha)+\tan(\beta)}{1-\tan(\alpha)\tan(\beta)}$\\\\\\
$\tan(\alpha-\beta)=\frac{\tan(\alpha)-\tan(\beta)}{1+\tan(\alpha)\tan(\beta)}$\\\\\\



$\sin(2\theta)=2\sin(\theta)\cos(\theta)$\\

$\cos(2\theta)=\cos^2(\theta)-\sin^2(\theta) = 2\cos^2(\theta)-1 = 1-2\sin^2(\theta)$\\\\

$\sin\left(\frac{\theta}{2}\right)=\pm\sqrt{\frac{1-\cos(\theta)}{2}}$\\

$\cos\left(\frac{\theta}{2}\right)=\pm\sqrt{\frac{1+\cos(\theta)}{2}}$\\

$\tan\left(\frac{\theta}{2}\right)=\frac{\sin(\theta)}{1+\cos(\theta)}=\frac{1-\cos(\theta)}{\sin(\theta)}$

\end{document}