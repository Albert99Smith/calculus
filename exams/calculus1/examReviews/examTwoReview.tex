\documentclass{ximera}
%\usepackage{todonotes}
%\usepackage{mathtools} %% Required for wide table Curl and Greens
%\usepackage{cuted} %% Required for wide table Curl and Greens
\newcommand{\todo}{}

\usepackage{esint} % for \oiint
\ifxake%%https://math.meta.stackexchange.com/questions/9973/how-do-you-render-a-closed-surface-double-integral
\renewcommand{\oiint}{{\large\bigcirc}\kern-1.56em\iint}
\fi


\graphicspath{
  {./}
  {ximeraTutorial/}
  {basicPhilosophy/}
  {functionsOfSeveralVariables/}
  {normalVectors/}
  {lagrangeMultipliers/}
  {vectorFields/}
  {greensTheorem/}
  {shapeOfThingsToCome/}
  {dotProducts/}
  {partialDerivativesAndTheGradientVector/}
  {../productAndQuotientRules/exercises/}
  {../normalVectors/exercisesParametricPlots/}
  {../continuityOfFunctionsOfSeveralVariables/exercises/}
  {../partialDerivativesAndTheGradientVector/exercises/}
  {../directionalDerivativeAndChainRule/exercises/}
  {../commonCoordinates/exercisesCylindricalCoordinates/}
  {../commonCoordinates/exercisesSphericalCoordinates/}
  {../greensTheorem/exercisesCurlAndLineIntegrals/}
  {../greensTheorem/exercisesDivergenceAndLineIntegrals/}
  {../shapeOfThingsToCome/exercisesDivergenceTheorem/}
  {../greensTheorem/}
  {../shapeOfThingsToCome/}
  {../separableDifferentialEquations/exercises/}
  {vectorFields/}
}

\newcommand{\mooculus}{\textsf{\textbf{MOOC}\textnormal{\textsf{ULUS}}}}

\usepackage{tkz-euclide}\usepackage{tikz}
\usepackage{tikz-cd}
\usetikzlibrary{arrows}
\tikzset{>=stealth,commutative diagrams/.cd,
  arrow style=tikz,diagrams={>=stealth}} %% cool arrow head
\tikzset{shorten <>/.style={ shorten >=#1, shorten <=#1 } } %% allows shorter vectors

\usetikzlibrary{backgrounds} %% for boxes around graphs
\usetikzlibrary{shapes,positioning}  %% Clouds and stars
\usetikzlibrary{matrix} %% for matrix
\usepgfplotslibrary{polar} %% for polar plots
\usepgfplotslibrary{fillbetween} %% to shade area between curves in TikZ
\usetkzobj{all}
\usepackage[makeroom]{cancel} %% for strike outs
%\usepackage{mathtools} %% for pretty underbrace % Breaks Ximera
%\usepackage{multicol}
\usepackage{pgffor} %% required for integral for loops



%% http://tex.stackexchange.com/questions/66490/drawing-a-tikz-arc-specifying-the-center
%% Draws beach ball
\tikzset{pics/carc/.style args={#1:#2:#3}{code={\draw[pic actions] (#1:#3) arc(#1:#2:#3);}}}



\usepackage{array}
\setlength{\extrarowheight}{+.1cm}
\newdimen\digitwidth
\settowidth\digitwidth{9}
\def\divrule#1#2{
\noalign{\moveright#1\digitwidth
\vbox{\hrule width#2\digitwidth}}}





\newcommand{\RR}{\mathbb R}
\newcommand{\R}{\mathbb R}
\newcommand{\N}{\mathbb N}
\newcommand{\Z}{\mathbb Z}

\newcommand{\sagemath}{\textsf{SageMath}}


%\renewcommand{\d}{\,d\!}
\renewcommand{\d}{\mathop{}\!d}
\newcommand{\dd}[2][]{\frac{\d #1}{\d #2}}
\newcommand{\pp}[2][]{\frac{\partial #1}{\partial #2}}
\renewcommand{\l}{\ell}
\newcommand{\ddx}{\frac{d}{\d x}}

\newcommand{\zeroOverZero}{\ensuremath{\boldsymbol{\tfrac{0}{0}}}}
\newcommand{\inftyOverInfty}{\ensuremath{\boldsymbol{\tfrac{\infty}{\infty}}}}
\newcommand{\zeroOverInfty}{\ensuremath{\boldsymbol{\tfrac{0}{\infty}}}}
\newcommand{\zeroTimesInfty}{\ensuremath{\small\boldsymbol{0\cdot \infty}}}
\newcommand{\inftyMinusInfty}{\ensuremath{\small\boldsymbol{\infty - \infty}}}
\newcommand{\oneToInfty}{\ensuremath{\boldsymbol{1^\infty}}}
\newcommand{\zeroToZero}{\ensuremath{\boldsymbol{0^0}}}
\newcommand{\inftyToZero}{\ensuremath{\boldsymbol{\infty^0}}}



\newcommand{\numOverZero}{\ensuremath{\boldsymbol{\tfrac{\#}{0}}}}
\newcommand{\dfn}{\textbf}
%\newcommand{\unit}{\,\mathrm}
\newcommand{\unit}{\mathop{}\!\mathrm}
\newcommand{\eval}[1]{\bigg[ #1 \bigg]}
\newcommand{\seq}[1]{\left( #1 \right)}
\renewcommand{\epsilon}{\varepsilon}
\renewcommand{\phi}{\varphi}


\renewcommand{\iff}{\Leftrightarrow}

\DeclareMathOperator{\arccot}{arccot}
\DeclareMathOperator{\arcsec}{arcsec}
\DeclareMathOperator{\arccsc}{arccsc}
\DeclareMathOperator{\si}{Si}
\DeclareMathOperator{\scal}{scal}
\DeclareMathOperator{\sign}{sign}


%% \newcommand{\tightoverset}[2]{% for arrow vec
%%   \mathop{#2}\limits^{\vbox to -.5ex{\kern-0.75ex\hbox{$#1$}\vss}}}
\newcommand{\arrowvec}[1]{{\overset{\rightharpoonup}{#1}}}
%\renewcommand{\vec}[1]{\arrowvec{\mathbf{#1}}}
\renewcommand{\vec}[1]{{\overset{\boldsymbol{\rightharpoonup}}{\mathbf{#1}}}\hspace{0in}}

\newcommand{\point}[1]{\left(#1\right)} %this allows \vector{ to be changed to \vector{ with a quick find and replace
\newcommand{\pt}[1]{\mathbf{#1}} %this allows \vec{ to be changed to \vec{ with a quick find and replace
\newcommand{\Lim}[2]{\lim_{\point{#1} \to \point{#2}}} %Bart, I changed this to point since I want to use it.  It runs through both of the exercise and exerciseE files in limits section, which is why it was in each document to start with.

\DeclareMathOperator{\proj}{\mathbf{proj}}
\newcommand{\veci}{{\boldsymbol{\hat{\imath}}}}
\newcommand{\vecj}{{\boldsymbol{\hat{\jmath}}}}
\newcommand{\veck}{{\boldsymbol{\hat{k}}}}
\newcommand{\vecl}{\vec{\boldsymbol{\l}}}
\newcommand{\uvec}[1]{\mathbf{\hat{#1}}}
\newcommand{\utan}{\mathbf{\hat{t}}}
\newcommand{\unormal}{\mathbf{\hat{n}}}
\newcommand{\ubinormal}{\mathbf{\hat{b}}}

\newcommand{\dotp}{\bullet}
\newcommand{\cross}{\boldsymbol\times}
\newcommand{\grad}{\boldsymbol\nabla}
\newcommand{\divergence}{\grad\dotp}
\newcommand{\curl}{\grad\cross}
%\DeclareMathOperator{\divergence}{divergence}
%\DeclareMathOperator{\curl}[1]{\grad\cross #1}
\newcommand{\lto}{\mathop{\longrightarrow\,}\limits}

\renewcommand{\bar}{\overline}

\colorlet{textColor}{black}
\colorlet{background}{white}
\colorlet{penColor}{blue!50!black} % Color of a curve in a plot
\colorlet{penColor2}{red!50!black}% Color of a curve in a plot
\colorlet{penColor3}{red!50!blue} % Color of a curve in a plot
\colorlet{penColor4}{green!50!black} % Color of a curve in a plot
\colorlet{penColor5}{orange!80!black} % Color of a curve in a plot
\colorlet{penColor6}{yellow!70!black} % Color of a curve in a plot
\colorlet{fill1}{penColor!20} % Color of fill in a plot
\colorlet{fill2}{penColor2!20} % Color of fill in a plot
\colorlet{fillp}{fill1} % Color of positive area
\colorlet{filln}{penColor2!20} % Color of negative area
\colorlet{fill3}{penColor3!20} % Fill
\colorlet{fill4}{penColor4!20} % Fill
\colorlet{fill5}{penColor5!20} % Fill
\colorlet{gridColor}{gray!50} % Color of grid in a plot

\newcommand{\surfaceColor}{violet}
\newcommand{\surfaceColorTwo}{redyellow}
\newcommand{\sliceColor}{greenyellow}




\pgfmathdeclarefunction{gauss}{2}{% gives gaussian
  \pgfmathparse{1/(#2*sqrt(2*pi))*exp(-((x-#1)^2)/(2*#2^2))}%
}


%%%%%%%%%%%%%
%% Vectors
%%%%%%%%%%%%%

%% Simple horiz vectors
\renewcommand{\vector}[1]{\left\langle #1\right\rangle}


%% %% Complex Horiz Vectors with angle brackets
%% \makeatletter
%% \renewcommand{\vector}[2][ , ]{\left\langle%
%%   \def\nextitem{\def\nextitem{#1}}%
%%   \@for \el:=#2\do{\nextitem\el}\right\rangle%
%% }
%% \makeatother

%% %% Vertical Vectors
%% \def\vector#1{\begin{bmatrix}\vecListA#1,,\end{bmatrix}}
%% \def\vecListA#1,{\if,#1,\else #1\cr \expandafter \vecListA \fi}

%%%%%%%%%%%%%
%% End of vectors
%%%%%%%%%%%%%

%\newcommand{\fullwidth}{}
%\newcommand{\normalwidth}{}



%% makes a snazzy t-chart for evaluating functions
%\newenvironment{tchart}{\rowcolors{2}{}{background!90!textColor}\array}{\endarray}

%%This is to help with formatting on future title pages.
\newenvironment{sectionOutcomes}{}{}



%% Flowchart stuff
%\tikzstyle{startstop} = [rectangle, rounded corners, minimum width=3cm, minimum height=1cm,text centered, draw=black]
%\tikzstyle{question} = [rectangle, minimum width=3cm, minimum height=1cm, text centered, draw=black]
%\tikzstyle{decision} = [trapezium, trapezium left angle=70, trapezium right angle=110, minimum width=3cm, minimum height=1cm, text centered, draw=black]
%\tikzstyle{question} = [rectangle, rounded corners, minimum width=3cm, minimum height=1cm,text centered, draw=black]
%\tikzstyle{process} = [rectangle, minimum width=3cm, minimum height=1cm, text centered, draw=black]
%\tikzstyle{decision} = [trapezium, trapezium left angle=70, trapezium right angle=110, minimum width=3cm, minimum height=1cm, text centered, draw=black]

\author{Tom Dinitz and Nela Lakos}
\license{Creative Commons 3.0 By-NC}
\title{Exam Two Review}

\begin{document}
\begin{abstract}
Review questions for exam 2.
\end{abstract}
\maketitle
%Exercise 1

\begin{exercise}
(a) Suppose $s(t)$ is the position of an object moving along a line at time $t\geq 0$. 

What is the average velocity between times $t=a$ and $t=b$?
\begin{prompt}
\begin{multipleChoice}
\choice{$s(b)-s(a)$}
\choice[correct]{$\frac{s(b)-s(a)}{b-a}$}
\choice{$\frac{b-a}{s(b)-s(a)}$}
\choice{$b-a$}
\end{multipleChoice}
\end{prompt}

(b)The table gives the position $s(t)$ of an object moving along a line at time $t$, over a three second interval.

\[
\begin{array}{|c|c|c|c|c|c|c|c|}
\hline
t& 0& 0.5& 1& 1.5& 2& 2.5& 3\\ \hline
s(t)& 0 & 22 & 32 & 48 & 54 & 64 & 74 \\ \hline
\end{array}
\]

The average velocity over the interval $[1,3]$ is $v_{av}=\begin{prompt}\answer{21}\end{prompt}$

(c) Make a conjecture about the value of the instantenous velcoty at $t=1$: $\begin{prompt}\answer{26}\end{prompt}$
\end{exercise}

%Exercise 2
\begin{exercise}
(a) Fill in the blanks: $\lim_{\answer[format=string]{h}\to\answer{0}}\frac{f(\answer{x+h})-f(\answer{x}}{\answer{h}}$

(b) Let $f(x)=\frac{1}{x-4}$. Using the (limit) defintion of the derivative in (a), compute $f'(x)=\answer{-(x-4)^{-2}}$
\end{exercise}

%Exercise 3
\begin{exercise}
The graph of a function $f$ is given below

INSERT GRAPH

(1) (a) Select the figure which has a secant line drawn through the points $(0,f(0))$ and $(4,f(4))$.

INSERT GRAPHS

(b) Select the figure which has the line tangent to the curve at $x=1$

INSERT GRAPHS

(2) Mutliple Choice:

(i) \begin{multipleChoice}
\choice{$f(1)=f(3)$}
\choice{$f(1)>f(3)$}
\choice[correct]{$f(1)<f(3)$} 
\end{multipleChoice}

(ii) \begin{multipleChoice}
\choice{$f'(1)=f'(3)$}
\choice[correct]{$f'(1)>f'(3)$} 
\choice{$f'(1)<f'(3)$} 
\end{multipleChoice}

(iii) Select the best approximation of $f'(3)$
\begin{multipleChoice}
\choice{$f'(3)\approx 2$}
\choice{$f'(3)\approx -2$}
\choice{$f'(3)\approx 1$}
\choice{$f'(3)\approx -1$}
\choice[correct]{$f'(3)\approx \frac{1}{2}$}
\choice{$f'(3)\approx -\frac{1}{2}}$  
\end{multipleChoice}
\end{exercise}

%Exercise 4
\begin{exercise}
Let $f$ be a function such that $f(4)=3$ and $f'(4)=-6$.

(a) Find the limit or say it does not exist (DNE).

$\lim_{x\to 4} \frac{f(x)-f(4)}{x-4}=\answer{-6}$

(b) An equation of the tangent line to the curve $y=f(x)$ at the point where $x=4$ is given by $y=\answer{-6x+27}$
\end{exercise}

%Exercise 5
\begin{exercise}
(i) Given $e^{xy}=e+x-y$, use implicit differentiation to find $\frac{dy}{dx}=\answer{\frac{1-ye^{xy}}{1+xe^{xy}}}$

(ii) Given y, find $\frac{dy}{dx}$:

(a) $y=\left(5\sin(x)\right)\arctan\left(\frac{x}{x+1}\right)$. $\frac{dy}{dx}=\answer{5\sin(x)\frac{(x+1)^{-1}-x(x+1)^{-2}}{1+\frac{x^2}{(x+1)^2}}+5\cos(x)\arctan(\frac{x}{x+1})}$

(b) $y=\frac{1}{\sqrt{\tan(x^2)+5)}}$. $\frac{dy}{dx}=\answer{\frac{-x\sec^2(x^2)}{(\tan(x^2)+5)^{1.5}}}$

(c) $y=xe^{ax}+\arcsin(ax)$. $\frac{dy}{dx}=\answer{e^{ax}+axe^{ax}+\frac{a}{\sqrt{1-a^2x^2}}}$

(d) $y(x)=\left(\sin(x)\right)^{ax}$. $\frac{dy}{dx}=\answer{\sin(x)^{ax}(ax\cot(x)+a\log(\sin(x)))}$

(ii) Find the following values or state the value is undefined (DNE):

(a) $\sin\left(\frac{17\pi}{6}\right)=\answer{\frac{1}{2}}$

(b) $\arcsin\left(\frac{17\pi}{6}\right)=\answer[format=string]{DNE}$

(c) $\arcsin\left(\frac{-\sqrt{3}}{2}\right)=\answer{-\frac{\pi}{3}}$

(d) $e^{-2\ln(3)}=\answer{3^{-2}}$

(e) $\ln\left(-e^2\right)=\answer[format=string]{DNE}$

(f) $\ln\left(e^{2t}\right)=\answer{2t}$

(g) $10^{3\log_10(4)}=\answer{4^3}$

(h) $\ln(1)=\answer{0}$

(i) $\arctan\left(-\sqrt{3}\right)=\answer{\frac{-\pi}{3}}$

(j) At $x=\frac{\pi}{3}$, $\ddx\left(\cot(x)\right)=\answer{-\frac{4}{3}}$

(k) At $x=-\sqrt{3}$, $\ddx\left(\arccot(x)\right)=\answer{-\frac{1}{4}}$
\end{exercise}

\begin{exercise}
A table of values for $f(x)$ and $f'(x)$, along with a graph of a function $g(x)$ is shown below.
\[
\begin{array}{c|c|c}
x & f(x)&f'(x)\\ \hline
1 & 2 & 4\\ \hline
2 & 3 & 5\\ \hline
3 & 4 & 1\\ \hline
\end{array}
\]
\begin{image}
\begin{tikzpicture}
    \begin{axis}[
            domain=0:6,
            ymax=5,
        ymin=-1,
            samples=100,
            axis lines =middle, xlabel=$x$, ylabel=$y$,
        ytick={1,2,3,4},
            every axis y label/.style={at=(current axis.above origin),anchor=south},
            every axis x label/.style={at=(current axis.right of origin),anchor=west}
          ]
          \addplot [very thick, penColor, smooth, domain=(0:1)] {3*x};
          \addplot [very thick, penColor, smooth, domain=(1:4)] {4-x};
      \addplot [very thick, penColor, smooth, domain=(4:6)] {(x-4)^2};
        \end{axis}
\end{tikzpicture}
\end{image}

(i) At $x=2$, $\ddx\left[f(x)g(x)\right]$=\begin{multipleChoice}
\choice{$\frac{13}{3}$}
\choice{$\frac{4}{3}$}
\choice{12}
\choice{4}
\choice[correct]{DNE}
\choice{None of the above}
\end{multipleChoice}

(ii) At $x=3$, $\ddx\left[f(g(x))\right]$=\begin{multipleChoice}
\choice{-1}
\choice[correct]{-4}
\choice{$\frac{1}{4}$}
\choice{4}
\choice{DNE}
\choice{None of the above}
\end{multipleChoice}

(iii) At $x=2$, $\ddx\left[g(f(x))\right]$=\begin{multipleChoice}
\choice{-1}
\choice[correct]{-5}
\choice{$\frac{1}{5}$}
\choice{0}
\choice{DNE}
\choice{None of the above}
\end{multipleChoice}

(iv) $f^{-1}(3)=$ \begin{multipleChoice}
\choice{4}
\choice{1}
\choice{$\frac{1}{3}$}
\choice[correct]{2}
\choice{DNE}
\choice{None of the above}
\end{multipleChoice}

(v) At $x=3$, $\ddx\left[f^{-1}(x)\right]$=\begin{multipleChoice}
\choice{$\frac{1}{4}$}
\choice[correct]{$\frac{1}{5}$}
\choice{2}
\choice{4}
\choice{DNE}
\choice{None of the above}
\end{multipleChoice}
\end{exercise}

%Exercise 7
\begin{exercise}
The function $f$ is defined by $f(x)=\frac{x}{\sqrt{x^2-9}}$

(a) The domain of $f$ is $(\answer{-\infty},\answer{-3})\cup (\answer{3},\answer{\infty})$
(b) The function $f$ is \begin{multipleChoice}
\choice[correct]{odd}
\choice{even}
\choice{neither}
\end{multipleChoice}

(c) Select all horizonal asymptotes
\begin{selectAll}
\choice{y=0}
\choice[correct]{y=1}
\choice{x=1}
\choice[correct]{y=-1}
\choice{There are no horizontal asymptotes}
\end{selectAll}

(d) Select all vertical asymptotes
\begin{selectAll}
\choice[correct]{x=-3}
\choice{y=1}
\choice[correct]{x=3}
\choice{y=3}
\choice{There are no vertical asymptotes}
\end{selectAll}  

(e) $f'(x)=\answer{\frac{-9}{(x^2-9)^{1.5}}}$

(f) $f''(x)=\answer{\frac{27x}{(x^2-9)^{2.5}}}$

(g) Fill in the blanks with the appropraite \underline{underlined} choices.

\begin{tabular}{|c|c|c|}\hline
 & f is \underline{increasing} & f is \underline{concave UP} \\
On this interval & f is \underline{decreasing} & f is \underline{concave DOWN} \\
 & f is \underline{Not Defined} & f is \underline{Not Defined} \\ \hline
$(-\infty,-3)$ & $\answer[format=string]{decreasing}$& $\answer[format=string]{concave DOWN}$\\ \hline
$(-3,3)$ & $\answer[format=string]{Not Defined}$& $\answer[format=string]{Not Defined}$\\ \hline
$(3,\infty)$ & $\answer[format=string]{decreasing}$& $\answer[format=string]{concave UP}$\\\hline 
\end{tabular}

(h) Select the correct graph for $f$:

INSERT GRAPHS

(i) Is the function $f$ one-to-one:
\begin{multipleChoice}
\choice[correct]{Yes}
\choice{No}
\end{multipleChoice}
\end{exercise}

%Exercise 8
\begin{exercise}
The graph of $f'$ (the derivative of $f$) on the interval $[-6,7]$ is shown in the figure.

INSERT GRAPH

Use the given graph of $f'$ to answer the following questions about f:

(a) On what interval(s) is $f$ decreasing: $[\answer{-6},\answer{-2}]$ and $[\answer{6},\answer{7}]$

(b) The critical points of $f$ are $x=\answer{-2},\answer{4},\answer{6}$

(c) Which critical points correspond to local maxima? $x=\answer{6}$

(d) Which critical poitns correspond to neither local maxima nor local minima? $x=\answer{4}$

(e) On what intervals is $f$ concave up? $[\answer{-6},\answer{0}]$ and $[\answer{4},\answer{5}]$
(f) At what points does $f$ have an inflection point? $x=\answer{0},\answer{4},\answer{6}$
\end{exercise}

%Exercise 9
\begin{exercise}
Given that $f(1)=2$, $f'(1)=3$, and $f''(1)=-1$, find the following values or state `cannot be determined':

(a) $\left(\ddx\frac{(x+5)f'(x)}{f(x)}\right)_{x=1}=\answer{\frac{-9}{4}}$

(b) $\lim_{x\to 1}f(x)=\answer{2}$

(c) $\left(\ddx f^{-1}(x)\right)_{x=2}=\answer{\frac{1}{3}}$

(d) $\lim_{h\to 0}\frac{f(1+h)-f(1)}{h}=\answer{3}$

\end{exercise}

%Exercise 10
\begin{exercise}
Given that $h(x)=\ln(\sec(x)+\tan(x))$, $f'(x) = \answer{\sec(x)}$
\end{exercise}

%Exercise 11

\begin{exercise}
The position, $s(t)$, of an object moving along a horizontal line is given by $s(t)=3\sin\left(\frac{\pi}{4}t\right)$, $t\geq0$, where $s$ is measured in feet and $t$ in seconds.

(a) The position of the particle at time $t=2$ is $\answer{3}$

(b) The \underline{average velocity}, $v_{av}$ of the object over the interval $[0,t]$ is $\answer{\frac{3\sin(\frac{\pi}{4}t)}{t}}$

(c) Using the expression found in part (b), evaluate $\lim_{t\to0^+} v_{av} = \answer{\frac{3\pi}{4}}$

(d) Find the velocity of the particle: $v(t)=\answer{\frac{3\pi}{4}\cos(\frac{\pi}{4}t)}$

(e) Find the acceleartion of the particle: $a(t)=\answer{-\frac{9\pi^2}{16}\sin(\frac{\pi}{4}t)}$

\end{exercise}

\end{document}
