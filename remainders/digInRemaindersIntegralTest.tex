\documentclass{ximera}

%\usepackage{todonotes}
%\usepackage{mathtools} %% Required for wide table Curl and Greens
%\usepackage{cuted} %% Required for wide table Curl and Greens
\newcommand{\todo}{}

\usepackage{esint} % for \oiint
\ifxake%%https://math.meta.stackexchange.com/questions/9973/how-do-you-render-a-closed-surface-double-integral
\renewcommand{\oiint}{{\large\bigcirc}\kern-1.56em\iint}
\fi


\graphicspath{
  {./}
  {ximeraTutorial/}
  {basicPhilosophy/}
  {functionsOfSeveralVariables/}
  {normalVectors/}
  {lagrangeMultipliers/}
  {vectorFields/}
  {greensTheorem/}
  {shapeOfThingsToCome/}
  {dotProducts/}
  {partialDerivativesAndTheGradientVector/}
  {../productAndQuotientRules/exercises/}
  {../normalVectors/exercisesParametricPlots/}
  {../continuityOfFunctionsOfSeveralVariables/exercises/}
  {../partialDerivativesAndTheGradientVector/exercises/}
  {../directionalDerivativeAndChainRule/exercises/}
  {../commonCoordinates/exercisesCylindricalCoordinates/}
  {../commonCoordinates/exercisesSphericalCoordinates/}
  {../greensTheorem/exercisesCurlAndLineIntegrals/}
  {../greensTheorem/exercisesDivergenceAndLineIntegrals/}
  {../shapeOfThingsToCome/exercisesDivergenceTheorem/}
  {../greensTheorem/}
  {../shapeOfThingsToCome/}
  {../separableDifferentialEquations/exercises/}
  {vectorFields/}
}

\newcommand{\mooculus}{\textsf{\textbf{MOOC}\textnormal{\textsf{ULUS}}}}

\usepackage{tkz-euclide}\usepackage{tikz}
\usepackage{tikz-cd}
\usetikzlibrary{arrows}
\tikzset{>=stealth,commutative diagrams/.cd,
  arrow style=tikz,diagrams={>=stealth}} %% cool arrow head
\tikzset{shorten <>/.style={ shorten >=#1, shorten <=#1 } } %% allows shorter vectors

\usetikzlibrary{backgrounds} %% for boxes around graphs
\usetikzlibrary{shapes,positioning}  %% Clouds and stars
\usetikzlibrary{matrix} %% for matrix
\usepgfplotslibrary{polar} %% for polar plots
\usepgfplotslibrary{fillbetween} %% to shade area between curves in TikZ
\usetkzobj{all}
\usepackage[makeroom]{cancel} %% for strike outs
%\usepackage{mathtools} %% for pretty underbrace % Breaks Ximera
%\usepackage{multicol}
\usepackage{pgffor} %% required for integral for loops



%% http://tex.stackexchange.com/questions/66490/drawing-a-tikz-arc-specifying-the-center
%% Draws beach ball
\tikzset{pics/carc/.style args={#1:#2:#3}{code={\draw[pic actions] (#1:#3) arc(#1:#2:#3);}}}



\usepackage{array}
\setlength{\extrarowheight}{+.1cm}
\newdimen\digitwidth
\settowidth\digitwidth{9}
\def\divrule#1#2{
\noalign{\moveright#1\digitwidth
\vbox{\hrule width#2\digitwidth}}}





\newcommand{\RR}{\mathbb R}
\newcommand{\R}{\mathbb R}
\newcommand{\N}{\mathbb N}
\newcommand{\Z}{\mathbb Z}

\newcommand{\sagemath}{\textsf{SageMath}}


%\renewcommand{\d}{\,d\!}
\renewcommand{\d}{\mathop{}\!d}
\newcommand{\dd}[2][]{\frac{\d #1}{\d #2}}
\newcommand{\pp}[2][]{\frac{\partial #1}{\partial #2}}
\renewcommand{\l}{\ell}
\newcommand{\ddx}{\frac{d}{\d x}}

\newcommand{\zeroOverZero}{\ensuremath{\boldsymbol{\tfrac{0}{0}}}}
\newcommand{\inftyOverInfty}{\ensuremath{\boldsymbol{\tfrac{\infty}{\infty}}}}
\newcommand{\zeroOverInfty}{\ensuremath{\boldsymbol{\tfrac{0}{\infty}}}}
\newcommand{\zeroTimesInfty}{\ensuremath{\small\boldsymbol{0\cdot \infty}}}
\newcommand{\inftyMinusInfty}{\ensuremath{\small\boldsymbol{\infty - \infty}}}
\newcommand{\oneToInfty}{\ensuremath{\boldsymbol{1^\infty}}}
\newcommand{\zeroToZero}{\ensuremath{\boldsymbol{0^0}}}
\newcommand{\inftyToZero}{\ensuremath{\boldsymbol{\infty^0}}}



\newcommand{\numOverZero}{\ensuremath{\boldsymbol{\tfrac{\#}{0}}}}
\newcommand{\dfn}{\textbf}
%\newcommand{\unit}{\,\mathrm}
\newcommand{\unit}{\mathop{}\!\mathrm}
\newcommand{\eval}[1]{\bigg[ #1 \bigg]}
\newcommand{\seq}[1]{\left( #1 \right)}
\renewcommand{\epsilon}{\varepsilon}
\renewcommand{\phi}{\varphi}


\renewcommand{\iff}{\Leftrightarrow}

\DeclareMathOperator{\arccot}{arccot}
\DeclareMathOperator{\arcsec}{arcsec}
\DeclareMathOperator{\arccsc}{arccsc}
\DeclareMathOperator{\si}{Si}
\DeclareMathOperator{\scal}{scal}
\DeclareMathOperator{\sign}{sign}


%% \newcommand{\tightoverset}[2]{% for arrow vec
%%   \mathop{#2}\limits^{\vbox to -.5ex{\kern-0.75ex\hbox{$#1$}\vss}}}
\newcommand{\arrowvec}[1]{{\overset{\rightharpoonup}{#1}}}
%\renewcommand{\vec}[1]{\arrowvec{\mathbf{#1}}}
\renewcommand{\vec}[1]{{\overset{\boldsymbol{\rightharpoonup}}{\mathbf{#1}}}\hspace{0in}}

\newcommand{\point}[1]{\left(#1\right)} %this allows \vector{ to be changed to \vector{ with a quick find and replace
\newcommand{\pt}[1]{\mathbf{#1}} %this allows \vec{ to be changed to \vec{ with a quick find and replace
\newcommand{\Lim}[2]{\lim_{\point{#1} \to \point{#2}}} %Bart, I changed this to point since I want to use it.  It runs through both of the exercise and exerciseE files in limits section, which is why it was in each document to start with.

\DeclareMathOperator{\proj}{\mathbf{proj}}
\newcommand{\veci}{{\boldsymbol{\hat{\imath}}}}
\newcommand{\vecj}{{\boldsymbol{\hat{\jmath}}}}
\newcommand{\veck}{{\boldsymbol{\hat{k}}}}
\newcommand{\vecl}{\vec{\boldsymbol{\l}}}
\newcommand{\uvec}[1]{\mathbf{\hat{#1}}}
\newcommand{\utan}{\mathbf{\hat{t}}}
\newcommand{\unormal}{\mathbf{\hat{n}}}
\newcommand{\ubinormal}{\mathbf{\hat{b}}}

\newcommand{\dotp}{\bullet}
\newcommand{\cross}{\boldsymbol\times}
\newcommand{\grad}{\boldsymbol\nabla}
\newcommand{\divergence}{\grad\dotp}
\newcommand{\curl}{\grad\cross}
%\DeclareMathOperator{\divergence}{divergence}
%\DeclareMathOperator{\curl}[1]{\grad\cross #1}
\newcommand{\lto}{\mathop{\longrightarrow\,}\limits}

\renewcommand{\bar}{\overline}

\colorlet{textColor}{black}
\colorlet{background}{white}
\colorlet{penColor}{blue!50!black} % Color of a curve in a plot
\colorlet{penColor2}{red!50!black}% Color of a curve in a plot
\colorlet{penColor3}{red!50!blue} % Color of a curve in a plot
\colorlet{penColor4}{green!50!black} % Color of a curve in a plot
\colorlet{penColor5}{orange!80!black} % Color of a curve in a plot
\colorlet{penColor6}{yellow!70!black} % Color of a curve in a plot
\colorlet{fill1}{penColor!20} % Color of fill in a plot
\colorlet{fill2}{penColor2!20} % Color of fill in a plot
\colorlet{fillp}{fill1} % Color of positive area
\colorlet{filln}{penColor2!20} % Color of negative area
\colorlet{fill3}{penColor3!20} % Fill
\colorlet{fill4}{penColor4!20} % Fill
\colorlet{fill5}{penColor5!20} % Fill
\colorlet{gridColor}{gray!50} % Color of grid in a plot

\newcommand{\surfaceColor}{violet}
\newcommand{\surfaceColorTwo}{redyellow}
\newcommand{\sliceColor}{greenyellow}




\pgfmathdeclarefunction{gauss}{2}{% gives gaussian
  \pgfmathparse{1/(#2*sqrt(2*pi))*exp(-((x-#1)^2)/(2*#2^2))}%
}


%%%%%%%%%%%%%
%% Vectors
%%%%%%%%%%%%%

%% Simple horiz vectors
\renewcommand{\vector}[1]{\left\langle #1\right\rangle}


%% %% Complex Horiz Vectors with angle brackets
%% \makeatletter
%% \renewcommand{\vector}[2][ , ]{\left\langle%
%%   \def\nextitem{\def\nextitem{#1}}%
%%   \@for \el:=#2\do{\nextitem\el}\right\rangle%
%% }
%% \makeatother

%% %% Vertical Vectors
%% \def\vector#1{\begin{bmatrix}\vecListA#1,,\end{bmatrix}}
%% \def\vecListA#1,{\if,#1,\else #1\cr \expandafter \vecListA \fi}

%%%%%%%%%%%%%
%% End of vectors
%%%%%%%%%%%%%

%\newcommand{\fullwidth}{}
%\newcommand{\normalwidth}{}



%% makes a snazzy t-chart for evaluating functions
%\newenvironment{tchart}{\rowcolors{2}{}{background!90!textColor}\array}{\endarray}

%%This is to help with formatting on future title pages.
\newenvironment{sectionOutcomes}{}{}



%% Flowchart stuff
%\tikzstyle{startstop} = [rectangle, rounded corners, minimum width=3cm, minimum height=1cm,text centered, draw=black]
%\tikzstyle{question} = [rectangle, minimum width=3cm, minimum height=1cm, text centered, draw=black]
%\tikzstyle{decision} = [trapezium, trapezium left angle=70, trapezium right angle=110, minimum width=3cm, minimum height=1cm, text centered, draw=black]
%\tikzstyle{question} = [rectangle, rounded corners, minimum width=3cm, minimum height=1cm,text centered, draw=black]
%\tikzstyle{process} = [rectangle, minimum width=3cm, minimum height=1cm, text centered, draw=black]
%\tikzstyle{decision} = [trapezium, trapezium left angle=70, trapezium right angle=110, minimum width=3cm, minimum height=1cm, text centered, draw=black]


\outcome{Determine if a series converges using the alternating series test.}

\title[Dig-In:]{Remainders and the Integral Test}
\author{Jim Talamo and Bart Snapp}

\begin{document}
\begin{abstract}
There is a nice result for approximating the remainder for series that converge by the integral test. 
\end{abstract}
\maketitle

%%%%%%%%%%%%%%%%%%%%%%%%%%%%%%%%%%%%%%%%%%%%%%%%%%%%%%%%%%%%%%%%%
%%%%%%%%%%%%%%%%%%%%%%%%%%%%%%%%%%%%%%%%%%%%%%%%%%%%%%%%%%%%%%%%%
%%%%%%%%%%%%%%%%%%%%%%%%%%%%%%%%%%%%%%%%%%%%%%%%%%%%%%%%%%%%%%%%%
%add Jenny's important points
%%%%%%%%%%%%%%%%%%%%%%%%%%%%%%%%%%%%%%%%%%%%%%%%%%%%%%%%%%%%%%%%%
%%%%%%%%%%%%%%%%%%%%%%%%%%%%%%%%%%%%%%%%%%%%%%%%%%%%%%%%%%%%%%%%%
%%%%%%%%%%%%%%%%%%%%%%%%%%%%%%%%%%%%%%%%%%%%%%%%%%%%%%%%%%%%%%%%%
When we have a convergent geometric series or a convergent telescoping series, we can find an explicit formula for the terms in the sequence of remainders since we can find an explicit formula for the terms in the sequence of partial sums.  One of the other important convergence tests we have studied so far is the integral test.

\begin{theorem}[Integral Test]
  Suppose that $\{a_n\}_{n=n_0}$ is a sequence, and let $f(x)$ be a continuous, positive, and decreasing function
    with $a_k = f(k)$ on $[N,\infty)$ for some index $N \geq n_0$.  Then, 
    \[
    \sum_{k=n_0}^\infty a_k\text{ and }\int_N^\infty f(x) \d x
    \]
    \textbf{either both converge or both diverge}.
\end{theorem}

The key to proving that a series converges by the integral test is to note that if all of the terms in $\{a_n\}_{n=n_0}$ are eventually positive, then $\{s_n\}_{n=n_0}$ will be eventually increasing.  The improper integral can thus be used to establish an upper bound for $\{s_n\}_{n=n_0}$, meaning that $\{s_n\}_{n=n_0}$ will be bounded and monotonic and thus have a limit.  In this argument, we never actually compute a formula for $s_n$.  As such we have the following important observation. 
  
\begin{quote}
When the assumptions for the integral test are met, we can use the integral test to determine if a series converges, but we cannot ever use it to find the value to which the series converges!
\end{quote}

What, then should we do?  Thankfully, the integral test comes with a nice remainder result.


%%%%%%%%%%%%%%%%%%%%%%%%%%%%%%%%%%%%%%%%%%%%%%%%%%%%%%%%%%%%%%%%%%%%%%

\section{Remainders and the Integral Test}

%%%%%%%%%%%%%%%%%%%%%%%%%%%%%%%%%%%%%%%%%%%%%%%%%%%%%%%%%%%%%%%%%
\begin{theorem}[Remainder Estimates for the Integral Test]{integral test remainder estimates}
If $f(x)$ is a function that is positive, increasing, and continuous for $x \geq n_0$,  and $f(n) = a_n$ for every $n \geq n_0$, then 

\[
\int_{n+1}^{\infty} f(x) \d x \leq  r_n \leq \int_{n}^{\infty} f(x) \d x \qquad \textrm{ for all } n \geq n_0,
\]
where $r_n = \sum_{k=n+1}^{\infty} a_k$.
\end{theorem}

Note that we actually have two results here.

\begin{itemize}
\item The inequality $\int_{n+1}^{\infty} f(x) \d x \leq  r_n$ tells us a \emph{lower bound} for the error; this means we know that the error we make in our approximation can be no \emph{less} than the value of $\int_{n+1}^{\infty} f(x) \d x$ (so we've made \emph{some} amount of error).

\item The inequality $r_n \leq \int_{n}^{\infty} f(x) \d x$ gives us an \emph{upper bound} for the error; this means we know that the error we make in our approximation can be no \emph{more} than the value of $\int_{n}^{\infty} f(x) \d x$.
\end{itemize}
%%%%%%%%%%%%%%%%%%%%%%%%%%%%%%%%%%%%%%%%%%%%%%%%%%%%%%%%%%%%%%%%%

Let's explore how this test works in the context of an example we have seen in previous sections.

\begin{model}
We have seen that  $\sum_{k=1}^\infty \frac{1}{k^2}$ converges, but we have not discussed a way to find its value (and
we will not learn how to do so in this course). 

What if we want to \textbf{approximate} this sum within an error of
$.01$.  How many terms should we sum?  Is it enough to sum
the first ten terms?  The first hundred?  We want to find a number $N$
where we can be sure that
\[
\sum_{k=1}^\infty \frac{1}{k^2}-\sum_{k=1}^N \frac{1}{k^2} \leq .01.
\]
While we could determine the smallest such value when we had an explicit formula for $r_N$, we are not so fortunate here.  However, computers can add terms very quickly, so we do not need the smallest number $N$; we only need one for which we can be sure that $-\sum_{k=1}^N \frac{1}{k^2}$ will approximate $\sum_{k=1}^\infty \frac{1}{k^2}$ to within $.01$.  

Set $r_N$ to be 
\[
r_N = \sum_{N+1}^\infty \frac{1}{k^2}.
\]

Consider the following graph, where the rectangles correspond to the terms of $r_N$.
\begin{image}
\begin{tikzpicture}
	\begin{axis}[
            domain=0:6,xmin=0,xmax=6,ymin=0,ymax=2,
            width=4in,
            height=2in,
            xtick={1,2,...,5},
            xticklabels={$N$,$N+1$,$N+2$, $N+3$, $N+4$, $N+5$},
            ytick = {},
            yticklabels = {,,},
            axis lines =middle, xlabel=$n$, ylabel=$a$,
            every axis y label/.style={at=(current axis.above origin),anchor=south},
            every axis x label/.style={at=(current axis.right of origin),anchor=west},
            clip=false,
            axis on top,
          ]          
	  \addplot[color=penColor,fill=penColor,only marks,mark=*] coordinates{(2,7/10)};  %% closed hole
	  \addplot [draw=penColor, fill = fill1] plot coordinates {(2,0) (3,0) (3, 7/10) (2,7/10) (2, 0)};          
                    
          \addplot[color=penColor,fill=penColor,only marks,mark=*] coordinates{(3,7/20)};  %% closed hole
          \addplot [draw=penColor, fill = fill1] plot coordinates {(3,0) (4,0) (4, 7/20) (3,7/20) (3, 0)};          
                    
          \addplot[color=penColor,fill=penColor,only marks,mark=*] coordinates{(4,7/40)};  %% closed hole        
          \addplot [draw=penColor, fill = fill1] plot coordinates {(4,0) (5,0) (5, 7/40) (4,7/40) (4, 0)};          
          
          \addplot[color=penColor,fill=penColor,only marks,mark=*] coordinates{(5,7/80)};  %% closed hole
          \addplot [draw=penColor, fill = fill1] plot coordinates {(5,0) (6,0) (6, 7/80) (5,7/80) (5, 0)};          

	  \addplot [draw=penColor,very thick, domain=0.5:6] {(14/5)*2^(-x)};
        \end{axis}
\end{tikzpicture}
\end{image}
Shifting these rectangles over one unit, we have the following.
\begin{image}
\begin{tikzpicture}
	\begin{axis}[
            domain=0:6,xmin=0,xmax=6,ymin=0,ymax=2,
            width=4in,
            height=2in,
            xtick={1,2,...,5},
            xticklabels={$N$,$N+1$,$N+2$, $N+3$, $N+4$, $N+5$},
            ytick = {},
            yticklabels = {,,},
            axis lines =middle, xlabel=$n$, ylabel=$a$,
            every axis y label/.style={at=(current axis.above origin),anchor=south},
            every axis x label/.style={at=(current axis.right of origin),anchor=west},
            clip=false,
            axis on top,
          ]
          \addplot [draw=penColor, fill = fill1] plot coordinates {(1,0) (2,0) (2, 7/10) (1,7/10) (1, 0)};          
          
	  \addplot[color=penColor,fill=penColor,only marks,mark=*] coordinates{(2,7/10)};  %% closed hole
	  \addplot [draw=penColor, fill = fill1] plot coordinates {(2,0) (3,0) (3, 7/20) (2,7/20) (2, 0)};          
          
          \addplot[color=penColor,fill=penColor,only marks,mark=*] coordinates{(3,7/20)};  %% closed hole
          \addplot [draw=penColor, fill = fill1] plot coordinates {(3,0) (4,0) (4, 7/40) (3,7/40) (3, 0)};          
          
          
          \addplot[color=penColor,fill=penColor,only marks,mark=*] coordinates{(4,7/40)};  %% closed hole        
          \addplot [draw=penColor, fill = fill1] plot coordinates {(4,0) (5,0) (5, 7/80) (4,7/80) (4, 0)};          

          \addplot[color=penColor,fill=penColor,only marks,mark=*] coordinates{(5,7/80)};  %% closed hole

	  \addplot [draw=penColor,very thick, domain=0.5:6] {(14/5)*2^(-x)};
        \end{axis}
\end{tikzpicture}
\end{image}
Now we know
\[
r_N \leq \int_N^\infty \frac{1}{x^2} \d x .
\]
In other words, the remainder of the series must be less than the given integral.  
So, if we can find a whole number $N$ for which
\[
\int_N^\infty \frac{1}{x^2} \d x \leq .01,
\]
we will be sure that we have summed enough terms in the series to get
to within $.01$.   From our work above, we see that we need to find $N$ such that
    \[
    \int_N^\infty \frac{1}{x^2} \d x \leq \answer[given]{.01}.
    \]
    Write with me.
    \begin{align*}
      \int_N^\infty \frac{1}{x^2} \d x & \leq .01\\
      \lim_{b \to \infty} \int_N^b \frac{1}{x^2} \d x &\leq .01\\
      \lim_{b \to \infty} \eval{\answer[given]{\frac{-1}{x}}}_N^b& \leq .01\\
      \lim_{b \to \infty} \frac{1}{N} - \frac{1}{b}& \leq .01\\
      \frac{1}{N} &\leq .01\\
      N&>\answer[given]{100}.
    \end{align*}
    So we need $N$ to be at least $100$.


This shows that if we compute $\sum_{k=1}^{100} \frac{1}{k^2}$, meaning that if we sum the $100$ terms of the series $\sum_{k=1}^\infty
\frac{1}{k^2}$, we will be within $.01$ of the true answer.

Using a computer to sum the first $100$ terms we find to five decimal places that 
\[
\sum_{k=1}^{100} \frac{1}{k^2} \approx 1.63498.
\]


Similarly, by considering the picture, we find that $r_n \geq \int_{N+1}^\infty \frac{1}{x^2}= \int_{101}^\infty \frac{1}{x^2}$.  By repeating the earlier calculation, we find that $\int_{101}^\infty \frac{1}{x^2} = \frac{1}{101}$, so we know that the error in our approximation is at least $\frac{1}{101} = .00990$ to five decimal places.

As such we have an estimate for $\sum_{k=1}^{\infty} \frac{1}{k^2}$.  Noting that the error $r_{100}$ satisfies

\begin{align*}
\int_{n+1}^{\infty} \frac{1}{x^2} \d x & \leq  r_n & \leq \int_{n}^{\infty} \frac{1}{x^2} \d x \\
\frac{1}{101} & \leq  r_n & \frac{1}{100}, \\
\end{align*}

and that $s_{100} = \sum_{k=1}^{100} \frac{1}{k^2} = \answer{1.6350}$ to four decimal places, we find that the actual value of the series $\sum_{k=1}^{\infty}$ should be between $1.63498 + \frac{1}{101}$ and $1.63498 + \frac{1}{100}$.  Reporting this to five decimal places, we find that

\[
1.64488 \leq \sum_{k=1}^{\infty} \frac{1}{k^2} \leq 1.64498. 
\]

As we have mentioned previously, the series actually converges to $\frac{\pi^2}{6}$, and using a calculator, we know that $\frac{\pi^2}{6} \approx 1.64493$.  

\end{model}

There are two important types of questions we have asked about remainders.

\begin{itemize}
\item[1.] How bad is the error made when we approximate a convergent infinite series by its first several terms?
In other words, if we specify $N$, how close is $\sum_{k=n_0}^{N} a_k$ to the exact value of $\sum_{k=n_0}^{\infty} a_k$?
\item[2.] How many terms should we specify if we want to know the value of a convergent series to obtain a desired precision?
Said another way, given an acceptable value for the error, what value should we pick for $N$ so $\sum_{k=n_0}^{N} a_k$ approximates $\sum_{k=n_0}^{\infty} a_k$ that accurately?
\end{itemize}

Now, let's work an example that addresses both.

\begin{example}
Consider the series $\sum_{k=2}^{\infty} \frac{\ln(k)}{k^2}$.  

\begin{example}
Show that this series converges. 
\begin{explanation}
We can check that the series above meets the hypothesis of the integral test, so we may apply it to investigate whether the series converges.
We must compute 
\begin{align*}
\int_{2}^{\infty} \frac{\ln(x)}{x^2} \d x &= \lim_{b \to \infty} \int_2^b  \frac{\ln(x)}{x^2} \d x .\\
\end{align*}
To find the necessary antiderivative, we use use integration by parts.

\begin{align*}
u&= \answer{\ln(x)} & \d v&= \answer{\frac{1}{x^2}} \d x \\
\d u &= \answer{\frac{1}{x}} & v&= \answer{-\frac{1}{x}}
\end{align*}
Thus, we find

\begin{align*}
\int  \frac{\ln(x)}{x^2} \d x &= -\frac{\ln(x)}{x} + \int \frac{1}{x^2} \d x \\
&= -\frac{\ln(x)}{x} - \frac{1}{x} +C
\end{align*}

Now, we can evaluate the improper integral needed for the integral test.

\begin{align*}
\int_{2}^{\infty} \frac{\ln(x)}{x^2} \d x = \lim_{b \to \infty} \int_2^b  \frac{\ln(x)}{x^2} \d x &= \lim_{b \to \infty} \eval{-\frac{\ln(x)}{x} - \frac{1}{x}}_2^b\\
&= \lim_{b \to \infty} \eval{\left(-\frac{\ln(b)}{b} - \frac{1}{b}\right) - \left(-\frac{\ln(2)}{2} - \frac{1}{2}\right)}\\
&= \lim_{b \to \infty} \eval{\frac{\ln(2)+1}{2}-\frac{\ln(b)}{b} - \frac{1}{b}}
\end{align*}
By growth rates, we have that $\lim_{b \to \infty} \frac{\ln(b)}{b} = \answer{0}$, so we find that $\int_{2}^{\infty} \frac{\ln(x)}{x^2} \d x$ \wordChoice{\choice[correct]{converges}\choice{diverges}} to $\answer{\frac{\ln(2)+1}{2}}$.
 
\end{explanation}
\end{example}

\begin{example}
Using the integral test (and a calculator or computer!), give an upper error bound if $\sum_{k=2}^{100}  \frac{\ln(k)}{k^2}$ is used to approximate the value of $\sum_{k=2}^{\infty}  \frac{\ln(k)}{k^2}$
\begin{explanation}
We know that an upper bound for the error is given by $r_n \leq \int_{\answer{100}}^{\infty} \frac{\ln(x)}{x^2} \d x$.  We've already found that 

\begin{align*}
\int  \frac{\ln(x)}{x^2} \d x &= -\frac{\ln(x)}{x} - \frac{1}{x} +C,
\end{align*}

so 

\[
\int_{100}^{\infty} \frac{\ln(x)}{x^2} \d x =\frac{\ln(100)+1}{100} 
\]

Using a calculator, we find to four decimal places that $\int_{100}^{\infty} \frac{\ln(x)}{x^2} \d x  \approx \answer[tolerance=.0005]{.0561}$.  Thus, the error made by approximating $\sum_{k=2}^{\infty}  \frac{\ln(k)}{k^2}$ by $\sum_{k=2}^{100}  \frac{\ln(k)}{k^2}$ is no \wordChoice{\choice{better}\choice[correct]{worse}} than the above result.

Using a calculator or a computer, we find (quite quickly) that to four decimal places, $\sum_{k=2}^{100}  \frac{\ln(k)}{k^2} \approx .8817$.

It doesn't look like we used enough terms here to get a really good sense of the value to which this series converge quite yet.

\end{explanation}
\end{example}

\begin{example}
Using technology, find a value for $N$ so we are guaranteed that  $\sum_{k=2}^{N}  \frac{\ln(k)}{k^2}$ will be accurate to within $.001$ of the exact value of $\sum_{k=2}^{\infty}  \frac{\ln(k)}{k^2}$.  Then, find the value of $\sum_{k=2}^{100} \frac{\ln(k)}{k^2}$ to within $.001$ of its exact value.

\begin{explanation}
We need to find $N$ so $r_N \leq .001$.  Since we know that $r_N \leq \int_N^\infty  \frac{\ln(x)}{x^2} \d x$, we can achieve this by setting $\int_N^\infty  \frac{\ln(x)}{x^2} \d x \leq .001$.  

By analogous computation to earlier, we find that $\int_N^\infty  \frac{\ln(x)}{x^2} \d x = \frac{\ln(N)+1}{N}$, so we need to find $N$ so

\[
\frac{\ln(N)+1}{N} \leq .001  .
\]

We cannot solve this equation exactly, but we can use a computer or a calculator to approximate $N$.  Since estimating is usually done with technology, it's perfectly reasonable to use it to find $N$ here, and doing so yields $N = 10233.4$, so we should use $N=10234$.

Thus, $\sum_{k=2}^{10234} \frac{\ln(k)}{k^2}$ will give the desired approximation, and using technology, we find that $\sum_{k=2}^{10234} \frac{\ln(k)}{k^2} \approx .9365$ to four decimal places.

\end{explanation}
\end{example}




\end{example}


\end{document}
