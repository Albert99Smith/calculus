\documentclass[11pt]{article}
%trivial edit
\usepackage{graphicx}
\usepackage{amssymb}



\begin{document}

\section*{Pre-Calculus Review}
\begin{enumerate}
	\item Find domain and range.
	\item Determine where a function is positive or negative.
	\item Find inverse functions (algebraically and graphically).
	\item Work with exponential and logarithmic functions.
	\item Evaluate expressions and solve equations involving trig functions and inverse trig functions.
	\item Define a function.
	\item Know the graphs and properties of ``famous'' functions.
	\item Define and work with inverse functions.
	\item Determine the intervals on which a function has an inverse.
	\item Understand the relationship between exponential and logarithmic functions.
	\item Understand the properties of trig functions.
	\item Know and use the properties of exponential and logarithmic functions.
	\item Graph basic functions and their inverses.
	\item Use the horizontal line test to determine if a function is one-to-one.
	\item Distinguish two functions by considering their domains.
	\item Recognize different representations of the same function.
\end{enumerate}


\section*{2.1: The Idea of Limits}
\begin{enumerate}
	\item Compute average velocity.
	\item Approximate instantaneous velocity.
	\item Calculate slope of a secant line.
	\item Compare average and instantaneous velocity.
	\item Compare secant and tangent lines.
	\item Understand the concept of a limit.
\end{enumerate}

\section*{2.2: Definitions of Limits}
\begin{enumerate}
	\item Calculate limits from a graph (or state that the limit does not exist).
	\item Estimate limits using nearby values.
	\item Define a one-sided limit.
	\item Explain the relationship between one-sided and two-sided limits.
	\item Distinguish between limit values and function values.
	\item Identify when a limit does not exist.
	\item Compute average rate of change for a function.
\end{enumerate}


\section*{2.3: Limit Laws}
\begin{enumerate}
	\item Calculate limits using the limit laws
	\item Calculate limits of the form $0/0$
	\item Calculate limits of piecewise functions.
	\item Calculate limits using the Squeeze Theorem.
	\item Understand what is meant by the form of a limit.
	\item Understand the Squeeze Theorem and how it can be used to find limit values.
\end{enumerate}


\section*{2.4: Infinite Limits}
\begin{enumerate}
	\item Recognize when a limit is indicating there is a vertical asymptote.
	\item Evaluate the limit as $x$ approaches a point where there is a vertical asymptote.
	\item Match graphs of functions with their equations based on vertical asymptotes.
	\item Discuss what it means for a limit to equal $\infty$.
	\item Define a vertical asymptote.
	\item Understand the relationship between limits and vertical asymptotes.
\end{enumerate}


\section*{2.5: Limits at Infinity}
\begin{enumerate}
	\item Calculate the limit as $x$ approaches $\pm \infty$ of common functions algebraically.
	\item Decide whether a form is determinate or indeterminate.
	\item Find the limit as $x$ approaches $\pm \infty$ from a graph.
	\item Identify determinate and indeterminate forms.
	\item Discuss why infinity is not a number.
\end{enumerate}


\section*{2.6: Continuity}
\begin{enumerate}
	\item Identify where a function is and is not continuous.
	\item List common continuous functions such as trigonometric, exponential, logarithmic, and polynomials.
	\item Determine if the Intermediate Value Theorem applies.
	\item Compute limits using continuity.
	\item Understand what it means for a function to be continuous.
	\item Identify when a function is right- or left-continuous at the endpoints of a closed interval.
	\item Understand the connection between continuity of a function and the value of a limit.
	\item State the Intermediate Value Theorem including hypotheses.
	\item Sketch pictures indicating why the Intermediate Value Theorem is true, and why all hypotheses are necessary.
        \item   Explain why certain points exist using the Intermediate Value Theorem.
\end{enumerate}

\section*{3.1: Introducing the Derivative}
\begin{enumerate}
	\item Use limits to find the slope of the tangent line at a point.
	\item Write the equation of the tangent line.
	\item Find the derivative function using the limit definition.
	\item Understand the difference between average and instantaneous velocity.
	\item Recognize and distinguish between secant and tangent lines.
	\item Understand the definition of the derivative at a point.
	\item Understand the derivative as a function.
\end{enumerate}

\section*{3.2: Working with Derivatives}
\begin{enumerate}
	\item Estimate the slope of the tangent line graphically.
	\item Graph the derivative function.
	\item Determine whether a piecewise function is differentiable.
	\item Understand the relationship between the graph of a function and the graph of its derivative.
	\item Explain the relationship between differentiability and continuity.
\end{enumerate}

\section*{3.3: Rules of Differentiation}
\begin{enumerate}
	\item Use ``shortcut'' rules to find and use derivatives.
	\item Calculate higher order derivatives.
	\item Use the definition of the derivative to develop shortcut rules to find the derivatives of constants, constant multiples, powers of $x$, sums of functions, and the natural exponential function.
	\item Define higher order derivatives.
\end{enumerate}


\section*{3.4: The Product and Quotient Rules}
\begin{enumerate}
	\item Use the product rule to calculate derivatives.
	\item Use the quotient rule to calculate derivatives.
	\item Identify products of functions.
	\item Identify quotients of functions.
	\item Combine derivative rules to take derivatives of more complicated functions.
\end{enumerate}

\section*{3.5: Derivatives of Trig Functions}
\begin{enumerate}
	\item Compute derivatives of trig functions.
	\item Compute limits using special trig limits and trig identities.
	\item Compute higher order derivatives of sine and cosine.
	\item Understand derivatives of trig functions.
	\item Understand special trig limits.
	\item Understand the cyclic nature of the derivatives of sine and cosine.
\end{enumerate}

\section*{3.6: Derivatives as Rates of Change}
\begin{enumerate}
	\item Find velocity and acceleration and use to determine information about position.
	\item Determine average and instantaneous growth rates.
	\item Calculate average and marginal costs.
	\item Identify applications of the derivative.
	\item Assign meaning to the first and second derivatives of a position function.
	\item Interpret the derivative as information about growth.
\end{enumerate}

\section*{3.7: The Chain Rule}
\begin{enumerate}
	\item Take derivatives of compositions of functions using the chain rule.
	\item Take derivatives that require the use of multiple derivative rules.
	\item Recognize a composition of functions.
	\item Understand rate of change when quantities are dependent upon each other.
	\item Use order of operations in situations requiring multiple derivative rules.
\end{enumerate}

\section*{3.8: Implicit Differentiation}
\begin{enumerate}
	\item Implicitly differentiate expressions.
	\item Solve equations for $\frac{dy}{dx}$
	\item Find the equation of the tangent line for curves that are not graphs of functions.
	\item Find derivatives of functions with rational exponents.
	\item Understand how changing the variable changes how we take the derivative.
	\item Understand the derivatives of expressions that are not functions or not solved for $y$.
	\item Use implicit differentiation to demonstrate the power rule for rational exponents.
\end{enumerate}

\section*{3.9: Derivatives of Logarithmic and Exponential Functions}
\begin{enumerate}
	\item Take derivatives of logarithms and exponents of all bases.
	\item Use logarithmic differentiation to simplify taking derivatives.
	\item Take derivatives of functions raised to functions.
	\item Apply the generalized power rule.
	\item Recognize the difference between a variable as the base and a variable as the exponent.
	\item Identify situations where logs can be used to help find derivatives.
	\item Work with the inverse properties of $e^x$ and $\ln(x)$.
\end{enumerate}

\section*{3.10: Derivatives of Inverse Trig Functions}
\begin{enumerate}
	\item Take derivatives which involve inverse trig functions.
	\item Find derivatives of inverse functions in general.
	\item Recall the meaning and properties of inverse trig functions.
	\item Derive the derivatives of inverse trig functions.
	\item Understand how the derivative of an inverse function relates to the original derivative.
\end{enumerate}

\section*{3.11: Related Rates}
\begin{enumerate}
	\item Solve related rates word problems.
	\item Identify word problems as related rates problems.
	\item Translate word problems into mathematical expressions.
	\item Calculate derivatives of expressions with multiple variables implicitly.
	\item Understand the process of solving related rates problems.
\end{enumerate}

\section*{4.1: Maxima and Minima}
\begin{enumerate}
	\item Find critical points.
	\item Find the absolute max or min of a continuous function on a closed interval.
	\item Solve basic word problems involving maxima or minima.
	\item Compare and contrast local and absolute maxima and minima.
	\item Identify situations in which an absolute maximum or minimum is guaranteed.
	\item Define a critical point.
	\item Define absolute maximum and absolute minimum.
	\item State the Extreme Value Theorem.
	\item Define local maximum and local minimum.
	\item Classify critical points.
\end{enumerate}

\section*{4.2: What Derivatives Tell Us}
\begin{enumerate}
	\item Find the intervals where a function is increasing or decreasing.
	\item Find the intervals where a function is concave up or down.
	\item Find all local maximums and minimums using the 1st and 2nd derivative tests.
	\item Determine when a local extremum is an absolute extremum.
	\item Sketch a graph of $f(x)$ with information from $f'(x)$.
	\item Understand what information the derivative gives concerning when a function is increasing or decreasing.
	\item Understand how to find local maximums and minimums.
	\item Define concavity and inflection points.
	\item Identify when we can find an absolute maximum or minimum on an open interval.
	\item Identify any asymptotic behaviors a function may have: vertical, horizontal, or slant.
\end{enumerate}

\section*{4.3: Graphing Functions}
\begin{enumerate}
	\item Determine how the graph of a function looks without using a calculator.
\end{enumerate}

\section*{4.4: Optimization Problems}
\begin{enumerate}
	\item Interpret an optimization problem as the procedure used to make a system or design as effective or functional as possible.
	\item Set up an optimization problem by identifying the objective function and appropriate constraints.
	\item Solve optimization problems by finding the appropriate absolute extremum.
\end{enumerate}

\section*{4.5: Linear Approximation and Differentials}
\begin{enumerate}
	\item Find the linear approximation to a function at a point and use it to approximate the function value.
	\item Label a graph with the linear approximation, $x$, $x+a$, $dx$, $dy$, $\Delta x$ and $\Delta y$.
	\item Find the error of a linear approximation.
	\item Calculate $\Delta y$ and $dy$.
	\item Define linear approximation as an application of the tangent to a curve.
	\item Identify when a linear approximation can be used.
	\item Understand how good an approximation is.
	\item Define a differential.
	\item Use the second derivative to discuss whether the linear approximation over- or under-estimates the actual value.
\end{enumerate}

\section*{4.6: Mean Value Theorem}
\begin{enumerate}
	\item Determine whether Rolle's Theorem and/or MVT can be applied.
	\item Find the values guaranteed by Rolle's Theorem or MVT.
	\item Use MVT to solve word problems.
	\item Compare and contrast the IVT, MVT, and Rolle's Theorem.
	\item Understand what the MVT tells us.
	\item Identify calculus ideas which are consequences of the MVT.
	\item Sketch pictures to illustrate why the MVT is true.
\end{enumerate}

\section*{4.7: L'Hopital's Rule}
\begin{enumerate}
	\item Convert indeterminate forms to $\frac{0}{0}$ or $\frac{\infty}{\infty}$
	\item Use L'Hopital's Rule to find limits.
	\item Recall how to find limits for forms that are not indeterminate.
	\item Determine if a form is indeterminate.
	\item Define an indeterminate form.
	\item Define L'Hopital's Rule and identify when it can be used.
\end{enumerate}

\section*{4.9: Antiderivatives}
\begin{enumerate}
	\item Compute basic antiderivatives
	\item Solve basic initial value problems.
	\item Use antiderivatives to solve simple word problems.
	\item Define an antiderivative.
	\item Compare and contrast finding derivatives and finding antiderivatives.
	\item Define an indefinite integral.
	\item Define initial value problems.
	\item Discuss the meaning of antiderivatives of a position function.
\end{enumerate}

\section*{5.1: Approximating Area under Curves}
\begin{enumerate}
	\item Add up a large number of terms quickly using sigma notation.
	\item Approximate area under a curve.
	\item Approximate displacement from velocity.
	\item Compute left, right, and midpoint Riemann Sums.
	\item Define area.
	\item Understand the relationship between area under a curve and sums of rectangles.
	\item Associate the components of the sum formula with their geometric meaning.
\end{enumerate}

\section*{5.2: Definite Integrals}
\begin{enumerate}
	\item Approximate net area.
	\item Compute definite integrals using limits of Riemann Sums.
	\item Compute definite integrals using geometry.
	\item Compute definite integrals using the properties of integrals.
	\item Define net area.
	\item Understand how Riemann sums are used to find exact area.
	\item Justify the properties of definite integrals using algebra or geometry.
\end{enumerate}

\section*{5.3: Fundamental Theorem of Calculus}
\begin{enumerate}
	\item Calculate and evaluate accumulation functions.
	\item Take derivatives of accumulation functions using the 1st Fundamental Theorem of Calculus.
	\item Evaluate definite integrals using the 2nd Fundamental Theorem of Calculus.
	\item Define accumulation functions.
	\item Use the accumulation function to find information about the original function.
	\item Understand the relationship between the function and the derivative of its accumulation function.
	\item Understand how the area under a curve is related to the antiderivative.
	\item Understand the relationship between indefinite and definite integrals.
\end{enumerate}

\section*{5.4: Working With Integrals}
\begin{enumerate}
	\item Use symmetry to calculate definite integrals.
	\item Find the average value of a function.
	\item Use the Mean Value Theorem for integrals.
	\item Explain geometrically why symmetry of a function simplifies calculation of some definite integrals.
	\item Define the average value of a function.
	\item State the Mean Value Theorem for integrals.
\end{enumerate}

\section*{5.5: Substitution Rule}
\begin{enumerate}
	\item Calculate indefinite integrals (antiderivatives) using $u$-substitution.
	\item Calculate definite integrals using $u$-substitution
	\item Practice until you are familiar with a lot of patterns.
	\item Undo the Chain Rule.
	\item Evaluate definite and indefinite integrals through a change of variables.
\end{enumerate}

\section*{6.1: Velocity and Net Change}
\begin{enumerate}
	\item Given a velocity function, calculate displacement and distance traveled.
	\item Given a velocity function, find the position function.
	\item Given an acceleration function, find the velocity function.
	\item Calculate net change and future value.
	\item Understand the difference between displacement and distance traveled.
	\item Understand the relationship between position, velocity and acceleration.
	\item Understand how the net change and future value of a function are related to that function's derivative.
\end{enumerate}


\end{document}

