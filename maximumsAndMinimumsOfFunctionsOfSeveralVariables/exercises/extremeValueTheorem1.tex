\documentclass{ximera}

%\usepackage{todonotes}
%\usepackage{mathtools} %% Required for wide table Curl and Greens
%\usepackage{cuted} %% Required for wide table Curl and Greens
\newcommand{\todo}{}

\usepackage{esint} % for \oiint
\ifxake%%https://math.meta.stackexchange.com/questions/9973/how-do-you-render-a-closed-surface-double-integral
\renewcommand{\oiint}{{\large\bigcirc}\kern-1.56em\iint}
\fi


\graphicspath{
  {./}
  {ximeraTutorial/}
  {basicPhilosophy/}
  {functionsOfSeveralVariables/}
  {normalVectors/}
  {lagrangeMultipliers/}
  {vectorFields/}
  {greensTheorem/}
  {shapeOfThingsToCome/}
  {dotProducts/}
  {partialDerivativesAndTheGradientVector/}
  {../productAndQuotientRules/exercises/}
  {../normalVectors/exercisesParametricPlots/}
  {../continuityOfFunctionsOfSeveralVariables/exercises/}
  {../partialDerivativesAndTheGradientVector/exercises/}
  {../directionalDerivativeAndChainRule/exercises/}
  {../commonCoordinates/exercisesCylindricalCoordinates/}
  {../commonCoordinates/exercisesSphericalCoordinates/}
  {../greensTheorem/exercisesCurlAndLineIntegrals/}
  {../greensTheorem/exercisesDivergenceAndLineIntegrals/}
  {../shapeOfThingsToCome/exercisesDivergenceTheorem/}
  {../greensTheorem/}
  {../shapeOfThingsToCome/}
  {../separableDifferentialEquations/exercises/}
  {vectorFields/}
}

\newcommand{\mooculus}{\textsf{\textbf{MOOC}\textnormal{\textsf{ULUS}}}}

\usepackage{tkz-euclide}\usepackage{tikz}
\usepackage{tikz-cd}
\usetikzlibrary{arrows}
\tikzset{>=stealth,commutative diagrams/.cd,
  arrow style=tikz,diagrams={>=stealth}} %% cool arrow head
\tikzset{shorten <>/.style={ shorten >=#1, shorten <=#1 } } %% allows shorter vectors

\usetikzlibrary{backgrounds} %% for boxes around graphs
\usetikzlibrary{shapes,positioning}  %% Clouds and stars
\usetikzlibrary{matrix} %% for matrix
\usepgfplotslibrary{polar} %% for polar plots
\usepgfplotslibrary{fillbetween} %% to shade area between curves in TikZ
\usetkzobj{all}
\usepackage[makeroom]{cancel} %% for strike outs
%\usepackage{mathtools} %% for pretty underbrace % Breaks Ximera
%\usepackage{multicol}
\usepackage{pgffor} %% required for integral for loops



%% http://tex.stackexchange.com/questions/66490/drawing-a-tikz-arc-specifying-the-center
%% Draws beach ball
\tikzset{pics/carc/.style args={#1:#2:#3}{code={\draw[pic actions] (#1:#3) arc(#1:#2:#3);}}}



\usepackage{array}
\setlength{\extrarowheight}{+.1cm}
\newdimen\digitwidth
\settowidth\digitwidth{9}
\def\divrule#1#2{
\noalign{\moveright#1\digitwidth
\vbox{\hrule width#2\digitwidth}}}





\newcommand{\RR}{\mathbb R}
\newcommand{\R}{\mathbb R}
\newcommand{\N}{\mathbb N}
\newcommand{\Z}{\mathbb Z}

\newcommand{\sagemath}{\textsf{SageMath}}


%\renewcommand{\d}{\,d\!}
\renewcommand{\d}{\mathop{}\!d}
\newcommand{\dd}[2][]{\frac{\d #1}{\d #2}}
\newcommand{\pp}[2][]{\frac{\partial #1}{\partial #2}}
\renewcommand{\l}{\ell}
\newcommand{\ddx}{\frac{d}{\d x}}

\newcommand{\zeroOverZero}{\ensuremath{\boldsymbol{\tfrac{0}{0}}}}
\newcommand{\inftyOverInfty}{\ensuremath{\boldsymbol{\tfrac{\infty}{\infty}}}}
\newcommand{\zeroOverInfty}{\ensuremath{\boldsymbol{\tfrac{0}{\infty}}}}
\newcommand{\zeroTimesInfty}{\ensuremath{\small\boldsymbol{0\cdot \infty}}}
\newcommand{\inftyMinusInfty}{\ensuremath{\small\boldsymbol{\infty - \infty}}}
\newcommand{\oneToInfty}{\ensuremath{\boldsymbol{1^\infty}}}
\newcommand{\zeroToZero}{\ensuremath{\boldsymbol{0^0}}}
\newcommand{\inftyToZero}{\ensuremath{\boldsymbol{\infty^0}}}



\newcommand{\numOverZero}{\ensuremath{\boldsymbol{\tfrac{\#}{0}}}}
\newcommand{\dfn}{\textbf}
%\newcommand{\unit}{\,\mathrm}
\newcommand{\unit}{\mathop{}\!\mathrm}
\newcommand{\eval}[1]{\bigg[ #1 \bigg]}
\newcommand{\seq}[1]{\left( #1 \right)}
\renewcommand{\epsilon}{\varepsilon}
\renewcommand{\phi}{\varphi}


\renewcommand{\iff}{\Leftrightarrow}

\DeclareMathOperator{\arccot}{arccot}
\DeclareMathOperator{\arcsec}{arcsec}
\DeclareMathOperator{\arccsc}{arccsc}
\DeclareMathOperator{\si}{Si}
\DeclareMathOperator{\scal}{scal}
\DeclareMathOperator{\sign}{sign}


%% \newcommand{\tightoverset}[2]{% for arrow vec
%%   \mathop{#2}\limits^{\vbox to -.5ex{\kern-0.75ex\hbox{$#1$}\vss}}}
\newcommand{\arrowvec}[1]{{\overset{\rightharpoonup}{#1}}}
%\renewcommand{\vec}[1]{\arrowvec{\mathbf{#1}}}
\renewcommand{\vec}[1]{{\overset{\boldsymbol{\rightharpoonup}}{\mathbf{#1}}}\hspace{0in}}

\newcommand{\point}[1]{\left(#1\right)} %this allows \vector{ to be changed to \vector{ with a quick find and replace
\newcommand{\pt}[1]{\mathbf{#1}} %this allows \vec{ to be changed to \vec{ with a quick find and replace
\newcommand{\Lim}[2]{\lim_{\point{#1} \to \point{#2}}} %Bart, I changed this to point since I want to use it.  It runs through both of the exercise and exerciseE files in limits section, which is why it was in each document to start with.

\DeclareMathOperator{\proj}{\mathbf{proj}}
\newcommand{\veci}{{\boldsymbol{\hat{\imath}}}}
\newcommand{\vecj}{{\boldsymbol{\hat{\jmath}}}}
\newcommand{\veck}{{\boldsymbol{\hat{k}}}}
\newcommand{\vecl}{\vec{\boldsymbol{\l}}}
\newcommand{\uvec}[1]{\mathbf{\hat{#1}}}
\newcommand{\utan}{\mathbf{\hat{t}}}
\newcommand{\unormal}{\mathbf{\hat{n}}}
\newcommand{\ubinormal}{\mathbf{\hat{b}}}

\newcommand{\dotp}{\bullet}
\newcommand{\cross}{\boldsymbol\times}
\newcommand{\grad}{\boldsymbol\nabla}
\newcommand{\divergence}{\grad\dotp}
\newcommand{\curl}{\grad\cross}
%\DeclareMathOperator{\divergence}{divergence}
%\DeclareMathOperator{\curl}[1]{\grad\cross #1}
\newcommand{\lto}{\mathop{\longrightarrow\,}\limits}

\renewcommand{\bar}{\overline}

\colorlet{textColor}{black}
\colorlet{background}{white}
\colorlet{penColor}{blue!50!black} % Color of a curve in a plot
\colorlet{penColor2}{red!50!black}% Color of a curve in a plot
\colorlet{penColor3}{red!50!blue} % Color of a curve in a plot
\colorlet{penColor4}{green!50!black} % Color of a curve in a plot
\colorlet{penColor5}{orange!80!black} % Color of a curve in a plot
\colorlet{penColor6}{yellow!70!black} % Color of a curve in a plot
\colorlet{fill1}{penColor!20} % Color of fill in a plot
\colorlet{fill2}{penColor2!20} % Color of fill in a plot
\colorlet{fillp}{fill1} % Color of positive area
\colorlet{filln}{penColor2!20} % Color of negative area
\colorlet{fill3}{penColor3!20} % Fill
\colorlet{fill4}{penColor4!20} % Fill
\colorlet{fill5}{penColor5!20} % Fill
\colorlet{gridColor}{gray!50} % Color of grid in a plot

\newcommand{\surfaceColor}{violet}
\newcommand{\surfaceColorTwo}{redyellow}
\newcommand{\sliceColor}{greenyellow}




\pgfmathdeclarefunction{gauss}{2}{% gives gaussian
  \pgfmathparse{1/(#2*sqrt(2*pi))*exp(-((x-#1)^2)/(2*#2^2))}%
}


%%%%%%%%%%%%%
%% Vectors
%%%%%%%%%%%%%

%% Simple horiz vectors
\renewcommand{\vector}[1]{\left\langle #1\right\rangle}


%% %% Complex Horiz Vectors with angle brackets
%% \makeatletter
%% \renewcommand{\vector}[2][ , ]{\left\langle%
%%   \def\nextitem{\def\nextitem{#1}}%
%%   \@for \el:=#2\do{\nextitem\el}\right\rangle%
%% }
%% \makeatother

%% %% Vertical Vectors
%% \def\vector#1{\begin{bmatrix}\vecListA#1,,\end{bmatrix}}
%% \def\vecListA#1,{\if,#1,\else #1\cr \expandafter \vecListA \fi}

%%%%%%%%%%%%%
%% End of vectors
%%%%%%%%%%%%%

%\newcommand{\fullwidth}{}
%\newcommand{\normalwidth}{}



%% makes a snazzy t-chart for evaluating functions
%\newenvironment{tchart}{\rowcolors{2}{}{background!90!textColor}\array}{\endarray}

%%This is to help with formatting on future title pages.
\newenvironment{sectionOutcomes}{}{}



%% Flowchart stuff
%\tikzstyle{startstop} = [rectangle, rounded corners, minimum width=3cm, minimum height=1cm,text centered, draw=black]
%\tikzstyle{question} = [rectangle, minimum width=3cm, minimum height=1cm, text centered, draw=black]
%\tikzstyle{decision} = [trapezium, trapezium left angle=70, trapezium right angle=110, minimum width=3cm, minimum height=1cm, text centered, draw=black]
%\tikzstyle{question} = [rectangle, rounded corners, minimum width=3cm, minimum height=1cm,text centered, draw=black]
%\tikzstyle{process} = [rectangle, minimum width=3cm, minimum height=1cm, text centered, draw=black]
%\tikzstyle{decision} = [trapezium, trapezium left angle=70, trapezium right angle=110, minimum width=3cm, minimum height=1cm, text centered, draw=black]


\author{Jim Talamo}
\license{Creative Commons 3.0 By-NC}


\outcome{Find the critical points of a function of two variables.}
\outcome{Use the second derivative test to identify local extrema.}
\outcome{Find local extrema of functions of two variables.}

\begin{document}
\begin{exercise}

Consider the function $F(x,y) = x^3+9x^2 + y^3-9y +1$, which is defined for all $(x,y) \in \R^2$.  

The function $F(x,y)$ has $\answer{4}$ critical points over $\R^2$.

\begin{itemize}
\item This function \wordChoice{\choice[correct]{has}\choice{does not have}} relative extrema over $\R^2$.
\item This function \wordChoice{\choice{has}\choice[correct]{does not have}} absolute extrema over $\R^2$.
\end{itemize}

%%%%%%%%%%%%%%%%%%%%%%%%%%%%%%%%%%%%%%%%%%%%%%%%%%%%%%%%
\begin{exercise}
The $x$-values for the critical points are $x=\answer{-6}$ and $x=\answer{0}$ (type the smaller $x$-value first) and the $y$-values for the critical points are $y=\answer{-\sqrt{3}}$ and $y= \answer{\sqrt{3}}$ (type the smaller $y$-value first).

\begin{exercise}
Now, let $R = \left\{ (x,y) \in \R^2 ~ \bigg| ~ x^2 \leq y \leq 4 \right\}$.

Select all of the following that are critical points for $F(x,y)$ over $\R^2$.

\begin{selectAll}
\choice[correct]{$(-6,-\sqrt{3})$}
\choice[correct]{$(-6,\sqrt{3})$}
\choice[correct]{$(0,-\sqrt{3})$}
\choice[correct]{$(0,\sqrt{3})$}
\end{selectAll}


Select all of the following that are critical points for $F(x,y)$ that lie in $R$.

\begin{selectAll}
\choice{$(-6,-\sqrt{3})$}
\choice{$(-6,\sqrt{3})$}
\choice{$(0,-\sqrt{3})$}
\choice[correct]{$(0,\sqrt{3})$}
\end{selectAll}


The function $F(x,y)$ \wordChoice{\choice[correct]{attains}\choice{does not attain}} any absolute extrema over $R$ because:

\begin{selectAll}
\choice{all functions that have critical points have absolute extrema.}
\choice{no functions that have critical points have absolute extrema.}
\choice{all continuous functions have absolute extrema over any subset of $\R^2$.}
\choice{all continuous functions have absolute extrema over any bounded subset of $\R^2$.}
\choice{all continuous functions have absolute extrema over any closed subset of $\R^2$.}
\choice[correct]{all continuous functions have absolute extrema over any closed and bounded subset of $\R^2$.}
\end{selectAll}


	%%%%%%%%%%%%%%%%%%%%%%%%%%%%%%%%%%%%%%%%%%%%%%%%%%%%%%%%
\begin{exercise}
The absolute extrema could occur at the critical point in $R$ or along the boundary.  

We first note that at the critical point $\left(0,\sqrt{3}\right)$, we have $F\left(0,\sqrt{3}\right) = \answer{-6 \sqrt{3}+1}$.

To analyze around the boundary, first sketch $R$, and click on the Hint below to verify if your sketch is correct.

\begin{hint}
The region $R$ is shown below, along with the critical points of $F(x,y)$.
\begin{image}
\begin{tikzpicture}

\begin{axis}
	[
	domain=-7:3, ymax=5,xmax=3, ymin=-2, xmin=-7,
	axis lines=center, xlabel=$x$, ylabel=$y$,
	xtick={-6,-4,...,2},
	ytick={2,4},
	every axis y label/.style={at=(current axis.above origin),anchor=south},
	every axis x label/.style={at=(current axis.right of origin),anchor=west},
	axis on top,
	]

	\addplot [draw=penColor,very thick, smooth,domain=-2:2] {x^2};	
	\addplot [draw=penColor2,very thick, smooth,domain=-2:2] {4};
	
	\draw [penColor,thick] (1,-1) -- (1,0);
	
	\addplot [name path=A,domain=-2:2,draw=none] {x^2};
	\addplot [name path=B,domain=-2:2,draw=none] {4};
	\addplot [fillp] fill between[of=A and B];
	
	\node at (axis cs:1,4.5) [penColor2] {$y=4$};
	\node at (axis cs:2,1) [penColor] {$y = x^2$};
	
	\addplot[color=black,fill=black,only marks,mark=x] coordinates{(-6,-1.73)};
	\addplot[color=black,fill=black,only marks,mark=x] coordinates{(-6,1.73)};
	\addplot[color=black,fill=black,only marks,mark=x] coordinates{(0,-1.73)};
	\addplot[color=black,fill=black,only marks,mark=*] coordinates{(0,1.73)};
	
	\node at (axis cs:-4.5,-1.7) [black] {\footnotesize $\left(-6,-\sqrt{3}\right)$};
	\node at (axis cs:-4.5,1.7) [black] {\footnotesize $\left(-6,\sqrt{3}\right)$};	
	\node at (axis cs:1.2,-1.7) [black] {\footnotesize $\left(0,-\sqrt{3}\right)$};	
	\node at (axis cs:.8,2.3) [black] {\footnotesize $\left(0,\sqrt{3}\right)$};	
\end{axis}

\end{tikzpicture}
\end{image}
\end{hint}


$\bullet$ Along the boundary $y=x^2$ for $-2 \leq x \leq 2$, we have

\[
F(x,y) = F(x,x^2) = \answer{  x^3 + x^6 +1 }
\]

Moving forward, let's call the function above $f(x)$ (i.e. define $f(x) := F(x,x^2)$).  The critical points here occur when $x=\answer{-\sqrt[3]{2}}$, $x=\answer{0}$, and $x=\answer{\sqrt[3]{2}}$ (list the critical points in the order of increasing $x$).

\begin{hint}
The critical points will occur when $f'(x) = \ddx \left[  x^3 + x^6 +1 \right] =0$.
\end{hint}

\begin{exercise}
Checking the value of $F(x,y)$ at each of these as well as at the endpoints gives the following values.

\begin{itemize}
\item $f\left(-2\right) = \answer{57}$
\item $f\left(-\sqrt[3]{2}\right) = \answer{3}$
\item $f\left(0\right) = \answer{1}$
\item $f\left(\sqrt[3]{2}  \right) = \answer{7}$
\item $f\left(2 \right) = \answer{73}$
\end{itemize}

\end{exercise}
%%%%%%%%

$\bullet$ Along the boundary $y=4$ for $-2 \leq x \leq 2$, we have

\[
F(x,y) = F(x,4) = \answer{  x^3+9x^2 + 29 }
\]

Moving forward,  let's call the function above $g(x)$ (i.e. define $g(x) := F(x,4)$).    The critical points here occur when $x=\answer{0}$.

\begin{hint}
The critical points will occur when $g'(x) = \ddx \left[ x^3+9x^2 + 29 \right] =0$.  Note this means that $x=0$ or $x=-3$, the latter of which is not along the boundary since we need $-2 \leq x \leq 2$.

\end{hint}

\begin{exercise}
Checking the value of $F(x,y)$ at each of these as well as at the endpoints gives the following values.

\begin{itemize}
\item $g\left(-2\right) = \answer{57}$
\item $g\left(0 \right) = \answer{29}$
\item $g\left(2 \right) = \answer{73}$
\end{itemize}


Note that $f(-2) = g(-2)$ and $f(2) = g(2)$.  This is to be expected; take a step back and think why this makes sense.  If you still have questions, ask your instructor!

\end{exercise}

From our list of candidates along each part of the boundary and remembering the value the function takes at the interior critical point, we see that the absolute extrema are as follows.

\begin{itemize}
\item The absolute maximum value $F(x,y)$ takes on $R$ is $ \answer{73}  $ and occurs at $(x,y) = \left( \answer{2} , \answer{4} \right)$.
\item The absolute minimum value $F(x,y)$ takes on $R$ is $ \answer{-6 \sqrt{3}+1}  $ and occurs at $(x,y) = \left( \answer{0} , \answer{\sqrt{3}} \right)$.
\end{itemize}
\end{exercise}
	%%%%%%%%%%%%%%%%%%%%%%%%%%%%%%%%%%%%%%%%%%%%%%%%%%%%%%%%

\end{exercise}

%%%%%%%%%%%%%%%%%%%%%%%%%%%%%%%%%%%%%%%%%%%%%%%%%%%%%%%%



\end{exercise}
\end{exercise}
\end{document}
