\documentclass[handout,hints,noauthor,nooutcomes]{ximera}

%\usepackage{todonotes}
%\usepackage{mathtools} %% Required for wide table Curl and Greens
%\usepackage{cuted} %% Required for wide table Curl and Greens
\newcommand{\todo}{}

\usepackage{esint} % for \oiint
\ifxake%%https://math.meta.stackexchange.com/questions/9973/how-do-you-render-a-closed-surface-double-integral
\renewcommand{\oiint}{{\large\bigcirc}\kern-1.56em\iint}
\fi


\graphicspath{
  {./}
  {ximeraTutorial/}
  {basicPhilosophy/}
  {functionsOfSeveralVariables/}
  {normalVectors/}
  {lagrangeMultipliers/}
  {vectorFields/}
  {greensTheorem/}
  {shapeOfThingsToCome/}
  {dotProducts/}
  {partialDerivativesAndTheGradientVector/}
  {../productAndQuotientRules/exercises/}
  {../normalVectors/exercisesParametricPlots/}
  {../continuityOfFunctionsOfSeveralVariables/exercises/}
  {../partialDerivativesAndTheGradientVector/exercises/}
  {../directionalDerivativeAndChainRule/exercises/}
  {../commonCoordinates/exercisesCylindricalCoordinates/}
  {../commonCoordinates/exercisesSphericalCoordinates/}
  {../greensTheorem/exercisesCurlAndLineIntegrals/}
  {../greensTheorem/exercisesDivergenceAndLineIntegrals/}
  {../shapeOfThingsToCome/exercisesDivergenceTheorem/}
  {../greensTheorem/}
  {../shapeOfThingsToCome/}
  {../separableDifferentialEquations/exercises/}
  {vectorFields/}
}

\newcommand{\mooculus}{\textsf{\textbf{MOOC}\textnormal{\textsf{ULUS}}}}

\usepackage{tkz-euclide}\usepackage{tikz}
\usepackage{tikz-cd}
\usetikzlibrary{arrows}
\tikzset{>=stealth,commutative diagrams/.cd,
  arrow style=tikz,diagrams={>=stealth}} %% cool arrow head
\tikzset{shorten <>/.style={ shorten >=#1, shorten <=#1 } } %% allows shorter vectors

\usetikzlibrary{backgrounds} %% for boxes around graphs
\usetikzlibrary{shapes,positioning}  %% Clouds and stars
\usetikzlibrary{matrix} %% for matrix
\usepgfplotslibrary{polar} %% for polar plots
\usepgfplotslibrary{fillbetween} %% to shade area between curves in TikZ
\usetkzobj{all}
\usepackage[makeroom]{cancel} %% for strike outs
%\usepackage{mathtools} %% for pretty underbrace % Breaks Ximera
%\usepackage{multicol}
\usepackage{pgffor} %% required for integral for loops



%% http://tex.stackexchange.com/questions/66490/drawing-a-tikz-arc-specifying-the-center
%% Draws beach ball
\tikzset{pics/carc/.style args={#1:#2:#3}{code={\draw[pic actions] (#1:#3) arc(#1:#2:#3);}}}



\usepackage{array}
\setlength{\extrarowheight}{+.1cm}
\newdimen\digitwidth
\settowidth\digitwidth{9}
\def\divrule#1#2{
\noalign{\moveright#1\digitwidth
\vbox{\hrule width#2\digitwidth}}}





\newcommand{\RR}{\mathbb R}
\newcommand{\R}{\mathbb R}
\newcommand{\N}{\mathbb N}
\newcommand{\Z}{\mathbb Z}

\newcommand{\sagemath}{\textsf{SageMath}}


%\renewcommand{\d}{\,d\!}
\renewcommand{\d}{\mathop{}\!d}
\newcommand{\dd}[2][]{\frac{\d #1}{\d #2}}
\newcommand{\pp}[2][]{\frac{\partial #1}{\partial #2}}
\renewcommand{\l}{\ell}
\newcommand{\ddx}{\frac{d}{\d x}}

\newcommand{\zeroOverZero}{\ensuremath{\boldsymbol{\tfrac{0}{0}}}}
\newcommand{\inftyOverInfty}{\ensuremath{\boldsymbol{\tfrac{\infty}{\infty}}}}
\newcommand{\zeroOverInfty}{\ensuremath{\boldsymbol{\tfrac{0}{\infty}}}}
\newcommand{\zeroTimesInfty}{\ensuremath{\small\boldsymbol{0\cdot \infty}}}
\newcommand{\inftyMinusInfty}{\ensuremath{\small\boldsymbol{\infty - \infty}}}
\newcommand{\oneToInfty}{\ensuremath{\boldsymbol{1^\infty}}}
\newcommand{\zeroToZero}{\ensuremath{\boldsymbol{0^0}}}
\newcommand{\inftyToZero}{\ensuremath{\boldsymbol{\infty^0}}}



\newcommand{\numOverZero}{\ensuremath{\boldsymbol{\tfrac{\#}{0}}}}
\newcommand{\dfn}{\textbf}
%\newcommand{\unit}{\,\mathrm}
\newcommand{\unit}{\mathop{}\!\mathrm}
\newcommand{\eval}[1]{\bigg[ #1 \bigg]}
\newcommand{\seq}[1]{\left( #1 \right)}
\renewcommand{\epsilon}{\varepsilon}
\renewcommand{\phi}{\varphi}


\renewcommand{\iff}{\Leftrightarrow}

\DeclareMathOperator{\arccot}{arccot}
\DeclareMathOperator{\arcsec}{arcsec}
\DeclareMathOperator{\arccsc}{arccsc}
\DeclareMathOperator{\si}{Si}
\DeclareMathOperator{\scal}{scal}
\DeclareMathOperator{\sign}{sign}


%% \newcommand{\tightoverset}[2]{% for arrow vec
%%   \mathop{#2}\limits^{\vbox to -.5ex{\kern-0.75ex\hbox{$#1$}\vss}}}
\newcommand{\arrowvec}[1]{{\overset{\rightharpoonup}{#1}}}
%\renewcommand{\vec}[1]{\arrowvec{\mathbf{#1}}}
\renewcommand{\vec}[1]{{\overset{\boldsymbol{\rightharpoonup}}{\mathbf{#1}}}\hspace{0in}}

\newcommand{\point}[1]{\left(#1\right)} %this allows \vector{ to be changed to \vector{ with a quick find and replace
\newcommand{\pt}[1]{\mathbf{#1}} %this allows \vec{ to be changed to \vec{ with a quick find and replace
\newcommand{\Lim}[2]{\lim_{\point{#1} \to \point{#2}}} %Bart, I changed this to point since I want to use it.  It runs through both of the exercise and exerciseE files in limits section, which is why it was in each document to start with.

\DeclareMathOperator{\proj}{\mathbf{proj}}
\newcommand{\veci}{{\boldsymbol{\hat{\imath}}}}
\newcommand{\vecj}{{\boldsymbol{\hat{\jmath}}}}
\newcommand{\veck}{{\boldsymbol{\hat{k}}}}
\newcommand{\vecl}{\vec{\boldsymbol{\l}}}
\newcommand{\uvec}[1]{\mathbf{\hat{#1}}}
\newcommand{\utan}{\mathbf{\hat{t}}}
\newcommand{\unormal}{\mathbf{\hat{n}}}
\newcommand{\ubinormal}{\mathbf{\hat{b}}}

\newcommand{\dotp}{\bullet}
\newcommand{\cross}{\boldsymbol\times}
\newcommand{\grad}{\boldsymbol\nabla}
\newcommand{\divergence}{\grad\dotp}
\newcommand{\curl}{\grad\cross}
%\DeclareMathOperator{\divergence}{divergence}
%\DeclareMathOperator{\curl}[1]{\grad\cross #1}
\newcommand{\lto}{\mathop{\longrightarrow\,}\limits}

\renewcommand{\bar}{\overline}

\colorlet{textColor}{black}
\colorlet{background}{white}
\colorlet{penColor}{blue!50!black} % Color of a curve in a plot
\colorlet{penColor2}{red!50!black}% Color of a curve in a plot
\colorlet{penColor3}{red!50!blue} % Color of a curve in a plot
\colorlet{penColor4}{green!50!black} % Color of a curve in a plot
\colorlet{penColor5}{orange!80!black} % Color of a curve in a plot
\colorlet{penColor6}{yellow!70!black} % Color of a curve in a plot
\colorlet{fill1}{penColor!20} % Color of fill in a plot
\colorlet{fill2}{penColor2!20} % Color of fill in a plot
\colorlet{fillp}{fill1} % Color of positive area
\colorlet{filln}{penColor2!20} % Color of negative area
\colorlet{fill3}{penColor3!20} % Fill
\colorlet{fill4}{penColor4!20} % Fill
\colorlet{fill5}{penColor5!20} % Fill
\colorlet{gridColor}{gray!50} % Color of grid in a plot

\newcommand{\surfaceColor}{violet}
\newcommand{\surfaceColorTwo}{redyellow}
\newcommand{\sliceColor}{greenyellow}




\pgfmathdeclarefunction{gauss}{2}{% gives gaussian
  \pgfmathparse{1/(#2*sqrt(2*pi))*exp(-((x-#1)^2)/(2*#2^2))}%
}


%%%%%%%%%%%%%
%% Vectors
%%%%%%%%%%%%%

%% Simple horiz vectors
\renewcommand{\vector}[1]{\left\langle #1\right\rangle}


%% %% Complex Horiz Vectors with angle brackets
%% \makeatletter
%% \renewcommand{\vector}[2][ , ]{\left\langle%
%%   \def\nextitem{\def\nextitem{#1}}%
%%   \@for \el:=#2\do{\nextitem\el}\right\rangle%
%% }
%% \makeatother

%% %% Vertical Vectors
%% \def\vector#1{\begin{bmatrix}\vecListA#1,,\end{bmatrix}}
%% \def\vecListA#1,{\if,#1,\else #1\cr \expandafter \vecListA \fi}

%%%%%%%%%%%%%
%% End of vectors
%%%%%%%%%%%%%

%\newcommand{\fullwidth}{}
%\newcommand{\normalwidth}{}



%% makes a snazzy t-chart for evaluating functions
%\newenvironment{tchart}{\rowcolors{2}{}{background!90!textColor}\array}{\endarray}

%%This is to help with formatting on future title pages.
\newenvironment{sectionOutcomes}{}{}



%% Flowchart stuff
%\tikzstyle{startstop} = [rectangle, rounded corners, minimum width=3cm, minimum height=1cm,text centered, draw=black]
%\tikzstyle{question} = [rectangle, minimum width=3cm, minimum height=1cm, text centered, draw=black]
%\tikzstyle{decision} = [trapezium, trapezium left angle=70, trapezium right angle=110, minimum width=3cm, minimum height=1cm, text centered, draw=black]
%\tikzstyle{question} = [rectangle, rounded corners, minimum width=3cm, minimum height=1cm,text centered, draw=black]
%\tikzstyle{process} = [rectangle, minimum width=3cm, minimum height=1cm, text centered, draw=black]
%\tikzstyle{decision} = [trapezium, trapezium left angle=70, trapezium right angle=110, minimum width=3cm, minimum height=1cm, text centered, draw=black]


\author{Bart Snapp}

\title[Collaborate:]{Derivatives of vector-valued functions}

\begin{document}
\begin{abstract}
  We think about the derivative of vector-valued functions.
\end{abstract}
\maketitle

\textbf{Work in groups of 3--4, writing your answers on a separate
  sheet of paper.}


Previously in calculus course you learned the following metaphor for
the derivatives:
\begin{quote}
  Given a function $f:\R\to\R$, the derivative of $f$ is the slope of
  the tangent line at any point on the graph $y = f(x)$.
\end{quote}

This is really a great metaphor for functions that map from $\R$ to
$\R$. However, now we are studying vector-valued functions. A new
metaphor is needed:
\begin{quote}
  Given a vector-valued function $\vec{f}:\R^n\to\R$, the derivative
  of $\vec{f}$ is a tangent vector at any point on the graph of
  $\vec{f}$.
\end{quote}
Let's see if we can figure what this is saying.

\section{Lines}

Suppose you have a line given by the vector valued function $\vecl$.

\begin{image}
  \begin{tikzpicture}
    \begin{axis}%
      [
	xmin=-4,xmax=5,
        ymin=-2,ymax=3,
        xlabel=$x$,ylabel=$y$,
        axis lines=center,
        every axis y label/.style={at=(current axis.above origin),anchor=south},
        every axis x label/.style={at=(current axis.right of origin),anchor=west},
        clip=false,
	grid =major,
        width=9cm,
        height=5cm,
        xtick={-4,-3,...,5},
        ytick={-2,-1,...,3},
      ]
      \addplot[penColor,ultra thick,domain=-4:5] {
        -x/3+1
      };
        \addplot[color=penColor,fill=penColor,only marks,mark=*] coordinates{(3,0)};  %% closed hole
        \addplot[color=penColor,fill=penColor,only marks,mark=*] coordinates{(-3,2)};  %% closed hole
        \node[penColor,above] at (axis cs: 3,0) {$\vecl(0)$};
        \node[penColor,above right] at (axis cs: -3,2) {$\vecl(2)$};
      \end{axis}
    \end{tikzpicture}
\end{image}


\begin{problem}
  Pencil-in some tangent vectors for $\vecl$ above. 
\end{problem}

\begin{problem}
  Someone has plotted $\vecl'$ below:
  \begin{image}
  \begin{tikzpicture}
    \begin{axis}%
      [
	xmin=-4,xmax=5,
        ymin=-2,ymax=3,
        xlabel=$x$,ylabel=$y$,
        axis lines=center,
        every axis y label/.style={at=(current axis.above origin),anchor=south},
        every axis x label/.style={at=(current axis.right of origin),anchor=west},
        clip=false,
	grid =major,
        width=9cm,
        height=5cm,
        xtick={-4,-3,...,5},
        ytick={-2,-1,...,3},
      ]
      \addplot[color=penColor,fill=penColor,only marks,mark=*] coordinates{(-3,1)};  %% closed hole
      \node[penColor,above] at (axis cs: -3,1) {$\vecl'$};
    \end{axis}
    \end{tikzpicture}
  \end{image}
  Make sense of this plot. Explain what is going on to someone else.
\end{problem}


\section{Circles}

The a circle of radius $2$ centered at $(3,1$ is given by
\[
\vec{c}(t) = \vector{3 + 2 \cos(t), 1+2\sin(t)}
\]
and here is a plot:

\begin{image}
  \begin{tikzpicture}
    \begin{axis}%
      [
	xmin=-4,xmax=6,
        ymin=-2,ymax=4,
        xlabel=$x$,ylabel=$y$,
        axis lines=center,
        every axis y label/.style={at=(current axis.above origin),anchor=south},
        every axis x label/.style={at=(current axis.right of origin),anchor=west},
        clip=false,
	grid =major,
        width=10cm,
        height=6.7cm,
        xtick={-4,-3,...,6},
        ytick={-2,-1,...,4},
      ]
      \addplot[penColor,smooth,ultra thick,domain=0:360] (
       {3+2*cos(x)},{1+2*sin(x)}
      );
    \end{axis}
    \end{tikzpicture}
\end{image}

\begin{problem}
  Pencil-in some tangent vectors for $\vec{c}$ above. 
\end{problem}



\begin{problem}
  Someone has plotted $\vec{c}'$ below:
  \begin{image}
    \begin{tikzpicture}
      \begin{axis}%
        [
	xmin=-4,xmax=6,
        ymin=-3,ymax=3,
        xlabel=$x$,ylabel=$y$,
        axis lines=center,
        every axis y label/.style={at=(current axis.above origin),anchor=south},
        every axis x label/.style={at=(current axis.right of origin),anchor=west},
        clip=false,
	grid =major,
        width=10cm,
        height=6.7cm,
        xtick={-4,-3,...,6},
        ytick={-3,-2,...,3},
      ]
      \addplot[penColor,smooth,ultra thick,domain=0:360] (
       {-2*sin(x)},{2*cos(x)}
      );
    \end{axis}
    \end{tikzpicture}
  \end{image}
  Make sense of this plot. Explain what is going on to someone else.
\end{problem}

\section{Projectile motion}

Vector-valued functions are excellent for modeling projectile
motion. The function below models the path of a calculus book being
thrown from an initial height of $1\unit{m}$ at an initial velocity of
$5\unit{m}/{s}$ at a $45^\circ$ angle:
\[
\vec{p}(t)=\vector{\frac{5t}{\sqrt{2}}, 1+ \frac{5t}{\sqrt{2}}-5t^2}
\]
For your viewing pleasure here is a plot:
  \begin{image}
    \begin{tikzpicture}
      \begin{axis}%
        [
	xmin=-.5,xmax=4,
        ymin=-.5,ymax=2,
        xlabel=$x$,ylabel=$y$,
        axis lines=center,
        every axis y label/.style={at=(current axis.above origin),anchor=south},
        every axis x label/.style={at=(current axis.right of origin),anchor=west},
        clip=false,
	grid =major,
        width=10cm,
        height=6.7cm,
        xtick={0,1,...,4},
        ytick={0,1,...,2},
      ]
      \addplot[penColor,smooth,ultra thick,domain=0:1] (
       {5*x/sqrt(2)},{1+ 5*x/sqrt(2)-5*x^2}
      );
      \end{axis}
    \end{tikzpicture}
  \end{image}

\begin{problem}
  Pencil-in some tangent vectors for $\vec{p}$ above. 
\end{problem}



\begin{problem}
  Someone has plotted $\vec{p}'$ below:
  \begin{image}
    \begin{tikzpicture}
      \begin{axis}%
        [
	xmin=-.5,xmax=4,
        ymin=-.5,ymax=2,
        xlabel=$x$,ylabel=$y$,
        axis lines=center,
        every axis y label/.style={at=(current axis.above origin),anchor=south},
        every axis x label/.style={at=(current axis.right of origin),anchor=west},
        clip=false,
	grid =major,
        width=10cm,
        height=6.7cm,
        xtick={0,1,...,4},
        ytick={0,1,...,2},
      ]
      \addplot[penColor,smooth,ultra thick,domain=.1:.4] (
              {5/sqrt(2)},{5/sqrt(2)-10*x}
      );
      \end{axis}
    \end{tikzpicture}
  \end{image}
  Make sense of this plot (note, this is not a complete plot). Explain
  what is going on to someone else.
\end{problem}

\begin{problem}
  If we were to plot $\vec{p}''$ what would it look like?
\end{problem}

\begin{problem}
  Rather than plotting $\vec{p}''$, what should you do?
\end{problem}


\section{The moral of the story}


The moral of the story is this: When studying, functions from $\R$ to
$\R$, it makes a lot of sense to plot their derivatives. When dealing
with vector-valued functions, plotting their derivatives might not be
the best idea. Instead you should be plotting tangent vectors.


\end{document}
