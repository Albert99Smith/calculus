\documentclass[]{ximera}
%handout:  for handout version with no solutions or instructor notes
%handout,instructornotes:  for instructor version with just problems and notes, no solutions
%noinstructornotes:  shows only problem and solutions

%% handout
%% space
%% newpage
%% numbers
%% nooutcomes

%I added the commands here so that I would't have to keep looking them up
%\newcommand{\RR}{\mathbb R}
%\renewcommand{\d}{\,d}
%\newcommand{\dd}[2][]{\frac{d #1}{d #2}}
%\renewcommand{\l}{\ell}
%\newcommand{\ddx}{\frac{d}{dx}}
%\everymath{\displaystyle}
%\newcommand{\dfn}{\textbf}
%\newcommand{\eval}[1]{\bigg[ #1 \bigg]}

%\begin{image}
%\includegraphics[trim= 170 420 250 180]{Figure1.pdf}
%\end{image}

%add a ``.'' below when used in a specific directory.

%\usepackage{todonotes}
%\usepackage{mathtools} %% Required for wide table Curl and Greens
%\usepackage{cuted} %% Required for wide table Curl and Greens
\newcommand{\todo}{}

\usepackage{esint} % for \oiint
\ifxake%%https://math.meta.stackexchange.com/questions/9973/how-do-you-render-a-closed-surface-double-integral
\renewcommand{\oiint}{{\large\bigcirc}\kern-1.56em\iint}
\fi


\graphicspath{
  {./}
  {ximeraTutorial/}
  {basicPhilosophy/}
  {functionsOfSeveralVariables/}
  {normalVectors/}
  {lagrangeMultipliers/}
  {vectorFields/}
  {greensTheorem/}
  {shapeOfThingsToCome/}
  {dotProducts/}
  {partialDerivativesAndTheGradientVector/}
  {../productAndQuotientRules/exercises/}
  {../normalVectors/exercisesParametricPlots/}
  {../continuityOfFunctionsOfSeveralVariables/exercises/}
  {../partialDerivativesAndTheGradientVector/exercises/}
  {../directionalDerivativeAndChainRule/exercises/}
  {../commonCoordinates/exercisesCylindricalCoordinates/}
  {../commonCoordinates/exercisesSphericalCoordinates/}
  {../greensTheorem/exercisesCurlAndLineIntegrals/}
  {../greensTheorem/exercisesDivergenceAndLineIntegrals/}
  {../shapeOfThingsToCome/exercisesDivergenceTheorem/}
  {../greensTheorem/}
  {../shapeOfThingsToCome/}
  {../separableDifferentialEquations/exercises/}
  {vectorFields/}
}

\newcommand{\mooculus}{\textsf{\textbf{MOOC}\textnormal{\textsf{ULUS}}}}

\usepackage{tkz-euclide}\usepackage{tikz}
\usepackage{tikz-cd}
\usetikzlibrary{arrows}
\tikzset{>=stealth,commutative diagrams/.cd,
  arrow style=tikz,diagrams={>=stealth}} %% cool arrow head
\tikzset{shorten <>/.style={ shorten >=#1, shorten <=#1 } } %% allows shorter vectors

\usetikzlibrary{backgrounds} %% for boxes around graphs
\usetikzlibrary{shapes,positioning}  %% Clouds and stars
\usetikzlibrary{matrix} %% for matrix
\usepgfplotslibrary{polar} %% for polar plots
\usepgfplotslibrary{fillbetween} %% to shade area between curves in TikZ
\usetkzobj{all}
\usepackage[makeroom]{cancel} %% for strike outs
%\usepackage{mathtools} %% for pretty underbrace % Breaks Ximera
%\usepackage{multicol}
\usepackage{pgffor} %% required for integral for loops



%% http://tex.stackexchange.com/questions/66490/drawing-a-tikz-arc-specifying-the-center
%% Draws beach ball
\tikzset{pics/carc/.style args={#1:#2:#3}{code={\draw[pic actions] (#1:#3) arc(#1:#2:#3);}}}



\usepackage{array}
\setlength{\extrarowheight}{+.1cm}
\newdimen\digitwidth
\settowidth\digitwidth{9}
\def\divrule#1#2{
\noalign{\moveright#1\digitwidth
\vbox{\hrule width#2\digitwidth}}}





\newcommand{\RR}{\mathbb R}
\newcommand{\R}{\mathbb R}
\newcommand{\N}{\mathbb N}
\newcommand{\Z}{\mathbb Z}

\newcommand{\sagemath}{\textsf{SageMath}}


%\renewcommand{\d}{\,d\!}
\renewcommand{\d}{\mathop{}\!d}
\newcommand{\dd}[2][]{\frac{\d #1}{\d #2}}
\newcommand{\pp}[2][]{\frac{\partial #1}{\partial #2}}
\renewcommand{\l}{\ell}
\newcommand{\ddx}{\frac{d}{\d x}}

\newcommand{\zeroOverZero}{\ensuremath{\boldsymbol{\tfrac{0}{0}}}}
\newcommand{\inftyOverInfty}{\ensuremath{\boldsymbol{\tfrac{\infty}{\infty}}}}
\newcommand{\zeroOverInfty}{\ensuremath{\boldsymbol{\tfrac{0}{\infty}}}}
\newcommand{\zeroTimesInfty}{\ensuremath{\small\boldsymbol{0\cdot \infty}}}
\newcommand{\inftyMinusInfty}{\ensuremath{\small\boldsymbol{\infty - \infty}}}
\newcommand{\oneToInfty}{\ensuremath{\boldsymbol{1^\infty}}}
\newcommand{\zeroToZero}{\ensuremath{\boldsymbol{0^0}}}
\newcommand{\inftyToZero}{\ensuremath{\boldsymbol{\infty^0}}}



\newcommand{\numOverZero}{\ensuremath{\boldsymbol{\tfrac{\#}{0}}}}
\newcommand{\dfn}{\textbf}
%\newcommand{\unit}{\,\mathrm}
\newcommand{\unit}{\mathop{}\!\mathrm}
\newcommand{\eval}[1]{\bigg[ #1 \bigg]}
\newcommand{\seq}[1]{\left( #1 \right)}
\renewcommand{\epsilon}{\varepsilon}
\renewcommand{\phi}{\varphi}


\renewcommand{\iff}{\Leftrightarrow}

\DeclareMathOperator{\arccot}{arccot}
\DeclareMathOperator{\arcsec}{arcsec}
\DeclareMathOperator{\arccsc}{arccsc}
\DeclareMathOperator{\si}{Si}
\DeclareMathOperator{\scal}{scal}
\DeclareMathOperator{\sign}{sign}


%% \newcommand{\tightoverset}[2]{% for arrow vec
%%   \mathop{#2}\limits^{\vbox to -.5ex{\kern-0.75ex\hbox{$#1$}\vss}}}
\newcommand{\arrowvec}[1]{{\overset{\rightharpoonup}{#1}}}
%\renewcommand{\vec}[1]{\arrowvec{\mathbf{#1}}}
\renewcommand{\vec}[1]{{\overset{\boldsymbol{\rightharpoonup}}{\mathbf{#1}}}\hspace{0in}}

\newcommand{\point}[1]{\left(#1\right)} %this allows \vector{ to be changed to \vector{ with a quick find and replace
\newcommand{\pt}[1]{\mathbf{#1}} %this allows \vec{ to be changed to \vec{ with a quick find and replace
\newcommand{\Lim}[2]{\lim_{\point{#1} \to \point{#2}}} %Bart, I changed this to point since I want to use it.  It runs through both of the exercise and exerciseE files in limits section, which is why it was in each document to start with.

\DeclareMathOperator{\proj}{\mathbf{proj}}
\newcommand{\veci}{{\boldsymbol{\hat{\imath}}}}
\newcommand{\vecj}{{\boldsymbol{\hat{\jmath}}}}
\newcommand{\veck}{{\boldsymbol{\hat{k}}}}
\newcommand{\vecl}{\vec{\boldsymbol{\l}}}
\newcommand{\uvec}[1]{\mathbf{\hat{#1}}}
\newcommand{\utan}{\mathbf{\hat{t}}}
\newcommand{\unormal}{\mathbf{\hat{n}}}
\newcommand{\ubinormal}{\mathbf{\hat{b}}}

\newcommand{\dotp}{\bullet}
\newcommand{\cross}{\boldsymbol\times}
\newcommand{\grad}{\boldsymbol\nabla}
\newcommand{\divergence}{\grad\dotp}
\newcommand{\curl}{\grad\cross}
%\DeclareMathOperator{\divergence}{divergence}
%\DeclareMathOperator{\curl}[1]{\grad\cross #1}
\newcommand{\lto}{\mathop{\longrightarrow\,}\limits}

\renewcommand{\bar}{\overline}

\colorlet{textColor}{black}
\colorlet{background}{white}
\colorlet{penColor}{blue!50!black} % Color of a curve in a plot
\colorlet{penColor2}{red!50!black}% Color of a curve in a plot
\colorlet{penColor3}{red!50!blue} % Color of a curve in a plot
\colorlet{penColor4}{green!50!black} % Color of a curve in a plot
\colorlet{penColor5}{orange!80!black} % Color of a curve in a plot
\colorlet{penColor6}{yellow!70!black} % Color of a curve in a plot
\colorlet{fill1}{penColor!20} % Color of fill in a plot
\colorlet{fill2}{penColor2!20} % Color of fill in a plot
\colorlet{fillp}{fill1} % Color of positive area
\colorlet{filln}{penColor2!20} % Color of negative area
\colorlet{fill3}{penColor3!20} % Fill
\colorlet{fill4}{penColor4!20} % Fill
\colorlet{fill5}{penColor5!20} % Fill
\colorlet{gridColor}{gray!50} % Color of grid in a plot

\newcommand{\surfaceColor}{violet}
\newcommand{\surfaceColorTwo}{redyellow}
\newcommand{\sliceColor}{greenyellow}




\pgfmathdeclarefunction{gauss}{2}{% gives gaussian
  \pgfmathparse{1/(#2*sqrt(2*pi))*exp(-((x-#1)^2)/(2*#2^2))}%
}


%%%%%%%%%%%%%
%% Vectors
%%%%%%%%%%%%%

%% Simple horiz vectors
\renewcommand{\vector}[1]{\left\langle #1\right\rangle}


%% %% Complex Horiz Vectors with angle brackets
%% \makeatletter
%% \renewcommand{\vector}[2][ , ]{\left\langle%
%%   \def\nextitem{\def\nextitem{#1}}%
%%   \@for \el:=#2\do{\nextitem\el}\right\rangle%
%% }
%% \makeatother

%% %% Vertical Vectors
%% \def\vector#1{\begin{bmatrix}\vecListA#1,,\end{bmatrix}}
%% \def\vecListA#1,{\if,#1,\else #1\cr \expandafter \vecListA \fi}

%%%%%%%%%%%%%
%% End of vectors
%%%%%%%%%%%%%

%\newcommand{\fullwidth}{}
%\newcommand{\normalwidth}{}



%% makes a snazzy t-chart for evaluating functions
%\newenvironment{tchart}{\rowcolors{2}{}{background!90!textColor}\array}{\endarray}

%%This is to help with formatting on future title pages.
\newenvironment{sectionOutcomes}{}{}



%% Flowchart stuff
%\tikzstyle{startstop} = [rectangle, rounded corners, minimum width=3cm, minimum height=1cm,text centered, draw=black]
%\tikzstyle{question} = [rectangle, minimum width=3cm, minimum height=1cm, text centered, draw=black]
%\tikzstyle{decision} = [trapezium, trapezium left angle=70, trapezium right angle=110, minimum width=3cm, minimum height=1cm, text centered, draw=black]
%\tikzstyle{question} = [rectangle, rounded corners, minimum width=3cm, minimum height=1cm,text centered, draw=black]
%\tikzstyle{process} = [rectangle, minimum width=3cm, minimum height=1cm, text centered, draw=black]
%\tikzstyle{decision} = [trapezium, trapezium left angle=70, trapezium right angle=110, minimum width=3cm, minimum height=1cm, text centered, draw=black]


\author{Jim Talamo}

\outcome{Find Taylor Polynomials.}
\outcome{Understand the relationship between the derivatives of a function and the coefficients of its Taylor Polynomial.}

\title{Taylor Series of Common Functions and Rules of Compositions}

\begin{document}
\begin{abstract}
\end{abstract}
\maketitle

\vspace{-0.9in}

\section{Discussion Questions}

\begin{problem} 
%This will go on the previous handout eventually

Suppose that $\displaystyle f(x) = \sum_{k=0}^{\infty} \frac{k+1}{3^k}x^{2k}.$

\begin{itemize}
\item[I.] A student claims that $f(0)=0$.  Determine if the student is correct or incorrect.  If the student is incorrect, what should $f(0)$ be?
%explain why it isn't 0; students will plug in x=0 and not realize that there is a constant term.  I explain that k is part of the notation that is used to represent the function and must be handled before plugging in for x.
\item[II.] Find the radius of convergence of this series.
\item[III.] Does the series represented by $f(2)$ converge or diverge?  What about the series represented by $f'(2)$? 
\item[IV.] Would you expect that a higher order Taylor polynomial approximates $f(2)$ better or worse than a lower order one? 
%This may not be rigorous enough, but the idea I'm going for is that we only use TP to appx the function wtihin the IOC of the series
\end{itemize}

\begin{solution}
I. Writing out the first few terms of $f(x)$, we have
$$
f(x)= \frac{0+1}{3^0}x^{2\cdot 0} + \frac{1+1}{3^1} x^{2\cdot 1}+ \frac{2+1}{3^2} x^{2 \cdot 2} + \cdots = 1 + \frac{2}{3}x^2 + \frac{1}{3}x^4 + \cdots,
$$
where all remaining terms have degree strictly larger than $4$. From this, we see that $f(0)=1$.

II. We have
$$
\lim_{k \rightarrow \infty} \frac{k+2}{3^{k+1}} x^{2k+2} \cdot \frac{3^k}{(k+1) x^{2k}}= \lim_{k \rightarrow \infty} \frac{k+2}{k+1} \cdot \frac{1}{3} \cdot x^2 = \frac{x^2}{3}.
$$
By the Ratio Test, a sufficient condition for convergence of the series is $x^2/3 < 1$, which tells us that the radius of convergence of the power series is $\sqrt{3}$.

III. Since the center of the power series is $0$, and the radius of convergence is $\sqrt{3} < 2$, we know that $f(2)$ must diverge. The radius of convergence of $f'(x)$ is the same as that of $f(x)$, so $f'(2)$ also must diverge.

IV. Since the radius of convergence of $f(x)$ is $\sqrt{3} < 2$, $f(2)$ is undefined. It therefore doesn't make sense to try to approximate $f(2)$ (something which doesn't exist) by \emph{any} Taylor polynomial.
\end{solution}
\end{problem}


\begin{problem} 
The Taylor series centered at $x=0$ for $\cos(x)$ is $\displaystyle \sum_{k=0}^{\infty} \frac{(-1)^k}{(2k)!} x^{2k}$

\begin{itemize}
\item[I.]  A student reasons that it follows that the fourth degree Taylor polynomial centered at $x=0$ is: $$p_4(x) = 1-\dfrac{1}{2}x^2+\dfrac{1}{24}x^4.$$  Is the student correct?  If not, what should the correct fourth degree Taylor polynomial be?
\item[II.] The student now reasons that the fourth degree Taylor polynomial centered at $x=1$ is: $$p_4(x) = 1-\dfrac{1}{2}(x-1)^2+\dfrac{1}{24}(x-1)^4.$$  Is the student correct?  If not, how should the student find this polynomial?
\end{itemize}
%Jim's Note: show that the student is not correct by showing that cos(1) \neq the p_4(1) in II.  Then, just note that the student needs to compute the appropriate derivatives and cannot simply use the formula since it only holds at x=0.

\begin{solution}
I. Writing out the first few terms of the series explicitly, we have
$$
\sum_{k=0}^{\infty} \frac{(-1)^k}{(2k)!} x^{2k} = 1 - \frac{1}{2} x^2 + \frac{1}{24}x^4 + \cdots.
$$
Therefore the student is correct, by the definitions of Taylor polynomials and Taylor series.

II. If the student were correct, we would have $p_4(1) = \cos(1)$. However,
$$
p_4(1) = 1 - \frac{1}{2}(0)^2 + \frac{1}{24}(0)^4 = 1
$$
is not equal to 
$$
\cos(1) = 0.54\ldots.
$$
To compute the Taylor series centered at $1$, the student would need to recompute the coefficents using derivatives of $\cos(x)$ at $x=1$. 
\end{solution}
\end{problem}

\begin{problem} 
Write down the fifth degree Taylor polynomial centered at $x=0$ for the following and simplify your answers completely.  

\begin{itemize}
\item[I.]  $f(x) = \sin(2x)$
\item[II.] $f(x) = x^2 \sin(x)$
\item[III.] $f(x) = \sin(2x)+ x^2 \sin(x)$
\end{itemize}

\begin{solution}
We begin by recalling that the Taylor series for $f(x)=\sin(x)$ centered at $x=0$ is 
\begin{equation}\label{eqn:sin_series}
\sum_{k=0}^\infty \frac{(-1)^k}{(2k+1)!} x^{2k+1}.
\end{equation}

I. We find the Taylor series for $f(x) = \sin(2x)$ by replacing the variable $x$ in \eqref{eqn:sin_series} with $2x$, yielding the series
$$
\sum_{k=0}^\infty \frac{(-1)^k}{(2k+1)!} (2x)^{2k+1} = \sum_{k=0}^\infty \frac{(-1)^k2^{2k+1}}{(2k+1)!} x^{2k+1}.
$$
The fifth degree Taylor polynomial is therefore
$$
2x - \frac{4}{3}x^3 + \frac{4}{15} x^5.
$$

II. The Taylor series for $f(x) = x^2 \sin(x)$ is obtained by multiplying each term in \eqref{eqn:sin_series} by $x^2$, which gives
$$
\sum_{k=0}^\infty x^2 \cdot \frac{(-1)^k}{(2k+1)!} x^{2k+1} = \sum_{k=0}^\infty \frac{(-1)^k}{(2k+1)!} x^{2k+3}.
$$
This gives us the fifth degree Taylor polynomial
$$
x^3 - \frac{1}{3!} x^5.
$$

III. The Taylor series for $f(x) = \sin(2x)+ x^2 \sin(x)$ is the sum of the series from parts I and II:
$$
\sum_{k=0}^\infty \frac{(-1)^k 2^{2k+1}}{(2k+1)!} x^{2k+1} + \sum_{k=0}^\infty \frac{(-1)^k}{(2k+1)!} x^{2k+3}.
$$
The fifth degree Taylor polynomial for $f(x)$ is the sum of the fifth degree Taylor polynomials from parts I and II:
$$
2x - \frac{4}{3}x^3 + \frac{4}{15} x^5 + x^3 - \frac{1}{3!} x^5 = 2x - \frac{1}{3} x^3 + \frac{1}{10} x^5.
$$

\end{solution}
\end{problem}


\section{Group Work}
%%%%%%%%%%%%%%%%%%%%%%%%%%%%%%%%

\begin{problem} Give the radius of convergence for the following:
\begin{itemize}
\item[I.] The Taylor series centered at $x=0$ for $f(x) = \dfrac{5}{1-2x}$.
\item[II.] The Taylor series centered at $x=0$ for $f(x) = \dfrac{1}{6-x}$.
\item[III.] The Taylor series centered at $x=5$ for $f(x) = \dfrac{1}{6-x}$.
%Jim's Note: Do this from the properties, not the ratio test.  for III, rewrite \dfrac{1}{6-x} = \dfrac{1}{1-(x-5)} 
\end{itemize}

\begin{solution}
Recall that the Taylor series centered at $x=0$ for $g(x) = 1/(1-x)$ has radius of convergence $1$. 

I. The function $f(x) = 5/(1-2x)$ satisfies $f(x) = 5 g(2x)$. The radius of convergence is then determined by the inequality $|2x| < 1$, or $|x| < 1/2$, and we conclude that the radius of convergence is $1/2$. 

II. The function $f(x) = 1/(6-x)$ satisfies
$$
f(x) = \frac{1}{6-x} = \frac{1}{6(1-x/6)} = \frac{1}{6} \cdot \frac{1}{1-x/6} = \frac{1}{6} \cdot g(x/6). 
$$
The radius of convergence is determined by the inequality $|x/6| < 1$, and is therefore equal to $6$.

III. Note that the function $f(x) = 1/(6-x)$ satisfies 
$$
f(x) = \frac{1}{6-x} = \frac{1}{1-(x-5)} = g(x-5).
$$
The Taylor series centered at $x=5$ for $g(x-5)$ is the same as the Taylor series for $g(x)$ based at $x=0$. Therefore the radius of convergence of $f(x)$ is $1$.
\end{solution}
\end{problem}



\begin{problem} 
Consider the function $f(x) = \dfrac{4x^2}{1+3x^2}.$

\begin{itemize}
\item[I.] Find the Taylor series centered at $x=0$ for $f(x)$.
\item[II.] What is $f^{(23)}(0)$?
\end{itemize}

\begin{solution}
Recall that the Taylor series centered at $x=0$ for $g(x) =1/(1-x)$ is $\sum_{k=0}^\infty x^k$. 

I. We rewrite $f(x)$ as
$$
f(x) = \frac{4x^2}{1+3x^2} = 4x^2 \cdot \frac{1}{1-(-3x^2)}.
$$
The Taylor series centered at $x=0$ for $f(x)$ is therefore
$$
\sum_{k=0}^\infty 4x^2 \cdot (-3x^2)^k = \sum_{k=0}^\infty 4  (-3)^k x^{2k+2}.
$$

II. Recall that the $k$th term in the Taylor series centered at $x=0$ of $f(x)$ is 
$$
\frac{f^{(k)}(0)}{k!} x^k.
$$
We can therefore determine $f^{(23)}(0)$ by examining the coefficient of the degree-$23$ term of the Taylor series we found in part I. From our formula, we see that the powers of $x$ occuring in the Taylor series for $f(x)$ are always even (since the expression is written in terms of $x^{2k+2}$). This implies that the coefficient of $x^{23}$ in the series is zero, so that $f^{(23)}(0)=0$.
\end{solution}
\end{problem}


\begin{problem} 
Consider the function $f(x) = 4x^2e^x+3\sin(2x)$.

\begin{itemize}
\item[I.] Find the sum of the first three nonzero terms in the Taylor series centered at $x=0$ for $f(x)$.
	%Ans: 2x^4+4x^2+6x (verified with Maple). Comment how using the definition would be a bad idea
\item[II.] Find $f^{(4)}(0)$.
	%Comment how using the coefficients in the 4th degree Taylor poly is a good idea rather than using the summation notation
\end{itemize}

\begin{solution}
We could, of course, compute the terms in the Taylor series by hand, using the general formula for its coefficients. It would instead be faster to use properties of Taylor series and the known series expansions of $e^x$ and $\sin(x)$. We recall that 
$$
e^x = \sum_{k=0}^\infty \frac{1}{k!}x^k = 1 + x + \frac{1}{2} x^2 + \frac{1}{3!} x^3 + \frac{1}{4!}x^4 + \cdots
$$
and
$$
\sin(x) = \sum_{k=0}^\infty \frac{(-1)^k}{(2k+1)!} x^{2k+1} = x - \frac{1}{3!}x^3 + \frac{1}{5!}x^5 + \cdots.
$$

I. Using the above, the Taylor series centered at $x=0$ for $f(x)$ is
\begin{align*}
&4x^2 \cdot \left(1 + x + \frac{1}{2} x^2 + \frac{1}{3!} x^3 + \frac{1}{4!}x^4 + \cdots\right) + 3 \cdot \left((2x) - \frac{1}{3!}(2x)^3 + \frac{1}{5!}(2x)^5 + \cdots\right) \\
&\hspace{.2in} = \left(4x^2 + 4x^3 + 2x^4 + \frac{4}{3!} x^5 + \frac{4}{4!}x^6 + \cdots\right) + \left(6x - \frac{3 \cdot 8}{3!} x^3 + \frac{3 \cdot 32}{5!} x^5 + \cdots \right) \\
&\hspace{.2in} = \left(4x^2 + 4x^3 + 2x^4 + \frac{2}{3} x^5 + \frac{1}{6}x^6 + \cdots\right) + \left(6x - 4 x^3 + \frac{5}{4} x^5 + \cdots \right).
\end{align*}
Combining terms of the same degree, we see that the sum of the first three nonzero terms is
$$
6x + 4x^2 + 2x^4.
$$

II. The degree-$4$ term of the Taylor series is given by the general formula 
$$
\frac{f^{(4)}(0)}{4!} x^4.
$$
On the other hand, we see from part I. that the degree-$4$ term is $2x^4$. Equating the coefficients of these expressions, we have
$$
2 = \frac{f^{(4)}(0)}{4!},
$$
and conclude that $f^{(4)}(0) = 2 \cdot 4! = 48$.
\end{solution}
\end{problem}



\end{document}
