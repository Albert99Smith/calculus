\documentclass[]{ximera}
%handout:  for handout version with no solutions or instructor notes
%handout,instructornotes:  for instructor version with just problems and notes, no solutions
%noinstructornotes:  shows only problem and solutions

%% handout
%% space
%% newpage
%% numbers
%% nooutcomes

%I added the commands here so that I would't have to keep looking them up
%\newcommand{\RR}{\mathbb R}
%\renewcommand{\d}{\,d}
%\newcommand{\dd}[2][]{\frac{d #1}{d #2}}
%\renewcommand{\l}{\ell}
%\newcommand{\ddx}{\frac{d}{dx}}
%\everymath{\displaystyle}
%\newcommand{\dfn}{\textbf}
%\newcommand{\eval}[1]{\bigg[ #1 \bigg]}

%\begin{image}
%\includegraphics[trim= 170 420 250 180]{Figure1.pdf}
%\end{image}

%add a ``.'' below when used in a specific directory.
%\usepackage{todonotes}
%\usepackage{mathtools} %% Required for wide table Curl and Greens
%\usepackage{cuted} %% Required for wide table Curl and Greens
\newcommand{\todo}{}

\usepackage{esint} % for \oiint
\ifxake%%https://math.meta.stackexchange.com/questions/9973/how-do-you-render-a-closed-surface-double-integral
\renewcommand{\oiint}{{\large\bigcirc}\kern-1.56em\iint}
\fi


\graphicspath{
  {./}
  {ximeraTutorial/}
  {basicPhilosophy/}
  {functionsOfSeveralVariables/}
  {normalVectors/}
  {lagrangeMultipliers/}
  {vectorFields/}
  {greensTheorem/}
  {shapeOfThingsToCome/}
  {dotProducts/}
  {partialDerivativesAndTheGradientVector/}
  {../productAndQuotientRules/exercises/}
  {../normalVectors/exercisesParametricPlots/}
  {../continuityOfFunctionsOfSeveralVariables/exercises/}
  {../partialDerivativesAndTheGradientVector/exercises/}
  {../directionalDerivativeAndChainRule/exercises/}
  {../commonCoordinates/exercisesCylindricalCoordinates/}
  {../commonCoordinates/exercisesSphericalCoordinates/}
  {../greensTheorem/exercisesCurlAndLineIntegrals/}
  {../greensTheorem/exercisesDivergenceAndLineIntegrals/}
  {../shapeOfThingsToCome/exercisesDivergenceTheorem/}
  {../greensTheorem/}
  {../shapeOfThingsToCome/}
  {../separableDifferentialEquations/exercises/}
  {vectorFields/}
}

\newcommand{\mooculus}{\textsf{\textbf{MOOC}\textnormal{\textsf{ULUS}}}}

\usepackage{tkz-euclide}\usepackage{tikz}
\usepackage{tikz-cd}
\usetikzlibrary{arrows}
\tikzset{>=stealth,commutative diagrams/.cd,
  arrow style=tikz,diagrams={>=stealth}} %% cool arrow head
\tikzset{shorten <>/.style={ shorten >=#1, shorten <=#1 } } %% allows shorter vectors

\usetikzlibrary{backgrounds} %% for boxes around graphs
\usetikzlibrary{shapes,positioning}  %% Clouds and stars
\usetikzlibrary{matrix} %% for matrix
\usepgfplotslibrary{polar} %% for polar plots
\usepgfplotslibrary{fillbetween} %% to shade area between curves in TikZ
\usetkzobj{all}
\usepackage[makeroom]{cancel} %% for strike outs
%\usepackage{mathtools} %% for pretty underbrace % Breaks Ximera
%\usepackage{multicol}
\usepackage{pgffor} %% required for integral for loops



%% http://tex.stackexchange.com/questions/66490/drawing-a-tikz-arc-specifying-the-center
%% Draws beach ball
\tikzset{pics/carc/.style args={#1:#2:#3}{code={\draw[pic actions] (#1:#3) arc(#1:#2:#3);}}}



\usepackage{array}
\setlength{\extrarowheight}{+.1cm}
\newdimen\digitwidth
\settowidth\digitwidth{9}
\def\divrule#1#2{
\noalign{\moveright#1\digitwidth
\vbox{\hrule width#2\digitwidth}}}





\newcommand{\RR}{\mathbb R}
\newcommand{\R}{\mathbb R}
\newcommand{\N}{\mathbb N}
\newcommand{\Z}{\mathbb Z}

\newcommand{\sagemath}{\textsf{SageMath}}


%\renewcommand{\d}{\,d\!}
\renewcommand{\d}{\mathop{}\!d}
\newcommand{\dd}[2][]{\frac{\d #1}{\d #2}}
\newcommand{\pp}[2][]{\frac{\partial #1}{\partial #2}}
\renewcommand{\l}{\ell}
\newcommand{\ddx}{\frac{d}{\d x}}

\newcommand{\zeroOverZero}{\ensuremath{\boldsymbol{\tfrac{0}{0}}}}
\newcommand{\inftyOverInfty}{\ensuremath{\boldsymbol{\tfrac{\infty}{\infty}}}}
\newcommand{\zeroOverInfty}{\ensuremath{\boldsymbol{\tfrac{0}{\infty}}}}
\newcommand{\zeroTimesInfty}{\ensuremath{\small\boldsymbol{0\cdot \infty}}}
\newcommand{\inftyMinusInfty}{\ensuremath{\small\boldsymbol{\infty - \infty}}}
\newcommand{\oneToInfty}{\ensuremath{\boldsymbol{1^\infty}}}
\newcommand{\zeroToZero}{\ensuremath{\boldsymbol{0^0}}}
\newcommand{\inftyToZero}{\ensuremath{\boldsymbol{\infty^0}}}



\newcommand{\numOverZero}{\ensuremath{\boldsymbol{\tfrac{\#}{0}}}}
\newcommand{\dfn}{\textbf}
%\newcommand{\unit}{\,\mathrm}
\newcommand{\unit}{\mathop{}\!\mathrm}
\newcommand{\eval}[1]{\bigg[ #1 \bigg]}
\newcommand{\seq}[1]{\left( #1 \right)}
\renewcommand{\epsilon}{\varepsilon}
\renewcommand{\phi}{\varphi}


\renewcommand{\iff}{\Leftrightarrow}

\DeclareMathOperator{\arccot}{arccot}
\DeclareMathOperator{\arcsec}{arcsec}
\DeclareMathOperator{\arccsc}{arccsc}
\DeclareMathOperator{\si}{Si}
\DeclareMathOperator{\scal}{scal}
\DeclareMathOperator{\sign}{sign}


%% \newcommand{\tightoverset}[2]{% for arrow vec
%%   \mathop{#2}\limits^{\vbox to -.5ex{\kern-0.75ex\hbox{$#1$}\vss}}}
\newcommand{\arrowvec}[1]{{\overset{\rightharpoonup}{#1}}}
%\renewcommand{\vec}[1]{\arrowvec{\mathbf{#1}}}
\renewcommand{\vec}[1]{{\overset{\boldsymbol{\rightharpoonup}}{\mathbf{#1}}}\hspace{0in}}

\newcommand{\point}[1]{\left(#1\right)} %this allows \vector{ to be changed to \vector{ with a quick find and replace
\newcommand{\pt}[1]{\mathbf{#1}} %this allows \vec{ to be changed to \vec{ with a quick find and replace
\newcommand{\Lim}[2]{\lim_{\point{#1} \to \point{#2}}} %Bart, I changed this to point since I want to use it.  It runs through both of the exercise and exerciseE files in limits section, which is why it was in each document to start with.

\DeclareMathOperator{\proj}{\mathbf{proj}}
\newcommand{\veci}{{\boldsymbol{\hat{\imath}}}}
\newcommand{\vecj}{{\boldsymbol{\hat{\jmath}}}}
\newcommand{\veck}{{\boldsymbol{\hat{k}}}}
\newcommand{\vecl}{\vec{\boldsymbol{\l}}}
\newcommand{\uvec}[1]{\mathbf{\hat{#1}}}
\newcommand{\utan}{\mathbf{\hat{t}}}
\newcommand{\unormal}{\mathbf{\hat{n}}}
\newcommand{\ubinormal}{\mathbf{\hat{b}}}

\newcommand{\dotp}{\bullet}
\newcommand{\cross}{\boldsymbol\times}
\newcommand{\grad}{\boldsymbol\nabla}
\newcommand{\divergence}{\grad\dotp}
\newcommand{\curl}{\grad\cross}
%\DeclareMathOperator{\divergence}{divergence}
%\DeclareMathOperator{\curl}[1]{\grad\cross #1}
\newcommand{\lto}{\mathop{\longrightarrow\,}\limits}

\renewcommand{\bar}{\overline}

\colorlet{textColor}{black}
\colorlet{background}{white}
\colorlet{penColor}{blue!50!black} % Color of a curve in a plot
\colorlet{penColor2}{red!50!black}% Color of a curve in a plot
\colorlet{penColor3}{red!50!blue} % Color of a curve in a plot
\colorlet{penColor4}{green!50!black} % Color of a curve in a plot
\colorlet{penColor5}{orange!80!black} % Color of a curve in a plot
\colorlet{penColor6}{yellow!70!black} % Color of a curve in a plot
\colorlet{fill1}{penColor!20} % Color of fill in a plot
\colorlet{fill2}{penColor2!20} % Color of fill in a plot
\colorlet{fillp}{fill1} % Color of positive area
\colorlet{filln}{penColor2!20} % Color of negative area
\colorlet{fill3}{penColor3!20} % Fill
\colorlet{fill4}{penColor4!20} % Fill
\colorlet{fill5}{penColor5!20} % Fill
\colorlet{gridColor}{gray!50} % Color of grid in a plot

\newcommand{\surfaceColor}{violet}
\newcommand{\surfaceColorTwo}{redyellow}
\newcommand{\sliceColor}{greenyellow}




\pgfmathdeclarefunction{gauss}{2}{% gives gaussian
  \pgfmathparse{1/(#2*sqrt(2*pi))*exp(-((x-#1)^2)/(2*#2^2))}%
}


%%%%%%%%%%%%%
%% Vectors
%%%%%%%%%%%%%

%% Simple horiz vectors
\renewcommand{\vector}[1]{\left\langle #1\right\rangle}


%% %% Complex Horiz Vectors with angle brackets
%% \makeatletter
%% \renewcommand{\vector}[2][ , ]{\left\langle%
%%   \def\nextitem{\def\nextitem{#1}}%
%%   \@for \el:=#2\do{\nextitem\el}\right\rangle%
%% }
%% \makeatother

%% %% Vertical Vectors
%% \def\vector#1{\begin{bmatrix}\vecListA#1,,\end{bmatrix}}
%% \def\vecListA#1,{\if,#1,\else #1\cr \expandafter \vecListA \fi}

%%%%%%%%%%%%%
%% End of vectors
%%%%%%%%%%%%%

%\newcommand{\fullwidth}{}
%\newcommand{\normalwidth}{}



%% makes a snazzy t-chart for evaluating functions
%\newenvironment{tchart}{\rowcolors{2}{}{background!90!textColor}\array}{\endarray}

%%This is to help with formatting on future title pages.
\newenvironment{sectionOutcomes}{}{}



%% Flowchart stuff
%\tikzstyle{startstop} = [rectangle, rounded corners, minimum width=3cm, minimum height=1cm,text centered, draw=black]
%\tikzstyle{question} = [rectangle, minimum width=3cm, minimum height=1cm, text centered, draw=black]
%\tikzstyle{decision} = [trapezium, trapezium left angle=70, trapezium right angle=110, minimum width=3cm, minimum height=1cm, text centered, draw=black]
%\tikzstyle{question} = [rectangle, rounded corners, minimum width=3cm, minimum height=1cm,text centered, draw=black]
%\tikzstyle{process} = [rectangle, minimum width=3cm, minimum height=1cm, text centered, draw=black]
%\tikzstyle{decision} = [trapezium, trapezium left angle=70, trapezium right angle=110, minimum width=3cm, minimum height=1cm, text centered, draw=black]
 %% we can turn off input when making a master document


\title{Integration By Parts}  

\begin{document}
\begin{abstract}		\end{abstract}
\author{Tom Needham and Jim Talamo}

\outcome{Apply the Integration by Parts technique to evaluate definite and indefinite integrals.}
\maketitle




\section{Discussion Questions:}


\begin{problem}
A student evaluates the integral $\int_0^1 x e^x \d x$ via Integration by Parts with $u=x$, $\d v = e^x \d x$ as follows:
\[
\int_0^1 x e^x \d x = x e^x - \int_0^1 e^x \d x = x e^x + \eval{e^x}_0^1 = x e^x + e^1 - e^0 = xe^x + e -1.
\]
Is the student's work correct?
\end{problem}

\begin{freeResponse}
The student's work is certainly not correct! Since they are evaluating a definite integral, their answer should be a number. However, their answer is a function of the integration variable $x$. In particular, the student forgot to evaluate the first term during the course of the integration by parts process. The correct solution is 
$$
\int_0^1 x e^x \d x = \eval{x e^x}_0^1 - \int_0^1 e^x \d x = e^1 - 0e^0 + \eval{e^x}_0^1 = e^1 + e^1 - e^0 = 2e -1.
$$
\end{freeResponse}

\begin{problem}
For each example, decide on an integration strategy (but don't actually evaluate the integrals).

\begin{center}
\begin{tabular}{lll}
I. $\int x \sin (x) \d x$ \hspace{0.2in} & II. $\int x \sin (x^2) \d x$  \hspace{0.2in} & III. $\int x^2 \sin (x) \d x$.
\end{tabular}
\end{center}
\end{problem}

\begin{freeResponse}
I. This integral is easily solved via Integration by Parts, with $u=x$ and $\d v = \sin(x) \d x$.

II. This integral can be solved via $u$-substitution with $u = x^2$, so that $\d u = 2 x \d x$.

III. Integration by Parts can be used for this example, with $u=x^2$ and $\d v = \sin(x) \d x$. After applying the method once, the integral reduces to 
$$
\int x^2 \sin (x) \d x = -x^2\cos(x) + 2\int x \cos(x) \d x.
$$
The second term can be evaluated by applying Integration by Parts once more with $u = x$ and $\d v = \cos (x) \d x$. 
\end{freeResponse}
\section{Group work:}


%problem 1
\begin{problem}
Evaluate the following integrals
\begin{center}
\begin{tabular}{lll}
I. $\int_1^3 x^2 e^{2x} \d x$ \hspace{0.2in} & II. $\int \arcsin(x) \d x$  \hspace{0.2in} & III. $\int x\sin\left(\frac{x^2}{2}\right) + x\sin\left(\frac{x}{2}\right) \d x$ %III. $\int x^{\frac{5}{3}} (\ln x)^2 \d x$.  
\end{tabular}
\end{center}
\end{problem}
	
\begin{freeResponse}
I. We use Integration by Parts with $u = x^2$ and $\d v = e^{2x} \d x$, so that $\d u = 2x \d x$ and $v = \frac{1}{2} e^{2x}$. Then
$$
\int_1^3 x^2 e^{2x} \d x = \eval{\frac{x^2}{2} e^{2x}}_1^3 - \int_1^3 x e^{2x} \d x.
$$
The second integral is evaluated by another application of Integration by Parts, with $u =x$, $\d v = e^{2x} \d x$, $\d u = \d x$ and $v = \frac{1}{2}e^{2x}$. Continuing our integration, we have 
\begin{align*}
\eval{\frac{x^2}{2} e^{2x}}_1^3 - \int_1^3 x e^{2x} \d x & = \frac{9}{2} e^6 - \frac{1}{2} e^2 - \eval{\frac{x}{2}e^{2x}}_1^3 + \int_1^3 \frac{1}{2} e^{2x} \d x \\
& = \frac{9}{2} e^6 - \frac{1}{2} e^2 - \frac{3}{2} e^6 + \frac{1}{2} e^2 + \eval{\frac{1}{4}e^{2x}}_1^3 \\
& = 3e^6 + \frac{1}{4} e^6 - \frac{1}{4} e^2 \\
& = \frac{13}{4} e^6 - \frac{1}{4} e^2.
\end{align*}

II. We do not know how to integrate $\arcsin(x)$, but we do know how to differentiate it, so we will use integration by parts. Let $u = \arcsin(x)$ and $\d v =  \d x$. Then $\d u = \frac{1}{\sqrt{1-x^2}} \d x$ and $v = x$. This gives us
$$
\int \arcsin(x) \d x = x\arcsin(x) - \int \frac{x \d x}{\sqrt{1-x^2}}.
$$
The second term can be evaluated by $u$-substitution, with $u = 1- x^2$ and $\d u = -2x \d x$.  This yields
	\begin{align*}
	\int \frac{x \d x}{\sqrt{1-x^2}} &= \int \frac{-\d u}{2\sqrt{u}}\\
	&= -\frac{1}{2} \int u^{-\frac{1}{2}} \d u \\
	&= -\frac{1}{2} \cdot 2u^{\frac{1}{2}} +C \\
	&= - \sqrt{1-x^2} +C.
	\end{align*}
Therefore
$$
	\int \arcsin(x) \d x  = x\arcsin(x) + \sqrt{1-x^2} +C.
	$$

III. We can split the integral up as follows.

\[
\int x\sin\left(\frac{x^2}{2}\right) + x\sin\left(\frac{x}{2}\right) \d x = \int x\sin\left(\frac{x^2}{2}\right)  \d x+\int  x\sin\left(\frac{x}{2}\right) \d x
\]

Notice that the first integral can be handled by the substitution $u = \frac{x^2}{2}$.  In this case $\d u = x \d x$, so

\[
 \int x\sin\left(\frac{x^2}{2}\right)  \d x = \int \sin(u) \d u = -\cos(u) +C=-\cos\left(\frac{x^2}{2}\right)+C
\]

For the second integral, we will use integration by parts. 

Let $u =x $ and $\d v =   \sin\left(\frac{x}{2}\right) \d x$. Then $\d u =  \d x$ and $v =  -2 \cos\left(\frac{x}{2}\right)$. This gives us

\begin{align*}
\int  x\sin\left(\frac{x}{2}\right) \d x &= -2x \cos\left(\frac{x}{2}\right) - \int -2 \cos\left(\frac{x}{2}\right) \d x \\
&= -2x \cos\left(\frac{x}{2}\right) + 4 \sin\left(\frac{x}{2}\right) +C
\end{align*}

Putting these both together gives the final result.	

\begin{align*}
\int x\sin\left(\frac{x^2}{2}\right) + x\sin\left(\frac{x}{2}\right) \d x &= \int x\sin\left(\frac{x^2}{2}\right)  \d x+\int  x\sin\left(\frac{x}{2}\right) \d x\\
&= -\cos\left(\frac{x^2}{2}\right)-2x \cos\left(\frac{x}{2}\right) + 4 \sin\left(\frac{x}{2}\right) +C
\end{align*}	
%III. We begin with the substitution $w = \ln x$, so that $\d w = \frac{1}{x} \d x$. Solving the substitution equation for $x$ yields $x = e^w$, and we have
%		\begin{align*}
%		\int x^{\frac{5}{3}} (\ln x)^2 \d x 
%		&= \int x^{\frac{8}{3}} (\ln x)^2 \cdot \frac{1}{x} \d x  \\
%		&= \int (e^w)^{\frac{8}{3}} w^2 \d w  \\
%		&= \int w^2 e^{\frac{8}{3} w} \d w.
%		\end{align*}
%	We now use integration by parts, with $u = w^2$ and $\d v = e^{\frac{8}{3} w} \d w$, whence $\d u = 2 w \d w$ and $v = \frac{3}{8} e^{\frac{8}{3} w}$. This gives us
%		\begin{align*}
%		\int x^{\frac{5}{3}} (\ln x)^2 \d x
%		&= \frac{3}{8} w^2 e^{\frac{8}{3} w} - \int \frac{3}{8} (2w) e^{\frac{8}{3} w} \d w  \\
%		&= \frac{3}{8} w^2 e^{\frac{8}{3} w} - \frac{3}{4} \int w e^{\frac{8}{3} w} \d w.
%		\end{align*}
%	To evaluate the second term, we apply integration by parts one last time with $u = w$, $\d u = \d w$, 	$\d v = e^{\frac{8}{3} w} \d w$ and $v = \frac{3}{8} e^{\frac{8}{3} w}$. This yields
%		\begin{align*}
%		\int x^{\frac{5}{3}} (\ln x)^2 \d x
%		&= \frac{3}{8}w^2 e^{\frac{8}{3} w} - \frac{3}{4} \left( \frac{3}{8} w e^{\frac{8}{3} w} - \frac{3}{8} \int e^{\frac{8}{3} w} \d w \right) \\
%		&= \frac{3}{8} w^2 e^{\frac{8}{3} w} - \frac{9}{32} w e^{\frac{8}{3} w} + \frac{27}{256} e^{\frac{8}{3} w} + C  \\
%		&= \frac{3}{8} e^{\frac{8}{3} \ln x} \left( (\ln x)^2 - \frac{3}{4} \ln x + \frac{9}{32} \right) + C  \\
%		&= \frac{3}{8} x^{\frac{8}{3}} \left( (\ln x)^2 - \frac{3}{4} \ln x + \frac{9}{32} \right) + C.
%		\end{align*}
\end{freeResponse}





\pagebreak

%problem 2
\begin{problem}

Evaluate the following integrals

\begin{center}
\begin{tabular}{ll}
I. $\int \sin(3x) e^{7x} \d x$ \hspace{.5in} II. $\int x^5 \cos \left( x^3 \right) \d x$ 
\end{tabular}
\end{center}
	
\end{problem}

\begin{freeResponse}
I. We begin by letting $I = \int \sin(3x) e^{7x} \d x$.  
	We then use integration by parts with
		\[
		u = e^{7x} \qquad \d v = \sin(3x) \d x
		\]
		\[
		\d u = 7e^{7x} \d x \qquad v = -\frac{1}{3} \cos(3x).
		\]
	Then
		\begin{align*}
		I = \int \sin(3x) e^{7x} \d x &= -\frac{1}{3} e^{7x} \cos(3x) - \int -\frac{1}{3} (7e^{7x}) \cos(3x) \d x  \\
		 &= -\frac{1}{3} e^{7x} \cos(3x) + \frac{7}{3} \int e^{7x} \cos(3x) \d x.
		\end{align*}
	We then apply integration by parts again, this time with
		\[
		u = e^{7x} 		\qquad	\d v = \cos(3x) \d x
		\]
		\[
		\d u = 7e^{7x} \d x 	\qquad	v = \frac{1}{3} \sin(3x).
		\]
	This gives us
		\begin{align*}
		I &= -\frac{1}{3} e^{7x} \cos(3x) + \frac{7}{3} \left[ \frac{1}{3} e^{7x} \sin(3x) - \int \frac{1}{3} (7e^{7x}) \sin(3x) \d x \right]  \\
				&= -\frac{1}{3} e^{7x} \cos(3x) + \frac{7}{9} e^{7x} \sin(3x) - \frac{49}{9} \int e^{7x} \sin(3x) \d x  \\
		&= -\frac{1}{3} e^{7x} \cos(3x) + \frac{7}{9} e^{7x} \sin(3x) - \frac{49}{9} I.
		\end{align*}
		Since we are interested in the integral $I$, we solve the above equation for $I$ to obtain
		$$
		\frac{58}{9} I = -\frac{1}{3} e^{7x} \cos(3x) + \frac{7}{9} e^{7x} \sin(3x),
		$$
		so that 
		$$
		I = \frac{9}{58} \left( -\frac{1}{3} e^{7x} \cos(3x) + \frac{7}{9} e^{7x} \sin(3x) \right) + C.
		$$
		
		II. We begin with the substitution $w = x^3$, so that $\d w = 3x^2 \d x$. It follows that 
		\begin{align*}
		\int x^5 \cos \left( x^3 \right) \d x 
		&= \int x^3 \cos \left( x^3 \right) \cdot x^2 \d x  \\
		&= \int w \cos(w) \cdot \frac{1}{3} \d w  \\
		&= \frac{1}{3} \int w \cos(w) \d w.
		\end{align*}
	We then use integration by parts, with
		\[
		u = w 		\qquad	\d v = \cos(w) \d w
		\]
		\[
		\d u = \d w 	\qquad	v = \sin(w),
		\]
	which yields
		\begin{align*}
		\int x^5 \cos \left( x^3 \right) \d x 
		&= \frac{1}{3} \left( w \sin(w) - \int \sin(w) \d w \right)  \\
		&= \frac{1}{3} \left( w \sin(w) + \cos(w) \right) + C  \\
		&= \frac{1}{3} \left( x^3 \sin \left( x^3 \right) + \cos \left( x^3 \right) \right) + C.  
		\end{align*}
		
		
\end{freeResponse}

\begin{problem}
The region $R$ is bounded by the curves $y=0$, $y=\ln (x)$ and $x=e$. Using the shell method, find the volume of the solid formed by rotating $R$ around the $y$-axis.
\end{problem}

\begin{freeResponse}
Since we are rotating around a vertical axis and we wish to use the Shell method, our slices should be vertical. The height of the shell at $x$ is given by $\ln (x)$, and our bounds of integration are $x=1$ (corresponding to $y=0$) and $x=e$. The volume integral is 
\[
V=\int_1^e 2 \pi x \ln (x) \d x = 2 \pi \int_1^e x \ln (x) \d x.
\]
To evaluate the integral, we use integration by parts with $u = \ln (x)$ and $\d v = x \d x$, so that $\d u = \frac{1}{x} \d x$ and $v = \frac{x^2}{2}$. This gives us 
\begin{align*}
\int_1^e 2 \pi x \ln (x) \d x &= 2 \pi \eval{\frac{x^2}{2} \ln x}_1^e - 2 \pi \int_1^e \frac{x}{2} \d x \\
&= 2 \pi \cdot \frac{e^2}{2}\ln(e) - 2 \pi \cdot \frac{1}{2}\ln (1) - 2 \pi \eval{\frac{x^2}{4}}_1^e \\
&= 2 \pi \cdot \frac{e^2}{2} - 2 \pi \cdot \frac{e^2}{4} + 2 \pi \cdot \frac{1}{4} \\
&= \frac{\pi e^2}{2} + \frac{\pi}{2}.
\end{align*}

\end{freeResponse}
\end{document} 


















