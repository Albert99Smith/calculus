\documentclass[noauthor,handout]{ximera}
%handout:  for handout version with no solutions or instructor notes
%handout,instructornotes:  for instructor version with just problems and notes, no solutions
%noinstructornotes:  shows only problem and solutions

%% handout
%% space
%% newpage
%% numbers
%% nooutcomes

%I added the commands here so that I would't have to keep looking them up
%\newcommand{\RR}{\mathbb R}
%\renewcommand{\d}{\,d}
%\newcommand{\dd}[2][]{\frac{d #1}{d #2}}
%\renewcommand{\l}{\ell}
%\newcommand{\ddx}{\frac{d}{dx}}
%\everymath{\displaystyle}
%\newcommand{\dfn}{\textbf}
%\newcommand{\eval}[1]{\bigg[ #1 \bigg]}

%\begin{image}
%\includegraphics[trim= 170 420 250 180]{Figure1.pdf}
%\end{image}

%add a ``.'' below when used in a specific directory.

%\usepackage{todonotes}
%\usepackage{mathtools} %% Required for wide table Curl and Greens
%\usepackage{cuted} %% Required for wide table Curl and Greens
\newcommand{\todo}{}

\usepackage{esint} % for \oiint
\ifxake%%https://math.meta.stackexchange.com/questions/9973/how-do-you-render-a-closed-surface-double-integral
\renewcommand{\oiint}{{\large\bigcirc}\kern-1.56em\iint}
\fi


\graphicspath{
  {./}
  {ximeraTutorial/}
  {basicPhilosophy/}
  {functionsOfSeveralVariables/}
  {normalVectors/}
  {lagrangeMultipliers/}
  {vectorFields/}
  {greensTheorem/}
  {shapeOfThingsToCome/}
  {dotProducts/}
  {partialDerivativesAndTheGradientVector/}
  {../productAndQuotientRules/exercises/}
  {../normalVectors/exercisesParametricPlots/}
  {../continuityOfFunctionsOfSeveralVariables/exercises/}
  {../partialDerivativesAndTheGradientVector/exercises/}
  {../directionalDerivativeAndChainRule/exercises/}
  {../commonCoordinates/exercisesCylindricalCoordinates/}
  {../commonCoordinates/exercisesSphericalCoordinates/}
  {../greensTheorem/exercisesCurlAndLineIntegrals/}
  {../greensTheorem/exercisesDivergenceAndLineIntegrals/}
  {../shapeOfThingsToCome/exercisesDivergenceTheorem/}
  {../greensTheorem/}
  {../shapeOfThingsToCome/}
  {../separableDifferentialEquations/exercises/}
  {vectorFields/}
}

\newcommand{\mooculus}{\textsf{\textbf{MOOC}\textnormal{\textsf{ULUS}}}}

\usepackage{tkz-euclide}\usepackage{tikz}
\usepackage{tikz-cd}
\usetikzlibrary{arrows}
\tikzset{>=stealth,commutative diagrams/.cd,
  arrow style=tikz,diagrams={>=stealth}} %% cool arrow head
\tikzset{shorten <>/.style={ shorten >=#1, shorten <=#1 } } %% allows shorter vectors

\usetikzlibrary{backgrounds} %% for boxes around graphs
\usetikzlibrary{shapes,positioning}  %% Clouds and stars
\usetikzlibrary{matrix} %% for matrix
\usepgfplotslibrary{polar} %% for polar plots
\usepgfplotslibrary{fillbetween} %% to shade area between curves in TikZ
\usetkzobj{all}
\usepackage[makeroom]{cancel} %% for strike outs
%\usepackage{mathtools} %% for pretty underbrace % Breaks Ximera
%\usepackage{multicol}
\usepackage{pgffor} %% required for integral for loops



%% http://tex.stackexchange.com/questions/66490/drawing-a-tikz-arc-specifying-the-center
%% Draws beach ball
\tikzset{pics/carc/.style args={#1:#2:#3}{code={\draw[pic actions] (#1:#3) arc(#1:#2:#3);}}}



\usepackage{array}
\setlength{\extrarowheight}{+.1cm}
\newdimen\digitwidth
\settowidth\digitwidth{9}
\def\divrule#1#2{
\noalign{\moveright#1\digitwidth
\vbox{\hrule width#2\digitwidth}}}





\newcommand{\RR}{\mathbb R}
\newcommand{\R}{\mathbb R}
\newcommand{\N}{\mathbb N}
\newcommand{\Z}{\mathbb Z}

\newcommand{\sagemath}{\textsf{SageMath}}


%\renewcommand{\d}{\,d\!}
\renewcommand{\d}{\mathop{}\!d}
\newcommand{\dd}[2][]{\frac{\d #1}{\d #2}}
\newcommand{\pp}[2][]{\frac{\partial #1}{\partial #2}}
\renewcommand{\l}{\ell}
\newcommand{\ddx}{\frac{d}{\d x}}

\newcommand{\zeroOverZero}{\ensuremath{\boldsymbol{\tfrac{0}{0}}}}
\newcommand{\inftyOverInfty}{\ensuremath{\boldsymbol{\tfrac{\infty}{\infty}}}}
\newcommand{\zeroOverInfty}{\ensuremath{\boldsymbol{\tfrac{0}{\infty}}}}
\newcommand{\zeroTimesInfty}{\ensuremath{\small\boldsymbol{0\cdot \infty}}}
\newcommand{\inftyMinusInfty}{\ensuremath{\small\boldsymbol{\infty - \infty}}}
\newcommand{\oneToInfty}{\ensuremath{\boldsymbol{1^\infty}}}
\newcommand{\zeroToZero}{\ensuremath{\boldsymbol{0^0}}}
\newcommand{\inftyToZero}{\ensuremath{\boldsymbol{\infty^0}}}



\newcommand{\numOverZero}{\ensuremath{\boldsymbol{\tfrac{\#}{0}}}}
\newcommand{\dfn}{\textbf}
%\newcommand{\unit}{\,\mathrm}
\newcommand{\unit}{\mathop{}\!\mathrm}
\newcommand{\eval}[1]{\bigg[ #1 \bigg]}
\newcommand{\seq}[1]{\left( #1 \right)}
\renewcommand{\epsilon}{\varepsilon}
\renewcommand{\phi}{\varphi}


\renewcommand{\iff}{\Leftrightarrow}

\DeclareMathOperator{\arccot}{arccot}
\DeclareMathOperator{\arcsec}{arcsec}
\DeclareMathOperator{\arccsc}{arccsc}
\DeclareMathOperator{\si}{Si}
\DeclareMathOperator{\scal}{scal}
\DeclareMathOperator{\sign}{sign}


%% \newcommand{\tightoverset}[2]{% for arrow vec
%%   \mathop{#2}\limits^{\vbox to -.5ex{\kern-0.75ex\hbox{$#1$}\vss}}}
\newcommand{\arrowvec}[1]{{\overset{\rightharpoonup}{#1}}}
%\renewcommand{\vec}[1]{\arrowvec{\mathbf{#1}}}
\renewcommand{\vec}[1]{{\overset{\boldsymbol{\rightharpoonup}}{\mathbf{#1}}}\hspace{0in}}

\newcommand{\point}[1]{\left(#1\right)} %this allows \vector{ to be changed to \vector{ with a quick find and replace
\newcommand{\pt}[1]{\mathbf{#1}} %this allows \vec{ to be changed to \vec{ with a quick find and replace
\newcommand{\Lim}[2]{\lim_{\point{#1} \to \point{#2}}} %Bart, I changed this to point since I want to use it.  It runs through both of the exercise and exerciseE files in limits section, which is why it was in each document to start with.

\DeclareMathOperator{\proj}{\mathbf{proj}}
\newcommand{\veci}{{\boldsymbol{\hat{\imath}}}}
\newcommand{\vecj}{{\boldsymbol{\hat{\jmath}}}}
\newcommand{\veck}{{\boldsymbol{\hat{k}}}}
\newcommand{\vecl}{\vec{\boldsymbol{\l}}}
\newcommand{\uvec}[1]{\mathbf{\hat{#1}}}
\newcommand{\utan}{\mathbf{\hat{t}}}
\newcommand{\unormal}{\mathbf{\hat{n}}}
\newcommand{\ubinormal}{\mathbf{\hat{b}}}

\newcommand{\dotp}{\bullet}
\newcommand{\cross}{\boldsymbol\times}
\newcommand{\grad}{\boldsymbol\nabla}
\newcommand{\divergence}{\grad\dotp}
\newcommand{\curl}{\grad\cross}
%\DeclareMathOperator{\divergence}{divergence}
%\DeclareMathOperator{\curl}[1]{\grad\cross #1}
\newcommand{\lto}{\mathop{\longrightarrow\,}\limits}

\renewcommand{\bar}{\overline}

\colorlet{textColor}{black}
\colorlet{background}{white}
\colorlet{penColor}{blue!50!black} % Color of a curve in a plot
\colorlet{penColor2}{red!50!black}% Color of a curve in a plot
\colorlet{penColor3}{red!50!blue} % Color of a curve in a plot
\colorlet{penColor4}{green!50!black} % Color of a curve in a plot
\colorlet{penColor5}{orange!80!black} % Color of a curve in a plot
\colorlet{penColor6}{yellow!70!black} % Color of a curve in a plot
\colorlet{fill1}{penColor!20} % Color of fill in a plot
\colorlet{fill2}{penColor2!20} % Color of fill in a plot
\colorlet{fillp}{fill1} % Color of positive area
\colorlet{filln}{penColor2!20} % Color of negative area
\colorlet{fill3}{penColor3!20} % Fill
\colorlet{fill4}{penColor4!20} % Fill
\colorlet{fill5}{penColor5!20} % Fill
\colorlet{gridColor}{gray!50} % Color of grid in a plot

\newcommand{\surfaceColor}{violet}
\newcommand{\surfaceColorTwo}{redyellow}
\newcommand{\sliceColor}{greenyellow}




\pgfmathdeclarefunction{gauss}{2}{% gives gaussian
  \pgfmathparse{1/(#2*sqrt(2*pi))*exp(-((x-#1)^2)/(2*#2^2))}%
}


%%%%%%%%%%%%%
%% Vectors
%%%%%%%%%%%%%

%% Simple horiz vectors
\renewcommand{\vector}[1]{\left\langle #1\right\rangle}


%% %% Complex Horiz Vectors with angle brackets
%% \makeatletter
%% \renewcommand{\vector}[2][ , ]{\left\langle%
%%   \def\nextitem{\def\nextitem{#1}}%
%%   \@for \el:=#2\do{\nextitem\el}\right\rangle%
%% }
%% \makeatother

%% %% Vertical Vectors
%% \def\vector#1{\begin{bmatrix}\vecListA#1,,\end{bmatrix}}
%% \def\vecListA#1,{\if,#1,\else #1\cr \expandafter \vecListA \fi}

%%%%%%%%%%%%%
%% End of vectors
%%%%%%%%%%%%%

%\newcommand{\fullwidth}{}
%\newcommand{\normalwidth}{}



%% makes a snazzy t-chart for evaluating functions
%\newenvironment{tchart}{\rowcolors{2}{}{background!90!textColor}\array}{\endarray}

%%This is to help with formatting on future title pages.
\newenvironment{sectionOutcomes}{}{}



%% Flowchart stuff
%\tikzstyle{startstop} = [rectangle, rounded corners, minimum width=3cm, minimum height=1cm,text centered, draw=black]
%\tikzstyle{question} = [rectangle, minimum width=3cm, minimum height=1cm, text centered, draw=black]
%\tikzstyle{decision} = [trapezium, trapezium left angle=70, trapezium right angle=110, minimum width=3cm, minimum height=1cm, text centered, draw=black]
%\tikzstyle{question} = [rectangle, rounded corners, minimum width=3cm, minimum height=1cm,text centered, draw=black]
%\tikzstyle{process} = [rectangle, minimum width=3cm, minimum height=1cm, text centered, draw=black]
%\tikzstyle{decision} = [trapezium, trapezium left angle=70, trapezium right angle=110, minimum width=3cm, minimum height=1cm, text centered, draw=black]


\author{Jim Talamo}

\outcome{Find Taylor Polynomials.}
\outcome{Understand the relationship between the derivatives of a function and the coefficients of its Taylor Polynomial.}

\title{Calculus with Taylor Series}

\begin{document}
\begin{abstract}
\end{abstract}
\maketitle

\vspace{-0.9in}

\section{Discussion Questions}

\begin{problem} 

A student wants to find the Taylor series centered at $x=0$ for $f'(x)$, where $f(x) = x^3 \sin(x)$, and makes the following argument:

Since $f(x) = x^3 \sin(x) = x^3 \sum_{k=0}^{\infty} \frac{(-1)^k}{(2k+1)!}x^{2k+1}$, then:

\begin{align*}
f'(x) = \ddx \left[  x^3 \sum_{k=0}^{\infty} \frac{(-1)^k}{(2k+1)!}x^{2k+1} \right] &=  3x^2 \sum_{k=0}^{\infty} \frac{(-1)^k}{(2k)!}x^{2k} \\ 
\end{align*}

Determine if the student is correct or incorrect.  If the student is incorrect, explain the error in the student's work, then express the requested Taylor series in summation notation correctly.

\begin{freeResponse}
The student is incorrect, because they forgot to apply the product rule when taking the derivative. Their work could be corrected by using the product rule, but it would be less work to simplify the expression before taking the derivative. That is,
$$
f(x) = x^3 \sum_{k=0}^{\infty} \frac{(-1)^k}{(2k+1)!}x^{2k+1} = \sum_{k=0}^\infty \frac{(-1)^k}{(2k+1)!}x^{2k+4},
$$
whence
$$
f'(x) = \sum_{k=0}^\infty (2k+4)\frac{(-1)^k}{(2k+1)!}x^{2k+3}.
$$
\end{freeResponse}
\end{problem}


\begin{problem} 
A student claims that since the fourth degree Taylor polynomial for $\frac{x^2}{1+x^2}$ is $p_4(x) = x^2-x^4$, then:

\[\int_0^2 \frac{x^2}{1+x^2} \d x \approx \int_0^2 x^2-x^4 \d x = \eval{\frac{1}{3}x^3-\frac{1}{5}x^5}_0^2 = \frac{8}{3}-\frac{32}{5} \approx -3.73333.\]

The student then uses software to calculate that the integral to five decimal places is $.89285$.  Perplexed by this discrepancy, the student writes out $p_6(x) = x^2-x^4+x^6$  and repeats the earlier calculation.

\[\int_0^2 \frac{x^2}{1+x^2} \d x \approx \int_0^2 x^2-x^4+x^6 \d x = \eval{\frac{1}{3}x^3-\frac{1}{5}x^5+\frac{1}{7}x^7}_0^2 = \frac{8}{3}-\frac{32}{5} +\frac{128}{7} \approx 14.55238.\]

The student proclaims that they were told that Taylor polynomials are supposed to be useful for approximations and that the higher order polynomials give worse results.  How would you explain to the student what is happening?

\begin{freeResponse}
The issue is that Taylor polynomials are only guaranteed to give good \emph{local} approximations of the function. This can be made more precise as follows. Recall that the Taylor series for $g(x)=1/(1-x)$ converges when $|x|<1$. Therefore the Taylor series of the function $f(x)=x^2/(1+x^2) = x^2 g(-x^2)$ converges when $|-x^2| <1$, so the radius of convergence of the Taylor series for $f(x)$ is still $1$. Since the upper bound on the integral lies outside the interval of convergence, one would not expect the integrals of the Taylor polynomials to be good approximations of the true integral. Indeed, plotting the function $f(x)$ and its degree-$n$ Taylor polynomial $p_n(x)$ over the interval $[0,2]$ for increasing values of $n$, one sees that the $p_n(x)$ give worse approximations of $f(x)$ outside of $[0,1)$ for larger values of $n$.
\end{freeResponse}
\end{problem}



\begin{problem} 
Given that $f(x) = \sum_{k=1}^{\infty} k x^{2k}$, find $f(0)$, $f'(0)$, $f''(0)$, and $f^{(50)}(0)$.

\begin{freeResponse}
The first few derivatives can be found directly. We have 
$$
f(x) = x^2 + 2x^4 + 3x^6 + \cdots,
$$
so $f(0)= 0$, $f'(0)=0$, and 
$$
f''(x) = 2 + 4 \cdot 3 \cdot 2 \cdot  x^2 + 6 \cdot 5 \cdot 3 \cdot x^4 + \cdots
$$
implies $f''(0) = 2$. 

The approach taken above will not be effective for computing $f^{(50)}(0)$, we have to be more clever. Recall that the degree-$k$ term for the Taylor series centered at $x=0$ of $f(x)$ is given by the formula
$$
\frac{f^{(k)}(0)}{k!}x^k.
$$
Comparing with the given formula, we see that all odd-degree coefficients must be zero. Moreover, even degree coefficients satisfy
$$
\frac{f^{2k}(0)}{(2k)!} = k,
$$
or $f^{2k}(0) = k \cdot (2k)!$. From this, we easily recover the answers from above, and we are also able to determine $f^{(50)}(0)$: 
\begin{align*}
f(0) &= f^{2 \cdot 0}(0) = 0 \\
f'(0) &= 0 \\
f''(0) &= f^{2 \cdot 1}(0) = 2 \\
f^{(50)}(0) &= f^{2 \cdot 25}(0) = 25 \cdot 50!.
\end{align*}
\end{freeResponse}
\end{problem}



\section{Group Work}
%%%%%%%%%%%%%%%%%%%%%%%%%%%%%%%%
\begin{problem} 
Consider the function $f(x) = \sum_{k=0}^{\infty} \frac{(-1)^k}{k}x^{k}.$

\begin{itemize}
\item[I.] Find the radius and interval of convergence for this power series.
\item[II.] Find the radius and interval of convergence for the series for $f'(x)$ centered at $x=0$.
\item[III.] Find the radius and interval of convergence for the series for $g(x) = \int_0^x f(x)$ centered at $x=0$.
\end{itemize}
%comment in solutions about the effect of int/diff a series has on its ROC and IOC

\begin{freeResponse}
I. To find the radius of convergence, we apply the Ratio Test to the series
$$
\sum_{k=0}^{\infty} \left|\frac{(-1)^k}{k}x^{k}\right| = \sum_{k=0}^\infty \frac{|x|^k}{k}.
$$
We have
$$
\lim_{k \rightarrow \infty} \frac{|x|^{k+1}}{k+1} \cdot \frac{k}{|x|^k} = \lim_{k \rightarrow \infty} \frac{k}{k+1} \cdot |x| = |x|.
$$
The series converges when $|x| < 1$, so the radius of convergence is $1$. To find the interval of convergence, we need to test the endpoints $x=\pm 1$. These series are given by
$$
f(1) = \sum_{k=0}^\infty \frac{(-1)^k}{k}
$$
and
$$
f(-1) = \sum_{k=0}^\infty \frac{(-1)^k}{k}(-1)^k = \sum_{k=0}^\infty \frac{(-1)^{k+1}}{k}.
$$
These are alternating harmonic series and are therefore convergent. We conclude that the interval of convergence for $f(x)$ is $[-1,1]$. 

II. The series for $f'(x)$ has the same radius of convergence. To find its interval of convergence, we note that 
$$
f'(x)=\sum_{k=0}^{\infty} (-1)^k x^{k-1},
$$
whence 
$$
f'(1) = \sum_{k=0}^{\infty} (-1)^k
$$
and
$$
f'(-1) = \sum_{k=0}^{\infty} (-1)^k (-1)^{k-1} = \sum_{k=0}^{\infty} (-1)^{2k-1}.
$$
Neither of these series converge (by the Divergence Test), so the interval of convergence for $f'(x)$ is $(-1,1)$. 

III. The series for $g(x)$ has the same radius of convergence as the series $f(x)$. Moreover, since $g(x)$ is an antiderivative of $f(x)$, the interval of convergence of the series for $g(x)$ must contain the interval of convergence for $f(x)$. We conclude that the interval of convergence for $g(x)$ is $[-1,1]$. 
\end{freeResponse}
\end{problem}


\begin{problem} 
Consider the function $f(x) = \dfrac{1}{1+2x^2}.$

\begin{itemize}
\item[I.] Find the Taylor series centered at $x=0$ for $f(x)$ and give the radius of convergence.
\item[II.] Find the Taylor series centered at $x=0$ for $g(x) = \frac{3x}{(1+2x)^2}$.  Does the series represented by $g(2)$ converge or diverge?
\end{itemize}

\begin{freeResponse}
I. Let $a(x) = \dfrac{1}{1-x}$, which has Taylor series centered at $x=0$ given by 
$$
\sum_{k=0}^\infty x^k,
$$
converging on the interval $(-1,1)$. We write 
$$
f(x) = \frac{1}{1+2x^2} = \frac{1}{1-(-2x^2)} = a(-2x^2),
$$
so that the Taylor series centered at $x=0$ for $f(x)$ is
$$
\sum_{k=0}^\infty (-2x^2)^k = \sum_{k=0}^\infty (-1)^k \cdot 2^k \cdot  x^{2k}.
$$
This series converges when $|-2x^2| < 1$, or $|x| < 1/2$. The radius of convergence of the series for $f(x)$ is therefore $1/2$. 

II. We first consider the function $h(x) = \frac{1}{(1+2x)^2}$. Using the substitution $u=1+2x$, whence $\d u = 2 \d x$, we have
$$
\int \frac{1}{(1+2x)^2} \d x = \frac{1}{2} \int u^{-2} \d u = -\frac{1}{2} u^{-1} + C = -\frac{1}{2} \frac{1}{1 + 2x} + C = -\frac{1}{2} a(-2x) + C,
$$
where $a(x)$ is the function from part I. It follows that the Taylor series centered at $x=0$ for $h(x)$ can be obtained by differentiating the Taylor series
$$
-\frac{1}{2} \sum_{k=0}^\infty (-2x)^k = \sum_{k=0}^\infty (-2)^{-1} \cdot (-2)^k \cdot x^k = \sum_{k=0}^\infty  (-2)^{k-1} \cdot x^k,
$$
and is therefore equal to
$$
\sum_{k=0}^\infty  k \cdot (-2)^{k-1} \cdot x^{k-1}.
$$
Finally, we have $g(x) = 3x \cdot h(x)$, so the Taylor series centered at $x=0$ for $g(x)$ is
$$
3x \cdot \sum_{k=0}^\infty  k \cdot (-2)^{k-1} \cdot x^{k-1} = 
\sum_{k=0}^\infty  3k \cdot (-2)^{k-1} \cdot x^{k}.
$$
Since the radius of convergence of the series for $a(x)$ is $1$, the radius of convergence for $a(-2x)$ is $1/2$. It follows that the radius of convergence for $h(x)$, and hence for $g(x)$, is $1/2$.
\end{freeResponse}
\end{problem}


\begin{problem} 
Consider the function $f(x) = \dfrac{2x^3}{1+2x^2}$.
\begin{itemize}
\item[I.] Show that the Taylor series centered at $x=0$ for $f(x)$ is $\displaystyle \sum_{k=0}^{\infty} (-1)^k2^{k+1}x^{2k+3}$.
\item[II.] Give the seventh degree Taylor polynomial for $f(x)$ centered at $x=0$.
\item[III.] Use the series in Part I to calculate $\displaystyle \lim_{x \rightarrow 0} \dfrac{f(x)-2x^3}{~ 8x^5+5x^7}$. 
%Jim's Note: Many students specifically will use the coefficicents of the x^7 term to find the limit (probably because they are thinking of the rules for limits of sequences and not thinking about what is the dominant term.  Please specifically address this in the solution.
\end{itemize}

\begin{freeResponse}
I. Let $g(x) = 1/(1-x)$. We have 
$$
f(x) = 2x^3 \cdot g(-2x^2).
$$
It follows that the Taylor series centered at $x=0$ for $f(x)$ is 
$$
2x^3 \cdot \sum_{k=0}^\infty (-2x^2)^k = \sum_{k=0}^\infty (-1)^k 2^{k+1} x^{2k+3}.
$$

II. The seventh degree Taylor polynomial centered at $x=0$ for $f(x)$ is obtained by writing out the first few terms from the series above. The polynomial is given by
$$
2x^3 - 4x^5 + 8x^7.
$$

III. We have 
$$
f(x) = 2x^3 - 4x^5 + 8x^7 + O(x^9),
$$
for $|x|<1/2$, where $O(x^9)$ indicates that the remaining terms in the Taylor series representation of $f(x)$ are degree $9$ or higher. Since we are taking the limit as $x$ approaches zero, these terms will vanish more quickly than the lower order terms and do not play a large role in the limit computation. The limit is computed as 
\begin{align*}
\lim_{x \rightarrow 0} \frac{f(x)-2x^3}{ 8x^5+5x^7} &= \lim_{x \rightarrow 0} \frac{2x^3 - 4x^5 + 8x^7 + O(x^9)-2x^3}{ 8x^5+5x^7} \\
&= \lim_{x \rightarrow 0} \frac{4x^5 + 8x^7 + O(x^9)}{ 8x^5+5x^7} \\
&= \lim_{x \rightarrow 0} \frac{x^5(4 + 8x^2 + O(x^4))}{ x^5(8+5x^2)} \\
&= \lim_{x \rightarrow 0} \frac{4 + 8x^2 + O(x^4)}{8 + 5x^2} \\
&= \frac{4}{8} \\
&= \frac{1}{2}.
\end{align*}
\end{freeResponse}
\end{problem}


\end{document}
