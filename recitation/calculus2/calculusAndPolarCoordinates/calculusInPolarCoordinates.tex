\documentclass[noauthor,handout]{ximera}
%handout:  for handout version with no solutions or instructor notes
%handout,instructornotes:  for instructor version with just problems and notes, no solutions
%noinstructornotes:  shows only problem and solutions

%% handout
%% space
%% newpage
%% numbers
%% nooutcomes

%I added the commands here so that I would't have to keep looking them up
%\newcommand{\RR}{\mathbb R}
%\renewcommand{\d}{\,d}
%\newcommand{\dd}[2][]{\frac{d #1}{d #2}}
%\renewcommand{\l}{\ell}
%\newcommand{\ddx}{\frac{d}{dx}}
%\everymath{\displaystyle}
%\newcommand{\dfn}{\textbf}
%\newcommand{\eval}[1]{\bigg[ #1 \bigg]}

%\begin{image}
%\includegraphics[trim= 170 420 250 180]{Figure1.pdf}
%\end{image}

%add a ``.'' below when used in a specific directory.

%%\usepackage{todonotes}
%\usepackage{mathtools} %% Required for wide table Curl and Greens
%\usepackage{cuted} %% Required for wide table Curl and Greens
\newcommand{\todo}{}

\usepackage{esint} % for \oiint
\ifxake%%https://math.meta.stackexchange.com/questions/9973/how-do-you-render-a-closed-surface-double-integral
\renewcommand{\oiint}{{\large\bigcirc}\kern-1.56em\iint}
\fi


\graphicspath{
  {./}
  {ximeraTutorial/}
  {basicPhilosophy/}
  {functionsOfSeveralVariables/}
  {normalVectors/}
  {lagrangeMultipliers/}
  {vectorFields/}
  {greensTheorem/}
  {shapeOfThingsToCome/}
  {dotProducts/}
  {partialDerivativesAndTheGradientVector/}
  {../productAndQuotientRules/exercises/}
  {../normalVectors/exercisesParametricPlots/}
  {../continuityOfFunctionsOfSeveralVariables/exercises/}
  {../partialDerivativesAndTheGradientVector/exercises/}
  {../directionalDerivativeAndChainRule/exercises/}
  {../commonCoordinates/exercisesCylindricalCoordinates/}
  {../commonCoordinates/exercisesSphericalCoordinates/}
  {../greensTheorem/exercisesCurlAndLineIntegrals/}
  {../greensTheorem/exercisesDivergenceAndLineIntegrals/}
  {../shapeOfThingsToCome/exercisesDivergenceTheorem/}
  {../greensTheorem/}
  {../shapeOfThingsToCome/}
  {../separableDifferentialEquations/exercises/}
  {vectorFields/}
}

\newcommand{\mooculus}{\textsf{\textbf{MOOC}\textnormal{\textsf{ULUS}}}}

\usepackage{tkz-euclide}\usepackage{tikz}
\usepackage{tikz-cd}
\usetikzlibrary{arrows}
\tikzset{>=stealth,commutative diagrams/.cd,
  arrow style=tikz,diagrams={>=stealth}} %% cool arrow head
\tikzset{shorten <>/.style={ shorten >=#1, shorten <=#1 } } %% allows shorter vectors

\usetikzlibrary{backgrounds} %% for boxes around graphs
\usetikzlibrary{shapes,positioning}  %% Clouds and stars
\usetikzlibrary{matrix} %% for matrix
\usepgfplotslibrary{polar} %% for polar plots
\usepgfplotslibrary{fillbetween} %% to shade area between curves in TikZ
\usetkzobj{all}
\usepackage[makeroom]{cancel} %% for strike outs
%\usepackage{mathtools} %% for pretty underbrace % Breaks Ximera
%\usepackage{multicol}
\usepackage{pgffor} %% required for integral for loops



%% http://tex.stackexchange.com/questions/66490/drawing-a-tikz-arc-specifying-the-center
%% Draws beach ball
\tikzset{pics/carc/.style args={#1:#2:#3}{code={\draw[pic actions] (#1:#3) arc(#1:#2:#3);}}}



\usepackage{array}
\setlength{\extrarowheight}{+.1cm}
\newdimen\digitwidth
\settowidth\digitwidth{9}
\def\divrule#1#2{
\noalign{\moveright#1\digitwidth
\vbox{\hrule width#2\digitwidth}}}





\newcommand{\RR}{\mathbb R}
\newcommand{\R}{\mathbb R}
\newcommand{\N}{\mathbb N}
\newcommand{\Z}{\mathbb Z}

\newcommand{\sagemath}{\textsf{SageMath}}


%\renewcommand{\d}{\,d\!}
\renewcommand{\d}{\mathop{}\!d}
\newcommand{\dd}[2][]{\frac{\d #1}{\d #2}}
\newcommand{\pp}[2][]{\frac{\partial #1}{\partial #2}}
\renewcommand{\l}{\ell}
\newcommand{\ddx}{\frac{d}{\d x}}

\newcommand{\zeroOverZero}{\ensuremath{\boldsymbol{\tfrac{0}{0}}}}
\newcommand{\inftyOverInfty}{\ensuremath{\boldsymbol{\tfrac{\infty}{\infty}}}}
\newcommand{\zeroOverInfty}{\ensuremath{\boldsymbol{\tfrac{0}{\infty}}}}
\newcommand{\zeroTimesInfty}{\ensuremath{\small\boldsymbol{0\cdot \infty}}}
\newcommand{\inftyMinusInfty}{\ensuremath{\small\boldsymbol{\infty - \infty}}}
\newcommand{\oneToInfty}{\ensuremath{\boldsymbol{1^\infty}}}
\newcommand{\zeroToZero}{\ensuremath{\boldsymbol{0^0}}}
\newcommand{\inftyToZero}{\ensuremath{\boldsymbol{\infty^0}}}



\newcommand{\numOverZero}{\ensuremath{\boldsymbol{\tfrac{\#}{0}}}}
\newcommand{\dfn}{\textbf}
%\newcommand{\unit}{\,\mathrm}
\newcommand{\unit}{\mathop{}\!\mathrm}
\newcommand{\eval}[1]{\bigg[ #1 \bigg]}
\newcommand{\seq}[1]{\left( #1 \right)}
\renewcommand{\epsilon}{\varepsilon}
\renewcommand{\phi}{\varphi}


\renewcommand{\iff}{\Leftrightarrow}

\DeclareMathOperator{\arccot}{arccot}
\DeclareMathOperator{\arcsec}{arcsec}
\DeclareMathOperator{\arccsc}{arccsc}
\DeclareMathOperator{\si}{Si}
\DeclareMathOperator{\scal}{scal}
\DeclareMathOperator{\sign}{sign}


%% \newcommand{\tightoverset}[2]{% for arrow vec
%%   \mathop{#2}\limits^{\vbox to -.5ex{\kern-0.75ex\hbox{$#1$}\vss}}}
\newcommand{\arrowvec}[1]{{\overset{\rightharpoonup}{#1}}}
%\renewcommand{\vec}[1]{\arrowvec{\mathbf{#1}}}
\renewcommand{\vec}[1]{{\overset{\boldsymbol{\rightharpoonup}}{\mathbf{#1}}}\hspace{0in}}

\newcommand{\point}[1]{\left(#1\right)} %this allows \vector{ to be changed to \vector{ with a quick find and replace
\newcommand{\pt}[1]{\mathbf{#1}} %this allows \vec{ to be changed to \vec{ with a quick find and replace
\newcommand{\Lim}[2]{\lim_{\point{#1} \to \point{#2}}} %Bart, I changed this to point since I want to use it.  It runs through both of the exercise and exerciseE files in limits section, which is why it was in each document to start with.

\DeclareMathOperator{\proj}{\mathbf{proj}}
\newcommand{\veci}{{\boldsymbol{\hat{\imath}}}}
\newcommand{\vecj}{{\boldsymbol{\hat{\jmath}}}}
\newcommand{\veck}{{\boldsymbol{\hat{k}}}}
\newcommand{\vecl}{\vec{\boldsymbol{\l}}}
\newcommand{\uvec}[1]{\mathbf{\hat{#1}}}
\newcommand{\utan}{\mathbf{\hat{t}}}
\newcommand{\unormal}{\mathbf{\hat{n}}}
\newcommand{\ubinormal}{\mathbf{\hat{b}}}

\newcommand{\dotp}{\bullet}
\newcommand{\cross}{\boldsymbol\times}
\newcommand{\grad}{\boldsymbol\nabla}
\newcommand{\divergence}{\grad\dotp}
\newcommand{\curl}{\grad\cross}
%\DeclareMathOperator{\divergence}{divergence}
%\DeclareMathOperator{\curl}[1]{\grad\cross #1}
\newcommand{\lto}{\mathop{\longrightarrow\,}\limits}

\renewcommand{\bar}{\overline}

\colorlet{textColor}{black}
\colorlet{background}{white}
\colorlet{penColor}{blue!50!black} % Color of a curve in a plot
\colorlet{penColor2}{red!50!black}% Color of a curve in a plot
\colorlet{penColor3}{red!50!blue} % Color of a curve in a plot
\colorlet{penColor4}{green!50!black} % Color of a curve in a plot
\colorlet{penColor5}{orange!80!black} % Color of a curve in a plot
\colorlet{penColor6}{yellow!70!black} % Color of a curve in a plot
\colorlet{fill1}{penColor!20} % Color of fill in a plot
\colorlet{fill2}{penColor2!20} % Color of fill in a plot
\colorlet{fillp}{fill1} % Color of positive area
\colorlet{filln}{penColor2!20} % Color of negative area
\colorlet{fill3}{penColor3!20} % Fill
\colorlet{fill4}{penColor4!20} % Fill
\colorlet{fill5}{penColor5!20} % Fill
\colorlet{gridColor}{gray!50} % Color of grid in a plot

\newcommand{\surfaceColor}{violet}
\newcommand{\surfaceColorTwo}{redyellow}
\newcommand{\sliceColor}{greenyellow}




\pgfmathdeclarefunction{gauss}{2}{% gives gaussian
  \pgfmathparse{1/(#2*sqrt(2*pi))*exp(-((x-#1)^2)/(2*#2^2))}%
}


%%%%%%%%%%%%%
%% Vectors
%%%%%%%%%%%%%

%% Simple horiz vectors
\renewcommand{\vector}[1]{\left\langle #1\right\rangle}


%% %% Complex Horiz Vectors with angle brackets
%% \makeatletter
%% \renewcommand{\vector}[2][ , ]{\left\langle%
%%   \def\nextitem{\def\nextitem{#1}}%
%%   \@for \el:=#2\do{\nextitem\el}\right\rangle%
%% }
%% \makeatother

%% %% Vertical Vectors
%% \def\vector#1{\begin{bmatrix}\vecListA#1,,\end{bmatrix}}
%% \def\vecListA#1,{\if,#1,\else #1\cr \expandafter \vecListA \fi}

%%%%%%%%%%%%%
%% End of vectors
%%%%%%%%%%%%%

%\newcommand{\fullwidth}{}
%\newcommand{\normalwidth}{}



%% makes a snazzy t-chart for evaluating functions
%\newenvironment{tchart}{\rowcolors{2}{}{background!90!textColor}\array}{\endarray}

%%This is to help with formatting on future title pages.
\newenvironment{sectionOutcomes}{}{}



%% Flowchart stuff
%\tikzstyle{startstop} = [rectangle, rounded corners, minimum width=3cm, minimum height=1cm,text centered, draw=black]
%\tikzstyle{question} = [rectangle, minimum width=3cm, minimum height=1cm, text centered, draw=black]
%\tikzstyle{decision} = [trapezium, trapezium left angle=70, trapezium right angle=110, minimum width=3cm, minimum height=1cm, text centered, draw=black]
%\tikzstyle{question} = [rectangle, rounded corners, minimum width=3cm, minimum height=1cm,text centered, draw=black]
%\tikzstyle{process} = [rectangle, minimum width=3cm, minimum height=1cm, text centered, draw=black]
%\tikzstyle{decision} = [trapezium, trapezium left angle=70, trapezium right angle=110, minimum width=3cm, minimum height=1cm, text centered, draw=black]

%\usepackage{todonotes}
%\usepackage{mathtools} %% Required for wide table Curl and Greens
%\usepackage{cuted} %% Required for wide table Curl and Greens
\newcommand{\todo}{}

\usepackage{esint} % for \oiint
\ifxake%%https://math.meta.stackexchange.com/questions/9973/how-do-you-render-a-closed-surface-double-integral
\renewcommand{\oiint}{{\large\bigcirc}\kern-1.56em\iint}
\fi


\graphicspath{
  {./}
  {ximeraTutorial/}
  {basicPhilosophy/}
  {functionsOfSeveralVariables/}
  {normalVectors/}
  {lagrangeMultipliers/}
  {vectorFields/}
  {greensTheorem/}
  {shapeOfThingsToCome/}
  {dotProducts/}
  {partialDerivativesAndTheGradientVector/}
  {../productAndQuotientRules/exercises/}
  {../normalVectors/exercisesParametricPlots/}
  {../continuityOfFunctionsOfSeveralVariables/exercises/}
  {../partialDerivativesAndTheGradientVector/exercises/}
  {../directionalDerivativeAndChainRule/exercises/}
  {../commonCoordinates/exercisesCylindricalCoordinates/}
  {../commonCoordinates/exercisesSphericalCoordinates/}
  {../greensTheorem/exercisesCurlAndLineIntegrals/}
  {../greensTheorem/exercisesDivergenceAndLineIntegrals/}
  {../shapeOfThingsToCome/exercisesDivergenceTheorem/}
  {../greensTheorem/}
  {../shapeOfThingsToCome/}
  {../separableDifferentialEquations/exercises/}
  {vectorFields/}
}

\newcommand{\mooculus}{\textsf{\textbf{MOOC}\textnormal{\textsf{ULUS}}}}

\usepackage{tkz-euclide}\usepackage{tikz}
\usepackage{tikz-cd}
\usetikzlibrary{arrows}
\tikzset{>=stealth,commutative diagrams/.cd,
  arrow style=tikz,diagrams={>=stealth}} %% cool arrow head
\tikzset{shorten <>/.style={ shorten >=#1, shorten <=#1 } } %% allows shorter vectors

\usetikzlibrary{backgrounds} %% for boxes around graphs
\usetikzlibrary{shapes,positioning}  %% Clouds and stars
\usetikzlibrary{matrix} %% for matrix
\usepgfplotslibrary{polar} %% for polar plots
\usepgfplotslibrary{fillbetween} %% to shade area between curves in TikZ
\usetkzobj{all}
\usepackage[makeroom]{cancel} %% for strike outs
%\usepackage{mathtools} %% for pretty underbrace % Breaks Ximera
%\usepackage{multicol}
\usepackage{pgffor} %% required for integral for loops



%% http://tex.stackexchange.com/questions/66490/drawing-a-tikz-arc-specifying-the-center
%% Draws beach ball
\tikzset{pics/carc/.style args={#1:#2:#3}{code={\draw[pic actions] (#1:#3) arc(#1:#2:#3);}}}



\usepackage{array}
\setlength{\extrarowheight}{+.1cm}
\newdimen\digitwidth
\settowidth\digitwidth{9}
\def\divrule#1#2{
\noalign{\moveright#1\digitwidth
\vbox{\hrule width#2\digitwidth}}}





\newcommand{\RR}{\mathbb R}
\newcommand{\R}{\mathbb R}
\newcommand{\N}{\mathbb N}
\newcommand{\Z}{\mathbb Z}

\newcommand{\sagemath}{\textsf{SageMath}}


%\renewcommand{\d}{\,d\!}
\renewcommand{\d}{\mathop{}\!d}
\newcommand{\dd}[2][]{\frac{\d #1}{\d #2}}
\newcommand{\pp}[2][]{\frac{\partial #1}{\partial #2}}
\renewcommand{\l}{\ell}
\newcommand{\ddx}{\frac{d}{\d x}}

\newcommand{\zeroOverZero}{\ensuremath{\boldsymbol{\tfrac{0}{0}}}}
\newcommand{\inftyOverInfty}{\ensuremath{\boldsymbol{\tfrac{\infty}{\infty}}}}
\newcommand{\zeroOverInfty}{\ensuremath{\boldsymbol{\tfrac{0}{\infty}}}}
\newcommand{\zeroTimesInfty}{\ensuremath{\small\boldsymbol{0\cdot \infty}}}
\newcommand{\inftyMinusInfty}{\ensuremath{\small\boldsymbol{\infty - \infty}}}
\newcommand{\oneToInfty}{\ensuremath{\boldsymbol{1^\infty}}}
\newcommand{\zeroToZero}{\ensuremath{\boldsymbol{0^0}}}
\newcommand{\inftyToZero}{\ensuremath{\boldsymbol{\infty^0}}}



\newcommand{\numOverZero}{\ensuremath{\boldsymbol{\tfrac{\#}{0}}}}
\newcommand{\dfn}{\textbf}
%\newcommand{\unit}{\,\mathrm}
\newcommand{\unit}{\mathop{}\!\mathrm}
\newcommand{\eval}[1]{\bigg[ #1 \bigg]}
\newcommand{\seq}[1]{\left( #1 \right)}
\renewcommand{\epsilon}{\varepsilon}
\renewcommand{\phi}{\varphi}


\renewcommand{\iff}{\Leftrightarrow}

\DeclareMathOperator{\arccot}{arccot}
\DeclareMathOperator{\arcsec}{arcsec}
\DeclareMathOperator{\arccsc}{arccsc}
\DeclareMathOperator{\si}{Si}
\DeclareMathOperator{\scal}{scal}
\DeclareMathOperator{\sign}{sign}


%% \newcommand{\tightoverset}[2]{% for arrow vec
%%   \mathop{#2}\limits^{\vbox to -.5ex{\kern-0.75ex\hbox{$#1$}\vss}}}
\newcommand{\arrowvec}[1]{{\overset{\rightharpoonup}{#1}}}
%\renewcommand{\vec}[1]{\arrowvec{\mathbf{#1}}}
\renewcommand{\vec}[1]{{\overset{\boldsymbol{\rightharpoonup}}{\mathbf{#1}}}\hspace{0in}}

\newcommand{\point}[1]{\left(#1\right)} %this allows \vector{ to be changed to \vector{ with a quick find and replace
\newcommand{\pt}[1]{\mathbf{#1}} %this allows \vec{ to be changed to \vec{ with a quick find and replace
\newcommand{\Lim}[2]{\lim_{\point{#1} \to \point{#2}}} %Bart, I changed this to point since I want to use it.  It runs through both of the exercise and exerciseE files in limits section, which is why it was in each document to start with.

\DeclareMathOperator{\proj}{\mathbf{proj}}
\newcommand{\veci}{{\boldsymbol{\hat{\imath}}}}
\newcommand{\vecj}{{\boldsymbol{\hat{\jmath}}}}
\newcommand{\veck}{{\boldsymbol{\hat{k}}}}
\newcommand{\vecl}{\vec{\boldsymbol{\l}}}
\newcommand{\uvec}[1]{\mathbf{\hat{#1}}}
\newcommand{\utan}{\mathbf{\hat{t}}}
\newcommand{\unormal}{\mathbf{\hat{n}}}
\newcommand{\ubinormal}{\mathbf{\hat{b}}}

\newcommand{\dotp}{\bullet}
\newcommand{\cross}{\boldsymbol\times}
\newcommand{\grad}{\boldsymbol\nabla}
\newcommand{\divergence}{\grad\dotp}
\newcommand{\curl}{\grad\cross}
%\DeclareMathOperator{\divergence}{divergence}
%\DeclareMathOperator{\curl}[1]{\grad\cross #1}
\newcommand{\lto}{\mathop{\longrightarrow\,}\limits}

\renewcommand{\bar}{\overline}

\colorlet{textColor}{black}
\colorlet{background}{white}
\colorlet{penColor}{blue!50!black} % Color of a curve in a plot
\colorlet{penColor2}{red!50!black}% Color of a curve in a plot
\colorlet{penColor3}{red!50!blue} % Color of a curve in a plot
\colorlet{penColor4}{green!50!black} % Color of a curve in a plot
\colorlet{penColor5}{orange!80!black} % Color of a curve in a plot
\colorlet{penColor6}{yellow!70!black} % Color of a curve in a plot
\colorlet{fill1}{penColor!20} % Color of fill in a plot
\colorlet{fill2}{penColor2!20} % Color of fill in a plot
\colorlet{fillp}{fill1} % Color of positive area
\colorlet{filln}{penColor2!20} % Color of negative area
\colorlet{fill3}{penColor3!20} % Fill
\colorlet{fill4}{penColor4!20} % Fill
\colorlet{fill5}{penColor5!20} % Fill
\colorlet{gridColor}{gray!50} % Color of grid in a plot

\newcommand{\surfaceColor}{violet}
\newcommand{\surfaceColorTwo}{redyellow}
\newcommand{\sliceColor}{greenyellow}




\pgfmathdeclarefunction{gauss}{2}{% gives gaussian
  \pgfmathparse{1/(#2*sqrt(2*pi))*exp(-((x-#1)^2)/(2*#2^2))}%
}


%%%%%%%%%%%%%
%% Vectors
%%%%%%%%%%%%%

%% Simple horiz vectors
\renewcommand{\vector}[1]{\left\langle #1\right\rangle}


%% %% Complex Horiz Vectors with angle brackets
%% \makeatletter
%% \renewcommand{\vector}[2][ , ]{\left\langle%
%%   \def\nextitem{\def\nextitem{#1}}%
%%   \@for \el:=#2\do{\nextitem\el}\right\rangle%
%% }
%% \makeatother

%% %% Vertical Vectors
%% \def\vector#1{\begin{bmatrix}\vecListA#1,,\end{bmatrix}}
%% \def\vecListA#1,{\if,#1,\else #1\cr \expandafter \vecListA \fi}

%%%%%%%%%%%%%
%% End of vectors
%%%%%%%%%%%%%

%\newcommand{\fullwidth}{}
%\newcommand{\normalwidth}{}



%% makes a snazzy t-chart for evaluating functions
%\newenvironment{tchart}{\rowcolors{2}{}{background!90!textColor}\array}{\endarray}

%%This is to help with formatting on future title pages.
\newenvironment{sectionOutcomes}{}{}



%% Flowchart stuff
%\tikzstyle{startstop} = [rectangle, rounded corners, minimum width=3cm, minimum height=1cm,text centered, draw=black]
%\tikzstyle{question} = [rectangle, minimum width=3cm, minimum height=1cm, text centered, draw=black]
%\tikzstyle{decision} = [trapezium, trapezium left angle=70, trapezium right angle=110, minimum width=3cm, minimum height=1cm, text centered, draw=black]
%\tikzstyle{question} = [rectangle, rounded corners, minimum width=3cm, minimum height=1cm,text centered, draw=black]
%\tikzstyle{process} = [rectangle, minimum width=3cm, minimum height=1cm, text centered, draw=black]
%\tikzstyle{decision} = [trapezium, trapezium left angle=70, trapezium right angle=110, minimum width=3cm, minimum height=1cm, text centered, draw=black]




\author{Tom Needham and Jim Talamo}

\outcome{Find slopes of tangent lines to polar curves.}
\outcome{Determine areas of regions using integrals in polar coordinates.}

\title[Collaborate:]{Calculus in Polar Coordinates}

\begin{document}
\begin{abstract}
\end{abstract}
\maketitle

\section{Discussion Questions}

\begin{problem}

On the axes below, plot the point on the polar curve $r = \cos (4 \theta)$ corresponding to $\theta = 3\pi/4$. 

\begin{image}  
  \begin{tikzpicture}  
    \begin{axis}[  
        xmin=-1.5,  
        xmax=1.5,  
        ymin=-1.2,  
        ymax=1.2,  
        xtick={-1,1},
        ytick={-1,1},
        axis lines=center,  
        xlabel=$x$,  
        ylabel=$y$,  
        every axis y label/.style={at=(current axis.above origin),anchor=south},  
        every axis x label/.style={at=(current axis.right of origin),anchor=west},  
      ]  
      \addplot[data cs=polar,penColor,domain=0:360,samples=360,smooth, thick] (x,{cos(4*x)});
      
            \end{axis}  
  \end{tikzpicture}  
\end{image} 


\begin{freeResponse}
Plugging $\theta = 3\pi/4$ into the equation, we have
$$
r = \cos (4 \cdot 3 \pi/4)  = \cos (3\pi) = -1. 
$$
It follows that the relevant point has Cartesian coordinates.
\begin{align*}
x &= -\cos(3 \pi /4) = \sqrt{2}/2 \\
y &= -\sin(3 \pi/4) = - \sqrt{2}/2.
\end{align*}
\end{freeResponse}
\end{problem}

\begin{problem}
Consider the curve given by the plot of the polar curve $r = 2 \cos(\theta)$. A student claims that the slope of the tangent line to the curve at $\theta = \pi/2$ is given by
$$
\left.\frac{\d r}{\d \theta}\right|_{\theta = \pi/2} =  \frac{\d}{\d \theta }  \left[2 \cos(\theta) \right|_{\theta = \pi/2}=  \left. -2 \sin(\theta) \right|_{\theta = \pi/2} = -2.
$$
Is the student correct? If not, explain the mistake in their reasoning and find the correct slope.
\begin{freeResponse}
The student is not correct, since the slope of the tangent line should be calculated in Cartesian coordinates. Writing
\begin{align*}
x &= r \cos(\theta) = 2 \cos(\theta) \cos(\theta) = 2 \cos^2(\theta) \\
y &= r \sin(\theta) = 2 \cos(\theta) \sin(\theta) =  \sin(2 \theta),
\end{align*}
we have
\begin{align*}
\frac{\d x}{\d \theta} &= -4 \cos (\theta)\sin(\theta) \\
\frac{\d y}{\d \theta} &= 2 \cos(2\theta).
\end{align*}
Evaluating these derivatives at $\theta = \pi/2$ yields
\begin{align*}
\left.\frac{\d x}{\d \theta}\right|_{\theta = \pi/2} &= 0 \\
\left.\frac{\d y}{\d \theta}\right|_{\theta = \pi/2} &= 2.
\end{align*}
By the chain rule, $\frac{\d y}{\d x} = \frac{\d y/ \d \theta}{\d x/ \d \theta}$, and it follows that there is a vertical tangent line at $\theta = \pi/2$ . 
\end{freeResponse}
\end{problem}

%%%%%%%%%%%%%%%%%%%%%%%%%%%%%%%%%%%%%%%%%%%%%
\begin{problem}
The shaded region shown below has boundary curve given by the plot of the polar curve $r = \cos(3\theta)$. 

\begin{image}  
  \begin{tikzpicture}  
    \begin{axis}[  
        xmin=-1.5,  
        xmax=1.5,  
        ymin=-1.5,  
        ymax=1.5,  
        axis lines=center,  
        xlabel=$x$,  
        ylabel=$y$,  
        every axis y label/.style={at=(current axis.above origin),anchor=south},  axis on top
        every axis x label/.style={at=(current axis.right of origin),anchor=west},  axis on top
      ]  
      \addplot [data cs=polar, very thick, mark=none,fill=fill1,domain=0:360 ,samples=360,smooth] (x, {cos(3*x)}) ;
      \addplot[data cs=polar,penColor,domain=0:360,samples=360,smooth, thick] (x,{cos(3*x)});
      
            \end{axis}  
  \end{tikzpicture}  
\end{image} 

A student claims that the area of the region is given by 
$$
\int_0^{2\pi} \frac{1}{2} \cos^2(3\theta) \d \theta.
$$
Is the student correct? Explain why or why not. If the student is incorrect, set up an integral which gives the correct area.

\begin{freeResponse}
The student is not correct. Inspecting the equation, we see that the full image of the curve is plotted as $\theta$ runs from $0$ to $\pi$. Therefore the area of the region is given by 
$$
\int_0^{\pi} \frac{1}{2} \cos^2(3\theta) \d \theta.
$$

Note that by utilizing symmetry, we can avoid this issue altogether since the area of the region can be computed by evaluating

$$
6 \cdot \int_0^{\pi/6} \frac{1}{2} \cos^2(3\theta) \d \theta.
$$

Compute each of these integrals of these for yourself to verify.
\end{freeResponse}
\end{problem}

%%%%%%%%%%%%%%%%%%%%%%%%%%%%%%%%%%%%%%%%%%%%%
\section{Group Work}

\begin{problem}
A plot of the polar curve $r=1+2\sin(\theta)$ is shown below.
\begin{image}  
  \begin{tikzpicture}  
    \begin{axis}[  
        xmin=-3,  
        xmax=3,  
        ymin=-1,  
        ymax=3.5,  
        axis lines=center,  
        xlabel=$x$,  
        ylabel=$y$,  
        every axis y label/.style={at=(current axis.above origin),anchor=south},  
        every axis x label/.style={at=(current axis.right of origin),anchor=west},  
      ]  
      \addplot[data cs=polar,penColor,domain=0:360,samples=360,smooth, thick] (x,{1+2*sin(x)});
      
            \end{axis}  
  \end{tikzpicture}  
\end{image} 

\begin{itemize}
\item[I.] Find the equation of the tangent line to the point corresponding to $\theta = \pi/6$.
\item[II.]  Determine the Cartesian coordinates of all points on the curve with a horizontal tangent line.
\end{itemize}
\begin{freeResponse}
I. We find the slope of the tangent line by first writing
\begin{align*}
x &= r \cos(\theta) = (1+ 2 \sin(\theta)) \cdot \cos(\theta) = \cos(\theta) + 2 \sin\theta \cos(\theta) = \cos(\theta) + \sin (2\theta), \\
y&= r \sin(\theta) = (1+ 2 \sin(\theta)) \cdot \sin(\theta) = \sin(\theta) + 2 \sin^2 \theta,
\end{align*}
so that
\begin{align*}
\frac{\d x}{\d \theta} &= -\sin(\theta) + 2 \cos(2 \theta), \\
\frac{\d y}{\d \theta} &= \cos(\theta) + 4 \sin(\theta) \cos(\theta) = \cos(\theta) + 2 \sin(2 \theta).
\end{align*}
Evaluating at $\theta = \pi/6$ yields 
\begin{align*}
\left.\frac{\d x}{\d \theta}\right|_{\theta = \pi/6} &= -\frac{1}{2} +  2 \frac{1}{2} = \frac{1}{2} \\
\left.\frac{\d y}{\d \theta}\right|_{\theta = \pi/6} &= \frac{\sqrt{3}}{2} + 2 \frac{\sqrt{3}}{2} = 3 \frac{\sqrt{3}}{2},
\end{align*}
and the chain rule therefore implies that the slope of the tangent line is $3 \sqrt{3}$.

The point that the tangent line passes through has Cartesian coordinates
\begin{align*}
x &= \cos(\pi/6) + \sin (\pi/3) = \frac{\sqrt{3}}{2} + \frac{\sqrt{3}}{2} = \sqrt{3} \\
y &= \sin(\pi/6) + 2 \sin^2(\pi/6) = \frac{1}{2} + 2 \cdot \frac{1}{4} = 1.
\end{align*}
We conclude that the equation of the tangent line is $y-1 = 3 \sqrt{3}(x-\sqrt{3})$.

II. The curve has a horizontal tangent line when $\d y/ \d \theta = 0$ and $\d x /\d \theta \neq 0$. Using our formulas above, we see that we need to solve
$$
\cos(\theta) + 2 \sin(2 \theta) = 0.
$$
Then 
$$
0 = \cos(\theta) + 4 \sin(\theta) \cos(\theta) = \cos(\theta) ( 1+ 4 \sin(\theta)),
$$
so that $\cos(\theta) = 0$ or $\sin(\theta) = - \frac{1}{4}$. The first equation has solutions (on the domain $[0,2\pi)$) $\theta = \pi/2$ or $3\pi/2$. The Cartesian coordinates of the corresponding points are
\begin{align*}
x &= \cos (\pi/2) + \sin(2 \cdot \pi/2) = 0\\
y &= \sin(\pi/2) + 2 \sin^2(\pi/2) = 3
\end{align*}
and, similarly, $(x,y) = (0,1)$. 

The equation $\sin(\theta) = -\frac{1}{4}$ has solutions $\theta = \arcsin (-1/4)$ and $\pi - \arcsin(-1/4)$ and the corresponding points in Cartesian coordinates are
\begin{align*}
x &= \cos (\arcsin(-1/4)) + \sin (2 \arcsin(-1/4)) = \frac{\sqrt{15}}{4} + 2 \frac{\sqrt{15}}{4} \cdot \frac{-1}{4} = \frac{\sqrt{15}}{8}, \\
y &= \sin(\arcsin(-1/4)) + 2 \sin^2 (\arcsin(-1/4)) = -\frac{1}{4} + 2 \frac{1}{16} = -\frac{1}{8}
\end{align*}
and $(x,y) = (-\sqrt{15}/8,-1/8)$. 
\end{freeResponse}
\end{problem}

%%%%%%%%%%%%%%%%%%%%%%%%%%%%%%%%%%%%%%%%%%%%%
\begin{problem}

The curves $r=2\cos(2\theta)$ and $r=1$ are shown below.

\begin{image}  
  \begin{tikzpicture}  
    \begin{axis}[  
        xmin=-3.5,  
        xmax=3.5,  
        ymin=-3,  
        ymax=3,  
        axis lines=center,  
        xlabel=$x$,  
        ylabel=$y$,  
        every axis y label/.style={at=(current axis.above origin),anchor=south},  axis on top
        every axis x label/.style={at=(current axis.right of origin),anchor=west},  axis on top
      ]  
%      \addplot [data cs=polar, very thick, mark=none,fill=fill1,domain=-45:45 ,samples=180,smooth] (x, {2*cos(2*x)}) ;
      \addplot[data cs=polar, very thick, mark=none, fill=white,  domain=0:360, samples=180, smooth] (x, {1});
%            \addplot [data cs=polar, very thick, mark=none,fill=fill1,domain=30:45 ,samples=180,smooth] (x, {2*cos(2*x)})-- (axis cs:0,0)[penColor4!50] ;
%            \addplot [data cs=polar, very thick, mark=none,fill=fill1,domain=-45:-30 ,samples=180,smooth] (x, {2*cos(2*x)})-- (axis cs:0,0)[penColor4!50] ;
%      \addplot[data cs=polar, very thick, mark=none, fill=fill1,  domain=-30:30, samples=360, smooth] (x, {1}) -- (axis cs:0,0)[penColor4!50];
      \addplot[data cs=polar,penColor,domain=0:360,samples=360,smooth, ultra thick] (x,{2*cos(2*x)});
   	   \addplot[data cs=polar,penColor2,domain=0:360,samples=360,smooth, thick] (x,{1});
	   
\node[penColor] at (axis cs: 1.5,2) {$r = 2\cos(2\theta)$};
\node[penColor2] at (axis cs: -1.5,1) {$r = 1$};
      
            \end{axis}  
  \end{tikzpicture}  
  \end{image}
  
I.  Find the areas of the region that is outside of the circle $r=1$ but inside $r=2\cos(2\theta)$.

II. Find the area of the region that is inside of both curves.


 \begin{freeResponse}
We begin by finding the intersection point of the curves. The intersections occur exactly when $2 \cos (2 \theta) = 1$, or $\cos (2 \theta) = 1/2$. There are many solutions to this equation, but we are interested in the smallest positive value of $\theta$ which satisfies the equation; namely, $2\theta = \pi/3$ or $\theta = \pi/6$. 

I.  The relevant area is shown below.

\begin{image}  
  \begin{tikzpicture}  
    \begin{axis}[  
        xmin=-3.5,  
        xmax=3.5,  
        ymin=-3,  
        ymax=3,  
        axis lines=center,  
        xlabel=$$,  
        ylabel=$y$,  
        every axis y label/.style={at=(current axis.above origin),anchor=south},  axis on top
        every axis x label/.style={at=(current axis.right of origin),anchor=west},  axis on top
      ]  
      \addplot [data cs=polar, very thick, mark=none,fill=fill1,domain=-0:360 ,samples=180,smooth] (x, {2*cos(2*x)}) ;
            \addplot [data cs=polar, very thick, mark=none,fill=penColor!70,domain=-0:45 ,samples=180,smooth] (x, {2*cos(2*x)}) ;
      \addplot[data cs=polar, very thick, mark=none, fill=white,  domain=0:360, samples=180, smooth] (x, {1});
%            \addplot [data cs=polar, very thick, mark=none,fill=fill1,domain=30:45 ,samples=180,smooth] (x, {2*cos(2*x)})-- (axis cs:0,0)[penColor4!50] ;
%            \addplot [data cs=polar, very thick, mark=none,fill=fill1,domain=-45:-30 ,samples=180,smooth] (x, {2*cos(2*x)})-- (axis cs:0,0)[penColor4!50] ;
%      \addplot[data cs=polar, very thick, mark=none, fill=fill1,  domain=-30:30, samples=360, smooth] (x, {1}) -- (axis cs:0,0)[penColor4!50];
      \addplot[data cs=polar,penColor,domain=0:360,samples=360,smooth, ultra thick] (x,{2*cos(2*x)});
   	   \addplot[data cs=polar,penColor2,domain=0:360,samples=360,smooth, thick] (x,{1});
	   
\addplot [draw=penColor5,domain=0:2,very thick,smooth] {tan(30)*x};
\addplot [draw=penColor5,domain=0:3,very thick,smooth] {0};
	   
\node[penColor] at (axis cs: 1.5,2) {$r = 2\cos(2\theta)$};
\node[penColor2] at (axis cs: -1.5,1) {$r = 1$};
\node[penColor5] at (axis cs: 2.5,1.4) {\small $\theta=\frac{\pi}{6}$};
\node[penColor5] at (axis cs: 2.8,.3) {\small$\theta=0$};
      
            \end{axis}  
  \end{tikzpicture}  
  \end{image}
  
Note that by using symmetry,

\[
\textrm{ Total Area } = 8 \cdot \left< \textrm{ Area of darkly shaded region } \right>.
\]  

The limits of integration to compute the dark area are $\theta = 0$ to $\pi/6$. The outer radius of the region is $2 \cos (2\theta)$ and the inner radius is $1$. Thus the area is
$$
A = 8 \cdot \int_0^{\pi/6} \frac{1}{2}(4 \cos^2 (2\theta) - 1) \d \theta = \frac{\sqrt{3}}{8}.
$$

Note that evaluating this integral requires some careful computations and use of the identity $$\cos^2(\alpha) = \frac{1}{2}+\frac{1}{2}\cos(2 \alpha).$$

\begin{align*}
A &= 8 \cdot \int_0^{\pi/6} \frac{1}{2}\big( (2\cos(2\theta))^2 - 1\big) \d \theta \\
&=   \int_0^{\pi/6} 16 \left[\frac{1}{2} + \frac{1}{2} \cos(4\theta)\right] - 4 \d \theta \\
&=   \int_0^{\pi/6} 4 + 8 \cos(4\theta) \d \theta \\
&=   \eval{4\theta + 2 \sin(4\theta)}_{0}^{\pi/6} \\
&= \left[\frac{2\pi}{3} + 2 \sin\left(\frac{2\pi}{3}\right) \right]- 0 \\
&= \frac{2\pi}{3} + \sqrt{3}
\end{align*}

II. The area of the region inside of both curves can be found by considering the region below and multiplying its area by $8$.

\begin{image}  
  \begin{tikzpicture}  
    \begin{axis}[  
        xmin=-3.5,  
        xmax=3.5,  
        ymin=-3,  
        ymax=3,  
        axis lines=center,  
        xlabel=$$,  
        ylabel=$y$,  
        every axis y label/.style={at=(current axis.above origin),anchor=south},  axis on top
        every axis x label/.style={at=(current axis.right of origin),anchor=west},  axis on top
      ]  
%      \addplot [data cs=polar, very thick, mark=none,fill=fill1,domain=-0:360 ,samples=180,smooth] (x, {2*cos(2*x)}) ;
%            \addplot [data cs=polar, very thick, mark=none,fill=penColor!70,domain=-0:45 ,samples=180,smooth] (x, {2*cos(2*x)}) ;
%      \addplot[data cs=polar, very thick, mark=none, fill=white,  domain=0:360, samples=180, smooth] (x, {1});
            \addplot [data cs=polar, very thick, mark=none,fill=fill1,domain=30:45 ,samples=180,smooth] (x, {2*cos(2*x)})-- (axis cs:0,0)[penColor4!50] ;
%            \addplot [data cs=polar, very thick, mark=none,fill=fill1,domain=-45:-30 ,samples=180,smooth] (x, {2*cos(2*x)})-- (axis cs:0,0)[penColor4!50] ;
      \addplot[data cs=polar, very thick, mark=none, fill=fill1,  domain=0:30, samples=360, smooth] (x, {1}) -- (axis cs:0,0)[penColor4!50];
      \addplot[data cs=polar,penColor,domain=0:360,samples=360,smooth, ultra thick] (x,{2*cos(2*x)});
   	   \addplot[data cs=polar,penColor2,domain=0:360,samples=360,smooth, thick] (x,{1});
	   
\addplot [draw=penColor5,domain=0:2,very thick,smooth] {tan(30)*x};
\addplot [draw=penColor5,domain=0:3,very thick,smooth] {0};
	   
\node[penColor] at (axis cs: 1.5,2) {$r = 2\cos(2\theta)$};
\node[penColor2] at (axis cs: -1.5,1) {$r = 1$};
%\node[penColor5] at (axis cs: 2.5,1.4) {\small $\theta=\frac{\pi}{6}$};
%\node[penColor5] at (axis cs: 2.8,.3) {\small$\theta=0$};
      
            \end{axis}  
  \end{tikzpicture}  
  \end{image}
  
Let's zoom in a bit on this region.

\begin{image}  
  \begin{tikzpicture}  
    \begin{axis}[  
        xmin=-.1,  
        xmax=1.5,  
        ymin=-.1,  
        ymax=1.3,  
        axis lines=center,  
        xlabel=$x$,  xtick=1,
        ylabel=$y$,  ytick =1,
        every axis y label/.style={at=(current axis.above origin),anchor=south},  axis on top
        every axis x label/.style={at=(current axis.right of origin),anchor=west},  axis on top
      ]  
%      \addplot [data cs=polar, very thick, mark=none,fill=fill1,domain=-0:360 ,samples=180,smooth] (x, {2*cos(2*x)}) ;
%            \addplot [data cs=polar, very thick, mark=none,fill=penColor!70,domain=-0:45 ,samples=180,smooth] (x, {2*cos(2*x)}) ;
%      \addplot[data cs=polar, very thick, mark=none, fill=white,  domain=0:360, samples=180, smooth] (x, {1});
            \addplot [data cs=polar, very thick, mark=none,fill=fill1,domain=30:45 ,samples=180,smooth] (x, {2*cos(2*x)})-- (axis cs:0,0)[penColor4!50] ;
%            \addplot [data cs=polar, very thick, mark=none,fill=fill1,domain=-45:-30 ,samples=180,smooth] (x, {2*cos(2*x)})-- (axis cs:0,0)[penColor4!50] ;
      \addplot[data cs=polar, very thick, mark=none, fill=fill1,  domain=0:30, samples=360, smooth] (x, {1}) -- (axis cs:0,0)[penColor4!50];
      \addplot[data cs=polar,penColor,domain=0:45,samples=360,smooth, ultra thick] (x,{2*cos(2*x)});
   	   \addplot[data cs=polar,penColor2,domain=0:360,samples=360,smooth, ultra thick] (x,{1});
	   
\addplot [draw=penColor5,domain=0:1.1,very thick,smooth] {tan(30)*x};
\addplot [draw=penColor5,domain=0:3,very thick,smooth] {0};
\addplot [draw=penColor5,domain=0:1,very thick,smooth] {tan(45)*x};	   
\node[penColor] at (axis cs: 1.5,2) {$r = 2\cos(2\theta)$};
\node[penColor2] at (axis cs: -1.5,1) {$r = 1$};
\node[right, penColor5] at (axis cs: 1.1,.65) {$\theta=\frac{\pi}{6}$};
\node[right, penColor5] at (axis cs: 1,1.05) {$\theta=\frac{\pi}{4}$};
\node[penColor5] at (axis cs: 1.2,.1) {$\theta=0$};
      
            \end{axis}  
  \end{tikzpicture}  
  \end{image}

For this region, we need to determine the $\theta$-value at which the curve $r=2\cos (2\theta)$ first passes through the origin. This is the smallest positive value of $\theta$ such that $0 = 2 \cos (2\theta)$, and therefore satisfies $2 \theta = \pi/2$ or $\theta = \pi/4$.
 
The outer curve for $0 \leq \theta \leq \pi/6$  is $r=1$, while the outer curve from $\pi/6 < \theta \leq \pi/4$ is $r=2\cos(2\theta)$. Since the outer curve changes, we will need two integrals to compute the area.

\[
A = 8 \cdot \left[ \int_{0}^{\pi/6} \frac{1}{2} (1)^2  \d \theta +  \int_{\pi/6}^{\pi/4} \frac{1}{2} (2\cos(2\theta))^2  \d \theta\right] = \frac{\pi}{2}.
\]

Carrying out the computation gives $A = \frac{2\pi}{3}-\sqrt{3}$.
\end{freeResponse}
\end{problem}

%%%%%%%%%%%%%%%%%%%%%%%%%%%%%%%%%%%%%%%%%%%%%
\begin{problem}
For each region given below, calculate its area using both Cartesian and polar coordinates.

\begin{itemize}
\item[I.] The rectangle bounded by $y=0$, $y=\sqrt{3}$, $x=0$ and $x=1$.

\item[II.] The region in the first quadrant bounded by the line $y=x$, the unit circle and the positive $y$-axis.
\end{itemize}

\begin{freeResponse}
I. From basic geometry, the area is clearly $\sqrt{3}$. To verify this with an integral in Cartesian coordinates, the area is 
$$
\int_0^1 \sqrt{3} \d x = \sqrt{3}.
$$
Finally, to  express the area integral in polar coordinates, we note that the vertical line $x = 1$ has polar equation $r = \sec \theta$. The area is therefore
$$
2 \int_0^{\pi/3} \frac{1}{2} \sec^2 \theta \d \theta = \eval{\tan \theta}_0^{\pi/3} = \sqrt{3}.
$$ 

II. From basic geometry, the area is $\frac{1}{8} \pi$. Using polar coordinates, the area is given by
$$
\int_{\pi/4}^{\pi/2} \frac{1}{2} \d \theta = \frac{1}{2}\left(\frac{\pi}{2} - \frac{\pi}{4}\right) = \frac{\pi}{8}.
$$
Using Cartesian coordinates, the area is given by
$$
\int_0^{\sqrt{2}/2} \sqrt{1- x^2}-x \d x.
$$
A tedious calculation involving trigonometric substitution shows that this integral also evaluates to $\pi/8$. 
\end{freeResponse}
\end{problem}

\end{document}
