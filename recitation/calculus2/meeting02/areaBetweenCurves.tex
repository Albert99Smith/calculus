\documentclass[handout]{ximera}
%handout:  for handout version with no solutions or instructor notes
%handout,instructornotes:  for instructor version with just problems and notes, no solutions
%noinstructornotes:  shows only problem and solutions

%% handout
%% space
%% newpage
%% numbers
%% nooutcomes

%I added the commands here so that I would't have to keep looking them up
%\newcommand{\RR}{\mathbb R}
%\renewcommand{\d}{\,d}
%\newcommand{\dd}[2][]{\frac{d #1}{d #2}}
%\renewcommand{\l}{\ell}
%\newcommand{\ddx}{\frac{d}{dx}}
%\everymath{\displaystyle}
%\newcommand{\dfn}{\textbf}
%\newcommand{\eval}[1]{\bigg[ #1 \bigg]}

%\begin{image}
%\includegraphics[trim= 170 420 250 180]{Figure1.pdf}
%\end{image}

%add a ``.'' below when used in a specific directory.

%\usepackage{todonotes}
%\usepackage{mathtools} %% Required for wide table Curl and Greens
%\usepackage{cuted} %% Required for wide table Curl and Greens
\newcommand{\todo}{}

\usepackage{esint} % for \oiint
\ifxake%%https://math.meta.stackexchange.com/questions/9973/how-do-you-render-a-closed-surface-double-integral
\renewcommand{\oiint}{{\large\bigcirc}\kern-1.56em\iint}
\fi


\graphicspath{
  {./}
  {ximeraTutorial/}
  {basicPhilosophy/}
  {functionsOfSeveralVariables/}
  {normalVectors/}
  {lagrangeMultipliers/}
  {vectorFields/}
  {greensTheorem/}
  {shapeOfThingsToCome/}
  {dotProducts/}
  {partialDerivativesAndTheGradientVector/}
  {../productAndQuotientRules/exercises/}
  {../normalVectors/exercisesParametricPlots/}
  {../continuityOfFunctionsOfSeveralVariables/exercises/}
  {../partialDerivativesAndTheGradientVector/exercises/}
  {../directionalDerivativeAndChainRule/exercises/}
  {../commonCoordinates/exercisesCylindricalCoordinates/}
  {../commonCoordinates/exercisesSphericalCoordinates/}
  {../greensTheorem/exercisesCurlAndLineIntegrals/}
  {../greensTheorem/exercisesDivergenceAndLineIntegrals/}
  {../shapeOfThingsToCome/exercisesDivergenceTheorem/}
  {../greensTheorem/}
  {../shapeOfThingsToCome/}
  {../separableDifferentialEquations/exercises/}
  {vectorFields/}
}

\newcommand{\mooculus}{\textsf{\textbf{MOOC}\textnormal{\textsf{ULUS}}}}

\usepackage{tkz-euclide}\usepackage{tikz}
\usepackage{tikz-cd}
\usetikzlibrary{arrows}
\tikzset{>=stealth,commutative diagrams/.cd,
  arrow style=tikz,diagrams={>=stealth}} %% cool arrow head
\tikzset{shorten <>/.style={ shorten >=#1, shorten <=#1 } } %% allows shorter vectors

\usetikzlibrary{backgrounds} %% for boxes around graphs
\usetikzlibrary{shapes,positioning}  %% Clouds and stars
\usetikzlibrary{matrix} %% for matrix
\usepgfplotslibrary{polar} %% for polar plots
\usepgfplotslibrary{fillbetween} %% to shade area between curves in TikZ
\usetkzobj{all}
\usepackage[makeroom]{cancel} %% for strike outs
%\usepackage{mathtools} %% for pretty underbrace % Breaks Ximera
%\usepackage{multicol}
\usepackage{pgffor} %% required for integral for loops



%% http://tex.stackexchange.com/questions/66490/drawing-a-tikz-arc-specifying-the-center
%% Draws beach ball
\tikzset{pics/carc/.style args={#1:#2:#3}{code={\draw[pic actions] (#1:#3) arc(#1:#2:#3);}}}



\usepackage{array}
\setlength{\extrarowheight}{+.1cm}
\newdimen\digitwidth
\settowidth\digitwidth{9}
\def\divrule#1#2{
\noalign{\moveright#1\digitwidth
\vbox{\hrule width#2\digitwidth}}}





\newcommand{\RR}{\mathbb R}
\newcommand{\R}{\mathbb R}
\newcommand{\N}{\mathbb N}
\newcommand{\Z}{\mathbb Z}

\newcommand{\sagemath}{\textsf{SageMath}}


%\renewcommand{\d}{\,d\!}
\renewcommand{\d}{\mathop{}\!d}
\newcommand{\dd}[2][]{\frac{\d #1}{\d #2}}
\newcommand{\pp}[2][]{\frac{\partial #1}{\partial #2}}
\renewcommand{\l}{\ell}
\newcommand{\ddx}{\frac{d}{\d x}}

\newcommand{\zeroOverZero}{\ensuremath{\boldsymbol{\tfrac{0}{0}}}}
\newcommand{\inftyOverInfty}{\ensuremath{\boldsymbol{\tfrac{\infty}{\infty}}}}
\newcommand{\zeroOverInfty}{\ensuremath{\boldsymbol{\tfrac{0}{\infty}}}}
\newcommand{\zeroTimesInfty}{\ensuremath{\small\boldsymbol{0\cdot \infty}}}
\newcommand{\inftyMinusInfty}{\ensuremath{\small\boldsymbol{\infty - \infty}}}
\newcommand{\oneToInfty}{\ensuremath{\boldsymbol{1^\infty}}}
\newcommand{\zeroToZero}{\ensuremath{\boldsymbol{0^0}}}
\newcommand{\inftyToZero}{\ensuremath{\boldsymbol{\infty^0}}}



\newcommand{\numOverZero}{\ensuremath{\boldsymbol{\tfrac{\#}{0}}}}
\newcommand{\dfn}{\textbf}
%\newcommand{\unit}{\,\mathrm}
\newcommand{\unit}{\mathop{}\!\mathrm}
\newcommand{\eval}[1]{\bigg[ #1 \bigg]}
\newcommand{\seq}[1]{\left( #1 \right)}
\renewcommand{\epsilon}{\varepsilon}
\renewcommand{\phi}{\varphi}


\renewcommand{\iff}{\Leftrightarrow}

\DeclareMathOperator{\arccot}{arccot}
\DeclareMathOperator{\arcsec}{arcsec}
\DeclareMathOperator{\arccsc}{arccsc}
\DeclareMathOperator{\si}{Si}
\DeclareMathOperator{\scal}{scal}
\DeclareMathOperator{\sign}{sign}


%% \newcommand{\tightoverset}[2]{% for arrow vec
%%   \mathop{#2}\limits^{\vbox to -.5ex{\kern-0.75ex\hbox{$#1$}\vss}}}
\newcommand{\arrowvec}[1]{{\overset{\rightharpoonup}{#1}}}
%\renewcommand{\vec}[1]{\arrowvec{\mathbf{#1}}}
\renewcommand{\vec}[1]{{\overset{\boldsymbol{\rightharpoonup}}{\mathbf{#1}}}\hspace{0in}}

\newcommand{\point}[1]{\left(#1\right)} %this allows \vector{ to be changed to \vector{ with a quick find and replace
\newcommand{\pt}[1]{\mathbf{#1}} %this allows \vec{ to be changed to \vec{ with a quick find and replace
\newcommand{\Lim}[2]{\lim_{\point{#1} \to \point{#2}}} %Bart, I changed this to point since I want to use it.  It runs through both of the exercise and exerciseE files in limits section, which is why it was in each document to start with.

\DeclareMathOperator{\proj}{\mathbf{proj}}
\newcommand{\veci}{{\boldsymbol{\hat{\imath}}}}
\newcommand{\vecj}{{\boldsymbol{\hat{\jmath}}}}
\newcommand{\veck}{{\boldsymbol{\hat{k}}}}
\newcommand{\vecl}{\vec{\boldsymbol{\l}}}
\newcommand{\uvec}[1]{\mathbf{\hat{#1}}}
\newcommand{\utan}{\mathbf{\hat{t}}}
\newcommand{\unormal}{\mathbf{\hat{n}}}
\newcommand{\ubinormal}{\mathbf{\hat{b}}}

\newcommand{\dotp}{\bullet}
\newcommand{\cross}{\boldsymbol\times}
\newcommand{\grad}{\boldsymbol\nabla}
\newcommand{\divergence}{\grad\dotp}
\newcommand{\curl}{\grad\cross}
%\DeclareMathOperator{\divergence}{divergence}
%\DeclareMathOperator{\curl}[1]{\grad\cross #1}
\newcommand{\lto}{\mathop{\longrightarrow\,}\limits}

\renewcommand{\bar}{\overline}

\colorlet{textColor}{black}
\colorlet{background}{white}
\colorlet{penColor}{blue!50!black} % Color of a curve in a plot
\colorlet{penColor2}{red!50!black}% Color of a curve in a plot
\colorlet{penColor3}{red!50!blue} % Color of a curve in a plot
\colorlet{penColor4}{green!50!black} % Color of a curve in a plot
\colorlet{penColor5}{orange!80!black} % Color of a curve in a plot
\colorlet{penColor6}{yellow!70!black} % Color of a curve in a plot
\colorlet{fill1}{penColor!20} % Color of fill in a plot
\colorlet{fill2}{penColor2!20} % Color of fill in a plot
\colorlet{fillp}{fill1} % Color of positive area
\colorlet{filln}{penColor2!20} % Color of negative area
\colorlet{fill3}{penColor3!20} % Fill
\colorlet{fill4}{penColor4!20} % Fill
\colorlet{fill5}{penColor5!20} % Fill
\colorlet{gridColor}{gray!50} % Color of grid in a plot

\newcommand{\surfaceColor}{violet}
\newcommand{\surfaceColorTwo}{redyellow}
\newcommand{\sliceColor}{greenyellow}




\pgfmathdeclarefunction{gauss}{2}{% gives gaussian
  \pgfmathparse{1/(#2*sqrt(2*pi))*exp(-((x-#1)^2)/(2*#2^2))}%
}


%%%%%%%%%%%%%
%% Vectors
%%%%%%%%%%%%%

%% Simple horiz vectors
\renewcommand{\vector}[1]{\left\langle #1\right\rangle}


%% %% Complex Horiz Vectors with angle brackets
%% \makeatletter
%% \renewcommand{\vector}[2][ , ]{\left\langle%
%%   \def\nextitem{\def\nextitem{#1}}%
%%   \@for \el:=#2\do{\nextitem\el}\right\rangle%
%% }
%% \makeatother

%% %% Vertical Vectors
%% \def\vector#1{\begin{bmatrix}\vecListA#1,,\end{bmatrix}}
%% \def\vecListA#1,{\if,#1,\else #1\cr \expandafter \vecListA \fi}

%%%%%%%%%%%%%
%% End of vectors
%%%%%%%%%%%%%

%\newcommand{\fullwidth}{}
%\newcommand{\normalwidth}{}



%% makes a snazzy t-chart for evaluating functions
%\newenvironment{tchart}{\rowcolors{2}{}{background!90!textColor}\array}{\endarray}

%%This is to help with formatting on future title pages.
\newenvironment{sectionOutcomes}{}{}



%% Flowchart stuff
%\tikzstyle{startstop} = [rectangle, rounded corners, minimum width=3cm, minimum height=1cm,text centered, draw=black]
%\tikzstyle{question} = [rectangle, minimum width=3cm, minimum height=1cm, text centered, draw=black]
%\tikzstyle{decision} = [trapezium, trapezium left angle=70, trapezium right angle=110, minimum width=3cm, minimum height=1cm, text centered, draw=black]
%\tikzstyle{question} = [rectangle, rounded corners, minimum width=3cm, minimum height=1cm,text centered, draw=black]
%\tikzstyle{process} = [rectangle, minimum width=3cm, minimum height=1cm, text centered, draw=black]
%\tikzstyle{decision} = [trapezium, trapezium left angle=70, trapezium right angle=110, minimum width=3cm, minimum height=1cm, text centered, draw=black]




\author{Tom Needham and Jim Talamo}

\outcome{Use definite integrals to compute areas of regions bounded between curves.}
\outcome{Determine if the area is better computed using an integral with respect to $x$ or $y$.}
\outcome{Set up integrals with respect to both $x$ and $y$ that give the area of a region.}

\title[]{Regions Between Curves}

\begin{document}
\begin{abstract}
\end{abstract}
\maketitle

\vspace{-0.9in}

\section{Discussion Questions}

\begin{problem} Determine the minimum number of integrals with respect to $x$ and the minimum number of integrals with respect to $y$ are needed to set up the area of the region below:
\begin{center}
\resizebox {6cm} {!} {
\begin{tikzpicture}
		\begin{axis}[
			domain=-2:4, ymax=4.5,xmax=3, ymin=-1, xmin=-0.5,
			axis lines =center, xlabel=$x$, ylabel=$y$,
            		every axis y label/.style={at=(current axis.above origin),anchor=south},
            		every axis x label/.style={at=(current axis.right of origin),anchor=west},
            		axis on top,
            		]
                      
            	\addplot [draw=penColor,very thick,smooth] {6*x};
            	\addplot [draw=penColor,very thick,smooth] {x};
		\addplot [domain=0.3:2,draw=penColor2,very thick,smooth] {1/x};
		\addplot [domain=0.3:2,draw=penColor2,very thick,smooth] {2/x};
                       
            	\addplot [name path=A,domain=6^(-1/2):3^(-1/2),draw=none] {6*x};   
            	\addplot [name path=B,domain=6^(-1/2):3^(-1/2),draw=none] {1/x};
		\addplot [name path=C,domain=3^(-1/1.9):1,draw=none] {1/x};
		\addplot [name path=D,domain=3^(-1/1.9):1,draw=none] {2/x};
		\addplot [name path=E,domain=0.99:2^(1/2),draw=none] {x};
		\addplot [name path=F,domain=0.99:2^(1/2),draw=none] {2/x};
            	\addplot [fillp] fill between[of=A and B];
		\addplot [fillp] fill between[of=C and D];
		\addplot [fillp] fill between[of=E and F];
		
		\node at (axis cs:1,3.75) [penColor] {$y=6x$};
            	\node at (axis cs:2.5,3) [penColor] {$y=x$};
		\node at (axis cs:2.3,0.5) [penColor2] {$y=\frac{1}{x}$};
            	\node at (axis cs:2,1.45) [penColor2] {$y=\frac{2}{x}$};
                      
            	\end{axis}
	\end{tikzpicture}}
	\end{center}

%%%%%%%%%%WHAT DOES THIS DO?%%%%%%%%%%%%%%%%%
\iffalse
\begin{image}
\begin{tikzpicture}
	\begin{axis}[
            domain=0:1.5, ymax=6,xmax=1.5,ymin=0, xmin=0,
            axis lines =left, xlabel=$x$, ylabel=$y$,
            xtick={0.5,1,2},
            %width=4in,
            %height=2in,
            %yticklabels={},
            %xticklabels={$1$, $2$},
            every axis y label/.style={at=(current axis.above origin),anchor=south},
            every axis x label/.style={at=(current axis.right of origin),anchor=west},
            axis on top,
          ]
          \addplot [draw=none,fill=fillp,domain=0.5:1] {-3*x^2+4*x+3} \closedcycle;
          \addplot [draw=none,fill=background,domain=0.5:1] {3*x^2-3*x +2} \closedcycle;
          \addplot [draw=penColor,very thick] {-3*x^2+4*x+3};
          \addplot [draw=penColor2,very thick] {3*x^2-3*x +2};
          \node at (axis cs:0.75,4.75) [penColor] {$y= -3x^2+4x+3$};
          \node at (axis cs:0.75,0.75) [penColor2] {$y=3x^2-3x+2$};
        \end{axis}
\end{tikzpicture}
\end{image} 
\fi
%%%%%%%%%%%%%%%%%%%%%%%%%%%%%%%%
\end{problem}

\begin{freeResponse}

\end{freeResponse}


%%%%%%%%%%%%%%%%%%%%%%%%%%%%%%%%

\begin{problem}

The area between curves $y=f(x)$ and $y=g(x)$ can frequently be computed either as an integral with respect to $x$ or as an integral with respect to $y$. For each of the following examples, explain which strategy you would use to calculate the area of the shaded region.

\begin{tabular}{ll}
\resizebox {6cm} {!} { 

\begin{tikzpicture}
	\begin{axis}[
            domain=0:1.5, ymax=1.5,xmax=1.5, ymin=0, xmin=0,
            axis lines =center, xlabel=$x$, ylabel=$y$,
            every axis y label/.style={at=(current axis.above origin),anchor=south},
            every axis x label/.style={at=(current axis.right of origin),anchor=west},
            axis on top,
          ]
          \addplot [ fill = fillp, smooth, samples=100, domain=(0:1.5)] ({1-x^2},{x}) \closedcycle;
          \addplot [draw=none,fill=background,domain=0:1.5] {x} \closedcycle;   
          \addplot [very thick, penColor2, smooth, samples=100, domain=(0:1.5)] ({1-x^2},{x});
          \addplot [draw=penColor,very thick,smooth] {x};
          
          \node at (axis cs:1.2,.25) [penColor2] {$y=\sqrt{1-x}$};
          \node at (axis cs:1.2,.95) [penColor] {$y=x$};
        \end{axis}
\end{tikzpicture}
}
  &
\resizebox {6cm} {!} { 
         \begin{tikzpicture}
	\begin{axis}[
            domain=0:5.5, ymax=2.8,xmax=5.5, ymin=0, xmin=0,
            axis lines =center, xlabel=$x$, ylabel=$y$,
            every axis y label/.style={at=(current axis.above origin),anchor=south},
            every axis x label/.style={at=(current axis.right of origin),anchor=west},
            axis on top,
          ]
          \addplot [ fill = fillp, smooth, samples=100, domain=(0:2)] ({1+x^2},{x}) \closedcycle;
          \addplot [draw=none,fill=background,domain=0:5.2] {x-3} \closedcycle;   
          \addplot [very thick, penColor2, smooth, samples=100, domain=(0:3)] ({1+x^2},{x});
          \addplot [draw=penColor,very thick,smooth] {x-3};
          
          \node at (axis cs:2,1.5) [penColor2] {$y=\sqrt{x-1}$};
          \node at (axis cs:4.5,0.7) [penColor] {$y=x-3$};
        \end{axis}
\end{tikzpicture}
} 
\end{tabular}
\end{problem}

%%%%%%%%%%%%%%%%%%%%%%%%%%%%%%





\begin{freeResponse} 
\end{freeResponse}


\begin{problem}
A student considers the region bounded by the curves $y=\sin (x)$, $y = 0$, $x=0$ and $x=2\pi$. After graphing the area in question, the student concludes that the area is zero. Explain why the student's answer is correct or explain a likely error in the student's reasoning and give the correct answer.
\begin{center}
\resizebox {4cm} {!} { 
          \begin{tikzpicture}
          
	    \begin{axis}[
            domain=-0.5:6.5,
            xmin=-0.5, xmax=6.5,
            ymin=-1.25, ymax=1.25
         ,
            axis lines =middle, xlabel=$x$, ylabel=$y$, yticklabels={,,}, xticklabels={,,},
            every axis y label/.style={at=(current axis.above origin),anchor=south},
            every axis x label/.style={at=(current axis.right of origin),anchor=west},
          ]
	  \addplot [draw=none,fill=fillp,domain=0:6.28, smooth] {sin(deg(x))} \closedcycle;
	  \addplot [very thick, penColor, smooth] {sin(deg(x))};
	  
        
        \end{axis}
\end{tikzpicture}}
\end{center}

\end{problem}

\begin{freeResponse}

\end{freeResponse}



\section{Group Work}

\begin{problem}
Let $R$ denote the region bounded by the curves $y=2-2x$, $y=-1$ and $x=0$. Set up an integral (either with respect to $x$ or with respect to $y$) which can be used to calculate the area of $R$. Either evaluate your integral or determine another way to calculate the area of $R$. 
\end{problem}

\begin{freeResponse}

\end{freeResponse}

\begin{problem}
(Calculating areas)
\begin{enumerate}
%\item[I.] Find the area of the region bounded between the curves $y=2-x^2$ and $y=x-1$.

\item[I.] Find the area of the region bounded by the curves $y=e^{x}$, $y=e^{-x}+2$ and $x=0$. 

\item[II.] Find the area of the region bounded by the curves $x=y^2$ and $x=8-y^2$. The following figure may be useful.

\begin{center}
\resizebox {6cm} {!} {
\begin{tikzpicture}
\begin{axis}[
            domain=-3:3, ymax=4.5,xmax=10.5, ymin=-3, xmin=-1,
            axis lines =center, xlabel=$x$, ylabel=$y$,
            every axis y label/.style={at=(current axis.above origin),anchor=south},
            every axis x label/.style={at=(current axis.right of origin),anchor=west},
            axis on top,
          ]
          
\addplot[draw=penColor,very thick,smooth] (8-x^2,x);
\addplot[draw=penColor2,very thick,smooth] (x^2,x);
           
\addplot [name path=A,domain=4:8,samples=50,draw=none] {sqrt(8-x)}; 
\addplot [name path=B,domain=4:8,samples=50,draw=none] {-sqrt(8-x)};  
\addplot [name path=C,domain=0:4.01,samples=50,draw=none] {sqrt(x)}; 
\addplot [name path=D,domain=0:4.01,samples=50,draw=none] {-sqrt(x)};  
\addplot [fillp] fill between[of=A and B];
\addplot [fillp] fill between[of=C and D];



\node at (axis cs:3,-2.75) [penColor] {$x=8-y^2$};
\node at (axis cs:7,2.25) [penColor2] {$x=y^2$};
\end{axis}
\end{tikzpicture}}
\end{center}
%
%\item[IV.] Find the area of the region bounded by the curves $y=\cos(x)$, $y=\cos (2x)$, $x=0$ and $x=\pi$. The following figure may be useful.

%\begin{center}
%\resizebox {6cm} {!} {
%\begin{tikzpicture}
%		\begin{axis}[
%			domain=-1:3.5, ymax=1.5,xmax=3.5, ymin=-1.5, xmin=-1,
%			axis lines =center, xlabel=$x$, ylabel=$y$,
%            		every axis y label/.style={at=(current axis.above origin),anchor=south},
%            		every axis x label/.style={at=(current axis.right of origin),anchor=west},
%            		axis on top,
%            		]
%                      
%            	\addplot [draw=penColor,very thick,smooth] {cos(deg(x))};
%            	\addplot [draw=penColor2,very thick,smooth] {cos(2*deg(x))};
%	
%		\addplot [name path=A,domain=0:3.14,draw=none] {cos(deg(x))};   
%            	\addplot [name path=B,domain=0:3.14,draw=none] {cos(2*deg(x))};
%		\addplot [fillp] fill between[of=A and B];
%		
%		\node at (axis cs:1.6,0.8) [penColor] {$y=\cos(x)$};
%            	\node at (axis cs:1.6,-1.25) [penColor2] {$y=\cos(2x)$};
%                       
%            	\end{axis}
%	\end{tikzpicture}}
%	\end{center}
\end{enumerate}
\end{problem}

\begin{freeResponse}
\end{freeResponse}

\begin{problem}
The region $R$ bounded by $x^2+y^2=1$ and $y=x$ is shown below. 

\begin{center}
\resizebox {6cm} {!} {
            \begin{tikzpicture}
            	\begin{axis}[
            		domain=-1.2:1.2, ymax=1.2,xmax=1.2, ymin=-1.1, xmin=-1.2,
            		axis lines =center, xlabel=$x$, ylabel=$y$,
            		every axis y label/.style={at=(current axis.above origin),anchor=south},
            		every axis x label/.style={at=(current axis.right of origin),anchor=west},
            		axis on top,
            		]
                      
            	\addplot [draw=penColor,domain=-1:1/sqrt(2),very thick,smooth,samples=200] {sqrt(1-x^2)};
	        \addplot [draw=penColor,domain=-1:-1/sqrt(2),very thick,smooth,samples=200] {-sqrt(1-x^2)};
            	\addplot [draw=penColor2,very thick,smooth] {x};
	                            
            	\addplot [name path=A,domain=-1:1/sqrt(2),draw=none,samples=200] {sqrt(1-x^2)};   
            	\addplot [name path=B,domain=-1:-1/sqrt(2),draw=none,samples=200] {-sqrt(1-x^2)};
	        \addplot [name path=C,domain=-2:1,draw=none] {x};
            	\addplot [fillp] fill between[of=A and B];
                                   
            	\node at (axis cs:-.8,1) [penColor] {$x^2+y^2=1$};
            	\node at (axis cs:.7,.3) [penColor2] {$y=x$};
	    
	      \end{axis}
            \end{tikzpicture}}
            \end{center}
            
\begin{enumerate}
\item[I.] Set up, but do not evaluate, an integral or sum of integrals with respect to $x$ that expresses the area of $R$.
\item[II.] Set up, but do not evaluate, an integral or sum of integrals with respect to $y$ that expresses the area of $R$.
\end{enumerate}
\end{problem}

\begin{freeResponse}

\end{freeResponse}

\end{document}
