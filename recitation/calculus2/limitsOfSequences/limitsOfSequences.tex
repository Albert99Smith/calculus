\documentclass[]{ximera}
%handout:  for handout version with no solutions or instructor notes
%handout,instructornotes:  for instructor version with just problems and notes, no solutions
%noinstructornotes:  shows only problem and solutions

%% handout
%% space
%% newpage
%% numbers
%% nooutcomes

%I added the commands here so that I would't have to keep looking them up
%\newcommand{\RR}{\mathbb R}
%\renewcommand{\d}{\,d}
%\newcommand{\dd}[2][]{\frac{d #1}{d #2}}
%\renewcommand{\l}{\ell}
%\newcommand{\ddx}{\frac{d}{dx}}
%\everymath{\displaystyle}
%\newcommand{\dfn}{\textbf}
%\newcommand{\eval}[1]{\bigg[ #1 \bigg]}

%\begin{image}
%\includegraphics[trim= 170 420 250 180]{Figure1.pdf}
%\end{image}

%add a ``.'' below when used in a specific directory.


%\usepackage{todonotes}
%\usepackage{mathtools} %% Required for wide table Curl and Greens
%\usepackage{cuted} %% Required for wide table Curl and Greens
\newcommand{\todo}{}

\usepackage{esint} % for \oiint
\ifxake%%https://math.meta.stackexchange.com/questions/9973/how-do-you-render-a-closed-surface-double-integral
\renewcommand{\oiint}{{\large\bigcirc}\kern-1.56em\iint}
\fi


\graphicspath{
  {./}
  {ximeraTutorial/}
  {basicPhilosophy/}
  {functionsOfSeveralVariables/}
  {normalVectors/}
  {lagrangeMultipliers/}
  {vectorFields/}
  {greensTheorem/}
  {shapeOfThingsToCome/}
  {dotProducts/}
  {partialDerivativesAndTheGradientVector/}
  {../productAndQuotientRules/exercises/}
  {../normalVectors/exercisesParametricPlots/}
  {../continuityOfFunctionsOfSeveralVariables/exercises/}
  {../partialDerivativesAndTheGradientVector/exercises/}
  {../directionalDerivativeAndChainRule/exercises/}
  {../commonCoordinates/exercisesCylindricalCoordinates/}
  {../commonCoordinates/exercisesSphericalCoordinates/}
  {../greensTheorem/exercisesCurlAndLineIntegrals/}
  {../greensTheorem/exercisesDivergenceAndLineIntegrals/}
  {../shapeOfThingsToCome/exercisesDivergenceTheorem/}
  {../greensTheorem/}
  {../shapeOfThingsToCome/}
  {../separableDifferentialEquations/exercises/}
  {vectorFields/}
}

\newcommand{\mooculus}{\textsf{\textbf{MOOC}\textnormal{\textsf{ULUS}}}}

\usepackage{tkz-euclide}\usepackage{tikz}
\usepackage{tikz-cd}
\usetikzlibrary{arrows}
\tikzset{>=stealth,commutative diagrams/.cd,
  arrow style=tikz,diagrams={>=stealth}} %% cool arrow head
\tikzset{shorten <>/.style={ shorten >=#1, shorten <=#1 } } %% allows shorter vectors

\usetikzlibrary{backgrounds} %% for boxes around graphs
\usetikzlibrary{shapes,positioning}  %% Clouds and stars
\usetikzlibrary{matrix} %% for matrix
\usepgfplotslibrary{polar} %% for polar plots
\usepgfplotslibrary{fillbetween} %% to shade area between curves in TikZ
\usetkzobj{all}
\usepackage[makeroom]{cancel} %% for strike outs
%\usepackage{mathtools} %% for pretty underbrace % Breaks Ximera
%\usepackage{multicol}
\usepackage{pgffor} %% required for integral for loops



%% http://tex.stackexchange.com/questions/66490/drawing-a-tikz-arc-specifying-the-center
%% Draws beach ball
\tikzset{pics/carc/.style args={#1:#2:#3}{code={\draw[pic actions] (#1:#3) arc(#1:#2:#3);}}}



\usepackage{array}
\setlength{\extrarowheight}{+.1cm}
\newdimen\digitwidth
\settowidth\digitwidth{9}
\def\divrule#1#2{
\noalign{\moveright#1\digitwidth
\vbox{\hrule width#2\digitwidth}}}





\newcommand{\RR}{\mathbb R}
\newcommand{\R}{\mathbb R}
\newcommand{\N}{\mathbb N}
\newcommand{\Z}{\mathbb Z}

\newcommand{\sagemath}{\textsf{SageMath}}


%\renewcommand{\d}{\,d\!}
\renewcommand{\d}{\mathop{}\!d}
\newcommand{\dd}[2][]{\frac{\d #1}{\d #2}}
\newcommand{\pp}[2][]{\frac{\partial #1}{\partial #2}}
\renewcommand{\l}{\ell}
\newcommand{\ddx}{\frac{d}{\d x}}

\newcommand{\zeroOverZero}{\ensuremath{\boldsymbol{\tfrac{0}{0}}}}
\newcommand{\inftyOverInfty}{\ensuremath{\boldsymbol{\tfrac{\infty}{\infty}}}}
\newcommand{\zeroOverInfty}{\ensuremath{\boldsymbol{\tfrac{0}{\infty}}}}
\newcommand{\zeroTimesInfty}{\ensuremath{\small\boldsymbol{0\cdot \infty}}}
\newcommand{\inftyMinusInfty}{\ensuremath{\small\boldsymbol{\infty - \infty}}}
\newcommand{\oneToInfty}{\ensuremath{\boldsymbol{1^\infty}}}
\newcommand{\zeroToZero}{\ensuremath{\boldsymbol{0^0}}}
\newcommand{\inftyToZero}{\ensuremath{\boldsymbol{\infty^0}}}



\newcommand{\numOverZero}{\ensuremath{\boldsymbol{\tfrac{\#}{0}}}}
\newcommand{\dfn}{\textbf}
%\newcommand{\unit}{\,\mathrm}
\newcommand{\unit}{\mathop{}\!\mathrm}
\newcommand{\eval}[1]{\bigg[ #1 \bigg]}
\newcommand{\seq}[1]{\left( #1 \right)}
\renewcommand{\epsilon}{\varepsilon}
\renewcommand{\phi}{\varphi}


\renewcommand{\iff}{\Leftrightarrow}

\DeclareMathOperator{\arccot}{arccot}
\DeclareMathOperator{\arcsec}{arcsec}
\DeclareMathOperator{\arccsc}{arccsc}
\DeclareMathOperator{\si}{Si}
\DeclareMathOperator{\scal}{scal}
\DeclareMathOperator{\sign}{sign}


%% \newcommand{\tightoverset}[2]{% for arrow vec
%%   \mathop{#2}\limits^{\vbox to -.5ex{\kern-0.75ex\hbox{$#1$}\vss}}}
\newcommand{\arrowvec}[1]{{\overset{\rightharpoonup}{#1}}}
%\renewcommand{\vec}[1]{\arrowvec{\mathbf{#1}}}
\renewcommand{\vec}[1]{{\overset{\boldsymbol{\rightharpoonup}}{\mathbf{#1}}}\hspace{0in}}

\newcommand{\point}[1]{\left(#1\right)} %this allows \vector{ to be changed to \vector{ with a quick find and replace
\newcommand{\pt}[1]{\mathbf{#1}} %this allows \vec{ to be changed to \vec{ with a quick find and replace
\newcommand{\Lim}[2]{\lim_{\point{#1} \to \point{#2}}} %Bart, I changed this to point since I want to use it.  It runs through both of the exercise and exerciseE files in limits section, which is why it was in each document to start with.

\DeclareMathOperator{\proj}{\mathbf{proj}}
\newcommand{\veci}{{\boldsymbol{\hat{\imath}}}}
\newcommand{\vecj}{{\boldsymbol{\hat{\jmath}}}}
\newcommand{\veck}{{\boldsymbol{\hat{k}}}}
\newcommand{\vecl}{\vec{\boldsymbol{\l}}}
\newcommand{\uvec}[1]{\mathbf{\hat{#1}}}
\newcommand{\utan}{\mathbf{\hat{t}}}
\newcommand{\unormal}{\mathbf{\hat{n}}}
\newcommand{\ubinormal}{\mathbf{\hat{b}}}

\newcommand{\dotp}{\bullet}
\newcommand{\cross}{\boldsymbol\times}
\newcommand{\grad}{\boldsymbol\nabla}
\newcommand{\divergence}{\grad\dotp}
\newcommand{\curl}{\grad\cross}
%\DeclareMathOperator{\divergence}{divergence}
%\DeclareMathOperator{\curl}[1]{\grad\cross #1}
\newcommand{\lto}{\mathop{\longrightarrow\,}\limits}

\renewcommand{\bar}{\overline}

\colorlet{textColor}{black}
\colorlet{background}{white}
\colorlet{penColor}{blue!50!black} % Color of a curve in a plot
\colorlet{penColor2}{red!50!black}% Color of a curve in a plot
\colorlet{penColor3}{red!50!blue} % Color of a curve in a plot
\colorlet{penColor4}{green!50!black} % Color of a curve in a plot
\colorlet{penColor5}{orange!80!black} % Color of a curve in a plot
\colorlet{penColor6}{yellow!70!black} % Color of a curve in a plot
\colorlet{fill1}{penColor!20} % Color of fill in a plot
\colorlet{fill2}{penColor2!20} % Color of fill in a plot
\colorlet{fillp}{fill1} % Color of positive area
\colorlet{filln}{penColor2!20} % Color of negative area
\colorlet{fill3}{penColor3!20} % Fill
\colorlet{fill4}{penColor4!20} % Fill
\colorlet{fill5}{penColor5!20} % Fill
\colorlet{gridColor}{gray!50} % Color of grid in a plot

\newcommand{\surfaceColor}{violet}
\newcommand{\surfaceColorTwo}{redyellow}
\newcommand{\sliceColor}{greenyellow}




\pgfmathdeclarefunction{gauss}{2}{% gives gaussian
  \pgfmathparse{1/(#2*sqrt(2*pi))*exp(-((x-#1)^2)/(2*#2^2))}%
}


%%%%%%%%%%%%%
%% Vectors
%%%%%%%%%%%%%

%% Simple horiz vectors
\renewcommand{\vector}[1]{\left\langle #1\right\rangle}


%% %% Complex Horiz Vectors with angle brackets
%% \makeatletter
%% \renewcommand{\vector}[2][ , ]{\left\langle%
%%   \def\nextitem{\def\nextitem{#1}}%
%%   \@for \el:=#2\do{\nextitem\el}\right\rangle%
%% }
%% \makeatother

%% %% Vertical Vectors
%% \def\vector#1{\begin{bmatrix}\vecListA#1,,\end{bmatrix}}
%% \def\vecListA#1,{\if,#1,\else #1\cr \expandafter \vecListA \fi}

%%%%%%%%%%%%%
%% End of vectors
%%%%%%%%%%%%%

%\newcommand{\fullwidth}{}
%\newcommand{\normalwidth}{}



%% makes a snazzy t-chart for evaluating functions
%\newenvironment{tchart}{\rowcolors{2}{}{background!90!textColor}\array}{\endarray}

%%This is to help with formatting on future title pages.
\newenvironment{sectionOutcomes}{}{}



%% Flowchart stuff
%\tikzstyle{startstop} = [rectangle, rounded corners, minimum width=3cm, minimum height=1cm,text centered, draw=black]
%\tikzstyle{question} = [rectangle, minimum width=3cm, minimum height=1cm, text centered, draw=black]
%\tikzstyle{decision} = [trapezium, trapezium left angle=70, trapezium right angle=110, minimum width=3cm, minimum height=1cm, text centered, draw=black]
%\tikzstyle{question} = [rectangle, rounded corners, minimum width=3cm, minimum height=1cm,text centered, draw=black]
%\tikzstyle{process} = [rectangle, minimum width=3cm, minimum height=1cm, text centered, draw=black]
%\tikzstyle{decision} = [trapezium, trapezium left angle=70, trapezium right angle=110, minimum width=3cm, minimum height=1cm, text centered, draw=black]




\author{Tom Needham}

\outcome{Evaluate limits of sequences.}

\title[]{Limits of Sequences}

\begin{document}
\begin{abstract}
\end{abstract}
\maketitle

\vspace{-0.4in}

\section{Discussion Questions}

\begin{problem}
Quickly evaluate the limits of the following sequences, with as few explicit calculations as possible. In each example, give a quick justification for your answer using what you know about growth rates of functions.
\begin{center}
\begin{tabular}{lll}
I. $\lim_{n \rightarrow \infty} \frac{(n+2)^3}{(2n+1)(n+2)}$ \hspace{.1in} II. $\lim_{n \rightarrow \infty} \frac{n^2 + \ln(n+2)-1000}{10n - n^{11/5}}$ \hspace{.1in} III. $\lim_{n\rightarrow \infty} \frac{5 \cdot 4^n}{(2^n + n^{50})^2}$
\end{tabular}
\end{center}
\end{problem}

\begin{freeResponse}
We will use growth rate comparisons to quickly evaluate the limits.

I. The dominant term in the numerator is $n^3$ and the dominant term in the denominator is $2n^2$, so a growth rate comparison shows that the limit is $\infty$.

II. The dominant term in the numerator is $n^2$ and the dominant term in the denominator is $n^{11/5}$. Since $\frac{11}{5} > 2$, the limit is $0$. 

III. The only term in the numerator is $5 \cdot 4^n$ and the dominant term in the denominator is $(2^n)^2 = 4^n$. It follows that the limit is $5$. 
\end{freeResponse}

\begin{problem}
Determine which of the following statements are true.
\begin{enumerate}[label=(\alph*)]
\item A sequence which has a limit must be bounded and monotonic.
\item A sequence which is bounded and monotonic must have a limit.
\item If the sequences $\{a_n\}$ and $\{b_n\}$ are both bounded, then the sequence $\{a_n + b_n\}$ must be bounded.
\item A sequence which has a limit must be bounded.
\item A bounded sequence must have a limit.
\item A sequence which takes only finitely many distinct values must have a limit.
\end{enumerate}
\end{problem}

\begin{freeResponse}
\begin{enumerate}[label=(\alph*)]
\item This statement is not true. For example, the sequence $\left\{\frac{1}{n} \cos (n) \right\}_{n=1}^\infty$ is not monotonic, but the limit of its terms is $0$. 
\item This statement is true (it is an important theorem in the text).
\item This statement is true. If we have bounds $A \leq a_n \leq A'$ and $B \leq b_n \leq B'$ which hold for all $n$, then $A + B \leq a_n + b_n \leq A' + B'$ for all $n$. It follows that the sequence $\{a_n+b_n\}$ is bounded.
\item This statement is true. Assume that the sequence is indexed as $\{a_n\}_{n=1}^\infty$ and that $\lim_{n \rightarrow \infty} a_n = L$. Then there exists an index $N$ such that $L-1 \leq a_k \leq L+1$ whenever $k \geq N$. This shows that the ``tail" of the sequence is bounded between $L-1$ and $L+1$. On the other hand, there are finitely many terms $a_1,a_2,\ldots,a_{N-1}$ with index less than $N$, and these finitely many terms must have upper and lower bounds (e.g., the largest number in the finite list is an upper bound for all the terms in the list). Putting these statements together, we see that the sequence $\{a_n\}$ must be bounded.
\item The statement is false. For example, consider the sequence defined by $a_n = 1$ if $n$ is even and $a_n = -1$ if $n$ is odd. The sequence is bounded, but doesn't have a limit. 
\item The same counterexample used for the previous statement also shows that this statement is false.
\end{enumerate}
\end{freeResponse}
\section{Group Work}

\begin{problem}
Consider the sequence $\{a_n\}_{n=1}^\infty$ defined by the formula
$$
a_n = n^2 + 10n -1.
$$
Compute the limit
$$
\lim_{n \rightarrow \infty} \frac{a_{n+1}}{a_n}.
$$
\end{problem}

\begin{freeResponse}
Using the method of growth rates comparison, we have
$$
\lim_{n \rightarrow \infty} \frac{a_{n+1}}{a_n} = \lim_{n \rightarrow \infty}  \frac{(n+1)^2 + 10(n+1) - 1}{n^2 + 10n -1} = 1.
$$
\end{freeResponse}

\begin{problem}
Consider the sequence $\{a_n\}_{n=1}^\infty$ defined by the formula
$$
a_n = \left(\frac{1}{n}\right)^{\ln n}.
$$
Compute the limit
$$
\lim_{n \rightarrow \infty} a_n.
$$
\end{problem}

\begin{freeResponse}
Consider the sequence $\{b_n\}_{n=1}^\infty$ defined by the formula
$$
b_n = \ln a_n = \ln  \left(\frac{1}{n}\right)^{\ln n} = \ln n \ln \frac{1}{n}.
$$
We know that $\lim_{n\rightarrow \infty} \ln n = \infty$ and $\lim_{n\rightarrow \infty} \ln \frac{1}{n} = -\infty$, and it follows that $\lim_{n\rightarrow \infty} b_n = -\infty$.  Finally, we have
$$
\lim_{n\rightarrow \infty} a_n = e^{\lim_{n \rightarrow \infty} b_n} =0.
$$
\end{freeResponse}

\begin{problem}
Consider the sequences $\{a_n\}_{n=1}^\infty$, $\{b_n\}_{n=1}^\infty$ and $\{s_n\}_{n=1}^\infty$ defined, respectively, by the formulas
$$
a_n = \frac{1}{n},
$$
$$
b_n = 5 + \sin(n)
$$
and
$$
s_n = \sum_{k=1}^n b_k.
$$
Determine which of the following properties each sequence has:
\begin{enumerate}[label=(\alph*)]
\item the sequence is bounded
\item the sequence is monotone increasing
\item the sequence is monotone decreasing
\item the sequence has a finite limit
\end{enumerate}
\end{problem}

\begin{freeResponse}
The sequence $\{a_n\}$ is bounded (above by $1$ and below by $0$), montonone decreasing, and therefore must have a finite limit (specifically, $0$). 

Since the function $f(x)=\sin(x)$ is bounded by $\pm 1$, the sequence $\{b_n\}$ is bounded below by $4$ and above by $6$. The sequence is neither monotone increasing nor monotone decreasing (compute the terms $n=4,5,6$ to verify this). It has no limit, since the terms realize large oscillations in the interval $[4,6]$ indefinitely. 

The sequence $\{s_n\}$ is monotone increasing, since all of the terms $b_n$ are positive. It is unbounded, since all of the terms $b_n$ are greater than or equal to $4$. Since the sequence is unbounded, it does not have a finite limit. 
\end{freeResponse}
\end{document}
