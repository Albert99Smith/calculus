\documentclass[noauthor]{ximera}
%handout:  for handout version with no solutions or instructor notes
%handout,instructornotes:  for instructor version with just problems and notes, no solutions
%noinstructornotes:  shows only problem and solutions

%% handout
%% space
%% newpage
%% numbers
%% nooutcomes

%I added the commands here so that I would't have to keep looking them up
%\newcommand{\RR}{\mathbb R}
%\renewcommand{\d}{\,d}
%\newcommand{\dd}[2][]{\frac{d #1}{d #2}}
%\renewcommand{\l}{\ell}
%\newcommand{\ddx}{\frac{d}{dx}}
%\everymath{\displaystyle}
%\newcommand{\dfn}{\textbf}
%\newcommand{\eval}[1]{\bigg[ #1 \bigg]}

%\begin{image}
%\includegraphics[trim= 170 420 250 180]{Figure1.pdf}
%\end{image}

%add a ``.'' below when used in a specific directory.

%\usepackage{todonotes}
%\usepackage{mathtools} %% Required for wide table Curl and Greens
%\usepackage{cuted} %% Required for wide table Curl and Greens
\newcommand{\todo}{}

\usepackage{esint} % for \oiint
\ifxake%%https://math.meta.stackexchange.com/questions/9973/how-do-you-render-a-closed-surface-double-integral
\renewcommand{\oiint}{{\large\bigcirc}\kern-1.56em\iint}
\fi


\graphicspath{
  {./}
  {ximeraTutorial/}
  {basicPhilosophy/}
  {functionsOfSeveralVariables/}
  {normalVectors/}
  {lagrangeMultipliers/}
  {vectorFields/}
  {greensTheorem/}
  {shapeOfThingsToCome/}
  {dotProducts/}
  {partialDerivativesAndTheGradientVector/}
  {../productAndQuotientRules/exercises/}
  {../normalVectors/exercisesParametricPlots/}
  {../continuityOfFunctionsOfSeveralVariables/exercises/}
  {../partialDerivativesAndTheGradientVector/exercises/}
  {../directionalDerivativeAndChainRule/exercises/}
  {../commonCoordinates/exercisesCylindricalCoordinates/}
  {../commonCoordinates/exercisesSphericalCoordinates/}
  {../greensTheorem/exercisesCurlAndLineIntegrals/}
  {../greensTheorem/exercisesDivergenceAndLineIntegrals/}
  {../shapeOfThingsToCome/exercisesDivergenceTheorem/}
  {../greensTheorem/}
  {../shapeOfThingsToCome/}
  {../separableDifferentialEquations/exercises/}
  {vectorFields/}
}

\newcommand{\mooculus}{\textsf{\textbf{MOOC}\textnormal{\textsf{ULUS}}}}

\usepackage{tkz-euclide}\usepackage{tikz}
\usepackage{tikz-cd}
\usetikzlibrary{arrows}
\tikzset{>=stealth,commutative diagrams/.cd,
  arrow style=tikz,diagrams={>=stealth}} %% cool arrow head
\tikzset{shorten <>/.style={ shorten >=#1, shorten <=#1 } } %% allows shorter vectors

\usetikzlibrary{backgrounds} %% for boxes around graphs
\usetikzlibrary{shapes,positioning}  %% Clouds and stars
\usetikzlibrary{matrix} %% for matrix
\usepgfplotslibrary{polar} %% for polar plots
\usepgfplotslibrary{fillbetween} %% to shade area between curves in TikZ
\usetkzobj{all}
\usepackage[makeroom]{cancel} %% for strike outs
%\usepackage{mathtools} %% for pretty underbrace % Breaks Ximera
%\usepackage{multicol}
\usepackage{pgffor} %% required for integral for loops



%% http://tex.stackexchange.com/questions/66490/drawing-a-tikz-arc-specifying-the-center
%% Draws beach ball
\tikzset{pics/carc/.style args={#1:#2:#3}{code={\draw[pic actions] (#1:#3) arc(#1:#2:#3);}}}



\usepackage{array}
\setlength{\extrarowheight}{+.1cm}
\newdimen\digitwidth
\settowidth\digitwidth{9}
\def\divrule#1#2{
\noalign{\moveright#1\digitwidth
\vbox{\hrule width#2\digitwidth}}}





\newcommand{\RR}{\mathbb R}
\newcommand{\R}{\mathbb R}
\newcommand{\N}{\mathbb N}
\newcommand{\Z}{\mathbb Z}

\newcommand{\sagemath}{\textsf{SageMath}}


%\renewcommand{\d}{\,d\!}
\renewcommand{\d}{\mathop{}\!d}
\newcommand{\dd}[2][]{\frac{\d #1}{\d #2}}
\newcommand{\pp}[2][]{\frac{\partial #1}{\partial #2}}
\renewcommand{\l}{\ell}
\newcommand{\ddx}{\frac{d}{\d x}}

\newcommand{\zeroOverZero}{\ensuremath{\boldsymbol{\tfrac{0}{0}}}}
\newcommand{\inftyOverInfty}{\ensuremath{\boldsymbol{\tfrac{\infty}{\infty}}}}
\newcommand{\zeroOverInfty}{\ensuremath{\boldsymbol{\tfrac{0}{\infty}}}}
\newcommand{\zeroTimesInfty}{\ensuremath{\small\boldsymbol{0\cdot \infty}}}
\newcommand{\inftyMinusInfty}{\ensuremath{\small\boldsymbol{\infty - \infty}}}
\newcommand{\oneToInfty}{\ensuremath{\boldsymbol{1^\infty}}}
\newcommand{\zeroToZero}{\ensuremath{\boldsymbol{0^0}}}
\newcommand{\inftyToZero}{\ensuremath{\boldsymbol{\infty^0}}}



\newcommand{\numOverZero}{\ensuremath{\boldsymbol{\tfrac{\#}{0}}}}
\newcommand{\dfn}{\textbf}
%\newcommand{\unit}{\,\mathrm}
\newcommand{\unit}{\mathop{}\!\mathrm}
\newcommand{\eval}[1]{\bigg[ #1 \bigg]}
\newcommand{\seq}[1]{\left( #1 \right)}
\renewcommand{\epsilon}{\varepsilon}
\renewcommand{\phi}{\varphi}


\renewcommand{\iff}{\Leftrightarrow}

\DeclareMathOperator{\arccot}{arccot}
\DeclareMathOperator{\arcsec}{arcsec}
\DeclareMathOperator{\arccsc}{arccsc}
\DeclareMathOperator{\si}{Si}
\DeclareMathOperator{\scal}{scal}
\DeclareMathOperator{\sign}{sign}


%% \newcommand{\tightoverset}[2]{% for arrow vec
%%   \mathop{#2}\limits^{\vbox to -.5ex{\kern-0.75ex\hbox{$#1$}\vss}}}
\newcommand{\arrowvec}[1]{{\overset{\rightharpoonup}{#1}}}
%\renewcommand{\vec}[1]{\arrowvec{\mathbf{#1}}}
\renewcommand{\vec}[1]{{\overset{\boldsymbol{\rightharpoonup}}{\mathbf{#1}}}\hspace{0in}}

\newcommand{\point}[1]{\left(#1\right)} %this allows \vector{ to be changed to \vector{ with a quick find and replace
\newcommand{\pt}[1]{\mathbf{#1}} %this allows \vec{ to be changed to \vec{ with a quick find and replace
\newcommand{\Lim}[2]{\lim_{\point{#1} \to \point{#2}}} %Bart, I changed this to point since I want to use it.  It runs through both of the exercise and exerciseE files in limits section, which is why it was in each document to start with.

\DeclareMathOperator{\proj}{\mathbf{proj}}
\newcommand{\veci}{{\boldsymbol{\hat{\imath}}}}
\newcommand{\vecj}{{\boldsymbol{\hat{\jmath}}}}
\newcommand{\veck}{{\boldsymbol{\hat{k}}}}
\newcommand{\vecl}{\vec{\boldsymbol{\l}}}
\newcommand{\uvec}[1]{\mathbf{\hat{#1}}}
\newcommand{\utan}{\mathbf{\hat{t}}}
\newcommand{\unormal}{\mathbf{\hat{n}}}
\newcommand{\ubinormal}{\mathbf{\hat{b}}}

\newcommand{\dotp}{\bullet}
\newcommand{\cross}{\boldsymbol\times}
\newcommand{\grad}{\boldsymbol\nabla}
\newcommand{\divergence}{\grad\dotp}
\newcommand{\curl}{\grad\cross}
%\DeclareMathOperator{\divergence}{divergence}
%\DeclareMathOperator{\curl}[1]{\grad\cross #1}
\newcommand{\lto}{\mathop{\longrightarrow\,}\limits}

\renewcommand{\bar}{\overline}

\colorlet{textColor}{black}
\colorlet{background}{white}
\colorlet{penColor}{blue!50!black} % Color of a curve in a plot
\colorlet{penColor2}{red!50!black}% Color of a curve in a plot
\colorlet{penColor3}{red!50!blue} % Color of a curve in a plot
\colorlet{penColor4}{green!50!black} % Color of a curve in a plot
\colorlet{penColor5}{orange!80!black} % Color of a curve in a plot
\colorlet{penColor6}{yellow!70!black} % Color of a curve in a plot
\colorlet{fill1}{penColor!20} % Color of fill in a plot
\colorlet{fill2}{penColor2!20} % Color of fill in a plot
\colorlet{fillp}{fill1} % Color of positive area
\colorlet{filln}{penColor2!20} % Color of negative area
\colorlet{fill3}{penColor3!20} % Fill
\colorlet{fill4}{penColor4!20} % Fill
\colorlet{fill5}{penColor5!20} % Fill
\colorlet{gridColor}{gray!50} % Color of grid in a plot

\newcommand{\surfaceColor}{violet}
\newcommand{\surfaceColorTwo}{redyellow}
\newcommand{\sliceColor}{greenyellow}




\pgfmathdeclarefunction{gauss}{2}{% gives gaussian
  \pgfmathparse{1/(#2*sqrt(2*pi))*exp(-((x-#1)^2)/(2*#2^2))}%
}


%%%%%%%%%%%%%
%% Vectors
%%%%%%%%%%%%%

%% Simple horiz vectors
\renewcommand{\vector}[1]{\left\langle #1\right\rangle}


%% %% Complex Horiz Vectors with angle brackets
%% \makeatletter
%% \renewcommand{\vector}[2][ , ]{\left\langle%
%%   \def\nextitem{\def\nextitem{#1}}%
%%   \@for \el:=#2\do{\nextitem\el}\right\rangle%
%% }
%% \makeatother

%% %% Vertical Vectors
%% \def\vector#1{\begin{bmatrix}\vecListA#1,,\end{bmatrix}}
%% \def\vecListA#1,{\if,#1,\else #1\cr \expandafter \vecListA \fi}

%%%%%%%%%%%%%
%% End of vectors
%%%%%%%%%%%%%

%\newcommand{\fullwidth}{}
%\newcommand{\normalwidth}{}



%% makes a snazzy t-chart for evaluating functions
%\newenvironment{tchart}{\rowcolors{2}{}{background!90!textColor}\array}{\endarray}

%%This is to help with formatting on future title pages.
\newenvironment{sectionOutcomes}{}{}



%% Flowchart stuff
%\tikzstyle{startstop} = [rectangle, rounded corners, minimum width=3cm, minimum height=1cm,text centered, draw=black]
%\tikzstyle{question} = [rectangle, minimum width=3cm, minimum height=1cm, text centered, draw=black]
%\tikzstyle{decision} = [trapezium, trapezium left angle=70, trapezium right angle=110, minimum width=3cm, minimum height=1cm, text centered, draw=black]
%\tikzstyle{question} = [rectangle, rounded corners, minimum width=3cm, minimum height=1cm,text centered, draw=black]
%\tikzstyle{process} = [rectangle, minimum width=3cm, minimum height=1cm, text centered, draw=black]
%\tikzstyle{decision} = [trapezium, trapezium left angle=70, trapezium right angle=110, minimum width=3cm, minimum height=1cm, text centered, draw=black]




\author{Tom Needham and Jim Talamo}

\outcome{Use trigonometric identities to evaluate integrals.}

\title[]{Trigonometric Integrals}

\begin{document}
\begin{abstract}
\end{abstract}
\maketitle


\section{Discussion Questions}

\begin{problem}
Determine an effective strategy for evaluating each of the following integrals (you do not actually need to evaluate the integrals).

\begin{center}
\begin{tabular}{lll}
I. $\int \cos^2(x) \d x$ \hspace{.1in} & II. $\int 2x+\sin (x) \cos^2(x) \d x$ \hspace{.1in} &  III. $\int \sin^2(x) \cos^2(x) \d x$
\end{tabular}
\end{center}
\end{problem}

\begin{freeResponse}
I. This integral can be solved using the cosine half-angle identity 
$$
\cos^2(x) = \frac{1+\cos(2x)}{2},
$$
together with basic integration rules.

II. This integral can be split into two integrals

\[
\int 2x+\sin (x) \cos^2(x) \d x = \int 2x \d x +\int \sin (x) \cos^2(x) \d x.
\]
The first can be evaluated directly and the second can be solved via $u$-substitution, with $u=\cos (x)$, so that $\d u = -\sin (x) \d x$.

III. To solve this integral, use the double-angle identity
$$
\sin (2x) = 2 \sin(x) \cos(x),
$$
which implies
$$
\int \sin^2(x) \cos^2(x) \d x= \int \frac{1}{4} \sin^2(2x) \d x.
$$
Next we use a half-angle identity for sine, following a similar process to that used in Part I.
\end{freeResponse}


\begin{problem}
Fill in the following table.

\[
\begin{array}{ll}
\int \sin(x) \d x = \underline{\hspace{30mm}} \qquad \qquad & \int \csc(x) \d x = \underline{\hspace{30mm}} \\ [2 ex]
\int \cos(x) \d x = \underline{\hspace{30mm}} \qquad & \int \sec(x) \d x = \underline{\hspace{30mm}} \\ [2 ex]
\int \tan(x) \d x = \underline{\hspace{30mm}} \qquad & \int \cot(x) \d x = \underline{\hspace{30mm}} \\ [2 ex]
\end{array}
\]

\end{problem}

\begin{freeResponse}
\[
\begin{array}{ll}
\int \sin(x) \d x = -\cos(x)+C \qquad \qquad \qquad & \int \csc(x) \d x = -\ln|\csc(x)+\cot(x)|+C \\ [2 ex]
\int \cos(x) \d x =\sin(x)+C \qquad & \int \sec(x) \d x = \ln|\sec(x)+\tan(x)| +C \\ [2 ex]
\int \tan(x) \d x =\ln|\sec(x)|+C \qquad & \int \cot(x) \d x = -\ln|\csc(x)|+C \\ [2 ex]
\end{array}
\]
Note that while you should commit these to memory, the formulas for the antiderivatives of $\tan(x)$, $\cot(x)$, $\sec(x)$, and $\csc(x)$ can be derived as follows.

\begin{itemize}
\item For $\int \tan(x) \d x$, write $\int \tan(x) \d x = \int\frac{\sin(x)}{\cos(x)} \d x$ and set $u=\cos(x)$.  After working through the substitution, note that

\[
-\ln|\cos(x)| = \ln\left(|\cos(x)|^{-1}\right) = \ln\left|\frac{1}{\cos(x)}\right| = \ln|\sec(x)|.
\]

\item The formula for $\int \cot(x) \d x$ can be found in a similar fashion as the formula for $\tan(x) \d x$ was derived.

\item For $\int \sec(x) \d x$, we can multiply and divide by $\sec(x) + \tan(x)$ as follows.

\[
\int \sec(x) \d x = \int \sec(x) \cdot \frac{\sec(x) +\tan(x)}{\sec(x) +\tan(x)} \d x = \int \frac{\sec^2(x) +\sec(x)\tan(x)}{\sec(x) +\tan(x)} \d x
\]
While this may seem strange, note that by letting $u = \sec(x) +\tan(x)$, we obtain 

\[
 \int \frac{\sec^2(x) +\sec(x)\tan(x)}{\sec(x) +\tan(x)} \d x = \int \frac{1}{u} \d u.
\]

\item The formula for $\int \csc(x) \d x$ can be found in a similar fashion as the formula for $\sec(x) \d x$ was derived.

\end{itemize}
\end{freeResponse}

\begin{problem}
Consider the integral
$$
\int \tan (x) \sec^2(x) \d x.
$$
Student A evaluates this integral by making the substitution $u=\tan(x)$, so that $\d u = \sec^2 (x) \d x$, and
$$
\int \tan(x) \sec^2(x) \d x = \int u \d u = \frac{u^2}{2} + C = \frac{\tan^2 (x)}{2} + C.
$$
Student B makes the substitution $u = \sec(x)$, so that $\d u = \sec(x) \tan(x)$, and the integral becomes
$$
\int \tan (x) \sec^2 (x) \d x = \int \sec(x) \sec (x) \tan (x) \d x = \int u \d u = \frac{u^2}{2} +C = \frac{\sec^2(x)}{2} +C.
$$
Which of the students' solutions is correct? Either find a mistake in a student's work, or reconcile the two answers.
\end{problem}

\begin{freeResponse}
Both solutions are correct. To reconcile the answers, note that the difference between the non-constant parts of their answers is 
$$
\frac{\tan^2 (x)}{2} - \frac{\sec^2(x)}{2} = \frac{\tan^2(x) - \sec^2(x)}{2}  = -\frac{1}{2},
$$
where the last equality follows by a trigonometric identity. We conclude that the students' answers are equal, since indefinite integrals are only defined up to an additive constant.
\end{freeResponse}

\section{Group Work}

\begin{problem}
Evaluate the following integrals

\begin{center}
\begin{tabular}{lll}
I. $\int \sin^2(4x) \d x$ \hspace{10mm} & II. $\int \sin^3 (4 \theta) \cos^6 (4\theta) \d \theta$ \hspace{.2in} & III. $\int \tan^{23}(x) \sec^6 (x) \d x$  \\
%IV. $\int \tan^2(x) \sec (x) \d x$ \hspace{.2in} & 
IV. $\int \tan^2(x) \sin (x) \d x$ & V. $\int x+ \tan^2(x) \d x$
\end{tabular}
\end{center}
\end{problem}

\begin{freeResponse}
I. We use the identity 

\[
\sin^2(\theta) = \frac{1}{2}-\frac{1}{2} \cos(2 \theta)
\]
with $\theta = 4x$ to write 
\[
\int \sin^2(4x) \d x = \int \frac{1}{2}-\frac{1}{2} \cos(8x) \d x = \frac{1}{2}x-\frac{1}{16} \sin(8x)+C
\]

II. Write $\sin^3(4\theta) = \sin(4\theta) \sin^2(4\theta) = \sin(4\theta) (1-\cos^2(4\theta))$ and let $u = \cos(4\theta)$, whence $\d u = -4 \sin (4\theta) \d \theta$. The integral becomes
\begin{align*}
\int \sin^3 (4 \theta) \cos^6 (4\theta) \, \mathrm{d}\theta &= -\frac{1}{4} \int (1-u^2)u^6 \d u \\
&= -\frac{1}{4} \int (1-u^2)u^6 \d u \\
&= -\frac{1}{4} \int u^6 - u^8 \d u \\
&= -\frac{1}{4}\left(\frac{u^7}{7} - \frac{u^9}{9}\right) + C \\
&= -\frac{1}{4}\left(\frac{\cos^7(4\theta)}{7} - \frac{\cos^9(4\theta)}{9}\right) + C.
\end{align*}

III. We first break the $\sec^6 x$ factor into pieces, as 
	\begin{align*}
		\int \tan^{23} x \sec^6 x \d x 
		&= \int \tan^{23} x \sec^4 x \sec^2 x \d x  \\
		&= \int \tan^{23} x \left( 1 + \tan^2(x) \right)^2 \sec^2 x \d x .
		\end{align*}
	Next we make the substitution $u = \tan(x)$, so that $\d u = \sec^2 x \d x$, and
		\begin{align*}
		\int \tan^{23} x \left( 1 + \tan^2(x) \right)^2 \sec^2 x \d x
		&= \int u^{23} (1+u^2)^2 \d u  \\
		&= \int u^{23} (1 + 2u^2 + u^4) \d u  \\
		&= \int \left( u^{23} + 2u^{25} + u^{27} \right) \d u  \\
		&= \frac{1}{24} u^{24} + \frac{1}{13} u^{26} + \frac{1}{28} u^{28} + C  \\
		&= \frac{1}{24} \tan^{24}(x) + \frac{1}{13} \tan^{26}(x) + \frac{1}{28} \tan^{28}(x) + C.
		\end{align*}
		
%IV. The trick here is to recall that 
%$$
%\int \sec (x) \d x = \ln | \sec (x) + \tan(x)| + C.
%$$
%It follows that
%\begin{align*}
%		\int \tan^2(x) \sec (x) \d x &= \int \left( \sec^2 x - 1 \right) \sec (x) \d x  	\\
%		&= \int \sec^3 x \d x - \int \sec (x) \d x  	\\
%		&= \int \sec^3 x \d x - \ln | \sec (x) + \tan(x)| 
%		\end{align*}
%	To evaluate $\int \sec^3 x \d x$, we use integration by parts with
%		{
%		\[
%		u = \sec (x) 				\qquad	\d v = \sec^2 x \d x
%		\]
%		\[
%		\d u = \sec (x) \tan (x) \d x	\qquad	v = \tan(x).
%		\]
%		}
%	So
%		\begin{equation}\label{equation 2}
%		\int \sec^3 x \d x = \sec (x) \tan (x) - \int \tan^2(x) \sec (x) \d x.
%		\end{equation}
%	Combining our work yields
%		\begin{align*}
%		\int \tan^2(x) \sec (x) \d x &= \int \sec^3 x \d x - \ln | \sec (x) + \tan(x)|   \\
%		\int \tan^2(x) \sec (x) \d x &= \sec (x) \tan (x) - \int \tan^2(x) \sec (x) \d x - \ln | \sec (x) + \tan (x) |  \\
%		2 \int \tan^2(x) \sec (x) \d x &= \sec (x) \tan (x) - \ln | \sec (x) + \tan (x) | + C  \\
%		\int \tan^2(x) \sec (x) \d x &= \frac{1}{2} \left( \sec (x) \tan (x) - \ln | \sec (x) + \tan (x) | \right) + C.
%		\end{align*}
		
IV. We first apply trigonometric identities to conclude that
		\begin{align*}
		\int \tan^2(x) \sin (x) \d x
		&= \int \frac{\sin^2 x}{\cos^2 x} \sin (x) \d x  \\
		&= \int \frac{1-\cos^2 x}{\cos^2 x} \sin (x) \d x.
		\end{align*}
	Now we substitute $u = \cos (x) $, so that $\d u = - \sin (x) \d x$. This gives us that
		\begin{align*}
		\int \tan^2(x) \sin (x) \d x &= \int \frac{1-u^2}{u^2} (-1) \d u  \\
		&= \int \frac{u^2 - 1}{u^2} \d u  \\
		&= \int \left( 1 - u^{-2} \right) \d u  \\
		&= u + \frac{1}{u} + C  \\
		&= \cos (x) + \sec (x) + C.
		\end{align*}
		
V. We first split the integral up.

\[ \int x+ \tan^2(x) \d x = \int x \d x +\int  \tan^2(x) \d x\]

To evaluate $\int \tan^2(x) \d x$, note that $\tan^2(x) = \sec^2(x)-1$, so

\begin{align*}
\int x+ \tan^2(x) \d x &= \int x \d x +\int  \tan^2(x) \d x \\
&= \frac{1}{2}x^2 +\int  \sec^2(x) -1 \d x \\
&=  \frac{1}{2}x^2 + \tan(x) -x+C \\
\end{align*}
\end{freeResponse}

\begin{problem}
Evaluate the definite integral
$$
\int_{- \pi}^0 \sqrt{1 - \cos^2 x} \d x.
$$
\end{problem}

\begin{freeResponse}
Using the Pythagorean identity, we have
\begin{align*}
		\int_{- \pi}^0 \sqrt{1 - \cos^2 x} \d x
		&= \int_{- \pi}^0 \sqrt{\sin^2 x} \d x  \\
		&= \int_{- \pi}^0 | \sin (x) | \d x.
		\end{align*}
	Now, when $-\pi \leq x \leq 0$, $\sin (x) \leq 0$.  
	Thus, on this region, $|\sin (x) | = - \sin x$.
	So we continue
		\begin{align*}
		\int_{- \pi}^0 \sqrt{1 - \cos^2 x} \d x
		&= \int_{- \pi}^0 - \sin (x) \d x  \\
		&= \eval{\cos x}_{-\pi}^0  \\
		&= \cos(0) - \cos(-\pi) = 1 - (-1) = 2.
		\end{align*}
\end{freeResponse}
\end{document}
