\documentclass{ximera}

%\usepackage{todonotes}
%\usepackage{mathtools} %% Required for wide table Curl and Greens
%\usepackage{cuted} %% Required for wide table Curl and Greens
\newcommand{\todo}{}

\usepackage{esint} % for \oiint
\ifxake%%https://math.meta.stackexchange.com/questions/9973/how-do-you-render-a-closed-surface-double-integral
\renewcommand{\oiint}{{\large\bigcirc}\kern-1.56em\iint}
\fi


\graphicspath{
  {./}
  {ximeraTutorial/}
  {basicPhilosophy/}
  {functionsOfSeveralVariables/}
  {normalVectors/}
  {lagrangeMultipliers/}
  {vectorFields/}
  {greensTheorem/}
  {shapeOfThingsToCome/}
  {dotProducts/}
  {partialDerivativesAndTheGradientVector/}
  {../productAndQuotientRules/exercises/}
  {../normalVectors/exercisesParametricPlots/}
  {../continuityOfFunctionsOfSeveralVariables/exercises/}
  {../partialDerivativesAndTheGradientVector/exercises/}
  {../directionalDerivativeAndChainRule/exercises/}
  {../commonCoordinates/exercisesCylindricalCoordinates/}
  {../commonCoordinates/exercisesSphericalCoordinates/}
  {../greensTheorem/exercisesCurlAndLineIntegrals/}
  {../greensTheorem/exercisesDivergenceAndLineIntegrals/}
  {../shapeOfThingsToCome/exercisesDivergenceTheorem/}
  {../greensTheorem/}
  {../shapeOfThingsToCome/}
  {../separableDifferentialEquations/exercises/}
  {vectorFields/}
}

\newcommand{\mooculus}{\textsf{\textbf{MOOC}\textnormal{\textsf{ULUS}}}}

\usepackage{tkz-euclide}\usepackage{tikz}
\usepackage{tikz-cd}
\usetikzlibrary{arrows}
\tikzset{>=stealth,commutative diagrams/.cd,
  arrow style=tikz,diagrams={>=stealth}} %% cool arrow head
\tikzset{shorten <>/.style={ shorten >=#1, shorten <=#1 } } %% allows shorter vectors

\usetikzlibrary{backgrounds} %% for boxes around graphs
\usetikzlibrary{shapes,positioning}  %% Clouds and stars
\usetikzlibrary{matrix} %% for matrix
\usepgfplotslibrary{polar} %% for polar plots
\usepgfplotslibrary{fillbetween} %% to shade area between curves in TikZ
\usetkzobj{all}
\usepackage[makeroom]{cancel} %% for strike outs
%\usepackage{mathtools} %% for pretty underbrace % Breaks Ximera
%\usepackage{multicol}
\usepackage{pgffor} %% required for integral for loops



%% http://tex.stackexchange.com/questions/66490/drawing-a-tikz-arc-specifying-the-center
%% Draws beach ball
\tikzset{pics/carc/.style args={#1:#2:#3}{code={\draw[pic actions] (#1:#3) arc(#1:#2:#3);}}}



\usepackage{array}
\setlength{\extrarowheight}{+.1cm}
\newdimen\digitwidth
\settowidth\digitwidth{9}
\def\divrule#1#2{
\noalign{\moveright#1\digitwidth
\vbox{\hrule width#2\digitwidth}}}





\newcommand{\RR}{\mathbb R}
\newcommand{\R}{\mathbb R}
\newcommand{\N}{\mathbb N}
\newcommand{\Z}{\mathbb Z}

\newcommand{\sagemath}{\textsf{SageMath}}


%\renewcommand{\d}{\,d\!}
\renewcommand{\d}{\mathop{}\!d}
\newcommand{\dd}[2][]{\frac{\d #1}{\d #2}}
\newcommand{\pp}[2][]{\frac{\partial #1}{\partial #2}}
\renewcommand{\l}{\ell}
\newcommand{\ddx}{\frac{d}{\d x}}

\newcommand{\zeroOverZero}{\ensuremath{\boldsymbol{\tfrac{0}{0}}}}
\newcommand{\inftyOverInfty}{\ensuremath{\boldsymbol{\tfrac{\infty}{\infty}}}}
\newcommand{\zeroOverInfty}{\ensuremath{\boldsymbol{\tfrac{0}{\infty}}}}
\newcommand{\zeroTimesInfty}{\ensuremath{\small\boldsymbol{0\cdot \infty}}}
\newcommand{\inftyMinusInfty}{\ensuremath{\small\boldsymbol{\infty - \infty}}}
\newcommand{\oneToInfty}{\ensuremath{\boldsymbol{1^\infty}}}
\newcommand{\zeroToZero}{\ensuremath{\boldsymbol{0^0}}}
\newcommand{\inftyToZero}{\ensuremath{\boldsymbol{\infty^0}}}



\newcommand{\numOverZero}{\ensuremath{\boldsymbol{\tfrac{\#}{0}}}}
\newcommand{\dfn}{\textbf}
%\newcommand{\unit}{\,\mathrm}
\newcommand{\unit}{\mathop{}\!\mathrm}
\newcommand{\eval}[1]{\bigg[ #1 \bigg]}
\newcommand{\seq}[1]{\left( #1 \right)}
\renewcommand{\epsilon}{\varepsilon}
\renewcommand{\phi}{\varphi}


\renewcommand{\iff}{\Leftrightarrow}

\DeclareMathOperator{\arccot}{arccot}
\DeclareMathOperator{\arcsec}{arcsec}
\DeclareMathOperator{\arccsc}{arccsc}
\DeclareMathOperator{\si}{Si}
\DeclareMathOperator{\scal}{scal}
\DeclareMathOperator{\sign}{sign}


%% \newcommand{\tightoverset}[2]{% for arrow vec
%%   \mathop{#2}\limits^{\vbox to -.5ex{\kern-0.75ex\hbox{$#1$}\vss}}}
\newcommand{\arrowvec}[1]{{\overset{\rightharpoonup}{#1}}}
%\renewcommand{\vec}[1]{\arrowvec{\mathbf{#1}}}
\renewcommand{\vec}[1]{{\overset{\boldsymbol{\rightharpoonup}}{\mathbf{#1}}}\hspace{0in}}

\newcommand{\point}[1]{\left(#1\right)} %this allows \vector{ to be changed to \vector{ with a quick find and replace
\newcommand{\pt}[1]{\mathbf{#1}} %this allows \vec{ to be changed to \vec{ with a quick find and replace
\newcommand{\Lim}[2]{\lim_{\point{#1} \to \point{#2}}} %Bart, I changed this to point since I want to use it.  It runs through both of the exercise and exerciseE files in limits section, which is why it was in each document to start with.

\DeclareMathOperator{\proj}{\mathbf{proj}}
\newcommand{\veci}{{\boldsymbol{\hat{\imath}}}}
\newcommand{\vecj}{{\boldsymbol{\hat{\jmath}}}}
\newcommand{\veck}{{\boldsymbol{\hat{k}}}}
\newcommand{\vecl}{\vec{\boldsymbol{\l}}}
\newcommand{\uvec}[1]{\mathbf{\hat{#1}}}
\newcommand{\utan}{\mathbf{\hat{t}}}
\newcommand{\unormal}{\mathbf{\hat{n}}}
\newcommand{\ubinormal}{\mathbf{\hat{b}}}

\newcommand{\dotp}{\bullet}
\newcommand{\cross}{\boldsymbol\times}
\newcommand{\grad}{\boldsymbol\nabla}
\newcommand{\divergence}{\grad\dotp}
\newcommand{\curl}{\grad\cross}
%\DeclareMathOperator{\divergence}{divergence}
%\DeclareMathOperator{\curl}[1]{\grad\cross #1}
\newcommand{\lto}{\mathop{\longrightarrow\,}\limits}

\renewcommand{\bar}{\overline}

\colorlet{textColor}{black}
\colorlet{background}{white}
\colorlet{penColor}{blue!50!black} % Color of a curve in a plot
\colorlet{penColor2}{red!50!black}% Color of a curve in a plot
\colorlet{penColor3}{red!50!blue} % Color of a curve in a plot
\colorlet{penColor4}{green!50!black} % Color of a curve in a plot
\colorlet{penColor5}{orange!80!black} % Color of a curve in a plot
\colorlet{penColor6}{yellow!70!black} % Color of a curve in a plot
\colorlet{fill1}{penColor!20} % Color of fill in a plot
\colorlet{fill2}{penColor2!20} % Color of fill in a plot
\colorlet{fillp}{fill1} % Color of positive area
\colorlet{filln}{penColor2!20} % Color of negative area
\colorlet{fill3}{penColor3!20} % Fill
\colorlet{fill4}{penColor4!20} % Fill
\colorlet{fill5}{penColor5!20} % Fill
\colorlet{gridColor}{gray!50} % Color of grid in a plot

\newcommand{\surfaceColor}{violet}
\newcommand{\surfaceColorTwo}{redyellow}
\newcommand{\sliceColor}{greenyellow}




\pgfmathdeclarefunction{gauss}{2}{% gives gaussian
  \pgfmathparse{1/(#2*sqrt(2*pi))*exp(-((x-#1)^2)/(2*#2^2))}%
}


%%%%%%%%%%%%%
%% Vectors
%%%%%%%%%%%%%

%% Simple horiz vectors
\renewcommand{\vector}[1]{\left\langle #1\right\rangle}


%% %% Complex Horiz Vectors with angle brackets
%% \makeatletter
%% \renewcommand{\vector}[2][ , ]{\left\langle%
%%   \def\nextitem{\def\nextitem{#1}}%
%%   \@for \el:=#2\do{\nextitem\el}\right\rangle%
%% }
%% \makeatother

%% %% Vertical Vectors
%% \def\vector#1{\begin{bmatrix}\vecListA#1,,\end{bmatrix}}
%% \def\vecListA#1,{\if,#1,\else #1\cr \expandafter \vecListA \fi}

%%%%%%%%%%%%%
%% End of vectors
%%%%%%%%%%%%%

%\newcommand{\fullwidth}{}
%\newcommand{\normalwidth}{}



%% makes a snazzy t-chart for evaluating functions
%\newenvironment{tchart}{\rowcolors{2}{}{background!90!textColor}\array}{\endarray}

%%This is to help with formatting on future title pages.
\newenvironment{sectionOutcomes}{}{}



%% Flowchart stuff
%\tikzstyle{startstop} = [rectangle, rounded corners, minimum width=3cm, minimum height=1cm,text centered, draw=black]
%\tikzstyle{question} = [rectangle, minimum width=3cm, minimum height=1cm, text centered, draw=black]
%\tikzstyle{decision} = [trapezium, trapezium left angle=70, trapezium right angle=110, minimum width=3cm, minimum height=1cm, text centered, draw=black]
%\tikzstyle{question} = [rectangle, rounded corners, minimum width=3cm, minimum height=1cm,text centered, draw=black]
%\tikzstyle{process} = [rectangle, minimum width=3cm, minimum height=1cm, text centered, draw=black]
%\tikzstyle{decision} = [trapezium, trapezium left angle=70, trapezium right angle=110, minimum width=3cm, minimum height=1cm, text centered, draw=black]


\title[Dig-In:]{Logarithmic differentiation}

\begin{document}
\begin{abstract}
\end{abstract}
\maketitle

Logarithms were originally developed as a computational tool. The key
fact that made this possible is that:
\[
\log_b(xy) = \log_b(x)+\log_b(y).
\]
\begin{image}
\begin{tikzpicture}
	\begin{axis}[
            xmin=-1,xmax=7,ymin=-4,ymax=3.5,
            axis lines=center,
            xlabel=$x$, ylabel=$y$,
            every axis y label/.style={at=(current axis.above origin),anchor=south},
            every axis x label/.style={at=(current axis.right of origin),anchor=west},
            xtick={1,...,7},
          ]        
          \addplot [very thick, penColor, samples=100, smooth, domain=(.01:7)] {ln(x)};
          \addplot [very thick, penColor4] plot coordinates {(2,0) (2,.693)};
          \addplot [very thick, penColor5] plot coordinates {(3,0) (3,1.099)};
          \addplot [very thick, penColor4] plot coordinates {(6,0) (6,.693)};
          \addplot [very thick, penColor5] plot coordinates {(6,.693) (6,1.792)};
        \end{axis}
\end{tikzpicture}
%% \caption{A plot of $\ln(x)$. Here we see that \[\ln(2\cdot 3) = \ln(2) + \ln(3).\]}
%% \label{plot:ln additive property}
\end{image}

Before the days of calculators and computers, this was critical
knowledge for anyone in a computational discipline.

\begin{example}
  Compute $138\cdot 23.4$ using logarithms.
  \begin{explanation}
    Start by writing both numbers in scientific notation
    \[
    \left(1.38\cdot 10^{\answer[given]{2}}\right)\cdot \left(2.34 \cdot 10^{\answer[given]{1}}\right).
    \]
    Next we use a log-table, which gives $\log_{10}(N)$ for values of
    $N$ ranging between $0$ and $9$. We've reproduced part of such a
    table below.
    \begin{image}
       \begin{tikzpicture}
         \matrix (m) [matrix of math nodes, row sep=-.3ex, column sep=-.3ex,ampersand replacement=\&]
         {N \& 0 \& 1 \& 2 \& 3 \& 4 \& 5 \& 6 \& 7 \& 8 \& 9 \\ \hline
         1.3 \& 0.1139 \& 0.1173 \& 0.1206 \& 0.1239 \& 0.1271 \& 0.1303 \& 0.1335 \& 0.1367 \& 0.1399 \& 0.1430 \\
         %\hdotsfor{11} \\ 
         2.3 \& 0.3617 \& 0.3636 \& 0.3655 \& 0.3674 \& 0.3692 \& 0.3711 \& 0.3729 \& 0.3747 \& 0.3766 \& 0.3784 \\
         %\hdotsfor{11} \\
         3.2 \& 0.5052 \& 0.5065 \& 0.5079 \& 0.5092 \& 0.5105 \& 0.5119 \& 0.5132 \& 0.5145 \& 0.5159 \& 0.5172\\};
       \end{tikzpicture}
    \end{image}
    From the table, we see that 
    \[
    \log_{10}(1.38) \approx \answer[given]{0.1399}\qquad\text{and}\qquad \log_{10}(2.34)\approx \answer[given]{0.3692}
    \]
    Add these numbers together to get $\answer[given]{0.5091}$.
    Essentially, we know the following at this point:
    \begin{center}
    \begin{tikzpicture}
      \matrix (m) [matrix of math nodes, row sep=-.3ex, column sep=-.3ex,ampersand replacement=\&]
      {\log_{10}(\mathrm{?}) \& = \& \log_{10}(1.38) \& + \& \log_{10}(2.34)\\
      \rotatebox[origin=c]{90}{$\approx$}  \& \& \rotatebox[origin=c]{90}{$\approx$}   \& \& \rotatebox[origin=c]{90}{$\approx$}  \\
      0.5091 \& = \& 0.1399 \& + \& 0.3692\\};
    \end{tikzpicture}
    \end{center}
    Using the table again, we see that
    $\log_{10}(\answer[given]{3.23})\approx 0.5091$. Since we were
    working in scientific notation, we need to multiply this by
    $10^3$. Our final answer is
    \[
    3230 \approx 138\cdot 23.4
    \]
    Since $138\cdot 23.4 = 3229.2$, this is a good approximation.
  \end{explanation}
\end{example}
The moral is:
\begin{quote}
      \textbf{Logarithms allow us to use addition in place of multiplication.}
\end{quote}


\section{Logarithmic differentiation}


When taking derivatives, both the product rule and the quotient rule
can be cumbersome to use. Logarithms will save the day. A key point is
the following
\[
\ddx \ln(f(x)) = \frac{1}{f(x)}\cdot f'(x) = \frac{f'(x)}{f(x)}
\]
which follows from the chain rule. Let's look at an illustrative
example to see how this is actually used.

\begin{example} 
Compute:
\[
\ddx \frac{x^9e^{4x}}{\sqrt{x^2+4}}
\]

Recall the properties of logarithms:
\begin{itemize}
\item $\log_b(xy) = \log_b(x) + \log_b(y)$
\item $\log_b(x/y) = \log_b(x) - \log_b(y)$
\item $\log_b(x^y) = y\log_b(x)$
\end{itemize}

While we could use the product and quotient rule to solve this
problem, it would be tedious. Start by taking the logarithm of the
function to be differentiated.
\begin{align*}
\ln\left(\frac{x^9e^{4x}}{\sqrt{x^2+4}} \right) &= \ln\left(\answer[given]{x^9e^{4x}}\right) - \ln\left(\answer[given]{\sqrt{x^2+4}}\right)\\
&= \ln\left(x^9\right)+\ln\left(e^{4x}\right) - \ln\left((x^2+4)^{1/2}\right)\\
&= \answer[given]{9}\ln(x)+4x - \answer[given]{\frac{1}{2}}\ln(x^2+4).
\end{align*}
Setting $f(x) = \frac{x^9e^{4x}}{\sqrt{x^2+4}}$, we can write
\[
\ln(f(x)) = 9\ln(x)+4x - \frac{1}{2}\ln(x^2+4).
\]
Differentiating both sides, we find
\[
\frac{f'(x)}{f(x)} = \answer[given]{\frac{9}{x}+4} - \frac{x}{x^2+4}.
\]
Finally we solve for $f'(x)$, write
\[
f'(x) = \left(\frac{9}{x}+4 - \frac{x}{x^2+4}\right)\left(\answer[given]{\frac{x^9e^{4x}}{\sqrt{x^2+4}}}\right).
\]
\end{example}

The process above is called \textit{logarithmic
  differentiation}. Logarithmic differentiation allows us to compute
new derivatives too.

\begin{example}
Compute:
\[
\ddx x^x
\]
\begin{explanation}
The function $x^x$ is tricky to differentiate. We cannot use the power
rule, as the exponent is not a constant. However, if we set $f(x) = x^x$ we can write
\begin{align*}
\ln(f(x)) &= \ln\left(x^x\right)\\
&=x\ln(x).
\end{align*}
Differentiating both sides, we find
\[
\frac{f'(x)}{f(x)} = \answer[given]{1 + \ln(x)}.
\]
Now we can solve for $f'(x)$, 
\[
f'(x) = x^x + x^x\ln(x).
\]
\end{explanation}
\end{example}






\section{A general explanation of the power rule}


Finally, recall that previously we only explained the power rule for
positive exponents. Now we'll use logarithmic differentiation to give
a explanation for all real-valued exponents. We restate the power rule
for convenience sake:

\begin{theorem}[Power Rule]\index{power rule}
For any real number $n$, 
\[
\ddx x^n = n x^{n-1}.
\]
\begin{explanation}
We will use logarithmic differentiation. Set $f(x) = x^n$. Write
\begin{align*}
\ln(f(x)) &= \ln\left(x^n\right) \\ 
&= n\ln(x).
\end{align*}
Now differentiate both sides, and solve for $f'(x)$
\begin{align*}
\frac{f'(x)}{f(x)} &= \frac{n}{x}\\
f'(x) &=\frac{n f(x)}{x}\\
&= \answer[given]{n x^{n-1}}.
\end{align*}
Thus we see that the power rule holds for all real-valued exponents.
\end{explanation}
\end{theorem}

While logarithmic differentiation might seem strange and new at
first, with a little practice it will seem much more natural to you.

\end{document}
