\documentclass{ximera}

%\usepackage{todonotes}
%\usepackage{mathtools} %% Required for wide table Curl and Greens
%\usepackage{cuted} %% Required for wide table Curl and Greens
\newcommand{\todo}{}

\usepackage{esint} % for \oiint
\ifxake%%https://math.meta.stackexchange.com/questions/9973/how-do-you-render-a-closed-surface-double-integral
\renewcommand{\oiint}{{\large\bigcirc}\kern-1.56em\iint}
\fi


\graphicspath{
  {./}
  {ximeraTutorial/}
  {basicPhilosophy/}
  {functionsOfSeveralVariables/}
  {normalVectors/}
  {lagrangeMultipliers/}
  {vectorFields/}
  {greensTheorem/}
  {shapeOfThingsToCome/}
  {dotProducts/}
  {partialDerivativesAndTheGradientVector/}
  {../productAndQuotientRules/exercises/}
  {../normalVectors/exercisesParametricPlots/}
  {../continuityOfFunctionsOfSeveralVariables/exercises/}
  {../partialDerivativesAndTheGradientVector/exercises/}
  {../directionalDerivativeAndChainRule/exercises/}
  {../commonCoordinates/exercisesCylindricalCoordinates/}
  {../commonCoordinates/exercisesSphericalCoordinates/}
  {../greensTheorem/exercisesCurlAndLineIntegrals/}
  {../greensTheorem/exercisesDivergenceAndLineIntegrals/}
  {../shapeOfThingsToCome/exercisesDivergenceTheorem/}
  {../greensTheorem/}
  {../shapeOfThingsToCome/}
  {../separableDifferentialEquations/exercises/}
  {vectorFields/}
}

\newcommand{\mooculus}{\textsf{\textbf{MOOC}\textnormal{\textsf{ULUS}}}}

\usepackage{tkz-euclide}\usepackage{tikz}
\usepackage{tikz-cd}
\usetikzlibrary{arrows}
\tikzset{>=stealth,commutative diagrams/.cd,
  arrow style=tikz,diagrams={>=stealth}} %% cool arrow head
\tikzset{shorten <>/.style={ shorten >=#1, shorten <=#1 } } %% allows shorter vectors

\usetikzlibrary{backgrounds} %% for boxes around graphs
\usetikzlibrary{shapes,positioning}  %% Clouds and stars
\usetikzlibrary{matrix} %% for matrix
\usepgfplotslibrary{polar} %% for polar plots
\usepgfplotslibrary{fillbetween} %% to shade area between curves in TikZ
\usetkzobj{all}
\usepackage[makeroom]{cancel} %% for strike outs
%\usepackage{mathtools} %% for pretty underbrace % Breaks Ximera
%\usepackage{multicol}
\usepackage{pgffor} %% required for integral for loops



%% http://tex.stackexchange.com/questions/66490/drawing-a-tikz-arc-specifying-the-center
%% Draws beach ball
\tikzset{pics/carc/.style args={#1:#2:#3}{code={\draw[pic actions] (#1:#3) arc(#1:#2:#3);}}}



\usepackage{array}
\setlength{\extrarowheight}{+.1cm}
\newdimen\digitwidth
\settowidth\digitwidth{9}
\def\divrule#1#2{
\noalign{\moveright#1\digitwidth
\vbox{\hrule width#2\digitwidth}}}





\newcommand{\RR}{\mathbb R}
\newcommand{\R}{\mathbb R}
\newcommand{\N}{\mathbb N}
\newcommand{\Z}{\mathbb Z}

\newcommand{\sagemath}{\textsf{SageMath}}


%\renewcommand{\d}{\,d\!}
\renewcommand{\d}{\mathop{}\!d}
\newcommand{\dd}[2][]{\frac{\d #1}{\d #2}}
\newcommand{\pp}[2][]{\frac{\partial #1}{\partial #2}}
\renewcommand{\l}{\ell}
\newcommand{\ddx}{\frac{d}{\d x}}

\newcommand{\zeroOverZero}{\ensuremath{\boldsymbol{\tfrac{0}{0}}}}
\newcommand{\inftyOverInfty}{\ensuremath{\boldsymbol{\tfrac{\infty}{\infty}}}}
\newcommand{\zeroOverInfty}{\ensuremath{\boldsymbol{\tfrac{0}{\infty}}}}
\newcommand{\zeroTimesInfty}{\ensuremath{\small\boldsymbol{0\cdot \infty}}}
\newcommand{\inftyMinusInfty}{\ensuremath{\small\boldsymbol{\infty - \infty}}}
\newcommand{\oneToInfty}{\ensuremath{\boldsymbol{1^\infty}}}
\newcommand{\zeroToZero}{\ensuremath{\boldsymbol{0^0}}}
\newcommand{\inftyToZero}{\ensuremath{\boldsymbol{\infty^0}}}



\newcommand{\numOverZero}{\ensuremath{\boldsymbol{\tfrac{\#}{0}}}}
\newcommand{\dfn}{\textbf}
%\newcommand{\unit}{\,\mathrm}
\newcommand{\unit}{\mathop{}\!\mathrm}
\newcommand{\eval}[1]{\bigg[ #1 \bigg]}
\newcommand{\seq}[1]{\left( #1 \right)}
\renewcommand{\epsilon}{\varepsilon}
\renewcommand{\phi}{\varphi}


\renewcommand{\iff}{\Leftrightarrow}

\DeclareMathOperator{\arccot}{arccot}
\DeclareMathOperator{\arcsec}{arcsec}
\DeclareMathOperator{\arccsc}{arccsc}
\DeclareMathOperator{\si}{Si}
\DeclareMathOperator{\scal}{scal}
\DeclareMathOperator{\sign}{sign}


%% \newcommand{\tightoverset}[2]{% for arrow vec
%%   \mathop{#2}\limits^{\vbox to -.5ex{\kern-0.75ex\hbox{$#1$}\vss}}}
\newcommand{\arrowvec}[1]{{\overset{\rightharpoonup}{#1}}}
%\renewcommand{\vec}[1]{\arrowvec{\mathbf{#1}}}
\renewcommand{\vec}[1]{{\overset{\boldsymbol{\rightharpoonup}}{\mathbf{#1}}}\hspace{0in}}

\newcommand{\point}[1]{\left(#1\right)} %this allows \vector{ to be changed to \vector{ with a quick find and replace
\newcommand{\pt}[1]{\mathbf{#1}} %this allows \vec{ to be changed to \vec{ with a quick find and replace
\newcommand{\Lim}[2]{\lim_{\point{#1} \to \point{#2}}} %Bart, I changed this to point since I want to use it.  It runs through both of the exercise and exerciseE files in limits section, which is why it was in each document to start with.

\DeclareMathOperator{\proj}{\mathbf{proj}}
\newcommand{\veci}{{\boldsymbol{\hat{\imath}}}}
\newcommand{\vecj}{{\boldsymbol{\hat{\jmath}}}}
\newcommand{\veck}{{\boldsymbol{\hat{k}}}}
\newcommand{\vecl}{\vec{\boldsymbol{\l}}}
\newcommand{\uvec}[1]{\mathbf{\hat{#1}}}
\newcommand{\utan}{\mathbf{\hat{t}}}
\newcommand{\unormal}{\mathbf{\hat{n}}}
\newcommand{\ubinormal}{\mathbf{\hat{b}}}

\newcommand{\dotp}{\bullet}
\newcommand{\cross}{\boldsymbol\times}
\newcommand{\grad}{\boldsymbol\nabla}
\newcommand{\divergence}{\grad\dotp}
\newcommand{\curl}{\grad\cross}
%\DeclareMathOperator{\divergence}{divergence}
%\DeclareMathOperator{\curl}[1]{\grad\cross #1}
\newcommand{\lto}{\mathop{\longrightarrow\,}\limits}

\renewcommand{\bar}{\overline}

\colorlet{textColor}{black}
\colorlet{background}{white}
\colorlet{penColor}{blue!50!black} % Color of a curve in a plot
\colorlet{penColor2}{red!50!black}% Color of a curve in a plot
\colorlet{penColor3}{red!50!blue} % Color of a curve in a plot
\colorlet{penColor4}{green!50!black} % Color of a curve in a plot
\colorlet{penColor5}{orange!80!black} % Color of a curve in a plot
\colorlet{penColor6}{yellow!70!black} % Color of a curve in a plot
\colorlet{fill1}{penColor!20} % Color of fill in a plot
\colorlet{fill2}{penColor2!20} % Color of fill in a plot
\colorlet{fillp}{fill1} % Color of positive area
\colorlet{filln}{penColor2!20} % Color of negative area
\colorlet{fill3}{penColor3!20} % Fill
\colorlet{fill4}{penColor4!20} % Fill
\colorlet{fill5}{penColor5!20} % Fill
\colorlet{gridColor}{gray!50} % Color of grid in a plot

\newcommand{\surfaceColor}{violet}
\newcommand{\surfaceColorTwo}{redyellow}
\newcommand{\sliceColor}{greenyellow}




\pgfmathdeclarefunction{gauss}{2}{% gives gaussian
  \pgfmathparse{1/(#2*sqrt(2*pi))*exp(-((x-#1)^2)/(2*#2^2))}%
}


%%%%%%%%%%%%%
%% Vectors
%%%%%%%%%%%%%

%% Simple horiz vectors
\renewcommand{\vector}[1]{\left\langle #1\right\rangle}


%% %% Complex Horiz Vectors with angle brackets
%% \makeatletter
%% \renewcommand{\vector}[2][ , ]{\left\langle%
%%   \def\nextitem{\def\nextitem{#1}}%
%%   \@for \el:=#2\do{\nextitem\el}\right\rangle%
%% }
%% \makeatother

%% %% Vertical Vectors
%% \def\vector#1{\begin{bmatrix}\vecListA#1,,\end{bmatrix}}
%% \def\vecListA#1,{\if,#1,\else #1\cr \expandafter \vecListA \fi}

%%%%%%%%%%%%%
%% End of vectors
%%%%%%%%%%%%%

%\newcommand{\fullwidth}{}
%\newcommand{\normalwidth}{}



%% makes a snazzy t-chart for evaluating functions
%\newenvironment{tchart}{\rowcolors{2}{}{background!90!textColor}\array}{\endarray}

%%This is to help with formatting on future title pages.
\newenvironment{sectionOutcomes}{}{}



%% Flowchart stuff
%\tikzstyle{startstop} = [rectangle, rounded corners, minimum width=3cm, minimum height=1cm,text centered, draw=black]
%\tikzstyle{question} = [rectangle, minimum width=3cm, minimum height=1cm, text centered, draw=black]
%\tikzstyle{decision} = [trapezium, trapezium left angle=70, trapezium right angle=110, minimum width=3cm, minimum height=1cm, text centered, draw=black]
%\tikzstyle{question} = [rectangle, rounded corners, minimum width=3cm, minimum height=1cm,text centered, draw=black]
%\tikzstyle{process} = [rectangle, minimum width=3cm, minimum height=1cm, text centered, draw=black]
%\tikzstyle{decision} = [trapezium, trapezium left angle=70, trapezium right angle=110, minimum width=3cm, minimum height=1cm, text centered, draw=black]


\title[Dig-In:]{Basic antiderivatives}

\begin{document}
\begin{abstract}
  We introduce antiderivatives.
\end{abstract}
\maketitle


Computing derivatives is not too difficult. At this point, you should
be able to take the derivative of almost any function you can write
down. However, undoing derivatives is much harder. This process of
undoing a derivative is called taking an \textit{antiderivative}.

\begin{definition}
A function $F$ is called an \dfn{antiderivative} of $f$ on an
interval if
\[
F'(x) = f(x)
\]
for all $x$ in the interval.
\end{definition}

%% \begin{question} BADBAD
%%   Give three antiderivatives of $f(x) = 2x$.  
%%   \[F_1(x) = \answer[antiderivative]{2x} \]
%%   \[F_2(x) = \answer[antiderivative]{2x} \]
%%   \[F_3(x) = \answer[antiderivative]{2x} \]
%% \end{question}

\begin{question}
  How many antiderivatives does $f(x) = 2x$ have?
  \begin{multipleChoice}
    \choice{none}
    \choice{one}
    \choice[correct]{infinitely many}
  \end{multipleChoice}
  \begin{feedback}
    The functions $x^2$, $x^2+1$, $x^2-5$, and so on, are all
    antiderivatives of $2x$.
  \end{feedback}
\end{question}

There are two common ways to notate antiderivatives, either with a
capital letter or with a funny symbol:

\begin{definition}\index{antiderivative!notation}\index{indefinite integral}
The antiderivative is denoted by
\[
\int f(x) \d x = F(x)+C,
\]
where $dx$ identifies $x$ as the variable and $C$ is a constant
indicating that there are many possible antiderivatives, each varying by
the addition of a constant.  This is often called the
\dfn{indefinite integral}.
\end{definition}

Fill out these basic antiderivatives. Note each of these examples comes
directly from our knowledge of basic derivatives.

%% BADBAD - we want this back.
%% \begin{theorem}[Basic Antiderivatives]\label{theorem:basicAnti} \hfil
%% %\begin{multicols}{3}
%% %\todo{Originally this was done with the multicols environment}
%% \begin{question}
%% \begin{itemize}
%% \item An antiderivative of  $k$ is $\answer[antiderivative]{k}$
%% \item An antiderivative of $x^n$, for $n \neq -1$,  is $\answer[antiderivative]{x^n}$
%% \item An antiderivative of  $e^x$ is $\answer[antiderivative]{e^x}$
%% % Is this really needed? \item $\int a^x \d x= \frac{a^x}{\ln(a)}+C$.
%% \item An antiderivative of  $\frac{1}{x}$ is $\answer[antiderivative]{1/x}$. (Be careful that your answer makes works on the full domain, not just $(0,\infty)$)
%% \item An antiderivative of $\cos(x)$ is  $\answer[antiderivative]{\cos(x)}$.
%% \item An antiderivative of $\sin(x)$ is $\answer[antiderivative]{\sin(x)}$.  
%% \item An antiderivative of $\tan(x)$ is $\answer[antiderivative]{\tan(x)}$.  
%% \item An antiderivative of $\sec^2(x)$ is $\answer[antiderivative]{\sec^2(x)}$. 
%% \item An antiderivative of $\csc^2(x)$ is $\answer[antiderivative]{\csc^2(x)}$.
%% \item An antiderivative of $\sec(x)\tan(x)$ is $\answer[antiderivative]{\sec(x)\tan(x)}$.
%% \item An antiderivative of $\csc(x)\cot(x)$ is $\answer[antiderivative]{\csc(x)\cot(x)}$.
%% \item An antiderivative of $\frac{1}{x^2+1}$ is $\answer[antiderivative]{\frac{1}{1+x^2}}$.
%% \item An antiderivative of $\frac{1}{\sqrt{1-x^2}}$ is $\answer[antiderivative]{\frac{1}{\sqrt{1-x^2}}}$.
%% \end{itemize}
%% \end{question}
%% \end{theorem}
\begin{theorem}[Basic Antiderivatives]\label{theorem:basicAnti} \hfil
%\begin{multicols}{3}
\begin{itemize}
\item $\int k \d x= kx+C$
\item $\int x^n \d x= \frac{x^{n+1}}{n+1}+C\qquad(n\ne-1)$
\item $\int e^x \d x= e^x + C$
\item $\int a^x \d x= \frac{a^x}{\ln(a)}+C$
\item $\int \frac{1}{x} \d x= \ln|x|+C$
\item $\int \cos(x) \d x = \sin(x) + C$
\item $\int \sin(x) \d x = -\cos(x) + C$  
\item $\int \tan(x) \d x = -\ln|\cos(x)| + C$  
\item $\int \sec^2(x) \d x = \tan(x) + C$ 
\item $\int \csc^2(x) \d x = -\cot(x) + C$
\item $\int \sec(x)\tan(x) \d x = \sec(x) + C$
\item $\int \csc(x)\cot(x) \d x = -\csc(x) + C$
\item $\int \frac{1}{x^2+1}\d x = \arctan(x) + C$
\item $\int \frac{1}{\sqrt{1-x^2}}\d x= \arcsin(x)+C$
\end{itemize}
%\end{multicols}
\end{theorem}

It may seem that one could simply memorize these antiderivatives and
antidifferentiating would be as easy as differentiating. This is
\textbf{not} the case. The issue comes up when trying to combine these
functions.  When taking derivatives we have the \textit{product rule}
and the \textit{chain rule}. The analogues of these two rules are much
more difficult to deal with when taking antiderivatives. However, not
all is lost. We have the following analogue of the sum rule for
derivatives and the constant factor rule.

\begin{theorem}[The Sum Rule for Antiderivatives]\label{theorem:SRA}
If $F$ is an antiderivative of $f$ and $G$ is an antiderivative of
$g$, then $F+G$ is an antiderivative of $f+g$.
\end{theorem}

\begin{theorem}[The Constant Factor Rule for Antiderivatives]\label{theorem:CFRA}
If $F$ is an antiderivative of $f$, and $k$ is a constant, then $kF$
is an antiderivative of $kf$.
\end{theorem}

Let's put these rules and our knowledge of basic derivatives to work.

\begin{example}
Find the antiderivative of $3 x^7$.
\begin{explanation}
By the theorems above , we see that
\begin{align*}
\int 3 x^7 \d x &= 3 \int x^7 \d x\\
&= 3 \cdot \answer[given]{\frac{x^8}{8}}+C.
\end{align*}
\end{explanation}
\end{example}

The sum rule for antiderivatives allows us to integrate
term-by-term. Let's see an example of this.

\begin{example}
Compute
\[
\int \left(x^4 + 5x^2 - \cos(x)\right) \d x.
\]
\begin{explanation}
Let's start by simplifying the problem using the sum rule for
antiderivatives, 
\begin{align*}
\int &\left(x^4 + 5x^2 - \cos(x)\right) \d x\\
&= \int x^4 \d x + 5\int x^2 \d x - \int \cos(x) \d x.
\end{align*}
Now we may integrate term-by-term to find
\[
= \answer[given]{\frac{x^5}{5} + \frac{5x^3}{3}  - \sin(x)}+C.
\]
\end{explanation}
\end{example}


\begin{warning}
While the sum rule for antiderivatives allows us to integrate
term-by-term, we cannot integrate \textit{factor-by-factor}, meaning
that in general
\[
\int f(x)g(x) \d x \ne \int f(x) \d x\cdot \int g(x) \d x.
\]
\end{warning}








\section{Computing antiderivatives}


Unfortunately, we cannot tell you how to compute every antiderivative.
We advise that the mathematician view antiderivatives as a sort of
\textit{puzzle}. Later we will learn a hand-full of techniques for
computing antiderivatives. However, a robust and simple way to compute
antiderivatives is guess-and-check.


%\begin{guessingAntiderivatives}\hfil
\paragraph{Tips for guessing antiderivatives}
\begin{enumerate}
\item Make a guess for the antiderivative.
\item Take the derivative of your guess.
\item Note how the above derivative is different from the function
  whose antiderivative you want to find.
\item Change your original guess by \textbf{multiplying} by constants
  or by \textbf{adding} in new functions.
\end{enumerate}
%\end{guessingAntiderivatives}

\begin{template}\label{template:powerchain}
If the indefinite integral looks \emph{something} like
\[
\int \mathrm{stuff}' \cdot (\mathrm{stuff})^n \d x
\]
guess
\[
\mathrm{stuff}^{n+1}
\]
where $n\ne -1$.
\end{template}

\begin{example} Compute
\[
\int \frac{x^3}{\sqrt{x^4 -6}} \d x.
\]
\begin{explanation}
  Start by rewriting the indefinite integral as
  \[
  \int x^3\left(x^4 -6\right)^{-1/2} \d x.
  \]
  Now start with a guess of 
  \[
  \int x^3\left(x^4 -6\right)^{-1/2} \d x \approx \left(x^4 -6\right)^{1/2}.
  \]
  Take the derivative of your guess to see if it is correct:
  \[
  \ddx  \left(x^4 -6\right)^{1/2} = (4/2)x^3\left(x^4 -6\right)^{-1/2}.
  \]
  We're off by a factor of $2/4$, so multiply our guess by this constant
  to get the solution,
  \[
  \int \frac{x^3}{\sqrt{x^4 -6}} \d x = \answer[given]{(2/4)(x^4 -6)^{1/2}}+C.
  \]
\end{explanation}
\end{example}


\begin{template}\label{template:echain}
If the indefinite integral looks \emph{something} like
\[
\int \mathrm{junk}\cdot e^{\mathrm{stuff}} \d x
\]
guess
\[
e^{\mathrm{stuff}} \text{ or }\mathrm{junk} \cdot e^{\mathrm{stuff}}.
\]
\end{template}


\begin{example}
Compute
\[
\int xe^{x} \d x.
\]
\begin{explanation}
We try to guess the antiderivative. Start with a guess of
\[
\int xe^x \d x \approx xe^x.
\]
Take the derivative of your guess to see if it is correct:
\[
\ddx xe^x = e^x + xe^x.
\]
Ah! So we need only subtract $e^x$ from our original guess.  We now
find
\[
\int xe^x \d x =\answer[given]{xe^x - e^x} + C.
\]
\end{explanation}
\end{example}





\begin{template}\label{template:lnchain}
If the indefinite integral looks \emph{something} like
\[
\int \frac{\mathrm{stuff}'}{\mathrm{stuff}}\d x
\]
guess
\[
\ln(\mathrm{stuff}).
\]
\end{template}

\begin{example}
Compute
\[
\int \frac{2x^2}{7x^3 + 3} \d x.
\]
\begin{explanation}
We'll start with a guess of
\[
\int \frac{2x^2}{7x^3 + 3} \d x \approx \ln(7x^3+3).
\]
Take the derivative of your guess to see if it is correct:
\[
\ddx \ln(7x^3+3) = \frac{21x^2}{7x^3 + 3}.
\]
We are only off by a factor of $2/21$, so we need to multiply our
original guess by this constant to get the solution,
\[
\int \frac{2x^2}{7x^3 + 3} \d x = \answer[given]{(2/21)\ln(7x^3+3)}+C.
\]
\end{explanation}
\end{example}




\begin{template}\label{template:trigchain}
If the indefinite integral looks \emph{something} like
\[
\int \mathrm{junk}\cdot \sin(\mathrm{stuff}) \d x
\]
guess
\[
\cos(\mathrm{stuff}) \text{ or }\mathrm{junk}\cdot \cos(\mathrm{stuff}),
\]
likewise if you have 
\[
\int \mathrm{junk}\cdot \cos(\mathrm{stuff}) \d x
\]
guess
\[
\sin(\mathrm{stuff}) \text{ or }\mathrm{junk}
\cdot \sin(\mathrm{stuff}).
\]
\end{template}



\begin{example}
Compute
\[
\int x^4\sin(3x^5+7) \d x.
\]
\begin{explanation}
Here we simply try to guess the antiderivative. Start with a guess of
\[
\int x^4\sin(3x^5+7)\d x \approx \cos(3x^5+7).
\]
To see if your guess is correct, take the derivative of $\cos(3x^5+7)$,
\[
\ddx \cos(3x^5+7) = -15x^4\sin(3x^5+7).
\]
We are off by a factor of $-1/15$. Hence we should multiply our
original guess by this constant to find
\[
\int x^4\sin(3x^5+7) \d x = \answer[given]{\frac{-\cos(3x^5+7)}{15}} + C.
\]
\end{explanation}
\end{example}





\section{Final thoughts}
Computing antiderivatives is a place where insight and rote
computation meet. We cannot teach you a method that will always
work. Moreover, merely \emph{understanding} the examples above will
probably not be enough for you to become proficient in computing
antiderivatives. You must practice, practice, practice!




\end{document}
