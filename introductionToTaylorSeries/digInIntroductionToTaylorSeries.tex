\documentclass{ximera}

%\usepackage{todonotes}
%\usepackage{mathtools} %% Required for wide table Curl and Greens
%\usepackage{cuted} %% Required for wide table Curl and Greens
\newcommand{\todo}{}

\usepackage{esint} % for \oiint
\ifxake%%https://math.meta.stackexchange.com/questions/9973/how-do-you-render-a-closed-surface-double-integral
\renewcommand{\oiint}{{\large\bigcirc}\kern-1.56em\iint}
\fi


\graphicspath{
  {./}
  {ximeraTutorial/}
  {basicPhilosophy/}
  {functionsOfSeveralVariables/}
  {normalVectors/}
  {lagrangeMultipliers/}
  {vectorFields/}
  {greensTheorem/}
  {shapeOfThingsToCome/}
  {dotProducts/}
  {partialDerivativesAndTheGradientVector/}
  {../productAndQuotientRules/exercises/}
  {../normalVectors/exercisesParametricPlots/}
  {../continuityOfFunctionsOfSeveralVariables/exercises/}
  {../partialDerivativesAndTheGradientVector/exercises/}
  {../directionalDerivativeAndChainRule/exercises/}
  {../commonCoordinates/exercisesCylindricalCoordinates/}
  {../commonCoordinates/exercisesSphericalCoordinates/}
  {../greensTheorem/exercisesCurlAndLineIntegrals/}
  {../greensTheorem/exercisesDivergenceAndLineIntegrals/}
  {../shapeOfThingsToCome/exercisesDivergenceTheorem/}
  {../greensTheorem/}
  {../shapeOfThingsToCome/}
  {../separableDifferentialEquations/exercises/}
  {vectorFields/}
}

\newcommand{\mooculus}{\textsf{\textbf{MOOC}\textnormal{\textsf{ULUS}}}}

\usepackage{tkz-euclide}\usepackage{tikz}
\usepackage{tikz-cd}
\usetikzlibrary{arrows}
\tikzset{>=stealth,commutative diagrams/.cd,
  arrow style=tikz,diagrams={>=stealth}} %% cool arrow head
\tikzset{shorten <>/.style={ shorten >=#1, shorten <=#1 } } %% allows shorter vectors

\usetikzlibrary{backgrounds} %% for boxes around graphs
\usetikzlibrary{shapes,positioning}  %% Clouds and stars
\usetikzlibrary{matrix} %% for matrix
\usepgfplotslibrary{polar} %% for polar plots
\usepgfplotslibrary{fillbetween} %% to shade area between curves in TikZ
\usetkzobj{all}
\usepackage[makeroom]{cancel} %% for strike outs
%\usepackage{mathtools} %% for pretty underbrace % Breaks Ximera
%\usepackage{multicol}
\usepackage{pgffor} %% required for integral for loops



%% http://tex.stackexchange.com/questions/66490/drawing-a-tikz-arc-specifying-the-center
%% Draws beach ball
\tikzset{pics/carc/.style args={#1:#2:#3}{code={\draw[pic actions] (#1:#3) arc(#1:#2:#3);}}}



\usepackage{array}
\setlength{\extrarowheight}{+.1cm}
\newdimen\digitwidth
\settowidth\digitwidth{9}
\def\divrule#1#2{
\noalign{\moveright#1\digitwidth
\vbox{\hrule width#2\digitwidth}}}





\newcommand{\RR}{\mathbb R}
\newcommand{\R}{\mathbb R}
\newcommand{\N}{\mathbb N}
\newcommand{\Z}{\mathbb Z}

\newcommand{\sagemath}{\textsf{SageMath}}


%\renewcommand{\d}{\,d\!}
\renewcommand{\d}{\mathop{}\!d}
\newcommand{\dd}[2][]{\frac{\d #1}{\d #2}}
\newcommand{\pp}[2][]{\frac{\partial #1}{\partial #2}}
\renewcommand{\l}{\ell}
\newcommand{\ddx}{\frac{d}{\d x}}

\newcommand{\zeroOverZero}{\ensuremath{\boldsymbol{\tfrac{0}{0}}}}
\newcommand{\inftyOverInfty}{\ensuremath{\boldsymbol{\tfrac{\infty}{\infty}}}}
\newcommand{\zeroOverInfty}{\ensuremath{\boldsymbol{\tfrac{0}{\infty}}}}
\newcommand{\zeroTimesInfty}{\ensuremath{\small\boldsymbol{0\cdot \infty}}}
\newcommand{\inftyMinusInfty}{\ensuremath{\small\boldsymbol{\infty - \infty}}}
\newcommand{\oneToInfty}{\ensuremath{\boldsymbol{1^\infty}}}
\newcommand{\zeroToZero}{\ensuremath{\boldsymbol{0^0}}}
\newcommand{\inftyToZero}{\ensuremath{\boldsymbol{\infty^0}}}



\newcommand{\numOverZero}{\ensuremath{\boldsymbol{\tfrac{\#}{0}}}}
\newcommand{\dfn}{\textbf}
%\newcommand{\unit}{\,\mathrm}
\newcommand{\unit}{\mathop{}\!\mathrm}
\newcommand{\eval}[1]{\bigg[ #1 \bigg]}
\newcommand{\seq}[1]{\left( #1 \right)}
\renewcommand{\epsilon}{\varepsilon}
\renewcommand{\phi}{\varphi}


\renewcommand{\iff}{\Leftrightarrow}

\DeclareMathOperator{\arccot}{arccot}
\DeclareMathOperator{\arcsec}{arcsec}
\DeclareMathOperator{\arccsc}{arccsc}
\DeclareMathOperator{\si}{Si}
\DeclareMathOperator{\scal}{scal}
\DeclareMathOperator{\sign}{sign}


%% \newcommand{\tightoverset}[2]{% for arrow vec
%%   \mathop{#2}\limits^{\vbox to -.5ex{\kern-0.75ex\hbox{$#1$}\vss}}}
\newcommand{\arrowvec}[1]{{\overset{\rightharpoonup}{#1}}}
%\renewcommand{\vec}[1]{\arrowvec{\mathbf{#1}}}
\renewcommand{\vec}[1]{{\overset{\boldsymbol{\rightharpoonup}}{\mathbf{#1}}}\hspace{0in}}

\newcommand{\point}[1]{\left(#1\right)} %this allows \vector{ to be changed to \vector{ with a quick find and replace
\newcommand{\pt}[1]{\mathbf{#1}} %this allows \vec{ to be changed to \vec{ with a quick find and replace
\newcommand{\Lim}[2]{\lim_{\point{#1} \to \point{#2}}} %Bart, I changed this to point since I want to use it.  It runs through both of the exercise and exerciseE files in limits section, which is why it was in each document to start with.

\DeclareMathOperator{\proj}{\mathbf{proj}}
\newcommand{\veci}{{\boldsymbol{\hat{\imath}}}}
\newcommand{\vecj}{{\boldsymbol{\hat{\jmath}}}}
\newcommand{\veck}{{\boldsymbol{\hat{k}}}}
\newcommand{\vecl}{\vec{\boldsymbol{\l}}}
\newcommand{\uvec}[1]{\mathbf{\hat{#1}}}
\newcommand{\utan}{\mathbf{\hat{t}}}
\newcommand{\unormal}{\mathbf{\hat{n}}}
\newcommand{\ubinormal}{\mathbf{\hat{b}}}

\newcommand{\dotp}{\bullet}
\newcommand{\cross}{\boldsymbol\times}
\newcommand{\grad}{\boldsymbol\nabla}
\newcommand{\divergence}{\grad\dotp}
\newcommand{\curl}{\grad\cross}
%\DeclareMathOperator{\divergence}{divergence}
%\DeclareMathOperator{\curl}[1]{\grad\cross #1}
\newcommand{\lto}{\mathop{\longrightarrow\,}\limits}

\renewcommand{\bar}{\overline}

\colorlet{textColor}{black}
\colorlet{background}{white}
\colorlet{penColor}{blue!50!black} % Color of a curve in a plot
\colorlet{penColor2}{red!50!black}% Color of a curve in a plot
\colorlet{penColor3}{red!50!blue} % Color of a curve in a plot
\colorlet{penColor4}{green!50!black} % Color of a curve in a plot
\colorlet{penColor5}{orange!80!black} % Color of a curve in a plot
\colorlet{penColor6}{yellow!70!black} % Color of a curve in a plot
\colorlet{fill1}{penColor!20} % Color of fill in a plot
\colorlet{fill2}{penColor2!20} % Color of fill in a plot
\colorlet{fillp}{fill1} % Color of positive area
\colorlet{filln}{penColor2!20} % Color of negative area
\colorlet{fill3}{penColor3!20} % Fill
\colorlet{fill4}{penColor4!20} % Fill
\colorlet{fill5}{penColor5!20} % Fill
\colorlet{gridColor}{gray!50} % Color of grid in a plot

\newcommand{\surfaceColor}{violet}
\newcommand{\surfaceColorTwo}{redyellow}
\newcommand{\sliceColor}{greenyellow}




\pgfmathdeclarefunction{gauss}{2}{% gives gaussian
  \pgfmathparse{1/(#2*sqrt(2*pi))*exp(-((x-#1)^2)/(2*#2^2))}%
}


%%%%%%%%%%%%%
%% Vectors
%%%%%%%%%%%%%

%% Simple horiz vectors
\renewcommand{\vector}[1]{\left\langle #1\right\rangle}


%% %% Complex Horiz Vectors with angle brackets
%% \makeatletter
%% \renewcommand{\vector}[2][ , ]{\left\langle%
%%   \def\nextitem{\def\nextitem{#1}}%
%%   \@for \el:=#2\do{\nextitem\el}\right\rangle%
%% }
%% \makeatother

%% %% Vertical Vectors
%% \def\vector#1{\begin{bmatrix}\vecListA#1,,\end{bmatrix}}
%% \def\vecListA#1,{\if,#1,\else #1\cr \expandafter \vecListA \fi}

%%%%%%%%%%%%%
%% End of vectors
%%%%%%%%%%%%%

%\newcommand{\fullwidth}{}
%\newcommand{\normalwidth}{}



%% makes a snazzy t-chart for evaluating functions
%\newenvironment{tchart}{\rowcolors{2}{}{background!90!textColor}\array}{\endarray}

%%This is to help with formatting on future title pages.
\newenvironment{sectionOutcomes}{}{}



%% Flowchart stuff
%\tikzstyle{startstop} = [rectangle, rounded corners, minimum width=3cm, minimum height=1cm,text centered, draw=black]
%\tikzstyle{question} = [rectangle, minimum width=3cm, minimum height=1cm, text centered, draw=black]
%\tikzstyle{decision} = [trapezium, trapezium left angle=70, trapezium right angle=110, minimum width=3cm, minimum height=1cm, text centered, draw=black]
%\tikzstyle{question} = [rectangle, rounded corners, minimum width=3cm, minimum height=1cm,text centered, draw=black]
%\tikzstyle{process} = [rectangle, minimum width=3cm, minimum height=1cm, text centered, draw=black]
%\tikzstyle{decision} = [trapezium, trapezium left angle=70, trapezium right angle=110, minimum width=3cm, minimum height=1cm, text centered, draw=black]


\outcome{Give the definition of a Maclaurin series.}
\outcome{Give the definition of a Taylor series.}
\outcome{Find the Maclaurin/Taylor series of a function.}
\outcome{Use given Maclaurin/Taylor series to find new power series.}
%\outcome{Use the binomial series to find Maclaurin series.}
\outcome{Find the interval and radius of convergence of a Maclaurin/Taylor series.}

\title[Dig-In:]{Introduction to Taylor series}

\begin{document}
\begin{abstract}
  We study Taylor and Maclaurin series.
\end{abstract}
\maketitle

We've seen that we can approximate functions with polynomials, given
that enough derivative information is available.  We have also seen
that certain functions can be represented by a power series.  In this
section we combine these concepts: If a function $f(x)$ is infinitely
differentiable, we show how to represent it with a power series
function.

\begin{definition}
  Let $f(x)$ have derivatives of all orders at $x=c$.  
  \begin{itemize}
  \item The \dfn{Taylor series} of $f(x)$, centered at $c$ is
    \[
    \sum_{n=0}^\infty \frac{f^{(n)}(c)}{n!}(x-c)^n.
    \]
  \item Setting $c=0$ gives the \dfn{Maclaurin series} of $f(x)$:
    \[
    \sum_{n=0}^\infty \frac{f^{(n)}(0)}{n!}x^n.
    \]
  \end{itemize}
\end{definition}

\begin{question}
  Quick: Write down the Taylor series for $f(x) = x^3-6x^2+1$ centered
  at $x=0$.
  \begin{prompt}
    \[
    \answer{x^3-6x^2+1}
    \]
  \end{prompt}
  \begin{question}
    Write down the Taylor series for $f(x) = x^3-6x^2+1$ centered at
    $x=1$.
    \begin{prompt}-
      \[
      \answer{-4} + \answer{-9}(x-1) + \answer{-3}(x-1)^2 + \answer{1}(x-1)^3
      \]
    \end{prompt}
  \end{question}
\end{question}

The difference between a Taylor polynomial and a Taylor series is the
former is a polynomial, containing only a \textbf{finite} number of
terms, whereas the latter is a series, a summation of an infinite set
of terms, any number of which (including an infinite number) may be
zero. When creating the Taylor polynomial of degree $n$ for a function
$f(x)$ at $x=c$, we needed to evaluate $f$, and the first $n$
derivatives of $f$, at $x=c$. When creating the Taylor series of $f$,
it helps to find a pattern that describes the $n$th derivative of $f$
at $x=c$.  Time for examples!

\begin{example}
  Compute the Taylor series for $f(x) = \sin(x)$ centered at $x=0$.
  \begin{explanation}
    We'll start by making a table of derivatives:
    \[
    \begin{array}{lcl}
      f(x) = \sin(x) & \Rightarrow &f(0) = 0\\
      f'(x) = \answer[given]{\cos(x)} & \Rightarrow & f'(0) = \answer[given]{1}\\
      f''(x) = \answer[given]{-\sin(x)} &\Rightarrow &f''(0) = \answer[given]{0}\\
      f'''(x) = \answer[given]{-\cos(x)} &\Rightarrow &f'''(0) = \answer[given]{-1}\\
      f^{(4)}(x) = \answer[given]{\sin(x)} &\Rightarrow &f^{(4)}(0) = \answer[given]{0}\\
      f^{(5)}(x) = \answer[given]{\cos(x)} &\Rightarrow &f^{(5)}(0) = \answer[given]{1}\\
      f^{(6)}(x) = \answer[given]{-\sin(x)} &\Rightarrow &f^{(6)}(0) =\answer[given]{0}\\
      f^{(7)}(x) = \answer[given]{-\cos(x)} &\Rightarrow &f^{(7)}(0) = \answer[given]{-1}\\
      f^{(8)}(x) = \answer[given]{\sin(x)} &\Rightarrow &f^{(8)}(0) = \answer[given]{0}\\
      f^{(9)}(x) = \answer[given]{\cos(x)} &\Rightarrow &f^{(9)}(0) = \answer[given]{1}\\
    \end{array}
    \]
    Since a repeating pattern has emerged, we see that the Maclaurin
    series for $\sin(x)$ is:
    \[
    x-\frac{x^3}{3!}+\frac{x^5}{5!}-\frac{x^7}{7!}+\cdots = \sum_{n=0}^\infty \frac{(-1)^{n+1}}{(2n+1)!} x^{2n+1}
    \]
  \end{explanation}
\end{example}


Let's see an example that is not centered at $x=0$:

\begin{example}
  Compute the Taylor series for $f(x) = \ln(x)$ centered at $x=1$.
  \begin{explanation}
    We'll start by making a table of derivatives:
    \[
    \begin{array}{lcl}
      f(x) = \ln(x) &\Rightarrow &f(1) = 0\\
      f'(x) = \answer[given]{1/x}&\Rightarrow &f'(1) = \answer[given]{1}\\
      f''(x) = \answer[given]{-1/x^2}&\Rightarrow &f''(1) = \answer[given]{-1}\\
      f'''(x) = \answer[given]{2/x^3}&\Rightarrow &f'''(1) = \answer[given]{2}\\
      f^{(4)}(x) = \answer[given]{-6/x^4}&\Rightarrow &f^{(4)}(1) = \answer[given]{-6}\\
      f^{(5)}(x) = \answer[given]{24/x^5}&\Rightarrow &f^{(5)}(1) = \answer[given]{24}\\
      \ \vdots & &\ \vdots\\
      f^{(n)}(x) = \answer[given]{\frac{(-1)^{n+1}(n-1)!}{x^n}} &\Rightarrow & f^{(n)}(1) = \answer[given]{(-1)^{n+1}(n-1)!}
    \end{array}
    \]
    \begin{hint}
      \[
      0! = 1,\ 1! = 1,\ 2!=2,\ 3! = 6,\ 4! = 24, \ 5!=120, \ 6! = 720.
      \]
    \end{hint}
    Since a pattern has emerged, we see that the Taylor series for
    $\ln(x)$ is:
    \begin{align*}
    (x-1) &- \frac{(x-1)^2}{\answer[given]{2}} + \frac{(x-1)^3}{\answer[given]{3}} - \frac{(x-1)^4}{\answer[given]{4}} + \cdots \\
    &= \sum_{n=1}^\infty (-1)^{n+1}\frac{(x-1)^n}{\answer[given]{n}}
    \end{align*}
  \end{explanation}
\end{example}

Finally, sometimes Taylor's formula may not be the best way to compute the
Taylor series.

\begin{example}
  Compute the Taylor series for $f(x) = \arctan(x)$ centered at $x=0$.
  \begin{explanation}
    If we try to use Taylor's formula, we must start by making a table
    of derivatives:
    \begin{align*}
      f(x) &= \arctan(x)\\
      f'(x) &= \answer[given]{\frac{1}{1+x^2}}\\
      f''(x) &= \answer[given]{\frac{-2x}{(1+x^2)^2}}
    \end{align*}
    Hmm. This is getting messy. Let's try to find the Taylor series
    via known power series. We know that
    \[
    \frac{1}{1-z} = 1+ z+ z^2 + z^3 + \cdots \qquad |z|< 1
    \]
    setting $z = -x^2$ we now have
    \[
    \frac{1}{1-(-x^2)} = 1 + (-x^2)+ (-x^2)^2 + (-x^2)^3 + \cdots
    \]
    when $|(-x^2)|< 1$ and
    \[
    \frac{1}{1+x^2} = 1 - x^2 + x^4 - x^6 + \cdots 
    \]
    when $|x|< 1$. Since
    \[
    \int\frac{1}{1+x^2} \d x = \arctan(x) + C
    \]
    we can find the desired power series by integrating. Write with me
    \begin{align*}
      \int &\left(1 - x^2 + x^4 - x^6 + \cdots\right)\d x \\
      &= \answer[given]{x - \frac{x^3}{3} + \frac{x^5}{5}} - \frac{x^7}{7} + \cdots +C
    \end{align*}
    since $\arctan(0) = 0$, $C=0$, and we have our desired power
    series, which converges with radius of convergence $R=1$. However,
    note the interval of convergence may be different, and it is in
    this case. First note that our power series can be written in summation notation as
    \[
    \sum_{n=0}^\infty \frac{(-1)^n x^{2n+1}}{2n+1}
    \]
    If $x=1$ or $x=-1$ we can see that this sequence is
    \[
    \sum_{n=0}^\infty \frac{(-1)^n}{2n+1}\qquad\text{or}\qquad\sum_{n=0}^\infty \frac{(-1)^{n+1}}{2n+1}
    \]
    In both cases, the series converges by the alternating series
    test. Hence the interval of convergence is $[\answer[given]{-1},\answer[given]{1}]$.
  \end{explanation}
\end{example}

Above we implicitly used the following theorem:


\begin{theorem}
  The Taylor series centered at $x=c$ of a polynomial in $(x-c)$ is
  exactly that polynomial and the Taylor series of power series
  centered at $(x-c)$ is exactly that power series.
\end{theorem}

This is just saying that if you know a power series for a function,
then using Taylor's formula will do nothing but give you the power
series.

\begin{warning}
  Since we designed Taylor polynomials to approximate functions, you
  might guess that the Taylor series of a function is equal to the
  function (at least on the interval of convergence for the Taylor
  series).  This is \textbf{false}.
\end{warning}

Here is an somewhat unsatisfying example:

\begin{example}
  Consider:
  \[
  f(x) = \begin{cases}
    1 &\text{if $|x|<1$},\\
    0 &\text{if $|x|\ge 1$}.
  \end{cases}
  \]
  \begin{enumerate}
  \item Compute the Maclaurin series of $f$.
  \item Find the radius of convergence.
  \item Is the Maclaurin series for $f$ equal to $f$ on the interval of convergence?
  \end{enumerate}
  \begin{explanation}
   We'll start by making a table of derivatives:
   \begin{align*}
     f(0)   &= \answer[given]{1}\\
     f'(0)  &= \answer[given]{0}\\
     f''(0) &=\answer[given]{0}\\
     &\vdots
   \end{align*}
   So our Maclaurin series for $f$ is:
    \[
    1+0+0+\cdots
    \]
    This converges for all values of $x$, and hence the radius of
    convergence is $R=\infty$, with interval of convergence
    $(-\infty,\infty)$. However, $1\ne f$ as functions.
\end{explanation}
\end{example}


A more satisfying example is the following:

\begin{example}  %% Give a plot?
	Let 
	\[
	f(x) = \begin{cases}
		e^{-1/x^2} &\text{if $x \neq 0$,}\\
		0 &\text{if $x = 0$.}
	\end{cases}
	\]
	It turns out that $f$ is infinitely differentiable everywhere,
        but all of its derivatives vanish at $x=0$. Thus the Maclaurin series for $f$ is just
        \[
        1 + 0 + 0+ \cdots
        \]
        The derivatives of $f$ ``go to $0$'' as $x$ goes to zero
        faster than any polynomial, and so no polynomial term
        ``detects'' that this function is not the horizontal line
        $y=1$.

        It is within your power to show that $f$ is infinitely
        differentiable everywhere, and to prove that $f^{(k)}(0) = 0$.
        This is quite involved, and we will not do it here.  If you
        have the gumption, and the willpower, it would make a
        fantastic exercise.
\end{example}

We will find that ``most of the time'' they are equal, but we need to
consider the conditions that allow us to conclude this.  Taylor's
theorem states that the error between a function $f(x)$ and its $n$th
degree Taylor polynomial is $R_n(x)$:
\[
f(x) = p_n(x) + R_n(x)
\]
and that 
\[
\left|R_n(x)\right| \leq \frac{M}{(n+1)!}|x-c|^{(n+1)}
\]
where $M$ is the maximum value of $|f^{(n+1)}|$ on $[c,x]$.  If
$R_n(x)$ goes to $0$ for each $x$ in an interval $I$ as $n$ approaches
infinity, we conclude that the function is equal to its Taylor series
expansion. This leads us to our next theorem:

\begin{theorem}[Function and Taylor Series Equality]
Let $f(x)$ have derivatives of all orders at $x=c$, let $R_n(x)$, and
let $I$ be an interval on which the Taylor series of $f$ converges.
If
\[
\lim_{n\to\infty} R_n(x) = 0
\]
for all $x$ in $I$ containing $c$, then
\[
f(x) = \sum_{n=0}^\infty \frac{f\,^{(n)}(c)}{n!}(x-c)^n
\]
on $I$.
\end{theorem}

We'll work a representative example of this theorem to see what is
going on, the general case is much the same.

\begin{example}
	The Maclaurin series for $e^x$ is 
	\[
	1+x+\frac{x^2}{2!} +\frac{x^3}{3!}+\dots
	\]
        show that:
        \[
        e^x = 1+x+\frac{x^2}{2!} +\frac{x^3}{3!}+\dots
        \]
	\begin{explanation}
	  To start, note that using the ratio test this power series
          has an infinite radius of convergence.  We want to know
          whether
          \[
          e^x = 1+x+\frac{x^2}{2!} +\frac{x^3}{3!}+\dots
          \]
          for every real number $x$.  Another way of phrasing this is
          that we want the remainder $R_n(x)$ to go to zero as $n$
          goes to infinity. We know that
          \[
          \left|R_n(x)\right| \le \frac{M}{(n+1)!}|x-c|^{(n+1)}
          \]
          where $M$ is the maximum value of $|f^{(n+1)}|$ on $[0,x]$.
          We know $f^{(n+1)}(b) = \answer[given]{e^b}$, since all the
          derivatives of $e^x$ are just $e^x$ Thus
	  \[
	  R_n(x) = \frac{\answer[given]{e^b}}{(n+1)!}x^{n+1}
	  \]
	  So how large can $R_n(x)$ be?  If $b$ is between $0$ and
          $x$, then $e^b$ is between $0$ and $e^{|b|}$, so
	  \[
	  0\le \left| R_n(x)\right| \leq \frac{e^{|b|}}{(n+1)!}|x|^{n+1}
	  \]
	  Thus by the Squeeze Theorem,
	  \[
	  \lim_{n \to \infty}\left| R_n(x)\right| = 0
          \]
          as
          \[
          \lim_{n \to \infty} \frac{e^{|b|}}{(n+1)!}|x|^{n+1} = 0,
	  \]
	  proving that the Maclaurin series for $e^x$ converges to
          $e^x$ everywhere!
        \end{explanation}
\end{example}

There is good news. A function $f$ that is equal to its Taylor series,
centered at any point the domain of $f(x)$, is said to be an
\dfn{analytic function}, and most, if not all, functions that we
encounter within this course are analytic functions. Generally
speaking, any function that one creates with elementary functions
(polynomials, exponentials, trigonometric functions, etc.) that is not
piecewise defined is probably analytic. For most functions, we assume
the function is equal to its Taylor series on the series' interval of
convergence and only check the remainder (as above) when we suspect
something may not work as expected.

%%%% In the future it might be good to add a section talking about
%%%% computing taylor series, versus manipulating power series to
%%%% compute taylor series. It would be cool to use the trig identity
%%%% sin(a+b) = sin(a)cos(b) + cos(a)sin(b) to help us change the
%%%% center of the expansion for the taylor series.







\end{document}
