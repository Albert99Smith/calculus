\documentclass{ximera}

%\usepackage{todonotes}
%\usepackage{mathtools} %% Required for wide table Curl and Greens
%\usepackage{cuted} %% Required for wide table Curl and Greens
\newcommand{\todo}{}

\usepackage{esint} % for \oiint
\ifxake%%https://math.meta.stackexchange.com/questions/9973/how-do-you-render-a-closed-surface-double-integral
\renewcommand{\oiint}{{\large\bigcirc}\kern-1.56em\iint}
\fi


\graphicspath{
  {./}
  {ximeraTutorial/}
  {basicPhilosophy/}
  {functionsOfSeveralVariables/}
  {normalVectors/}
  {lagrangeMultipliers/}
  {vectorFields/}
  {greensTheorem/}
  {shapeOfThingsToCome/}
  {dotProducts/}
  {partialDerivativesAndTheGradientVector/}
  {../productAndQuotientRules/exercises/}
  {../normalVectors/exercisesParametricPlots/}
  {../continuityOfFunctionsOfSeveralVariables/exercises/}
  {../partialDerivativesAndTheGradientVector/exercises/}
  {../directionalDerivativeAndChainRule/exercises/}
  {../commonCoordinates/exercisesCylindricalCoordinates/}
  {../commonCoordinates/exercisesSphericalCoordinates/}
  {../greensTheorem/exercisesCurlAndLineIntegrals/}
  {../greensTheorem/exercisesDivergenceAndLineIntegrals/}
  {../shapeOfThingsToCome/exercisesDivergenceTheorem/}
  {../greensTheorem/}
  {../shapeOfThingsToCome/}
  {../separableDifferentialEquations/exercises/}
  {vectorFields/}
}

\newcommand{\mooculus}{\textsf{\textbf{MOOC}\textnormal{\textsf{ULUS}}}}

\usepackage{tkz-euclide}\usepackage{tikz}
\usepackage{tikz-cd}
\usetikzlibrary{arrows}
\tikzset{>=stealth,commutative diagrams/.cd,
  arrow style=tikz,diagrams={>=stealth}} %% cool arrow head
\tikzset{shorten <>/.style={ shorten >=#1, shorten <=#1 } } %% allows shorter vectors

\usetikzlibrary{backgrounds} %% for boxes around graphs
\usetikzlibrary{shapes,positioning}  %% Clouds and stars
\usetikzlibrary{matrix} %% for matrix
\usepgfplotslibrary{polar} %% for polar plots
\usepgfplotslibrary{fillbetween} %% to shade area between curves in TikZ
\usetkzobj{all}
\usepackage[makeroom]{cancel} %% for strike outs
%\usepackage{mathtools} %% for pretty underbrace % Breaks Ximera
%\usepackage{multicol}
\usepackage{pgffor} %% required for integral for loops



%% http://tex.stackexchange.com/questions/66490/drawing-a-tikz-arc-specifying-the-center
%% Draws beach ball
\tikzset{pics/carc/.style args={#1:#2:#3}{code={\draw[pic actions] (#1:#3) arc(#1:#2:#3);}}}



\usepackage{array}
\setlength{\extrarowheight}{+.1cm}
\newdimen\digitwidth
\settowidth\digitwidth{9}
\def\divrule#1#2{
\noalign{\moveright#1\digitwidth
\vbox{\hrule width#2\digitwidth}}}





\newcommand{\RR}{\mathbb R}
\newcommand{\R}{\mathbb R}
\newcommand{\N}{\mathbb N}
\newcommand{\Z}{\mathbb Z}

\newcommand{\sagemath}{\textsf{SageMath}}


%\renewcommand{\d}{\,d\!}
\renewcommand{\d}{\mathop{}\!d}
\newcommand{\dd}[2][]{\frac{\d #1}{\d #2}}
\newcommand{\pp}[2][]{\frac{\partial #1}{\partial #2}}
\renewcommand{\l}{\ell}
\newcommand{\ddx}{\frac{d}{\d x}}

\newcommand{\zeroOverZero}{\ensuremath{\boldsymbol{\tfrac{0}{0}}}}
\newcommand{\inftyOverInfty}{\ensuremath{\boldsymbol{\tfrac{\infty}{\infty}}}}
\newcommand{\zeroOverInfty}{\ensuremath{\boldsymbol{\tfrac{0}{\infty}}}}
\newcommand{\zeroTimesInfty}{\ensuremath{\small\boldsymbol{0\cdot \infty}}}
\newcommand{\inftyMinusInfty}{\ensuremath{\small\boldsymbol{\infty - \infty}}}
\newcommand{\oneToInfty}{\ensuremath{\boldsymbol{1^\infty}}}
\newcommand{\zeroToZero}{\ensuremath{\boldsymbol{0^0}}}
\newcommand{\inftyToZero}{\ensuremath{\boldsymbol{\infty^0}}}



\newcommand{\numOverZero}{\ensuremath{\boldsymbol{\tfrac{\#}{0}}}}
\newcommand{\dfn}{\textbf}
%\newcommand{\unit}{\,\mathrm}
\newcommand{\unit}{\mathop{}\!\mathrm}
\newcommand{\eval}[1]{\bigg[ #1 \bigg]}
\newcommand{\seq}[1]{\left( #1 \right)}
\renewcommand{\epsilon}{\varepsilon}
\renewcommand{\phi}{\varphi}


\renewcommand{\iff}{\Leftrightarrow}

\DeclareMathOperator{\arccot}{arccot}
\DeclareMathOperator{\arcsec}{arcsec}
\DeclareMathOperator{\arccsc}{arccsc}
\DeclareMathOperator{\si}{Si}
\DeclareMathOperator{\scal}{scal}
\DeclareMathOperator{\sign}{sign}


%% \newcommand{\tightoverset}[2]{% for arrow vec
%%   \mathop{#2}\limits^{\vbox to -.5ex{\kern-0.75ex\hbox{$#1$}\vss}}}
\newcommand{\arrowvec}[1]{{\overset{\rightharpoonup}{#1}}}
%\renewcommand{\vec}[1]{\arrowvec{\mathbf{#1}}}
\renewcommand{\vec}[1]{{\overset{\boldsymbol{\rightharpoonup}}{\mathbf{#1}}}\hspace{0in}}

\newcommand{\point}[1]{\left(#1\right)} %this allows \vector{ to be changed to \vector{ with a quick find and replace
\newcommand{\pt}[1]{\mathbf{#1}} %this allows \vec{ to be changed to \vec{ with a quick find and replace
\newcommand{\Lim}[2]{\lim_{\point{#1} \to \point{#2}}} %Bart, I changed this to point since I want to use it.  It runs through both of the exercise and exerciseE files in limits section, which is why it was in each document to start with.

\DeclareMathOperator{\proj}{\mathbf{proj}}
\newcommand{\veci}{{\boldsymbol{\hat{\imath}}}}
\newcommand{\vecj}{{\boldsymbol{\hat{\jmath}}}}
\newcommand{\veck}{{\boldsymbol{\hat{k}}}}
\newcommand{\vecl}{\vec{\boldsymbol{\l}}}
\newcommand{\uvec}[1]{\mathbf{\hat{#1}}}
\newcommand{\utan}{\mathbf{\hat{t}}}
\newcommand{\unormal}{\mathbf{\hat{n}}}
\newcommand{\ubinormal}{\mathbf{\hat{b}}}

\newcommand{\dotp}{\bullet}
\newcommand{\cross}{\boldsymbol\times}
\newcommand{\grad}{\boldsymbol\nabla}
\newcommand{\divergence}{\grad\dotp}
\newcommand{\curl}{\grad\cross}
%\DeclareMathOperator{\divergence}{divergence}
%\DeclareMathOperator{\curl}[1]{\grad\cross #1}
\newcommand{\lto}{\mathop{\longrightarrow\,}\limits}

\renewcommand{\bar}{\overline}

\colorlet{textColor}{black}
\colorlet{background}{white}
\colorlet{penColor}{blue!50!black} % Color of a curve in a plot
\colorlet{penColor2}{red!50!black}% Color of a curve in a plot
\colorlet{penColor3}{red!50!blue} % Color of a curve in a plot
\colorlet{penColor4}{green!50!black} % Color of a curve in a plot
\colorlet{penColor5}{orange!80!black} % Color of a curve in a plot
\colorlet{penColor6}{yellow!70!black} % Color of a curve in a plot
\colorlet{fill1}{penColor!20} % Color of fill in a plot
\colorlet{fill2}{penColor2!20} % Color of fill in a plot
\colorlet{fillp}{fill1} % Color of positive area
\colorlet{filln}{penColor2!20} % Color of negative area
\colorlet{fill3}{penColor3!20} % Fill
\colorlet{fill4}{penColor4!20} % Fill
\colorlet{fill5}{penColor5!20} % Fill
\colorlet{gridColor}{gray!50} % Color of grid in a plot

\newcommand{\surfaceColor}{violet}
\newcommand{\surfaceColorTwo}{redyellow}
\newcommand{\sliceColor}{greenyellow}




\pgfmathdeclarefunction{gauss}{2}{% gives gaussian
  \pgfmathparse{1/(#2*sqrt(2*pi))*exp(-((x-#1)^2)/(2*#2^2))}%
}


%%%%%%%%%%%%%
%% Vectors
%%%%%%%%%%%%%

%% Simple horiz vectors
\renewcommand{\vector}[1]{\left\langle #1\right\rangle}


%% %% Complex Horiz Vectors with angle brackets
%% \makeatletter
%% \renewcommand{\vector}[2][ , ]{\left\langle%
%%   \def\nextitem{\def\nextitem{#1}}%
%%   \@for \el:=#2\do{\nextitem\el}\right\rangle%
%% }
%% \makeatother

%% %% Vertical Vectors
%% \def\vector#1{\begin{bmatrix}\vecListA#1,,\end{bmatrix}}
%% \def\vecListA#1,{\if,#1,\else #1\cr \expandafter \vecListA \fi}

%%%%%%%%%%%%%
%% End of vectors
%%%%%%%%%%%%%

%\newcommand{\fullwidth}{}
%\newcommand{\normalwidth}{}



%% makes a snazzy t-chart for evaluating functions
%\newenvironment{tchart}{\rowcolors{2}{}{background!90!textColor}\array}{\endarray}

%%This is to help with formatting on future title pages.
\newenvironment{sectionOutcomes}{}{}



%% Flowchart stuff
%\tikzstyle{startstop} = [rectangle, rounded corners, minimum width=3cm, minimum height=1cm,text centered, draw=black]
%\tikzstyle{question} = [rectangle, minimum width=3cm, minimum height=1cm, text centered, draw=black]
%\tikzstyle{decision} = [trapezium, trapezium left angle=70, trapezium right angle=110, minimum width=3cm, minimum height=1cm, text centered, draw=black]
%\tikzstyle{question} = [rectangle, rounded corners, minimum width=3cm, minimum height=1cm,text centered, draw=black]
%\tikzstyle{process} = [rectangle, minimum width=3cm, minimum height=1cm, text centered, draw=black]
%\tikzstyle{decision} = [trapezium, trapezium left angle=70, trapezium right angle=110, minimum width=3cm, minimum height=1cm, text centered, draw=black]


\author{Bart Snapp \and Elizabeth Miller \and Jenny Sheldon \and Jim Talamo}

\outcome{Find explicit formulas for tangent planes.}


\title[Dig-In:]{Tangent planes}

\begin{document}
\begin{abstract}
  We find tangent planes.
\end{abstract}
\maketitle

As we are hoping is becoming a habit, let's start by looking again at
the case of functions of one variable, and use what we know from our
previous calculus courses to answer our questions in higher
dimensions.  As usual with this strategy, we proceed with care.  In
the past you may have learned:
\begin{quote}\index{tangent line}
Given a function $f$ and a number $a$ in the domain of $f$, if one can
``zoom in'' on the graph at $(a, f(a))$ sufficiently so that it
appears to be a straight line, then the function is
\dfn{differentiable}, and that line is the \dfn{tangent line} to
$f$ at the point $(a,f(a))$.
\end{quote}
We illustrate this informal definition with the following diagram:
\begin{image}
\begin{tikzpicture}
 \begin{axis}[
           domain=0:6, range=0:7,
           ymin=-.2,ymax=7,
           width=6in,
           height=2.5in, %% Hard coded height! Moreover this effects the aspect ratio of the zoom--sort of BAD
           axis lines=none,
         ]   
         \addplot [draw=none, fill=textColor!10!background] plot coordinates {(.8,1.6) (2.834,5)} \closedcycle; %% zoom fill
         \addplot [draw=none, fill=textColor!10!background] plot coordinates {(2.834,5) (4.166,5)} \closedcycle; %% zoom fill
         \addplot [draw=none, fill=background] plot coordinates {(1.2,1.6) (4.166,5)} \closedcycle; %% zoom fill
         \addplot [draw=none, fill=background] plot coordinates {(.8,1.6) (1.2,1.6)} \closedcycle; %% zoom fill

         \addplot [draw=none, fill=textColor!10!background] plot coordinates {(3.3,3.6) (5.334,5)} \closedcycle; %% zoom fill
         \addplot [draw=none, fill=textColor!10!background] plot coordinates {(5.334,5) (6.666,5)} \closedcycle; %% zoom fill
         \addplot [draw=none, fill=background] plot coordinates {(3.7,3.6) (6.666,5)} \closedcycle; %% zoom fill
         \addplot [draw=none, fill=background] plot coordinates {(3.3,3.6) (3.7,3.6)} \closedcycle; %% zoom fill
         
         \addplot [draw=none, fill=textColor!10!background] plot coordinates {(3.7,2.4) (6.666,1)} \closedcycle; %% zoom fill
         \addplot [draw=none, fill=textColor!10!background] plot coordinates {(3.3,2.4) (3.7,2.4)} \closedcycle; %% zoom fill
         \addplot [draw=none, fill=background] plot coordinates {(3.3,2.4) (5.334,1)} \closedcycle; %% zoom fill          
         \addplot [draw=none, fill=background] plot coordinates {(5.334,1) (6.666,1)} \closedcycle; %% zoom fill
         

         \addplot [draw=none, fill=textColor!10!background] plot coordinates {(.8,.4) (2.834,1)} \closedcycle; %% zoom fill
         \addplot [draw=none, fill=textColor!10!background] plot coordinates {(2.834,1) (4.166,1)} \closedcycle; %% zoom fill
         \addplot [draw=none, fill=background] plot coordinates {(1.2,.4) (4.166,1)} \closedcycle; %% zoom fill
         \addplot [draw=none, fill=background] plot coordinates {(.8,.4) (1.2,.4)} \closedcycle; %% zoom fill

         \addplot[very thick,penColor, smooth,domain=(0:1.833)] {-1/(x-2)};
         \addplot[very thick,penColor, smooth,domain=(2.834:4.166)] {3.333/(2.050-.3*x)-0.333}; %% 2.5 to 4.333
         %\addplot[very thick,penColor, smooth,domain=(5.334:6.666)] {11.11/(1.540-.09*x)-8.109}; %% 5 to 6.833
         \addplot[very thick,penColor, smooth,domain=(5.334:6.666)] {x-3}; %% 5 to 6.833
         
         \addplot[color=penColor,fill=penColor,only marks,mark=*] coordinates{(1,1)};  %% point to be zoomed
         \addplot[color=penColor,fill=penColor,only marks,mark=*] coordinates{(3.5,3)};  %% zoomed pt 1
         \addplot[color=penColor,fill=penColor,only marks,mark=*] coordinates{(6,3)};  %% zoomed pt 2

         \addplot [->,textColor] plot coordinates {(0,0) (0,6)}; %% axis
         \addplot [->,textColor] plot coordinates {(0,0) (2,0)}; %% axis
         
         \addplot [textColor!50!background] plot coordinates {(.8,.4) (.8,1.6)}; %% box around pt
         \addplot [textColor!50!background] plot coordinates {(1.2,.4) (1.2,1.6)}; %% box around pt
         \addplot [textColor!50!background] plot coordinates {(.8,1.6) (1.2,1.6)}; %% box around pt
         \addplot [textColor!50!background] plot coordinates {(.8,.4) (1.2,.4)}; %% box around pt
         
         \addplot [textColor!50!background] plot coordinates {(2.834,1) (2.834,5)}; %% zoomed box 1
         \addplot [textColor!50!background] plot coordinates {(4.166,1) (4.166,5)}; %% zoomed box 1
         \addplot [textColor!50!background] plot coordinates {(2.834,1) (4.166,1)}; %% zoomed box 1
         \addplot [textColor!50!background] plot coordinates {(2.834,5) (4.166,5)}; %% zoomed box 1

         \addplot [textColor] plot coordinates {(3.3,2.4) (3.3,3.6)}; %% box around zoomed pt
         \addplot [textColor] plot coordinates {(3.7,2.4) (3.7,3.6)}; %% box around zoomed pt
         \addplot [textColor] plot coordinates {(3.3,3.6) (3.7,3.6)}; %% box around zoomed pt
         \addplot [textColor] plot coordinates {(3.3,2.4) (3.7,2.4)}; %% box around zoomed pt

         \addplot [textColor] plot coordinates {(5.334,1) (5.334,5)}; %% zoomed box 2
         \addplot [textColor] plot coordinates {(6.666,1) (6.666,5)}; %% zoomed box 2
         \addplot [textColor] plot coordinates {(5.334,1) (6.666,1)}; %% zoomed box 2
         \addplot [textColor] plot coordinates {(5.334,5) (6.666,5)}; %% zoomed box 2

         \node at (axis cs:2.2,0) [anchor=east] {$x$};
         \node at (axis cs:0,6.6) [anchor=north] {$y$};
       \end{axis}
\end{tikzpicture}
%% \caption{Given a function $f$, if one can ``zoom in'' at $(a, f(a))$
%%   sufficiently so that the graph seems to be a line, then that line is
%%   the \textbf{tangent line} to $f$ at the point determined by $(a,
%%   f(a))$.}
%% \label{figure:informal-tangent}
\end{image}

We can give a similar informal definition for functions of two
variables.
\begin{quote}\index{tangent plane}
  Given a function $F:\R^2\to\R$ and a vector $\vec{a}$ in the domain
  of $F$, if one can ``zoom in'' on the graph at $(\vec{a},
  F(\vec{a}))$ sufficiently so that it appears to be a plane, then the
  function is \dfn{differentiable}, and that plane is the \dfn{tangent
    plane} to $F$ at the point $(\vec{a},F(\vec{a}))$.
\end{quote}

Let's turn our attention to finding an equation for this tangent
plane.


\section{The single variable case}

Given a function $f : \R \to \R$ and a point of interest $x=a$ in the
domain of $f$, we have previously found an equation for the tangent
line to $f$ at $x=a$, which we also called the linear approximation to
$f$ at $x=a$.
\[
\l(x) = f(a) + f'(a) \cdot (x-a)
\]

There are two things we should notice about this linear approximation.

\paragraph{First, think about geometry}  We know that:
\begin{itemize}
\item $f(a)$ is the ``height'' of $f$ at $x=a$.
\item $f'(a)$ is the rate of change of $f$ at $a$.
\item $(x-a)$ is the distance we have moved from the point $a$.
\end{itemize}

Multiplying $f'(a)$ and $(x-a)$ gives us an approximate change in the
value of the function, so to find the new approximate function value,
we need to add on our original height, or $f(a)$.
    \begin{image}
      \begin{tikzpicture}
        \begin{axis}[
            domain=0:2, range=0:2.5,ymax=2.5,ymin=0,
            axis lines =left, xlabel=$x$, ylabel=$y$,
            every axis y label/.style={at=(current axis.above origin),anchor=south},
            every axis x label/.style={at=(current axis.right of origin),anchor=west},
            xtick={1,1.5}, ytick={1,1.5},
            xticklabels={$a$,$x$}, yticklabels={$f(a)$,$\l(x)$},
            axis on top,clip=false
          ]         
      	  \addplot [textColor,dashed] plot coordinates {(1,0) (1,1)};
          \addplot [textColor,dashed] plot coordinates {(0,1) (1,1)};
          \addplot [textColor,dashed] plot coordinates {(0,1.5) (1.5,1.5)};
          \addplot [textColor,dashed] plot coordinates {(1,1) (1.5,1)};
          \addplot [textColor,dashed] plot coordinates {(1.5,0) (1.5,1.5)};
          \addplot [ultra thick,penColor, smooth,domain=(0:1.6)] {-1/(x-2)};
          \addplot[color=penColor,fill=penColor,only marks,mark=*] coordinates{(1,1)};  %% closed hole
          \addplot[color=penColor2,fill=penColor2,only marks,mark=*] coordinates{(1.5,1.5)};  %% closed hole          
          \addplot [very thick,penColor2, smooth,domain=(0:2)] {x};

          \addplot[decoration={brace,mirror,raise=.1cm},decorate,thin] plot coordinates {(1.05,1) (1.45,1)};
          \node[below] at (axis cs: 1.25,.9) {$(x-a)$};

          \addplot[decoration={brace,mirror,raise=.2cm},decorate,thin] plot coordinates {(1.5,1) (1.5,1.5)};
          \node[right] at (axis cs: 1.6,1.25) {$f'(a)\cdot(x-a)$};

          \addplot[decoration={brace,mirror,raise=.2cm},decorate,thin] plot coordinates {(1.5,.1) (1.5,.9)};
          \node[right] at (axis cs: 1.6,.5) {$f(a)$};
        \end{axis}
\end{tikzpicture}
    \end{image}
    So we see that:
    \[
    \l(x) = \underbrace{f(a)}_{\text{height}} + \overbrace{f'(a)}^{\text{rate}} \cdot \underbrace{(x-a)}_{\text{change}} 
    \]

    
\paragraph{Second, this is an approximation}

Specifically the tangent line $\l$ to a function $f$ is a line
constructed so that:
\begin{itemize}
\item $\l(a) = f(a)$
\item $\l'(a) = f'(a)$
\end{itemize}
When the values of $\l$ and $f$ are equal at $x=a$, and they are
changing at the same rates at $x=a$, then it is only reasonable that
$\l$ is a good approximation of $f$.  Moreover, $\l$ is also the
first-order Taylor approximation to $f$. Recall, the degree $n$ Taylor
polynomial for $f$ at $x=a$ is:
\begin{align*}
  p_n(x) = f(a) &+ f'(a) (x-a) && \text{(linear approximation)} \\
  &+ \frac{f''(a)(x-a)^2}{2!}\\
  &+ \cdots  && (\text{higher degree terms})\\
  &+ \frac{f^{n}(a)(x-a)^n}{n!}
\end{align*}
This polynomial starts with the linear approximation, and then adds higher degree terms. 


\section{The two variable case}

Now consider a differentiable function $F : \R^2 \to \R$. Here the
formula for the tangent plane at $\vec{a} =\vector{a,b}$ is given by:
\[
L(x,y) =F(\vec{a})+ F^{(1,0)}(\vec{a}) (x-a)+ F^{(0,1)}(\vec{a}) (y-b)
\]
We will work as we did above.



\paragraph{First, think about geometry}


Finally, let's look at some of the geometry of the equation we just
found.  Working in an entirely similar way as before, we know that:

\begin{itemize}
\item $F(\vec{a})$ is the ``height'' of $F$ at $\vector{x,y} = \vec{a}$.
\item $F^{(1,0)}(\vec{a})$ is the rate of change of $F$ at
  $\vector{x,y} = \vec{a}$ in the $x$-direction.
\item $F^{(0,1)}(\vec{a})$ is the rate of change of $F$ at
  $\vector{x,y} = \vec{a}$ in the $y$-direction.
\item $(x-a)$ is the distance we have moved from the point $a$ in the $x$-direction.
\item $(y-b)$ is the distance we have moved from the point $a$ in the $y$-direction.
\end{itemize}

Multiplying $F^{(1,0)}(\vec{a})$ and $(x-a)$ gives us an approximate
change in the value of the function when moving in the
$x$-direction. Multiplying $F^{(0,1)}(\vec{a})$ and $(y-b)$ gives us
an approximate change in the value of the function when moving in the
$y$-direction. Adding these two together, we see that we have moved
away from the point $\vec{a}$, and we are combining these rates of
change to get an approximate change in the value of $F$.  Adding this
to our original function value $F(\vec{a})$ gives us our new
approximate function value.  In other words, $L$ approximates
$F$ near $\vector{x,y} = \vec{a}$.

\begin{image}
  \begin{tikzpicture}
    \begin{axis}%
      [
        tick label style={font=\scriptsize},axis on top,
	axis lines=center,
	view={145}{35},
	name=myplot,
	xtick={-2},
        xticklabels={$a$},
        ytick={-2},
        yticklabels={$b$},
	%ytick={1,2,3,4,5,6},
	ztick={},
        width=6in,
        zticklabels={},
	ymin=-4.9,ymax=4.9,
	xmin=-4.9,xmax=4.9,
	zmin=-1, zmax=4,
	every axis x label/.style={at={(axis cs:\pgfkeysvalueof{/pgfplots/xmax},0,0)},xshift=-1pt,yshift=-4pt},
	xlabel={\scriptsize $x$},
	every axis y label/.style={at={(axis cs:0,\pgfkeysvalueof{/pgfplots/ymax},0)},xshift=5pt,yshift=-3pt},
	ylabel={\scriptsize $y$},
	every axis z label/.style={at={(axis cs:0,0,\pgfkeysvalueof{/pgfplots/zmax})},xshift=0pt,yshift=4pt},
	zlabel={\scriptsize $z$},
        colormap/cool,
      ]      
      \addplot3[domain=-4:2,,y domain=-4:4,surf,opacity=.5,samples=15,samples y=15,very thin,z buffer=sort] {-.2*(x+1)^2-.05*y^2+2};
      
      
      \draw[dashed] (axis cs:-2,0,0) -- (axis cs:-2,-2,0);
      \draw[dashed] (axis cs:0,-2,0) -- (axis cs:-2,-2,0);
      \draw[dashed] (axis cs:-2,-2,0) -- (axis cs:-2,-2,1.6);

      \draw[dashed] (axis cs:-2,-2,1.6) -- (axis cs:3,-2,1.6);
      \draw[penColor4,ultra thick] (axis cs:3,-2,1.6) -- (axis cs:3,-2,3.6);
      
      \draw[dashed] (axis cs:-2,-2,1.6) -- (axis cs:-2,3,1.6);
      \draw[penColor5,ultra thick] (axis cs:-2,3,1.6) -- (axis cs:-2,3,2.6);

      \draw[dashed] (axis cs:-2,3,1.6) -- (axis cs:3,3,1.6);
      \draw[dashed] (axis cs:3,-2,1.6) -- (axis cs:3,3,1.6);


      \filldraw [penColor] (axis cs:-2,-2,1.6) circle (2pt);

      \filldraw [penColor2] (axis cs:3,3,4.6) circle (2pt);

      \addplot3[domain=-4:4,,y domain=-4:4,surf,red,opacity=.5,samples=15,samples y=15,very thin,z buffer=sort] {2.8+.4*x+.2*y};
      
      
      \draw[penColor4,ultra thick] (axis cs:3,3,1.6) -- (axis cs:3,3,3.6);
      \draw[penColor5,ultra thick] (axis cs:3,3,3.6) -- (axis cs:3,3,4.6);
      

      \node[right] at (axis cs: -2,-2,1.6) {$F(\vec{a})$};

      \node[left,penColor4] at (axis cs: 3,-2,2.6) {$F^{(1,0)}(\vec{a})(x-a)$};
      \node[right,penColor5] at (axis cs: -2,3,2.1) {$F^{(0,1)}(\vec{a})(x-b)$};
      \node[above right,penColor2] at (axis cs: 3,3,4.6) {$F(\vec{a}) + F^{(1,0)}(\vec{a})(x-a) + F^{(0,1)}(\vec{a})(x-b)$};
    \end{axis}
  \end{tikzpicture}
\end{image}
So we see that:
\[
L(x,y) = \underbrace{F(\vec{a})}_{\text{height}} + \overbrace{F^{(1,0)}(\vec{a})}^{\text{rate}} \cdot \underbrace{(x-a)}_{\text{change}} + \overbrace{F^{(0,1)}(\vec{a})}^{\text{rate}} \cdot \underbrace{(y-b)}_{\text{change}} 
\]


\paragraph{Second, this is an approximation}

Note, first that $z=L (x,y)$ is a plane. To be a bit pedantic, we can write
\begin{align*}
z &= L(x,y)\\
z &= F(\vec{a})+ F^{(1,0)}(\vec{a}) \answer[given]{(x-a)}+ F^{(0,1)}(\vec{a}) \answer[given]{(y-b)}
\end{align*}
and rearranging, we find:
\[
a\cdot F^{(1,0)}(\vec{a}) + b\cdot F^{(0,1)}(\vec{a})- F(\vec{a}) = F^{(1,0)}(\vec{a}) \answer[given]{x}+ F^{(0,1)}(\vec{a}) \answer[given]{y} -z 
\]
and this is the equation of a plane, as it is:
\[
\text{number} = \text{number}\cdot x + \text{number}\cdot y + \text{number}\cdot z
\]
Let's see if you get this.
\begin{question}
  Given a differentiable function $F:\R^2\to\R$, find a vector normal
  to the plane $z= L (x,y)$.
  \begin{prompt}
    \[
    \vec{n} = \vector{F^{(1,0)}(\vec{a}), F^{(0,1)}(\vec{a}), \answer[given]{-1}}.
    \]
  \end{prompt}
\end{question}

Now note that the function $L$ is cooked-up so that:
\begin{itemize}
\item $L(a,b) = F(a,b)$
\item $L^{(1,0)} (a,b) = F^{(1,0)}(a,b)$
\item $L^{(0,1)} (a,b) = F^{(0,1)}(a,b)$
\end{itemize}
So, $z=L(x,y)$ is a plane that approximates $z=F(x,y)$  near
the point $(a,b,F(a,b)$.

\begin{question}
  Let $F(x,y) = 5-2(x-3)^2 -3(y-2)^2$. Find a tangent plane at $(x,y) = (2,1)$.
  \begin{prompt}
    \[
    L(x,y) = \answer{4x+6y-14}
    \]
  \end{prompt}
  \begin{question}
    Find a vector normal to the plane you found above.
    \begin{prompt}
    \[
    \vec{n} = \vector{\answer{-4},\answer{-6},1}
    \]
    \end{prompt}
    \begin{hint}
      From above you see:
      \[
      z=4x+6y-14
      \]
      So now write:
      \[
      z-4x-6y = -14
      \]
    \end{hint}
  \end{question}
\end{question}

Foreshadowing a bit, our formula for our linear approximation is:
\[
L(x,y) = F(\vec{a})+ F^{(1,0)}(\vec{a}) (x-a)+ F^{(0,1)}(\vec{a}) (y-b)
\]

This look like what we might want from a first-order Taylor
approximation to our function $F$ as:
\begin{itemize}
\item $L(a,b) = F(a,b)$
\item $L^{(1,0)} (a,b) = F^{(1,0)}(a,b)$
\item $L^{(0,1)} (a,b) = F^{(0,1)}(a,b)$
\end{itemize}
Said in words, the tangent plane at $\vec{a}$ touches the surface at
$\vec{a}$ and the first partial derivatives of $L$ and $F$ agree
at $\vec{a}$.

Saying the tangent plane has the same partial derivatives as the
function is another way of saying that tangent plane has to contain
the tangent lines we can find using the partial derivatives.



Try your hand at this question:


\begin{question}
  Find an equation for the tangent plane to $F(x,y) = 3\cos(x)\sin(y)$ at $(x,y) =
  (\pi/3,\pi/6)$.
  \begin{prompt}
    \[
    z = \answer{(-3\sqrt{3}/4)\cdot (x-\pi/3) + (3\sqrt{3}/4)\cdot (y-\pi/6) + 3/4}
    \]
  \end{prompt}
\end{question}

Moreover, tangent planes are \textit{linear approximations} of
differentiable surfaces. 


\begin{question}
  Let $F:\R^2\to\R$ be a differentiable function where you know that
  $F(1,-3) = 5$ and that $F^{(1,0)}(1,-3) = -4$ and $F^{(0,1)}(1,-3) = 2$. Give a plane
  tangent to $z =F(x,y)$ at the point $(1,-3)$ and use this tangent
  plane to approximate the value of $F(2,-4)$.
  \begin{prompt}
    First we write the linear approximation, meaning the tangent plane:
    \[
    L(x,y) = \answer{5 -4(x-1) + 2(y+3)}
    \]
    So we believe that $F(2,-4) \approx L\left(\answer{2},\answer{-4}\right) =
    \answer{-1}$.
  \end{prompt}
\end{question}


\section{Generalizing with the gradient vector}

Above, given a function $F:\R^2\to\R$, we wrote our linear approximation as:
\[
L(x,y) =F(\vec{a})+ F^{(1,0)}(\vec{a}) (x-a)+ F^{(0,1)}(\vec{a}) (y-b)
\]
We can rewrite this using the gradient vector as follows:
\[
L(\vec{x}) = F(\vec{a})+ \grad F(\vec{a})\dotp (\vec{x}-\vec{a})
\]
where $\vec{x} = \vector{x,y}$ and $\vec{a} = \vector{a,b}$. Now check this out, the formula
\[
L(\vec{x}) = F(\vec{a})+ \grad F(\vec{a})\dotp (\vec{x}-\vec{a})
\]
actually works in \textbf{all} dimensions! So now if you have a
function of three variable, we set $\vec{x} = \vector{x,y,z}$ and
$\vec{a} = \vector{a,b,c}$ and we can unpack this formula as:
\[
L(\vec{x}) = F(\vec{a})+ \grad F(\vec{a})\dotp (\vec{x}-\vec{a})
\]
\begin{align*}
  =F(\vec{a}) &+ F^{(1,0,0)}(\vec{a}) (x-a)\\
  &+ F^{(0,1,0)}(\vec{a}) (y-b)\\
  &+ F^{(0,0,1)}(\vec{a}) (z-c)
\end{align*}
Moreover, if we are working with functions of a single variable, then 
\begin{align*}
\l(x) &= f(a) + f'(a) \cdot (x-a) \\
&=f(\vec{a})+ \grad f(\vec{a})\dotp (\vec{x}-\vec{a})
\end{align*}
where $\vec{a} = \vector{a}$ and $\vec{x} = \vector{x}$. I like to say that the formula
\[
L(\vec{x}) = F(\vec{a})+ \grad F(\vec{a})\dotp (\vec{x}-\vec{a})
\]
is ``one formula to rule them all,'' since when you know this formula,
you know how to compute linear approximations in all dimensions.


\section{An in-depth example}

Our final example brings together several concepts.
\begin{example}
  Let $F(x,y) = x^2 + 3xy + 4$.
  \begin{enumerate}
  \item\label{eg:tpa} Find an equation for the plane tangent to $F$ at $(x,y) =
    (1,2)$.
  \item\label{eg:tpb} Show that the line $y = 3x - 1$ passes through the point $(1,2)$.
  \item\label{eg:tpc} Find a parametric description for the curve
    lying above $y = 3x-1$ on the surface $z = F(x,y)$. Call this
    description $\vec{p}$.
  \item\label{eg:tpd} Find a vector-valued formula for the line tangent to the curve
    $\vec{p}(t)$ at the point $(1, 2, F(1,2))$. Call this $\vecl$.
  \item\label{eg:tpe} Does the tangent line drawn by $\vecl(t)$ lie in the plane
    tangent to $F$ at $(x,y)=(1,2)$?
  \end{enumerate}
  \begin{explanation}
    For part \ref{eg:tpa}, compute $F(1,2) = \answer[given]{11}$,
    $F^{(1,0)}(1,2) = \answer[given]{8}$, $F^{(0,1)}(1,2) =
    \answer[given]{3}$, and set $z = \answer[given]{11 + 8(x-1) +
      3(y-2)}$.

    For part \ref{eg:tpb}, note that we are working here in
    \wordChoice{\choice[correct]{the $(x,y)$-plane}
      \choice{$(x,y,z)$-space}}.  We would like to see that the point
    $(1,2)$ lies on the line given.  To do this, we plug in $x =
    \answer[given]{1}$ to the equation $y = 3x - 1$.  We see that
          \begin{align*}
            y &= 3\left(\answer[given]{1}\right) - 1 \\
            &= \answer[given]{3} - 1 \\
            &= \answer[given]{2}.
          \end{align*}
          So we conclude that the point $(1, 2)$
          \wordChoice{\choice[correct]{is} \choice{is not}} on the
          line $y = 3x-1$.
 
   For part \ref{eg:tpc}, we would like a parametric description $\vec{m}(t)$ of the line $y = 
          3x-1$ in the $(x,y)$-plane.  Since this line is already expressed in terms of 
          $x$, the simplest choice is to take $x = \answer[given]{t}$.  We find
          \[
          \vec{m}(t) = \vector{\answer[given]{t}, \answer[given]{3t-1}}.
          \]
          The curve lying on the surface is the collection of all of
          the images of the points on the line $y = 3x-1$, so we would
          like to evaluate $F$ on all of these points.  Since we
          already have a parametric description $\vec{m}(t) =
          \vector{x(t), y(t)}$, we plug these coordinates in for $x$
          and $y$ to find our $z$- coordinate.  We find
          \[
          \vec{p}(t) = F\left( \vec{m}(t) \right ) = \vector{\answer[given]{t}, \answer[given]{3t-1}, \answer[given]{10t^2-3t+4}}
          \]
      For part \ref{eg:tpd} note that since our tangent line is in
      $(x,y,z)$-space, we will need to find a parametric equation for
      this line.  Recall that the most straightforward way to write
      such an equation is
          \[
          \vecl(t) = \vec{p} + t\vec{v},
          \]
          where $\vec{p}$ is a point on the line and $\vec{v}$ is a direction vector for the 
          line.  The direction of $\vecl(t)$ is tangent to the curve.  We first find the 
          tangent vector for any value of $t$.
          \[
          \vec{p}'(t) = \vector{\answer[given]{1}, \answer[given]{3}, \answer[given]{20t-3}}
          \]
          To find the tangent vector at $(1,2, F(1,2))$, we plug in $t = \answer[given]{1}$ 
          and find $\vec{p}'(1) = \vector{1, 3, 17}$.  We know that the line goes through 
          the point $\vector{1, 2, \answer[given]{11}}$, and so now we can write the 
          equation of our tangent line.
          \begin{align*}
          \vecl(t) &= \vector{1, 2, \answer[given]{11}} + t \vector{\answer[given]{1}, \answer[given]{3}, \answer[given]{17}} \\
          &= \vector{\answer[given]{1+t}, \answer[given]{2+3t}, \answer[given]{11+17t}}
          \end{align*}
  
          Does the tangent line $\vecl(t)$ lie in the tangent plane to
          $F$ at $(1,2)$?


          We could solve this problem by simply plugging in the
          coordinate functions of $\vecl(t)$ to the equation $z =
          L(x,y)$ to see if the two sides of the equals sign agree.
          Instead, let's use a little geometry!
              
          We know that both the tangent plane $L(x,y)$ and the line
          $\vecl(t)$ pass through the point $(1, 2, 11)$.  Since
          $\vecl(t)$ is a line, its tangent vector $\vec{v}$ tells us
          everything we need to know about its direction.  Since
          $L(x,y)$ is a plane, its normal vector $\vec{n}$ tells us
          everything we need to know about its direction.  Then, if
          $\vec{v} \dotp \vec{n} = \answer[given]{0}$, we know that
          the vector $\vec{v}$ must lie in the plane---and thus the
          entire line $\vecl(t)$ must line in the plane as well!
          First, find $\vec{v}$.
          \[
          \vec{v} = \vector{\answer[given]{1}, \answer[given]{3}, \answer[given]{17}}
          \]
          Next, find $\vec{n}$.
          \[
          \vec{n} = \vector{\answer[given]{8}, \answer[given]{3}, \answer[given]{-1}}
          \]
          Finally, compute their dot product.
          \[
          \vec{v} \dotp \vec{n} = \answer[given]{0}
          \]                  
  \end{explanation}
\end{example}



%% Because it is beneficial to recognize the dot product when it arises, notice that we 
%% can rewrite the equation of the tangent plane to look more like the equation from 
%% single variable calculus by using the dot product.
%% \[
%% T_{\vec{a}}(\vec{x})=F(\vec{a})+ \vector{F^{(1,0)}(\vec{a}), F^{(0,1)}(\vec{a})} \dotp (\vec{x}-\vec{a})
%% \]








%% \section{Parametric surfaces}

%% We can also find tangent planes of surfaces that are defined
%% parameterically. Recall that the vector-valued
%% formula\index{plane!vector-valued formula}\index{plane!parametric formula} for a plane is:
%% \[
%% \vec{L}(s,t) = \vec{p} + s\cdot \vec{v} + t\cdot\vec{w},
%% \]
%% where $\vec{p}$ is a vector whose ``tip'' is on the plane, and
%% $\vec{v}$ and $\vec{w}$ are in the plane. Given a vector-valued
%% surface $\vec{P}(s,t)$, and a point given by a specific value, say
%% $\vec{p} = \vec{P}(a,b)$, we can produce a tangent plane by writing:
%% \[
%% \vec{L}(s,t) = \vec{p} + s\cdot \eval{\pp[\vec{P}(s,t)]{s}}_{} + t\cdot\vec{w},
%% \]


%% HERE



%% \begin{example}
%% Consider the parameterization of the unit sphere,
%% centered at the origin where $0\le \theta< 2\pi$ and $0\le\phi\le \pi$:
%% \begin{align*}
%%   x(\theta,\phi) &= \cos(\theta)\sin(\phi)\\
%%   y(\theta,\phi) &= \sin(\theta)\sin(\phi)\\
%%   z(\theta,\phi) &= \cos(\phi)
%% \end{align*}
%% Find a plane tangent to the sphere when $\theta = \pi/4$ and $\phi =
%% \pi/3$.
%% \begin{explanation}
%%   Here we'll use the parametric formula for a plane:
%%   \[
%%   \vec{L}(s,t) = \vec{p}+ s \vec{v} + t\cdot \vec{w}
%%   \]
%%   The the point $\vec{p}$ (denoted by a vector) is:
%%   \[
%%   \vec{p} = \vector{\answer[given]{\sqrt{3/8}},\answer[given]{\sqrt{3/8}},\answer[given]{1/2}}
%%   \]
%%   The vector $\vec{v}$ is given by
%%   \begin{align*}
%%     \vec{v} &=\eval{\pp{\theta}\vector{x(\theta,\phi),y(\theta,\phi),z(\theta,\phi)}}_{\substack{\theta = \pi/4\\ \phi = \pi/3}}\\
%%       &= \eval{\vector{\answer[given]{-\sin(\theta)\sin(\phi)},\answer[given]{\cos(\theta)\sin(\phi)},\answer[given]{0}}}_{\substack{\theta = \pi/4\\ \phi = \pi/3}}\\
%%       &= \vector{-\sqrt{3/8},\sqrt{3/8},0}
%%   \end{align*}
%%   And vector $\vec{w}$ is given by
%%   \begin{align*}
%%     \vec{w} &=\eval{\pp{\phi}\vector{x(\theta,\phi),y(\theta,\phi),z(\theta,\phi)}}_{\substack{\theta = \pi/4\\ \phi = \pi/3}}\\
%%       &= \eval{\vector{\answer[given]{\cos(\theta)\cos(\phi)},\answer[given]{\sin(\theta)\cos(\phi)},\answer[given]{-\sin(\phi)}}}_{\substack{\theta = \pi/4\\ \phi = \pi/3}}\\
%%       &= \vector{1/\sqrt{8},1/\sqrt{8},-\sqrt{3}/2}
%%   \end{align*}
%%   Hence our desired plane is given by:
%%   \[
%%   \vec{L}(s,t) = \begin{bmatrix}
%%     \answer[given]{\sqrt{3/8}-s\sqrt{3/8}+ t/\sqrt{8}}\\
%%     \answer[given]{\sqrt{3/8} +s\sqrt{3/8} + t/\sqrt{8}}\\
%%     \answer[given]{1/2 -t\sqrt{3}/2}
%%   \end{bmatrix}
%%   \]
%% \end{explanation}
%% \end{example}





\end{document}
