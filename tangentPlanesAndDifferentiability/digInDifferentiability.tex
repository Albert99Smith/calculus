\documentclass{ximera}

%\usepackage{todonotes}
%\usepackage{mathtools} %% Required for wide table Curl and Greens
%\usepackage{cuted} %% Required for wide table Curl and Greens
\newcommand{\todo}{}

\usepackage{esint} % for \oiint
\ifxake%%https://math.meta.stackexchange.com/questions/9973/how-do-you-render-a-closed-surface-double-integral
\renewcommand{\oiint}{{\large\bigcirc}\kern-1.56em\iint}
\fi


\graphicspath{
  {./}
  {ximeraTutorial/}
  {basicPhilosophy/}
  {functionsOfSeveralVariables/}
  {normalVectors/}
  {lagrangeMultipliers/}
  {vectorFields/}
  {greensTheorem/}
  {shapeOfThingsToCome/}
  {dotProducts/}
  {partialDerivativesAndTheGradientVector/}
  {../productAndQuotientRules/exercises/}
  {../normalVectors/exercisesParametricPlots/}
  {../continuityOfFunctionsOfSeveralVariables/exercises/}
  {../partialDerivativesAndTheGradientVector/exercises/}
  {../directionalDerivativeAndChainRule/exercises/}
  {../commonCoordinates/exercisesCylindricalCoordinates/}
  {../commonCoordinates/exercisesSphericalCoordinates/}
  {../greensTheorem/exercisesCurlAndLineIntegrals/}
  {../greensTheorem/exercisesDivergenceAndLineIntegrals/}
  {../shapeOfThingsToCome/exercisesDivergenceTheorem/}
  {../greensTheorem/}
  {../shapeOfThingsToCome/}
  {../separableDifferentialEquations/exercises/}
  {vectorFields/}
}

\newcommand{\mooculus}{\textsf{\textbf{MOOC}\textnormal{\textsf{ULUS}}}}

\usepackage{tkz-euclide}\usepackage{tikz}
\usepackage{tikz-cd}
\usetikzlibrary{arrows}
\tikzset{>=stealth,commutative diagrams/.cd,
  arrow style=tikz,diagrams={>=stealth}} %% cool arrow head
\tikzset{shorten <>/.style={ shorten >=#1, shorten <=#1 } } %% allows shorter vectors

\usetikzlibrary{backgrounds} %% for boxes around graphs
\usetikzlibrary{shapes,positioning}  %% Clouds and stars
\usetikzlibrary{matrix} %% for matrix
\usepgfplotslibrary{polar} %% for polar plots
\usepgfplotslibrary{fillbetween} %% to shade area between curves in TikZ
\usetkzobj{all}
\usepackage[makeroom]{cancel} %% for strike outs
%\usepackage{mathtools} %% for pretty underbrace % Breaks Ximera
%\usepackage{multicol}
\usepackage{pgffor} %% required for integral for loops



%% http://tex.stackexchange.com/questions/66490/drawing-a-tikz-arc-specifying-the-center
%% Draws beach ball
\tikzset{pics/carc/.style args={#1:#2:#3}{code={\draw[pic actions] (#1:#3) arc(#1:#2:#3);}}}



\usepackage{array}
\setlength{\extrarowheight}{+.1cm}
\newdimen\digitwidth
\settowidth\digitwidth{9}
\def\divrule#1#2{
\noalign{\moveright#1\digitwidth
\vbox{\hrule width#2\digitwidth}}}





\newcommand{\RR}{\mathbb R}
\newcommand{\R}{\mathbb R}
\newcommand{\N}{\mathbb N}
\newcommand{\Z}{\mathbb Z}

\newcommand{\sagemath}{\textsf{SageMath}}


%\renewcommand{\d}{\,d\!}
\renewcommand{\d}{\mathop{}\!d}
\newcommand{\dd}[2][]{\frac{\d #1}{\d #2}}
\newcommand{\pp}[2][]{\frac{\partial #1}{\partial #2}}
\renewcommand{\l}{\ell}
\newcommand{\ddx}{\frac{d}{\d x}}

\newcommand{\zeroOverZero}{\ensuremath{\boldsymbol{\tfrac{0}{0}}}}
\newcommand{\inftyOverInfty}{\ensuremath{\boldsymbol{\tfrac{\infty}{\infty}}}}
\newcommand{\zeroOverInfty}{\ensuremath{\boldsymbol{\tfrac{0}{\infty}}}}
\newcommand{\zeroTimesInfty}{\ensuremath{\small\boldsymbol{0\cdot \infty}}}
\newcommand{\inftyMinusInfty}{\ensuremath{\small\boldsymbol{\infty - \infty}}}
\newcommand{\oneToInfty}{\ensuremath{\boldsymbol{1^\infty}}}
\newcommand{\zeroToZero}{\ensuremath{\boldsymbol{0^0}}}
\newcommand{\inftyToZero}{\ensuremath{\boldsymbol{\infty^0}}}



\newcommand{\numOverZero}{\ensuremath{\boldsymbol{\tfrac{\#}{0}}}}
\newcommand{\dfn}{\textbf}
%\newcommand{\unit}{\,\mathrm}
\newcommand{\unit}{\mathop{}\!\mathrm}
\newcommand{\eval}[1]{\bigg[ #1 \bigg]}
\newcommand{\seq}[1]{\left( #1 \right)}
\renewcommand{\epsilon}{\varepsilon}
\renewcommand{\phi}{\varphi}


\renewcommand{\iff}{\Leftrightarrow}

\DeclareMathOperator{\arccot}{arccot}
\DeclareMathOperator{\arcsec}{arcsec}
\DeclareMathOperator{\arccsc}{arccsc}
\DeclareMathOperator{\si}{Si}
\DeclareMathOperator{\scal}{scal}
\DeclareMathOperator{\sign}{sign}


%% \newcommand{\tightoverset}[2]{% for arrow vec
%%   \mathop{#2}\limits^{\vbox to -.5ex{\kern-0.75ex\hbox{$#1$}\vss}}}
\newcommand{\arrowvec}[1]{{\overset{\rightharpoonup}{#1}}}
%\renewcommand{\vec}[1]{\arrowvec{\mathbf{#1}}}
\renewcommand{\vec}[1]{{\overset{\boldsymbol{\rightharpoonup}}{\mathbf{#1}}}\hspace{0in}}

\newcommand{\point}[1]{\left(#1\right)} %this allows \vector{ to be changed to \vector{ with a quick find and replace
\newcommand{\pt}[1]{\mathbf{#1}} %this allows \vec{ to be changed to \vec{ with a quick find and replace
\newcommand{\Lim}[2]{\lim_{\point{#1} \to \point{#2}}} %Bart, I changed this to point since I want to use it.  It runs through both of the exercise and exerciseE files in limits section, which is why it was in each document to start with.

\DeclareMathOperator{\proj}{\mathbf{proj}}
\newcommand{\veci}{{\boldsymbol{\hat{\imath}}}}
\newcommand{\vecj}{{\boldsymbol{\hat{\jmath}}}}
\newcommand{\veck}{{\boldsymbol{\hat{k}}}}
\newcommand{\vecl}{\vec{\boldsymbol{\l}}}
\newcommand{\uvec}[1]{\mathbf{\hat{#1}}}
\newcommand{\utan}{\mathbf{\hat{t}}}
\newcommand{\unormal}{\mathbf{\hat{n}}}
\newcommand{\ubinormal}{\mathbf{\hat{b}}}

\newcommand{\dotp}{\bullet}
\newcommand{\cross}{\boldsymbol\times}
\newcommand{\grad}{\boldsymbol\nabla}
\newcommand{\divergence}{\grad\dotp}
\newcommand{\curl}{\grad\cross}
%\DeclareMathOperator{\divergence}{divergence}
%\DeclareMathOperator{\curl}[1]{\grad\cross #1}
\newcommand{\lto}{\mathop{\longrightarrow\,}\limits}

\renewcommand{\bar}{\overline}

\colorlet{textColor}{black}
\colorlet{background}{white}
\colorlet{penColor}{blue!50!black} % Color of a curve in a plot
\colorlet{penColor2}{red!50!black}% Color of a curve in a plot
\colorlet{penColor3}{red!50!blue} % Color of a curve in a plot
\colorlet{penColor4}{green!50!black} % Color of a curve in a plot
\colorlet{penColor5}{orange!80!black} % Color of a curve in a plot
\colorlet{penColor6}{yellow!70!black} % Color of a curve in a plot
\colorlet{fill1}{penColor!20} % Color of fill in a plot
\colorlet{fill2}{penColor2!20} % Color of fill in a plot
\colorlet{fillp}{fill1} % Color of positive area
\colorlet{filln}{penColor2!20} % Color of negative area
\colorlet{fill3}{penColor3!20} % Fill
\colorlet{fill4}{penColor4!20} % Fill
\colorlet{fill5}{penColor5!20} % Fill
\colorlet{gridColor}{gray!50} % Color of grid in a plot

\newcommand{\surfaceColor}{violet}
\newcommand{\surfaceColorTwo}{redyellow}
\newcommand{\sliceColor}{greenyellow}




\pgfmathdeclarefunction{gauss}{2}{% gives gaussian
  \pgfmathparse{1/(#2*sqrt(2*pi))*exp(-((x-#1)^2)/(2*#2^2))}%
}


%%%%%%%%%%%%%
%% Vectors
%%%%%%%%%%%%%

%% Simple horiz vectors
\renewcommand{\vector}[1]{\left\langle #1\right\rangle}


%% %% Complex Horiz Vectors with angle brackets
%% \makeatletter
%% \renewcommand{\vector}[2][ , ]{\left\langle%
%%   \def\nextitem{\def\nextitem{#1}}%
%%   \@for \el:=#2\do{\nextitem\el}\right\rangle%
%% }
%% \makeatother

%% %% Vertical Vectors
%% \def\vector#1{\begin{bmatrix}\vecListA#1,,\end{bmatrix}}
%% \def\vecListA#1,{\if,#1,\else #1\cr \expandafter \vecListA \fi}

%%%%%%%%%%%%%
%% End of vectors
%%%%%%%%%%%%%

%\newcommand{\fullwidth}{}
%\newcommand{\normalwidth}{}



%% makes a snazzy t-chart for evaluating functions
%\newenvironment{tchart}{\rowcolors{2}{}{background!90!textColor}\array}{\endarray}

%%This is to help with formatting on future title pages.
\newenvironment{sectionOutcomes}{}{}



%% Flowchart stuff
%\tikzstyle{startstop} = [rectangle, rounded corners, minimum width=3cm, minimum height=1cm,text centered, draw=black]
%\tikzstyle{question} = [rectangle, minimum width=3cm, minimum height=1cm, text centered, draw=black]
%\tikzstyle{decision} = [trapezium, trapezium left angle=70, trapezium right angle=110, minimum width=3cm, minimum height=1cm, text centered, draw=black]
%\tikzstyle{question} = [rectangle, rounded corners, minimum width=3cm, minimum height=1cm,text centered, draw=black]
%\tikzstyle{process} = [rectangle, minimum width=3cm, minimum height=1cm, text centered, draw=black]
%\tikzstyle{decision} = [trapezium, trapezium left angle=70, trapezium right angle=110, minimum width=3cm, minimum height=1cm, text centered, draw=black]


\author{Bart Snapp}

\outcome{Be familiar with theorems related to differentiability.}

\title[Dig-In:]{Differentiability}

\begin{document}
\begin{abstract}
  We introduce differentiability for functions of several variables and find tangent planes.
\end{abstract}
\maketitle


\section{Differentiability}

Previously in your calculus experience, we often asked the question,
``Is this function differentiable?''  When looking at $f: \R \to \R$,
the answer to that question was the same as the answer to,``Does the
derivitive exist?''  Now that we are working with multivariable
functions, we've already seen that there is more than one derivative
we can define at any given point, giving us different information
about the surface.  Therefore, we need a different strategy. Another
way we could have answered whether a function $f: \R \to \R$ was
differentiable was to ask whether there exists a tangent line (that
is, a linear approximation) to the function at that point.  This idea
we can generalize to higher dimensions.

Recall, for $f: \R \to \R$, we say a function is differentiable if the
limit
\[
f'(a)=\lim_{x \to a} \frac{f(x)-f(a)}{x-a} 
\]
exists.  This is equivalent to saying that
\begin{align*}
\lim_{x \to a} \frac{f(x)-f(a)}{x-a} -f'(a)=0 \\
\lim_{x \to a} \frac{f(x)-f(a) -f'(a)(x-a)}{x-a}=0\\
\lim_{x \to a} \frac{f(x)-(f(a) +f'(a)(x-a))}{x-a}=0\\
\lim_{x \to a} \frac{f(x)-\l(x)}{x-a}=0
\end{align*}
where $\l(x)$ is the equation of the tangent line to $f$ at $a$. 

Let's use this same idea to define differentiability in higher
dimensions.

\begin{definition}
Let $F: \R^n\to \R$ and let $\vec{x}=\vector{x_1, x_2, \dots, x_n}$ and
$\vec{a} = \vector{a_1, a_2, \dots, a_n}$ be vectors in the interior of
the domain of $F$.  We say $F$ is \dfn{differentiable} at a vector
$\vec{a}$ if there exists $L(\vec{x})=c_0+\sum_{i=1}^n c_i
(x_i-a_i)$ for some real constants $c_i$ such that
\[
\lim_{\vec{x} \to \vec{a}} \frac{F(\vec{x})-L(\vec{x})}{|\vec{x}-\vec{a}|} = 0
\]
A function is \dfn{differentiable} on an open set if it is
differentiable at every point in that set.
\end{definition}


%In the past you may have learned
%\begin{quote}\index{tangent line}
%Given a function $f$ and a number $a$ in the domain of $f$, if one can
%``zoom in'' on the graph at $(a, f(a))$ sufficiently so that it
%appears to be a straight line, then the function is
%\dfn{differentiable}, and that line is the \dfn{tangent line} to
%$f(x)$ at the point $(a,f(a))$.
%\end{quote}
%We illustrate this informal definition with the following diagram:
%\begin{image}
%\begin{tikzpicture}
%  \begin{axis}[
%            domain=0:6, range=0:7,
%            ymin=-.2,ymax=7,
%            width=6in,
%            height=2.5in, %% Hard coded height! Moreover this effects the aspect ratio of the zoom--sort of BAD
%            axis lines=none,
%          ]   
%          \addplot [draw=none, fill=textColor!10!background] plot coordinates {(.8,1.6) (2.834,5)} \closedcycle; %% zoom fill
%          \addplot [draw=none, fill=textColor!10!background] plot coordinates {(2.834,5) (4.166,5)} \closedcycle; %% zoom fill
%          \addplot [draw=none, fill=background] plot coordinates {(1.2,1.6) (4.166,5)} \closedcycle; %% zoom fill
%          \addplot [draw=none, fill=background] plot coordinates {(.8,1.6) (1.2,1.6)} \closedcycle; %% zoom fill
%
%          \addplot [draw=none, fill=textColor!10!background] plot coordinates {(3.3,3.6) (5.334,5)} \closedcycle; %% zoom fill
%          \addplot [draw=none, fill=textColor!10!background] plot coordinates {(5.334,5) (6.666,5)} \closedcycle; %% zoom fill
%          \addplot [draw=none, fill=background] plot coordinates {(3.7,3.6) (6.666,5)} \closedcycle; %% zoom fill
%          \addplot [draw=none, fill=background] plot coordinates {(3.3,3.6) (3.7,3.6)} \closedcycle; %% zoom fill
%          
%          \addplot [draw=none, fill=textColor!10!background] plot coordinates {(3.7,2.4) (6.666,1)} \closedcycle; %% zoom fill
%          \addplot [draw=none, fill=textColor!10!background] plot coordinates {(3.3,2.4) (3.7,2.4)} \closedcycle; %% zoom fill
%          \addplot [draw=none, fill=background] plot coordinates {(3.3,2.4) (5.334,1)} \closedcycle; %% zoom fill          
%          \addplot [draw=none, fill=background] plot coordinates {(5.334,1) (6.666,1)} \closedcycle; %% zoom fill
%          
%
%          \addplot [draw=none, fill=textColor!10!background] plot coordinates {(.8,.4) (2.834,1)} \closedcycle; %% zoom fill
%          \addplot [draw=none, fill=textColor!10!background] plot coordinates {(2.834,1) (4.166,1)} \closedcycle; %% zoom fill
%          \addplot [draw=none, fill=background] plot coordinates {(1.2,.4) (4.166,1)} \closedcycle; %% zoom fill
%          \addplot [draw=none, fill=background] plot coordinates {(.8,.4) (1.2,.4)} \closedcycle; %% zoom fill
%
%          \addplot[very thick,penColor, smooth,domain=(0:1.833)] {-1/(x-2)};
%          \addplot[very thick,penColor, smooth,domain=(2.834:4.166)] {3.333/(2.050-.3*x)-0.333}; %% 2.5 to 4.333
%          %\addplot[very thick,penColor, smooth,domain=(5.334:6.666)] {11.11/(1.540-.09*x)-8.109}; %% 5 to 6.833
%          \addplot[very thick,penColor, smooth,domain=(5.334:6.666)] {x-3}; %% 5 to 6.833
%          
%          \addplot[color=penColor,fill=penColor,only marks,mark=*] coordinates{(1,1)};  %% point to be zoomed
%          \addplot[color=penColor,fill=penColor,only marks,mark=*] coordinates{(3.5,3)};  %% zoomed pt 1
%          \addplot[color=penColor,fill=penColor,only marks,mark=*] coordinates{(6,3)};  %% zoomed pt 2
%
%          \addplot [->,textColor] plot coordinates {(0,0) (0,6)}; %% axis
%          \addplot [->,textColor] plot coordinates {(0,0) (2,0)}; %% axis
%          
%          \addplot [textColor!50!background] plot coordinates {(.8,.4) (.8,1.6)}; %% box around pt
%          \addplot [textColor!50!background] plot coordinates {(1.2,.4) (1.2,1.6)}; %% box around pt
%          \addplot [textColor!50!background] plot coordinates {(.8,1.6) (1.2,1.6)}; %% box around pt
%          \addplot [textColor!50!background] plot coordinates {(.8,.4) (1.2,.4)}; %% box around pt
%          
%          \addplot [textColor!50!background] plot coordinates {(2.834,1) (2.834,5)}; %% zoomed box 1
%          \addplot [textColor!50!background] plot coordinates {(4.166,1) (4.166,5)}; %% zoomed box 1
%          \addplot [textColor!50!background] plot coordinates {(2.834,1) (4.166,1)}; %% zoomed box 1
%          \addplot [textColor!50!background] plot coordinates {(2.834,5) (4.166,5)}; %% zoomed box 1
%
%          \addplot [textColor] plot coordinates {(3.3,2.4) (3.3,3.6)}; %% box around zoomed pt
%          \addplot [textColor] plot coordinates {(3.7,2.4) (3.7,3.6)}; %% box around zoomed pt
%          \addplot [textColor] plot coordinates {(3.3,3.6) (3.7,3.6)}; %% box around zoomed pt
%          \addplot [textColor] plot coordinates {(3.3,2.4) (3.7,2.4)}; %% box around zoomed pt
%
%          \addplot [textColor] plot coordinates {(5.334,1) (5.334,5)}; %% zoomed box 2
%          \addplot [textColor] plot coordinates {(6.666,1) (6.666,5)}; %% zoomed box 2
%          \addplot [textColor] plot coordinates {(5.334,1) (6.666,1)}; %% zoomed box 2
%          \addplot [textColor] plot coordinates {(5.334,5) (6.666,5)}; %% zoomed box 2
%
%          \node at (axis cs:2.2,0) [anchor=east] {$x$};
%          \node at (axis cs:0,6.6) [anchor=north] {$y$};
%        \end{axis}
%\end{tikzpicture}
%%% \caption{Given a function $f$, if one can ``zoom in'' at $(a, f(a))$
%%% sufficiently so that the graph seems to be a straight line,
%%% then that line is the \textbf{tangent line} to $f$ at the point
%%% determined by $(a, f(a))$.}
%%% \label{figure:informal-tangent}
%\end{image}
%%% \todo{This image should be interactive.}
%
%**INTERACTIVE IMAGE**
%
%We would like to generalize this notion of a tangent line to functions of several variables.  We would like to say $F$  is differentiable when one can ``zoom-in'' and the graph of the function looks "flat".  
%
%Let's start by considering functions from $\R^2\to \R$ as these are the easiest to visualize. In this case, if our function is differentiable, when we "zoom-in", graph of $F: \R^2\to \R$ should look like a plane. We call this plane the \dfn{tangent plane}. 
%
%**INTERACTIVE IMAGE**
%%%BADBAD would like an image

%In fact, this is what the technical definition of differentiability of a function $F: \R^2\to \R$ says.

%\begin{definition}
%Let $F: \R^2\to \R$.  We say $F$ is  \dfn{differentiable} at a vector $\vec{a}$ if there exists a tangent plane $T_a(\vec{x})$ such that 
%\[
%\lim_{\vec{x} \rightarrow \vec{a}} \frac{F(\vec{x})-T_a(\vec{x})}{|\vec{x}-\vec{a}|} = 0
%\]
%A function is \dfn{differentiable} on an open set if it is differentiable at every point in that set.
%\end{definition}

In $\R^2$, we can interpret this definition as saying that a function
$F$ is differentiable at a vector $\vec{a}$ if there is a plane
$L(\vec{x})$ at that point such that $F$ approaches
$L(\vec{x})$ faster than $\vec{x}$ approaches $\vec{a}$.  In
this case, we call this plane the \dfn{tangent plane}.  We interpret
this differentiability as, if one ``zooms in'' on the graph of $F$ at
$(\vec{a}, F(\vec{a}))$ sufficiently, it looks more and more like the
tangent plane.

\begin{onlineOnly}
  You can use this interactive to visualize a tangent plane.
  \begin{center}
    \geogebra{aurqzZWx}{800}{600}%https://www.geogebra.org/3d/aurqzZWx
  \end{center}
\end{onlineOnly}

How do we interpret this definition in higher dimensions?  If we zoom
in on a differentiable curve $f:\R \to \R$, we will see $f$ approach
its tangent line. If we zoom in on a differentiable surface $F:\R^2
\to \R$, we will see $F$ approach its tangent plane.  Similarly, if we
zoom in on a differentiable surface $F:\R^n \to \R$, we should expect
to see $F$ approach a flat surface $L(x)$ which looks like $\R^n$.

Keep in mind that this statement of differentiability is much stronger
than just saying the partial derivatives of $F$ exist at $\vec{a}$.
Because this is a limit in a higher dimension, we are saying this
limit must exist and be equal for every path by which $\vec{x}$ could
approach $\vec{a}$, not just the paths along the axes.  Because of
this, it is very difficult to use this definition to show that a
function is differentiable.  Thankfully we have the following theorem.

\begin{theorem}[A Sufficient Criterion for Differentiability]
  Let $F:\R^n\to\R$ be defined on an open set $S$ containing
  $\vec{a}$.  If
  \[
  \pp[F]{x_i}(x_1, x_2, \dots, x_n)
  \]
 for $i$ from $1$ to $n$ are all continuous on $S$, then $F$ is differentiable on $S$.
\end{theorem}

In words, this theorem is saying that if all of the partial derivatives are all continuous on all of $S$, then $F$ is differentiable on $S$.  Notice how much easier this is to use than the definition! This theorem assures us that essentially all functions that we see in
the course of our studies here are differentiable on their natural domains.

Remember, we were trying to generalize the concept of differentiability for $f:\R \to \R$?  Our concept of differentiability for $f:\R \to \R$ had a lot of nice properties.  Many of these properties also hold for differentiability in higher dimensions.  In particular, here are a few examples.

\begin{theorem}[Differentiability Implies Partial Derivatives Exist]
Let $F:\R^n\to\R$ be defined on an open set $S$ containing
$\vec{a}$.  If $F$ is differentiable at $\vec{a}$, then all the partial derivatives of $F$ exist at $\vec{a}$.
\end{theorem}

\begin{theorem}[Differentiability Implies Continuity]
Let $F:\R^n\to\R$ be defined on an open set $S$ containing
$\vec{a}$.  If $F$ is differentiable at $\vec{a}$, then $F$ is
continuous at $\vec{a}$.
\end{theorem}

Since most of the functions we see in this course are differentiable
on their domains, they are also continuous.

A note of caution: differentiability in higher dimensions is a much
more subtle notion.  For instance, it is possible for a function $F$
to be differentiable yet some partials may not be continuous. It is
also possible for all the partial derivatives to exist at a point and
the function still fail to be continuous, let alone differentiable, at
that point.  Such strange behavior of functions is a source of delight
for many mathematicians.



\end{document}
