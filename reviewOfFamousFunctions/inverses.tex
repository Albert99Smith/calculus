\documentclass{ximera}

%\usepackage{todonotes}
%\usepackage{mathtools} %% Required for wide table Curl and Greens
%\usepackage{cuted} %% Required for wide table Curl and Greens
\newcommand{\todo}{}

\usepackage{esint} % for \oiint
\ifxake%%https://math.meta.stackexchange.com/questions/9973/how-do-you-render-a-closed-surface-double-integral
\renewcommand{\oiint}{{\large\bigcirc}\kern-1.56em\iint}
\fi


\graphicspath{
  {./}
  {ximeraTutorial/}
  {basicPhilosophy/}
  {functionsOfSeveralVariables/}
  {normalVectors/}
  {lagrangeMultipliers/}
  {vectorFields/}
  {greensTheorem/}
  {shapeOfThingsToCome/}
  {dotProducts/}
  {partialDerivativesAndTheGradientVector/}
  {../productAndQuotientRules/exercises/}
  {../normalVectors/exercisesParametricPlots/}
  {../continuityOfFunctionsOfSeveralVariables/exercises/}
  {../partialDerivativesAndTheGradientVector/exercises/}
  {../directionalDerivativeAndChainRule/exercises/}
  {../commonCoordinates/exercisesCylindricalCoordinates/}
  {../commonCoordinates/exercisesSphericalCoordinates/}
  {../greensTheorem/exercisesCurlAndLineIntegrals/}
  {../greensTheorem/exercisesDivergenceAndLineIntegrals/}
  {../shapeOfThingsToCome/exercisesDivergenceTheorem/}
  {../greensTheorem/}
  {../shapeOfThingsToCome/}
  {../separableDifferentialEquations/exercises/}
  {vectorFields/}
}

\newcommand{\mooculus}{\textsf{\textbf{MOOC}\textnormal{\textsf{ULUS}}}}

\usepackage{tkz-euclide}\usepackage{tikz}
\usepackage{tikz-cd}
\usetikzlibrary{arrows}
\tikzset{>=stealth,commutative diagrams/.cd,
  arrow style=tikz,diagrams={>=stealth}} %% cool arrow head
\tikzset{shorten <>/.style={ shorten >=#1, shorten <=#1 } } %% allows shorter vectors

\usetikzlibrary{backgrounds} %% for boxes around graphs
\usetikzlibrary{shapes,positioning}  %% Clouds and stars
\usetikzlibrary{matrix} %% for matrix
\usepgfplotslibrary{polar} %% for polar plots
\usepgfplotslibrary{fillbetween} %% to shade area between curves in TikZ
\usetkzobj{all}
\usepackage[makeroom]{cancel} %% for strike outs
%\usepackage{mathtools} %% for pretty underbrace % Breaks Ximera
%\usepackage{multicol}
\usepackage{pgffor} %% required for integral for loops



%% http://tex.stackexchange.com/questions/66490/drawing-a-tikz-arc-specifying-the-center
%% Draws beach ball
\tikzset{pics/carc/.style args={#1:#2:#3}{code={\draw[pic actions] (#1:#3) arc(#1:#2:#3);}}}



\usepackage{array}
\setlength{\extrarowheight}{+.1cm}
\newdimen\digitwidth
\settowidth\digitwidth{9}
\def\divrule#1#2{
\noalign{\moveright#1\digitwidth
\vbox{\hrule width#2\digitwidth}}}





\newcommand{\RR}{\mathbb R}
\newcommand{\R}{\mathbb R}
\newcommand{\N}{\mathbb N}
\newcommand{\Z}{\mathbb Z}

\newcommand{\sagemath}{\textsf{SageMath}}


%\renewcommand{\d}{\,d\!}
\renewcommand{\d}{\mathop{}\!d}
\newcommand{\dd}[2][]{\frac{\d #1}{\d #2}}
\newcommand{\pp}[2][]{\frac{\partial #1}{\partial #2}}
\renewcommand{\l}{\ell}
\newcommand{\ddx}{\frac{d}{\d x}}

\newcommand{\zeroOverZero}{\ensuremath{\boldsymbol{\tfrac{0}{0}}}}
\newcommand{\inftyOverInfty}{\ensuremath{\boldsymbol{\tfrac{\infty}{\infty}}}}
\newcommand{\zeroOverInfty}{\ensuremath{\boldsymbol{\tfrac{0}{\infty}}}}
\newcommand{\zeroTimesInfty}{\ensuremath{\small\boldsymbol{0\cdot \infty}}}
\newcommand{\inftyMinusInfty}{\ensuremath{\small\boldsymbol{\infty - \infty}}}
\newcommand{\oneToInfty}{\ensuremath{\boldsymbol{1^\infty}}}
\newcommand{\zeroToZero}{\ensuremath{\boldsymbol{0^0}}}
\newcommand{\inftyToZero}{\ensuremath{\boldsymbol{\infty^0}}}



\newcommand{\numOverZero}{\ensuremath{\boldsymbol{\tfrac{\#}{0}}}}
\newcommand{\dfn}{\textbf}
%\newcommand{\unit}{\,\mathrm}
\newcommand{\unit}{\mathop{}\!\mathrm}
\newcommand{\eval}[1]{\bigg[ #1 \bigg]}
\newcommand{\seq}[1]{\left( #1 \right)}
\renewcommand{\epsilon}{\varepsilon}
\renewcommand{\phi}{\varphi}


\renewcommand{\iff}{\Leftrightarrow}

\DeclareMathOperator{\arccot}{arccot}
\DeclareMathOperator{\arcsec}{arcsec}
\DeclareMathOperator{\arccsc}{arccsc}
\DeclareMathOperator{\si}{Si}
\DeclareMathOperator{\scal}{scal}
\DeclareMathOperator{\sign}{sign}


%% \newcommand{\tightoverset}[2]{% for arrow vec
%%   \mathop{#2}\limits^{\vbox to -.5ex{\kern-0.75ex\hbox{$#1$}\vss}}}
\newcommand{\arrowvec}[1]{{\overset{\rightharpoonup}{#1}}}
%\renewcommand{\vec}[1]{\arrowvec{\mathbf{#1}}}
\renewcommand{\vec}[1]{{\overset{\boldsymbol{\rightharpoonup}}{\mathbf{#1}}}\hspace{0in}}

\newcommand{\point}[1]{\left(#1\right)} %this allows \vector{ to be changed to \vector{ with a quick find and replace
\newcommand{\pt}[1]{\mathbf{#1}} %this allows \vec{ to be changed to \vec{ with a quick find and replace
\newcommand{\Lim}[2]{\lim_{\point{#1} \to \point{#2}}} %Bart, I changed this to point since I want to use it.  It runs through both of the exercise and exerciseE files in limits section, which is why it was in each document to start with.

\DeclareMathOperator{\proj}{\mathbf{proj}}
\newcommand{\veci}{{\boldsymbol{\hat{\imath}}}}
\newcommand{\vecj}{{\boldsymbol{\hat{\jmath}}}}
\newcommand{\veck}{{\boldsymbol{\hat{k}}}}
\newcommand{\vecl}{\vec{\boldsymbol{\l}}}
\newcommand{\uvec}[1]{\mathbf{\hat{#1}}}
\newcommand{\utan}{\mathbf{\hat{t}}}
\newcommand{\unormal}{\mathbf{\hat{n}}}
\newcommand{\ubinormal}{\mathbf{\hat{b}}}

\newcommand{\dotp}{\bullet}
\newcommand{\cross}{\boldsymbol\times}
\newcommand{\grad}{\boldsymbol\nabla}
\newcommand{\divergence}{\grad\dotp}
\newcommand{\curl}{\grad\cross}
%\DeclareMathOperator{\divergence}{divergence}
%\DeclareMathOperator{\curl}[1]{\grad\cross #1}
\newcommand{\lto}{\mathop{\longrightarrow\,}\limits}

\renewcommand{\bar}{\overline}

\colorlet{textColor}{black}
\colorlet{background}{white}
\colorlet{penColor}{blue!50!black} % Color of a curve in a plot
\colorlet{penColor2}{red!50!black}% Color of a curve in a plot
\colorlet{penColor3}{red!50!blue} % Color of a curve in a plot
\colorlet{penColor4}{green!50!black} % Color of a curve in a plot
\colorlet{penColor5}{orange!80!black} % Color of a curve in a plot
\colorlet{penColor6}{yellow!70!black} % Color of a curve in a plot
\colorlet{fill1}{penColor!20} % Color of fill in a plot
\colorlet{fill2}{penColor2!20} % Color of fill in a plot
\colorlet{fillp}{fill1} % Color of positive area
\colorlet{filln}{penColor2!20} % Color of negative area
\colorlet{fill3}{penColor3!20} % Fill
\colorlet{fill4}{penColor4!20} % Fill
\colorlet{fill5}{penColor5!20} % Fill
\colorlet{gridColor}{gray!50} % Color of grid in a plot

\newcommand{\surfaceColor}{violet}
\newcommand{\surfaceColorTwo}{redyellow}
\newcommand{\sliceColor}{greenyellow}




\pgfmathdeclarefunction{gauss}{2}{% gives gaussian
  \pgfmathparse{1/(#2*sqrt(2*pi))*exp(-((x-#1)^2)/(2*#2^2))}%
}


%%%%%%%%%%%%%
%% Vectors
%%%%%%%%%%%%%

%% Simple horiz vectors
\renewcommand{\vector}[1]{\left\langle #1\right\rangle}


%% %% Complex Horiz Vectors with angle brackets
%% \makeatletter
%% \renewcommand{\vector}[2][ , ]{\left\langle%
%%   \def\nextitem{\def\nextitem{#1}}%
%%   \@for \el:=#2\do{\nextitem\el}\right\rangle%
%% }
%% \makeatother

%% %% Vertical Vectors
%% \def\vector#1{\begin{bmatrix}\vecListA#1,,\end{bmatrix}}
%% \def\vecListA#1,{\if,#1,\else #1\cr \expandafter \vecListA \fi}

%%%%%%%%%%%%%
%% End of vectors
%%%%%%%%%%%%%

%\newcommand{\fullwidth}{}
%\newcommand{\normalwidth}{}



%% makes a snazzy t-chart for evaluating functions
%\newenvironment{tchart}{\rowcolors{2}{}{background!90!textColor}\array}{\endarray}

%%This is to help with formatting on future title pages.
\newenvironment{sectionOutcomes}{}{}



%% Flowchart stuff
%\tikzstyle{startstop} = [rectangle, rounded corners, minimum width=3cm, minimum height=1cm,text centered, draw=black]
%\tikzstyle{question} = [rectangle, minimum width=3cm, minimum height=1cm, text centered, draw=black]
%\tikzstyle{decision} = [trapezium, trapezium left angle=70, trapezium right angle=110, minimum width=3cm, minimum height=1cm, text centered, draw=black]
%\tikzstyle{question} = [rectangle, rounded corners, minimum width=3cm, minimum height=1cm,text centered, draw=black]
%\tikzstyle{process} = [rectangle, minimum width=3cm, minimum height=1cm, text centered, draw=black]
%\tikzstyle{decision} = [trapezium, trapezium left angle=70, trapezium right angle=110, minimum width=3cm, minimum height=1cm, text centered, draw=black]


\title{Inverses of functions}

\begin{document}
\begin{abstract}
	Functions are inverses if they compose to the identity function
\end{abstract}
\maketitle

\begin{question}
	Let $f(x) = x^3+1$. Let $a$ be the input into $f$ which generates an output of $28$.
	
	\begin{hint}
		 $f(a) = 28$
	\end{hint}
	\begin{hint}
		\begin{align*}
			a^3+1 &= 28\\
			a^3 &= 27\\
			a &= 3 
		\end{align*}
	\end{hint}
	  \[a = \answer{3}\] 
\end{question}

The situation in the question above is very common.  We often need to know which inputs for a function generate a given output.

\begin{question}
		Let $f(x) = x^3+1$ again. Let $a$ be the input into $f$ which generates an output of $b$.
	
	\begin{hint}
		 $f(a) = 28$
	\end{hint}
	\begin{hint}
		\begin{align*}
			a^3+1 &= b\\
			a^3 &= b-1\\
			a &= \sqrt[3]{b-1} 
		\end{align*}
	\end{hint}
	  \[a = \answer{(b-1)^(1/3)}\] 
\end{question}

In the last question, you found a way to find the input $a$ which generates any desired output $b$ for the function $f$.  This is what inverse functions are all about!

\begin{question}
	Let $f(x) = x^3+1$ and $g(x) = \sqrt[3]{x-1}$.  What is $f(g(x))$?  How about $g(f(x))$?
	
	\begin{question}
		\begin{hint}
			\begin{align*}
				f(g(x)) &= f(\sqrt[3]{x-1})\\
				&= (\sqrt[3]{x-1})^3+1\\
				&=(x-1)+1\\
				&=x
			\end{align*}
		\end{hint}
	\[f(g(x)) = \answer{x}\]
	\end{question}
	
	\begin{question}
	\begin{hint}
			\begin{align*}
				g(f(x)) &= g(x^3+1)\\
				&= \sqrt[3]{(x^3+1) - 1}\\
				&=\sqrt[3]{x^3}\\
				&=x
			\end{align*}
		\end{hint}
	\[g(f(x)) = \answer{x}\]
	\end{question}
\end{question}

In the example above, $f$ and $g$ exactly undo each other!  For example, $f(3) = 28$, and $g(28) = 3$.  So $g$ ``undoes" $f$, and $f$ also ``undoes'' $g$.  We say these functions are \textbf{inverses} of each other.

\begin{definition}
	Let $f$ be a function with domain $A$ and range $B$, and let $g$ be a function with domain $B$ and range $A$.  We say that $f$ and $g$ are \textbf{inverses} of each other if $f(g(b)) = b$ for all $b$ in $B$, and also $g(f(a)) = a$ for all $a$ in $A$. 
	
	Sometimes we write $g = f^{-1}$ in this case.  So we could rephrase the conditions as $f(f^{-1}(x)) = x$ and $f^{-1}(f(x)) = x$.  
\end{definition}

\begin{warning}
	\[f^{-1}(x) \neq \frac{1}{f(x)}\]
	
	This is an incredibly common mistake.  Admittedly, the notation is misleading, but that is the fault of tradition.  Get used to it!

\end{warning}

\begin{question}
	Suppose you know that $f(3) = 7$.
	
	\begin{hint}
		Intuitively, an inverse function should take an output and return the input.  So $f^{-1}(7) = 3$.  
		
		If we want to be more rigorous, we could work from the definition.  
		
		\[f^{-7}(7) = f^{-1}(f(3)) = 3\]
	\end{hint}
	
	\[f^{-1}(7) = \answer{3}\]
	
\end{question}

\begin{example}
	The formula for converting a temperature from Celsius to Fahrenheit is 
	
	\[F = \frac{9}{5} C + 32\]
	
	In the language of functions, we could say $f(t) = \frac{9}{5} t + 32$ is the function which takes Celsius measurements and converts them to Fahrenheit measurements of temperature.
	
	What if we had a Fahrenheit measurement, and wanted to find the corresponding Celsius measurement?  We would have to find the inverse function of $f$.   We can do so as follows:
	
	\begin{align*}
		f(f^{-1}(t)) &= t \text{ by the definition of inverse functions}\\
		\frac{9}{5} f^{-1}(t)+32 &= t \text{ by the  rule for $f$}\\
		\frac{9}{5} f^{-1}(t)&= t -32\\
		f^{-1}(t) = \frac{5}{9}(t - 32)
	\end{align*}
	
	So $f^{-1}(t) = \frac{5}{9}(t - 32)$ is the inverse function of $f$, which converts a Fahrenheit measurement back into a Celsius measurement.  For example,  $f(0) = 32$, and $f^{-1}(32) = 0$, which shows that $0$ Celsius  is $32$ Fahrenheight (and vica versa)!
\end{example}

\begin{question}
	Let $f(x) = 2x+7$.
	
	\begin{hint}
		$f(f^{-1}(x))=x$
	\end{hint}
	\begin{hint}
		\begin{align*}
			2f^{-1}(x)+7 &= x\\
			2f^{-1}(x) &= x-7\\
			f^{-1}(x) &= \frac{1}{2}(x-7)
		\end{align*}
	\end{hint}
	\[
	f^{-1}(x) = \answer{(1/2)(x-7)}
	\]
\end{question}

\begin{question}
	Let $f(x) = x^2$ with domain $\RR$ and $g(x = \sqrt{x})$ with domain $[0,\infty]$.  Are $f$ and $g$ inverses of each other?
	\begin{multipleChoice}
		\choice{Yes}
		\choice[correct]{No}
	\end{multipleChoice}
	\feedback{Note that $f$ has domain $\RR$ and range $[0,\infty]$, while $g$ has domain $[0,\infty]$ and range $[0,\infty]$.  Already this tells us that the functions are not inverses, since they do not conform to the domain and range criteria.  Even more devastatingly, although we have $f(g(x)) = (\sqrt{x})^2 = x$, we do not have $g(f(x)) = x$, since for example $g(f(-2)) = \sqrt{(-2)^2} = \sqrt{4} = 2 \neq -2$.
	
	If the domain of $f$ had been restricted to only the non-negative real numbers $[0,\infty]$, then $f$ would have been the inverse of $g$.
	}
\end{question}

\begin{question}
	Let $f$ be a function.  If the point $(1,9)$ is on the graph of $f$, what point must be the the graph of $f^{-1}$?
	
	\[\left( \answer{9}, \answer{1} \right)\]
	
	\feedback{Since $f(1) = 9$, we must have $f^{-1}(f(1)) = 1$, so $f^{-1}(9) = 1$.  Thus $(9,1)$ is on the graph of $f^{-1}$.  This is a general rule.  If $(a,b)$ is on the graph of $f$, and $(b,a)$ will be the graph of $f^{-1}$. }
\end{question}

\begin{question}
	Let $f$ be a function, and imagine that the points $(2,3)$ and $(7,3)$ are both on its graph.  Could $f$ have an inverse function?
	
	\begin{multipleChoice}
		\choice{Yes}
		\choice[correct]{No}
	\end{multipleChoice}
	
	\feedback{$f$ could \textbf{not} have an inverse function.  Imagine that it did.  Then $f^{-1}(f(2)) = 2$ and $f^{-1}(f(7)) = 7$.  Then we have both $f^{-1}(3) = 2$ and $f^{-1}(3) = 7$.  Since a \textbf{function} cannot send the same input to two different outputs, $f$ must not have an inverse function}
\end{question}

The last question highlights something interesting about inverse functions.  If two different inputs into a function have the same output, there is no hope  of that function having an inverse, since a function can only have one output.  This leads us to the following definition:

\begin{definition}
A function $f:A \to B$ is said to be one to one (a.k.a $1-1$, a.k.a. injective) if $f(a_1) = f(a_2)$ implies that $a_1 = a_2$.  In words, $f$ is $1-1$ if each output generated by a unique input.
\end{definition}

\begin{question}
Which of the following functions are one to one?  Select all that apply.

\begin{selectAll}
\choice[correct]{$f(x) = x$}
\choice{$f(x) = x^2$}
\choice{$f(x) = x^3 - 4x$}
\choice[correct]{$f(x) = x^3+4$}
\end{selectAll}

\end{question}

\begin{question}
\end{question}
\end{document}