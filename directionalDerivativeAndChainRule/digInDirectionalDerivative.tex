\documentclass{ximera}

%\usepackage{todonotes}
%\usepackage{mathtools} %% Required for wide table Curl and Greens
%\usepackage{cuted} %% Required for wide table Curl and Greens
\newcommand{\todo}{}

\usepackage{esint} % for \oiint
\ifxake%%https://math.meta.stackexchange.com/questions/9973/how-do-you-render-a-closed-surface-double-integral
\renewcommand{\oiint}{{\large\bigcirc}\kern-1.56em\iint}
\fi


\graphicspath{
  {./}
  {ximeraTutorial/}
  {basicPhilosophy/}
  {functionsOfSeveralVariables/}
  {normalVectors/}
  {lagrangeMultipliers/}
  {vectorFields/}
  {greensTheorem/}
  {shapeOfThingsToCome/}
  {dotProducts/}
  {partialDerivativesAndTheGradientVector/}
  {../productAndQuotientRules/exercises/}
  {../normalVectors/exercisesParametricPlots/}
  {../continuityOfFunctionsOfSeveralVariables/exercises/}
  {../partialDerivativesAndTheGradientVector/exercises/}
  {../directionalDerivativeAndChainRule/exercises/}
  {../commonCoordinates/exercisesCylindricalCoordinates/}
  {../commonCoordinates/exercisesSphericalCoordinates/}
  {../greensTheorem/exercisesCurlAndLineIntegrals/}
  {../greensTheorem/exercisesDivergenceAndLineIntegrals/}
  {../shapeOfThingsToCome/exercisesDivergenceTheorem/}
  {../greensTheorem/}
  {../shapeOfThingsToCome/}
  {../separableDifferentialEquations/exercises/}
  {vectorFields/}
}

\newcommand{\mooculus}{\textsf{\textbf{MOOC}\textnormal{\textsf{ULUS}}}}

\usepackage{tkz-euclide}\usepackage{tikz}
\usepackage{tikz-cd}
\usetikzlibrary{arrows}
\tikzset{>=stealth,commutative diagrams/.cd,
  arrow style=tikz,diagrams={>=stealth}} %% cool arrow head
\tikzset{shorten <>/.style={ shorten >=#1, shorten <=#1 } } %% allows shorter vectors

\usetikzlibrary{backgrounds} %% for boxes around graphs
\usetikzlibrary{shapes,positioning}  %% Clouds and stars
\usetikzlibrary{matrix} %% for matrix
\usepgfplotslibrary{polar} %% for polar plots
\usepgfplotslibrary{fillbetween} %% to shade area between curves in TikZ
\usetkzobj{all}
\usepackage[makeroom]{cancel} %% for strike outs
%\usepackage{mathtools} %% for pretty underbrace % Breaks Ximera
%\usepackage{multicol}
\usepackage{pgffor} %% required for integral for loops



%% http://tex.stackexchange.com/questions/66490/drawing-a-tikz-arc-specifying-the-center
%% Draws beach ball
\tikzset{pics/carc/.style args={#1:#2:#3}{code={\draw[pic actions] (#1:#3) arc(#1:#2:#3);}}}



\usepackage{array}
\setlength{\extrarowheight}{+.1cm}
\newdimen\digitwidth
\settowidth\digitwidth{9}
\def\divrule#1#2{
\noalign{\moveright#1\digitwidth
\vbox{\hrule width#2\digitwidth}}}





\newcommand{\RR}{\mathbb R}
\newcommand{\R}{\mathbb R}
\newcommand{\N}{\mathbb N}
\newcommand{\Z}{\mathbb Z}

\newcommand{\sagemath}{\textsf{SageMath}}


%\renewcommand{\d}{\,d\!}
\renewcommand{\d}{\mathop{}\!d}
\newcommand{\dd}[2][]{\frac{\d #1}{\d #2}}
\newcommand{\pp}[2][]{\frac{\partial #1}{\partial #2}}
\renewcommand{\l}{\ell}
\newcommand{\ddx}{\frac{d}{\d x}}

\newcommand{\zeroOverZero}{\ensuremath{\boldsymbol{\tfrac{0}{0}}}}
\newcommand{\inftyOverInfty}{\ensuremath{\boldsymbol{\tfrac{\infty}{\infty}}}}
\newcommand{\zeroOverInfty}{\ensuremath{\boldsymbol{\tfrac{0}{\infty}}}}
\newcommand{\zeroTimesInfty}{\ensuremath{\small\boldsymbol{0\cdot \infty}}}
\newcommand{\inftyMinusInfty}{\ensuremath{\small\boldsymbol{\infty - \infty}}}
\newcommand{\oneToInfty}{\ensuremath{\boldsymbol{1^\infty}}}
\newcommand{\zeroToZero}{\ensuremath{\boldsymbol{0^0}}}
\newcommand{\inftyToZero}{\ensuremath{\boldsymbol{\infty^0}}}



\newcommand{\numOverZero}{\ensuremath{\boldsymbol{\tfrac{\#}{0}}}}
\newcommand{\dfn}{\textbf}
%\newcommand{\unit}{\,\mathrm}
\newcommand{\unit}{\mathop{}\!\mathrm}
\newcommand{\eval}[1]{\bigg[ #1 \bigg]}
\newcommand{\seq}[1]{\left( #1 \right)}
\renewcommand{\epsilon}{\varepsilon}
\renewcommand{\phi}{\varphi}


\renewcommand{\iff}{\Leftrightarrow}

\DeclareMathOperator{\arccot}{arccot}
\DeclareMathOperator{\arcsec}{arcsec}
\DeclareMathOperator{\arccsc}{arccsc}
\DeclareMathOperator{\si}{Si}
\DeclareMathOperator{\scal}{scal}
\DeclareMathOperator{\sign}{sign}


%% \newcommand{\tightoverset}[2]{% for arrow vec
%%   \mathop{#2}\limits^{\vbox to -.5ex{\kern-0.75ex\hbox{$#1$}\vss}}}
\newcommand{\arrowvec}[1]{{\overset{\rightharpoonup}{#1}}}
%\renewcommand{\vec}[1]{\arrowvec{\mathbf{#1}}}
\renewcommand{\vec}[1]{{\overset{\boldsymbol{\rightharpoonup}}{\mathbf{#1}}}\hspace{0in}}

\newcommand{\point}[1]{\left(#1\right)} %this allows \vector{ to be changed to \vector{ with a quick find and replace
\newcommand{\pt}[1]{\mathbf{#1}} %this allows \vec{ to be changed to \vec{ with a quick find and replace
\newcommand{\Lim}[2]{\lim_{\point{#1} \to \point{#2}}} %Bart, I changed this to point since I want to use it.  It runs through both of the exercise and exerciseE files in limits section, which is why it was in each document to start with.

\DeclareMathOperator{\proj}{\mathbf{proj}}
\newcommand{\veci}{{\boldsymbol{\hat{\imath}}}}
\newcommand{\vecj}{{\boldsymbol{\hat{\jmath}}}}
\newcommand{\veck}{{\boldsymbol{\hat{k}}}}
\newcommand{\vecl}{\vec{\boldsymbol{\l}}}
\newcommand{\uvec}[1]{\mathbf{\hat{#1}}}
\newcommand{\utan}{\mathbf{\hat{t}}}
\newcommand{\unormal}{\mathbf{\hat{n}}}
\newcommand{\ubinormal}{\mathbf{\hat{b}}}

\newcommand{\dotp}{\bullet}
\newcommand{\cross}{\boldsymbol\times}
\newcommand{\grad}{\boldsymbol\nabla}
\newcommand{\divergence}{\grad\dotp}
\newcommand{\curl}{\grad\cross}
%\DeclareMathOperator{\divergence}{divergence}
%\DeclareMathOperator{\curl}[1]{\grad\cross #1}
\newcommand{\lto}{\mathop{\longrightarrow\,}\limits}

\renewcommand{\bar}{\overline}

\colorlet{textColor}{black}
\colorlet{background}{white}
\colorlet{penColor}{blue!50!black} % Color of a curve in a plot
\colorlet{penColor2}{red!50!black}% Color of a curve in a plot
\colorlet{penColor3}{red!50!blue} % Color of a curve in a plot
\colorlet{penColor4}{green!50!black} % Color of a curve in a plot
\colorlet{penColor5}{orange!80!black} % Color of a curve in a plot
\colorlet{penColor6}{yellow!70!black} % Color of a curve in a plot
\colorlet{fill1}{penColor!20} % Color of fill in a plot
\colorlet{fill2}{penColor2!20} % Color of fill in a plot
\colorlet{fillp}{fill1} % Color of positive area
\colorlet{filln}{penColor2!20} % Color of negative area
\colorlet{fill3}{penColor3!20} % Fill
\colorlet{fill4}{penColor4!20} % Fill
\colorlet{fill5}{penColor5!20} % Fill
\colorlet{gridColor}{gray!50} % Color of grid in a plot

\newcommand{\surfaceColor}{violet}
\newcommand{\surfaceColorTwo}{redyellow}
\newcommand{\sliceColor}{greenyellow}




\pgfmathdeclarefunction{gauss}{2}{% gives gaussian
  \pgfmathparse{1/(#2*sqrt(2*pi))*exp(-((x-#1)^2)/(2*#2^2))}%
}


%%%%%%%%%%%%%
%% Vectors
%%%%%%%%%%%%%

%% Simple horiz vectors
\renewcommand{\vector}[1]{\left\langle #1\right\rangle}


%% %% Complex Horiz Vectors with angle brackets
%% \makeatletter
%% \renewcommand{\vector}[2][ , ]{\left\langle%
%%   \def\nextitem{\def\nextitem{#1}}%
%%   \@for \el:=#2\do{\nextitem\el}\right\rangle%
%% }
%% \makeatother

%% %% Vertical Vectors
%% \def\vector#1{\begin{bmatrix}\vecListA#1,,\end{bmatrix}}
%% \def\vecListA#1,{\if,#1,\else #1\cr \expandafter \vecListA \fi}

%%%%%%%%%%%%%
%% End of vectors
%%%%%%%%%%%%%

%\newcommand{\fullwidth}{}
%\newcommand{\normalwidth}{}



%% makes a snazzy t-chart for evaluating functions
%\newenvironment{tchart}{\rowcolors{2}{}{background!90!textColor}\array}{\endarray}

%%This is to help with formatting on future title pages.
\newenvironment{sectionOutcomes}{}{}



%% Flowchart stuff
%\tikzstyle{startstop} = [rectangle, rounded corners, minimum width=3cm, minimum height=1cm,text centered, draw=black]
%\tikzstyle{question} = [rectangle, minimum width=3cm, minimum height=1cm, text centered, draw=black]
%\tikzstyle{decision} = [trapezium, trapezium left angle=70, trapezium right angle=110, minimum width=3cm, minimum height=1cm, text centered, draw=black]
%\tikzstyle{question} = [rectangle, rounded corners, minimum width=3cm, minimum height=1cm,text centered, draw=black]
%\tikzstyle{process} = [rectangle, minimum width=3cm, minimum height=1cm, text centered, draw=black]
%\tikzstyle{decision} = [trapezium, trapezium left angle=70, trapezium right angle=110, minimum width=3cm, minimum height=1cm, text centered, draw=black]


\author{Bart Snapp \and Jim Talamo}

\outcome{Compute directional derivatives.}
\outcome{Use the directional derivative to show that the gradient vector points in the initial direction of greatest increase for the function.}

\title[Dig-In:]{The directional derivative}

\begin{document}
\begin{abstract}
  We introduce a way of analyzing the rate of change in a given
  direction.
\end{abstract}
\maketitle

<<<<<<< HEAD

For functions of several variables, we can ask how changing only one of the inputs changes the outputs of the function.  We've seen that the partial derivatives measure this rate of change.  For functions of two variables, we've also seen that there is a geometric interpretation of this if we consider the surface $z=F(x,y)$ and the partial derivatives at a point $(a,b)$ in the domain of the function.  
Let's imagine that our surface is a hill and consider $F_y(a,b)$.  This tell us we should hold \wordChoice{\choice[correct]{$x$}\choice{$y$}} constant.  The goal is the same as before; we consider the path \wordChoice{\choice[correct]{$x=a$}\choice{$y=b$}} in the domain and wish to consider the curve on the surface $z=F(x,y)$ that lies above it.  A picture is shown below.

\begin{image}
  \begin{tikzpicture}
    \begin{axis}[tick label style={font=\scriptsize},axis on top,
	axis lines=center,
	view={110}{25},
	name=myplot,
	xtick=1.2,
	xticklabels={$a$},
        ytick=.8,
        yticklabels={$b$},
        ztick=\empty,
	ymin=-1,ymax=2.2,
	xmin=-.8,xmax=2.2,
	zmin=-.5, zmax=5.1,
	every axis x label/.style={at={(axis cs:\pgfkeysvalueof{/pgfplots/xmax},0,0)},xshift=-1pt,yshift=-4pt},
	xlabel={\scriptsize $x$},
	every axis y label/.style={at={(axis cs:0,\pgfkeysvalueof{/pgfplots/ymax},0)},xshift=5pt,yshift=-3pt},
	ylabel={\scriptsize $y$},
	every axis z label/.style={at={(axis cs:0,0,\pgfkeysvalueof{/pgfplots/zmax})},xshift=0pt,yshift=4pt},
	zlabel={\scriptsize $z$},
        colormap/cool,
      ]
%      \addplot3[gray,domain=0:2,samples y=0,dashed] ({1*cos(45)},{1*sin(45)},x); %% line for z
%      \addplot3[gray,domain=0:cos(45),samples y=0,dashed] ({x},{1*sin(45)},0); %% line for x
%      \addplot3[gray,domain=0:cos(45),samples y=0,dashed] ({sin(45)},{x},0); %% line for y
\addplot3[domain=-.8:2.2,y domain=-.8:2.5,mesh,samples y=25,very thin,z buffer=sort,  samples=25,] (x,y,{2-x^3+y^2});   
\addplot3 [very thick,penColor, smooth,domain=-.89:3,samples=20,samples y=0] ({1.2*cos(45)},{x},{2-(1.2*cos(45))^3+x^2});
\addplot3 [very thick,penColor2, dashed,domain=-1:2.5,samples=20,samples y=0] ({1.2*cos(45)},{x},{0});
     

%points they plotted in the xy plane and point on curve%
	%\node[below,black!50!white] at (axis cs:{-.5*cos(45)},{-1*sin(45)},0) {$(-.5,-1)$};
		      \filldraw [black] (axis cs:{1.2*cos(45)},{-1*sin(45)},0) circle (2.5pt); 
		      \filldraw [black] (axis cs:{1.2*cos(45)},{-1*sin(45)},1.889) circle (2.5pt);  
	       	      \addplot3[black,thick, domain=0:1.889,samples y=0,dashed] ({1.2*cos(45)},{-1*sin(45)},x); %% line for z 
      	%\node[below,black!50!white] at (axis cs:{0*cos(45)},{0*sin(45)},0) {$(0,0)$};
		  \filldraw [black] (axis cs:{1.2*cos(45)},{0*sin(45)},0) circle (2.5pt); 
		  \filldraw [black] (axis cs:{1.2*cos(45)},{0*sin(45)},1.389) circle (2.5pt);  
		  \addplot3[black,thick, domain=0:1.389,samples y=0,dashed] ({1.2*cos(45)},{0*sin(45)},x); %% line for z 
      	\node[below,black] at (axis cs:{1.5*cos(45)},{1*sin(45)},0) {$(a,b)$};
		  \filldraw [black] (axis cs:{1.2*cos(45)},{1*sin(45)},0) circle (2.5pt); 
		  \filldraw [black] (axis cs:{1.2*cos(45)},{1*sin(45)},1.889) circle (2.5pt); 
		  \addplot3[black,thick, domain=0:1.889,samples y=0,dashed] ({1.2*cos(45)},{1*sin(45)},x); %% line for z  
      	%\node[below,black!50!white] at (axis cs:{1*cos(45)},{2*sin(45)},0) {$(1,2)$};
	  	\filldraw [black] (axis cs:{1.2*cos(45)},{2*sin(45)},0) circle (2.5pt); 
		\filldraw [black] (axis cs:{1.2*cos(45)},{2*sin(45)},3.389) circle (2.5pt);  
		\addplot3[black,thick, domain=0:3.389,samples y=0,dashed] ({1.2*cos(45)},{2*sin(45)},x); %% line for z 
      	%\node[below,black!50!white] at (axis cs:{1.5*cos(45)},{3*sin(45)},0) {$(1.5,3)$};
	  	\filldraw [black] (axis cs:{1.2*cos(45)},{3*sin(45)},0) circle (2.5pt); 
		\filldraw [black] (axis cs:{1.2*cos(45)},{3*sin(45)},5.889) circle (2.5pt);
		\addplot3[black,thick, domain=0:5.889,samples y=0,dashed] ({1.2*cos(45)},{3*sin(45)},x); %% line for z   		

    \end{axis}
  \end{tikzpicture}
  
\end{image}


We can now interpret $F_y(a,b)$ as either:

\begin{itemize}
\item the slope of the hill if we walk along it in a direction parallel to the \wordChoice{\choice{$x$-axis}\choice[correct]{$y$-axis}}. 
\item the instantaneous rate of change of $F(x,y)$ at $(a,b)$ as we approach $(a,b)$ along the line $x=a$.
\end{itemize}
We have a similar interpretation of $F_x(a,b)$.
However, there is no reason that we \emph{must} approach $(a,b)$ along a line that is parallel to one of the coordinate axes.  One such line is shown in the image below.

\begin{image}
  \begin{tikzpicture}
    \begin{axis}[tick label style={font=\scriptsize},axis on top,
	axis lines=center,
	view={110}{25},
	name=myplot,
	xtick=1.2,
	xticklabels={$a$},
        ytick=.8,
        yticklabels={$b$},
        ztick=\empty,
	ymin=-1,ymax=2.2,
	xmin=-.8,xmax=2.2,
	zmin=-.5, zmax=5.1,
	every axis x label/.style={at={(axis cs:\pgfkeysvalueof{/pgfplots/xmax},0,0)},xshift=-1pt,yshift=-4pt},
	xlabel={\scriptsize $x$},
	every axis y label/.style={at={(axis cs:0,\pgfkeysvalueof{/pgfplots/ymax},0)},xshift=5pt,yshift=-3pt},
	ylabel={\scriptsize $y$},
	every axis z label/.style={at={(axis cs:0,0,\pgfkeysvalueof{/pgfplots/zmax})},xshift=0pt,yshift=4pt},
	zlabel={\scriptsize $z$},
        colormap/cool,
      ]
%      \addplot3[gray,domain=0:2,samples y=0,dashed] ({1*cos(45)},{1*sin(45)},x); %% line for z
%      \addplot3[gray,domain=0:cos(45),samples y=0,dashed] ({x},{1*sin(45)},0); %% line for x
%      \addplot3[gray,domain=0:cos(45),samples y=0,dashed] ({sin(45)},{x},0); %% line for y
\addplot3[domain=-.8:2.2,y domain=-.8:2.5,mesh,samples y=25,very thin,z buffer=sort,  samples=25,] (x,y,{2-x^3+y^2});   
\addplot3 [very thick,penColor, smooth,domain=-.6:.7,samples=20,samples y=0] ({3*x+1.2*cos(45)},{-x+1*sin(45)},{2-(2*x)^3+x^2});
\addplot3 [very thick,penColor2, dashed,domain=-.6:.7,samples=20,samples y=0] ({3*x+1.2*cos(45)},{-x+1*sin(45)},{0});
     

%points they plotted in the xy plane and point on curve%
	%\node[below,black!50!white] at (axis cs:{-.5*cos(45)},{-1*sin(45)},0) {$(-.5,-1)$};
		      \filldraw [black] (axis cs:{1.2+1.2*cos(45)},{-.4+sin(45)},0) circle (2.5pt); 
		      \filldraw [black] (axis cs:{1.2+1.2*cos(45)},{-.4+sin(45)},1.589) circle (2.5pt);  
	       	      \addplot3[black,thick, domain=0:1.589,samples y=0,dashed] ({1.2+1.2*cos(45)},{-.4+sin(45)},x); %% line for z 
      	%\node[below,black!50!white] at (axis cs:{0*cos(45)},{0*sin(45)},0) {$(0,0)$};
		  \filldraw [black] (axis cs:{.6+1.2*cos(45)},{-.2+sin(45)},0) circle (2.5pt); 
		  \filldraw [black] (axis cs:{.6+1.2*cos(45)},{-.2+sin(45)},1.944) circle (2.5pt);  
		  \addplot3[black,thick, domain=0:1.944,samples y=0,dashed] ({.6+1.2*cos(45)},{-.2+sin(45)},x); %% line for z 
      	\node[below,black] at (axis cs:{1.5*cos(45)},{1.2*sin(45)},0) {$(a,b)$};
		  \filldraw [black] (axis cs:{1.2*cos(45)},{1*sin(45)},0) circle (2.5pt); 
		  \filldraw [black] (axis cs:{1.2*cos(45)},{1*sin(45)},1.989) circle (2.5pt); 
		  \addplot3[black,thick, domain=0:1.989,samples y=0,dashed] ({1.2*cos(45)},{1*sin(45)},x); %% line for z  
      	%\node[below,black!50!white] at (axis cs:{1*cos(45)},{2*sin(45)},0) {$(1,2)$};
	  	\filldraw [black] (axis cs:{-.6+1.2*cos(45)},{.2+sin(45)},0) circle (2.5pt); 
		\filldraw [black] (axis cs:{-.6+1.2*cos(45)},{.2+sin(45)},2.096) circle (2.5pt);  
		\addplot3[black,thick, domain=0:2.096,samples y=0,dashed] ({-.6+1.2*cos(45)},{.2+sin(45)},x); %% line for z 
      	%\node[below,black!50!white] at (axis cs:{1.5*cos(45)},{3*sin(45)},0) {$(1.5,3)$};
	  	\filldraw [black] (axis cs:{-1.2+1.2*cos(45)},{.4+sin(45)},0) circle (2.5pt); 
		\filldraw [black] (axis cs:{-1.2+1.2*cos(45)},{.4+sin(45)},2.691) circle (2.5pt);
		\addplot3[black,thick, domain=0:2.691,samples y=0,dashed] ({-1.2+1.2*cos(45)},{.4+sin(45)},x); %% line for z   		

      	\node[below,penColor2] at (axis cs:{.2},{1.2},0) {$l$};
    \end{axis}
  \end{tikzpicture}
  
\end{image}

Indeed, once we are at a point on the surface above $\vec{a}=\vector{a,b}$, there are actually many different directions that we can travel along the hill.  Let's consider a line $l$ that passes through $\vec{a}$ as our path in the domain.  Our rate of change will be given by
\[
\textrm{ rate of change } = \frac{\textrm{rise}}{\textrm{run}}  ,
\] 
where the ``run'' is the distance travelled along the line $l$ and the ``rise'' is the corresponding change in the $z$-values of the function.  Since we are ultimately concerned about a curve on the surface, a good first step is to parameterize the line in the domain, then use the function to find a parametric description of the curve on the surface above $l$.

In order to make computing the run most efficiently, we pick a unit vector $\uvec{u}$ in the direction of $l$.  We'll see how to do this in the next example, but we can always start at $\vec{a}$ and draw a unit vector that extends from $\vec{a}$ along $l$.  

To find a parameterization of $l$, note that $\uvec{u}$ is parallel to $l$ and $\vec{a}$ is a point on the line, so letting $h$ denote the parameter, a description of $l$ is given by

\[
\vec{l}(h) = h \uvec{u} + \vec{a}
\]

For the sake of example, let $h>0$; a similar argument can be given if $h<0$.  One convenient consequence of using a unit vector in the direction of $l$ is that the ``run'', which is the distance between $\vec{a}$ and $\vec{a}+h\uvec{u}$ is simply

\[
\textrm{``run'' } = \left|\vec{a}+h\uvec{u} - \vec{a}\right| = |h \uvec{u}| = |h| |\uvec{u}| = h
\]
since $h>0$ and $\uvec{u}$ is a unit vector.

The ``rise'' is computed by noting that it is the corresponding change in $z$-values.

\[
\textrm{ ``rise'' } = F(\vec{a}+h\uvec{u})-F(\vec{a})
\] 

These are shown in the image below.



\begin{image}
  \begin{tikzpicture}
    \begin{axis}[tick label style={font=\scriptsize},axis on top,
	axis lines=center,
	view={110}{25},
	name=myplot,
	xtick=1.2,
	xticklabels={},
        ytick=.8,
        yticklabels={},
        ztick=\empty,
	ymin=-1,ymax=2.2,
	xmin=-.8,xmax=2.2,
	zmin=-.5, zmax=5.1,
	every axis x label/.style={at={(axis cs:\pgfkeysvalueof{/pgfplots/xmax},0,0)},xshift=-1pt,yshift=-4pt},
	xlabel={\scriptsize $x$},
	every axis y label/.style={at={(axis cs:0,\pgfkeysvalueof{/pgfplots/ymax},0)},xshift=5pt,yshift=-3pt},
	ylabel={\scriptsize $y$},
	every axis z label/.style={at={(axis cs:0,0,\pgfkeysvalueof{/pgfplots/zmax})},xshift=0pt,yshift=4pt},
	zlabel={\scriptsize $z$},
        colormap/cool,
      ]
%      \addplot3[gray,domain=0:2,samples y=0,dashed] ({1*cos(45)},{1*sin(45)},x); %% line for z
%      \addplot3[gray,domain=0:cos(45),samples y=0,dashed] ({x},{1*sin(45)},0); %% line for x
%      \addplot3[gray,domain=0:cos(45),samples y=0,dashed] ({sin(45)},{x},0); %% line for y
\addplot3[domain=-.8:2.2,y domain=-.8:2.5,mesh,samples y=25,very thin,z buffer=sort,  samples=25,] (x,y,{2-x^3+y^2});   
\addplot3 [very thick,penColor, smooth,domain=-.6:.7,samples=20,samples y=0] ({3*x+1.2*cos(45)},{-x+1*sin(45)},{2-(2*x)^3+x^2});
\addplot3 [very thick,penColor2, dashed,domain=-.6:.7,samples=20,samples y=0] ({3*x+1.2*cos(45)},{-x+1*sin(45)},{0});
     

%points they plotted in the xy plane and point on curve%

      	\node[below,black] at (axis cs:{1.5*cos(45)},{1.2*sin(45)},0) {$\vec{a}$};
		  \filldraw [black] (axis cs:{1.2*cos(45)},{1*sin(45)},0) circle (2.5pt); 
		  \filldraw [black] (axis cs:{1.2*cos(45)},{1*sin(45)},1.989) circle (2.5pt); 
		  \addplot3[black,thick, domain=0:1.989,samples y=0,dashed] ({1.2*cos(45)},{1*sin(45)},x); %% line for z  
%      	\node[below,black] at (axis cs:{.3*cos(45)},{2*sin(45)},0) {$\vec{a}+h\uvec{u}$};
	  	\filldraw [black] (axis cs:{-.6+1.2*cos(45)},{.2+sin(45)},0) circle (2.5pt); 
		\filldraw [black] (axis cs:{-.6+1.2*cos(45)},{.2+sin(45)},2.096) circle (2.5pt);  
		\addplot3[black,thick, domain=0:2.096,samples y=0,dashed] ({-.6+1.2*cos(45)},{.2+sin(45)},x); %% line for z 


      	\node[below,black] at (axis cs:{.1},{1.4},0) {$\vec{a}+h\uvec{u}$};
    \end{axis}
  \end{tikzpicture}
  
\end{image}

To find the instantaneous rate of change, we take the limit $h \to 0$ (since $F$ is differentiable, it can be shown this limit must exist).  We call the result the \emph{directional derivative of $F$ at $\vec{a}$ in the direction $\uvec{u}$} and will henceforth denote this by $D_{\uvec{u}}F(\vec{a})$.  

=======
One way to think about partial derivatives is as follows: Given a
differentiable function $F:\R^2\to\R$,
\begin{itemize}
  \item $\pp[F]{x}$ gives the rate of change of $F$ in the $\veci$
    direction.
  \item $\pp[F]{y}$ gives the rate of change of $F$ in the $\vecj$
    direction.
\end{itemize}

Thus if we want to know the rate of change in \textit{any} direction, we could use the scalar projection:
\[
\scal_\uvec{u} \vector{\pp[F]{x},\pp[F]{y}} = \grad F(x,y) \dotp \uvec{u}
\]
this will give the rate of change of $F$ in the direction $\uvec{u}$.
This is a new type of derivative called the directional derivative.
>>>>>>> b7dfc1aec8f1a24ec5acb1c56659c37ee990e8ba
\begin{definition}
  Suppose that $F: \R^2 \to \R$ is a differentiable function.  Given a \emph{unit} vector $\uvec{u}$ and a point $\vec{a}$ in the domain of $F$, we define the \dfn{directional derivative} \emph{ of $F$ at $\vec{a}$ in the direction $\uvec{u}$}, denoted by $D_{\uvec{u}} F(\vec{a})$ is
  \[
  D_\uvec{u}F(\vec{a}) = \lim_{h \to 0} \frac{F(\vec{a}+h\uvec{u})-F(\vec{a})}{h},
  \]
<<<<<<< HEAD
  
which is the instantaneous rate of change of $F$ at $\vec{a}$ as we approach $\vec{a}$ in the direction of $\uvec{u}$.
  \end{definition}
  
  There's actually a quick way to compute this limit by using the definition of differentiability.  We first give the result and save the derivation of the formula until the end of the section.
  
\begin{theorem}
 Suppose that $F: \R^2 \to \R$ is a differentiable function and let $\uvec{u}$ be a unit vector.   Then, we compute $D_{\uvec{u}}F(\vec{a})$ by the formula
 
\[
  D_\uvec{u}F(\vec{a}) = \grad F(\vec{x})\dotp \uvec{u}.
\]
\end{theorem}


\begin{remark}
While writing down the definition above might seem tricky, notice that the qualitative idea of finding the instantaneous rate of change as a limit is exactly the same as what we did with functions of a single variable.  It's really just an old problem in a new setting!  The difficulty lies in using the tools we have been developing to write down the actual limit that must be computed.
\end{remark}

%Interactive Figure!%

%Examples


%The directional derivative tells us how $F$ changes if we move one
%direction. If we want to make $D_\uvec{u}(F)$ as large as
%possible, it makes sense to let $\uvec{u}$ be the direction of
%gradient since the dot product
%
%is largest when $\uvec{u}$ is in the same direction as $\grad F$.
%
%This tells us something very important about the gradient vector:
%
%\begin{quote}
%Given a function $F:\R^n\to\R$ \textbf{the gradient vector points in the
%initial direction of greatest increase} for $F$.
%\end{quote}
%\begin{onlineOnly}
%  We can see this in the interactive below. 
%  \begin{center}
%    \geogebra{wd5mrudh}{800}{600} %https://ggbm.at/wd5mrudh
%  \end{center}
%  The gradient at each point is a vector pointing in the
%  $(x,y)$-plane. The direction of the vector tells us which initial
%  direction to leave a point in the $(x,y)$-plane in order to find the
%  greatest increase for $F$.
%\end{onlineOnly}

%
%
%
%\begin{example}
%  Let
%  \[
%  F(x,y) = -x^2+2x-y^2+2y+1.
%  \]
%  Compute and interpret $\grad F(1,1)$.
%  \begin{explanation}
%    Write with me
%    \[
%    \grad F(x,y) = \vector{\answer[given]{-2x+2},\answer[given]{-2y+2}}.
%    \]
%    However, we see $\grad F(1,1) = \vector{\answer[given]{0},\answer[given]{0}}$. Since there is no
%    initial direction of greatest increase, we must be at a local
%    maximum for the function. Indeed we are, behold:
%    \begin{image}
%      \begin{tikzpicture}
%        \begin{axis}%
%          [tick label style={font=\scriptsize},axis on top,
%	    axis lines=center,
%	    view={100}{25},
%	    name=myplot,
%	    %xtick=\empty,
%	    %ytick={5},
%	    %ztick={.7,-.7},
%	    minor xtick=1,
%	    minor ytick=1,
%	    ymin=-.1,ymax=2.1,
%	    xmin=-.1,xmax=2.1,
%	    zmin=-.5, zmax=3.5,
%	    every axis x label/.style={at={(axis cs:\pgfkeysvalueof{/pgfplots/xmax},0,0)},xshift=-5pt,yshift=-1pt},
%	    xlabel={\scriptsize $x$},
%	    every axis y label/.style={at={(axis cs:0,\pgfkeysvalueof{/pgfplots/ymax},0)},xshift=4pt,yshift=-4pt},
%	    ylabel={\scriptsize $y$},
%	    every axis z label/.style={at={(axis cs:0,0,\pgfkeysvalueof{/pgfplots/zmax})},xshift=0pt,yshift=4pt},
%	    zlabel={\scriptsize $z$},
%            colormap/cool
%	  ]
%          
%          \addplot3[domain=0:2,,y domain=0:2,
%            mesh,samples=25,samples y=25,very thin,z buffer=sort] {-x^2-y^2+2*x+2*y+1};
%          \filldraw [black,] (axis cs:1,1,3) circle (1pt);
%        \end{axis}
%      \end{tikzpicture}
%    \end{image}
%    Ah! The point $(1,1)$ lies at the top of a paraboloid. In all
%    directions, the instantaneous rate of change is $0$.
%  \end{explanation}
%\end{example}

\begin{example}
Find $D_{\uvec{u}}F(2,1)$ for the function $F(x,y) = x^2-3xy+4y^2+7$ and $\uvec{u}$ is parallel to the line $2x-y=3$.

\begin{explanation}
We want to use the result $D_{\uvec{u}}F(2,1) =- \grad{F}(2,1) \dotp \uvec{u}$.  To do this we need two quantities : $\uvec{u}$ and $\grad{F}(2,1)$.

\begin{itemize}
\item Finding $\uvec{u}$

Vectors have both a magnitude and a direction.  We've seen that it is much more challenging to find a vector in the appropriate direction than it is to scale a vector appropriately, so let's start by finding a vector $\vec{u}$ parallel to the line $2x-y=3$.  

There are many ways we can do this and one such way is to parameterize the line.  Since we can explicitly find $y=\answer[given]{2x-3}$, we set $x(t)=t$ and $y(t) = \answer[given]{2t-3}$.  A parameterization is thus

\[
\vec{p}(t) = \vector{x(t),y(t)} = \vector{\answer[given]{t},\answer[given]{2t-3}}
\]
Now, we need a $t$-value for which $x(t) = 2$, $y(t)=1$.  By inspecting the first component of the parameterization, we find $t=\answer[given]{2}$.  Thus, a vector parallel to the line will be $\vec{p}'(2)$.  We note

\[
\vec{p}'(t) = \vector{1,2}
\]

So $\vec{p}'(2) = \vector{\answer[given]{1},\answer[given]{2}}$.  This is the vector $\vec{u}$ we will use to be parallel to the line.  We now note that $\vec{u}$ \wordChoice{\choice{is}\choice[correct]{is not}} a unit vector.  

We find the unit vector the usual way by computing 
\[
\uvec{u} = \frac{\vec{u}}{|u|} = \frac{\vector{1,2}}{\answer[given]{\sqrt{5}}}.
\]

\item Finding $\grad{F}(2,1)$.

Since $F(x,y) = x^2-3xy+4y^2+7$, we find $F_x = \answer[given]{2x-3y}$ and $F_y = -3x+8y$, so 

\[
\grad{F}(x,y) = \vector{\answer[given]{2x-3y},\answer[given]{-3x+8y}}
\]
Thus, $\grad{F}(2,1) = \vector{\answer[given]{1},\answer[given]{2}}$.

\end{itemize}

Now, using the formula $D_{\uvec{u}}F(2,1) =- \grad{F}(2,1) \dotp \uvec{u}$ gives $D_{\uvec{u}}F(2,1) = \vector{1,2} \dotp \vector{\frac{1}{\sqrt{5}},\frac{2}{\sqrt{5}} } = \answer[given]{\frac{5}{\sqrt{5}}}$.

\end{explanation}

Now that we have defined and worked with the directional derivative, what does it tell us?
\begin{selectAll}
\choice[correct]{The instantaneous rate of change of $F(x,y)$ at the point $(1,2)$ as we approach it in the direction parallel to $\vector{\frac{1}{\sqrt{5}},\frac{2}{\sqrt{5}}}$.}
\choice[correct]{The slope of the tangent line to the curve on the surface $z= x^2-3xy+4y^2+7$ above the line $2x-y=3$ in its domain.}
\choice{The normal vector to the surface at the point.}
\choice{The slope of the tangent plane.}
\end{selectAll}
\begin{feedback}
The first two choices are two ways of thinking about the directional derivative.  Since the directional derivative is a scalar, not a vector, the third option cannot be correct.  The fourth option is also not correct because there is no ``slope'' associated to a plane.
\end{feedback}

\end{example}

%In the previous example, we picked a particular direction along which to approach $(2,1)$, but there are many others.  Experiment with the interactive figure below.
%
%IMAGE
%
%While you are experimenting, what happens to the slopes as $\uvec{u}$ is varied?  Can you find a reasonable estimate for the largest the slope can be and the direction in which it occurs?  Can you find directions in which there is no change in the heights ($z$-values)?

\section{Initial directions of change}
If we think of the surface $z=F(x,y)$ as a hill and are standing at the point $(a,b,F(a,b))$, we've seen how walking up the hill

Given a particular point $(a,b,F(a,b))$ on the surface where $\grad{F}(a,b) \neq \vec{0}$, there are a few questions we can ask.

\begin{itemize}

\item In which initial direction should we travel from $(a,b,F(a,b))$ if we want to head up the hill the fastest? What if we want to go down the hill?

\item In which direction should we travel if we do not want our current elevation to change?

\end{itemize}

In order to answer these questions, we have to consider \emph{every} possible direction we can travel from the point $(a,b,F(a,b))$ along the surface. This may seem daunting, but remember that we have a nice formula for the directional derivative as a dot product, and dot products capture important geometric information. Recall that given vectors $\vec{v} = \vector{v_1, \ldots , v_n}$ and $\vec{w} = \vector{w_1, \ldots , w_n}$ in $\R^n$, there are actually two equivalent ways to find $\vec{v} \dotp \vec{w}$.

\[
\vec{v} \dotp \vec{w} = v_1w_1+\ldots + v_nw_n = |\vec{v}||\vec{w}|\cos(\theta),
=======
  where $\uvec{u}$ is a unit vector.
\end{definition}
The directional derivative tells us how $F$ changes if we move some
direction. If we want to make $D_\uvec{u}(F)$ as large as possible, it
makes sense to let $\uvec{u}$ be the direction of gradient since the
dot product
\[
\grad F(\vec{x})\dotp \uvec{u}
>>>>>>> b7dfc1aec8f1a24ec5acb1c56659c37ee990e8ba
\]

where $\theta$ is the interior angle between the vectors. Using the second expression with $\vec{v} = \grad{F}(a,b)$ and $\vec{w} = \uvec{u}$ gives

\[
D_{\uvec{u}} F(a,b) = \grad{F}(a,b) \dotp \uvec{u} = |\grad{F}(a,b)||\uvec{u}|\cos(\theta).
\]

Since $\uvec{u}$ is a unit vector, $|\uvec{u}|=\answer[given]{1}$, and thus

\[
D_{\uvec{u}} F(a,b) = |\grad{F}(a,b)|\cos(\theta).
\]

Let's tackle the first question. To find the initial direction of greatest increase, we need to find a choice for $\uvec{u}$ that makes $D_{\uvec{u}} F(a,b)$ as large as possible. Since we are focusing only at the point on the surface above $(a,b)$, so $\grad{F}(a,b)$ \wordChoice{\choice{does}\choice[correct]{does not}} depend on the direction $\uvec{u}$ along which we travel from $(a,b,F(a,b))$. Since $\theta$ is the angle between the gradient $\grad{F}(a,b)$ and $\uvec{u}$, this \wordChoice{\choice[correct]{does}\choice{does not}} depend on the direction $\uvec{u}$. Since $\cos(\theta) \leq 1$, we find that the largest $D_{\uvec{u}}F(a,b)$ can be occurs for the angle $\theta$ between $\grad{F}(a,b)$ and $\uvec{u}$ that makes $\cos(\theta)=1$. This occurs when $\theta=\answer[given]{0}$, which means a vector $\uvec{u}$ that points in the direction of greatest increase parallel to the gradient.  This is important enough that we say it again.

\begin{quote}
Given a differentiable function $F:\R^2 \to \R$, the gradient $\grad{F}(a,b)$ points in the initial direction of greatest increase.
\end{quote}

As another upshot, we actually know exactly what the maximum rate of increase is at $(a,b)$ too. Since $D_{\uvec{u}}F(a,b) = |\grad{F}(a,b)| \cos(\theta)$ and $\cos(\theta) = 1$ when $\uvec{u}$ is in the same direction as $\grad{F}(a,b)$, we have that the maximum rate of change is $D_{\uvec{u}}F(a,b) = |\grad{F}(a,b)|$.

We can use similar logic to determine that the maximum rate of decrease, or the ``most negative'' rate of change occurs in the direction $\uvec{u}$ opposite the gradient, and that this most negative rate of change is $D_{\uvec{u}}F(a,b) = -|\grad{F}(a,b)|$.

To tackle the direction of no change, we need to find the directions $\uvec{u}$ for which $D_{\uvec{u}} F(a,b) =0$. Once again, the formula $D_{\uvec{u}}F(a,b) = \grad{F}(a,b) \dotp \uvec{u}$ comes to the rescue. Setting $D_{\uvec{u}} F(a,b) =0$ gives that $= \grad{F}(a,b) \dotp \uvec{u} = 0$, which means that the directions of no change are \wordChoice{\choice{parallel}\choice[correct]{orthogonal}} to $\grad{F}(a,b)$.

Let's summarize these results.

\begin{theorem}

Given a function $F:\R^2 \to \R$ and a point $(a,b)$ at which $\grad{F}(a,b) \neq \vec{0}$, we have the following.

\begin{itemize}

\item The initial direction of maximum increase, or the direction vector $\uvec{u}$ for which $D_{\uvec{u}} F(a,b)$ is the largest, is in the direction of $\grad{F}(a,b)$. The actual maximum rate of change is $\grad{F}(a,b)$.

\item The initial direction of maximum decrease, or the direction vector $\uvec{u}$ for which $D_{\uvec{u}} F(a,b)$ is the most negative, is in the direction of $-\grad{F}(a,b)$. The actual minimum rate of change is $\grad{F}(a,b)$.

\item The initial directions of no change, or the direction vector $\uvec{u}$ for which $D_{\uvec{u}} F(a,b)=0$ are orthogonal to $\grad{F}(a,b)$.

\end{itemize}

\end{theorem}

\begin{remark}

We talk about an ``initial'' direction because, as we head along the hill, this direction might change due to the fact that the hill is not flat.

\end{remark}

\begin{example}
<<<<<<< HEAD

Suppose that $F(x,y) = \sin(xy)+y^2$. Give a unit vector in the initial direction of maximum increase, decrease, and no change at $(0,1)$. What are the corresponding maximum and minimum rates of change at $(0,1)$?

\begin{explanation}

We first compute the gradient. Since $F(x,y) = \sin(xy)+y^2$, 

\begin{itemize}

\item $F_x(x,y) = \answer[given]{y\cos(xy)}$ so $F_x(0,1) = 1$.

\item $F_y(x,y) = x\cos(xy)+2y$, so $F_y(0,1) = \answer[given]{2}$.

\end{itemize}

Thus, $\grad{F}(0,1) = \vector{\answer[given]{1},\answer[given]{2}}$.

We can now use this to find the requested directions and rates.

\begin{itemize}

\item The initial direction of maximum increase is in the \wordChoice{\choice[correct]{same}\choice{opposite}\choice{orthogonal}} direction of the gradient. Since $|\grad{F}(0,1) = \sqrt{(1)^2+(2)^2} = \sqrt{5}$, a unit vector $\uvec{u}$ in the direction of maximum increase is $\uvec{u} = \vector{\answer[given]{\frac{1}{\sqrt{5}},\answer[given]{\frac{2}{\sqrt{5}}}}}$ and the maximum rate of change is $|\grad{F}(0,1)| = \sqrt{5}$.

\item The initial direction of maximum decrease is in the \wordChoice{\choice[correct]{same}\choice[correct]{opposite}\choice{orthogonal}} direction of the gradient. A unit vector $\uvec{u}$ in the direction of maximum increase is $\uvec{u} = \vector{-\frac{1}{\sqrt{5}},-\frac{2}{\sqrt{5}}}$ and the maximum rate of decrease is $-|\grad{F}(0,1)| = -\sqrt{5}$.

\item There are two unit vectors in the initial direction of no change. To see why, note that $\grad{F}(0,1) = \vector{1,2}$, so both the vectors $\vec{u}_1 =\vector{-2,1}$ and $\vec{u}_2 =\vector{2,-1}$ are orthogonal to $\vec{u}$ (Notice that for two dimensional vectors, we can always find a vector orthogonal to a given one by inspection; just flip the components and negate one of them).

The magnitude of both of these vectors is $\answer[given]{\sqrt{5}}$, so the two unit vectors in the initial direction of no change are $\uvec{u}_1 = \vector{-\frac{2}{\sqrt{5}},\frac{1}{\sqrt{5}}}$ and $\uvec{u}_2 = \vector{\frac{2}{\sqrt{5}},-\frac{1}{\sqrt{5}}}$.

\end{itemize}

\end{explanation}

\end{example}

Let's consider a visual example now that ties together some concepts that have been discussed so far.

\begin{example}

Level curve - TO BE ADDED

=======
  Let
  \[
  F(x,y) = -x^2+2x-y^2+2y+1.
  \]
  Compute and interpret $\grad F(1,1)$.
  \begin{explanation}
    Write with me
    \[
    \grad F(x,y) = \vector{\answer[given]{-2x+2},\answer[given]{-2y+2}}.
    \]
    However, we see $\grad F(1,1) =
    \vector{\answer[given]{0},\answer[given]{0}}$. Since there is no
    initial direction of greatest increase, we must be at a local
    maximum for the function. Indeed we are, behold:
    \begin{image}
      \begin{tikzpicture}
        \begin{axis}%
          [tick label style={font=\scriptsize},axis on top,
	    axis lines=center,
	    view={100}{25},
	    name=myplot,
	    %xtick=\empty,
	    %ytick={5},
	    %ztick={.7,-.7},
	    minor xtick=1,
	    minor ytick=1,
	    ymin=-.1,ymax=2.1,
	    xmin=-.1,xmax=2.1,
	    zmin=-.5, zmax=3.5,
	    every axis x label/.style={at={(axis cs:\pgfkeysvalueof{/pgfplots/xmax},0,0)},xshift=-5pt,yshift=-1pt},
	    xlabel={\scriptsize $x$},
	    every axis y label/.style={at={(axis cs:0,\pgfkeysvalueof{/pgfplots/ymax},0)},xshift=4pt,yshift=-4pt},
	    ylabel={\scriptsize $y$},
	    every axis z label/.style={at={(axis cs:0,0,\pgfkeysvalueof{/pgfplots/zmax})},xshift=0pt,yshift=4pt},
	    zlabel={\scriptsize $z$},
            colormap/cool
	  ]
          
          \addplot3[domain=0:2,,y domain=0:2,
            mesh,samples=25,samples y=25,very thin,z buffer=sort] {-x^2-y^2+2*x+2*y+1};
          \filldraw [black,] (axis cs:1,1,3) circle (1pt);
        \end{axis}
      \end{tikzpicture}
    \end{image}
    Ah! The point $(1,1)$ lies at the top of a paraboloid. In all
    directions, the instantaneous rate of change is $0$.
  \end{explanation}
>>>>>>> b7dfc1aec8f1a24ec5acb1c56659c37ee990e8ba
\end{example}

\section{The formula for $D_{\uvec{u}}F(\vec{a})$}
We conclude this section by giving the derivation of the formula $D_{\uvec{u}}F(\vec{a})=\grad{F}(\vec{a})\dotp\uvec{u}$.  Since our function $F$ is differentiable, we know that when we ``zoom in'' on the graph of the surface $z=F(x,y)$, the surface looks like its tangent plane, $z=T_{\vec{a}}(\vec{x})$,  which is mathematized in the definition of differentiability below.

\[
\lim_{\vec{x} \to \vec{a} } \frac{F(\vec{x})-T_{\vec{a}}(\vec{x})}{|\vec{x}-\vec{a}|} = 0
\]

We have seen that we can use the gradient to write the formula for the tangent plane as $T_{\vec{a}}(\vec{x}) = F(\vec{a}) + \grad{F}(\vec{a}) \dotp (\vec{x}-\vec{a})$.  Substituting into the above limit gives

\[
\lim_{\vec{x} \to \vec{a} } \frac{F(\vec{x})-F(\vec{a}) - \grad{F}(\vec{a}) \dotp (\vec{x}-\vec{a})}{|\vec{x}-\vec{a}|}=0 
\]

Now, recall that the directional derivative $D_{\uvec{u}}F(\vec{a})$ requires that we approach $\vec{a}$ along the line $\vec{l}(t) = \vec{a}+t\uvec{u}$.  Since the above limit exists, the result holds along \emph{any} path along which $\vec{x} \to \vec{a}$, so it certainly holds along this path.  Letting $\vec{x}$ approach $\vec{a}$ along this path is found by setting $\vec{x} = \vec{a}+t\uvec{u}$, and the limit $\vec{x} \to \vec{a}$ is now found by taking $t \to 0$.  To simplify, we will consider as $t \to 0^+$; the argument for the other sided limit is very similar.  Now, we update our limit along the chosen path.

\begin{align*}
\lim_{\vec{x} \to \vec{a} } \frac{F(\vec{x})-F(\vec{a}) - \grad{F}(\vec{a}) \dotp (\vec{x}-\vec{a})}{|\vec{x}-\vec{a}|} &= \lim_{t \to 0} \frac{F( \vec{a}+t\uvec{u})-F(\vec{a}) + \grad{F}(\vec{a}) \dotp ( \vec{a}+t\uvec{u}-\vec{a})}{| \vec{a}+t\uvec{u}-\vec{a}|}\\
&= \lim_{t \to 0} \frac{F( \vec{a}+t\uvec{u})-F(\vec{a}) - \grad{F}(\vec{a}) \dotp (t\uvec{u}}{| t\uvec{u}|}//
&= \lim_{t \to 0} \frac{F( \vec{a}+t\uvec{u})-F(\vec{a}) - \grad{F}(\vec{a}) \dotp (t\uvec{u})}{| t\uvec{u}|}\\
&= \lim_{t \to 0} \frac{F( \vec{a}+t\uvec{u})-F(\vec{a})}{t} - \grad{F}(\vec{a}) \dotp \uvec{u}\\
\end{align*}
where in the last step, we have used the fact that $|\uvec{u}|=1$ since $\uvec{u}$ is a unit vector.

Recalling that this limit is 0 in the first place gives

\[
\lim_{t \to 0} \frac{F( \vec{a}+t\uvec{u})-F(\vec{a})}{t} - \grad{F}(\vec{a}) \dotp \uvec{u} = 0,
\]
and since by definition, $D_{\uvec{u}}(\vec{a}) = \lim_{t \to 0} \frac{F( \vec{a}+t\uvec{u})-F(\vec{a})}{t}$, we have

\[
D_{\uvec{u}} f(\vec{a}) - \grad{F}(\vec{a}) \dotp \uvec{u} = 0.
\]

We may thus conclude that $D_{\uvec{u}} f(\vec{a}) = \grad{F}(\vec{a}) \dotp \uvec{u}$.

\end{document}
