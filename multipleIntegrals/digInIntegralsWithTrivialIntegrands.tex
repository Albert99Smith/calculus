\documentclass{ximera}

%\usepackage{todonotes}
%\usepackage{mathtools} %% Required for wide table Curl and Greens
%\usepackage{cuted} %% Required for wide table Curl and Greens
\newcommand{\todo}{}

\usepackage{esint} % for \oiint
\ifxake%%https://math.meta.stackexchange.com/questions/9973/how-do-you-render-a-closed-surface-double-integral
\renewcommand{\oiint}{{\large\bigcirc}\kern-1.56em\iint}
\fi


\graphicspath{
  {./}
  {ximeraTutorial/}
  {basicPhilosophy/}
  {functionsOfSeveralVariables/}
  {normalVectors/}
  {lagrangeMultipliers/}
  {vectorFields/}
  {greensTheorem/}
  {shapeOfThingsToCome/}
  {dotProducts/}
  {partialDerivativesAndTheGradientVector/}
  {../productAndQuotientRules/exercises/}
  {../normalVectors/exercisesParametricPlots/}
  {../continuityOfFunctionsOfSeveralVariables/exercises/}
  {../partialDerivativesAndTheGradientVector/exercises/}
  {../directionalDerivativeAndChainRule/exercises/}
  {../commonCoordinates/exercisesCylindricalCoordinates/}
  {../commonCoordinates/exercisesSphericalCoordinates/}
  {../greensTheorem/exercisesCurlAndLineIntegrals/}
  {../greensTheorem/exercisesDivergenceAndLineIntegrals/}
  {../shapeOfThingsToCome/exercisesDivergenceTheorem/}
  {../greensTheorem/}
  {../shapeOfThingsToCome/}
  {../separableDifferentialEquations/exercises/}
  {vectorFields/}
}

\newcommand{\mooculus}{\textsf{\textbf{MOOC}\textnormal{\textsf{ULUS}}}}

\usepackage{tkz-euclide}\usepackage{tikz}
\usepackage{tikz-cd}
\usetikzlibrary{arrows}
\tikzset{>=stealth,commutative diagrams/.cd,
  arrow style=tikz,diagrams={>=stealth}} %% cool arrow head
\tikzset{shorten <>/.style={ shorten >=#1, shorten <=#1 } } %% allows shorter vectors

\usetikzlibrary{backgrounds} %% for boxes around graphs
\usetikzlibrary{shapes,positioning}  %% Clouds and stars
\usetikzlibrary{matrix} %% for matrix
\usepgfplotslibrary{polar} %% for polar plots
\usepgfplotslibrary{fillbetween} %% to shade area between curves in TikZ
\usetkzobj{all}
\usepackage[makeroom]{cancel} %% for strike outs
%\usepackage{mathtools} %% for pretty underbrace % Breaks Ximera
%\usepackage{multicol}
\usepackage{pgffor} %% required for integral for loops



%% http://tex.stackexchange.com/questions/66490/drawing-a-tikz-arc-specifying-the-center
%% Draws beach ball
\tikzset{pics/carc/.style args={#1:#2:#3}{code={\draw[pic actions] (#1:#3) arc(#1:#2:#3);}}}



\usepackage{array}
\setlength{\extrarowheight}{+.1cm}
\newdimen\digitwidth
\settowidth\digitwidth{9}
\def\divrule#1#2{
\noalign{\moveright#1\digitwidth
\vbox{\hrule width#2\digitwidth}}}





\newcommand{\RR}{\mathbb R}
\newcommand{\R}{\mathbb R}
\newcommand{\N}{\mathbb N}
\newcommand{\Z}{\mathbb Z}

\newcommand{\sagemath}{\textsf{SageMath}}


%\renewcommand{\d}{\,d\!}
\renewcommand{\d}{\mathop{}\!d}
\newcommand{\dd}[2][]{\frac{\d #1}{\d #2}}
\newcommand{\pp}[2][]{\frac{\partial #1}{\partial #2}}
\renewcommand{\l}{\ell}
\newcommand{\ddx}{\frac{d}{\d x}}

\newcommand{\zeroOverZero}{\ensuremath{\boldsymbol{\tfrac{0}{0}}}}
\newcommand{\inftyOverInfty}{\ensuremath{\boldsymbol{\tfrac{\infty}{\infty}}}}
\newcommand{\zeroOverInfty}{\ensuremath{\boldsymbol{\tfrac{0}{\infty}}}}
\newcommand{\zeroTimesInfty}{\ensuremath{\small\boldsymbol{0\cdot \infty}}}
\newcommand{\inftyMinusInfty}{\ensuremath{\small\boldsymbol{\infty - \infty}}}
\newcommand{\oneToInfty}{\ensuremath{\boldsymbol{1^\infty}}}
\newcommand{\zeroToZero}{\ensuremath{\boldsymbol{0^0}}}
\newcommand{\inftyToZero}{\ensuremath{\boldsymbol{\infty^0}}}



\newcommand{\numOverZero}{\ensuremath{\boldsymbol{\tfrac{\#}{0}}}}
\newcommand{\dfn}{\textbf}
%\newcommand{\unit}{\,\mathrm}
\newcommand{\unit}{\mathop{}\!\mathrm}
\newcommand{\eval}[1]{\bigg[ #1 \bigg]}
\newcommand{\seq}[1]{\left( #1 \right)}
\renewcommand{\epsilon}{\varepsilon}
\renewcommand{\phi}{\varphi}


\renewcommand{\iff}{\Leftrightarrow}

\DeclareMathOperator{\arccot}{arccot}
\DeclareMathOperator{\arcsec}{arcsec}
\DeclareMathOperator{\arccsc}{arccsc}
\DeclareMathOperator{\si}{Si}
\DeclareMathOperator{\scal}{scal}
\DeclareMathOperator{\sign}{sign}


%% \newcommand{\tightoverset}[2]{% for arrow vec
%%   \mathop{#2}\limits^{\vbox to -.5ex{\kern-0.75ex\hbox{$#1$}\vss}}}
\newcommand{\arrowvec}[1]{{\overset{\rightharpoonup}{#1}}}
%\renewcommand{\vec}[1]{\arrowvec{\mathbf{#1}}}
\renewcommand{\vec}[1]{{\overset{\boldsymbol{\rightharpoonup}}{\mathbf{#1}}}\hspace{0in}}

\newcommand{\point}[1]{\left(#1\right)} %this allows \vector{ to be changed to \vector{ with a quick find and replace
\newcommand{\pt}[1]{\mathbf{#1}} %this allows \vec{ to be changed to \vec{ with a quick find and replace
\newcommand{\Lim}[2]{\lim_{\point{#1} \to \point{#2}}} %Bart, I changed this to point since I want to use it.  It runs through both of the exercise and exerciseE files in limits section, which is why it was in each document to start with.

\DeclareMathOperator{\proj}{\mathbf{proj}}
\newcommand{\veci}{{\boldsymbol{\hat{\imath}}}}
\newcommand{\vecj}{{\boldsymbol{\hat{\jmath}}}}
\newcommand{\veck}{{\boldsymbol{\hat{k}}}}
\newcommand{\vecl}{\vec{\boldsymbol{\l}}}
\newcommand{\uvec}[1]{\mathbf{\hat{#1}}}
\newcommand{\utan}{\mathbf{\hat{t}}}
\newcommand{\unormal}{\mathbf{\hat{n}}}
\newcommand{\ubinormal}{\mathbf{\hat{b}}}

\newcommand{\dotp}{\bullet}
\newcommand{\cross}{\boldsymbol\times}
\newcommand{\grad}{\boldsymbol\nabla}
\newcommand{\divergence}{\grad\dotp}
\newcommand{\curl}{\grad\cross}
%\DeclareMathOperator{\divergence}{divergence}
%\DeclareMathOperator{\curl}[1]{\grad\cross #1}
\newcommand{\lto}{\mathop{\longrightarrow\,}\limits}

\renewcommand{\bar}{\overline}

\colorlet{textColor}{black}
\colorlet{background}{white}
\colorlet{penColor}{blue!50!black} % Color of a curve in a plot
\colorlet{penColor2}{red!50!black}% Color of a curve in a plot
\colorlet{penColor3}{red!50!blue} % Color of a curve in a plot
\colorlet{penColor4}{green!50!black} % Color of a curve in a plot
\colorlet{penColor5}{orange!80!black} % Color of a curve in a plot
\colorlet{penColor6}{yellow!70!black} % Color of a curve in a plot
\colorlet{fill1}{penColor!20} % Color of fill in a plot
\colorlet{fill2}{penColor2!20} % Color of fill in a plot
\colorlet{fillp}{fill1} % Color of positive area
\colorlet{filln}{penColor2!20} % Color of negative area
\colorlet{fill3}{penColor3!20} % Fill
\colorlet{fill4}{penColor4!20} % Fill
\colorlet{fill5}{penColor5!20} % Fill
\colorlet{gridColor}{gray!50} % Color of grid in a plot

\newcommand{\surfaceColor}{violet}
\newcommand{\surfaceColorTwo}{redyellow}
\newcommand{\sliceColor}{greenyellow}




\pgfmathdeclarefunction{gauss}{2}{% gives gaussian
  \pgfmathparse{1/(#2*sqrt(2*pi))*exp(-((x-#1)^2)/(2*#2^2))}%
}


%%%%%%%%%%%%%
%% Vectors
%%%%%%%%%%%%%

%% Simple horiz vectors
\renewcommand{\vector}[1]{\left\langle #1\right\rangle}


%% %% Complex Horiz Vectors with angle brackets
%% \makeatletter
%% \renewcommand{\vector}[2][ , ]{\left\langle%
%%   \def\nextitem{\def\nextitem{#1}}%
%%   \@for \el:=#2\do{\nextitem\el}\right\rangle%
%% }
%% \makeatother

%% %% Vertical Vectors
%% \def\vector#1{\begin{bmatrix}\vecListA#1,,\end{bmatrix}}
%% \def\vecListA#1,{\if,#1,\else #1\cr \expandafter \vecListA \fi}

%%%%%%%%%%%%%
%% End of vectors
%%%%%%%%%%%%%

%\newcommand{\fullwidth}{}
%\newcommand{\normalwidth}{}



%% makes a snazzy t-chart for evaluating functions
%\newenvironment{tchart}{\rowcolors{2}{}{background!90!textColor}\array}{\endarray}

%%This is to help with formatting on future title pages.
\newenvironment{sectionOutcomes}{}{}



%% Flowchart stuff
%\tikzstyle{startstop} = [rectangle, rounded corners, minimum width=3cm, minimum height=1cm,text centered, draw=black]
%\tikzstyle{question} = [rectangle, minimum width=3cm, minimum height=1cm, text centered, draw=black]
%\tikzstyle{decision} = [trapezium, trapezium left angle=70, trapezium right angle=110, minimum width=3cm, minimum height=1cm, text centered, draw=black]
%\tikzstyle{question} = [rectangle, rounded corners, minimum width=3cm, minimum height=1cm,text centered, draw=black]
%\tikzstyle{process} = [rectangle, minimum width=3cm, minimum height=1cm, text centered, draw=black]
%\tikzstyle{decision} = [trapezium, trapezium left angle=70, trapezium right angle=110, minimum width=3cm, minimum height=1cm, text centered, draw=black]


\outcome{Understand the definiton of a multiple integral.}

\outcome{Use iterated integrals to compute multiple integrals.}

\outcome{Apply Fubini's Theorem.}

\title[Dig-In:]{Integrals with trivial integrands}

\begin{document}
\begin{abstract}
  We study integrals over general regions by integrating $1$.
\end{abstract}
\maketitle


Now we will integrate over regions that are more complex than
rectangles and boxes. In particular we will find areas of regions
bounded by curves in $\R^2$, and volumes of regions bounded by
surfaces in $\R^3$. To do this we will use integrals whose integrand
is $1$. The \dfn{integrand} is the ``thing'' you are integrating:
\[
\iint_R \underbrace{F(x,y)}_{\text{integrand}} \d A
\]
Think about this: The integral
\[
\int_a^b \d x = \int_a^b 1 \d x = b-a
\]
computes the length between $a$ and $b$. This is because the area
between the constant curve of height one and the $x$-axis is just the
length of the curve! In a similar way:
\[
\iint_R \d A = \iint_R 1 \d A = \text{Area of $R$}
\]
Here you imagine a surface of constant height $1$ above a region $R$:
  \begin{image}
    \begin{tikzpicture}
      \begin{axis}%
        [
          tick label style={font=\scriptsize},%axis on top,
	  axis lines=center,
	  view={135}{25},
	  name=myplot,
	  %% xtick=\empty,
	  %% ytick=\empty,
	  %% ztick=\empty,
	  %% extra x ticks={1},
	  %% extra x tick labels={$a$},
	  %% extra y ticks={1},
	  %% extra y tick labels={$a$},
	  %% extra z ticks={1},
	  %% extra z tick labels={$h$},
	  ymin=-8,ymax=8,
	  xmin=-8,xmax=8,
	  zmin=-.1, zmax=8,
	  every axis x label/.style={at={(axis cs:\pgfkeysvalueof{/pgfplots/xmax},0,0)},xshift=-1pt,yshift=-4pt},
	  xlabel={\scriptsize $x$},
	  every axis y label/.style={at={(axis cs:0,\pgfkeysvalueof{/pgfplots/ymax},0)},xshift=5pt,yshift=-3pt},
	  ylabel={\scriptsize $y$},
	  every axis z label/.style={at={(axis cs:0,0,\pgfkeysvalueof{/pgfplots/zmax})},xshift=0pt,yshift=4pt},
	  zlabel={\scriptsize $z$},colormap/cool
	]
        
        \addplot3[smooth cycle,ultra thick, penColor] coordinates{
          (-4,6,0) (-7,4,0) (-5,1,0) (-6,-3,0) (1,-5,0) (7,1,0) (4,6,0) (0,5,0)};
        \addplot3[smooth cycle,ultra thick, penColor,fill=fill1] coordinates{
          (-4,6,1) (-7,4,1) (-5,1,1) (-6,-3,1) (1,-5,1) (7,1,1) (4,6,1) (0,5,1)};
        \node at (axis cs: 0,0,1.2) {$R$};
      \end{axis}
    \end{tikzpicture}
  \end{image}
The volume under the plane of constant height $1$ above the region $R$
is numerically equal to the area of $R$. In an entirely similar way, if $R$ is a subset of $\R^3$, then 
\[
\iiint_R \d V = \iiint_R 1 \d V= \text{Volume of $R$}
\]
In this section we will focus on integrals with a trivial integrand,
meaning it is $1$, and hence the section is called ``Integrals with
trivial integrands.''




\section{Double integrals and area}

We start with a third version of Fubini's Theorem. 

\begin{theorem}[Fubini's Theorem]\index{Fubini's Theorem}
  Let $R$ be a closed, bounded region in the $(x,y)$-plane and let
  $F(x,y)$ be a continuous function on $R$.
  \begin{itemize}
  \item If $R=\{(x,y):\text{$a\leq x\leq b$ and $g_1(x)\leq y\leq g_2(x)$}\}$ 
    \begin{image}
      \begin{tikzpicture}
        \begin{axis}[
          tick label style={font=\scriptsize},
          axis y line=middle,axis x line=middle,
          name=myplot,
          width=3in,
          height=2in,
	  xtick={1,4},
          xticklabels={$a$,$b$},
          ytick = \empty,
          %ytick={1,3},
          %yticklabels={$g$,$d$},
          ymin=-.2,ymax=4.2,%
	  xmin=-.2,xmax=5.2,
          rounded corners=.5pt
          ]
          \addplot [ultra thick,penColor, smooth] coordinates{(1,1) (2.5,2) (4,1.4)};
          \addplot [ultra thick,penColor, smooth] coordinates{(1,3) (2,4) (4,2.5)};
          \addplot [ultra thick,penColor, smooth] coordinates{(1,3) (1,1)};
          \addplot [ultra thick,penColor, smooth] coordinates{(4,1.4) (4,2.5)};
          
          \draw (axis cs: 2,3) node {$R$};
          \draw (axis cs: 2,1.2) node {$g_1(x)$};
          \draw (axis cs: 3,4) node {$g_2(x)$};
        \end{axis}
        
      \node [right] at (myplot.right of origin) {\scriptsize $x$};
      \node [above] at (myplot.above origin) {\scriptsize $y$};
      \end{tikzpicture}
    \end{image}    
    where $g_1$ and $g_2$ are continuous functions on $[a,b]$, then
    \[
    \iint_R F(x,y) \d A = \int_a^b\int_{g_1(x)}^{g_2(x)} F(x,y)\d y\d x.
    \]
  \item If $R=\{(x,y):\text{$c\leq y\leq d$ and $g_3(y)\leq x\leq g_4(y)$}\}$
    \begin{image}
      \begin{tikzpicture}
        \begin{axis}[
          tick label style={font=\scriptsize},
          axis y line=middle,axis x line=middle,
          name=myplot,
          width=3in,
          height=2in,
	  xtick=\empty,
          ytick={1,3},
          yticklabels={$c$,$d$},
          ymin=-.2,ymax=4.2,%
	  xmin=-.2,xmax=5.2,
          rounded corners=.5pt,
          clip=false,
          ]
          \addplot [ultra thick,penColor, smooth] coordinates{(1,1) (2.5,2.5) (2,3)};
          \addplot [ultra thick,penColor, smooth] coordinates{(4,1) (4.5,2) (4,3)};
          \addplot [ultra thick,penColor, smooth] coordinates{(1,1) (4,1)};
          \addplot [ultra thick,penColor, smooth] coordinates{(2,3) (4,3)};
          
          \draw (axis cs: 3,2) node {$R$};
          \draw (axis cs: 1.3,2) node {$g_3(y)$};
          \draw (axis cs: 5,2) node {$g_4(y)$};
        \end{axis}
        
      \node [right] at (myplot.right of origin) {\scriptsize $x$};
      \node [above] at (myplot.above origin) {\scriptsize $y$};
      \end{tikzpicture}
    \end{image}    
    where $g_3$ and $g_4$ are continuous functions on $[c,d]$, then
    \[
    \iint_R F(x,y)\d A = \int_c^d\int_{g_3(y)}^{g_4(y)} F(x,y)\d x\d y.
    \]
\end{itemize}
\end{theorem}

It is important to note that when using Fubini's Theorem, we must
always have numbers as the limits of the outer-most integral and
curves (note constant curves are numbers) as the limits of the inner-most integral:
\[
\int_{\text{number}}^{\text{number}}\int_{\text{curve}}^{\text{curve}} F(x,y) \d A
\]
Note, if we set $g_1$ and $g_2$ (or $g_3$ and $g_4$) to be constant
functions, we recover the version of Fubini's Theorem from the
previous section.

Whenever you learn a new technique, you should always ``try it out''
on a computation where you know the answer through a different
method. So let's used an integral to find the area of a triangle.

\begin{example}
  Set-up and evaluate an iterated integral that will compute the area
  of the region below via a double integral:
  \begin{image}
    \begin{tikzpicture}
      \begin{axis}[
          tick label style={font=\scriptsize},
          axis y line=middle,axis x line=middle,
          name=myplot,
          xtick={1,2,3},
          ytick={1,2,3},
          grid=major,
          grid style={dashed, gridColor},
	  ymin=-.5,ymax=2.5,%
	  xmin=-.5,xmax=3.5,
          rounded corners=.5pt
        ]
        \draw [ultra thick,penColor] (axis cs:1,1) -- (axis cs: 3,1)-- (axis cs: 1,2)  -- cycle;

        \draw [ultra thick,penColor] (axis cs:1,1) -- (axis cs: 3,1)-- (axis cs: 1,2)  -- cycle;
        
        \draw (axis cs: 1.5,1.3) node {$R$};
      \end{axis}
      
      \node [right] at (myplot.right of origin) {\scriptsize $x$};
      \node [above] at (myplot.above origin) {\scriptsize $y$};
    \end{tikzpicture}
  \end{image}
  \begin{explanation}
    Our region above can be defined by:
    \[
    R=\left\{(x,y):\text{$1\leq x\leq \answer[given]{3}$ and $1\leq y\leq \answer[given]{-x/2+5/2}$}\right\}
    \]
    The area of this region is given by, write with me, 
    \begin{align*}
      \int_{\answer[given]{1}}^{\answer[given]{3}}\int_{\answer[given]{1}}^{\answer[given]{-x/2+5/2}}\d y \d x &= \int_{\answer[given]{1}}^{\answer[given]{3}} \eval{\answer[given]{y}}_{\answer[given]{1}}^{\answer[given]{-x/2+5/2}} \d x\\
      &=  \int_{\answer[given]{1}}^{\answer[given]{3}} \left(\answer[given]{-x/2+3/2}\right) \d x\\
      &=  \eval{\answer[given]{-x^2/4+3x/2}}_{\answer[given]{1}}^{\answer[given]{3}}\\
      &= \answer[given]{1}. 
    \end{align*}
  \end{explanation}
\end{example}

\begin{example}
  Set-up and evaluate an iterated integral that will compute the area
  of the region below via a double integral:
  \begin{image}
    \begin{tikzpicture}
      \begin{axis}[
          tick label style={font=\scriptsize},axis y line=middle,axis x line=middle,name=myplot,axis on top,%
	  xtick={2,4,...,12},
	  ytick={-2,-4,-6,6,4,2},
	  ymin=-6.9,ymax=6.9,%
	  xmin=-.5,xmax=12.9%
        ]
        \draw (axis cs: 6,2) node {$R$}
	(axis cs: 4,4.5) node [rotate=23] {\scriptsize $x=y^2/3$};
        
        \addplot [penColor,ultra thick, smooth,domain=-6:6,samples=20] ({x^2/3},{x}) -- cycle;
      \end{axis}
      \node [right] at (myplot.right of origin) {\scriptsize $x$};
      \node [above] at (myplot.above origin) {\scriptsize $y$};
    \end{tikzpicture}
  \end{image}
  \begin{explanation}
    Our region above can be defined by:
    \[
    R=\{(x,y):\text{$-6\leq y\leq 6$ and $y^2/3\leq x\leq 12$}\}
    \]
    The area of this region is given by, write with me, 
    \begin{align*}
      \int_{\answer[given]{-6}}^{\answer[given]{6}}\int_{\answer[given]{y^2/3}}^{\answer[given]{12}}\d x \d y &= \int_{\answer[given]{-6}}^{\answer[given]{6}} \eval{\answer[given]{x}}_{\answer[given]{y^2/3}}^{\answer[given]{12}} \d y\\
      &=  \int_{\answer[given]{-6}}^{\answer[given]{6}} \left(\answer[given]{12-y^2/3}\right) \d y\\
      &=  \eval{\answer[given]{12y-y^3/9}}_{\answer[given]{-6}}^{\answer[given]{6}}\\
      &= \answer[given]{96}. 
    \end{align*}
  \end{explanation}
\end{example}

\subsection{Changing the order of integration}

We can change the order of integration. This can make a big difference
in the difficulty of the integral in question. This can be somewhat
challenging. If you have a graph of the region in question, this can
help quite a bit. If not, then a strategy is to
\begin{itemize}
\item Find absolute bounds for each variable.
\item Solve inequalities that will allow you to integrate:
  \[
  \int_{\text{number}}^{\text{number}}\int_{\text{curve}}^{\text{curve}} F(x,y) \d A
  \]
\end{itemize}

Let's see an example.

\begin{example}
  Consider:
  \[
  \int_1^3 \int_1^{-x/2 +5/2}\d y \d x
  \]
  Set-up an iterated integral that integrates with respect to $x$ and
  then integrates with respect to $y$.
  \begin{explanation}
    Start by finding absolute bounds for our variables. In this case:
    \begin{align*}
      1\le &x \le 3\\
      1\le &y \le -x/2 + 5/2 \le \answer{2}
    \end{align*}
    Now, write $x$ in terms of $y$. Write with me:
    \begin{align*}
      \answer{-3/2}\le &y -5/2 \le -x/2  \le \answer{-1/2}\\
      \answer{1/2}\le &x/2 \le 5/2-y  \le \answer{3/2}\\
      1\le & x\le \answer{5-2y}  \le 3
    \end{align*}
    We may now write our desired integral:
    \[
    \iint_R \d A = \int_{\answer{1}}^{\answer{2}}\int_{\answer{1}}^{\answer{5-2y}} \d x \d y
    \]
  \end{explanation}
\end{example}

Let's see another example.


\begin{example}
  Consider:
  \[
  \int_{-6}^6 \int_{y^2/3}^{12} \d x \d y 
  \]
  Set-up an iterated integral that computes the area of $R$ that
  integrates with respect to $y$ and then with respect to $x$.
  \begin{explanation}
    Start by finding absolute bounds for our variables. In this case:
    \begin{align*}
      \answer{0}\le y^2/3\le &x \le 12\\
      -6\le &y \le 6
    \end{align*}
    Now, write $y$ in terms of $x$. Write with me:
    \begin{align*}
      \answer{0}\le &y^2 \le 3x \le \answer{36}\\
      \answer{-6}\le -\sqrt{3x}\le &y \le \sqrt{3x} \le \answer{6}\\
    \end{align*}
    We may now write our desired integral:
    \[
    \iint_R \d A = \int_{\answer{0}}^{\answer{12}}\int_{\answer{-\sqrt{3x}}}^{\answer{\sqrt{3x}}} \d y \d x
    \]
  \end{explanation}
\end{example}

Sometimes when you switch the order of integration, you will need to
write the sum of iterated integrals. You can recognize this by the
fact that you won't be able to find a single curve to bound your inner
integral. This will be more clear with an example.

\begin{example}
  Consider the region:
  \begin{image}
    \begin{tikzpicture}
      \begin{axis}[
          tick label style={font=\scriptsize},
          axis y line=middle,axis x line=middle,
          name=myplot,
	  xtick={1,2,...,5},
	  ytick={1,2,...,4},
          grid=major,
          grid style={dashed, gridColor},
	  ymin=-.5,ymax=4.5,%
	  xmin=-.5,xmax=5.5,
          rounded corners=.5pt
        ]
        \draw [ultra thick,penColor] (axis cs:1,1) -- (axis cs: 1,4)-- (axis cs: 5,2)  -- cycle;
        
        \draw (axis cs: 2.5,2.5) node {$R$};
      \end{axis}
      
      \node [right] at (myplot.right of origin) {\scriptsize $x$};
      \node [above] at (myplot.above origin) {\scriptsize $y$};
    \end{tikzpicture}
  \end{image}
  Set-up iterated integrals that compute the area of $R$.
  \begin{explanation}
    First we'll integrate with respect to $y$ and then $x$. Here
    \[
    \answer[given]{1}\le x \le \answer[given]{5}
    \]
    and
    \[
    \answer[given]{(x+3)/4}\le y \le \answer[given]{(9-x)/2}
    \]
    so our integral is:
    \[
    \int_{\answer[given]{1}}^{\answer[given]{5}} \int_{\answer[given]{(x+3)/4}}^{\answer[given]{(9-x)/2}}\d y \d x
    \]
  
    Now let's integrate with respect to $x$ and then $y$.  If we give
    absolute bounds to our inequalites above, we see something
    strange:
    \[
    1 \le \answer[given]{(x+3)/4}\le y \le \answer[given]{(9-x)/2} \le 4 
    \]
    This is actually \textbf{two} inequalities (note we can see this
    from the graph):
    \begin{align*}
      1 &\le \answer[given]{(x+3)/4}\le y && \text{and}\\
      y &\le \answer[given]{(9-x)/2} \le 4 
    \end{align*}
    Solving for $x$ in these inequalities we find:
    \begin{align*}
      1 &\le x \le 4y-3 && \text{and}\\
      1 &\le x \le 9-2y
    \end{align*}
    Here $x$ is bounded by two \textbf{different} curves! When this
    happens, it is probably best to look at the graph of the
    situation:
    \begin{image}
    \begin{tikzpicture}
      \begin{axis}[
          tick label style={font=\scriptsize},
          axis y line=middle,axis x line=middle,
          name=myplot,
	  xtick={1,2,...,5},
	  ytick={1,2,...,4},
          grid=major,
          grid style={dashed, gridColor},
	  ymin=-.5,ymax=4.5,%
	  xmin=-.5,xmax=5.5,
          rounded corners=.5pt
        ]
        \draw [ultra thick,penColor] (axis cs:1,1) -- (axis cs: 1,4)-- (axis cs: 5,2)  -- cycle;
        
        \draw (axis cs: 2.5,2.5) node {$R$};
        \node [above right,penColor] at (axis cs: 3,3) {$x=9-2y$};
        \node [below right, penColor] at (axis cs: 3,1.5) {$x=4y-3$};
      \end{axis} 
      \node [right] at (myplot.right of origin) {\scriptsize $x$};
      \node [above] at (myplot.above origin) {\scriptsize $y$};
    \end{tikzpicture}
    \end{image}
    Thus we see that the area can be found by the sum:
    \[
    \int_1^2 \int_{\answer[given]{1}}^{\answer[given]{4y-3}}\d x \d y +\int_2^4 \int_{\answer[given]{1}}^{\answer[given]{9-2y}}\d x \d y
    \]
  \end{explanation}
\end{example}




\subsection{Why change the order of integration?}

You may ask yourself, ``Why change the order of integration?''
Sometimes changing the region can make a difficult (impossible)
antiderivative easier (not impossible). Let's see an example.

\begin{example}
  Compute:
  \[
  \int_0^{\pi/2}\int_y^{\pi/2} \frac{\sin(x)}{x} \d x \d y
  \]
  \begin{explanation}
    Here you don't stand a chance if you try to antidifferentiate with
    respect to $x$. So the trick is to immediately swap the order of
    antidifferentation. Starting by writing absolute bounds for $x$
    and $y$:
    \[
    0\le \answer[given]{y}\le \answer[given]{x}\le\pi/2
    \]
    Now we see that:
    \begin{align*}
      \int_0^{\pi/2}\int_y^{\pi/2} \frac{\sin(x)}{x} \d x \d y &= \int_{\answer[given]{0}}^{\answer[given]{\pi/2}}\int_{\answer[given]{0}}^{\answer[given]{x}} \frac{\sin(x)}{x} \d y \d x\\
      &= \int_{\answer[given]{0}}^{\answer[given]{\pi/2}} \answer[given]{\sin(x)} \d x\\
      &=\answer[given]{1}
    \end{align*}
  \end{explanation}
\end{example}

This author does not know how to solve the problem above without
changing the order of integration.









\section{Triple integrals and volume}

We start by introducing a fourth version of Fubini's Theorem.

\begin{theorem}[Fubini]\index{Fubini's Theorem}
  Let $R$ be a closed, bounded region in $\R^3$ and let $F(x,y,z)$ be
  a continuous function on $R$. If
    \begin{align*}
      R=\{(x,y,z):&\text{$a\leq x\leq b$, $g_1(x)\leq y\leq g_2(x)$,}\\
        &\text{and $H_1(x,y) \leq z \leq H_2(x,y)$}\}
    \end{align*}
    where $g_1$ and $g_2$ are continuous functions on $[a,b]$, and $H_1$ and $H_2$ are continuous on the region
    \[
    S =\{(x,y):\text{$a\leq x\leq b$ and $g_1(x)\leq y\leq g_2(x)$}\}
    \]
    then
    \[
    \iiint_R F(x,y,z) \d V = \int_a^b\int_{g_1(x)}^{g_2(x)}\int_{H_1(x,y)}^{H_2(x,y)} F(x,y,z)\d z\d y\d x.
    \]
    There are \textit{six} reorderings total with three variables. We will spare
    the young mathematician the details, and trust that you will sort
    it out.
\end{theorem}

Again note that when using Fubini's Theorem, we must always have
numbers as the limits of the outer-most integral, curves as the limits
of the middle integral, and surfaces as the limits of the inner-most
integral:
\[
\int_{\text{number}}^{\text{number}}\int_{\text{curve}}^{\text{curve}}\int_{\text{surface}}^{\text{surface}}
F(x,y,z) \d V
\]


Now with Fubini's help, we will use triple integrals to compute
volumes.

\begin{example}
  Set-up and evaluate an iterated integral that will compute the
  volume of the region bounded by:
  \begin{itemize}
  \item The plane $x=0$.
  \item The plane $y=0$.
  \item The plane $z=0$.
  \item The plane $3x + 4y + 6z = 12$.
  \end{itemize}
  \begin{explanation}
    Our region above produces a
    \textit{tetrahedron}\index{tetrahedron}, a triangular-based
    pyramid. It intersects the $x$-axis at $(4,0,0)$, the $y$-axis at
    $(0,3,0)$, and the $z$-axis at $(0,0,2)$.  The region can be
    defined by:
    \begin{align*}
      R=\Big\{(x,y,z):&0\leq x\leq 4, \\
      &0\leq y\leq -3x/4+3, \\
      &0\leq z \leq \frac{12-3x-4y}{6}\Big\}
    \end{align*}
    The volume of this region is given by, write with me, 
    \[
    \int_{\answer[given]{0}}^{\answer[given]{4}} \int_{\answer[given]{0}}^{\answer[given]{-3x/4+3}} \int_{\answer[given]{0}}^{\answer[given]{\frac{12-3x-4y}{6}}}\d z \d y\d x 
    \]
    \begin{align*}
      &=\int_{\answer[given]{0}}^{\answer[given]{4}} \int_{\answer[given]{0}}^{\answer[given]{-3x/4+3}} \answer[given]{\frac{12-3x-4y}{6}} \d y \d x\\
      &= \int_{\answer[given]{0}}^{\answer[given]{4}}  \eval{\answer[given]{2y-\frac{xy}{2}-\frac{y^2}{3}}}_{\answer[given]{0}}^{\answer[given]{-3x/4+3}} \d x\\
      &= \int_{\answer[given]{0}}^{\answer[given]{4}}  \left(\answer[given]{\frac{3x^2}{16} -\frac{3x}{2}+3}\right) \d x\\
      &= \eval{\answer[given]{x^3/16-3x^2/4+3x}}_{\answer[given]{0}}^{\answer[given]{4}} \\
      &=\answer[given]{4}.
    \end{align*}    
  \end{explanation}
\end{example}

\begin{question}
  In the previous example, we integrated with respect to $z$, then
  $y$, then $x$. Set-up an integral that computes the volume of $R$
  that integrates with respect to $x$, then $z$, then $y$.
  \begin{prompt}
    Start by finding overall bounds for our variables. In this case:
    \begin{align*}
      0\le &x \le 4\\
      0\le &y \le -3x/4+3 \le \answer{3}\\
      0\le &z \le \frac{12-3x-4y}{6}\le \answer{2}
    \end{align*}
    At this point we see that our bounds for $y$ are $\answer{0}$ to
    $\answer{3}$. Now we will find our bounds for $x$. We must find an
    expression for $x$ in terms of $y$ and $z$. Write with me:
    \begin{align*}
      0\le z \le &\frac{12-3x-4y}{6}\le 2\\
      0\le 6z \le &\answer{12-3x-4y} \le 12\\
      -12+4y\le \answer{6z+4y-12}\le &-3x \le 4y\\
      \frac{12-4y}{3}\ge \answer{\frac{12-6z-4y}{3}}\ge &x \ge -4y/3
    \end{align*}
    However, we know that $y$ is nonnegative, so $-4y/3\le \answer{0}$, and $x$
    is bounded below by $\answer{0}$. So $x$ runs from $0$ to
    $\answer{\frac{12-6z-4y}{3}}$.

    Finally we must write $z$ in terms of $y$. Unfortunately, from our
    inequalities above, there is no direct way to get this. We must
    think about what our solid looks like. Recall that the plane
    bounding the solid is $3x + 4y + 6z = 12$. If $x=0$, then our
    plane is the line $\answer{4y+6z}=12$. Hence
    \[
    0\le z \le \answer{2-2y/3}
    \]
    We may now write our desired integral:
    \[
    \iiint_R \d V = \int_{\answer{0}}^{\answer{3}}\int_{\answer{0}}^{\answer{-2y/3+2}} \int_{\answer{0}}^{\answer{\frac{12-4y-6z}{3}}} \d x \d z \d y
    \]
  \end{prompt}
\end{question}


\begin{example}
  Set-up an iterated integral that will compute the volume of the region
  bounded by the cone below:
  \begin{image}
    \begin{tikzpicture}
      \begin{axis}%
        [
          tick label style={font=\scriptsize},%axis on top,
	  axis lines=center,
	  view={135}{25},
	  name=myplot,
	  %% xtick=\empty,
	  %% ytick=\empty,
	  %% ztick=\empty,
	  %% extra x ticks={1},
	  %% extra x tick labels={$a$},
	  %% extra y ticks={1},
	  %% extra y tick labels={$a$},
	  %% extra z ticks={1},
	  %% extra z tick labels={$h$},
	  ymin=-1.3,ymax=1.3,
	  xmin=-1.3,xmax=1.3,
	  zmin=-.1, zmax=1.1,
	  every axis x label/.style={at={(axis cs:\pgfkeysvalueof{/pgfplots/xmax},0,0)},xshift=-1pt,yshift=-4pt},
	  xlabel={\scriptsize $x$},
	  every axis y label/.style={at={(axis cs:0,\pgfkeysvalueof{/pgfplots/ymax},0)},xshift=5pt,yshift=-3pt},
	  ylabel={\scriptsize $y$},
	  every axis z label/.style={at={(axis cs:0,0,\pgfkeysvalueof{/pgfplots/zmax})},xshift=0pt,yshift=4pt},
	  zlabel={\scriptsize $z$},colormap/cool
	]
        
        \addplot3[domain=0:1,,y domain=0:360,mesh,samples=10,samples y=36,very thin,z buffer=sort] ({x*cos(y)}, {x*sin(y)},{1-x});
      \end{axis}
    \end{tikzpicture}
  \end{image}
  \begin{explanation}
    First note that 
    \begin{align*}
      R=\{(x,y,z):&\text{$-1\leq x\leq 1$, $-\sqrt{1-x^2}\leq y\leq \sqrt{1-x^2}$}, \\
        &\text{and $0\leq z \leq 1-\sqrt{x^2+y^2}$}\}
    \end{align*}
    Hence, the volume of this region is given by
    \[
    \int_{\answer[given]{-1}}^{\answer[given]{1}}
    \int_{\answer[given]{-\sqrt{1-x^2}}}^{\answer[given]{\sqrt{1-x^2}}}
    \int_{\answer[given]{0}}^{\answer[given]{1-\sqrt{x^2+y^2}}}\d z \d y\d x.
    \]
    \end{explanation}
\end{example}

\begin{question}
  In the previous example, we integrated with respect to $z$, then
  $y$, then $x$. Set-up an iterated integral that computes the volume
  of $R$ that integrates with respect to $x$, then $y$, then $z$.
  \begin{prompt}
    \[
    \iiint_R \d V =
    \int_{\answer{0}}^{\answer{1}}
    \int_{\answer{-1+z}}^{\answer{1-z}}
    \int_{\answer{-\sqrt{(1-z)^2-y^2}}}^{\answer{\sqrt{(1-z)^2-y^2}}}
    \d x \d y \d z
    \]
  \end{prompt}
\end{question}



\end{document}
