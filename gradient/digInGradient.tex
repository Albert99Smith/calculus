\documentclass{ximera}

%\usepackage{todonotes}
%\usepackage{mathtools} %% Required for wide table Curl and Greens
%\usepackage{cuted} %% Required for wide table Curl and Greens
\newcommand{\todo}{}

\usepackage{esint} % for \oiint
\ifxake%%https://math.meta.stackexchange.com/questions/9973/how-do-you-render-a-closed-surface-double-integral
\renewcommand{\oiint}{{\large\bigcirc}\kern-1.56em\iint}
\fi


\graphicspath{
  {./}
  {ximeraTutorial/}
  {basicPhilosophy/}
  {functionsOfSeveralVariables/}
  {normalVectors/}
  {lagrangeMultipliers/}
  {vectorFields/}
  {greensTheorem/}
  {shapeOfThingsToCome/}
  {dotProducts/}
  {partialDerivativesAndTheGradientVector/}
  {../productAndQuotientRules/exercises/}
  {../normalVectors/exercisesParametricPlots/}
  {../continuityOfFunctionsOfSeveralVariables/exercises/}
  {../partialDerivativesAndTheGradientVector/exercises/}
  {../directionalDerivativeAndChainRule/exercises/}
  {../commonCoordinates/exercisesCylindricalCoordinates/}
  {../commonCoordinates/exercisesSphericalCoordinates/}
  {../greensTheorem/exercisesCurlAndLineIntegrals/}
  {../greensTheorem/exercisesDivergenceAndLineIntegrals/}
  {../shapeOfThingsToCome/exercisesDivergenceTheorem/}
  {../greensTheorem/}
  {../shapeOfThingsToCome/}
  {../separableDifferentialEquations/exercises/}
  {vectorFields/}
}

\newcommand{\mooculus}{\textsf{\textbf{MOOC}\textnormal{\textsf{ULUS}}}}

\usepackage{tkz-euclide}\usepackage{tikz}
\usepackage{tikz-cd}
\usetikzlibrary{arrows}
\tikzset{>=stealth,commutative diagrams/.cd,
  arrow style=tikz,diagrams={>=stealth}} %% cool arrow head
\tikzset{shorten <>/.style={ shorten >=#1, shorten <=#1 } } %% allows shorter vectors

\usetikzlibrary{backgrounds} %% for boxes around graphs
\usetikzlibrary{shapes,positioning}  %% Clouds and stars
\usetikzlibrary{matrix} %% for matrix
\usepgfplotslibrary{polar} %% for polar plots
\usepgfplotslibrary{fillbetween} %% to shade area between curves in TikZ
\usetkzobj{all}
\usepackage[makeroom]{cancel} %% for strike outs
%\usepackage{mathtools} %% for pretty underbrace % Breaks Ximera
%\usepackage{multicol}
\usepackage{pgffor} %% required for integral for loops



%% http://tex.stackexchange.com/questions/66490/drawing-a-tikz-arc-specifying-the-center
%% Draws beach ball
\tikzset{pics/carc/.style args={#1:#2:#3}{code={\draw[pic actions] (#1:#3) arc(#1:#2:#3);}}}



\usepackage{array}
\setlength{\extrarowheight}{+.1cm}
\newdimen\digitwidth
\settowidth\digitwidth{9}
\def\divrule#1#2{
\noalign{\moveright#1\digitwidth
\vbox{\hrule width#2\digitwidth}}}





\newcommand{\RR}{\mathbb R}
\newcommand{\R}{\mathbb R}
\newcommand{\N}{\mathbb N}
\newcommand{\Z}{\mathbb Z}

\newcommand{\sagemath}{\textsf{SageMath}}


%\renewcommand{\d}{\,d\!}
\renewcommand{\d}{\mathop{}\!d}
\newcommand{\dd}[2][]{\frac{\d #1}{\d #2}}
\newcommand{\pp}[2][]{\frac{\partial #1}{\partial #2}}
\renewcommand{\l}{\ell}
\newcommand{\ddx}{\frac{d}{\d x}}

\newcommand{\zeroOverZero}{\ensuremath{\boldsymbol{\tfrac{0}{0}}}}
\newcommand{\inftyOverInfty}{\ensuremath{\boldsymbol{\tfrac{\infty}{\infty}}}}
\newcommand{\zeroOverInfty}{\ensuremath{\boldsymbol{\tfrac{0}{\infty}}}}
\newcommand{\zeroTimesInfty}{\ensuremath{\small\boldsymbol{0\cdot \infty}}}
\newcommand{\inftyMinusInfty}{\ensuremath{\small\boldsymbol{\infty - \infty}}}
\newcommand{\oneToInfty}{\ensuremath{\boldsymbol{1^\infty}}}
\newcommand{\zeroToZero}{\ensuremath{\boldsymbol{0^0}}}
\newcommand{\inftyToZero}{\ensuremath{\boldsymbol{\infty^0}}}



\newcommand{\numOverZero}{\ensuremath{\boldsymbol{\tfrac{\#}{0}}}}
\newcommand{\dfn}{\textbf}
%\newcommand{\unit}{\,\mathrm}
\newcommand{\unit}{\mathop{}\!\mathrm}
\newcommand{\eval}[1]{\bigg[ #1 \bigg]}
\newcommand{\seq}[1]{\left( #1 \right)}
\renewcommand{\epsilon}{\varepsilon}
\renewcommand{\phi}{\varphi}


\renewcommand{\iff}{\Leftrightarrow}

\DeclareMathOperator{\arccot}{arccot}
\DeclareMathOperator{\arcsec}{arcsec}
\DeclareMathOperator{\arccsc}{arccsc}
\DeclareMathOperator{\si}{Si}
\DeclareMathOperator{\scal}{scal}
\DeclareMathOperator{\sign}{sign}


%% \newcommand{\tightoverset}[2]{% for arrow vec
%%   \mathop{#2}\limits^{\vbox to -.5ex{\kern-0.75ex\hbox{$#1$}\vss}}}
\newcommand{\arrowvec}[1]{{\overset{\rightharpoonup}{#1}}}
%\renewcommand{\vec}[1]{\arrowvec{\mathbf{#1}}}
\renewcommand{\vec}[1]{{\overset{\boldsymbol{\rightharpoonup}}{\mathbf{#1}}}\hspace{0in}}

\newcommand{\point}[1]{\left(#1\right)} %this allows \vector{ to be changed to \vector{ with a quick find and replace
\newcommand{\pt}[1]{\mathbf{#1}} %this allows \vec{ to be changed to \vec{ with a quick find and replace
\newcommand{\Lim}[2]{\lim_{\point{#1} \to \point{#2}}} %Bart, I changed this to point since I want to use it.  It runs through both of the exercise and exerciseE files in limits section, which is why it was in each document to start with.

\DeclareMathOperator{\proj}{\mathbf{proj}}
\newcommand{\veci}{{\boldsymbol{\hat{\imath}}}}
\newcommand{\vecj}{{\boldsymbol{\hat{\jmath}}}}
\newcommand{\veck}{{\boldsymbol{\hat{k}}}}
\newcommand{\vecl}{\vec{\boldsymbol{\l}}}
\newcommand{\uvec}[1]{\mathbf{\hat{#1}}}
\newcommand{\utan}{\mathbf{\hat{t}}}
\newcommand{\unormal}{\mathbf{\hat{n}}}
\newcommand{\ubinormal}{\mathbf{\hat{b}}}

\newcommand{\dotp}{\bullet}
\newcommand{\cross}{\boldsymbol\times}
\newcommand{\grad}{\boldsymbol\nabla}
\newcommand{\divergence}{\grad\dotp}
\newcommand{\curl}{\grad\cross}
%\DeclareMathOperator{\divergence}{divergence}
%\DeclareMathOperator{\curl}[1]{\grad\cross #1}
\newcommand{\lto}{\mathop{\longrightarrow\,}\limits}

\renewcommand{\bar}{\overline}

\colorlet{textColor}{black}
\colorlet{background}{white}
\colorlet{penColor}{blue!50!black} % Color of a curve in a plot
\colorlet{penColor2}{red!50!black}% Color of a curve in a plot
\colorlet{penColor3}{red!50!blue} % Color of a curve in a plot
\colorlet{penColor4}{green!50!black} % Color of a curve in a plot
\colorlet{penColor5}{orange!80!black} % Color of a curve in a plot
\colorlet{penColor6}{yellow!70!black} % Color of a curve in a plot
\colorlet{fill1}{penColor!20} % Color of fill in a plot
\colorlet{fill2}{penColor2!20} % Color of fill in a plot
\colorlet{fillp}{fill1} % Color of positive area
\colorlet{filln}{penColor2!20} % Color of negative area
\colorlet{fill3}{penColor3!20} % Fill
\colorlet{fill4}{penColor4!20} % Fill
\colorlet{fill5}{penColor5!20} % Fill
\colorlet{gridColor}{gray!50} % Color of grid in a plot

\newcommand{\surfaceColor}{violet}
\newcommand{\surfaceColorTwo}{redyellow}
\newcommand{\sliceColor}{greenyellow}




\pgfmathdeclarefunction{gauss}{2}{% gives gaussian
  \pgfmathparse{1/(#2*sqrt(2*pi))*exp(-((x-#1)^2)/(2*#2^2))}%
}


%%%%%%%%%%%%%
%% Vectors
%%%%%%%%%%%%%

%% Simple horiz vectors
\renewcommand{\vector}[1]{\left\langle #1\right\rangle}


%% %% Complex Horiz Vectors with angle brackets
%% \makeatletter
%% \renewcommand{\vector}[2][ , ]{\left\langle%
%%   \def\nextitem{\def\nextitem{#1}}%
%%   \@for \el:=#2\do{\nextitem\el}\right\rangle%
%% }
%% \makeatother

%% %% Vertical Vectors
%% \def\vector#1{\begin{bmatrix}\vecListA#1,,\end{bmatrix}}
%% \def\vecListA#1,{\if,#1,\else #1\cr \expandafter \vecListA \fi}

%%%%%%%%%%%%%
%% End of vectors
%%%%%%%%%%%%%

%\newcommand{\fullwidth}{}
%\newcommand{\normalwidth}{}



%% makes a snazzy t-chart for evaluating functions
%\newenvironment{tchart}{\rowcolors{2}{}{background!90!textColor}\array}{\endarray}

%%This is to help with formatting on future title pages.
\newenvironment{sectionOutcomes}{}{}



%% Flowchart stuff
%\tikzstyle{startstop} = [rectangle, rounded corners, minimum width=3cm, minimum height=1cm,text centered, draw=black]
%\tikzstyle{question} = [rectangle, minimum width=3cm, minimum height=1cm, text centered, draw=black]
%\tikzstyle{decision} = [trapezium, trapezium left angle=70, trapezium right angle=110, minimum width=3cm, minimum height=1cm, text centered, draw=black]
%\tikzstyle{question} = [rectangle, rounded corners, minimum width=3cm, minimum height=1cm,text centered, draw=black]
%\tikzstyle{process} = [rectangle, minimum width=3cm, minimum height=1cm, text centered, draw=black]
%\tikzstyle{decision} = [trapezium, trapezium left angle=70, trapezium right angle=110, minimum width=3cm, minimum height=1cm, text centered, draw=black]



\title[Dig-In:]{The gradient}

\begin{document}
\begin{abstract}
We introduce the gradient vector. 
\end{abstract}
\maketitle

The gradient is a useful tool in vector calculus. In some sense, it
has the role that the derivative has in single-variable calculus.
Let's state the definition:

\begin{definition}
  Let $F:\R^n\to \R$ be a differentiable function, the \dfn{gradient}
  \[
  \grad F = \vector{\pp[F]{x_1},\pp[F]{x_2},\dots,\pp[F]{x_n}}
  \]
  is a vector-valued function of $n$ variables. 
\end{definition}

Try your hand at some casual computations.

\begin{question}
  Let $F(x,y) = \sin(x)\cos(y)$, compute:
  \[
  \grad F(x,y)
  \begin{prompt}
    = \vector{\answer{\cos(x)\cos(y)},\answer{-\sin(x)\sin(y)}}
  \end{prompt}
  \]
  \begin{question}
    Let $\vec{p}= \vector{\pi/3,\pi/3}$. Compute:
    \[
    \grad F(\vec{p})
    \begin{prompt}
      =\vector{\answer{1/4},\answer{-3/4}}
    \end{prompt}
    \]
  \end{question}
\end{question}

And now in three variables.

\begin{question}
  Let $F(x,y,z) = ze^{-7xy}$, compute:
  \[
  \grad F(x,y,z)
  \begin{prompt}
    = \vector{\answer{z e^{-7xy}(-7)y},\answer{z e^{-7xy}(-7)x}, \answer{e^{-7xy}}}
  \end{prompt}
  \]
  \begin{question}
    Let $\vec{p}= \vector{1,0,1/7}$. Compute:
    \[
    \grad F(\vec{p})
    \begin{prompt}
      =\vector{\answer{0},\answer{-1},\answer{1}}
    \end{prompt}
    \]
  \end{question}
\end{question}
Now that we can compute the gradient, let's see if we can figure out
what it \textbf{means}.

\section{The initial greatest increase}

First recall what it means for a function to be \textit{differentiable}:
\begin{quote}\index{tangent plane}%%BADBAD would like an image
  Given a function $F:\R^3\to\R$ and a vector $\vec{a}$ in the domain
  of $F$, if one can ``zoom in'' on the graph at $(\vec{a}, F(\vec{a}))$
  sufficiently so that it appears to be a plane, then the
  function is \dfn{differentiable}, and that plane is the \dfn{tangent plane}
  to $F$ at the point $(\vec{a},F(\vec{a}))$.
\end{quote}

Now let's imagine what the gradient is telling us about the
plane. First let $\vec{p}=\vector{a,b}$ and consider the tangent plane:
\[
z = F^{(1,0)}(\vec{p}) (x-a) + F^{(0,1)}(\vec{p})(y-b) + F(\vec{p})
\]
\begin{itemize}
  \item If a component of $\grad F(\vec{p})$ is positive, traveling in
    its direction will ensure that you are ``traveling uphill,''
    meaning you are raising the $z$-value (at least initially).
\item If a component of $\grad F(\vec{p})$ is negative, traveling in its
  (negative) direction will also ensure that you are ``traveling
  uphill,'' meaning you are raising the $z$-value (at least
  initially).
\end{itemize}
Expressing the tangent plane in terms of vectors $\vec{x}
=\vector{x,y}$ and the dot product, we see
\[
z = \grad F(\vec{p})\dotp (\vec{x}-\vec{p}) + F(\vec{p})
\]
From this we see that the components of $\grad F(\vec{p})$ form a
vector that simultaneously points ``most uphill'' in the
$x$-direction, and ``most uphill'' in the $y$-direction. The upshot?
\begin{quote}
  \textbf{The gradient points in the initial direction of the greatest
    increase of the function.}
\end{quote}

You may be wondering, ``Why does the gradient only point in the
`initial' direction of greatest increase?''  It can only be
``initial'' because the reasoning above relies on the fact that we can
be ``zoomed-in'' enough for the surface to look like a plane.
\begin{onlineOnly}
For your viewing pleasure, we have included a graph where you can see
the $x$-component of the gradient and the $y$-component of the
gradient combining to give the initial the direction one should leave
from a point, and find the initial greatest increase in the function.
\begin{sageCell}
f(x,y) = 3*x+4*y
vectorx=arrow3d((0,0,0),(3,0,0),7,color='blue');
vectory=arrow3d((0,0,0),(0,4,0),7,color='blue');
grad=arrow3d((0,0,0),(3,4,0),7,color='red');

plot3d( f, (x,-3,3), (y,-3,3) ) + vectorx + vectory + grad
\end{sageCell}
\end{onlineOnly}
\begin{remark}
  Given a function $F:\R\to\R^3$, the vectors $\grad F(x,y)$ are only
  pointing in two dimensions. Consider
  \[
  z = F(x,y) = 3x+4y.
  \]
  This is an explicit function that can be described as a surface in
  three-dimensional space, yet the gradient vector
  \[
  \grad F = \vector{3, 4}
  \]
  is a vector in $\R^2$, meaning two-dimensional space. Similarly, if
  you are working with a function $F:\R\to\R^n$, this can be described
  as a ``surface'' in $(n+1)$-dimensional space, but the gradient
  vectors
  \[
  \vector{\pp[F]{x_1},\pp[F]{x_2},\dots,\pp[F]{x_n}}
  \]
  are vectors in $n$-dimensional space.
\end{remark}


The gradient can help us understand how a function changes in a
particular direction.
\begin{definition}
  The \dfn{directional derivative} is computed by
  \[
  D_\vec{\hat{u}}(F) = \grad F(\vec{x})\dotp \vec{\hat{u}}
  \]
  where $\vec{\hat{u}}$ is a unit vector.
\end{definition}

The directional derivative tells us how $F$ changes if we move one
direction. If we want to make $D_\vec{\hat{u}}(F)$ as large as
possible, it makes sense to let $\vec{\hat{u}}$ be the direction of
gradient since the dot product
\[
\grad F(\vec{x})\dotp \vec{\hat{u}}
\]
is largest when $\vec{\hat{u}}$ is in the same direction as $\grad F$.

\begin{example}
  Let
  \[
  F(x,y) = -x^2+2x-y^2+2y+1.
  \]
  Compute and interpret $\grad F(1,1)$.
  \begin{explanation}
    Write with me
    \[
    \grad F(x,y) = \vector{\answer[given]{-2x+2},\answer[given]{-2y+2}}.
    \]
    However, we see $\grad F(1,1) = \vector{\answer[given]{0},\answer[given]{0}}$. Since there is no
    initial direction of greatest increase, we must be at a local
    maximum for the function. Indeed we are, behold:
    \begin{image}
      \begin{tikzpicture}
        \begin{axis}%
          [tick label style={font=\scriptsize},axis on top,
	    axis lines=center,
	    view={100}{25},
	    name=myplot,
	    %xtick=\empty,
	    %ytick={5},
	    %ztick={.7,-.7},
	    minor xtick=1,
	    minor ytick=1,
	    ymin=-.1,ymax=2.1,
	    xmin=-.1,xmax=2.1,
	    zmin=-.5, zmax=3.5,
	    every axis x label/.style={at={(axis cs:\pgfkeysvalueof{/pgfplots/xmax},0,0)},xshift=-5pt,yshift=-1pt},
	    xlabel={\scriptsize $x$},
	    every axis y label/.style={at={(axis cs:0,\pgfkeysvalueof{/pgfplots/ymax},0)},xshift=4pt,yshift=-4pt},
	    ylabel={\scriptsize $y$},
	    every axis z label/.style={at={(axis cs:0,0,\pgfkeysvalueof{/pgfplots/zmax})},xshift=0pt,yshift=4pt},
	    zlabel={\scriptsize $z$},
            colormap/cool
	  ]
          
          \addplot3[domain=0:2,,y domain=0:2,
            mesh,samples=25,samples y=25,very thin,z buffer=sort] {-x^2-y^2+2*x+2*y+1};
          \filldraw [black,] (axis cs:1,1,3) circle (1pt);
        \end{axis}
      \end{tikzpicture}
    \end{image}
    Ah! The point $(1,1)$ lies at the top of a paraboloid. In all
    directions, the instantaneous rate of change is $0$.
  \end{explanation}
\end{example}

Stand back. We're going to do some serious calculus now. Just read,
relax and enjoy.

\begin{example}
  Consider the surface given by $F(x,y)= 20-x^2-2y^2$:
  \begin{image}
    \begin{tikzpicture}
      \begin{axis}%
        [tick label style={font=\scriptsize},axis on top,
	  axis lines=center,
	  view={155}{25},
	  name=myplot,
	  %xtick=\empty,
	  %ytick={5},
	  %ztick={.7,-.7},
	  %minor xtick=1,
	  %minor ytick=1,
	  ymin=-1,ymax=5.5,
	  xmin=-1,xmax=5.5,
	  zmin=-1.1, zmax=21,
	  every axis x label/.style={at={(axis cs:\pgfkeysvalueof{/pgfplots/xmax},0,0)},xshift=-5pt,yshift=-1pt},
	  xlabel={\scriptsize $x$},
	  every axis y label/.style={at={(axis cs:0,\pgfkeysvalueof{/pgfplots/ymax},0)},xshift=4pt,yshift=-4pt},
	  ylabel={\scriptsize $y$},
	  every axis z label/.style={at={(axis cs:0,0,\pgfkeysvalueof{/pgfplots/zmax})},xshift=0pt,yshift=4pt},
	  zlabel={\scriptsize $z$},colormap/cool
        ]
        
        %\addplot3[domain=0:180,smooth,y domain=0:360,surf,%fill=white,
        %colormap={mp2}{\colormapplaneone},faceted color=black!40,samples=30,samples y=25,very thin,z buffer=sort] ({cos(x)*1.5*cos(y)},{sin(x)*cos(y)},{sin(y)});
        
        \addplot3[domain=-1:4,y domain=-1:3,mesh,samples y=15,very thin,z buffer=sort,%opacity=.6,
          samples=15,] (x,y,{20-x^2-2*y^2});
        
        \addplot3 [thick, penColor, smooth,domain=1:4,samples=20,samples y=0] ({x},{x^2/4},{20-x^2-0.125*x^4});
        %%        
        \filldraw [black] (axis cs:1,.25,18.875) circle (1pt);
        %\filldraw [black] (axis cs:.5,.25,19.625) circle (1pt);
        %
        %\filldraw [black] (axis cs:.5,.5,19.25) circle (1pt);
        
        
        %\addplot3 [thick,{\colorone}, smooth,domain=-3:3,samples=20,samples y=0] ({x},{2},{x^2+8});
        %
        %\addplot3 [thick,{\colorone}, smooth,domain=-30:170,samples=60,samples y=0] ({2.93*(cos(x))},{1.96*(sin(x))},.2);
        %
        %\addplot3 [thick,{\colorone}, smooth,domain=-30:170,samples=60,samples y=0] ({2.75*(cos(x))},{1.83*(sin(x))},.4);
        %
        %\addplot3 [thick,{\colorone}, smooth,domain=-35:170,samples=60,samples y=0] ({2.4*(cos(x))},{1.6*(sin(x))},.6);
        %
        %\addplot3 [thick,{\colorone}, smooth,domain=-40:170,samples=60,samples y=0] ({1.8*(cos(x))},{1.2*(sin(x))},.8);
        %
        %\filldraw [{\colorone}] (axis cs: 0,0,1) circle (1pt);
      \end{axis}
    \end{tikzpicture}
  \end{image}
  
  Water is poured on the surface at $(1,1/4)$. What path does it take
  as it flows downhill?
  \begin{explanation}
    Let $\vec{w}(t) = \vector{x(t), y(t)}$ be the vector-valued
    function describing the path of the water in the $(x,y)$-plane. We
    seek $x(t)$ and $y(t)$. We know that water will always flow
    downhill in the initial steepest direction. Therefore, at any
    point on its path, it will be moving in the direction of
    \[
    -\grad F(x,y)
    \]
    We'll ignore the physical effects of momentum on the water.  Thus
    $\vec{w}(t)$ will be parallel to $\grad F$. Ah! This means there
    is some constant $c$ such that
    \[
    c\grad F(x(t),y(t)) = \vec{w}'(t) = \vector{x'(t), y'(t)}.
    \]
    Computing the gradient,
    \[
    \grad F(x(t),y(t)) = \vector{-2x(t), -4y(t)}
    \]
    Then
    \begin{align*}
      c\cdot \grad F(x(t),y(t)) &= \vector{ x'(t), y'(t)}\\
      c\cdot \vector{-2x(t),-4y(t)} &= \vector{ x'(t), y'(t)}\\
      \vector{-2cx(t),-4cy(t)} &= \vector{ x'(t), y'(t)}\\
          \end{align*}
    This implies
    \[
    -2cx(t) = x'(t) \quad \text{and} \quad  -4cy(t) =y'(t)
    \]
    so
    \[
    c = -\frac{x'(t)}{2x(t)} \quad \text{and} \quad  c =-\frac{y'(t)}{4y(t)}.
    \]
    Now recall that the differentials $\d x = x'(t) \d t$, and $\d
    y=y'(t)\d t$, so we may write
    \begin{align*}
      \int \frac{1}{2x}x'(t)\d t &=\int \frac{1}{4y} y'(t)\d t \\
      \int \frac{1}{2x}\d x &=\int\frac{1}{4y}\d y \\
      \frac{1}{2}\ln|x| +C &= \frac{1}{4}\ln|y|\\
      2\ln|x| + C &= \ln|y|\\
      \ln|x^2| + C &= \ln|y|
    \end{align*}
    Raising $e$ to the left-hand and right hand sides, we see
    \begin{align*}
    e^{\ln|x^2| + C} &= e^{\ln|y|}\\
    x^2\cdot e^C + C &= \ln|y|,
    \end{align*}
    setting $K = e^C$, we write
    \[
    K\cdot x^2 = y.
    \]
  We are so close to being done, $y=K\cdot x^2$, this is the path
  described in the $(x,y)$-plane. Since the water started at the point
  $(1,1/4)$, we can solve for $K$:
\[
K\cdot 1^2 = \frac14 \quad \Rightarrow \quad K = \frac14.
\]
Thus the water follows the curve
\[
y=x^2/4
\]
in the $(x,y)$-plane.
  \end{explanation}
\end{example}


\section{Perpendicularity and the gradient}

Now that we know gradient vectors point in the direction of the
greatest initial increase of the function, let's learn about their
geometry. Consider again a plane. This time we'll think about
\[
z = F(x,y) = x+y
\]
This plane increase with both the $x$-values and $y$-values and passes
through the line $y=-x$ in the $(x,y)$-plane.
\begin{onlineOnly}
\begin{sageCell}
f(x,y) = x+y
plot3d(f, (x,-3,3), (y,-3,3))
\end{sageCell}
\end{onlineOnly}
Let's look at a contour plot:
    \begin{image}
      \begin{tikzpicture}
        \begin{axis}%
          [tick label style={font=\scriptsize},axis on top,
	    axis lines=center,
            width=3in,
            height=3in,
	    xtick={-3,-2,...,3},
            ytick={-3,-2,...,3},
	    %ytick={5},
	    %ztick={.7,-.7},
	    ymin=-1.2,ymax=1.2,
	    xmin=-1.2,xmax=1.2,
            grid=major,%width=3in,height=3in,
            grid style={dashed, gridColor},
	    every axis x label/.style={at={(axis cs:\pgfkeysvalueof{/pgfplots/xmax},0,0)},xshift=5pt,yshift=0pt},
	    xlabel={\scriptsize $x$},
	    every axis y label/.style={at={(axis cs:0,\pgfkeysvalueof{/pgfplots/ymax},0)},xshift=4pt,yshift=2pt},
	    ylabel={\scriptsize $y$},
	  ]
          \addplot[very thick, penColor,smooth] {-x};
          \addplot[very thick, penColor,smooth] {-x+1};
          \addplot[very thick, penColor,smooth] {-x+2};
          \addplot[very thick, penColor,smooth] {-x+3};
          \addplot[very thick, penColor,smooth] {-x-1};
          \addplot[very thick, penColor,smooth] {-x-2};
          
          \node[penColor,fill=white] at (axis cs:.5,-.5) {$0$};
          \node[penColor,fill=white] at (axis cs:0.8,.2) {$1$};
          \node[penColor,fill=white] at (axis cs:1,1) {$2$};
          \node[penColor,fill=white] at (axis cs:-.2,-.8) {$-1$};
          \node[penColor,fill=white] at (axis cs:-1,-1) {$-2$};

          \addplot[color=black,fill=black,only marks,mark=*] coordinates{(4,2)};  %% closed hole          
        \end{axis}
      \end{tikzpicture}
    \end{image}
    Computing the gradient we find
    \[
    \grad F(x,y) = \vector{1,1}
    \]
    This vector is perpendicular to all the level curves.
        \begin{image}
      \begin{tikzpicture}
        \begin{axis}%
          [tick label style={font=\scriptsize},axis on top,
	    axis lines=center,
	    %view={30}{30},
	    %name=myplot,
            width=3in,
            height=3in,
	    xtick={-3,-2,...,3},
            ytick={-3,-2,...,3},
	    %ytick={5},
	    %ztick={.7,-.7},
	    ymin=-1.2,ymax=1.2,
	    xmin=-1.2,xmax=1.2,
            grid=major,%width=3in,height=3in,
            grid style={dashed, gridColor},
	    every axis x label/.style={at={(axis cs:\pgfkeysvalueof{/pgfplots/xmax},0,0)},xshift=5pt,yshift=0pt},
	    xlabel={\scriptsize $x$},
	    every axis y label/.style={at={(axis cs:0,\pgfkeysvalueof{/pgfplots/ymax},0)},xshift=4pt,yshift=2pt},
	    ylabel={\scriptsize $y$},
	  ]
          \addplot[very thick, penColor,smooth] {-x};
          \addplot[very thick, penColor,smooth] {-x+1};
          \addplot[very thick, penColor,smooth] {-x+2};
          \addplot[very thick, penColor,smooth] {-x+3};
          \addplot[very thick, penColor,smooth] {-x-1};
          \addplot[very thick, penColor,smooth] {-x-2};
          
          \node[penColor,fill=white] at (axis cs:.5,-.5) {$0$};
          \node[penColor,fill=white] at (axis cs:0.8,.2) {$1$};
          \node[penColor,fill=white] at (axis cs:1,1) {$2$};
          \node[penColor,fill=white] at (axis cs:-.2,-.8) {$-1$};
          \node[penColor,fill=white] at (axis cs:-1,-1) {$-2$};

          \addplot[penColor2,ultra thick, ->] coordinates{(0,0) (1,1)};

          \addplot[color=black,fill=black,only marks,mark=*] coordinates{(4,2)};  %% closed hole          
        \end{axis}
      \end{tikzpicture}
    \end{image}
    This is \textbf{not} an accident, rather it is a general
    rule. Gradient vectors point in the initial direction of greatest
    increase. \textbf{Every} point on the level curve $1 = x+y$ gives
    the same $z$-value. Because of this if the gradient pointed any
    direction other than, ``directly away,'' and hence
    perpendicularly, from the level curve, it would not be pointing
    in the initial direction of greatest increase. The up-shot?
    \begin{quote}
      \textbf{Gradient vectors are always orthogonal to level
        sets.}
    \end{quote}
    Let's use this fact to find a tangent plane to an implicit surface.
    
    \begin{example}
      Find an implicit equation for the tangent plane to the ellipsoid
      \[
      \frac{x^2}{12} +\frac{y^2}{6}+\frac{z^2}{4}=1
      \]
      \begin{image}
        \begin{tikzpicture}
          \begin{axis}%
            [tick label style={font=\scriptsize},axis on top,
	      axis lines=center,
	      view={125}{25},
	      name=myplot,
	      %xtick=\empty,
	      %ytick={5},
	      %ztick={.7,-.7},
	      minor xtick=1,
	      minor ytick=1,
	      ymin=-3.5,ymax=3.5,
	      xmin=-3.5,xmax=3.5,
	      zmin=-2.5, zmax=2.5,
	      every axis x label/.style={at={(axis cs:\pgfkeysvalueof{/pgfplots/xmax},0,0)},xshift=-5pt,yshift=-1pt},
	      xlabel={\scriptsize $x$},
	      every axis y label/.style={at={(axis cs:0,\pgfkeysvalueof{/pgfplots/ymax},0)},xshift=4pt,yshift=-4pt},
	      ylabel={\scriptsize $y$},
	      every axis z label/.style={at={(axis cs:0,0,\pgfkeysvalueof{/pgfplots/zmax})},xshift=0pt,yshift=4pt},
	      zlabel={\scriptsize $z$},
              colormap/cool
	    ]

            \addplot3[domain=0:180,smooth,y domain=0:360,mesh,samples=19,samples y=19,very thin,z buffer=sort] ({cos(y)*(3.46*cos(x))},{cos(y)*(2.45*sin(x))},{2*sin(y)});
            
            
            \filldraw [black] (axis cs:1,2,1) circle (1pt);
\end{axis}
\end{tikzpicture}
\end{image}
      at $\vec{p} = \vector{1,2,1}$.
      \begin{explanation}
        Consider
        \[
        F(x,y,z) =\frac{x^2}{12} +\frac{y^2}{6}+\frac{z^2}{4}.
        \]
        and imagine the ellipsoid as the level surface
        \[
        F(x,y,z) = \answer[given]{1}
        \]
        Remember, the gradient is perpendicular to level surfaces, so
        we'll use this to find a normal vector to the tangent plane.
        The gradient is:
        \begin{align*}
          \grad F(x,y,z) &= \vector{\pp[F]{x}, \pp[F]{y},\pp[F]{z}}\\
          &= \vector{\answer[given]{\frac{x}{6}},\answer[given]{\frac{y}{3}}, \answer[given]{\frac{z}{2}}}.
        \end{align*}
        Since this vector is normal to the surface, we can use it to
        find an implicit formula for the tangent plane to the surface
        by computing
        \[
        \vec{n}\dotp(\vec{x}-\vec{p}) = 0
        \]
        where $\vec{p} = \vector{1,2,1}$ and
        \begin{align*}
          \vec{n} &= \grad F(\vec{p})\\
          &=\vector{\answer[given]{1/6}, \answer[given]{2/3},\answer[given]{1/2}}
        \end{align*}
        Thus the equation of the plane tangent to the ellipsoid at
        $\vec{p}$ is:
        \[
        \answer[given]{\frac{1}{6}(x-1) + \frac{2}{3}(y-2) + \frac{1}{2}(z-1)} = 0
        \]
      \end{explanation}
\end{example}

Finally, sometimes we are given discrete data that can be
adequately described and approximated by differentiable
functions. In this case, the \textbf{concept} of the gradient will
lead you to success.

\begin{example}
  Let $F:\R^2\to\R$ be a differentiable function described by the
    following table of values:
    \begin{image}
      \begin{tikzpicture}[x=1cm,y=.75cm]
        \draw (0,0) grid [step=1] (6,5);
        
        \draw[ultra thick] (0,1)--(6,1);
        \draw[ultra thick] (1,0)--(1,5);
        
        \draw (0,0) -- (1,1);
        \node at (.4,.9) [below left,inner sep=1pt] {\small$y$};
        \node at (0.6,.1) [above right,inner sep=1pt] {\small$x$};
    
        %% y-values
        \node at (0.5,4.5) {$7$};
        \node at (0.5,3.5) {$6$};
        \node at (0.5,2.5) {$5$};
        \node at (0.5,1.5) {$4$};
        
        
        %% z-values
        %% top
        \node at (1.5,4.5) {$18$};
        \node at (2.5,4.5) {$10$};
        \node at (3.5,4.5) {$-1$};
        \node at (4.5,4.5) {$-3$};
        \node at (5.5,4.5) {$-7$};
        
        %% 
        \node at (1.5,3.5) {$24$};
        \node at (2.5,3.5) {$16$};
        \node at (3.5,3.5) {$5$};
        \node at (4.5,3.5) {$1$};
        \node at (5.5,3.5) {$-2$};
        
        %% 
        \node at (1.5,2.5) {$18$};
        \node at (2.5,2.5) {$5$};
        \node at (3.5,2.5) {$0$};
        \node at (4.5,2.5) {$-1$};
        \node at (5.5,2.5) {$-4$};
        
        %% 
        \node at (1.5,1.5) {$12$};
        \node at (2.5,1.5) {$2$};
        \node at (3.5,1.5) {$-3$};
        \node at (4.5,1.5) {$-4$};
        \node at (5.5,1.5) {$-6$};
        
        %% bottom row
        \node at (1.5,.5) {$1$};
        \node at (2.5,.5) {$2$};
        \node at (3.5,.5) {$3$};
        \node at (4.5,.5) {$4$};
        \node at (5.5,.5) {$5$};
      \end{tikzpicture}
    \end{image}
    Estimate $\grad F(3,5)$.
    \begin{explanation}
      We estimate $\grad F(3,5)$ by estimating the partial derivatives. 
      To estimate $F^{(1,0)}(3,5)$, we examine the change in $F(x,5)$
    between $x=4$ and $x=3$:
    \[
    \frac{F(4,5)-F(\answer[given]{3},5)}{\answer[given]{4}-3}= \answer[given]{-1}
    \]
    We should also examine the change in $F(x,5)$ between $x=3$ and
    $x=2$:
    \[
      \frac{F(3,5)-F(\answer[given]{2},5)}{\answer[given]{3}-2} =\answer[given]{-5}  
    \]
    Now if we average these values together, we see:
    \[
    \eval{\pp{x} F(x,y)}_{(x,y)=(3,5)} \approx \answer[given]{-3}
    \]
    On the other hand, using a similar procedure, we find that:
    \[
    \eval{\pp{y} F(x,y)}_{(x,y)=(3,5)} \approx \answer[given]{4}
    \]
    Thus the gradient is
    \[
    \grad F(3,5) = \vector{\answer[given]{-3},\answer[given]{4}}
    \]
    Note if you leave the point $(3,5)$ in the direction of $\grad
    F(3,5) = \vector{\answer[given]{-3},\answer[given]{4}}$, you head
    directly toward $F(2,6)= 16$, the greatest initial increase from
    $(3,5)$.
    \end{explanation}
\end{example}



\end{document}
