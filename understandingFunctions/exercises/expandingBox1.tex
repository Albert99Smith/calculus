\documentclass{ximera}

%\usepackage{todonotes}
%\usepackage{mathtools} %% Required for wide table Curl and Greens
%\usepackage{cuted} %% Required for wide table Curl and Greens
\newcommand{\todo}{}

\usepackage{esint} % for \oiint
\ifxake%%https://math.meta.stackexchange.com/questions/9973/how-do-you-render-a-closed-surface-double-integral
\renewcommand{\oiint}{{\large\bigcirc}\kern-1.56em\iint}
\fi


\graphicspath{
  {./}
  {ximeraTutorial/}
  {basicPhilosophy/}
  {functionsOfSeveralVariables/}
  {normalVectors/}
  {lagrangeMultipliers/}
  {vectorFields/}
  {greensTheorem/}
  {shapeOfThingsToCome/}
  {dotProducts/}
  {partialDerivativesAndTheGradientVector/}
  {../productAndQuotientRules/exercises/}
  {../normalVectors/exercisesParametricPlots/}
  {../continuityOfFunctionsOfSeveralVariables/exercises/}
  {../partialDerivativesAndTheGradientVector/exercises/}
  {../directionalDerivativeAndChainRule/exercises/}
  {../commonCoordinates/exercisesCylindricalCoordinates/}
  {../commonCoordinates/exercisesSphericalCoordinates/}
  {../greensTheorem/exercisesCurlAndLineIntegrals/}
  {../greensTheorem/exercisesDivergenceAndLineIntegrals/}
  {../shapeOfThingsToCome/exercisesDivergenceTheorem/}
  {../greensTheorem/}
  {../shapeOfThingsToCome/}
  {../separableDifferentialEquations/exercises/}
  {vectorFields/}
}

\newcommand{\mooculus}{\textsf{\textbf{MOOC}\textnormal{\textsf{ULUS}}}}

\usepackage{tkz-euclide}\usepackage{tikz}
\usepackage{tikz-cd}
\usetikzlibrary{arrows}
\tikzset{>=stealth,commutative diagrams/.cd,
  arrow style=tikz,diagrams={>=stealth}} %% cool arrow head
\tikzset{shorten <>/.style={ shorten >=#1, shorten <=#1 } } %% allows shorter vectors

\usetikzlibrary{backgrounds} %% for boxes around graphs
\usetikzlibrary{shapes,positioning}  %% Clouds and stars
\usetikzlibrary{matrix} %% for matrix
\usepgfplotslibrary{polar} %% for polar plots
\usepgfplotslibrary{fillbetween} %% to shade area between curves in TikZ
\usetkzobj{all}
\usepackage[makeroom]{cancel} %% for strike outs
%\usepackage{mathtools} %% for pretty underbrace % Breaks Ximera
%\usepackage{multicol}
\usepackage{pgffor} %% required for integral for loops



%% http://tex.stackexchange.com/questions/66490/drawing-a-tikz-arc-specifying-the-center
%% Draws beach ball
\tikzset{pics/carc/.style args={#1:#2:#3}{code={\draw[pic actions] (#1:#3) arc(#1:#2:#3);}}}



\usepackage{array}
\setlength{\extrarowheight}{+.1cm}
\newdimen\digitwidth
\settowidth\digitwidth{9}
\def\divrule#1#2{
\noalign{\moveright#1\digitwidth
\vbox{\hrule width#2\digitwidth}}}





\newcommand{\RR}{\mathbb R}
\newcommand{\R}{\mathbb R}
\newcommand{\N}{\mathbb N}
\newcommand{\Z}{\mathbb Z}

\newcommand{\sagemath}{\textsf{SageMath}}


%\renewcommand{\d}{\,d\!}
\renewcommand{\d}{\mathop{}\!d}
\newcommand{\dd}[2][]{\frac{\d #1}{\d #2}}
\newcommand{\pp}[2][]{\frac{\partial #1}{\partial #2}}
\renewcommand{\l}{\ell}
\newcommand{\ddx}{\frac{d}{\d x}}

\newcommand{\zeroOverZero}{\ensuremath{\boldsymbol{\tfrac{0}{0}}}}
\newcommand{\inftyOverInfty}{\ensuremath{\boldsymbol{\tfrac{\infty}{\infty}}}}
\newcommand{\zeroOverInfty}{\ensuremath{\boldsymbol{\tfrac{0}{\infty}}}}
\newcommand{\zeroTimesInfty}{\ensuremath{\small\boldsymbol{0\cdot \infty}}}
\newcommand{\inftyMinusInfty}{\ensuremath{\small\boldsymbol{\infty - \infty}}}
\newcommand{\oneToInfty}{\ensuremath{\boldsymbol{1^\infty}}}
\newcommand{\zeroToZero}{\ensuremath{\boldsymbol{0^0}}}
\newcommand{\inftyToZero}{\ensuremath{\boldsymbol{\infty^0}}}



\newcommand{\numOverZero}{\ensuremath{\boldsymbol{\tfrac{\#}{0}}}}
\newcommand{\dfn}{\textbf}
%\newcommand{\unit}{\,\mathrm}
\newcommand{\unit}{\mathop{}\!\mathrm}
\newcommand{\eval}[1]{\bigg[ #1 \bigg]}
\newcommand{\seq}[1]{\left( #1 \right)}
\renewcommand{\epsilon}{\varepsilon}
\renewcommand{\phi}{\varphi}


\renewcommand{\iff}{\Leftrightarrow}

\DeclareMathOperator{\arccot}{arccot}
\DeclareMathOperator{\arcsec}{arcsec}
\DeclareMathOperator{\arccsc}{arccsc}
\DeclareMathOperator{\si}{Si}
\DeclareMathOperator{\scal}{scal}
\DeclareMathOperator{\sign}{sign}


%% \newcommand{\tightoverset}[2]{% for arrow vec
%%   \mathop{#2}\limits^{\vbox to -.5ex{\kern-0.75ex\hbox{$#1$}\vss}}}
\newcommand{\arrowvec}[1]{{\overset{\rightharpoonup}{#1}}}
%\renewcommand{\vec}[1]{\arrowvec{\mathbf{#1}}}
\renewcommand{\vec}[1]{{\overset{\boldsymbol{\rightharpoonup}}{\mathbf{#1}}}\hspace{0in}}

\newcommand{\point}[1]{\left(#1\right)} %this allows \vector{ to be changed to \vector{ with a quick find and replace
\newcommand{\pt}[1]{\mathbf{#1}} %this allows \vec{ to be changed to \vec{ with a quick find and replace
\newcommand{\Lim}[2]{\lim_{\point{#1} \to \point{#2}}} %Bart, I changed this to point since I want to use it.  It runs through both of the exercise and exerciseE files in limits section, which is why it was in each document to start with.

\DeclareMathOperator{\proj}{\mathbf{proj}}
\newcommand{\veci}{{\boldsymbol{\hat{\imath}}}}
\newcommand{\vecj}{{\boldsymbol{\hat{\jmath}}}}
\newcommand{\veck}{{\boldsymbol{\hat{k}}}}
\newcommand{\vecl}{\vec{\boldsymbol{\l}}}
\newcommand{\uvec}[1]{\mathbf{\hat{#1}}}
\newcommand{\utan}{\mathbf{\hat{t}}}
\newcommand{\unormal}{\mathbf{\hat{n}}}
\newcommand{\ubinormal}{\mathbf{\hat{b}}}

\newcommand{\dotp}{\bullet}
\newcommand{\cross}{\boldsymbol\times}
\newcommand{\grad}{\boldsymbol\nabla}
\newcommand{\divergence}{\grad\dotp}
\newcommand{\curl}{\grad\cross}
%\DeclareMathOperator{\divergence}{divergence}
%\DeclareMathOperator{\curl}[1]{\grad\cross #1}
\newcommand{\lto}{\mathop{\longrightarrow\,}\limits}

\renewcommand{\bar}{\overline}

\colorlet{textColor}{black}
\colorlet{background}{white}
\colorlet{penColor}{blue!50!black} % Color of a curve in a plot
\colorlet{penColor2}{red!50!black}% Color of a curve in a plot
\colorlet{penColor3}{red!50!blue} % Color of a curve in a plot
\colorlet{penColor4}{green!50!black} % Color of a curve in a plot
\colorlet{penColor5}{orange!80!black} % Color of a curve in a plot
\colorlet{penColor6}{yellow!70!black} % Color of a curve in a plot
\colorlet{fill1}{penColor!20} % Color of fill in a plot
\colorlet{fill2}{penColor2!20} % Color of fill in a plot
\colorlet{fillp}{fill1} % Color of positive area
\colorlet{filln}{penColor2!20} % Color of negative area
\colorlet{fill3}{penColor3!20} % Fill
\colorlet{fill4}{penColor4!20} % Fill
\colorlet{fill5}{penColor5!20} % Fill
\colorlet{gridColor}{gray!50} % Color of grid in a plot

\newcommand{\surfaceColor}{violet}
\newcommand{\surfaceColorTwo}{redyellow}
\newcommand{\sliceColor}{greenyellow}




\pgfmathdeclarefunction{gauss}{2}{% gives gaussian
  \pgfmathparse{1/(#2*sqrt(2*pi))*exp(-((x-#1)^2)/(2*#2^2))}%
}


%%%%%%%%%%%%%
%% Vectors
%%%%%%%%%%%%%

%% Simple horiz vectors
\renewcommand{\vector}[1]{\left\langle #1\right\rangle}


%% %% Complex Horiz Vectors with angle brackets
%% \makeatletter
%% \renewcommand{\vector}[2][ , ]{\left\langle%
%%   \def\nextitem{\def\nextitem{#1}}%
%%   \@for \el:=#2\do{\nextitem\el}\right\rangle%
%% }
%% \makeatother

%% %% Vertical Vectors
%% \def\vector#1{\begin{bmatrix}\vecListA#1,,\end{bmatrix}}
%% \def\vecListA#1,{\if,#1,\else #1\cr \expandafter \vecListA \fi}

%%%%%%%%%%%%%
%% End of vectors
%%%%%%%%%%%%%

%\newcommand{\fullwidth}{}
%\newcommand{\normalwidth}{}



%% makes a snazzy t-chart for evaluating functions
%\newenvironment{tchart}{\rowcolors{2}{}{background!90!textColor}\array}{\endarray}

%%This is to help with formatting on future title pages.
\newenvironment{sectionOutcomes}{}{}



%% Flowchart stuff
%\tikzstyle{startstop} = [rectangle, rounded corners, minimum width=3cm, minimum height=1cm,text centered, draw=black]
%\tikzstyle{question} = [rectangle, minimum width=3cm, minimum height=1cm, text centered, draw=black]
%\tikzstyle{decision} = [trapezium, trapezium left angle=70, trapezium right angle=110, minimum width=3cm, minimum height=1cm, text centered, draw=black]
%\tikzstyle{question} = [rectangle, rounded corners, minimum width=3cm, minimum height=1cm,text centered, draw=black]
%\tikzstyle{process} = [rectangle, minimum width=3cm, minimum height=1cm, text centered, draw=black]
%\tikzstyle{decision} = [trapezium, trapezium left angle=70, trapezium right angle=110, minimum width=3cm, minimum height=1cm, text centered, draw=black]


\author{Nela Lakos \and Bart Snapp}

\outcome{Find the domain and range of a function.}
\outcome{Perform basic operations and compositions on functions.}

\begin{document}
\begin{exercise}
  A box with an open top is to be constructed from a rectangular piece
  of cardboard with dimensions $12\unit{in}$ by $20\unit{in}$ by
  cutting out equal squares of side $x$ at each corner and then folding
  up the sides:
  \begin{image}
    \begin{tikzpicture}
      \draw [dashed] (0,0) rectangle (1,1);
      \draw [dashed] (6,0) rectangle (7,1);
      \draw [dashed] (6,4) rectangle (7,5);
      \draw [dashed] (0,4) rectangle (1,5);

      \draw [penColor, very thick, fill=fill3] (0,1) rectangle (1,4);
      \draw [penColor, very thick, fill=fill3] (1,0) rectangle (6,1);
      \draw [penColor, very thick, fill=fill3] (1,4) rectangle (6,5);
      \draw [penColor, very thick, fill=fill3] (6,1) rectangle (7,4);
      
      \draw [penColor,very thick,fill=fill1] (1,1) rectangle (6,4);

      \draw[decoration={brace,mirror,raise=.2cm},decorate,thin] (0,0)--(7,0);
      \draw[decoration={brace,raise=.2cm},decorate,thin] (0,0)--(0,5);
      
      \node [penColor] at (3.5,-.5) {$20$};
      \node [penColor] at (-.5,2.5) {$12$};

      \draw[decoration={brace,mirror,raise=.2cm},decorate,thin] (7,4)--(7,5);
      \draw[decoration={brace,raise=.2cm},decorate,thin] (6,5)--(7,5);
      
      \node [penColor] at (7.5,4.5) {$x$};
      \node [penColor] at (6.5,5.5) {$x$};
    \end{tikzpicture}
  \end{image}
  Express the volume $V$ of the box as a function of $x$.
  \[
  V(x) = \answer{x(20-2x)(12-2x)}
  \]
\end{exercise}
\begin{exercise}
  Simplify your answer above:
  \[
  V(x) = \answer{4}x^3 + \answer{-64}x^2 + \answer{240}x
  \]
\end{exercise}
\begin{exercise}
  What is the (natural) domain of $V$?
  \[
    \left[\answer{0},\answer{6}\right]
  \]
\end{exercise}
\begin{exercise}
  Is the function $V$ a polynomial?
  \begin{multipleChoice}
    \choice{yes}
    \choice[correct]{no}
  \end{multipleChoice}
  \begin{feedback}
    $V$ is not a polynomial. The domain of a polynomial is
    $(-\infty,\infty)$. $V$ is a polynomial on a restricted domain.
  \end{feedback}
\end{exercise}
\begin{exercise}
  Evaluate (either write a number or ``DNE'' (Does Not Exist)):
  \begin{align*}
    V(1) &= \answer{180}\\
    V(5) &= \answer{100}\\
    V(10) &= \answer{DNE}
  \end{align*}
\end{exercise}
\begin{exercise}
  If $x = 1$ and increases by $h$, by how much will volume of the box change? Simplify.
  \begin{hint}
    You may find the following formulas useful:
    \begin{align*}
      (a+b)^2 &= a^2 + 2ab + b^2\\
      (a + b)^3 &= a^3 + 3 a^2 b + 3 ab^2 + b^3
    \end{align*}
  \end{hint}
  \[
  V(1+h) - V(1) =4(1+h)^3 -64(1+h)^2 +240(1+h)-180=\answer{4}h^3 +\answer{-52}h^2 + \answer{124}h
  \]
\end{exercise}

\begin{exercise}
  As $x$ increases from $x=1$ to $x=5$, what is $h$?
  \[
  h=\answer{4}
  \]
  Now, compute
  \begin{align*}
    V(5) &= \answer{100}\\
    V(1) &= \answer{180}\\
    V(5) - V(1) &= \answer{-80}
  \end{align*}
  \begin{exercise}
    We also know from your work above that
    \begin{align*}
      V(1+h) -V(1) &= 4h^3 -52h^2 + 124h\\
      V(1+4) -V(1) &= 4(4)^3 -52(4)^2 + 124(4)\\
      V(1+4) - V(1) &= \answer{-80}
    \end{align*}
  \end{exercise}
\end{exercise}

\begin{exercise}
  Express the total surface area (inside and outside) of the open box
  as a function of $x$.
  \[
  S(x) = (20-2x)(12-2x)+2(12-2x)\cdot\left(\answer{x}\right)+2x\cdot\left(\answer{20-2x}\right)=\answer{480-8x^2}
  \]
\end{exercise}
\begin{exercise}
  As $x$ increases from $1$ to $(1+h)$, the total surface area of the
  box changes by
  \[
  S(1+h) - S(1) = \answer{-8}h^2 + \answer{-16}h
  \]
  So as $x$ increases from $1$ to  $(1+h)$, the surface areas decreases by
  $\answer{16h+8h^2}$.

  As a final exercise, illustrate this change in a picture and share your picture with a classmate.
  \begin{multipleChoice}
    \choice[correct]{I have done this.}
    \choice{I have not done this.}
  \end{multipleChoice}
\end{exercise}
\end{document}
