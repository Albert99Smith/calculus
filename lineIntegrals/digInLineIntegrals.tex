\documentclass{ximera}

%\usepackage{todonotes}
%\usepackage{mathtools} %% Required for wide table Curl and Greens
%\usepackage{cuted} %% Required for wide table Curl and Greens
\newcommand{\todo}{}

\usepackage{esint} % for \oiint
\ifxake%%https://math.meta.stackexchange.com/questions/9973/how-do-you-render-a-closed-surface-double-integral
\renewcommand{\oiint}{{\large\bigcirc}\kern-1.56em\iint}
\fi


\graphicspath{
  {./}
  {ximeraTutorial/}
  {basicPhilosophy/}
  {functionsOfSeveralVariables/}
  {normalVectors/}
  {lagrangeMultipliers/}
  {vectorFields/}
  {greensTheorem/}
  {shapeOfThingsToCome/}
  {dotProducts/}
  {partialDerivativesAndTheGradientVector/}
  {../productAndQuotientRules/exercises/}
  {../normalVectors/exercisesParametricPlots/}
  {../continuityOfFunctionsOfSeveralVariables/exercises/}
  {../partialDerivativesAndTheGradientVector/exercises/}
  {../directionalDerivativeAndChainRule/exercises/}
  {../commonCoordinates/exercisesCylindricalCoordinates/}
  {../commonCoordinates/exercisesSphericalCoordinates/}
  {../greensTheorem/exercisesCurlAndLineIntegrals/}
  {../greensTheorem/exercisesDivergenceAndLineIntegrals/}
  {../shapeOfThingsToCome/exercisesDivergenceTheorem/}
  {../greensTheorem/}
  {../shapeOfThingsToCome/}
  {../separableDifferentialEquations/exercises/}
  {vectorFields/}
}

\newcommand{\mooculus}{\textsf{\textbf{MOOC}\textnormal{\textsf{ULUS}}}}

\usepackage{tkz-euclide}\usepackage{tikz}
\usepackage{tikz-cd}
\usetikzlibrary{arrows}
\tikzset{>=stealth,commutative diagrams/.cd,
  arrow style=tikz,diagrams={>=stealth}} %% cool arrow head
\tikzset{shorten <>/.style={ shorten >=#1, shorten <=#1 } } %% allows shorter vectors

\usetikzlibrary{backgrounds} %% for boxes around graphs
\usetikzlibrary{shapes,positioning}  %% Clouds and stars
\usetikzlibrary{matrix} %% for matrix
\usepgfplotslibrary{polar} %% for polar plots
\usepgfplotslibrary{fillbetween} %% to shade area between curves in TikZ
\usetkzobj{all}
\usepackage[makeroom]{cancel} %% for strike outs
%\usepackage{mathtools} %% for pretty underbrace % Breaks Ximera
%\usepackage{multicol}
\usepackage{pgffor} %% required for integral for loops



%% http://tex.stackexchange.com/questions/66490/drawing-a-tikz-arc-specifying-the-center
%% Draws beach ball
\tikzset{pics/carc/.style args={#1:#2:#3}{code={\draw[pic actions] (#1:#3) arc(#1:#2:#3);}}}



\usepackage{array}
\setlength{\extrarowheight}{+.1cm}
\newdimen\digitwidth
\settowidth\digitwidth{9}
\def\divrule#1#2{
\noalign{\moveright#1\digitwidth
\vbox{\hrule width#2\digitwidth}}}





\newcommand{\RR}{\mathbb R}
\newcommand{\R}{\mathbb R}
\newcommand{\N}{\mathbb N}
\newcommand{\Z}{\mathbb Z}

\newcommand{\sagemath}{\textsf{SageMath}}


%\renewcommand{\d}{\,d\!}
\renewcommand{\d}{\mathop{}\!d}
\newcommand{\dd}[2][]{\frac{\d #1}{\d #2}}
\newcommand{\pp}[2][]{\frac{\partial #1}{\partial #2}}
\renewcommand{\l}{\ell}
\newcommand{\ddx}{\frac{d}{\d x}}

\newcommand{\zeroOverZero}{\ensuremath{\boldsymbol{\tfrac{0}{0}}}}
\newcommand{\inftyOverInfty}{\ensuremath{\boldsymbol{\tfrac{\infty}{\infty}}}}
\newcommand{\zeroOverInfty}{\ensuremath{\boldsymbol{\tfrac{0}{\infty}}}}
\newcommand{\zeroTimesInfty}{\ensuremath{\small\boldsymbol{0\cdot \infty}}}
\newcommand{\inftyMinusInfty}{\ensuremath{\small\boldsymbol{\infty - \infty}}}
\newcommand{\oneToInfty}{\ensuremath{\boldsymbol{1^\infty}}}
\newcommand{\zeroToZero}{\ensuremath{\boldsymbol{0^0}}}
\newcommand{\inftyToZero}{\ensuremath{\boldsymbol{\infty^0}}}



\newcommand{\numOverZero}{\ensuremath{\boldsymbol{\tfrac{\#}{0}}}}
\newcommand{\dfn}{\textbf}
%\newcommand{\unit}{\,\mathrm}
\newcommand{\unit}{\mathop{}\!\mathrm}
\newcommand{\eval}[1]{\bigg[ #1 \bigg]}
\newcommand{\seq}[1]{\left( #1 \right)}
\renewcommand{\epsilon}{\varepsilon}
\renewcommand{\phi}{\varphi}


\renewcommand{\iff}{\Leftrightarrow}

\DeclareMathOperator{\arccot}{arccot}
\DeclareMathOperator{\arcsec}{arcsec}
\DeclareMathOperator{\arccsc}{arccsc}
\DeclareMathOperator{\si}{Si}
\DeclareMathOperator{\scal}{scal}
\DeclareMathOperator{\sign}{sign}


%% \newcommand{\tightoverset}[2]{% for arrow vec
%%   \mathop{#2}\limits^{\vbox to -.5ex{\kern-0.75ex\hbox{$#1$}\vss}}}
\newcommand{\arrowvec}[1]{{\overset{\rightharpoonup}{#1}}}
%\renewcommand{\vec}[1]{\arrowvec{\mathbf{#1}}}
\renewcommand{\vec}[1]{{\overset{\boldsymbol{\rightharpoonup}}{\mathbf{#1}}}\hspace{0in}}

\newcommand{\point}[1]{\left(#1\right)} %this allows \vector{ to be changed to \vector{ with a quick find and replace
\newcommand{\pt}[1]{\mathbf{#1}} %this allows \vec{ to be changed to \vec{ with a quick find and replace
\newcommand{\Lim}[2]{\lim_{\point{#1} \to \point{#2}}} %Bart, I changed this to point since I want to use it.  It runs through both of the exercise and exerciseE files in limits section, which is why it was in each document to start with.

\DeclareMathOperator{\proj}{\mathbf{proj}}
\newcommand{\veci}{{\boldsymbol{\hat{\imath}}}}
\newcommand{\vecj}{{\boldsymbol{\hat{\jmath}}}}
\newcommand{\veck}{{\boldsymbol{\hat{k}}}}
\newcommand{\vecl}{\vec{\boldsymbol{\l}}}
\newcommand{\uvec}[1]{\mathbf{\hat{#1}}}
\newcommand{\utan}{\mathbf{\hat{t}}}
\newcommand{\unormal}{\mathbf{\hat{n}}}
\newcommand{\ubinormal}{\mathbf{\hat{b}}}

\newcommand{\dotp}{\bullet}
\newcommand{\cross}{\boldsymbol\times}
\newcommand{\grad}{\boldsymbol\nabla}
\newcommand{\divergence}{\grad\dotp}
\newcommand{\curl}{\grad\cross}
%\DeclareMathOperator{\divergence}{divergence}
%\DeclareMathOperator{\curl}[1]{\grad\cross #1}
\newcommand{\lto}{\mathop{\longrightarrow\,}\limits}

\renewcommand{\bar}{\overline}

\colorlet{textColor}{black}
\colorlet{background}{white}
\colorlet{penColor}{blue!50!black} % Color of a curve in a plot
\colorlet{penColor2}{red!50!black}% Color of a curve in a plot
\colorlet{penColor3}{red!50!blue} % Color of a curve in a plot
\colorlet{penColor4}{green!50!black} % Color of a curve in a plot
\colorlet{penColor5}{orange!80!black} % Color of a curve in a plot
\colorlet{penColor6}{yellow!70!black} % Color of a curve in a plot
\colorlet{fill1}{penColor!20} % Color of fill in a plot
\colorlet{fill2}{penColor2!20} % Color of fill in a plot
\colorlet{fillp}{fill1} % Color of positive area
\colorlet{filln}{penColor2!20} % Color of negative area
\colorlet{fill3}{penColor3!20} % Fill
\colorlet{fill4}{penColor4!20} % Fill
\colorlet{fill5}{penColor5!20} % Fill
\colorlet{gridColor}{gray!50} % Color of grid in a plot

\newcommand{\surfaceColor}{violet}
\newcommand{\surfaceColorTwo}{redyellow}
\newcommand{\sliceColor}{greenyellow}




\pgfmathdeclarefunction{gauss}{2}{% gives gaussian
  \pgfmathparse{1/(#2*sqrt(2*pi))*exp(-((x-#1)^2)/(2*#2^2))}%
}


%%%%%%%%%%%%%
%% Vectors
%%%%%%%%%%%%%

%% Simple horiz vectors
\renewcommand{\vector}[1]{\left\langle #1\right\rangle}


%% %% Complex Horiz Vectors with angle brackets
%% \makeatletter
%% \renewcommand{\vector}[2][ , ]{\left\langle%
%%   \def\nextitem{\def\nextitem{#1}}%
%%   \@for \el:=#2\do{\nextitem\el}\right\rangle%
%% }
%% \makeatother

%% %% Vertical Vectors
%% \def\vector#1{\begin{bmatrix}\vecListA#1,,\end{bmatrix}}
%% \def\vecListA#1,{\if,#1,\else #1\cr \expandafter \vecListA \fi}

%%%%%%%%%%%%%
%% End of vectors
%%%%%%%%%%%%%

%\newcommand{\fullwidth}{}
%\newcommand{\normalwidth}{}



%% makes a snazzy t-chart for evaluating functions
%\newenvironment{tchart}{\rowcolors{2}{}{background!90!textColor}\array}{\endarray}

%%This is to help with formatting on future title pages.
\newenvironment{sectionOutcomes}{}{}



%% Flowchart stuff
%\tikzstyle{startstop} = [rectangle, rounded corners, minimum width=3cm, minimum height=1cm,text centered, draw=black]
%\tikzstyle{question} = [rectangle, minimum width=3cm, minimum height=1cm, text centered, draw=black]
%\tikzstyle{decision} = [trapezium, trapezium left angle=70, trapezium right angle=110, minimum width=3cm, minimum height=1cm, text centered, draw=black]
%\tikzstyle{question} = [rectangle, rounded corners, minimum width=3cm, minimum height=1cm,text centered, draw=black]
%\tikzstyle{process} = [rectangle, minimum width=3cm, minimum height=1cm, text centered, draw=black]
%\tikzstyle{decision} = [trapezium, trapezium left angle=70, trapezium right angle=110, minimum width=3cm, minimum height=1cm, text centered, draw=black]


\outcome{Identify a line integral.}
\outcome{Reason about the sign of a line integral.}
\outcome{Compute line integrals.}
\outcome{Estimate line integrals.}
\outcome{State the Fundamental Theorem of calculus for line integrals.}
\outcome{Use the Fundamental Theorem of calculus to simplify computation.}
\outcome{Identify conservative fields to compute integrals.}

\title[Dig-In:]{Line integrals}

\begin{document}
\begin{abstract}
We accumulate vectors along a path.
\end{abstract}
\maketitle

In this section we introduce a new type of integrals, \textit{line
  integrals} also known as \textit{path integrals}. Let's start with a
mental model for what a line integral is.

\section{The idea of line integrals}

Suppose you are flying to \link[Olinda Brazil]{https://en.wikipedia.org/wiki/Olinda} for a mathematics
conference. Flying out of Columbus Ohio, you have a layover in Atlanta. Because
you bought your tickets as cheaply as possible, you have very little
time to catch your plane in Atlanta. Hence, your flight time is of crucial
interest to you. Fortunately or unfortunately, hurricane
\textit{Gauss} (note this name is not actually
\link[possible]{https://oceanservice.noaa.gov/facts/storm-names.html}
for a hurricane) is creating ``interesting''a wind patterns. Of crucial
interest to you is the following question:
\begin{quote}
  Is the flow of the wind going with the path of your airplane, or
  against? 
\end{quote}
If the flow of the air is with the path and direction of your plane,
you will pick up a tailwind and get to Atlanta in plenty of time to
catch your next flight. If flow of the wind is against the path and
direction of your plane, you might miss your connecting flight and be
forced to spend days in the Atlanta airport reading science fiction
novels.

Let $\vec{W}(x,y)$ be a vector field that represents \link[wind
  currents]{https://www.windy.com/}. Let $\vec{p}(t)$ be a
vector-valued function describing the path of your plane where $t$
represents the times you are in flight assuming no wind.  The velocity
vector of the plane is given by $\vec{p}'(t)$. To see if the wind
currents are pushing with the plane or against the plane for any given
time, you compute:
\[
\underbrace{\vec{W}(\vec{p}(t))}_{\text{wind currents}}\dotp \underbrace{\vec{p}'(t)}_{\text{velocity of the plane}}
\]
Here the dot product measures ``how aligned'' these two vectors are.
If the dot product is postive, the wind is pushing with the plane and
speeding-up your flight. If the dot product is negative, the wind is
pushing against the plane and is slowing your flight.

To see the net accumulation of the wind on the plane's flight, you
should hence integrate with respect to time:
\[
\int_{t_\text{takeoff}}^{t_\text{landing}}
\vec{W}(\vec{p}(t)) \dotp\vec{p}'(t) \d t
\]
This integral will measure the accumulated contribution of the wind to
the flight of the plane. If the integral is positive, you will arrive
in Atlanta early, if the integral is negative you will arrive late.



Let's start with
the definition of a \textit{line integral}:


\begin{definition}
Let $\vec{F}:\R^2\to\R^2$ be a vector field, $\vec{p}:\R\to\R^2$ be a
smooth vector valued function tracing a curve $C$ exactly once as $t$
runs from $a$ to $b$,
\begin{align*}
  \vec{F}(x,y) &= \vector{M(x,y), N(x,y)}\\
  \vec{p}(t) &= \vector{x(t),y(t)}.
\end{align*}
A \dfn{line integral} is an integral of the form:
\begin{align*}
\int_C \vec{F}\dotp \d \vec{p} &= \int_C \vector{M,N}\dotp\vector{\d x,\d y}\\
&= \int_C M(x,y)\d x + N(x,y)\d y
\end{align*}
Since $\d x = x'(t)\d t$ and $\d y = y'(t)\d t$, we may write
\begin{align*}
&= \int_a^b \left(M(x(t),y(t))\cdot x'(t) + N(x(t),y(t))\cdot  y'(t)\right) \d t\\
&= \int_a^b \vector{M(x(t),y(t)),N(x(t),y(t))}\dotp\vector{x'(t), y'(t)} \d t
\end{align*}
\end{definition}

If the path $C$ is closed, then sometimes people write a ``circle'' on the
integral sign:
\[
\oint_C \vec{F}\dotp \d \vec{p}
\]
This notation is not critical, but it can sometimes help us from
making silly mistakes.

\begin{question}
  Which of the following are line integrals?
  \begin{selectAll}
    \choice{$\int_R (x^2+y^2) \d A$}
    \choice[correct]{$\int_C \left( -y\d x + x\d y \right)$}
    \choice{$\int_3^4\int_2^3 \ln(xy) \d x \d y$}
    \choice[correct]{$\int_0^{2\pi} \vector{-\sin(t),\cos(t)}\dotp\vector{-\sin(t),\cos(t)}\d t$}
  \end{selectAll}
\end{question}



Read on to learn the meaning of this new integral.


\section{What do line integrals measure?}

A line integral measures the flow of a vector field along a path. The
basic idea is that there is some vector field given by $\vec{F}$:
\begin{image}
\begin{tikzpicture}
      \begin{axis}%
        [hide axis,
          ymin=-4.5,ymax=2.5,
          xmin=-6,xmax=5.5,
        ]
        \addplot[penColor,thick, ->] coordinates{(-6,2) (-.5,2)};
        \addplot[penColor,thick, ->] coordinates{(0,2) (5.5,2)};

        \addplot[penColor,thick, ->] coordinates{(-6,1) (-2.5,1)};
        \addplot[penColor,thick, ->] coordinates{(-2,1) (1.5,1)};
        \addplot[penColor,thick, ->] coordinates{(2,1) (5.5,1)};

        \addplot[penColor,thick, ->] coordinates{(-6,0) (-3.5,0)};
        \addplot[penColor,thick, ->] coordinates{(-3,0) (-.5,0)};
        \addplot[penColor,thick, ->] coordinates{(0,0) (2.5,0)};
        \addplot[penColor,thick, ->] coordinates{(3,0) (5.5,0)};

        \addplot[penColor,thick, ->] coordinates{(-6,-1) (-4.5,-1)};
        \addplot[penColor,thick, ->] coordinates{(-4,-1) (-2.5,-1)};
        \addplot[penColor,thick, ->] coordinates{(-2,-1) (-.5,-1)};
        \addplot[penColor,thick, ->] coordinates{(0,-1) (1.5,-1)};
        \addplot[penColor,thick, ->] coordinates{(2,-1) (3.5,-1)};
        \addplot[penColor,thick, ->] coordinates{(4,-1) (5.5,-1)};

        \addplot[penColor,thick, ->] coordinates{(-6,-2) (-5.5,-2)};
        \addplot[penColor,thick, ->] coordinates{(-5,-2) (-4.5,-2)};
        \addplot[penColor,thick, ->] coordinates{(-4,-2) (-3.5,-2)};
        \addplot[penColor,thick, ->] coordinates{(-3,-2) (-2.5,-2)};
        \addplot[penColor,thick, ->] coordinates{(-2,-2) (-1.5,-2)};
        \addplot[penColor,thick, ->] coordinates{(-1,-2) (-.5,-2)};
        \addplot[penColor,thick, ->] coordinates{(0,-2) (.5,-2)};
        \addplot[penColor,thick, ->] coordinates{(1,-2) (1.5,-2)};
        \addplot[penColor,thick, ->] coordinates{(2,-2) (2.5,-2)};
        \addplot[penColor,thick, ->] coordinates{(3,-2) (3.5,-2)};
        \addplot[penColor,thick, ->] coordinates{(4,-2) (4.5,-2)};
        \addplot[penColor,thick, ->] coordinates{(5,-2) (5.5,-2)};
        \node[inner sep=0pt,text width=8cm,right,scale=.85] at (axis cs:-6,-3.5)
             {\footnotesize One should imagine a vector at
               \textbf{every} point. We'll assume that the magnitudes
               of the vectors are constant along horizontal lines.};
      \end{axis}
 \end{tikzpicture}
\end{image}


Now we add an oriented path $C$ that is parameterized by $\vec{p}(t) =
\vector{x(t),y(t)}$. This can be thought of as a path that an object
takes through the field. To reason via a specific example, we'll add a
path below:
\begin{image}
  \begin{tikzpicture}
    \begin{axis}%
      [hide axis,
	ymin=-3,ymax=2.5,
	xmin=-6,xmax=5.5,
	]
      \addplot[penColor,thick, ->] coordinates{(-6,2) (-.5,2)};
      \addplot[penColor,thick, ->] coordinates{(0,2) (5.5,2)};
      
      \addplot[penColor,thick, ->] coordinates{(-6,1) (-2.5,1)};
      \addplot[penColor,thick, ->] coordinates{(-2,1) (1.5,1)};
      \addplot[penColor,thick, ->] coordinates{(2,1) (5.5,1)};
      
      \addplot[penColor,thick, ->] coordinates{(-6,0) (-3.5,0)};
      \addplot[penColor,thick, ->] coordinates{(-3,0) (-.5,0)};
      \addplot[penColor,thick, ->] coordinates{(0,0) (2.5,0)};
      \addplot[penColor,thick, ->] coordinates{(3,0) (5.5,0)};
      
      \addplot[penColor,thick, ->] coordinates{(-6,-1) (-4.5,-1)};
      \addplot[penColor,thick, ->] coordinates{(-4,-1) (-2.5,-1)};
      \addplot[penColor,thick, ->] coordinates{(-2,-1) (-.5,-1)};
      \addplot[penColor,thick, ->] coordinates{(0,-1) (1.5,-1)};
      \addplot[penColor,thick, ->] coordinates{(2,-1) (3.5,-1)};
      \addplot[penColor,thick, ->] coordinates{(4,-1) (5.5,-1)};
      
      \addplot[penColor,thick, ->] coordinates{(-6,-2) (-5.5,-2)};
      \addplot[penColor,thick, ->] coordinates{(-5,-2) (-4.5,-2)};
      \addplot[penColor,thick, ->] coordinates{(-4,-2) (-3.5,-2)};
      \addplot[penColor,thick, ->] coordinates{(-3,-2) (-2.5,-2)};
      \addplot[penColor,thick, ->] coordinates{(-2,-2) (-1.5,-2)};
      \addplot[penColor,thick, ->] coordinates{(-1,-2) (-.5,-2)};
      \addplot[penColor,thick, ->] coordinates{(0,-2) (.5,-2)};
      \addplot[penColor,thick, ->] coordinates{(1,-2) (1.5,-2)};
      \addplot[penColor,thick, ->] coordinates{(2,-2) (2.5,-2)};
      \addplot[penColor,thick, ->] coordinates{(3,-2) (3.5,-2)};
      \addplot[penColor,thick, ->] coordinates{(4,-2) (4.5,-2)};
      \addplot[penColor,thick, ->] coordinates{(5,-2) (5.5,-2)};
        
      \addplot[penColor2,ultra thick] coordinates{
        (-3,2.3) (3,2.3)
        (3,-2.3) (-3,-2.3)
      };
      \addplot[penColor2,ultra thick, ->] coordinates{(-3,2.3) (0,2.3)};
      \addplot[penColor2,ultra thick, ->] coordinates{(3,2.3) (3,0)};
      \addplot[penColor2,ultra thick, ->] coordinates{(3,-2.3) (0,-2.3)};
    \end{axis}
  \end{tikzpicture}
\end{image}

To figure out if the flow of the vector field is ``with'' the
direction of the path, we use the dot product:
\[
\underbrace{\vec{F}(x(t),y(t))}_{\text{direction of field}} \dotp \underbrace{\vector{x'(t) \d t,y'(t) \d t}}_{\text{direction of path}}
\]
\begin{question}
When the direction of the field and the direction of the path are in
alignment, the dot product is\dots
\begin{prompt}
  \begin{multipleChoice}
    \choice[correct]{positive}
    \choice{zero}
    \choice{negative}
  \end{multipleChoice}
\end{prompt}
\begin{question}
  When the direction of the
  field and the direction of the path are orthogonal, the dot product is\dots
  \begin{prompt}
  \begin{multipleChoice}
    \choice{positive}
    \choice[correct]{zero}
    \choice{negative}
  \end{multipleChoice}
\end{prompt}
\begin{question}
  When the direction of the field and the direction of the path are in
  opposite direction, the dot product is\dots
  \begin{prompt}
  \begin{multipleChoice}
    \choice{positive}
    \choice{zero}
    \choice[correct]{negative}
  \end{multipleChoice}
  \end{prompt}
\end{question}
\end{question}
\end{question}
Integrating over the path sums these infinitesimal measurements:
\[
\vec{F}(x(t),y(t))\dotp \vector{x'(t),y'(t)} \d t = \vec{F} \dotp\d\vec{p}
\]
Thus the line integral
\[
\int_C \vec{F}\dotp \d \vec{p}
\]
measures the flow of a field along a path. In particular, if the value
of the line integral is positive, then the flow is with the path; if
the value is negative, then the flow is against the path.



\begin{question}
  Consider the following vector field along with a (directed) curve
  $C$.
  \begin{image}
    \begin{tikzpicture}
      \begin{axis}%
        [hide axis,
	  ymin=-3,ymax=2.5,
	  xmin=-6,xmax=5.5,
	]
        \addplot[penColor,thick, ->] coordinates{(-6,2) (-.5,2)};
        \addplot[penColor,thick, ->] coordinates{(0,2) (5.5,2)};

        \addplot[penColor,thick, ->] coordinates{(-6,1) (-2.5,1)};
        \addplot[penColor,thick, ->] coordinates{(-2,1) (1.5,1)};
        \addplot[penColor,thick, ->] coordinates{(2,1) (5.5,1)};

        \addplot[penColor,thick, ->] coordinates{(-6,0) (-3.5,0)};
        \addplot[penColor,thick, ->] coordinates{(-3,0) (-.5,0)};
        \addplot[penColor,thick, ->] coordinates{(0,0) (2.5,0)};
        \addplot[penColor,thick, ->] coordinates{(3,0) (5.5,0)};

        \addplot[penColor,thick, ->] coordinates{(-6,-1) (-4.5,-1)};
        \addplot[penColor,thick, ->] coordinates{(-4,-1) (-2.5,-1)};
        \addplot[penColor,thick, ->] coordinates{(-2,-1) (-.5,-1)};
        \addplot[penColor,thick, ->] coordinates{(0,-1) (1.5,-1)};
        \addplot[penColor,thick, ->] coordinates{(2,-1) (3.5,-1)};
        \addplot[penColor,thick, ->] coordinates{(4,-1) (5.5,-1)};
        
        \addplot[penColor,thick, ->] coordinates{(-6,-2) (-5.5,-2)};
        \addplot[penColor,thick, ->] coordinates{(-5,-2) (-4.5,-2)};
        \addplot[penColor,thick, ->] coordinates{(-4,-2) (-3.5,-2)};
        \addplot[penColor,thick, ->] coordinates{(-3,-2) (-2.5,-2)};
        \addplot[penColor,thick, ->] coordinates{(-2,-2) (-1.5,-2)};
        \addplot[penColor,thick, ->] coordinates{(-1,-2) (-.5,-2)};
        \addplot[penColor,thick, ->] coordinates{(0,-2) (.5,-2)};
        \addplot[penColor,thick, ->] coordinates{(1,-2) (1.5,-2)};
        \addplot[penColor,thick, ->] coordinates{(2,-2) (2.5,-2)};
        \addplot[penColor,thick, ->] coordinates{(3,-2) (3.5,-2)};
        \addplot[penColor,thick, ->] coordinates{(4,-2) (4.5,-2)};
        \addplot[penColor,thick, ->] coordinates{(5,-2) (5.5,-2)};
        
        \addplot[penColor2,ultra thick] coordinates{
          (-3,2.3) (3,2.3)
          (3,-2.3) (-3,-2.3)
        };
        \addplot[penColor2,ultra thick, ->] coordinates{(-3,2.3) (0,2.3)};
        \addplot[penColor2,ultra thick, ->] coordinates{(3,2.3) (3,0)};
        \addplot[penColor2,ultra thick, ->] coordinates{(3,-2.3) (0,-2.3)};
      \end{axis}
    \end{tikzpicture}
  \end{image}
  Do you expect 
  \[
  \int_C \vec{F}\dotp \d \vec{p} 
  \]
  to be positive, zero, or negative?
  \begin{prompt}
  \begin{multipleChoice}
    \choice[correct]{positive}
    \choice{zero}
    \choice{negative}
  \end{multipleChoice}
  \begin{hint}
    We can think about this better if we break the path into pieces:
    $C_1$, $C_2$, $C_3$.
      \begin{image}
    \begin{tikzpicture}
      \begin{axis}%
        [hide axis,
	  ymin=-3,ymax=2.5,
	  xmin=-6,xmax=5.5,
	]
        \addplot[penColor,thick, ->] coordinates{(-6,2) (-.5,2)};
        \addplot[penColor,thick, ->] coordinates{(0,2) (5.5,2)};

        \addplot[penColor,thick, ->] coordinates{(-6,1) (-2.5,1)};
        \addplot[penColor,thick, ->] coordinates{(-2,1) (1.5,1)};
        \addplot[penColor,thick, ->] coordinates{(2,1) (5.5,1)};

        \addplot[penColor,thick, ->] coordinates{(-6,0) (-3.5,0)};
        \addplot[penColor,thick, ->] coordinates{(-3,0) (-.5,0)};
        \addplot[penColor,thick, ->] coordinates{(0,0) (2.5,0)};
        \addplot[penColor,thick, ->] coordinates{(3,0) (5.5,0)};

        \addplot[penColor,thick, ->] coordinates{(-6,-1) (-4.5,-1)};
        \addplot[penColor,thick, ->] coordinates{(-4,-1) (-2.5,-1)};
        \addplot[penColor,thick, ->] coordinates{(-2,-1) (-.5,-1)};
        \addplot[penColor,thick, ->] coordinates{(0,-1) (1.5,-1)};
        \addplot[penColor,thick, ->] coordinates{(2,-1) (3.5,-1)};
        \addplot[penColor,thick, ->] coordinates{(4,-1) (5.5,-1)};
        
        \addplot[penColor,thick, ->] coordinates{(-6,-2) (-5.5,-2)};
        \addplot[penColor,thick, ->] coordinates{(-5,-2) (-4.5,-2)};
        \addplot[penColor,thick, ->] coordinates{(-4,-2) (-3.5,-2)};
        \addplot[penColor,thick, ->] coordinates{(-3,-2) (-2.5,-2)};
        \addplot[penColor,thick, ->] coordinates{(-2,-2) (-1.5,-2)};
        \addplot[penColor,thick, ->] coordinates{(-1,-2) (-.5,-2)};
        \addplot[penColor,thick, ->] coordinates{(0,-2) (.5,-2)};
        \addplot[penColor,thick, ->] coordinates{(1,-2) (1.5,-2)};
        \addplot[penColor,thick, ->] coordinates{(2,-2) (2.5,-2)};
        \addplot[penColor,thick, ->] coordinates{(3,-2) (3.5,-2)};
        \addplot[penColor,thick, ->] coordinates{(4,-2) (4.5,-2)};
        \addplot[penColor,thick, ->] coordinates{(5,-2) (5.5,-2)};
        
        \addplot[penColor2,ultra thick] coordinates{
          (-3,2.3) (3,2.3)
        };
        \addplot[penColor4,ultra thick] coordinates{
          (3,2.3)
          (3,-2.3) 
        };
        \addplot[penColor5,ultra thick] coordinates{
          (3,-2.3) (-3,-2.3)
          };
        \addplot[penColor2,ultra thick, ->] coordinates{(-3,2.3) (0,2.3)};
        \addplot[penColor4,ultra thick, ->] coordinates{(3,2.3) (3,0)};
        \addplot[penColor5,ultra thick, ->] coordinates{(3,-2.3) (0,-2.3)};

        \node[below,penColor2] at (axis cs: 0,2) {$C_1$};
        \node[above right,penColor4] at (axis cs: 3,0) {$C_2$};
        \node[above,penColor5] at (axis cs: 0,-2) {$C_3$};
      \end{axis}
    \end{tikzpicture}
      \end{image}
      We can see that the vectors are flowing with the direction of
      $C_1$. Note that the the magnitude of these vectors is large, so
      this contributes a large positive value to our integral.

      The field vectors are orthogonal to the direction of the path
      $C_2$. So this part contributes nothing to the integral.

      The field vectors are flowing against the direction of
      $C_3$. However, their magnitude is much less than the vectors
      that flowed with $C_1$. So this contributes a small negative
      value to our integral.
  \end{hint}
  \end{prompt}
\end{question}






\begin{question}
  Consider the following vector field along with a (directed) curve
  $C$.
  \begin{image}
    \begin{tikzpicture}
      \begin{axis}%
        [hide axis,
	  ymin=-3,ymax=2.5,
	  xmin=-6,xmax=5.5,
	]
        \addplot[penColor,thick, ->] coordinates{(-4,0) (-4,2.5)};
        \addplot[penColor,thick, ->] coordinates{(-3,0) (-3,2.5)};
        \addplot[penColor,thick, ->] coordinates{(-2,0) (-2,2.5)};
        \addplot[penColor,thick, ->] coordinates{(-1,0) (-1,2.5)};
        \addplot[penColor,thick, ->] coordinates{(0,0) (0,2.5)};
        \addplot[penColor,thick, ->] coordinates{(1,0) (1,2.5)};
        \addplot[penColor,thick, ->] coordinates{(2,0) (2,2.5)};
        \addplot[penColor,thick, ->] coordinates{(3,0) (3,2.5)};
        \addplot[penColor,thick, ->] coordinates{(4,0) (4,2.5)};

        \addplot[penColor,thick, ->] coordinates{(-4,-2) (-4,-.25)};
        \addplot[penColor,thick, ->] coordinates{(-3,-2) (-3,-.25)};
        \addplot[penColor,thick, ->] coordinates{(-2,-2) (-2,-.25)};
        \addplot[penColor,thick, ->] coordinates{(-1,-2) (-1,-.25)};
        \addplot[penColor,thick, ->] coordinates{(0,-2) (0,-.25)};
        \addplot[penColor,thick, ->] coordinates{(1,-2) (1,-.25)};
        \addplot[penColor,thick, ->] coordinates{(2,-2) (2,-.25)};
        \addplot[penColor,thick, ->] coordinates{(3,-2) (3,-.25)};
        \addplot[penColor,thick, ->] coordinates{(4,-2) (4,-.25)};

        \addplot[penColor,thick, ->] coordinates{(-4,-3) (-4,-2.2)};
        \addplot[penColor,thick, ->] coordinates{(-3,-3) (-3,-2.2)};
        \addplot[penColor,thick, ->] coordinates{(-2,-3) (-2,-2.2)};
        \addplot[penColor,thick, ->] coordinates{(-1,-3) (-1,-2.2)};
        \addplot[penColor,thick, ->] coordinates{(0,-3) (0,-2.2)};
        \addplot[penColor,thick, ->] coordinates{(1,-3) (1,-2.2)};
        \addplot[penColor,thick, ->] coordinates{(2,-3) (2,-2.2)};
        \addplot[penColor,thick, ->] coordinates{(3,-3) (3,-2.2)};
        \addplot[penColor,thick, ->] coordinates{(4,-3) (4,-2.2)};

        
        
        
        \addplot[penColor2,ultra thick] coordinates{
          (-3,1.7) (3.5,1.7)
          (3.5,-2.5) (-3,-2.5)
        };
        \addplot[penColor2,ultra thick, ->] coordinates{(-3,1.7) (0,1.7)};
        \addplot[penColor2,ultra thick, ->] coordinates{(3.5,1.7) (3.5,0)};
        \addplot[penColor2,ultra thick, ->] coordinates{(3.5,-2.5) (0,-2.5)};
      \end{axis}
    \end{tikzpicture}
  \end{image}
  Do you expect 
  \[
  \int_C \vec{F}\dotp \d \vec{p} 
  \]
  to be positive, zero, or negative?
  \begin{prompt}
  \begin{multipleChoice}
    \choice{positive}
    \choice{zero}
    \choice[correct]{negative}
  \end{multipleChoice}
  \begin{hint}
    We can think about this better if we break the path into pieces:
    $C_1$, $C_2$, $C_3$.
    \begin{image}
      \begin{tikzpicture}
        \begin{axis}%
          [hide axis,
	    ymin=-3,ymax=2.5,
	    xmin=-6,xmax=5.5,
	  ]
          \addplot[penColor,thick, ->] coordinates{(-4,0) (-4,2.5)};
          \addplot[penColor,thick, ->] coordinates{(-3,0) (-3,2.5)};
          \addplot[penColor,thick, ->] coordinates{(-2,0) (-2,2.5)};
          \addplot[penColor,thick, ->] coordinates{(-1,0) (-1,2.5)};
          \addplot[penColor,thick, ->] coordinates{(0,0) (0,2.5)};
          \addplot[penColor,thick, ->] coordinates{(1,0) (1,2.5)};
          \addplot[penColor,thick, ->] coordinates{(2,0) (2,2.5)};
          \addplot[penColor,thick, ->] coordinates{(3,0) (3,2.5)};
          \addplot[penColor,thick, ->] coordinates{(4,0) (4,2.5)};
          
          \addplot[penColor,thick, ->] coordinates{(-4,-2) (-4,-.25)};
          \addplot[penColor,thick, ->] coordinates{(-3,-2) (-3,-.25)};
          \addplot[penColor,thick, ->] coordinates{(-2,-2) (-2,-.25)};
          \addplot[penColor,thick, ->] coordinates{(-1,-2) (-1,-.25)};
          \addplot[penColor,thick, ->] coordinates{(0,-2) (0,-.25)};
          \addplot[penColor,thick, ->] coordinates{(1,-2) (1,-.25)};
          \addplot[penColor,thick, ->] coordinates{(2,-2) (2,-.25)};
          \addplot[penColor,thick, ->] coordinates{(3,-2) (3,-.25)};
          \addplot[penColor,thick, ->] coordinates{(4,-2) (4,-.25)};
          
          \addplot[penColor,thick, ->] coordinates{(-4,-3) (-4,-2.2)};
          \addplot[penColor,thick, ->] coordinates{(-3,-3) (-3,-2.2)};
          \addplot[penColor,thick, ->] coordinates{(-2,-3) (-2,-2.2)};
          \addplot[penColor,thick, ->] coordinates{(-1,-3) (-1,-2.2)};
          \addplot[penColor,thick, ->] coordinates{(0,-3) (0,-2.2)};
          \addplot[penColor,thick, ->] coordinates{(1,-3) (1,-2.2)};
          \addplot[penColor,thick, ->] coordinates{(2,-3) (2,-2.2)};
          \addplot[penColor,thick, ->] coordinates{(3,-3) (3,-2.2)};
          \addplot[penColor,thick, ->] coordinates{(4,-3) (4,-2.2)};
          
          \addplot[penColor2,ultra thick] coordinates{
            (-3,1.7) (3.5,1.7)
          };

          \addplot[penColor4,ultra thick] coordinates{
            (3.5,1.7)
            (3.5,-2.5) 
          };

          \addplot[penColor5,ultra thick] coordinates{
            (3.5,-2.5) (-3,-2.5)
          };
         
          \addplot[penColor2,ultra thick, ->] coordinates{(-3,1.7) (0,1.7)};
          \addplot[penColor4,ultra thick, ->] coordinates{(3.5,1.7) (3.5,0)};
          \addplot[penColor5,ultra thick, ->] coordinates{(3.5,-2.5) (0,-2.5)};

          \node[above left,penColor2] at (axis cs: 0,1.7) {$C_1$};
          \node[above right,penColor4] at (axis cs: 2,0) {$C_2$};
          \node[above left,penColor5] at (axis cs: 0,-2.3) {$C_3$};
        \end{axis}
      \end{tikzpicture}
    \end{image}
    The field vectors are orthogonal to the direction of the path
    $C_1$. So this part contributes nothing to the integral.

    The field vectors are flowing against the direction of $C_2$. This
    contributes a negative value to our integral.

    The field vectors are again orthogonal to the direction of the
    path $C_3$. So this part contributes nothing to the integral.
  \end{hint}
  \end{prompt}
\end{question}




\begin{question}
  Consider the following vector field along with a (directed) curve
  $C$.
  \begin{image}
    \begin{tikzpicture}
      \begin{axis}%
        [hide axis,
	  ymin=-3,ymax=2.5,
	  xmin=-6,xmax=5.5,
	]
        \addplot[penColor,thick, ->] coordinates{(-6,2) (-.5,2)};
        \addplot[penColor,thick, ->] coordinates{(0,2) (5.5,2)};

        \addplot[penColor,thick, ->] coordinates{(-6,1) (-2.5,1)};
        \addplot[penColor,thick, ->] coordinates{(-2,1) (1.5,1)};
        \addplot[penColor,thick, ->] coordinates{(2,1) (5.5,1)};

        \addplot[penColor,thick, ->] coordinates{(-6,0) (-3.5,0)};
        \addplot[penColor,thick, ->] coordinates{(-3,0) (-.5,0)};
        \addplot[penColor,thick, ->] coordinates{(0,0) (2.5,0)};
        \addplot[penColor,thick, ->] coordinates{(3,0) (5.5,0)};

        \addplot[penColor,thick, ->] coordinates{(-6,-1) (-4.5,-1)};
        \addplot[penColor,thick, ->] coordinates{(-4,-1) (-2.5,-1)};
        \addplot[penColor,thick, ->] coordinates{(-2,-1) (-.5,-1)};
        \addplot[penColor,thick, ->] coordinates{(0,-1) (1.5,-1)};
        \addplot[penColor,thick, ->] coordinates{(2,-1) (3.5,-1)};
        \addplot[penColor,thick, ->] coordinates{(4,-1) (5.5,-1)};
        
        \addplot[penColor,thick, ->] coordinates{(-6,-2) (-5.5,-2)};
        \addplot[penColor,thick, ->] coordinates{(-5,-2) (-4.5,-2)};
        \addplot[penColor,thick, ->] coordinates{(-4,-2) (-3.5,-2)};
        \addplot[penColor,thick, ->] coordinates{(-3,-2) (-2.5,-2)};
        \addplot[penColor,thick, ->] coordinates{(-2,-2) (-1.5,-2)};
        \addplot[penColor,thick, ->] coordinates{(-1,-2) (-.5,-2)};
        \addplot[penColor,thick, ->] coordinates{(0,-2) (.5,-2)};
        \addplot[penColor,thick, ->] coordinates{(1,-2) (1.5,-2)};
        \addplot[penColor,thick, ->] coordinates{(2,-2) (2.5,-2)};
        \addplot[penColor,thick, ->] coordinates{(3,-2) (3.5,-2)};
        \addplot[penColor,thick, ->] coordinates{(4,-2) (4.5,-2)};
        \addplot[penColor,thick, ->] coordinates{(5,-2) (5.5,-2)};
        
        
        \addplot[penColor2,ultra thick,samples=100] {-sqrt(16-x^2)+2.5};

        \addplot[penColor2,->,ultra thick,domain=-1:0,samples=10] {-sqrt(16-x^2)+2.5};
        
      \end{axis}
    \end{tikzpicture}
  \end{image}
  Do you expect 
  \[
  \int_C \vec{F}\dotp \d \vec{p} 
  \]
  to be positive, zero, or negative?
  \begin{prompt}
  \begin{multipleChoice}
    \choice[correct]{positive}
    \choice{zero}
    \choice{negative}
  \end{multipleChoice}
  \begin{hint}
    Think about what the tangent vectors to the parameterized curve
    look like, and whether they point with the field or against the
    field.
  \end{hint}
  \end{prompt}
\end{question}


\begin{question}
  Consider the following vector field along with a (directed) curve
  $C$.
  \begin{image}
    \begin{tikzpicture}
      \begin{axis}%
        [hide axis,
          width=3in,
          height=2in,
	  ymin=-3,ymax=2.5,
	  xmin=-6,xmax=5.5,
	]
        \addplot[penColor,thick, ->] coordinates{(-4,0) (-4,2.5)};
        \addplot[penColor,thick, ->] coordinates{(-3,0) (-3,2.5)};
        \addplot[penColor,thick, ->] coordinates{(-2,0) (-2,2.5)};
        \addplot[penColor,thick, ->] coordinates{(-1,0) (-1,2.5)};
        \addplot[penColor,thick, ->] coordinates{(0,0) (0,2.5)};
        \addplot[penColor,thick, ->] coordinates{(1,0) (1,2.5)};
        \addplot[penColor,thick, ->] coordinates{(2,0) (2,2.5)};
        \addplot[penColor,thick, ->] coordinates{(3,0) (3,2.5)};
        \addplot[penColor,thick, ->] coordinates{(4,0) (4,2.5)};

        \addplot[penColor,thick, ->] coordinates{(-4,-2) (-4,-.25)};
        \addplot[penColor,thick, ->] coordinates{(-3,-2) (-3,-.25)};
        \addplot[penColor,thick, ->] coordinates{(-2,-2) (-2,-.25)};
        \addplot[penColor,thick, ->] coordinates{(-1,-2) (-1,-.25)};
        \addplot[penColor,thick, ->] coordinates{(0,-2) (0,-.25)};
        \addplot[penColor,thick, ->] coordinates{(1,-2) (1,-.25)};
        \addplot[penColor,thick, ->] coordinates{(2,-2) (2,-.25)};
        \addplot[penColor,thick, ->] coordinates{(3,-2) (3,-.25)};
        \addplot[penColor,thick, ->] coordinates{(4,-2) (4,-.25)};

        \addplot[penColor,thick, ->] coordinates{(-4,-3) (-4,-2.2)};
        \addplot[penColor,thick, ->] coordinates{(-3,-3) (-3,-2.2)};
        \addplot[penColor,thick, ->] coordinates{(-2,-3) (-2,-2.2)};
        \addplot[penColor,thick, ->] coordinates{(-1,-3) (-1,-2.2)};
        \addplot[penColor,thick, ->] coordinates{(0,-3) (0,-2.2)};
        \addplot[penColor,thick, ->] coordinates{(1,-3) (1,-2.2)};
        \addplot[penColor,thick, ->] coordinates{(2,-3) (2,-2.2)};
        \addplot[penColor,thick, ->] coordinates{(3,-3) (3,-2.2)};
        \addplot[penColor,thick, ->] coordinates{(4,-3) (4,-2.2)};

        \addplot[penColor2,ultra thick,samples=100] {-1.5*sqrt(9-x^2)+2};

        \addplot[penColor2,->,ultra thick,domain=-1:0,samples=10] {-1.5*sqrt(9-x^2)+2};
      \end{axis}
    \end{tikzpicture}
  \end{image}
  Do you expect 
  \[
  \int_C \vec{F}\dotp \d \vec{p} 
  \]
  to be positive, zero, or negative?
  \begin{prompt}
  \begin{multipleChoice}
    \choice{positive}
    \choice[correct]{zero}
    \choice{negative}
  \end{multipleChoice}
  \begin{hint}
    Think about what the tangent vectors to the parameterized curve
    look like, and whether they point with the field or against the
    field.
  \end{hint}
  \end{prompt}
\end{question}

Let's attempt to solve a discrete problem.

\begin{example}
  Below we have a very simple directed curve $C$ (it's a line) along
  with field vectors from a vector field $\vec{F}$. 
  \begin{image}
    \begin{tikzpicture}
      \begin{axis}%
        [
	  ymin=-4.5,ymax=4.5,
	  xmin=-6.5,xmax=6.5,
          axis lines =middle, xlabel=$x$, ylabel=$y$,
          every axis y label/.style={at=(current axis.above origin),anchor=south},
          every axis x label/.style={at=(current axis.right of origin),anchor=west},
          grid=both,
          grid style={dashed, gridColor},
          xtick={-8,...,8},
          ytick={-6,...,6},
	]
        \addplot[penColor,thick,->] coordinates{
          (6,2) (6,4) 
        };
        \addplot[penColor,thick,->] coordinates{
          (4,1) (3,3) 
        };
        \addplot[penColor,thick,->] coordinates{
          (2,0) (0,1) 
        };
        \addplot[penColor,thick,->] coordinates{
          (0,-1) (-3,-1) 
        };
        \addplot[penColor,thick,->] coordinates{
          (-2,-2) (-3,-3) 
        };
        \addplot[penColor,thick,->] coordinates{
          (-4,-3) (-4,-4) 
        };


        
        \addplot[penColor2,ultra thick] coordinates{
          (6,2) (-6,-4) 
        };
        \addplot[penColor2,ultra thick,->] coordinates{
          (6,2) (0,-1) 
        };
        
      \end{axis}
    \end{tikzpicture}
  \end{image}
  Setting $\d x= -2$, estimate:
  \[
  \int_C \vec{F}\dotp\d\vec{p}
  \]
  \begin{explanation}
    If $\vec{F}(x,y) = \vector{M(x,y),N(x,y)}$, we have that
    \[
    \int_C \vec{F}\dotp\d\vec{p} = \int_C M(x,y) \d x + N(x,y)\d y.
    \]
    We know that $\d x= \answer[given]{-2}$ and
    \[
    \d y = y'(x) \d x
    \]
    Since $C$ is a line of slope $\answer[given]{1/2}$, 
    \[
    \d y = \answer[given]{-1}.
    \]
    Since we want to estimate
    \[
    \int_C M(x,y) \d x + N(x,y)\d y.
    \]
    We will compute:
    \[
    \sum \left(M(x,y) \d x + N(x,y) \d y\right)
    \]
    Write with me
    \begin{align*}
    \answer[given]{0}\cdot (-2) &+ \answer[given]{2} \cdot (-1) \\
    &+ (-1) \cdot (-2)  + 2 \cdot (-1) \\
    &+ (-2)\cdot (-2) + 1 \cdot (-1) \\
    &+ (-3)\cdot (-2) + 0\cdot (-1) \\
    &+ (-1)\cdot (-2) + (-1)\cdot (-1)\\ 
    &+ (0)\cdot (-2) + (-1)\cdot (-1)
    \end{align*}
    \[
    = \answer[given]{11}.
    \]
  \end{explanation}
\end{example}






\section{Computations with line integrals}




\begin{example}
  Let $\vec{F}(x,y) = \vector{-y,x}$ and let $C$ be the unit circle
  centered at the origin. Compute
  \[
  \oint_C \vec{F}\dotp\d\vec{p}
  \]
  \begin{explanation}
    The path $C$ can be parameterized by
    \begin{align*}
      x(\theta) &= \cos(\theta)\\
      y(\theta) &= \sin(\theta)
    \end{align*}
    with $0\le \theta\le 2\pi$. To compute the integral, write with me
    \begin{align*}
      \oint_C \vec{F}\dotp\d\vec{p} &= \int_0^{2\pi} F(x(\theta),y(\theta))\dotp \vector{x'(\theta),y'(\theta)}\d \theta\\
      &= \int_0^{2\pi} \vector{\answer[given]{-\sin(\theta)},\answer[given]{\cos(\theta)}}\dotp \vector{\answer[given]{-\sin(\theta)},\answer[given]{\cos(\theta)}}\d \theta\\
      &= \int_0^{2\pi}\left(\sin^2(\theta)+\cos^2(\theta)\right)\d \theta\\
      &= \int_0^{2\pi} 1\d \theta\\
      &=\answer[given]{2\pi}.
    \end{align*}
  \end{explanation}
\end{example}

Any smooth path can be approximated with a polygonal path. These can
be quite easy to integrate. Check out our next example.

\begin{example}
  Let $\vec{F}(x,y) = \vector{0,x}$ and let $C$ be the polygonal path
  below parameterized in a counterclockwise direction:
  \begin{image}
    \begin{tikzpicture}
      \begin{axis}%
        [
	  ymin=-.5,ymax=2.5,
	  xmin=-.5,xmax=4.5,
          axis lines =middle, xlabel=$x$, ylabel=$y$,
          every axis y label/.style={at=(current axis.above origin),anchor=south},
          every axis x label/.style={at=(current axis.right of origin),anchor=west},
          grid=both,
          grid style={dashed, gridColor},
         % xtick={-2,...,4},
         % ytick={-3,...,3},
	]
        \addplot[penColor,ultra thick] coordinates{
            (0,0) (1,2) (3,2) (4,0) (0,0) (1,2)
          };

      \end{axis}
          \end{tikzpicture}
  \end{image}
  Compute
  \[
  \oint_C \vec{F}\dotp\d\vec{p}
  \]
  \begin{explanation}
    We need to parameterize our paths in a counterclockwise
    direction. We'll break it into four line segments each parameterized
    as $t$ runs from $0$ to $1$.
    \begin{image}
    \begin{tikzpicture}
      \begin{axis}%
        [
	  ymin=-.5,ymax=2.5,
	  xmin=-.5,xmax=4.5,
          axis lines =middle, xlabel=$x$, ylabel=$y$,
          every axis y label/.style={at=(current axis.above origin),anchor=south},
          every axis x label/.style={at=(current axis.right of origin),anchor=west},
          grid=both,
          grid style={dashed, gridColor},
         % xtick={-2,...,4},
         % ytick={-3,...,3},
	]
        \addplot[penColor,ultra thick] coordinates{
            (0,0) (1,2) 
        };
        \addplot[penColor,->,ultra thick] coordinates{
            (1,2) (.5,1) 
        };
        
        \addplot[penColor2,ultra thick] coordinates{
            (1,2) (3,2) 
        };
        \addplot[penColor2,ultra thick,->] coordinates{
            (3,2) (2,2) 
        };
        
        \addplot[penColor4,ultra thick] coordinates{
            (3,2) (4,0) 
        };
        \addplot[penColor4,ultra thick,->] coordinates{
            (4,0) (3.5,1) 
        };
        
        \addplot[penColor5,ultra thick] coordinates{
            (4,0) (0,0) 
        };
        \addplot[penColor5,ultra thick,->] coordinates{
            (0,0) (2,0) 
        };
        
        \node[above,penColor5] at (axis cs: 2,0) {$\vecl_1$};
        \node[penColor4] at (axis cs: 3.2,1) {$\vecl_2$};
        \node[below,penColor5] at (axis cs: 2,2) {$\vecl_3$};
        \node[penColor] at (axis cs: .8,1) {$\vecl_4$};

      \end{axis}
    \end{tikzpicture}    
    \end{image}
    Where:
    \begin{align*}
      \vecl_1(t) &= \vector{4t,\answer[given]{0}}\\
      \vecl_2(t) &= \vector{\answer[given]{4-t},2t}\\
      \vecl_3(t) &= \vector{3-2t,\answer[given]{2}}\\
      \vecl_4(t) &= \vector{\answer[given]{1-t},2-2t}
    \end{align*}
    and each draws the line as $t$ runs from $0$ to $1$.  Write:
    \begin{align*}
    \oint_C \vec{F}\dotp\d\vec{p} = \int_0^1 &\vec{F}(\vecl_1(t))\dotp \vecl_1'(t) \d t \\
    &+ \int_0^1 \vec{F}(\vecl_2(t))\dotp \vecl_2'(t) \d t\\
    &+ \int_0^1 \vec{F}(\vecl_3(t))\dotp \vecl_3'(t) \d t\\
    &+ \int_0^1 \vec{F}(\vecl_4(t))\dotp \vecl_4'(t) \d t
    \end{align*}
    For each of the integrands above, say $\vecl(t) =
    \vector{x(t),y(t)}$, we will write
    \[
    \vec{F}(\vecl(t))\dotp \vecl'(t) = \vector{0,x(t)} \dotp \vector{x'(t),y'(t)}
    \]
    and combine them into a single integral. Write with me
    \begin{align*}
      \oint_C \vec{F}\dotp\d\vec{p} &= \int_0^1 \left(\answer[given]{2(4-t) -2(1-t)} \right)\d t\\
      &= \eval{\answer[given]{6t}}_0^1\\
      &=\answer[given]{6}. 
    \end{align*}
  \end{explanation}
\end{example}



\section{The fundamental theorems of calculus}


We will soon see that there are many ``Fundamental Theorems of
calculus.'' What makes them similar is that they all share the
following rather vague description:
\begin{quote}
  To compute a certain sort of integral over a region, we may do a
  computation on the boundary of the region that involves one fewer
  integrations.
\end{quote}

Each version of the Fundamental Theorem of calculus makes the
``vague'' statement above precise. For example, when working with a
single variable, the Fundamental Theorem concludes:
\[
\int_a^b f'(x) \d x = f(b) - f(a)
\]
In this case we are doing an integral over the ``region'' $[a,b]$, and
the ``computation'' that allows ``one fewer integrations'' is
antidifferentiation. This is the only Fundamental Theorem you have
known so far in your studies.  However with additional dimensions,
there come \textit{additional derivatives}. When working with function
$F:\R^n\to\R$ we have the gradient as a ``derivative.'' This brings us
to our first of several new fundamental theorems.

\begin{theorem}[Fundamental Theorem for Line Integrals]
  If $C$ is a curve that starts at $\vec{a}$ and ends at $\vec{b}$, then:
  \[
  \int_C \grad F\dotp \d \vec{p} = F(\vec{b}) - F(\vec{a})
  \]
\end{theorem}

Like the first fundamental theorem we met in our very first calculus
class, the fundamental theorem for line integrals says that if we can
find a potential function for a gradient field, we can evaluate a line
integral over this gradient field by evaluating at the end-points.
The up-shot is that whenever you are dealing with a line integral, you
should always start by checking to see if you are working with a
gradient field.


\begin{question}
  Let $F(x,y) = x\cos(xy)$. Compute: $\grad F$
  \begin{prompt}
  \[
  \grad F(x,y) = \vector{\answer{\cos(xy) -xy\sin(xy)},\answer{ -x^2\sin(xy)}}
  \]
  \end{prompt}
  \begin{question}
    Now let:
    \[
    \vec{G}(x,y) = \vector{\cos(xy) -xy\sin(xy), -x^2\sin(xy)}
    \]
    Compute
    \[
    \int_C \vec G \dotp \d \vec{p}
    \]
    where $C$ is shown below:
    \begin{image}
      \begin{tikzpicture}
      \begin{axis}%
        [
	  ymin=-.5,ymax=2.5,
	  xmin=-.5,xmax=4.5,
          axis lines =middle, xlabel=$x$, ylabel=$y$,
          every axis y label/.style={at=(current axis.above origin),anchor=south},
          every axis x label/.style={at=(current axis.right of origin),anchor=west},
          grid=both,
          grid style={dashed, gridColor},
          % xtick={-2,...,4},
          % ytick={-3,...,3},
	]
        \addplot[penColor,ultra thick,smooth] coordinates{
          (0,2) (.5,2) (2,.7) (3,1.4) (4,0)
        };

        \addplot[penColor,ultra thick,->] coordinates{
          (1.05,1.485) (1.55,1) 
        };
        
        \fill[black,draw=black] (axis cs:0,2) circle (2.5pt);
        \fill[black,draw=black] (axis cs:4,0) circle (2.5pt);
      \end{axis}
    \end{tikzpicture}
      \end{image}
      \begin{prompt}
        \begin{align*}
          \int_C \vec{G}\dotp \d \vec{p} &= F(4,0) - F(0,2)\\
          &= \answer{4}.
        \end{align*}
      \end{prompt}
  \end{question}
\end{question}


\subsection{Conservative fields}

When dealing with gradient fields and closed curve something very nice
happens.


\begin{question}
  Suppose that $C$ is a closed curve, one that starts and stops at the
  same location. Compute:
  \[
  \oint_C \grad F\dotp \d \vec{p}
  \begin{prompt}
    = \answer{0}
  \end{prompt}
  \]
\end{question}

This leads us to our definition:
\begin{definition}
  A \dfn{conservative field} is just another name for a gradient
  field.
\end{definition}

If something is special enough to be named twice, we ought to do some more
examples.


\begin{example}
  Let $\vec{F}(x,y) = \vector{-y,-x}$. Let $C$ be the polygonal path
  below parameterized in a counterclockwise direction:
  \begin{image}
    \begin{tikzpicture}
      \begin{axis}%
        [
	  ymin=-.5,ymax=2.5,
	  xmin=-.5,xmax=4.5,
          axis lines =middle, xlabel=$x$, ylabel=$y$,
          every axis y label/.style={at=(current axis.above origin),anchor=south},
          every axis x label/.style={at=(current axis.right of origin),anchor=west},
          grid=both,
          grid style={dashed, gridColor},
          % xtick={-2,...,4},
          % ytick={-3,...,3},
	]
        \addplot[penColor,ultra thick] coordinates{
          (0,0) (1,2) (3,2) (4,0) (0,0) (1,2)
        };
        
      \end{axis}
    \end{tikzpicture}
  \end{image}
  Compute
  \[
  \oint_C \vec{F}\dotp\d\vec{p}
  \]
  \begin{explanation}
    In light of the Fundamental Theorem for line integrals, we should
    check to see if our field is a gradient field using the Clairaut
    gradient test:
    \[
    \pp{x}(-x)-\pp{y}(-y) = \answer[given]{0}.
    \]
    Since our field is a gradient field, and our curve is closed, we
    see that
    \[
    \oint_C \vec{F}\dotp \d\vec{p} = \answer[given]{0}.
    \]
  \end{explanation}
\end{example}


\begin{example}
  Below we see a directed curve with some field vectors
  attached.
  \begin{image}
    \begin{tikzpicture}
      \begin{axis}%
        [
	  ymin=-6,ymax=6,
	  xmin=-10,xmax=4,
          axis lines =middle, xlabel=$x$, ylabel=$y$,
          every axis y label/.style={at=(current axis.above origin),anchor=south},
          every axis x label/.style={at=(current axis.right of origin),anchor=west},
          grid=both,
          grid style={dashed, gridColor},
          xtick={-10,...,4},
          ytick={-6,...,6},
	]

        \addplot[penColor,thick,->] coordinates{
          (3,3) (2,4) 
        };
        \addplot[penColor,thick,->] coordinates{
          (2,3) (1,4) 
        };
        \addplot[penColor,thick,->] coordinates{
          (1,3) (0,4) 
        };
        \addplot[penColor,thick,->] coordinates{
          (0,3) (-1,4) 
        };
        \addplot[penColor,thick,->] coordinates{
          (-1,3) (-2,4) 
        };
        \addplot[penColor,thick,->] coordinates{
          (-2,3) (-3,4) 
        };
        \addplot[penColor,thick,->] coordinates{
          (-3,3) (-4,4) 
        };
        \addplot[penColor,thick,->] coordinates{
          (-4,3) (-5,4) 
        };

        \addplot[penColor,thick,->] coordinates{
          (-5.5,2) (-4.5,1) 
        };
        \addplot[penColor,thick,->] coordinates{
          (-6,1) (-5,0) 
        };
        \addplot[penColor,thick,->] coordinates{
          (-6.5,0) (-5.5,-1) 
        };
        \addplot[penColor,thick,->] coordinates{
          (-7,-1) (-6,-2) 
        };
        \addplot[penColor,thick,->] coordinates{
          (-7.5,-2) (-6.5,-3) 
        };
        \addplot[penColor,thick,->] coordinates{
          (-8,-3) (-7,-4) 
        };
        \addplot[penColor,thick,->] coordinates{
          (-8.5,-4) (-7.5,-5) 
        };
        \addplot[penColor,thick,->] coordinates{
          (-9,-5) (-8,-6) 
        };

        

        \addplot[penColor2,ultra thick] coordinates{
          (3,3) (-5,3) (-9,-5)
        };

        \addplot[penColor2,ultra thick,->] coordinates{
          (3,3) (-1,3) 
        };

        \addplot[penColor2,ultra thick,->] coordinates{
          (-5,3) (-7,-1) 
        };
      \end{axis}
    \end{tikzpicture}
  \end{image}
  Assuming that the field vectors are constant along each ``side'' of
  our polygonal path, compute:
  \[
  \int_C\vec{F}\dotp\d\vec{p}
  \]
  \begin{explanation}
    This problem would be easier if our field was a gradient field,
    and while it surely isn't, the field is constant along two paths
    that together make up all of $C$:
      \begin{image}
    \begin{tikzpicture}
      \begin{axis}%
        [
	  ymin=-6,ymax=6,
	  xmin=-10,xmax=4,
          axis lines =middle, xlabel=$x$, ylabel=$y$,
          every axis y label/.style={at=(current axis.above origin),anchor=south},
          every axis x label/.style={at=(current axis.right of origin),anchor=west},
          grid=both,
          grid style={dashed, gridColor},
          xtick={-10,...,4},
          ytick={-6,...,6},
	]

        \addplot[penColor,thick,->] coordinates{
          (3,3) (2,4) 
        };
        \addplot[penColor,thick,->] coordinates{
          (2,3) (1,4) 
        };
        \addplot[penColor,thick,->] coordinates{
          (1,3) (0,4) 
        };
        \addplot[penColor,thick,->] coordinates{
          (0,3) (-1,4) 
        };
        \addplot[penColor,thick,->] coordinates{
          (-1,3) (-2,4) 
        };
        \addplot[penColor,thick,->] coordinates{
          (-2,3) (-3,4) 
        };
        \addplot[penColor,thick,->] coordinates{
          (-3,3) (-4,4) 
        };
        \addplot[penColor,thick,->] coordinates{
          (-4,3) (-5,4) 
        };

        \addplot[penColor,thick,->] coordinates{
          (-5.5,2) (-4.5,1) 
        };
        \addplot[penColor,thick,->] coordinates{
          (-6,1) (-5,0) 
        };
        \addplot[penColor,thick,->] coordinates{
          (-6.5,0) (-5.5,-1) 
        };
        \addplot[penColor,thick,->] coordinates{
          (-7,-1) (-6,-2) 
        };
        \addplot[penColor,thick,->] coordinates{
          (-7.5,-2) (-6.5,-3) 
        };
        \addplot[penColor,thick,->] coordinates{
          (-8,-3) (-7,-4) 
        };
        \addplot[penColor,thick,->] coordinates{
          (-8.5,-4) (-7.5,-5) 
        };
        \addplot[penColor,thick,->] coordinates{
          (-9,-5) (-8,-6) 
        };

        

        \addplot[penColor4,ultra thick] coordinates{
          (3,3) (-5,3) 
        };


        \addplot[penColor5,ultra thick] coordinates{
          (-5,3) (-9,-5)
        };

        \addplot[penColor4,ultra thick,->] coordinates{
          (3,3) (-1,3) 
        };

        \addplot[penColor5,ultra thick,->] coordinates{
          (-5,3) (-7,-1) 
        };

        \node[below,penColor4] at (axis cs: -2,3) {$C_1$};

        \node[penColor5] at (axis cs: -7,1) {$C_2$};
      \end{axis}
    \end{tikzpicture}
      \end{image}
      So now we see that
      \[
      \int_C \vec{F}\dotp\d\vec{p} = \int_{C_1} \vec{F}\dotp\d\vec{p}+\int_{C_2} \vec{F}\dotp\d\vec{p}.
      \]
      so we have that
      \[
      \vec{F}(x,y) =
      \begin{cases}
        \vector{-1,1} &\text{on $C_1$,}\\
        \vector{1,-1} &\text{on $C_2$.}
      \end{cases}
      \]
      Using the Clairaut gradient test we see that $\vec{F}$
      \wordChoice{\choice[correct]{is}\choice{is not}} a gradient
      field when restricted to either part of $C$. Let $F$ be a
      potential function for $\vec{F}$. One candidate for the the
      potential function is
      \[
      F(x,y) =
      \begin{cases}
        -x+y  &\text{on $C_1$,}\\
        x-y  &\text{on $C_2$.}
      \end{cases}
      \]
      By the Fundamental Theorem for Line Integrals,
      \begin{align*}
        \int_{C_1} \vec{F}\dotp\d\vec{p}
        &= \eval{-x+y}_{\vector{3,3}}^{\vector{-5,3}}\\
        &= \answer[given]{8},
      \end{align*}
      and
      \begin{align*}
        \int_{C_2} \vec{F}\dotp\d\vec{p}
        &= \eval{x-y}_{\vector{-5,3}}^{\vector{-9,-5}}\\
        &= \answer[given]{4}.
      \end{align*}
      So
      \[
      \int_{C} \vec{F}\dotp\d\vec{p} = \answer[given]{12}.
      \]
  \end{explanation}
\end{example}


\end{document}
