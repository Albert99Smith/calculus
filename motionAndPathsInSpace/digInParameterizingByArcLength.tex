\documentclass{ximera}

%\usepackage{todonotes}
%\usepackage{mathtools} %% Required for wide table Curl and Greens
%\usepackage{cuted} %% Required for wide table Curl and Greens
\newcommand{\todo}{}

\usepackage{esint} % for \oiint
\ifxake%%https://math.meta.stackexchange.com/questions/9973/how-do-you-render-a-closed-surface-double-integral
\renewcommand{\oiint}{{\large\bigcirc}\kern-1.56em\iint}
\fi


\graphicspath{
  {./}
  {ximeraTutorial/}
  {basicPhilosophy/}
  {functionsOfSeveralVariables/}
  {normalVectors/}
  {lagrangeMultipliers/}
  {vectorFields/}
  {greensTheorem/}
  {shapeOfThingsToCome/}
  {dotProducts/}
  {partialDerivativesAndTheGradientVector/}
  {../productAndQuotientRules/exercises/}
  {../normalVectors/exercisesParametricPlots/}
  {../continuityOfFunctionsOfSeveralVariables/exercises/}
  {../partialDerivativesAndTheGradientVector/exercises/}
  {../directionalDerivativeAndChainRule/exercises/}
  {../commonCoordinates/exercisesCylindricalCoordinates/}
  {../commonCoordinates/exercisesSphericalCoordinates/}
  {../greensTheorem/exercisesCurlAndLineIntegrals/}
  {../greensTheorem/exercisesDivergenceAndLineIntegrals/}
  {../shapeOfThingsToCome/exercisesDivergenceTheorem/}
  {../greensTheorem/}
  {../shapeOfThingsToCome/}
  {../separableDifferentialEquations/exercises/}
  {vectorFields/}
}

\newcommand{\mooculus}{\textsf{\textbf{MOOC}\textnormal{\textsf{ULUS}}}}

\usepackage{tkz-euclide}\usepackage{tikz}
\usepackage{tikz-cd}
\usetikzlibrary{arrows}
\tikzset{>=stealth,commutative diagrams/.cd,
  arrow style=tikz,diagrams={>=stealth}} %% cool arrow head
\tikzset{shorten <>/.style={ shorten >=#1, shorten <=#1 } } %% allows shorter vectors

\usetikzlibrary{backgrounds} %% for boxes around graphs
\usetikzlibrary{shapes,positioning}  %% Clouds and stars
\usetikzlibrary{matrix} %% for matrix
\usepgfplotslibrary{polar} %% for polar plots
\usepgfplotslibrary{fillbetween} %% to shade area between curves in TikZ
\usetkzobj{all}
\usepackage[makeroom]{cancel} %% for strike outs
%\usepackage{mathtools} %% for pretty underbrace % Breaks Ximera
%\usepackage{multicol}
\usepackage{pgffor} %% required for integral for loops



%% http://tex.stackexchange.com/questions/66490/drawing-a-tikz-arc-specifying-the-center
%% Draws beach ball
\tikzset{pics/carc/.style args={#1:#2:#3}{code={\draw[pic actions] (#1:#3) arc(#1:#2:#3);}}}



\usepackage{array}
\setlength{\extrarowheight}{+.1cm}
\newdimen\digitwidth
\settowidth\digitwidth{9}
\def\divrule#1#2{
\noalign{\moveright#1\digitwidth
\vbox{\hrule width#2\digitwidth}}}





\newcommand{\RR}{\mathbb R}
\newcommand{\R}{\mathbb R}
\newcommand{\N}{\mathbb N}
\newcommand{\Z}{\mathbb Z}

\newcommand{\sagemath}{\textsf{SageMath}}


%\renewcommand{\d}{\,d\!}
\renewcommand{\d}{\mathop{}\!d}
\newcommand{\dd}[2][]{\frac{\d #1}{\d #2}}
\newcommand{\pp}[2][]{\frac{\partial #1}{\partial #2}}
\renewcommand{\l}{\ell}
\newcommand{\ddx}{\frac{d}{\d x}}

\newcommand{\zeroOverZero}{\ensuremath{\boldsymbol{\tfrac{0}{0}}}}
\newcommand{\inftyOverInfty}{\ensuremath{\boldsymbol{\tfrac{\infty}{\infty}}}}
\newcommand{\zeroOverInfty}{\ensuremath{\boldsymbol{\tfrac{0}{\infty}}}}
\newcommand{\zeroTimesInfty}{\ensuremath{\small\boldsymbol{0\cdot \infty}}}
\newcommand{\inftyMinusInfty}{\ensuremath{\small\boldsymbol{\infty - \infty}}}
\newcommand{\oneToInfty}{\ensuremath{\boldsymbol{1^\infty}}}
\newcommand{\zeroToZero}{\ensuremath{\boldsymbol{0^0}}}
\newcommand{\inftyToZero}{\ensuremath{\boldsymbol{\infty^0}}}



\newcommand{\numOverZero}{\ensuremath{\boldsymbol{\tfrac{\#}{0}}}}
\newcommand{\dfn}{\textbf}
%\newcommand{\unit}{\,\mathrm}
\newcommand{\unit}{\mathop{}\!\mathrm}
\newcommand{\eval}[1]{\bigg[ #1 \bigg]}
\newcommand{\seq}[1]{\left( #1 \right)}
\renewcommand{\epsilon}{\varepsilon}
\renewcommand{\phi}{\varphi}


\renewcommand{\iff}{\Leftrightarrow}

\DeclareMathOperator{\arccot}{arccot}
\DeclareMathOperator{\arcsec}{arcsec}
\DeclareMathOperator{\arccsc}{arccsc}
\DeclareMathOperator{\si}{Si}
\DeclareMathOperator{\scal}{scal}
\DeclareMathOperator{\sign}{sign}


%% \newcommand{\tightoverset}[2]{% for arrow vec
%%   \mathop{#2}\limits^{\vbox to -.5ex{\kern-0.75ex\hbox{$#1$}\vss}}}
\newcommand{\arrowvec}[1]{{\overset{\rightharpoonup}{#1}}}
%\renewcommand{\vec}[1]{\arrowvec{\mathbf{#1}}}
\renewcommand{\vec}[1]{{\overset{\boldsymbol{\rightharpoonup}}{\mathbf{#1}}}\hspace{0in}}

\newcommand{\point}[1]{\left(#1\right)} %this allows \vector{ to be changed to \vector{ with a quick find and replace
\newcommand{\pt}[1]{\mathbf{#1}} %this allows \vec{ to be changed to \vec{ with a quick find and replace
\newcommand{\Lim}[2]{\lim_{\point{#1} \to \point{#2}}} %Bart, I changed this to point since I want to use it.  It runs through both of the exercise and exerciseE files in limits section, which is why it was in each document to start with.

\DeclareMathOperator{\proj}{\mathbf{proj}}
\newcommand{\veci}{{\boldsymbol{\hat{\imath}}}}
\newcommand{\vecj}{{\boldsymbol{\hat{\jmath}}}}
\newcommand{\veck}{{\boldsymbol{\hat{k}}}}
\newcommand{\vecl}{\vec{\boldsymbol{\l}}}
\newcommand{\uvec}[1]{\mathbf{\hat{#1}}}
\newcommand{\utan}{\mathbf{\hat{t}}}
\newcommand{\unormal}{\mathbf{\hat{n}}}
\newcommand{\ubinormal}{\mathbf{\hat{b}}}

\newcommand{\dotp}{\bullet}
\newcommand{\cross}{\boldsymbol\times}
\newcommand{\grad}{\boldsymbol\nabla}
\newcommand{\divergence}{\grad\dotp}
\newcommand{\curl}{\grad\cross}
%\DeclareMathOperator{\divergence}{divergence}
%\DeclareMathOperator{\curl}[1]{\grad\cross #1}
\newcommand{\lto}{\mathop{\longrightarrow\,}\limits}

\renewcommand{\bar}{\overline}

\colorlet{textColor}{black}
\colorlet{background}{white}
\colorlet{penColor}{blue!50!black} % Color of a curve in a plot
\colorlet{penColor2}{red!50!black}% Color of a curve in a plot
\colorlet{penColor3}{red!50!blue} % Color of a curve in a plot
\colorlet{penColor4}{green!50!black} % Color of a curve in a plot
\colorlet{penColor5}{orange!80!black} % Color of a curve in a plot
\colorlet{penColor6}{yellow!70!black} % Color of a curve in a plot
\colorlet{fill1}{penColor!20} % Color of fill in a plot
\colorlet{fill2}{penColor2!20} % Color of fill in a plot
\colorlet{fillp}{fill1} % Color of positive area
\colorlet{filln}{penColor2!20} % Color of negative area
\colorlet{fill3}{penColor3!20} % Fill
\colorlet{fill4}{penColor4!20} % Fill
\colorlet{fill5}{penColor5!20} % Fill
\colorlet{gridColor}{gray!50} % Color of grid in a plot

\newcommand{\surfaceColor}{violet}
\newcommand{\surfaceColorTwo}{redyellow}
\newcommand{\sliceColor}{greenyellow}




\pgfmathdeclarefunction{gauss}{2}{% gives gaussian
  \pgfmathparse{1/(#2*sqrt(2*pi))*exp(-((x-#1)^2)/(2*#2^2))}%
}


%%%%%%%%%%%%%
%% Vectors
%%%%%%%%%%%%%

%% Simple horiz vectors
\renewcommand{\vector}[1]{\left\langle #1\right\rangle}


%% %% Complex Horiz Vectors with angle brackets
%% \makeatletter
%% \renewcommand{\vector}[2][ , ]{\left\langle%
%%   \def\nextitem{\def\nextitem{#1}}%
%%   \@for \el:=#2\do{\nextitem\el}\right\rangle%
%% }
%% \makeatother

%% %% Vertical Vectors
%% \def\vector#1{\begin{bmatrix}\vecListA#1,,\end{bmatrix}}
%% \def\vecListA#1,{\if,#1,\else #1\cr \expandafter \vecListA \fi}

%%%%%%%%%%%%%
%% End of vectors
%%%%%%%%%%%%%

%\newcommand{\fullwidth}{}
%\newcommand{\normalwidth}{}



%% makes a snazzy t-chart for evaluating functions
%\newenvironment{tchart}{\rowcolors{2}{}{background!90!textColor}\array}{\endarray}

%%This is to help with formatting on future title pages.
\newenvironment{sectionOutcomes}{}{}



%% Flowchart stuff
%\tikzstyle{startstop} = [rectangle, rounded corners, minimum width=3cm, minimum height=1cm,text centered, draw=black]
%\tikzstyle{question} = [rectangle, minimum width=3cm, minimum height=1cm, text centered, draw=black]
%\tikzstyle{decision} = [trapezium, trapezium left angle=70, trapezium right angle=110, minimum width=3cm, minimum height=1cm, text centered, draw=black]
%\tikzstyle{question} = [rectangle, rounded corners, minimum width=3cm, minimum height=1cm,text centered, draw=black]
%\tikzstyle{process} = [rectangle, minimum width=3cm, minimum height=1cm, text centered, draw=black]
%\tikzstyle{decision} = [trapezium, trapezium left angle=70, trapezium right angle=110, minimum width=3cm, minimum height=1cm, text centered, draw=black]


\title[Dig-In:]{Parameterizing by arc length}

\outcome{Reparametrize a curve.}
\outcome{Parameterize a curve in terms of arc length.}

\author{Bart Snapp}

\begin{document}
\begin{abstract}
  We find a new description of curves that trivializes arc length
  computations.
\end{abstract}
\maketitle

For any given a curve in space, there are many different vector-valued
functions that draw this curve. For example, consider a circle of
radius $3$ centered at the origin. Each of the following vector-valued
functions will draw this circle:
\begin{align*}
  \vec{f}(\theta) &=\vector{3\cos(\theta),3\sin(\theta)} &  &\text{for $0\le \theta<  2\pi$}  \\
  \vec{g}(t) &= \vector{3\cos(2\pi t),3\sin(2\pi t)} & &\text{for $0\le t< 1$}  \\
  \vec{h}(s) &= \vector{3\cos(s/3),3\sin(s/3)} & &\text{for $0\le s<  6\pi$}
\end{align*}
Each of these functions is a different \textit{parameterization} of
the circle. This means that while these vector-valued functions draw
the same circle, they do so at different rates.

\begin{question}
  Considering $\vec{f}$, $\vec{g}$, and $\vec{h}$, which draws the
  circle of radius $3$ quickest?
  \begin{prompt}
    \begin{multipleChoice}
      \choice{$\vec{f}(\theta) =\vector{3\cos(\theta),3\sin(\theta)}$}
      \choice[correct]{$\vec{g}(t) = \vector{3\cos(2\pi t),3\sin(2\pi t)}$}
      \choice{$\vec{h}(s) = \vector{3\cos(s/3),3\sin(s/3)}$}
    \end{multipleChoice}
  \end{prompt}
  \begin{question}
    Which draws the circle of radius $3$ slowest?
    \begin{prompt}
      \begin{multipleChoice}
        \choice{$\vec{f}(\theta) =\vector{3\cos(\theta),3\sin(\theta)}$}
        \choice{$\vec{g}(t) = \vector{3\cos(2\pi t),3\sin(2\pi t)}$}
        \choice[correct]{$\vec{h}(s) = \vector{3\cos(s/3),3\sin(s/3)}$}
      \end{multipleChoice}
    \end{prompt}
  \end{question}
\end{question}


In this section, we are going to be interested in parameterizations of
curves where there is a one-to-one ratio between the parameter (the
variable) and distance drawn (the arc length) from the start of the
curve.  Recall that if $\vec{f}$ is a continuous vector-valued
function where the curve drawn by $\vec{f}(t)$ is traversed once for
$a\le t\le b$, then the arc length of the curve from $\vec{f}(a)$ to
$\vec{f}(b)$ is given by
\[
\text{arc length} = \int_a^b |\vec{f}'(t)|\d t.
\]
This is all good and well, but the integral
\[
\int_a^b |\vec{f}'(t)|\d t
\]
could be quite difficult to compute. On the other hand, if $\vec{f}$
were an arc length parameterization, this would be \textbf{simple} to
compute, because then the arc length is in a one-to-one ratio with the
variables. Hence
\[
\int_a^b |\vec{f}'(t)|\d t = b-a.
\]
Let's state this as a definition.

\begin{definition}
  A vector-valued function $\vec{g}(s)$ is \dfn{parameterized by arc
    length} if
  \[
  s = \int_0^s |\vec{g}'(t)|\d t.
  \]
  Such a parameterization is called an \dfn{arc length
    parameterization}.
\end{definition}
It is nice to work with functions parameterized by arc length, because
computing the arc length is easy. If $g$ is parameterized by arc
length, then the length of $g(s)$ when $a\le s\le b$, is simply
$b-a$. No integral computations need to be done. Also we should point
out that $s$ is typically (though \textit{not necessarily}) the name
of the variable when a function is parameterized by arc length, as $s$
often represents ``distance.''

\begin{question}
  Suppose the curve below has an arc length parameterization given
  by $\vec{p}(s)$.
  \begin{image}
    \begin{tikzpicture}
      \begin{axis}%
        [
	  xmin=-3,xmax=5,
          ymin=-2,ymax=3,
          xlabel=$x$,ylabel=$y$,
          axis lines=center,
          every axis y label/.style={at=(current axis.above origin),anchor=south},
          every axis x label/.style={at=(current axis.right of origin),anchor=west},
          clip=false,
	  grid =major,
          xtick={-3,-2,...,5},
          ytick={-2,-1,...,3},
	]
        \addplot[line join =bevel,penColor,ultra thick] coordinates{
          (-2,2) (2,-1) (2,2) (4,2)
        };
        \addplot[color=penColor,fill=penColor,only marks,mark=*] coordinates{(-2,2)};  %% closed hole
        \addplot[color=penColor,fill=penColor,only marks,mark=*] coordinates{(4,2)};  %% closed hole
        \node[penColor,below] at (axis cs: 4,2) {$\vec{p}(0)$};
        \node[penColor,above] at (axis cs: -2,2) {$\vec{p}(10)$};
      \end{axis}
    \end{tikzpicture}
  \end{image}
  Compute: $\vec{p}(2)$, $\vec{p}(4)$, and $\vec{p}(7.5)$
  \begin{prompt}
    \begin{align*}
      \vec{p}(2) &= \vector{\answer{2},\answer{2}}\\
      \vec{p}(4) &= \vector{\answer{2},\answer{0}}\\
      \vec{p}(7.5) &= \vector{\answer{0},\answer{1/2}}
    \end{align*}
  \end{prompt}
\end{question}


Consider the following example:

\begin{example}
  Let $\vec{f}(t) = \vector{\cos(t), \sin(t)}$ for $0\le t<
  2\pi$. Show that $\vec{f}$ is parameterized by arc length.
  \begin{explanation}
    Here we need to show that
    \[
    s = \int_0^s |\vec{f}'(t)| \d t.
    \]
    We'll just compute the right-hand side of the equation above and
    see what happens. Write with me,
    \[
    \vec{f}'(t) = \vector{\answer[given]{-\sin(t)},\answer[given]{\cos(t)}}
    \]
    and so
    \begin{align*}
      |\vec{f}'(t)| &= \sqrt{\vec{f}'(t)\dotp \vec{f}'(t)}\\
      &= \sqrt{\vector{\answer[given]{-\sin(t)},\answer[given]{\cos(t)}}\dotp\vector{\answer[given]{-\sin(t)},\answer[given]{\cos(t)}}}\\
      &= \answer[given]{1}.
    \end{align*}
    Now our integral becomes:
    \begin{align*}
      \int_0^s  |\vec{f}'(t)| \d t &= \int_0^s \d t\\
      &= \answer[given]{s}.
    \end{align*}
    Hence $\vec{f}$ is parameterized by arc length.
  \end{explanation}
\end{example}

From your own experience and the work above, we think the next
theorem should be quite sensible.

\begin{theorem}
  A vector-valued function $\vec{f}:\R\to \R^2$ is parameterized by
  arc length if and only if $|\vec{f}'| = 1$.
\end{theorem}

If we imagine our vector-valued function as giving the position of a
particle, then this theorem says that the path is parameterized by arc
length exactly when the particle is moving at a speed of $1$.

\begin{question}
  Which of the following vector-valued functions are parameterized
  by arc length?
  \begin{selectAll}
    \choice[correct]{$t\vector{11/61,60/61}$}
    \choice[correct]{$\vector{3\sin(t/3),3\cos(t/3)}$}
    \choice{$t\vector{16/113,112/113}$}
    \choice[correct]{$\vector{7,t17/145,t144/145}$}
    \choice{$\vector{3\cos(t),3\sin(t)}$}
  \end{selectAll}
\end{question}


\begin{question}
  Consider $\vec{f}(t) = \vector{3\sin(a t),t/2,3\cos(a t)}$ for
  $0\le t$. Find $a$ that makes this parameterized by arc length.
  \begin{hint}
    Set $|\vec{f}'(t)| = 1$ and solve for $a$.
  \end{hint}
  \begin{prompt}
    \[
    a = \pm \answer{\frac{1}{2\sqrt{3}}}
    \]
  \end{prompt}
\end{question}

Often given a curve one wishes to have an arc length parameterization
of the curve.  We proceed by discussing several special cases, and
then by giving a general method.

\section{Disguised lines}

Sometimes you have a vector-valued function that is merely a line in
disguise.\index{line!in disguise} How could this be? Well consider
the vector-valued function:
\[
\vec{f}(t) = \vector{2-3\sin(t),1+4\sin(t)}\quad \text{for $-\pi/2\le t\le \pi/2$}
\]
This doesn't look very much like a line, for one thing it has the
function $\sin(t)$ in each component. On the other hand, if we look
at $\vec{f}'$, we see
\[
\vec{f}'(t) = \vector{-3\cos(t),4\cos(t)}
\]
Ah, we can now factor a $\cos(t)$ out of each component to get:
\[
\underbrace{\cos(t)}_{\text{scalar function}}\cdot \overbrace{\vector{-3,4}}^{\text{constant vector}}
\]
this is a scalar-function times a constant vector. The fact that we
can ``pull-out'' the scalar function, and are left with a constant
vector tells us that the line segment plotted by $\vec{f}$ for
$-\pi/2\le t\le \pi/2$ is identical to the line segment plotted by:
\begin{align*}
  \vec{g}(s) &=\vector{2,1} + s \vector{-3,4}\quad \text{for $-1\le s\le 1$}\\
  &=\vector{2-3s,1+4s}
\end{align*}
\begin{question}
  Which of the following are line segments in disguise?
  \begin{selectAll}
    \choice[correct]{$\vector{2+e^t,4,-2-e^t}$ for $0\le t\le 2$}
    \choice{$\vector{3+e^t,-1+2e^t,2-e^{2t}}$ for $0\le t\le 1$}
    \choice[correct]{$\vector{5-3\cos(t),4+2\cos(t),1+\cos(t)}$ for $0\le t\le 2\pi$}
    \choice{$\vector{-1 -\sin(t),3+\sin(t),2-\cos(t)}$ for $0\le t\le \pi/2$}
    \choice{$\vector{2+5t,3t^2}$ for $-1\le t\le 1$}
    \choice[correct]{$\vector{3-t^3,1+2t^3}$ for $-1\le t\le 1$}
  \end{selectAll}
\end{question}
Once we identify a vector-valued function as a disguised line, we
can rewrite it as
\[
\text{point}+ s \cdot \left(\text{unit vector}\right)
\]
and we have an arc length parameterization. Note, we need a unit
vector to ensure that the magnitude of the derivative is one!
\begin{example}
  Consider $\vec{f}(t) = \vector{3t^2,4t^2}$ for $0\le t\le
  1$. Parameterize this curve by arc length.
  \begin{explanation}
    If we think about $\vec{f}$ we see that the variable $t$ only
    appears in the expression as $t^2$. This means as $t$ grows, it
    will grow \textit{identically} in each component of $\vec{f}$.
    \begin{onlineOnly}
      Indeed a quick check with a graph will show that:
      \[
      \graph{(3t, 4t)}
      \]
      \[
      \graph{(3t^2,4t^2)}
      \]
      produce the same graph.
    \end{onlineOnly}
    Ah, so this is a line in disguise! To parameterize a line by arc
    length you need to write something like:
    \[
    \text{point}+ s \cdot \left(\text{unit vector}\right)
    \]
    So let's find two points on the line. Setting $t=0$, we see that
    $(0,0)$ is on the line. Setting $t = 1$ we see that $(3,4)$ is
    also on the line. The unit vector that runs from $(0,0)$ to
    $(3,4)$ is:
    \[
    \vector{\answer[given]{3/5},\answer[given]{4/5}}
    \]
    Thus as $s$ runs from $\answer[given]{0}$ to $\answer[given]{5}$, 
    $\vec{g}(s) = \vector{\answer[given]{3s/5},\answer[given]{4s/5}}$
    draws the same curve as $\vec{f}$ as $t$ runs from $\answer[given]{0}$ to $\answer[given]{1}$.
  \end{explanation}
\end{example}

\begin{question}
  Give an arc length parameterization of $\vec{f}(t) =
  \vector{3-4t^3,2+t^3,5-t^3}$ for $0\le t\le 1$.
  \begin{prompt}
    \[
    \vec{g}(s) =
    \vector{\answer{3-4s/\sqrt{18}},\answer{2+s/\sqrt{18}},\answer{5-s/\sqrt{18}}}
    \]
    for
    \[
    \answer{0} \le s \le \answer{\sqrt{18}}
    \]
  \end{prompt}
\end{question}

Try your hand at this one now:

\begin{question}
  Give an arc length parameterization of $\vec{f}(t) =
  \vector{1-e^t,3+e^t,5}$ for $0\le t\le 1$.
  \begin{hint}
    Check the values of $\vec{f}(0)$ and $\vec{f}(1)$.
    \end{hint}
  \begin{prompt}
    \[
    \vec{g}(s) =
    \vector{\answer{-s/\sqrt{2}},\answer{4+s/\sqrt{2}},\answer{5}}
    \]
    for
    \[
    \answer{0} \le s \le \answer{\sqrt{2(e-1)^2}}
    \]
  \end{prompt}
\end{question}


\section{Disguised circles}

Sometimes the curve we are given is a circle in disguise.
\index{circle!in disguise}

\begin{example}
  Consider $\vec{f}(t) = \vector{\sin(2\pi t^2),\cos(2\pi t^2)}$ for
  $0\le t\le 1$. Parameterize this curve by arc length.
  \begin{explanation}
    Here, we should recognize this curve a unit circle, being drawn
    in a counterclockwise fashion, starting (when $t=0$) at the
    point $\left(\answer[given]{0},\answer[given]{1}\right)$. Ah! So
    an arc length parameterization is given by
    \[
    \vec{g}(s) = \vector{\answer[given]{\sin(s)},\answer[given]{\cos(s)}}.
    \]
  \end{explanation}
\end{example}


\begin{question}
  Consider $\vec{f}(t) = \vector{5\cos(t),5\sin(t)}$ for $0\le t<
  2\pi$. Parameterize this curve by arc length.
  \begin{prompt}
    \[
    \vec{g}(s) = \vector{\answer{5\cos(s/5)},\answer{5\sin(s/5)}}
    \]
    for
    \[
    \answer{0}\le s < \answer{10\pi}
    \]
  \end{prompt}
\end{question}

\begin{example}
  The Moon travels in a orbit around the Earth that can be
  approximated by a circle. The distance from the Earth to the Moon is
  around $240$ thousand miles. Make the following assumptions:
  \begin{itemize}
  \item The Earth will be at the origin.
  \item At the starting time, $t=0$, the Moon will be at the point
    $(240,0)$ in the $(x,y)$-plane.
  \item The Moon will travel in a counterclockwise direction around
    the Earth.
  \end{itemize}
  Give a parameterization of the Moon's orbit that will model the
  Moon's position in terms of $s$, the distance traveled in thousands
  of miles.
  \begin{explanation}
    First compute the circumference of the Moon's orbit in thousands
    of miles:
    \[
    \text{circumference} = \answer[given]{480\pi}
    \]
    Now we write:
    \[
    \vec{m}(s) = \vector{\answer[given]{\cos(s/240)},\answer[given]{\sin(s/240)}}
    \]
  \end{explanation}
\end{example}





\section{A general method}

While we are about to present a general method for finding
representations of functions parameterized by arc length, one must not
overestimate its strength.

Regardless, if you want an arc length parameterization of $\vec{f}(t)$
starting at $t=a$ here is the idea:
\begin{enumerate}
\item Compute
  \[
  L(t)  = \int_a^t |\vec{f}'(u)| \d u
  \]
\item Now write
  \[
  s = L(t)
  \]
  and solve for $t$. In this case you will have
  \[
  t = L^{-1}(s)
  \]
\item The function
  \[
  \vec{g}(s) = \vec{f}(L^{-1}(s))
  \]
  will be parameterized by arc length.
\end{enumerate}

Try your hand at it.

\begin{example}
  Parameterize $\vec{f}(t) = \vector{\cos(t),\sin(t),t}$ for $t\ge 0$ by
  arc length.
  \begin{explanation}
    First we'll compute the magnitude of $\vec{f}$. Write with me:
    \[
    |\vec{f}'(t)| = \sqrt{\answer[given]{2}}
    \]
    So now:
    \begin{align*}
      L(t) &= \int_0^t \sqrt{2} \d u\\
      &= \eval{\answer[given]{u\sqrt{2}}}_0^t\\
      &= \answer[given]{t\sqrt{2}}.
    \end{align*}
    So now set $s = L(t)$:
    \[
    s = \answer[given]{t\sqrt{2}}
    \]
    Now compute $L^{-1}(s)$, in essence just solve for $t$:
    \[
    t = \answer[given]{s/\sqrt{2}}
    \]
    Our curve, now parameterized by arc length is
    \[
    \vec{g}(s) = \vector{\answer[given]{\cos(s/\sqrt{2})},\answer[given]{\sin(s/\sqrt{2})},\answer[given]{s/\sqrt{2}}}.
    \]
  \end{explanation}
\end{example}


  
\end{document}
