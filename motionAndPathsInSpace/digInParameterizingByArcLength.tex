\documentclass{ximera}

%\usepackage{todonotes}
%\usepackage{mathtools} %% Required for wide table Curl and Greens
%\usepackage{cuted} %% Required for wide table Curl and Greens
\newcommand{\todo}{}

\usepackage{esint} % for \oiint
\ifxake%%https://math.meta.stackexchange.com/questions/9973/how-do-you-render-a-closed-surface-double-integral
\renewcommand{\oiint}{{\large\bigcirc}\kern-1.56em\iint}
\fi


\graphicspath{
  {./}
  {ximeraTutorial/}
  {basicPhilosophy/}
  {functionsOfSeveralVariables/}
  {normalVectors/}
  {lagrangeMultipliers/}
  {vectorFields/}
  {greensTheorem/}
  {shapeOfThingsToCome/}
  {dotProducts/}
  {partialDerivativesAndTheGradientVector/}
  {../productAndQuotientRules/exercises/}
  {../normalVectors/exercisesParametricPlots/}
  {../continuityOfFunctionsOfSeveralVariables/exercises/}
  {../partialDerivativesAndTheGradientVector/exercises/}
  {../directionalDerivativeAndChainRule/exercises/}
  {../commonCoordinates/exercisesCylindricalCoordinates/}
  {../commonCoordinates/exercisesSphericalCoordinates/}
  {../greensTheorem/exercisesCurlAndLineIntegrals/}
  {../greensTheorem/exercisesDivergenceAndLineIntegrals/}
  {../shapeOfThingsToCome/exercisesDivergenceTheorem/}
  {../greensTheorem/}
  {../shapeOfThingsToCome/}
  {../separableDifferentialEquations/exercises/}
  {vectorFields/}
}

\newcommand{\mooculus}{\textsf{\textbf{MOOC}\textnormal{\textsf{ULUS}}}}

\usepackage{tkz-euclide}\usepackage{tikz}
\usepackage{tikz-cd}
\usetikzlibrary{arrows}
\tikzset{>=stealth,commutative diagrams/.cd,
  arrow style=tikz,diagrams={>=stealth}} %% cool arrow head
\tikzset{shorten <>/.style={ shorten >=#1, shorten <=#1 } } %% allows shorter vectors

\usetikzlibrary{backgrounds} %% for boxes around graphs
\usetikzlibrary{shapes,positioning}  %% Clouds and stars
\usetikzlibrary{matrix} %% for matrix
\usepgfplotslibrary{polar} %% for polar plots
\usepgfplotslibrary{fillbetween} %% to shade area between curves in TikZ
\usetkzobj{all}
\usepackage[makeroom]{cancel} %% for strike outs
%\usepackage{mathtools} %% for pretty underbrace % Breaks Ximera
%\usepackage{multicol}
\usepackage{pgffor} %% required for integral for loops



%% http://tex.stackexchange.com/questions/66490/drawing-a-tikz-arc-specifying-the-center
%% Draws beach ball
\tikzset{pics/carc/.style args={#1:#2:#3}{code={\draw[pic actions] (#1:#3) arc(#1:#2:#3);}}}



\usepackage{array}
\setlength{\extrarowheight}{+.1cm}
\newdimen\digitwidth
\settowidth\digitwidth{9}
\def\divrule#1#2{
\noalign{\moveright#1\digitwidth
\vbox{\hrule width#2\digitwidth}}}





\newcommand{\RR}{\mathbb R}
\newcommand{\R}{\mathbb R}
\newcommand{\N}{\mathbb N}
\newcommand{\Z}{\mathbb Z}

\newcommand{\sagemath}{\textsf{SageMath}}


%\renewcommand{\d}{\,d\!}
\renewcommand{\d}{\mathop{}\!d}
\newcommand{\dd}[2][]{\frac{\d #1}{\d #2}}
\newcommand{\pp}[2][]{\frac{\partial #1}{\partial #2}}
\renewcommand{\l}{\ell}
\newcommand{\ddx}{\frac{d}{\d x}}

\newcommand{\zeroOverZero}{\ensuremath{\boldsymbol{\tfrac{0}{0}}}}
\newcommand{\inftyOverInfty}{\ensuremath{\boldsymbol{\tfrac{\infty}{\infty}}}}
\newcommand{\zeroOverInfty}{\ensuremath{\boldsymbol{\tfrac{0}{\infty}}}}
\newcommand{\zeroTimesInfty}{\ensuremath{\small\boldsymbol{0\cdot \infty}}}
\newcommand{\inftyMinusInfty}{\ensuremath{\small\boldsymbol{\infty - \infty}}}
\newcommand{\oneToInfty}{\ensuremath{\boldsymbol{1^\infty}}}
\newcommand{\zeroToZero}{\ensuremath{\boldsymbol{0^0}}}
\newcommand{\inftyToZero}{\ensuremath{\boldsymbol{\infty^0}}}



\newcommand{\numOverZero}{\ensuremath{\boldsymbol{\tfrac{\#}{0}}}}
\newcommand{\dfn}{\textbf}
%\newcommand{\unit}{\,\mathrm}
\newcommand{\unit}{\mathop{}\!\mathrm}
\newcommand{\eval}[1]{\bigg[ #1 \bigg]}
\newcommand{\seq}[1]{\left( #1 \right)}
\renewcommand{\epsilon}{\varepsilon}
\renewcommand{\phi}{\varphi}


\renewcommand{\iff}{\Leftrightarrow}

\DeclareMathOperator{\arccot}{arccot}
\DeclareMathOperator{\arcsec}{arcsec}
\DeclareMathOperator{\arccsc}{arccsc}
\DeclareMathOperator{\si}{Si}
\DeclareMathOperator{\scal}{scal}
\DeclareMathOperator{\sign}{sign}


%% \newcommand{\tightoverset}[2]{% for arrow vec
%%   \mathop{#2}\limits^{\vbox to -.5ex{\kern-0.75ex\hbox{$#1$}\vss}}}
\newcommand{\arrowvec}[1]{{\overset{\rightharpoonup}{#1}}}
%\renewcommand{\vec}[1]{\arrowvec{\mathbf{#1}}}
\renewcommand{\vec}[1]{{\overset{\boldsymbol{\rightharpoonup}}{\mathbf{#1}}}\hspace{0in}}

\newcommand{\point}[1]{\left(#1\right)} %this allows \vector{ to be changed to \vector{ with a quick find and replace
\newcommand{\pt}[1]{\mathbf{#1}} %this allows \vec{ to be changed to \vec{ with a quick find and replace
\newcommand{\Lim}[2]{\lim_{\point{#1} \to \point{#2}}} %Bart, I changed this to point since I want to use it.  It runs through both of the exercise and exerciseE files in limits section, which is why it was in each document to start with.

\DeclareMathOperator{\proj}{\mathbf{proj}}
\newcommand{\veci}{{\boldsymbol{\hat{\imath}}}}
\newcommand{\vecj}{{\boldsymbol{\hat{\jmath}}}}
\newcommand{\veck}{{\boldsymbol{\hat{k}}}}
\newcommand{\vecl}{\vec{\boldsymbol{\l}}}
\newcommand{\uvec}[1]{\mathbf{\hat{#1}}}
\newcommand{\utan}{\mathbf{\hat{t}}}
\newcommand{\unormal}{\mathbf{\hat{n}}}
\newcommand{\ubinormal}{\mathbf{\hat{b}}}

\newcommand{\dotp}{\bullet}
\newcommand{\cross}{\boldsymbol\times}
\newcommand{\grad}{\boldsymbol\nabla}
\newcommand{\divergence}{\grad\dotp}
\newcommand{\curl}{\grad\cross}
%\DeclareMathOperator{\divergence}{divergence}
%\DeclareMathOperator{\curl}[1]{\grad\cross #1}
\newcommand{\lto}{\mathop{\longrightarrow\,}\limits}

\renewcommand{\bar}{\overline}

\colorlet{textColor}{black}
\colorlet{background}{white}
\colorlet{penColor}{blue!50!black} % Color of a curve in a plot
\colorlet{penColor2}{red!50!black}% Color of a curve in a plot
\colorlet{penColor3}{red!50!blue} % Color of a curve in a plot
\colorlet{penColor4}{green!50!black} % Color of a curve in a plot
\colorlet{penColor5}{orange!80!black} % Color of a curve in a plot
\colorlet{penColor6}{yellow!70!black} % Color of a curve in a plot
\colorlet{fill1}{penColor!20} % Color of fill in a plot
\colorlet{fill2}{penColor2!20} % Color of fill in a plot
\colorlet{fillp}{fill1} % Color of positive area
\colorlet{filln}{penColor2!20} % Color of negative area
\colorlet{fill3}{penColor3!20} % Fill
\colorlet{fill4}{penColor4!20} % Fill
\colorlet{fill5}{penColor5!20} % Fill
\colorlet{gridColor}{gray!50} % Color of grid in a plot

\newcommand{\surfaceColor}{violet}
\newcommand{\surfaceColorTwo}{redyellow}
\newcommand{\sliceColor}{greenyellow}




\pgfmathdeclarefunction{gauss}{2}{% gives gaussian
  \pgfmathparse{1/(#2*sqrt(2*pi))*exp(-((x-#1)^2)/(2*#2^2))}%
}


%%%%%%%%%%%%%
%% Vectors
%%%%%%%%%%%%%

%% Simple horiz vectors
\renewcommand{\vector}[1]{\left\langle #1\right\rangle}


%% %% Complex Horiz Vectors with angle brackets
%% \makeatletter
%% \renewcommand{\vector}[2][ , ]{\left\langle%
%%   \def\nextitem{\def\nextitem{#1}}%
%%   \@for \el:=#2\do{\nextitem\el}\right\rangle%
%% }
%% \makeatother

%% %% Vertical Vectors
%% \def\vector#1{\begin{bmatrix}\vecListA#1,,\end{bmatrix}}
%% \def\vecListA#1,{\if,#1,\else #1\cr \expandafter \vecListA \fi}

%%%%%%%%%%%%%
%% End of vectors
%%%%%%%%%%%%%

%\newcommand{\fullwidth}{}
%\newcommand{\normalwidth}{}



%% makes a snazzy t-chart for evaluating functions
%\newenvironment{tchart}{\rowcolors{2}{}{background!90!textColor}\array}{\endarray}

%%This is to help with formatting on future title pages.
\newenvironment{sectionOutcomes}{}{}



%% Flowchart stuff
%\tikzstyle{startstop} = [rectangle, rounded corners, minimum width=3cm, minimum height=1cm,text centered, draw=black]
%\tikzstyle{question} = [rectangle, minimum width=3cm, minimum height=1cm, text centered, draw=black]
%\tikzstyle{decision} = [trapezium, trapezium left angle=70, trapezium right angle=110, minimum width=3cm, minimum height=1cm, text centered, draw=black]
%\tikzstyle{question} = [rectangle, rounded corners, minimum width=3cm, minimum height=1cm,text centered, draw=black]
%\tikzstyle{process} = [rectangle, minimum width=3cm, minimum height=1cm, text centered, draw=black]
%\tikzstyle{decision} = [trapezium, trapezium left angle=70, trapezium right angle=110, minimum width=3cm, minimum height=1cm, text centered, draw=black]


\title[Dig-In:]{Parameterizing by arc length}

\outcome{Reparametrize a curve.}
\outcome{Parameterize a curve in terms of arc length.}

\begin{document}
\begin{abstract}
  We find a new description of curves that trivializes arc length
  computations.
\end{abstract}
\maketitle

Recall that if $\vec{f}$ is a vector-valued function where
\begin{itemize}
\item $\vec{f}'$ is continuous.
\item The curve defined by $\vec{f}(t)$ is traversed once for $a\le
  t\le b$.
\end{itemize}
  The arc length of the curve from
  \[
  \vec{f}(a)\quad\text{to}\quad\vec{f}(b)
  \]
  is given by
  \[
  \text{arc length} = \int_a^b |\vec{f}'(t)|\d t.
  \]
  This is all good and well; however, the integral
  \[
  \int_a^b |\vec{f}'(t)|\d t
  \]
  could be quite difficult to compute. In this section, we see a new
  description of the curve drawn by $\vec{f}(t)$, we'll call it
  $\vec{g}(s)$ where the \textbf{same} curve is drawn by both $\vec{f}$ and
  $\vec{g}$ and we have that
  \[
  s =  \int_0^s |\vec{g}'(t)|\d t.
  \]
  This is called an \dfn{arc length parameterization}. It is nice to
  work with functions parameterized by arc length, because computing
  the arc length is easy. If $g$ is parameterized by arc length, then
  the length of $g(s)$ when $a\le s\le b$, is simply $b-a$. No
  integral computations need to be done.  Consider the following
  example:

  \begin{example}
    Let $\vec{f}(t) = \vector{\cos(t), \sin(t)}$ for $0\le t<
    2\pi$. Show that $\vec{f}$ is parameterized by arc length.
    \begin{explanation}
      Here we need to show that
      \[
      s = \int_0^s |\vec{f}'(t)| \d t.
      \]
      We'll just compute the right-hand side of the equation above and
      see what happens. Write with me,
      \[
      \vec{f}'(t) = \vector{\answer[given]{-\sin(t)},\answer[given]{\cos(t)}}
      \]
      and so
      \begin{align*}
      |\vec{f}'(t)| &= \sqrt{\vec{f}'(t)\dotp \vec{f}'(t)}\\
      &= \sqrt{\vector{\answer[given]{-\sin(t)},\answer[given]{\cos(t)}}\dotp\vector{\answer[given]{-\sin(t)},\answer[given]{\cos(t)}}}\\
      &= \answer[given]{1}.
      \end{align*}
      Now our integral becomes:
      \begin{align*}
        \int_0^s  |\vec{f}'(t)| \d t &= \int_0^s \d t\\
        &= \answer[given]{s}.
      \end{align*}
      Hence $\vec{f}$ is parameterized by arc length.
    \end{explanation}
  \end{example}

  From your own experience and the work above, we think the next
  theorem should be quite sensible.

  \begin{theorem}
    A vector-valued function $\vec{f}:\R\to \R^2$ is parameterized by
    arc length if and only if $|\vec{f}'| = 1$.
  \end{theorem}

  \begin{question}
    Which of the following vector-valued functions are parameterized
    by arc length?
    \begin{selectAll}
      \choice[correct]{$t\vector{11/61,60/61}$}
      \choice[correct]{$\vector{3\sin(t/3),3\cos(t/3)}$}
      \choice{$t\vector{16/113,112/113}$}
      \choice[correct]{$\vector{7,t17/145,t144/145}$}
      \choice{$\vector{3\cos(t),3\sin(t)}$}
    \end{selectAll}
  \end{question}

  We proceed by discussing several special cases, and then by giving a
  general method.

  \section{Disguised lines}

  Sometimes you have a vector-valued function that is merely a line
  in disguise.\index{line!in disguise}

  \begin{example}
    Consider $\vec{f}(t) = \vector{3t^2,4t^2}$ for $0\le t\le
    1$. Parameterize this curve by arc length.
    \begin{explanation}
      If we think about $\vec{f}$ we see that the variable $t$ only
      appears in the expression as $t^2$. This means as $t$ grows, it
      will grow \textit{identically} in each component of $\vec{f}$.
      \begin{onlineOnly}
        Indeed a quick check with a graph will show that:
        \[
        \graph{(3t, 4t)}
        \]
        \[
        \graph{(3t^2,4t^2)}
        \]
        produce the same graph.
      \end{onlineOnly}
      Now, by our theorem above, $\vec{g}$ is parameterized by arc
      length if and only if
      \[
      |\vec{g}'(t)| = \answer[given]{1}
      \]
      Hence we need to find a unit vector in the the same direction as
      the line drawn by $\vec{f}$. Write with me,
      \begin{align*}
        \frac{\vec{f}'(t)}{|\vec{f}'(t)|} &= \frac{\vector{\answer[given]{6t},\answer[given]{8t}}}{\answer[given]{10t}}\\
        &=\vector{\answer[given]{3/5},\answer[given]{4/5}}.
      \end{align*}
      Hence $\vec{g}(s) =
      s\vector{\answer[given]{3/5},\answer[given]{4/5}}$ draws the
      same curve as $\vec{f}$ and is parameterized by arc length.
    \end{explanation}
  \end{example}

  \begin{question}
    Give an arc length parameterization of $\vec{f}(t) =
    \vector{3-4t^3,2+t^3,5-t^3}$ for $0\le t\le 1$.
    \begin{prompt}
      \[
      \vec{g}(s) =
      \vector{\answer{3-4s/\sqrt{18}},\answer{2+s/\sqrt{18}},\answer{5-s/\sqrt{18}}}
      \]
    \end{prompt}
  \end{question}

  Try your hand at this one now:

    \begin{question}
    Give an arc length parameterization of $\vec{f}(t) =
    \vector{1-e^t,3+e^t,5}$ for $0\le t$.
    \begin{hint}
      Check the values of $\vec{f}(0)$ and $\vec{f}(1)$.
    \end{hint}
    \begin{prompt}
      \[
      \vec{g}(s) =
      \vector{\answer{-s/\sqrt{2}},\answer{4+s/\sqrt{2}},\answer{5}}
      \]
    \end{prompt}
  \end{question}

  
  \section{Disguised circles}

  Sometimes the curve we are given is a circle in disguise.
  \index{circle!in disguise}

  \begin{example}
    Consider $\vec{f}(t) = \vector{\sin(2\pi t^2),\cos(2\pi t^2)}$ for
    $0\le t\le 1$. Parameterize this curve by arc length.
    \begin{explanation}
      Here, we should recognize this curve a unit circle, being drawn
      in a counterclockwise fashion, starting (when $t=0$) at the
      point $\left(\answer[given]{0},\answer[given]{1}\right)$. Ah! So
      an arc length parameterization is given by
      \[
      \vec{g}(s) = \vector{\answer[given]{\sin(s)},\answer[given]{\cos(s)}}.
      \]
    \end{explanation}
  \end{example}


  \begin{question}
    Consider $\vec{f}(t) = \vector{3\sin(a t),t/2,3\cos(a t)}$ for
    $0\le t$. Find $a$ that makes this parameterized by arc length.
    \begin{hint}
      Set $|\vec{f}'(t)| = 1$ and solve for $a$.
    \end{hint}
    \begin{prompt}
      \[
      a = \pm \answer{\frac{1}{2\sqrt{3}}}
      \]
    \end{prompt}
  \end{question}
  

  
  \section{A general method}

  While we are about to present a general method for finding
  representations of functions parameterized by arc length, one must
  not overestimate its strength.

  Regardless, here is the idea:
  \begin{enumerate}
  \item Given $\vec{f}(t)$ compute
    \[
    L(t)  = \int_a^t |\vec{f}'(u)| \d u
    \]
  \item Now write
    \[
    s = L(t)
    \]
    and solve for $t$. In this case you will have
    \[
    t = L^{-1}(s)
    \]
  \item The function
  \[
  \vec{g}(s) = \vec{f}(L^{-1}(s))
  \]
  will be parameterized by arc length.
  \end{enumerate}

  Try your hand at it.

  \begin{example}
    Parameterize $\vec{f}(t) = \vector{6t^3,8t^3,3}$ for $t\ge 0$ by
    arc length.
    \begin{explanation}
      Write with me,
      \begin{align*}
      |\vec{f}'(t)| &= \sqrt{\answer[given]{18^2t^4+24^2t^4}}\\
      &=\answer[given]{30t^2}
      \end{align*}
      \begin{align*}
      L(t) &= \int_0^t 30 u^2 \d u\\
      &= \eval{\answer[given]{10 u^3}}_0^t\\
      &= \answer[given]{10t^3}.
      \end{align*}
      So now set $s = \answer[given]{10t^3}$ and solve for $t$.
      \[
      t = \sqrt[3]{\answer[given]{s/10}}
      \]
      Our function parameterized by arc length is
      \[
      \vec{g}(s) = \vector{\answer[given]{6s/10},\answer[given]{8s/10},\answer[given]{3}}.
      \]
    \end{explanation}
  \end{example}


  
\end{document}
