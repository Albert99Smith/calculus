\documentclass{ximera}

%\usepackage{todonotes}
%\usepackage{mathtools} %% Required for wide table Curl and Greens
%\usepackage{cuted} %% Required for wide table Curl and Greens
\newcommand{\todo}{}

\usepackage{esint} % for \oiint
\ifxake%%https://math.meta.stackexchange.com/questions/9973/how-do-you-render-a-closed-surface-double-integral
\renewcommand{\oiint}{{\large\bigcirc}\kern-1.56em\iint}
\fi


\graphicspath{
  {./}
  {ximeraTutorial/}
  {basicPhilosophy/}
  {functionsOfSeveralVariables/}
  {normalVectors/}
  {lagrangeMultipliers/}
  {vectorFields/}
  {greensTheorem/}
  {shapeOfThingsToCome/}
  {dotProducts/}
  {partialDerivativesAndTheGradientVector/}
  {../productAndQuotientRules/exercises/}
  {../normalVectors/exercisesParametricPlots/}
  {../continuityOfFunctionsOfSeveralVariables/exercises/}
  {../partialDerivativesAndTheGradientVector/exercises/}
  {../directionalDerivativeAndChainRule/exercises/}
  {../commonCoordinates/exercisesCylindricalCoordinates/}
  {../commonCoordinates/exercisesSphericalCoordinates/}
  {../greensTheorem/exercisesCurlAndLineIntegrals/}
  {../greensTheorem/exercisesDivergenceAndLineIntegrals/}
  {../shapeOfThingsToCome/exercisesDivergenceTheorem/}
  {../greensTheorem/}
  {../shapeOfThingsToCome/}
  {../separableDifferentialEquations/exercises/}
  {vectorFields/}
}

\newcommand{\mooculus}{\textsf{\textbf{MOOC}\textnormal{\textsf{ULUS}}}}

\usepackage{tkz-euclide}\usepackage{tikz}
\usepackage{tikz-cd}
\usetikzlibrary{arrows}
\tikzset{>=stealth,commutative diagrams/.cd,
  arrow style=tikz,diagrams={>=stealth}} %% cool arrow head
\tikzset{shorten <>/.style={ shorten >=#1, shorten <=#1 } } %% allows shorter vectors

\usetikzlibrary{backgrounds} %% for boxes around graphs
\usetikzlibrary{shapes,positioning}  %% Clouds and stars
\usetikzlibrary{matrix} %% for matrix
\usepgfplotslibrary{polar} %% for polar plots
\usepgfplotslibrary{fillbetween} %% to shade area between curves in TikZ
\usetkzobj{all}
\usepackage[makeroom]{cancel} %% for strike outs
%\usepackage{mathtools} %% for pretty underbrace % Breaks Ximera
%\usepackage{multicol}
\usepackage{pgffor} %% required for integral for loops



%% http://tex.stackexchange.com/questions/66490/drawing-a-tikz-arc-specifying-the-center
%% Draws beach ball
\tikzset{pics/carc/.style args={#1:#2:#3}{code={\draw[pic actions] (#1:#3) arc(#1:#2:#3);}}}



\usepackage{array}
\setlength{\extrarowheight}{+.1cm}
\newdimen\digitwidth
\settowidth\digitwidth{9}
\def\divrule#1#2{
\noalign{\moveright#1\digitwidth
\vbox{\hrule width#2\digitwidth}}}





\newcommand{\RR}{\mathbb R}
\newcommand{\R}{\mathbb R}
\newcommand{\N}{\mathbb N}
\newcommand{\Z}{\mathbb Z}

\newcommand{\sagemath}{\textsf{SageMath}}


%\renewcommand{\d}{\,d\!}
\renewcommand{\d}{\mathop{}\!d}
\newcommand{\dd}[2][]{\frac{\d #1}{\d #2}}
\newcommand{\pp}[2][]{\frac{\partial #1}{\partial #2}}
\renewcommand{\l}{\ell}
\newcommand{\ddx}{\frac{d}{\d x}}

\newcommand{\zeroOverZero}{\ensuremath{\boldsymbol{\tfrac{0}{0}}}}
\newcommand{\inftyOverInfty}{\ensuremath{\boldsymbol{\tfrac{\infty}{\infty}}}}
\newcommand{\zeroOverInfty}{\ensuremath{\boldsymbol{\tfrac{0}{\infty}}}}
\newcommand{\zeroTimesInfty}{\ensuremath{\small\boldsymbol{0\cdot \infty}}}
\newcommand{\inftyMinusInfty}{\ensuremath{\small\boldsymbol{\infty - \infty}}}
\newcommand{\oneToInfty}{\ensuremath{\boldsymbol{1^\infty}}}
\newcommand{\zeroToZero}{\ensuremath{\boldsymbol{0^0}}}
\newcommand{\inftyToZero}{\ensuremath{\boldsymbol{\infty^0}}}



\newcommand{\numOverZero}{\ensuremath{\boldsymbol{\tfrac{\#}{0}}}}
\newcommand{\dfn}{\textbf}
%\newcommand{\unit}{\,\mathrm}
\newcommand{\unit}{\mathop{}\!\mathrm}
\newcommand{\eval}[1]{\bigg[ #1 \bigg]}
\newcommand{\seq}[1]{\left( #1 \right)}
\renewcommand{\epsilon}{\varepsilon}
\renewcommand{\phi}{\varphi}


\renewcommand{\iff}{\Leftrightarrow}

\DeclareMathOperator{\arccot}{arccot}
\DeclareMathOperator{\arcsec}{arcsec}
\DeclareMathOperator{\arccsc}{arccsc}
\DeclareMathOperator{\si}{Si}
\DeclareMathOperator{\scal}{scal}
\DeclareMathOperator{\sign}{sign}


%% \newcommand{\tightoverset}[2]{% for arrow vec
%%   \mathop{#2}\limits^{\vbox to -.5ex{\kern-0.75ex\hbox{$#1$}\vss}}}
\newcommand{\arrowvec}[1]{{\overset{\rightharpoonup}{#1}}}
%\renewcommand{\vec}[1]{\arrowvec{\mathbf{#1}}}
\renewcommand{\vec}[1]{{\overset{\boldsymbol{\rightharpoonup}}{\mathbf{#1}}}\hspace{0in}}

\newcommand{\point}[1]{\left(#1\right)} %this allows \vector{ to be changed to \vector{ with a quick find and replace
\newcommand{\pt}[1]{\mathbf{#1}} %this allows \vec{ to be changed to \vec{ with a quick find and replace
\newcommand{\Lim}[2]{\lim_{\point{#1} \to \point{#2}}} %Bart, I changed this to point since I want to use it.  It runs through both of the exercise and exerciseE files in limits section, which is why it was in each document to start with.

\DeclareMathOperator{\proj}{\mathbf{proj}}
\newcommand{\veci}{{\boldsymbol{\hat{\imath}}}}
\newcommand{\vecj}{{\boldsymbol{\hat{\jmath}}}}
\newcommand{\veck}{{\boldsymbol{\hat{k}}}}
\newcommand{\vecl}{\vec{\boldsymbol{\l}}}
\newcommand{\uvec}[1]{\mathbf{\hat{#1}}}
\newcommand{\utan}{\mathbf{\hat{t}}}
\newcommand{\unormal}{\mathbf{\hat{n}}}
\newcommand{\ubinormal}{\mathbf{\hat{b}}}

\newcommand{\dotp}{\bullet}
\newcommand{\cross}{\boldsymbol\times}
\newcommand{\grad}{\boldsymbol\nabla}
\newcommand{\divergence}{\grad\dotp}
\newcommand{\curl}{\grad\cross}
%\DeclareMathOperator{\divergence}{divergence}
%\DeclareMathOperator{\curl}[1]{\grad\cross #1}
\newcommand{\lto}{\mathop{\longrightarrow\,}\limits}

\renewcommand{\bar}{\overline}

\colorlet{textColor}{black}
\colorlet{background}{white}
\colorlet{penColor}{blue!50!black} % Color of a curve in a plot
\colorlet{penColor2}{red!50!black}% Color of a curve in a plot
\colorlet{penColor3}{red!50!blue} % Color of a curve in a plot
\colorlet{penColor4}{green!50!black} % Color of a curve in a plot
\colorlet{penColor5}{orange!80!black} % Color of a curve in a plot
\colorlet{penColor6}{yellow!70!black} % Color of a curve in a plot
\colorlet{fill1}{penColor!20} % Color of fill in a plot
\colorlet{fill2}{penColor2!20} % Color of fill in a plot
\colorlet{fillp}{fill1} % Color of positive area
\colorlet{filln}{penColor2!20} % Color of negative area
\colorlet{fill3}{penColor3!20} % Fill
\colorlet{fill4}{penColor4!20} % Fill
\colorlet{fill5}{penColor5!20} % Fill
\colorlet{gridColor}{gray!50} % Color of grid in a plot

\newcommand{\surfaceColor}{violet}
\newcommand{\surfaceColorTwo}{redyellow}
\newcommand{\sliceColor}{greenyellow}




\pgfmathdeclarefunction{gauss}{2}{% gives gaussian
  \pgfmathparse{1/(#2*sqrt(2*pi))*exp(-((x-#1)^2)/(2*#2^2))}%
}


%%%%%%%%%%%%%
%% Vectors
%%%%%%%%%%%%%

%% Simple horiz vectors
\renewcommand{\vector}[1]{\left\langle #1\right\rangle}


%% %% Complex Horiz Vectors with angle brackets
%% \makeatletter
%% \renewcommand{\vector}[2][ , ]{\left\langle%
%%   \def\nextitem{\def\nextitem{#1}}%
%%   \@for \el:=#2\do{\nextitem\el}\right\rangle%
%% }
%% \makeatother

%% %% Vertical Vectors
%% \def\vector#1{\begin{bmatrix}\vecListA#1,,\end{bmatrix}}
%% \def\vecListA#1,{\if,#1,\else #1\cr \expandafter \vecListA \fi}

%%%%%%%%%%%%%
%% End of vectors
%%%%%%%%%%%%%

%\newcommand{\fullwidth}{}
%\newcommand{\normalwidth}{}



%% makes a snazzy t-chart for evaluating functions
%\newenvironment{tchart}{\rowcolors{2}{}{background!90!textColor}\array}{\endarray}

%%This is to help with formatting on future title pages.
\newenvironment{sectionOutcomes}{}{}



%% Flowchart stuff
%\tikzstyle{startstop} = [rectangle, rounded corners, minimum width=3cm, minimum height=1cm,text centered, draw=black]
%\tikzstyle{question} = [rectangle, minimum width=3cm, minimum height=1cm, text centered, draw=black]
%\tikzstyle{decision} = [trapezium, trapezium left angle=70, trapezium right angle=110, minimum width=3cm, minimum height=1cm, text centered, draw=black]
%\tikzstyle{question} = [rectangle, rounded corners, minimum width=3cm, minimum height=1cm,text centered, draw=black]
%\tikzstyle{process} = [rectangle, minimum width=3cm, minimum height=1cm, text centered, draw=black]
%\tikzstyle{decision} = [trapezium, trapezium left angle=70, trapezium right angle=110, minimum width=3cm, minimum height=1cm, text centered, draw=black]


\title[Dig-In:]{Motion and paths in space}


\outcome{View a vector valued function as a position function.}
\outcome{Compute the length of a parametric curve.}
\outcome{For a position vector function of time, interpret
  the derivative as velocity.}
\outcome{For a position vector function of time, interpret the second
  derivative as acceleration.}
\outcome{Compute the length of a parametric curve.}
    
\begin{document}
\begin{abstract}
  We interpret vector-valued functions as paths of objects in space.
\end{abstract}
\maketitle

\section{Position, velocity, and acceleration}

From single-variable calculus, we know that if $v(t)$ is a function that
represents the \dfn{velocity} (signed speed) of an object at time $t$,
then
\begin{itemize}
\item $v'(t)$ tells us the \dfn{acceleration} (instantaneous change in
  velocity) of the object, and
\item $\int_a^b v(t) \d t$ tells us the \dfn{displacement} (position
  with respect to an origin) of the object.
\end{itemize}

There is a similar story to be told with vector-valued functions.

\begin{definition}
Let $\vec{v}(t)$ be a vector-valued function denoting the velocity of
some object in $\R^2$ or $\R^3$:
\begin{itemize}
\item $\vec{v}'(t)$ tells us the \dfn{acceleration} (instantaneous
  change in velocity) of the object, and 
\item $\int_a^b \vec{v}(t) \d t$ tells us the \dfn{displacement} (position
  with respect to an origin) of the object.
\end{itemize}
Additionally, the \dfn{speed} of the object is the magnitude of
velocity, $|\vec v(t)|$.
\end{definition}

\begin{question}
  An object is moving with velocity $\vec{v}(t) =
  \vector{t^2-4t,4-t^2}$ for $0\le t\le 3$. Let speed be measured in
  the units of ``feet-per-second.'' Give a vector valued formula for
  the acceleration of our object at time $t$.
  \begin{prompt}
    The acceleration is given by $\vector{\answer{2t-4},\answer{-2t}}$
    feet per second per second.
  \end{prompt}
  \begin{question}
    Supposing further that our object starts at the point $(2,3)$
    relative to the origin, give a vector valued formula for the
    position of of our object at time $t$.
    \begin{prompt}
      The position is given by
      $\vector{\answer{t^3/3-2t^2+2},\answer{3+4t-t^3/3}}$.
    \end{prompt}
    \begin{question}
      What is speed of our object at time $t$?
      \begin{prompt}
        The speed of our object is $\answer{\sqrt{2t^4-8t^3+8t^2+16}}$
        feet per second.
      \end{prompt}
    \begin{question}
      When is the object's speed maximized?
      \begin{hint}
        It might be easiest to maximize the square of the formula for
        speed.
      \end{hint}
      \begin{prompt}
        The speed is maximized when $t=\answer{1}$.
      \end{prompt}
    \end{question}
    \end{question}
  \end{question}
\end{question}


Note, from the definition above, we also see that if the
\textit{position} of an object with respect to some origin is given by
a vector-valued function $\vec{p}(t)$, then
\begin{itemize}
\item $\vec{p}'(t)$ gives the instantaneous velocity of the object at
  time $t$, and
\item $\vec{p}''(t)$ give the instantaneous acceleration of the object
  at time $t$.
\end{itemize}



Now let's see an example.
\begin{example}
You whirl a ball, attached to a string, above your head in a
counter-clockwise circle. The ball follows a circular path and makes
$2$ revolutions per second. The string has length $2$ft.
\begin{itemize}
\item Find the position function $\vec{p}(t)$ that describes this
  situation.
\item Find the acceleration of the ball and give a physical
  interpretation.
\end{itemize}
\begin{explanation}
  The ball whirls in a circle. Since the string is $2$ft long, the
  radius of the circle is $2$. We start by writing down a position
  function that describes
  \begin{itemize}
  \item a circle with radius $2$,
  \item centered at the origin of the $(x,y)$-plane,
  \item that makes a full revolution every $2\pi$ seconds.
  \end{itemize}
  \[
  \vector{\answer[given]{2\cos(t)}, \answer[given]{2\sin t}}
  \]
  However, we want two revolutions per second, not one revolution
  every $2\pi$ seconds. We modify the period by multiplying $t$ by
  $\answer[given]{4\pi}$. The final position function is
  \[
  \vec{p}(t) =\vector{\answer[given]{2\cos (4\pi t)},
    \answer[given]{2\sin (4\pi t)}}.
  \] 
  To find $\vec{a}(t)$, we differentiate $\vec{p}(t)$ twice.
  \begin{align*}
    \vec{v}(t) = \vec{p}'(t) &= \vector{\answer[given]{-8\pi \sin (4\pi t)}, \answer[given]{8\pi \cos (4\pi t)}}\\
    \vec{a}(t) =\vec{p}''(t) &= \vector{\answer[given]{-32\pi^2 \cos (4\pi t)}, \answer[given]{-32\pi^2 \sin (4\pi t) }}
    %&= -32\pi^2\vector{\cos (4\pi t), \sin (4\pi t)}.
  \end{align*}
  
  For the physical interpretation, recall the classic physics
  equation, ``Force $=$ mass $\times$ acceleration.'' A force acting
  on a mass induces acceleration (i.e., the mass moves); acceleration
  acting on a mass induces a force (gravity gives our mass a
  \emph{weight}). Thus force and acceleration are closely related. A
  moving ball ``wants'' to travel in a straight line. Why does the
  ball in our example move in a circle? It is attached to your hand by
  a string. The string applies a force to the ball, affecting it's
  motion: the string \emph{accelerates} the ball. This is not
  acceleration in the sense of ``it travels faster;'' rather, this
  acceleration is \textbf{changing the velocity} of the ball. In what
  direction is this force/acceleration being applied? In the direction
  of the string, towards your hand.

  Hence it makes sense $\vec{a}(t)$ is parallel to $\vec{p}(t)$, but
  has a different magnitude and points in the opposite direction.  The
  magnitude of the acceleration is related to the speed at which the
  ball is traveling. A ball whirling quickly is rapidly changing
  direction/velocity. When velocity is changing rapidly, the
  acceleration must be \wordChoice{\choice{small}\choice{large}}.
\end{explanation}
\end{example}

\begin{question}
  An object moves in a spiral with position function
  \[
  \vec{p}(t) = \vector{\cos t, \sin t, t},
  \]
  where distances are measured in meters and time is in
  minutes. Describe the object's velocity and acceleration at time
  $t$.
  \begin{prompt}
    \begin{align*}
      \vec{v}(t) &= \vector{\answer{-\sin t}, \answer{\cos t},\answer{1}}  \\
      \vec{a}(t) &= \vector{\answer{-\cos t}, \answer{-\sin t},\answer{0}}.
    \end{align*}
  \end{prompt}
  \begin{question}
    What is the speed of this object?
    \begin{prompt}
    \[
    |\vec{v}(t)| = \answer{\sqrt{2}}\unit{m}/\unit{min}
    \]
    \end{prompt}
    \begin{question}
      What is the angle between $\vec{v}$ and $\vec{a}$?
      \begin{prompt}
        \[
        \text{The angle equals } \answer{\pi/2}\unit{radians}
        \]
      \end{prompt}
      \begin{feedback}[correct]
        Since the speed is constant, the velocity and the acceleration
        are perpendicular.
      \end{feedback}
    \end{question}
  \end{question}
\end{question}
              
\section{Projectile motion}

An important application of vector-valued position functions is
\textit{projectile motion}: the motion of objects under the influence
of gravity. We will measure time in seconds, and distances will either
be in meters or feet. We will show that we can completely describe the
path of such an object knowing its initial position and initial
velocity (where it \textbf{is} and where it \textbf{is going.})
\index{vector-valued function!projectile motion}\index{projectile
  motion}

Suppose an object has initial position
\[
\vec{p}(0) = \vector{x(0),y(0)}
\]
and initial velocity of
\[
\vec{v}(0) = v_0\vector{\cos(\theta),\sin(\theta)}.
\]
Here, $\theta$ is often called the \dfn{angle of elevation}.  Since
the acceleration of the object is known, namely
\[
\vec{a}(t) =\vector{0,-g},
\]
where $g$ is the gravitational constant, we can find $\vec{p}(t)$ knowing
our two initial conditions. We first find $\vec{v}(t)$:
\begin{align*}
  \vec{v}(t) &= \int \vec{a}(t) \d t\\
  \vec{v}(t) &= \int \vector{0,-g} \d t\\
  \vec{v}(t) &= \vector{0,-gt} + \vec{v}(0).
\end{align*}
We integrate once more to find $\vec{p}(t)$:
\begin{align*}
\vec{p}(t) &= \int \vec{v}(t)\d t \\
\vec{p}(t) &= \int \vector{0,-gt} + \vec C_1 \d t\\
\vec{p}(t) &= \vector{0, \frac{-gt^2}{2}} +t\cdot \vec{v}(0)+ \vec{p}(0).
\end{align*}

\begin{onlineOnly}
  You can adjust the initial position, $P_0$, angle, magnitude of the
  velocity, and magnitude of the acceleration below:
  \begin{center}
    \geogebra{ukruZ8BS}{800}{600} %% https://ggbm.at/ukruZ8BS
  \end{center}
\end{onlineOnly}


We demonstrate how to solve for a position function in the context of
projectile motion in the next example.

\begin{example}
Sydney shoots her \textit{Red Ryder} BB gun across level ground from
an elevation of $4\unit{ft}$, where the barrel of the gun makes a
$0.1$ radian angle with the horizontal. Find how far the BB travels
before landing, assuming the BB is fired at the advertised rate of
$350\unit{ft/s}$ and ignoring air resistance. Assume that acceleration
due to gravity is $32\unit{ft/sec^2}$
\begin{explanation}
  Write with me
  \begin{align*}
    \vec{v}(t) &= \int \vec{a}(t) \d t\\
    \vec{v}(t) &= \int \vector{\answer[given]{0},\answer[given]{-32}} \d t\\
    \vec{v}(t) &= \vector{\answer[given]{0},\answer[given]{-32t}} + \vec{v}(0).
  \end{align*}
  But from the statement of the problem, we see that $\vec{v}(0) =
  350\vector{\cos(0.1),\sin(0.1)}$. So
  \[
  \vec{v}(t) = \vector{0,\answer[given]{-32t}}+ 350\vector{\cos(0.1),\sin(0.1)}.
  \]
  Please continuing writing with me
  \begin{align*}
    \vec{p}(t) &= \int \vec{v}(t)\d t \\
    \vec{p}(t) &= \int \vector{0,-32t} + 350\vector{\cos(0.1),\sin(0.1)}\d t\\
    \vec{p}(t) &= \vector{\answer[given]{0},\answer[given]{-16t^2}} +t\cdot 350\vector{\cos(0.1),\sin(0.1)}+ \vec{p}(0).
  \end{align*}
  and $\vec{p}(0) = \vector{\answer[given]{0},\answer[given]{4}}$, so
  \[
  \vec{p}(t) = \vector{0,-16t^2} +t\cdot 350\vector{\cos(0.1),\sin(0.1)}+ \vector{0,4}.
  \]
  Now we need to find \textbf{when} the BB lands, then we can find
  \textbf{where} it lands. We accomplish this by setting the
  $y$-component equal to $0$ 
  \[
  -16t^2+350\sin(0.1)t+4 = 0
  \]
  and solving for $t$:
  \[
  t = \frac{-\answer[given]{350\sin(0.1)} \pm \sqrt{\left(\answer[given]{350\sin(0.1)}\right)^2-4\answer[given]{(-16)(4)}}}{\answer[given]{-32}}
  \]
Discarding the negative solution that resulted from our quadratic
equation, we have found that the BB lands approximately
$2.3\unit{s}$ after firing; with $t\approx 2.3$, we find the
$x$-component of our position function is $2.3\cdot 350\cos(0.1) =
  800.98\unit{ft}$. 
\end{explanation}
\end{example}


\section{From distance traveled to arc length}

Consider a driver who sets her cruise-control to $60\unit{mph}$, and
travels at this speed for an hour. We can ask:
\begin{itemize}
\item How far did the driver travel?
\item How far is the driver from her starting position?
\end{itemize} 
The first is easy to answer: she traveled $60$ miles. The second is
impossible to answer with the given information. We do not know if she
traveled in a straight line, on an oval racetrack, or along a
slowly-winding highway.

This highlights an important fact: \textbf{to compute distance traveled, we
need only to know the speed}, given by $|\vec{v}(t)|$.

\begin{theorem}
Let $\vec{v}(t)$ be a velocity function for a moving object. The
distance traveled by the object for $a\le t\le b$ is:
\index{distance!traveled}\index{vector-valued function!distance
  traveled}\index{integration!distance traveled}
\[
\text{distance traveled} = \int_a^b |\vec{v}(t)|\d t.
\]
\end{theorem}

This theorem is a specific instance of the more general theorem for
arc length:
\begin{theorem}
  Let $\vec{f}(t) = \vector{x(t),y(t),z(t)}$ be a vector-valued function where
  \begin{itemize}
  \item $x'(t)$, $y'(t)$, and $z'(t)$ are continuous.
  \item The curve defined by $\vec{f}(t)$ is traversed once for $a\le
    t\le b$.
  \end{itemize}
  The arc length of the curve from
  \[
  \vector{x(a),y(a),z(a)}\quad\text{to}\quad\vector{x(b),y(b),z(b)}
  \]
  is given by
  \[
  \text{arc length} = \int_a^b |\vec{f}'(t)|\d t.
  \]
\end{theorem}


\begin{question}
  Knowing that
  \[
  \vector{R\cdot \cos(\theta), R \cdot \sin(\theta)}
  \]
  is a vector valued function that plots a circle of radius $R$ for
  $0\le \theta < 2\pi$, can you use an arc length integral to confirm
  its circumference is $2\pi R$?
  \begin{prompt}
    First note that
    \[
    |\vector{R\cdot \cos(\theta), R \cdot \sin(\theta)}| = \answer{R},
    \]
    and now compute
    \[
    \int_0^{2\pi} R \d \theta = \answer{2\pi R}.
    \]
  \end{prompt}
\end{question}

One more example, again interpreting our vector-valued function as
giving the position of an object in space.

\begin{example}
  A particle moves in space with position function
  \[
  \vec{p}(t) =\vector{t,t^2,\sin (\pi t)}
  \]
  on $[-2,2]$, where $t$ is measured in seconds and distances are in
  meters.
  \begin{image}
    \begin{tikzpicture}
      \begin{axis}%
        [width=175pt,tick label style={font=\scriptsize},axis on top,
	  axis lines=center,
	  view={115}{25},
	  name=myplot,
	  %xtick={-3,3},minor tick num=2,
	  %ytick={-3,3},
	  %ztick={-3,3},
	  ymin=-.5,ymax=4.5,
	  xmin=-3.5,xmax=3.5,
	  zmin=-1.5, zmax=1.5,
	  every axis x label/.style={at={(axis cs:\pgfkeysvalueof{/pgfplots/xmax},0,0)},xshift=-3pt,yshift=-3pt},
	  xlabel={\scriptsize $x$},
	  every axis y label/.style={at={(axis cs:0,\pgfkeysvalueof{/pgfplots/ymax},0)},xshift=0pt,yshift=-5pt},
	  ylabel={\scriptsize $y$},
	  every axis z label/.style={at={(axis cs:0,0,\pgfkeysvalueof{/pgfplots/zmax})},xshift=0pt,yshift=4pt},
	  zlabel={\scriptsize $z$}
	]
        
        \addplot3[domain=-2:2,,thick,smooth,samples y=0,penColor,samples=30,] ({x},{x^2},{sin(180*x)});
        
        \draw[thick,->] (axis cs: 1,1,0) -- (axis cs: 1.01,1.02,-.0314);
        %\draw[thick,->,{\colortwo}] (axis cs: 0,0,0) -- (axis cs: -1,0,1) node (B) {};
      \end{axis}
    \end{tikzpicture}
\end{image}

  Find:
\begin{itemize}
\item An integral that computes the distance traveled by the particle
  on $[-2,2]$.
\item The displacement of the particle on $[-2,2]$.
\item An integral that computes the particle's average speed.
\end{itemize}
\begin{explanation}
  First we'll compute the distance traveled.
  \begin{align*}
    \text{distance traveled} &= \int_{-2}^2 |\vec{v}(t)|\d t \\
    &= \int_{-2}^2 \sqrt{\vec{v}(t)\dotp \vec{v}(t)}\d t.\\
    &= \int_{-2}^2 \answer[given]{\sqrt{1+(2t)^2+ \pi^2\cos^2(t\pi)}}\d t.
  \end{align*}
  This cannot be solved in terms of elementary functions so we turn
  to numerical integration, finding the distance to be
  $12.88\unit{m}$.
  
  The displacement is the vector
  \begin{align*}
    \vec{p}(2)-\vec{p}(-2) &= \vector{\answer[given]{2},\answer[given]{4},\answer[given]{0}} - \vector{\answer[given]{-2},\answer[given]{4},\answer[given]{0}}\\
      &= \vector{\answer[given]{4},\answer[given]{0},\answer[given]{0}}.
  \end{align*}
  That is, the particle ends with an $x$-value increased by $4$ and
  with unchanged $(y,z)$-values the same.
  
  We found above that the particle traveled $12.88\unit{m}$ over $4$
  seconds. We can compute average speed by dividing: $12.88/4 =
  \answer[given]{3.22}\unit{m/s}$.
\end{explanation}
\end{example}
    
\section{Looking back}
    
Finally, let's think about what we learned in our previous calculus
courses, in terms of what we know now.

\subsection{Arc length}

When you first learned how to compute arc length in calculus, you
probably use a formula like
\[
\text{arc length} = \int_a^b \sqrt{1+ f'(x)^2} \d x
\]
next you learned that for parametric functions, the arc length was
\[
\text{arc length} = \int_a^b \sqrt{x'(t)^2+ y'(x)^2} \d t
\]
Note, the first formula, is really just the second one, where $x(t) =
t$ so that $\d x = \d t$ and $y(t) = f(t)$. Finally in this class, we view arc length as
\[
\text{arc length} = \int_a^b |\vec{f}'(t)|\d t.
\]
Since the magnitude of a vector can be given via the dot product, this
is a very general formula.




\subsection{Average value}


In your first calculus course, you defined the \dfn{average value of a
  function} to be
\[
\frac{1}{b-a}\int_a^bf(x)\d x.
\]
Above we computed the \dfn{average speed} as
\[
\frac{\text{distance traveled}}{\text{travel time}} =
\frac1{2-(-2)}\int_{-2}^2|\vec{v}(t)|\d t;
\]
that is, we just found the average value of $|\vec{v}(t)|$ on
$[-2,2]$.

Likewise, given position function $\vec{p}(t)$, the \dfn{average velocity} on $[a,b]$ is
\begin{align*}
  \frac{\text{displacement}}{\text{travel time}} &= \frac1{b-a}\int_a^b \vec{p}'(t)\d t\\
  &= \frac{\vec{p}(b)-\vec{p}(a)}{b-a};
\end{align*}
that is, it is the average value of $\vec{p}'(t)$, or $\vec{v}(t)$, on
$[a,b]$. As we learn new material, we must constantly reconcile, and
reintegrate what have learned before.

\end{document}
