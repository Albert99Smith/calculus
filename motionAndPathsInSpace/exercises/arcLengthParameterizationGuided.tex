\documentclass{ximera}

%\usepackage{todonotes}
%\usepackage{mathtools} %% Required for wide table Curl and Greens
%\usepackage{cuted} %% Required for wide table Curl and Greens
\newcommand{\todo}{}

\usepackage{esint} % for \oiint
\ifxake%%https://math.meta.stackexchange.com/questions/9973/how-do-you-render-a-closed-surface-double-integral
\renewcommand{\oiint}{{\large\bigcirc}\kern-1.56em\iint}
\fi


\graphicspath{
  {./}
  {ximeraTutorial/}
  {basicPhilosophy/}
  {functionsOfSeveralVariables/}
  {normalVectors/}
  {lagrangeMultipliers/}
  {vectorFields/}
  {greensTheorem/}
  {shapeOfThingsToCome/}
  {dotProducts/}
  {partialDerivativesAndTheGradientVector/}
  {../productAndQuotientRules/exercises/}
  {../normalVectors/exercisesParametricPlots/}
  {../continuityOfFunctionsOfSeveralVariables/exercises/}
  {../partialDerivativesAndTheGradientVector/exercises/}
  {../directionalDerivativeAndChainRule/exercises/}
  {../commonCoordinates/exercisesCylindricalCoordinates/}
  {../commonCoordinates/exercisesSphericalCoordinates/}
  {../greensTheorem/exercisesCurlAndLineIntegrals/}
  {../greensTheorem/exercisesDivergenceAndLineIntegrals/}
  {../shapeOfThingsToCome/exercisesDivergenceTheorem/}
  {../greensTheorem/}
  {../shapeOfThingsToCome/}
  {../separableDifferentialEquations/exercises/}
  {vectorFields/}
}

\newcommand{\mooculus}{\textsf{\textbf{MOOC}\textnormal{\textsf{ULUS}}}}

\usepackage{tkz-euclide}\usepackage{tikz}
\usepackage{tikz-cd}
\usetikzlibrary{arrows}
\tikzset{>=stealth,commutative diagrams/.cd,
  arrow style=tikz,diagrams={>=stealth}} %% cool arrow head
\tikzset{shorten <>/.style={ shorten >=#1, shorten <=#1 } } %% allows shorter vectors

\usetikzlibrary{backgrounds} %% for boxes around graphs
\usetikzlibrary{shapes,positioning}  %% Clouds and stars
\usetikzlibrary{matrix} %% for matrix
\usepgfplotslibrary{polar} %% for polar plots
\usepgfplotslibrary{fillbetween} %% to shade area between curves in TikZ
\usetkzobj{all}
\usepackage[makeroom]{cancel} %% for strike outs
%\usepackage{mathtools} %% for pretty underbrace % Breaks Ximera
%\usepackage{multicol}
\usepackage{pgffor} %% required for integral for loops



%% http://tex.stackexchange.com/questions/66490/drawing-a-tikz-arc-specifying-the-center
%% Draws beach ball
\tikzset{pics/carc/.style args={#1:#2:#3}{code={\draw[pic actions] (#1:#3) arc(#1:#2:#3);}}}



\usepackage{array}
\setlength{\extrarowheight}{+.1cm}
\newdimen\digitwidth
\settowidth\digitwidth{9}
\def\divrule#1#2{
\noalign{\moveright#1\digitwidth
\vbox{\hrule width#2\digitwidth}}}





\newcommand{\RR}{\mathbb R}
\newcommand{\R}{\mathbb R}
\newcommand{\N}{\mathbb N}
\newcommand{\Z}{\mathbb Z}

\newcommand{\sagemath}{\textsf{SageMath}}


%\renewcommand{\d}{\,d\!}
\renewcommand{\d}{\mathop{}\!d}
\newcommand{\dd}[2][]{\frac{\d #1}{\d #2}}
\newcommand{\pp}[2][]{\frac{\partial #1}{\partial #2}}
\renewcommand{\l}{\ell}
\newcommand{\ddx}{\frac{d}{\d x}}

\newcommand{\zeroOverZero}{\ensuremath{\boldsymbol{\tfrac{0}{0}}}}
\newcommand{\inftyOverInfty}{\ensuremath{\boldsymbol{\tfrac{\infty}{\infty}}}}
\newcommand{\zeroOverInfty}{\ensuremath{\boldsymbol{\tfrac{0}{\infty}}}}
\newcommand{\zeroTimesInfty}{\ensuremath{\small\boldsymbol{0\cdot \infty}}}
\newcommand{\inftyMinusInfty}{\ensuremath{\small\boldsymbol{\infty - \infty}}}
\newcommand{\oneToInfty}{\ensuremath{\boldsymbol{1^\infty}}}
\newcommand{\zeroToZero}{\ensuremath{\boldsymbol{0^0}}}
\newcommand{\inftyToZero}{\ensuremath{\boldsymbol{\infty^0}}}



\newcommand{\numOverZero}{\ensuremath{\boldsymbol{\tfrac{\#}{0}}}}
\newcommand{\dfn}{\textbf}
%\newcommand{\unit}{\,\mathrm}
\newcommand{\unit}{\mathop{}\!\mathrm}
\newcommand{\eval}[1]{\bigg[ #1 \bigg]}
\newcommand{\seq}[1]{\left( #1 \right)}
\renewcommand{\epsilon}{\varepsilon}
\renewcommand{\phi}{\varphi}


\renewcommand{\iff}{\Leftrightarrow}

\DeclareMathOperator{\arccot}{arccot}
\DeclareMathOperator{\arcsec}{arcsec}
\DeclareMathOperator{\arccsc}{arccsc}
\DeclareMathOperator{\si}{Si}
\DeclareMathOperator{\scal}{scal}
\DeclareMathOperator{\sign}{sign}


%% \newcommand{\tightoverset}[2]{% for arrow vec
%%   \mathop{#2}\limits^{\vbox to -.5ex{\kern-0.75ex\hbox{$#1$}\vss}}}
\newcommand{\arrowvec}[1]{{\overset{\rightharpoonup}{#1}}}
%\renewcommand{\vec}[1]{\arrowvec{\mathbf{#1}}}
\renewcommand{\vec}[1]{{\overset{\boldsymbol{\rightharpoonup}}{\mathbf{#1}}}\hspace{0in}}

\newcommand{\point}[1]{\left(#1\right)} %this allows \vector{ to be changed to \vector{ with a quick find and replace
\newcommand{\pt}[1]{\mathbf{#1}} %this allows \vec{ to be changed to \vec{ with a quick find and replace
\newcommand{\Lim}[2]{\lim_{\point{#1} \to \point{#2}}} %Bart, I changed this to point since I want to use it.  It runs through both of the exercise and exerciseE files in limits section, which is why it was in each document to start with.

\DeclareMathOperator{\proj}{\mathbf{proj}}
\newcommand{\veci}{{\boldsymbol{\hat{\imath}}}}
\newcommand{\vecj}{{\boldsymbol{\hat{\jmath}}}}
\newcommand{\veck}{{\boldsymbol{\hat{k}}}}
\newcommand{\vecl}{\vec{\boldsymbol{\l}}}
\newcommand{\uvec}[1]{\mathbf{\hat{#1}}}
\newcommand{\utan}{\mathbf{\hat{t}}}
\newcommand{\unormal}{\mathbf{\hat{n}}}
\newcommand{\ubinormal}{\mathbf{\hat{b}}}

\newcommand{\dotp}{\bullet}
\newcommand{\cross}{\boldsymbol\times}
\newcommand{\grad}{\boldsymbol\nabla}
\newcommand{\divergence}{\grad\dotp}
\newcommand{\curl}{\grad\cross}
%\DeclareMathOperator{\divergence}{divergence}
%\DeclareMathOperator{\curl}[1]{\grad\cross #1}
\newcommand{\lto}{\mathop{\longrightarrow\,}\limits}

\renewcommand{\bar}{\overline}

\colorlet{textColor}{black}
\colorlet{background}{white}
\colorlet{penColor}{blue!50!black} % Color of a curve in a plot
\colorlet{penColor2}{red!50!black}% Color of a curve in a plot
\colorlet{penColor3}{red!50!blue} % Color of a curve in a plot
\colorlet{penColor4}{green!50!black} % Color of a curve in a plot
\colorlet{penColor5}{orange!80!black} % Color of a curve in a plot
\colorlet{penColor6}{yellow!70!black} % Color of a curve in a plot
\colorlet{fill1}{penColor!20} % Color of fill in a plot
\colorlet{fill2}{penColor2!20} % Color of fill in a plot
\colorlet{fillp}{fill1} % Color of positive area
\colorlet{filln}{penColor2!20} % Color of negative area
\colorlet{fill3}{penColor3!20} % Fill
\colorlet{fill4}{penColor4!20} % Fill
\colorlet{fill5}{penColor5!20} % Fill
\colorlet{gridColor}{gray!50} % Color of grid in a plot

\newcommand{\surfaceColor}{violet}
\newcommand{\surfaceColorTwo}{redyellow}
\newcommand{\sliceColor}{greenyellow}




\pgfmathdeclarefunction{gauss}{2}{% gives gaussian
  \pgfmathparse{1/(#2*sqrt(2*pi))*exp(-((x-#1)^2)/(2*#2^2))}%
}


%%%%%%%%%%%%%
%% Vectors
%%%%%%%%%%%%%

%% Simple horiz vectors
\renewcommand{\vector}[1]{\left\langle #1\right\rangle}


%% %% Complex Horiz Vectors with angle brackets
%% \makeatletter
%% \renewcommand{\vector}[2][ , ]{\left\langle%
%%   \def\nextitem{\def\nextitem{#1}}%
%%   \@for \el:=#2\do{\nextitem\el}\right\rangle%
%% }
%% \makeatother

%% %% Vertical Vectors
%% \def\vector#1{\begin{bmatrix}\vecListA#1,,\end{bmatrix}}
%% \def\vecListA#1,{\if,#1,\else #1\cr \expandafter \vecListA \fi}

%%%%%%%%%%%%%
%% End of vectors
%%%%%%%%%%%%%

%\newcommand{\fullwidth}{}
%\newcommand{\normalwidth}{}



%% makes a snazzy t-chart for evaluating functions
%\newenvironment{tchart}{\rowcolors{2}{}{background!90!textColor}\array}{\endarray}

%%This is to help with formatting on future title pages.
\newenvironment{sectionOutcomes}{}{}



%% Flowchart stuff
%\tikzstyle{startstop} = [rectangle, rounded corners, minimum width=3cm, minimum height=1cm,text centered, draw=black]
%\tikzstyle{question} = [rectangle, minimum width=3cm, minimum height=1cm, text centered, draw=black]
%\tikzstyle{decision} = [trapezium, trapezium left angle=70, trapezium right angle=110, minimum width=3cm, minimum height=1cm, text centered, draw=black]
%\tikzstyle{question} = [rectangle, rounded corners, minimum width=3cm, minimum height=1cm,text centered, draw=black]
%\tikzstyle{process} = [rectangle, minimum width=3cm, minimum height=1cm, text centered, draw=black]
%\tikzstyle{decision} = [trapezium, trapezium left angle=70, trapezium right angle=110, minimum width=3cm, minimum height=1cm, text centered, draw=black]


\author{Jim Talamo}

\outcome{Compute the length of a parametric curve.}

\begin{document}
\begin{exercise}
 
Suppose that the curve $C$ in the $xy$-plane is traced out by the vector-valued function 

\[
\vec{r}(t) = \vector{2t^{3/2},4t+1}, 0 \leq t \leq 4.
\] 

In order to determine if the curve is parameterized by arclength, we could check either if

\begin{itemize}
\item $\int_0^s \left|\vec{r}'(t)\right| \d s = s$
\item $\left|\vec{r}'(t)\right|=1$ for all $t$.
\end{itemize}

If either of these holds, then the curve uses arclength as a parameter.  Furthermore, in order to establish the first result, we would have to compute $\left|\vec{r}'(t)\right|$ anyways, so let's take this approach.

We calculate that $\vec{r}'(t) = \vector{\answer{3t^{1/2}},\answer{4}}$, and hence $\left|\vec{r}'(t)\right| = \answer{\sqrt{9t+16}}$.

Since $\left|\vec{r}'(t)\right| \neq 1$ for all $t$, the curve \wordChoice{\choice{does}\choice[correct]{does not}} use arclength as a parameter.

\begin{exercise}
In order to find a description $\vec{p}(s)$ that does use arclength as a parameter, we can take the following steps.

\begin{itemize}
\item[1.] Find $s$ in terms of $t$ by computing $s = \int_0^t \left|\vec{r}'(\tau)\right| \d \tau$.

Since $\left|\vec{r}'(t)\right| = \sqrt{9t+16}$, $\left|\vec{r}'(\tau)\right| = \sqrt{9\tau+16}$, and thus

\begin{align*}
s &= \int_0^t \left|\vec{r}'(\tau)\right| \d \tau \\
&= \int_0^t \sqrt{9\tau+16} \d \tau
\end{align*}

Note that we need to find the antiderivative $\int \sqrt{9x+16} \d x$ to proceed, and 

\[
\int \sqrt{9x+16} \d x = \answer{\frac{2}{27} (9x+16)^{3/2}} +C.
\] 

Thus, we find that $s = \answer{\frac{2}{27} (9t+16)^{3/2} - \frac{128}{27}}$.

\begin{hint}
We can calculate $\int \sqrt{9x+16} \d x$ with the substitution $u=9x+16$.  Don't forget to evaluate the antiderivative at $t=0$ when you find $s$.

\end{hint}

\item[2.] Solve for $t$ in terms of $s$.

This requires some careful algebra, after which we find

\[
t = \answer{\frac{\left(\frac{27}{2}s+64\right)^{2/3}-16}{9}}
\]

\begin{hint}
Let's work through the start of the algebra one step at a time.

\begin{align*}
s &= \frac{2}{27} (9t+16)^{3/2} - \frac{128}{27} \\
s +\frac{128}{27} &=  \frac{2}{27} (9t+16)^{3/2} \\
\frac{27}{2}s +64 &= (9t+16)^{3/2}& (\textrm{ multiply both sides by } \frac{27}{2} \textrm{ and simplify. } ) \\
\left(\frac{27}{2}s +64\right)^{\answer{2/3}} &= 9t+16 \\
\end{align*}

From here, there's not too much more work necessary to solve for $t$.
\end{hint}

\item[3.] We can now find the parameterization $\vec{p}(s)$ by substituting the expression above for $t$ in the original parameterization. 

\begin{align*}
\vec{r}(t) &= \vector{2t^{3/2},4t+1} , 0 \leq t \leq 4 \\
\vec{p}(s) &= \vector{2\left(\answer{\frac{\left(\frac{27}{2}s+64\right)^{2/3}-16}{9} } \right)^{3/2},4\left( \answer{\frac{\left(\frac{27}{2}s+64\right)^{2/3}-16}{9}} \right)+1}
\end{align*} 

Note that we also need to transform the domain as well.  Since the original domain is $0 \leq t \leq 4$, we have $0 \leq \frac{\left(\frac{27}{2}s+64\right)^{2/3}-16}{9} \leq 4$, or 

\[
\textrm{ Domain in } s: ~ 0 \leq s \leq \answer{\frac{2}{27}\left((52)^{3/2}-64\right)}
\]
\end{itemize}

\begin{feedback}[correct]
Note that the arclength parameterization here is quite painful.  In practice, the best we can hope for is that this parameterization is painful, but possible.  Since finding $s$ in terms of $t$ in general requires that we evaluate an integral that has a square root, we often will not be able to compute the necessary antiderivative explicitly.  Even if we can, we still have to solve for $t$ in terms of $s$.  

In general, arclength is not generally a useful parameter to use explicitly, but it is a fundamentally important one to use when formulating definitions that have physical meaning, as a later exercise will explore.  A common theme in results that require this is to take the definition in terms of arclength and establish a computational result in terms of the parameter you want to use.  This will be the subject of a later problem in this assignment.

\end{feedback}
\end{exercise}
 \end{exercise}
\end{document}
