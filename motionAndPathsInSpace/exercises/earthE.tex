\documentclass{ximera}

%\usepackage{todonotes}
%\usepackage{mathtools} %% Required for wide table Curl and Greens
%\usepackage{cuted} %% Required for wide table Curl and Greens
\newcommand{\todo}{}

\usepackage{esint} % for \oiint
\ifxake%%https://math.meta.stackexchange.com/questions/9973/how-do-you-render-a-closed-surface-double-integral
\renewcommand{\oiint}{{\large\bigcirc}\kern-1.56em\iint}
\fi


\graphicspath{
  {./}
  {ximeraTutorial/}
  {basicPhilosophy/}
  {functionsOfSeveralVariables/}
  {normalVectors/}
  {lagrangeMultipliers/}
  {vectorFields/}
  {greensTheorem/}
  {shapeOfThingsToCome/}
  {dotProducts/}
  {partialDerivativesAndTheGradientVector/}
  {../productAndQuotientRules/exercises/}
  {../normalVectors/exercisesParametricPlots/}
  {../continuityOfFunctionsOfSeveralVariables/exercises/}
  {../partialDerivativesAndTheGradientVector/exercises/}
  {../directionalDerivativeAndChainRule/exercises/}
  {../commonCoordinates/exercisesCylindricalCoordinates/}
  {../commonCoordinates/exercisesSphericalCoordinates/}
  {../greensTheorem/exercisesCurlAndLineIntegrals/}
  {../greensTheorem/exercisesDivergenceAndLineIntegrals/}
  {../shapeOfThingsToCome/exercisesDivergenceTheorem/}
  {../greensTheorem/}
  {../shapeOfThingsToCome/}
  {../separableDifferentialEquations/exercises/}
  {vectorFields/}
}

\newcommand{\mooculus}{\textsf{\textbf{MOOC}\textnormal{\textsf{ULUS}}}}

\usepackage{tkz-euclide}\usepackage{tikz}
\usepackage{tikz-cd}
\usetikzlibrary{arrows}
\tikzset{>=stealth,commutative diagrams/.cd,
  arrow style=tikz,diagrams={>=stealth}} %% cool arrow head
\tikzset{shorten <>/.style={ shorten >=#1, shorten <=#1 } } %% allows shorter vectors

\usetikzlibrary{backgrounds} %% for boxes around graphs
\usetikzlibrary{shapes,positioning}  %% Clouds and stars
\usetikzlibrary{matrix} %% for matrix
\usepgfplotslibrary{polar} %% for polar plots
\usepgfplotslibrary{fillbetween} %% to shade area between curves in TikZ
\usetkzobj{all}
\usepackage[makeroom]{cancel} %% for strike outs
%\usepackage{mathtools} %% for pretty underbrace % Breaks Ximera
%\usepackage{multicol}
\usepackage{pgffor} %% required for integral for loops



%% http://tex.stackexchange.com/questions/66490/drawing-a-tikz-arc-specifying-the-center
%% Draws beach ball
\tikzset{pics/carc/.style args={#1:#2:#3}{code={\draw[pic actions] (#1:#3) arc(#1:#2:#3);}}}



\usepackage{array}
\setlength{\extrarowheight}{+.1cm}
\newdimen\digitwidth
\settowidth\digitwidth{9}
\def\divrule#1#2{
\noalign{\moveright#1\digitwidth
\vbox{\hrule width#2\digitwidth}}}





\newcommand{\RR}{\mathbb R}
\newcommand{\R}{\mathbb R}
\newcommand{\N}{\mathbb N}
\newcommand{\Z}{\mathbb Z}

\newcommand{\sagemath}{\textsf{SageMath}}


%\renewcommand{\d}{\,d\!}
\renewcommand{\d}{\mathop{}\!d}
\newcommand{\dd}[2][]{\frac{\d #1}{\d #2}}
\newcommand{\pp}[2][]{\frac{\partial #1}{\partial #2}}
\renewcommand{\l}{\ell}
\newcommand{\ddx}{\frac{d}{\d x}}

\newcommand{\zeroOverZero}{\ensuremath{\boldsymbol{\tfrac{0}{0}}}}
\newcommand{\inftyOverInfty}{\ensuremath{\boldsymbol{\tfrac{\infty}{\infty}}}}
\newcommand{\zeroOverInfty}{\ensuremath{\boldsymbol{\tfrac{0}{\infty}}}}
\newcommand{\zeroTimesInfty}{\ensuremath{\small\boldsymbol{0\cdot \infty}}}
\newcommand{\inftyMinusInfty}{\ensuremath{\small\boldsymbol{\infty - \infty}}}
\newcommand{\oneToInfty}{\ensuremath{\boldsymbol{1^\infty}}}
\newcommand{\zeroToZero}{\ensuremath{\boldsymbol{0^0}}}
\newcommand{\inftyToZero}{\ensuremath{\boldsymbol{\infty^0}}}



\newcommand{\numOverZero}{\ensuremath{\boldsymbol{\tfrac{\#}{0}}}}
\newcommand{\dfn}{\textbf}
%\newcommand{\unit}{\,\mathrm}
\newcommand{\unit}{\mathop{}\!\mathrm}
\newcommand{\eval}[1]{\bigg[ #1 \bigg]}
\newcommand{\seq}[1]{\left( #1 \right)}
\renewcommand{\epsilon}{\varepsilon}
\renewcommand{\phi}{\varphi}


\renewcommand{\iff}{\Leftrightarrow}

\DeclareMathOperator{\arccot}{arccot}
\DeclareMathOperator{\arcsec}{arcsec}
\DeclareMathOperator{\arccsc}{arccsc}
\DeclareMathOperator{\si}{Si}
\DeclareMathOperator{\scal}{scal}
\DeclareMathOperator{\sign}{sign}


%% \newcommand{\tightoverset}[2]{% for arrow vec
%%   \mathop{#2}\limits^{\vbox to -.5ex{\kern-0.75ex\hbox{$#1$}\vss}}}
\newcommand{\arrowvec}[1]{{\overset{\rightharpoonup}{#1}}}
%\renewcommand{\vec}[1]{\arrowvec{\mathbf{#1}}}
\renewcommand{\vec}[1]{{\overset{\boldsymbol{\rightharpoonup}}{\mathbf{#1}}}\hspace{0in}}

\newcommand{\point}[1]{\left(#1\right)} %this allows \vector{ to be changed to \vector{ with a quick find and replace
\newcommand{\pt}[1]{\mathbf{#1}} %this allows \vec{ to be changed to \vec{ with a quick find and replace
\newcommand{\Lim}[2]{\lim_{\point{#1} \to \point{#2}}} %Bart, I changed this to point since I want to use it.  It runs through both of the exercise and exerciseE files in limits section, which is why it was in each document to start with.

\DeclareMathOperator{\proj}{\mathbf{proj}}
\newcommand{\veci}{{\boldsymbol{\hat{\imath}}}}
\newcommand{\vecj}{{\boldsymbol{\hat{\jmath}}}}
\newcommand{\veck}{{\boldsymbol{\hat{k}}}}
\newcommand{\vecl}{\vec{\boldsymbol{\l}}}
\newcommand{\uvec}[1]{\mathbf{\hat{#1}}}
\newcommand{\utan}{\mathbf{\hat{t}}}
\newcommand{\unormal}{\mathbf{\hat{n}}}
\newcommand{\ubinormal}{\mathbf{\hat{b}}}

\newcommand{\dotp}{\bullet}
\newcommand{\cross}{\boldsymbol\times}
\newcommand{\grad}{\boldsymbol\nabla}
\newcommand{\divergence}{\grad\dotp}
\newcommand{\curl}{\grad\cross}
%\DeclareMathOperator{\divergence}{divergence}
%\DeclareMathOperator{\curl}[1]{\grad\cross #1}
\newcommand{\lto}{\mathop{\longrightarrow\,}\limits}

\renewcommand{\bar}{\overline}

\colorlet{textColor}{black}
\colorlet{background}{white}
\colorlet{penColor}{blue!50!black} % Color of a curve in a plot
\colorlet{penColor2}{red!50!black}% Color of a curve in a plot
\colorlet{penColor3}{red!50!blue} % Color of a curve in a plot
\colorlet{penColor4}{green!50!black} % Color of a curve in a plot
\colorlet{penColor5}{orange!80!black} % Color of a curve in a plot
\colorlet{penColor6}{yellow!70!black} % Color of a curve in a plot
\colorlet{fill1}{penColor!20} % Color of fill in a plot
\colorlet{fill2}{penColor2!20} % Color of fill in a plot
\colorlet{fillp}{fill1} % Color of positive area
\colorlet{filln}{penColor2!20} % Color of negative area
\colorlet{fill3}{penColor3!20} % Fill
\colorlet{fill4}{penColor4!20} % Fill
\colorlet{fill5}{penColor5!20} % Fill
\colorlet{gridColor}{gray!50} % Color of grid in a plot

\newcommand{\surfaceColor}{violet}
\newcommand{\surfaceColorTwo}{redyellow}
\newcommand{\sliceColor}{greenyellow}




\pgfmathdeclarefunction{gauss}{2}{% gives gaussian
  \pgfmathparse{1/(#2*sqrt(2*pi))*exp(-((x-#1)^2)/(2*#2^2))}%
}


%%%%%%%%%%%%%
%% Vectors
%%%%%%%%%%%%%

%% Simple horiz vectors
\renewcommand{\vector}[1]{\left\langle #1\right\rangle}


%% %% Complex Horiz Vectors with angle brackets
%% \makeatletter
%% \renewcommand{\vector}[2][ , ]{\left\langle%
%%   \def\nextitem{\def\nextitem{#1}}%
%%   \@for \el:=#2\do{\nextitem\el}\right\rangle%
%% }
%% \makeatother

%% %% Vertical Vectors
%% \def\vector#1{\begin{bmatrix}\vecListA#1,,\end{bmatrix}}
%% \def\vecListA#1,{\if,#1,\else #1\cr \expandafter \vecListA \fi}

%%%%%%%%%%%%%
%% End of vectors
%%%%%%%%%%%%%

%\newcommand{\fullwidth}{}
%\newcommand{\normalwidth}{}



%% makes a snazzy t-chart for evaluating functions
%\newenvironment{tchart}{\rowcolors{2}{}{background!90!textColor}\array}{\endarray}

%%This is to help with formatting on future title pages.
\newenvironment{sectionOutcomes}{}{}



%% Flowchart stuff
%\tikzstyle{startstop} = [rectangle, rounded corners, minimum width=3cm, minimum height=1cm,text centered, draw=black]
%\tikzstyle{question} = [rectangle, minimum width=3cm, minimum height=1cm, text centered, draw=black]
%\tikzstyle{decision} = [trapezium, trapezium left angle=70, trapezium right angle=110, minimum width=3cm, minimum height=1cm, text centered, draw=black]
%\tikzstyle{question} = [rectangle, rounded corners, minimum width=3cm, minimum height=1cm,text centered, draw=black]
%\tikzstyle{process} = [rectangle, minimum width=3cm, minimum height=1cm, text centered, draw=black]
%\tikzstyle{decision} = [trapezium, trapezium left angle=70, trapezium right angle=110, minimum width=3cm, minimum height=1cm, text centered, draw=black]


\author{Bart Snapp and Jim Talamo}

\outcome{Reparametrize a curve.}
\outcome{Parameterize a curve in terms of arc length.}

\begin{document}
\begin{exercise}
  The Earth travels in an orbit around the Sun that can be
  approximated by a circle. The distance from the Earth to the Sun is
  (on average) $1$
  \link[au]{https://en.wikipedia.org/wiki/Astronomical_unit}. We are
  going to make some models of the Earth's orbit in the
  $(x,y)$-plane. For all of our models, we will make the following
  assumptions:
  \begin{itemize}
  \item The Sun will be at the origin.
  \item At the starting time, $t=0$, the Earth will be at the point
    $(1,0)$ in the $(x,y)$-plane.
  \item The Earth will travel in a counterclockwise direction around
    the Sun.
  \end{itemize}
  \begin{exercise}
    Give a parametrization of the Earth's orbit that will model the
    Earth's position in terms of $t$, where the units are years.
    \[
    \vec{y}(t) = \vector{\answer{\cos(2\pi t)},\answer{\sin(2\pi t)}} 
    \]
  \end{exercise}
  \begin{exercise}
    Give a parametrization of the Earth's orbit that will model the
    Earth's position in terms of $t$, where the units are months.
    \[
    \vec{m}(t) = \vector{\answer{\cos(2\pi t/12)},\answer{\sin(2\pi t/12)}} 
    \]
  \end{exercise}
  
  \begin{exercise}
    Give a parametrization of the Earth's orbit that will model the
    Earth's position in terms of $t$, where the units are days.
    \[
    \vec{d}(t) = \vector{\answer{\cos(2\pi t/365)},\answer{\sin(2\pi t/365)}} 
    \]
  \end{exercise}
  
  \begin{exercise}
    Given that the Earth's orbit has a circumference of $940$ million
    miles, give a parametrization of the Earth's orbit that will model
    the Earth's position in terms of $S$, the distance traveled in
    millions of miles.
    \[
    \vec{e}(S) = \vector{\answer{\cos(2\pi S/940)},\answer{\sin(2\pi S/940)}} 
    \]
    \begin{exercise}
 %The above needs to have the radius listed in terms of miles!!!!  Fix everything below this eventually - Thanks to Sohail for making this observation.
 
      True or False: This is an arc length parameterization.
    \begin{multipleChoice}
      \choice{True}
      \choice[correct]{False}
    \end{multipleChoice}
    \begin{exercise}
    It may be tempting to think that since we are using ``distance'' in our parameterization that this means we have parameterized by arclength, but we can actually check whether we have an arclength parameterization by checking if
    
    \begin{selectAll}
    \choice[correct]{$\int_0^S \left|\vec{e}'(t)\right| \d t = S$.}
    \choice[correct]{$ \left|\vec{e}'(S)\right| =1$ for all $S$.}
    \end{selectAll}
        
      Note that $\vec{e}'(S) = \vector{\answer{(-\pi/470) \sin(2\pi S/940)},\answer{(\pi/470) \cos(2\pi S/940)}}$, so $\left|\vec{e}'(S)\right| = \answer{\pi/470}$.
      
While this is enough to determine that we have not used arclength as a parameter, we also can note that 

\[
\int_0^S \left|\vec{e}'(t)\right| \d t = \int_0^S \frac{\pi}{470}  \d t = \answer{\frac{\pi}{470} S} \neq S.
\]      

Let's now find a parameterization $\vec{p}(s)$ of the orbit that uses arclength as a parameter.

Above, we found that $\left|\vec{e}'(S)\right| = \frac{\pi}{470}$, so $s = \int_0^S \left|\vec{e}'(t)\right| \d t = \answer{\frac{\pi}{470} s}$ or, solving for $s$, we find

\[
s = \answer{\frac{470}{\pi}S}.
\]

Thus, a parameterization that uses arclength as a parameter is
\[
\vec{p}(s) = \vector{\answer{\cos(s) },\answer{\sin(s)}}. 
\]

\begin{feedback}[correct]
 Note that at the \emph{start} of the problem, we chose the length unit $\unit{au}$ to describe the orbit of the Earth; that is, we assume that the Earth's initial position is at $(1,0)$.  Since the unit $\unit{au}$ is the distance the Earth is from the Sun, and we assume that the Earth travels around the Sun in a circular orbit, our units on the $x$ and $y$ axis must be in $\unit{au}$.  
 
The parameter $S$ in this problem was millions of miles, and we found that $s = \frac{470}{\pi}S$.  Since $1 \unit{au} = 150\unit{million~miles}$, note that:
 
\[ \pi/470\cdot 150 \unit{million~miles}     \approx 1\unit{au} , \]

so the $\pi/470$ is the approximate conversion factor from millions of miles to $\unit{au}$.

An important point this brings up in arclength applications in STEM fields is the following:

\begin{quote}
``Parameterization by arclength'' only has meaning if you first choose your units!
\end{quote}

\end{feedback}
 
                   \end{exercise}
        \end{exercise}
      \end{exercise}
        \end{exercise}
\end{document}
